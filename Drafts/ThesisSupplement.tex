\appendix

\chapter{Supplementary Analysis and Figures}
\label{app:futurecovcond}

\paragraph{Oak-Conifer Forest and Woodland} Oak-conifer forests and woodlands were within the simulated historic and future ranges of variability. Median proportions approached the current proportion as median climate parameter value increased, and the scenario with the most extreme climate trend, ESM2M, actually had a median value exceeding that of the current, in contrast to the simulated historical median, which was lower than the current condition. The mid development seral stages are now generally outside the HRV and the FRV. In the future, the proportion of mid closed and mid moderate is expected to decrease relative to the simulated HRV, while the seven FRVs are fairly similar to the HRV. The amount of late development closed on the landscape is currently very low, and the FRV results show the landscape trending this way under increasingly hot and dry climate trends, to the point where the current conditions are within the FRV for some scenarios. The proportion of late development moderate is also predicted to become less common in the future scenarios, but would still be higher than the current condition. The proportion of late development open is projected to sharply increase relevant to both the current conditions and the simulated HRV across all simulated FRV scenarios.

When we pooled the future scenario results, we found that the projected future proportions of oak-conifer forests and woodlands are much higher than the HRV for the late development open stage. The early and mid open seral stages are projected to occupy similar proportions across the future and historical scenarios. For all remaining seral stages, the future proportion is projected to be less than under the historical scenario (Figure \ref{fig:covcond_ocfw}).

\begin{figure}[htbp]
  \centering
  \subfloat[][]{
    \centering
    \includegraphics[width=0.4\textwidth]{/Users/mmallek/Documents/Thesis/Plots/covcond-byscenario/1410-boxplots.png}
  }%
  \qquad
  \subfloat[][]{
    \includegraphics[width=0.4\textwidth]{/Users/mmallek/Documents/Thesis/Plots/covcond-byscenario/1420-boxplots.png}
  } \\
    \subfloat[][]{
    \centering
    \includegraphics[width=0.4\textwidth]{/Users/mmallek/Documents/Thesis/Plots/covcond-byscenario/1421-boxplots.png}
  }%
  \qquad
  \subfloat[][]{
    \includegraphics[width=0.4\textwidth]{/Users/mmallek/Documents/Thesis/Plots/covcond-byscenario/1422-boxplots.png}
  } \\
    \subfloat[][]{
    \centering
    \includegraphics[width=0.4\textwidth]{/Users/mmallek/Documents/Thesis/Plots/covcond-byscenario/1430-boxplots.png}
  }%
      \subfloat[][]{
    \centering
    \includegraphics[width=0.4\textwidth]{/Users/mmallek/Documents/Thesis/Plots/covcond-byscenario/1431-boxplots.png}
  } \\
  \qquad
  \subfloat[][]{
    \includegraphics[width=0.4\textwidth]{/Users/mmallek/Documents/Thesis/Plots/covcond-byscenario/1432-boxplots.png}
  }
    \qquad
  \subfloat[][]{
    \includegraphics[width=0.4\textwidth]{/Users/mmallek/Documents/Thesis/Plots/covcond-frvhrv/OCFW-frvhrv-boxplots.png}
  }
    \caption{Boxplots illustrating the range of variability across historical and future climate trajectories. The horizontal black bar represents the current condition. Boxplot whiskers extend from the $5^{th}$ to $95^{th}$ range of variability for each model. }
  \label{fig:covcond_ocfw}
\end{figure} %ocfw

\paragraph{Red Fir - Mesic} Mesic red fir forests in the early seral stage were within both the simulated historical and future ranges of variability. However, all of the mid development stages are currently outside the HRV and FRV. The HRV and FRV are similar, although in general the FRVs are characterized by greater spread than the HRV. Closed canopies are now less common, and moderate to open canopies more common, during the simulated periods than on the current landscape. Interpretation for the late development stages is less straightforward. The late closed seral stage is far outside the simulated HRV, but within the simulated FRV for all 7 future climate scenarios. The moderate canopy cover stage, however, is outside all simulated ranges of variability, and the HRV and FRV differ primarily in their spread: the future scenarios have wider 90\% ranges of variability than does the HRV. The late open stage is outside and more prevalent today than the HRV. However, the current proportion is within, even near the median, of all of the future simulations. Again, the future simulations have very wide ranges of variability (near 0\% to around 20\% in the future scenarios, compared to perhaps 0\% to 4\% for the HRV). 

When we pooled the future scenario results, we found the greatest difference between the historical and future projections in the proportion of the cover type in the late development, closed canopy stage, with much less area devoted to this seral stage under the future scenario, putting the current proportion within the FRV. We also see projected increases compared to the HRV in the early seral stage and the mid closed stage (Figure \ref{fig:covcond_rfrm}).

\begin{figure}[htbp]
  \centering
  \subfloat[][]{
    \centering
    \includegraphics[width=0.4\textwidth]{/Users/mmallek/Documents/Thesis/Plots/covcond-byscenario/1710-boxplots.png}
  }%
  \qquad
  \subfloat[][]{
    \includegraphics[width=0.4\textwidth]{/Users/mmallek/Documents/Thesis/Plots/covcond-byscenario/1720-boxplots.png}
  } \\
    \subfloat[][]{
    \centering
    \includegraphics[width=0.4\textwidth]{/Users/mmallek/Documents/Thesis/Plots/covcond-byscenario/1721-boxplots.png}
  }%
  \qquad
  \subfloat[][]{
    \includegraphics[width=0.4\textwidth]{/Users/mmallek/Documents/Thesis/Plots/covcond-byscenario/1722-boxplots.png}
  } \\
    \subfloat[][]{
    \centering
    \includegraphics[width=0.4\textwidth]{/Users/mmallek/Documents/Thesis/Plots/covcond-byscenario/1730-boxplots.png}
  }%
      \subfloat[][]{
    \centering
    \includegraphics[width=0.4\textwidth]{/Users/mmallek/Documents/Thesis/Plots/covcond-byscenario/1731-boxplots.png}
  } \\
  \qquad
  \subfloat[][]{
    \includegraphics[width=0.4\textwidth]{/Users/mmallek/Documents/Thesis/Plots/covcond-byscenario/1732-boxplots.png}
  } 
    \qquad
  \subfloat[][]{
    \includegraphics[width=0.4\textwidth]{/Users/mmallek/Documents/Thesis/Plots/covcond-frvhrv/RFRM-frvhrv-boxplots.png}
  }
    \caption{Boxplots illustrating the range of variability across historical and future climate trajectories. The horizontal black bar represents the current condition. Boxplot whiskers extend from the $5^{th}$ to $95^{th}$ range of variability for each model. }
  \label{fig:covcond_rfrm}
\end{figure} %rfrm

\paragraph{Red Fir - Xeric} Xeric red fir forests in the early seral stage were within both the simulated historical and future ranges of variability, which were fairly similar to one another with the exception of the ESM2M model. The mid development, closed canopy seral stage is currently far outside the HRV and 4 of 7 future scenario RVs. The 90\% range of variability in the future scenarios in some models encompasses 0\% to over 75\%, but the medians are all rather low, around 10\%, so being within the RV should be viewed in this context. Moderate canopy forests exhibit less extreme behavior, though they are all outside the HRV and far more prevalent that projected under either the HRV or FRV. Open canopies are less common than under the HRV or most FRV scenarios, but is within the FRV for all future scenarios. Late development closed and moderate canopy cover is within the HRV and FRV, generally speaking, but in both cases the FRV medians are well below the HRV, so the current proportion is on the high end of the range. Conversely, late development open canopies are rare on the current landscape. They were more common durin gthe HRV and much more important in all of the FRV scenarios. The current condition is completely outside any simulated range of variability. 

When we pooled the future scenario results, we observe a range of variability under the future condition that often overlaps the much smaller range of variability in the historical scenario. Medians differ substantially between the future and historical scenarios in the late closed and moderate canopy cover stages (HRV projected proportion is higher) and the late open stage (HRV projected proportion is lower) (Figure \ref{fig:covcond_rfrx}).

\begin{figure}[htbp]
  \centering
  \subfloat[][]{
    \centering
    \includegraphics[width=0.4\textwidth]{/Users/mmallek/Documents/Thesis/Plots/covcond-byscenario/1910-boxplots.png}
  }%
  \qquad
  \subfloat[][]{
    \includegraphics[width=0.4\textwidth]{/Users/mmallek/Documents/Thesis/Plots/covcond-byscenario/1920-boxplots.png}
  } \\
    \subfloat[][]{
    \centering
    \includegraphics[width=0.4\textwidth]{/Users/mmallek/Documents/Thesis/Plots/covcond-byscenario/1921-boxplots.png}
  }%
  \qquad
  \subfloat[][]{
    \includegraphics[width=0.4\textwidth]{/Users/mmallek/Documents/Thesis/Plots/covcond-byscenario/1922-boxplots.png}
  } \\
    \subfloat[][]{
    \centering
    \includegraphics[width=0.4\textwidth]{/Users/mmallek/Documents/Thesis/Plots/covcond-byscenario/1930-boxplots.png}
  }%
   \subfloat[][]{
    \includegraphics[width=0.4\textwidth]{/Users/mmallek/Documents/Thesis/Plots/covcond-byscenario/1931-boxplots.png}
  } \\
  \qquad
  \subfloat[][]{
    \includegraphics[width=0.4\textwidth]{/Users/mmallek/Documents/Thesis/Plots/covcond-byscenario/1932-boxplots.png}
  }
    \qquad
  \subfloat[][]{
    \includegraphics[width=0.4\textwidth]{/Users/mmallek/Documents/Thesis/Plots/covcond-frvhrv/RFRX-frvhrv-boxplots.png}
 }
    \caption{Boxplots illustrating the range of variability across historical and future climate trajectories. The horizontal black bar represents the current condition. Boxplot whiskers extend from the $5^{th}$ to $95^{th}$ range of variability for each model. }
  \label{fig:covcond_rfrx}
\end{figure} %rfrx

\paragraph{Mixed Evergreen Forest} Mesic and xeric mixed evergreen forests had less early during the HRV than on the current landscape, but in 4 of 7 climate scenarios\todo{Is it ok to say scenario when they are really sets of runs? Do I just add a note to clarify? Saying 'runs' feels weird.} the current proportion is within the FRV. In all cases the proportion of mixed evergreen forests in the mid development stage is higher now than during either the historic or future simulated period. For the late moderate and open seral stages, the current proportion of these forests is at the limit of the HRV and the FRV. In the case of the late open stage, the cover type is near the HRV, but under the future climate trajectories the amount of open is projected to increase beyond the HRV value. During the simulated HRV, the late closed stage dominated the lansdcape, but during all of our future scenarios, that seral stage is currently within the FRV. 

When we look at the consolidated future scenarios versus the historical scenario, we find that the proportion of the landscape currently in a given seral stage is always outside the simulated HRV, and within the simulated FRV for the early seral stage and the late development closed stage only. HRV and FRV results have similar medians for the early, middle and late development closed stages, although the FRV has a much wider range. A greater proportion of the both the mesic and xeric mixed evergreen forests are projected to be in late open in the FRV compared to the HRV, and the reverse is true for the late closed canopy stage (Figures \ref{fig:covcond_megm} and \ref{fig:covcond_megx}).

\begin{figure}[htbp]
  \centering
  \subfloat[][]{
    \centering
    \includegraphics[width=0.4\textwidth]{/Users/mmallek/Documents/Thesis/Plots/covcond-byscenario/910-boxplots.png}
  }%
  \qquad
  \subfloat[][]{
    \includegraphics[width=0.4\textwidth]{/Users/mmallek/Documents/Thesis/Plots/covcond-byscenario/920-boxplots.png}
  } \\
    \subfloat[][]{
    \centering
    \includegraphics[width=0.4\textwidth]{/Users/mmallek/Documents/Thesis/Plots/covcond-byscenario/921-boxplots.png}
  }%
  \qquad
  \subfloat[][]{
    \includegraphics[width=0.4\textwidth]{/Users/mmallek/Documents/Thesis/Plots/covcond-byscenario/922-boxplots.png}
  } \\
    \subfloat[][]{
    \centering
    \includegraphics[width=0.4\textwidth]{/Users/mmallek/Documents/Thesis/Plots/covcond-byscenario/930-boxplots.png}
  }%
      \subfloat[][]{
    \centering
    \includegraphics[width=0.4\textwidth]{/Users/mmallek/Documents/Thesis/Plots/covcond-byscenario/931-boxplots.png}
  } \\
  \qquad
  \subfloat[][]{
    \includegraphics[width=0.4\textwidth]{/Users/mmallek/Documents/Thesis/Plots/covcond-byscenario/932-boxplots.png}
  }
    \qquad
  \subfloat[][]{
    \includegraphics[width=0.4\textwidth]{/Users/mmallek/Documents/Thesis/Plots/covcond-frvhrv/MEGM-frvhrv-boxplots.png}
  }
    \caption{Boxplots illustrating the range of variability across historical and future climate trajectories. The horizontal black bar represents the current condition. Boxplot whiskers extend from the $5^{th}$ to $95^{th}$ range of variability for each model. }
  \label{fig:covcond_megm} %megm
\end{figure} %megm

\begin{figure}[htbp]
  \centering
  \subfloat[][]{
    \centering
    \includegraphics[width=0.4\textwidth]{/Users/mmallek/Documents/Thesis/Plots/covcond-byscenario/1110-boxplots.png}
  }%
  \qquad
  \subfloat[][]{
    \includegraphics[width=0.4\textwidth]{/Users/mmallek/Documents/Thesis/Plots/covcond-byscenario/1120-boxplots.png}
  } \\
    \subfloat[][]{
    \centering
    \includegraphics[width=0.4\textwidth]{/Users/mmallek/Documents/Thesis/Plots/covcond-byscenario/1121-boxplots.png}
  }%
  \qquad
  \subfloat[][]{
    \includegraphics[width=0.4\textwidth]{/Users/mmallek/Documents/Thesis/Plots/covcond-byscenario/1122-boxplots.png}
  } \\
    \subfloat[][]{
    \centering
    \includegraphics[width=0.4\textwidth]{/Users/mmallek/Documents/Thesis/Plots/covcond-byscenario/1130-boxplots.png}
  }%
      \subfloat[][]{
    \centering
    \includegraphics[width=0.4\textwidth]{/Users/mmallek/Documents/Thesis/Plots/covcond-byscenario/1131-boxplots.png}
  } \\
  \qquad
  \subfloat[][]{
    \includegraphics[width=0.4\textwidth]{/Users/mmallek/Documents/Thesis/Plots/covcond-byscenario/1132-boxplots.png}
  }
    \qquad
  \subfloat[][]{
    \includegraphics[width=0.4\textwidth]{/Users/mmallek/Documents/Thesis/Plots/covcond-frvhrv/MEGX-frvhrv-boxplots.png}
  }
    \caption{Boxplots illustrating the range of variability across historical and future climate trajectories. The horizontal black bar represents the current condition. Boxplot whiskers extend from the $5^{th}$ to $95^{th}$ range of variability for each model. }
  \label{fig:covcond_megx} 
\end{figure} %megx 


\paragraph{Sierran Mixed Conifer - Ultramafic} Early seral ultramafic mixed conifer forests have, in general, wider 90\% ranges of variability in the future scenarios than the historical scenario. are above and outside the simulated RV for all scenarios except the most extreme future model, ESM2M. The mid development, closed canopy stage is currently within all FRVs except the ESM2M RV, and the moderate canopy cover stage is within all simulated FRVs. Conversely, the mid development, open canopy stage is currently rare and outside the RV for all simulated scenarios, and the proportion of this seral stage is project to increase under all future scenarios. Late development, closed canopy ultramafic mixed conifer forest is currently quite prevalent and above the simulated RVs. It is projected to become even less common during the future scenarios than during the historical scenario. While the moderate canopy cover is below and outside the HRV, it falls within the simulated FRVs, which have medians at lower proportions than the HRV. The proportion of late development, open canopy is currently fairly low, but is projected to be more common across all simulated RVs. The spread of some of the future scenarios is great enough that the current conditions do fall within them. 

When we pooled the future scenario results, we found that the projected future proportions of oak-conifer forests and woodlands are much higher than the HRV for the mid development open stage. This is compensated for by a decrease in the simulated FRV of the proportion of the cover type in the other stages, particularly the late development stages (Figure \ref{fig:covcond_smcu}).

\begin{figure}[htbp]
  \centering
  \subfloat[][]{
    \centering
    \includegraphics[width=0.4\textwidth]{/Users/mmallek/Documents/Thesis/Plots/covcond-byscenario/2510-boxplots.png}
  }%
  \qquad
  \subfloat[][]{
    \includegraphics[width=0.4\textwidth]{/Users/mmallek/Documents/Thesis/Plots/covcond-byscenario/2520-boxplots.png}
  } \\
    \subfloat[][]{
    \centering
    \includegraphics[width=0.4\textwidth]{/Users/mmallek/Documents/Thesis/Plots/covcond-byscenario/2521-boxplots.png}
  }%
  \qquad
  \subfloat[][]{
    \includegraphics[width=0.4\textwidth]{/Users/mmallek/Documents/Thesis/Plots/covcond-byscenario/2522-boxplots.png}
  } \\
    \subfloat[][]{
    \centering
    \includegraphics[width=0.4\textwidth]{/Users/mmallek/Documents/Thesis/Plots/covcond-byscenario/2530-boxplots.png}
  }%
      \subfloat[][]{
    \centering
    \includegraphics[width=0.4\textwidth]{/Users/mmallek/Documents/Thesis/Plots/covcond-byscenario/2531-boxplots.png}
  } \\
  \qquad
  \subfloat[][]{
    \includegraphics[width=0.4\textwidth]{/Users/mmallek/Documents/Thesis/Plots/covcond-byscenario/2532-boxplots.png}
  }
    \qquad
  \subfloat[][]{
    \includegraphics[width=0.4\textwidth]{/Users/mmallek/Documents/Thesis/Plots/covcond-frvhrv/SMCU-frvhrv-boxplots.png}
  }
    \caption{Boxplots illustrating the range of variability across historical and future climate trajectories. The horizontal black bar represents the current condition. Boxplot whiskers extend from the $5^{th}$ to $95^{th}$ range of variability for each model. }
  \label{fig:covcond_smcu}
\end{figure} %smcu



\paragraph{Oak-Conifer Forest and Woodland - Ultramafic}\todo{I actually think this cover type is too small and uninteresting to bother covering. Looking for affirmation.}

\begin{figure}[htbp]
  \centering
  \subfloat[][]{
    \centering
    \includegraphics[width=0.4\textwidth]{/Users/mmallek/Documents/Thesis/Plots/covcond-byscenario/1510-boxplots.png}
  }%
  \qquad
  \subfloat[][]{
    \includegraphics[width=0.4\textwidth]{/Users/mmallek/Documents/Thesis/Plots/covcond-byscenario/1520-boxplots.png}
  } \\
    \subfloat[][]{
    \centering
    \includegraphics[width=0.4\textwidth]{/Users/mmallek/Documents/Thesis/Plots/covcond-byscenario/1521-boxplots.png}
  }%
  \qquad
  \subfloat[][]{
    \includegraphics[width=0.4\textwidth]{/Users/mmallek/Documents/Thesis/Plots/covcond-byscenario/1522-boxplots.png}
  } \\
    \subfloat[][]{
    \centering
    \includegraphics[width=0.4\textwidth]{/Users/mmallek/Documents/Thesis/Plots/covcond-byscenario/1530-boxplots.png}
  }%
      \subfloat[][]{
    \centering
    \includegraphics[width=0.4\textwidth]{/Users/mmallek/Documents/Thesis/Plots/covcond-byscenario/1531-boxplots.png}
  } \\
  \qquad
  \subfloat[][]{
    \includegraphics[width=0.4\textwidth]{/Users/mmallek/Documents/Thesis/Plots/covcond-byscenario/1532-boxplots.png}
  }
    \qquad
  \subfloat[][]{
    \includegraphics[width=0.4\textwidth]{/Users/mmallek/Documents/Thesis/Plots/covcond-frvhrv/OCFWU-frvhrv-boxplots.png}
  }
    \caption{Boxplots illustrating the range of variability across historical and future climate trajectories. The horizontal black bar represents the current condition. Boxplot whiskers extend from the $5^{th}$ to $95^{th}$ range of variability for each model. }
  \label{fig:covcond_ocfwu}
\end{figure} %ocfwu







