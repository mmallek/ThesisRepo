\documentclass[12pt]{article}

%\usepackage{geometry}
%\geometry{verbose,letterpaper,tmargin=2.54cm,bmargin=2.54cm,lmargin=2.54cm,rmargin=2.54cm}

\usepackage[left=1in,right=1in,top=1in,bottom=1in]{geometry}
%\usepackage[table,xcdraw]{xcolor}

%\usepackage{booktabs}
%\usepackage[utf8]{inputenc}
%\usepackage{amsmath}
%\usepackage{graphicx}
\usepackage{todonotes}
\usepackage{setspace}
%\usepackage{listings}
%\usepackage{hyperref}
\usepackage{natbib}
%\usepackage{wrapfig,subfig,graphicx}


%\usepackage{lineno}
%\linenumbers

%\doublespacing

\title{Effect of climate change on future landscape composition and configuration, Tahoe National Forest, California, USA}
\author{ Maritza Mallek }
\date{\today}

\begin{document}
\maketitle
%\begin{spacing}{1.9}
\begin{spacing}{1}

%\begin{flushleft}

\section*{Abstract}


\section*{Introduction}


\subsection{Test}
% 1. Importance of fire in western forests, especially the Sierra Nevada
In the Sierra Nevada, cycles of fire and vegetation recovery occur variably over large extents, as well as over long periods of time. Ongoing disturbance on a landscape results in increased heterogeneity captured by various metrics used to describe vegetation composition and configuration \citep{Monica2008}. Prior to European settlement, wildfire was the major source of disturbance in Sierran forests, shaping the composition and configuration of vegetation communities. Fires were primarily lightning-caused, although indigenous peoples are thought to have set fires for vegetation management, especially in the lower elevations. In general, fire was frequent, with a mean rotation as short as 20 years in Ponderosa Pine-dominated forests. Wetter mixed conifer areas are predicted to have had a mean fire rotation of 30 years. Fire rotation is thought to increase gradually with elevation. For example, mesic Red Fir forests, which exist around 2,000 feet higher in elevation than Ponderosa Pine forests, had a mean fire rotation of 60 years. Variance around these means can be significant, as some parts of the forest experience fire much more frequently, while other escape fire for long periods. 

In general, regardless of vegetation type, high mortality fires were thought to be rare, with the vast majority of fires killing under 70\% of overstory trees. Under this disturbance regime, stand-replacing fire initiated early development conditions on the landscape. Low mortality fire tended to open forest canopies, especially in more xeric parts of the forest, while vegetation succession closed them again. The rarity of high mortality fire allowed large forest stands to succeed into late development and old growth conditions \citep{SNEP1996,Mallek2013,Safford2014,SNEP1996a}. Since then, fire suppression, logging, grazing, and mining have all interacted to alter the historical fire regime and vegetation patterns \citep{Stephens2015,Knapp2013}. For the drier forest types within this landscape, frequent fires (usually having low mortality) were the norm. After large-scale fire suppression became the norm in the second half of the 19th century, less fire-tolerant species (such as Douglas fir and white fir) have come to dominate areas where they were once a minor part of the vegetation community. Grazing and development made fires less common by altering or removing the fine fuels that carried fire. Timber harvest, especially of fire-tolerant species such as ponderosa and sugar pines, accelerated the increased cover of species such as white fir. Finally, fire suppression allowed the buildup of medium size fuels and ladder fuels, which promotes larger and hotter fires when they do occur. Moreover, the lack of natural fires has meant that variation in fuel loading has decreased, which allows large fires to spread over very large areas \citep{Hessburg2005}.

% 2. relevance to planning needs


With the popularization (and to some extent forced adoption) of ecosystem management in the early 1990s, the need to recognize ecosystems as dynamic and constantly-changing became well accepted, and calls to manage forests sustainably became common (Chistensen et al. 1996). Within the context of forest and land management planning, the restoration of ecosystems to their pre-European settlement states was incorporated as a goal or desired future condition into various plans, including the Sierra Nevada Ecosystem Project \cite{SNEP1996a}. By 2000, the U.S. Forest Service's formal Planning Rule explicitly required the agency to manage for ecological characteristics within the range of variability expected under natural disturbance regimes (federal register). The need to consider the natural range of variability was maintained through various amendments to the rule, and is still present in the new 2012 rule, finalized in early 2015 (federal register). 


% 3. Historic range of variability analysis, models, fragstats

Methods for quantifying the natural range of variability for a diversity of landscapes in the United States augmented the development of research focused on this task \citep{Landres1999}. Of these, simulation of the historical dynamics became fairly popular. By 2004, some 45 landscape fire and succession models alone had been developed \citep{Keane2004}. Many of these, such as \textsc{landis} \citep{He1999}, \textsc{zelig-l} \citep{Miller1999}, \textsc{safe-forests} (Sessions et al. 1997) and LANDSUM \citep{Keane2012} are still in use today. Landscape fire and succession models are used to create spatially-explicit simulations of both of these key forest processes, typically outputting a set of GIS layers for each timestep of the model that can then be analyzed to quantify trajectories and patterns in the disturbance regime, seral stage composition, and landscape configuration over time \citep{Keane2004}. Most historical range of variability analyses in the United States have focused on the Rocky Mountains and the Oregon Coast Range. Only one historical range of variability study has been carried out within the Sierra Nevada \citep{Miller1999}, which took place in Sequoia National Park in the southern Sierra. Most information about the historical range of variability of Sierran wildfire regimes and vegetation was compiled as part of the Sierra Nevada Ecosystem Project \citep{SNEP1996a}. 

% Prior Work
In a previous study, we adapted a landscape fire and succession model, \textsc{RMLands}, developed for the Rocky Mountains, to the Sierra Nevada (Mallek, unpublished thesis). As part of this work we quantified the historical range of variability (HRV) for a sublandscape on the Tahoe National Forest. Although many such range of variability analyses have been completed, research that offers a complementary analysis of future scenarios under climate change are rare (but see \cite{Keane2008} and \cite{Duveneck2014}). Concern about the impact of changes to precipitation and temperature anticipated under climate change in the northern Sierra on local disturbance regimes, and subsequently, seral stage distribution and patch configuration motivates studies that look not only at the current and historical conditions, but also into the future \citep{Fule2008,North2012}. Quantifying and describing a ``Future Range of Variability'' (FRV) can inform how realistic restoration toward an HRV may be \citep{Duncan2010}.

% Objectives
In this study, our objectives were to evaluate the effect of climate change on the wildfire regime and landscape composition and configuration for the area previously studied on the Tahoe National Forest. We held model parameters constant but changed the climate parameter, incorporating Palmer Drought Severity Index (PDSI) values from a suite of seven climate trajectories developed by the National Center for Atmospheric Research (USA) and the Canadian Centre for Climate Modelling and Analysis to the year 2090 \citep{Cook2014}. We used \textsc{Fragstats} software and R to analyze outputs and report the 90\% range of variability for both simulated historical and future metrics. Ultimately, we evaluate our results for a series of simulations for these future scenarios to the historical range of variability and to the current conditions. 



\section*{Methods}
\subsection*{LDSMs, PDSI}




\subsection*{Study area}
The project landscape (see Figure~\ref{projectarea}) is located on the northern part of the Tahoe National Forest, on the Yuba River and Sierraville Ranger Districts, and comprises about 181,550 hectares. The topography of the project landscape consists of rugged mountains incised by two major and a few minor river drainages. Elevation ranges from about 350 to 2500 meters. The area receives 30--260 cm of precipitation annually, most of which falls as snow in the middle to upper elevations (Storer and Usinger 1963). Some areas in the mid-elevation band receive high precipitation compared to the region, resulting in patches of exceptionally productive forest \citetext{Alan\ Doerr,\ pers.\ comm.}. Vegetation is tremendously diverse and changes slowly along an elevational gradient and in response to local changes in drainage, aspect, and soil structure. Grasslands, chaparral, oak woodlands, mixed conifer forests, and subalpine forests are all found within the study area. 


\subsection*{RMLands, parameterization}

% model
\textsc{RMLands} is a spatially-explicit, stochastic, landscape-level disturbance and succession model capable of simulating fine-grained processes over large spatial and long temporal extents \citep{McGarigal2005}. It is grid-based but simulates disturbance and succession at the patch level, and employs a dichotomous high vs. low mortality effect of fire at the cell level in place of the uniformly low, mixed, and high severity regimes used elsewhere \citep{McGarigal2012}. Originally developed for use in the Rocky Mountains of southern Colorado to provide a quantitative description of HRV \citep{McGarigal2012}, \textsc{RMLands} has also been used to simulate wildfire and vegetation succession in northern Idaho \citep{Cushman2011}. 

% on mortality/severity
Some fire ecologists combine fire attributes such as flame length and fire size into their interpretation of the relative "severity" of a particular fire (Agee 1993). However, in our model the relevance of severity is the outcome of the fire: does it reset the area burned to an early seral state, open the stand, or have no effect on the overstory?  Ecologists working at other scales and not working with models often describe ``mixed severity'' regimes (Kane et al 2012), which Collins and Stephens (2010) define as ``stand-replacing patches within a matrix of low to moderate fire-induced effects.'' Because at the 30 m cell size of our model, most real fires would be classified as ``mixed severity'' by the prior definition, it becomes moot. The most consensus exists on the definition of high severity (but see X and X for discussion on the appropriate cutoff value), and so we elected to separate fires into two categories: those resulting in high mortality (which we define as over 75\%) and those that do not.

% input layers
In collaboration with USDA Forest Service Ecology program staff and Tahoe National Forest staff, we developed a system of land cover and seral stage classification based on LandFire. We used Forest Service corporate spatial data as inputs to the model. State and transition models for all 31 land cover types were developed based on the VDDT models associated with the LandFire project, and refined with input from local experts \todo{how to cite? list everyone?}. [Mallek 2015] contains a detailed description of input spatial data and model parameters. In general model parameters were developed using meta-analyses published in the literature. For example, LandFire data was used to calculate transition probabilities, and wind direction information was obtained from multiple area weather stations.

% climate parameter
A climate variable unique to each timestep in the model is the key parameter that varies across the scenarios in this study. The Palmer Drought Severity Index forms the basis of this parameter. PDSI calculations from the period 1550-1850 (the three hundred years prior to European establishment) were acquired from the National Oceanic and Atmosphereic Administration. Values from multiple points were used on an inverse distance-weighted basis, which were then rescaled to a mean of 1 and inverted so that they could be used as a multiplier within \textsc{RMLands}. Within the model, climate affects probabilities regulating initiation, susceptibility, and spread. 

The Intergovernmental Panel on Climate Change (IPCC) Working Group III developed 40 different scenarios involving different assumptions about demographic, social, economic, technological, and environmental development, which are then grouped into four broad categories (Nakicenovic et al. 2000). This strategy was updated in 2013, as the IPCC Working Group I developed a new set of scenarios, so-called Representative Concentration Pathways (RCPs). The RCPs represent different targets for radiative forcing at the year 2100: 2.6, 4.5, 6.0 and 8.5 W m2 (Stocker et al. 2013). Cook et al. (2014) calculated future values of the Palmer Drought Severity Index (PDSI) fit to historical data through 2099, based on the RCP8.5 model outputs. They used the same methodology as the North American Climate Atlas (Cook et al. 2008). An examination of landscape composition and configuration under potential climate change scenarios is important because it can provide additional information about what restoration strategies are likely to remain resilient and make sense ecologically for the area under study. 

% model calibration
We calibrated the model by manipulating the ignition coefficient (number of attempted fire starts per timestep) and by adjusting the relative magnitude of the scale parameter in the Weibull cumulative distribution function that governs the susceptibility of each land cover type and seral stage [more details in appendix?]. We repeated this process until we achieved our predicted overall rotation values for the nine land cover types greater than 1000 ha within the project landscape. Visual analysis of the seral stage distribution was used to determine when the model was equilibrated; we have ommited the first 40 timesteps of the model from the results. No true equilibration was done for the future simulations, but we elected to include only the final 5 timesteps (25 years) of the results in order to achieve some distance from the starting (current) conditions, maintain focus on the ending trajectory of climate and the range of variation possible in the landscape at the end of the 21st century, yet still capture the variability inherent to the PDSI-based climate parameter.

\subsection*{Evaluating fire regimes}


\subsection*{Landscape composition}

* Looks like the absolute range of variability is a lot wider in the future than during the HRV!
* In some cases the landscape is out of HRV but within most of the FRVs


\subsection*{Landscape configuration}

\section*{Results}

\subsection*{Natural fire regime}
The average amount of land burned by wildfire during simulations of seven alternative future climate trajectories ranged from 13\% to 23\% of the landscape, while during the simulated historical period the average area burned was 18\% of the landscape. However, in the future scenarios, we observe a slight shift from low mortality to high mortality in terms of area burned. This shift is more pronounced in the scenarios with higher PDSI values. The effect holds when all future scenario outputs are combined, in which case the mean values for high mortality, low mortality, and any mortality area burned by fire exceed the values for the simulated historical period.




\subsection*{Climate scenarios}


\section*{Discussion}

\subsection*{Scope \& Limitations}
- This study looks at the effect of changing PDSI values on the disturbance regime and landscape composition and configuration of part of the Yuba River Watershed. 
- We utilize only one RCP, 8.5.
- PDSI is a proxy for climate, not a direct measure of it
- PDSI is measured monthly, and we collapsed PDSI into 5 year summer averages.
- We removed some of the variabilities and extremes that PDSI indicates by rescaling the data.
- Since we are looking at historical and future, we reduce the effect of the current conditions, and the influence of the past ~100 years of fire suppression on fuels and fire regimes, except where we compare historical and future conditions to current. Thus we cannot say directly that the observed disturbance regime would occur if we let fires burn.
- Still that does mean that we have isolated simply the effect of increased temperatures and decreased rainfall, which is predicted under climate change.
- It tells us how different the future would be from the historic period if natural fire regimes were allowed to occur. Since this is not the case, it is not a simple roadmap. It's more about resilience. 
- It also tell us if restoration toward a historical regime/composition/configuration appears likely to succeed under climate change, or whether the future is ``on the other side'' of the past.
- We do not simulate changes in the spatial configuration of land cover types. That is, we do not model upward movement of forest types or type conversions after fire or drought events. Many studies indicate that such change will occur in the next 100 years and even has occurred in many places (cite). Thus the areas in 'early development' may be especially susceptible to type conversions, especially by 2099.
- We do not adjust for differential ownership within the study area, such as timber company holdings (Nonaka and Spies 2005).



%\backmatter 
%\include{glossary} 
%\include{notat} 
\bibliographystyle{humannat}%amsalpha} %The style you want to use for references. 
\bibliography{bibliography} %The files containing all the articles and books you ever referenced. 
\setcitestyle{notesep={:},aysep={}} 

%\end{flushleft}
\end{spacing}
\end{document}