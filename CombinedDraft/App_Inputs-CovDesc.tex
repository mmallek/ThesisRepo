% !TEX root = master.tex
\appendix


\chapter{Input Layers to \textsc{RMLands}}
\label{app:inputs}

\paragraph{Technical details on \textsc{RMLands} Data Structure}
\label{app:rmlspecs}
\textsc{RMLands} uses raster GeoTiffs (.tif files) as its data structure. Rasters are based on uniform square units called cells (or pixels). Each cell represents an actual portion of geographic space. In this application, we use the Universal Transverse Mercator (UTM) projection, Zone 10 North. The extent of the raster is rectangular although the area of study is not. Cells outside of the buffered project area are assigned a null value.\footnote{Latitude and longitude are commonly pictured when describing coordinates. In such cases the X value refers to longitude and Y refers to latitude. However, because we use UTMs in this project, the correct convention is actually that the X value is the Easting and the Y value is the Northing. For simplicity we discuss X and Y only in this document.} In the Yuba River watershed landscape, each grid cell is 30 meters on a side (i.e., 900 m$^2$ or 0.09 ha), and the input grid measures 2910 by 2245 pixels. \textsc{RMLands} requires that all input grids are perfectly aligned. We accomplished this by setting the Extent and Snap Raster to the same parameters whenever we manipulated the layers in ArcMap. This ``base'' spatial layer was created by taking the primary elevation layer used on the Tahoe National Forest, resampling it to a 30 meter grid, and clipping its extent to match that of the buffered project area. Each cell is assigned a single class value, where valid class values are positive non-zero integers. Integer values are mapped to more descriptive class names using csv files with names identical to the grid name. All grids are created in ArcMap and saved as GeoTiff files before being loaded into to the model. 

\section{Input Layer Maps}
\label{app:sec:inputmaps}

% cover layer
\begin{figure}[!htbp]
\centering
\includegraphics[height=0.4\textheight]{/Users/mmallek/Tahoe/Report2/images/cover.png}
\caption{Cover Type Map for the project area. Also shows the 10 km buffer from the project area boundary. See Table~\ref{covertable} for full land cover type names.}
%\label{covermap}
\end{figure}

% condition layer
\begin{figure}[!htbp]
\centering
\includegraphics[height=0.4\textheight]{/Users/mmallek/Tahoe/Report2/images/condition.png}
\caption{Condition Class Map for the project area. Also shows the 10 km buffer from the project area boundary.} 
\label{conditionmap}
\end{figure}

% age layer
\begin{figure}[htbp]
\centering
\includegraphics[height=0.4\textheight]{/Users/mmallek/Tahoe/Report2/images/age.png}
\caption{Age map at Timestep 0 for the project area. Also shows the 10 km buffer from the project area boundary.} 
\label{agemap}
\end{figure}

% condition-age layer
\begin{figure}[htbp]
\centering
\includegraphics[height=0.4\textheight]{/Users/mmallek/Tahoe/Report2/images/condage.png}
\caption{Condition-Age map at Timestep 0 for the project area. Also shows the 10 km buffer from the project area boundary.} 
\label{condagemap}
\end{figure}

% tpi 
\begin{figure}[htbp]
\centering
\includegraphics[height=0.4\textheight]{/Users/mmallek/Tahoe/Report2/images/tpi.png}
\caption{Topographic Position Index for the project area. Also shows the 10 km buffer from the project area boundary.} 
\label{tpimap}
\end{figure}

% elevation layer
\begin{figure}[htbp]
\centering
\includegraphics[height=0.4\textheight]{/Users/mmallek/Tahoe/Report2/images/elevation.png}
\caption{Elevation for the project area. Also shows the 10 km buffer from the project area boundary.} 
\label{elevationmap}
\end{figure}

% slope layer
\begin{figure}[htbp]
\centering
\includegraphics[height=0.4\textheight]{/Users/mmallek/Tahoe/Report2/images/slope.png}
\caption{Slope for the project area, which ranges from flat to 126\%. Also shows the 10 km buffer from the project area boundary.} 
\label{slopemap}
\end{figure}

% aspect layer
\begin{figure}[htbp]
\centering
\includegraphics[height=0.4\textheight]{/Users/mmallek/Tahoe/Report2/images/aspect.png}
\caption{Aspect for the project area. Also shows the 10 km buffer from the project area boundary.} 
\label{aspectmap}
\end{figure}

% streams layer
\begin{figure}[htbp]
\centering
\includegraphics[height=0.4\textheight]{/Users/mmallek/Tahoe/Report2/images/streams.png}
\caption{Streams in the project area. Also shows the 10 km buffer from the project area boundary.} 
\label{streamsmap}
\end{figure}

%%%%%%%%%%%%%%%%%%%%%%%%%%%%%%%%%%%%%%%%%%%%%%%%%%%%%%%%%%%%%%%%%%%%%%%%%%%%%%%%%%%%%%%%%%%%%%%%%%%%%%%%%%%%%%%%%%%%%%%%%%%%%%%%%%%%%%%%%%%%%%%%%%%%%%%%%%%%%%%%%%%%%%%%%%%%%%%%%%%%%%%%%%%%%%%%%%%%%%%%%%%%%%%%%%%%%%%%%%%%%%%%%%%%%%%%%%%%%%%%%%%%%%%%%%%%%%%%%%%%%%%%%%%%%%%%%%%%%%%%%%%%%%%%%%%%%%%%%%%%%%%%%%%%%%%%%%%%%%%%%%%%%%%%%%%%%%%%%%%%%%%%%%%%%%%%%%%%%%%%%%%%%%%%%%%%%%%%%%%%%%%%%%%%%%%%
\chapter{Cover Type Descriptions}
\label{sec:covertypedesc}

% !TEX root = master.tex

\section{Big Sagebrush (SAGE)}
\label{sage-description}
\subsection*{General Information}

\subsubsection{Cover Type Overview}

\textbf{Big Sagebrush (SAGE)}
\newline
Crosswalks
\begin{itemize}
	\item EVeg: Regional Dominance Type 1
	\begin{itemize}
		\item Bitterbrush 
		\item Basin Sagebrush
		\item Great Basin Mixed Scrub
		\item Bitterbrush - Sagebrush
	\end{itemize}

	\item LandFire BpS Model
	\begin{itemize}
		\item 0610800 Inter-Mountain Basins Big Sagebrush Shrubland
	\end{itemize}

	\item Presettlement Fire Regime Type
	\begin{itemize}
		\item Big Sagebrush
	\end{itemize}
\end{itemize}

\noindent Reviewed by Michele Slaton, GIS Specialist, Inyo National Forest, USDA Forest Service

\subsubsection{Vegetation Description}
The Big Sagebrush landcover type is typified by large, open, discontinuous stands of \emph{Artemisia tridentata} of fairly uniform height. \emph{A. tridentata} tends to have a single short, thick, stem that branches into a nearly globular crown (Neal 1988). \emph{Ericameria nauseosa} is a frequent associate or co-dominant (LandFire 2007).

Shrub canopy cover generally ranges from very open, widely spaced, small plants to large, closely spaced plants with canopies touching. Cover may be greater at higher elevations and in areas receiving more precipitation. In addition to a deep root system, \emph{A. tridentata} has a well-developed system of lateral roots near the soil surface (LandFire 2007, Neal 1988). Consequently, well-established sagebrush plants exclude most other shrubs in an area up to three times their crown area. Forbs and graminoids are often more abundant beneath these crowns (Slaton pers. comm. 2013). This produces stands of shrubs of very uniform size and spacing (Neal 1988).

Often the habitat is composed of pure stands of \emph{A. tridentata}, but many stands include other species of \emph{Artemisia, Ericameria, Tetradymia, Ribes, Prunus, Cercocarpus,} and \emph{Purshia}. In communities not fully occupied by \emph{Artemisia}, various amounts of herbaceous understory are found. Perennial forb cover is usually less than 10\% with perennial grass cover reaching 20-25\% on the more productive sites. \emph{Pseudoroegneria spicata} may be a dominant species following replacement fires and a co-dominant after 20 years. \emph{Elymus elymoides} and \emph{Oryzopsis hymenoides} are common on more xeric sites. \emph{Festuca, Stipa, Poa}, and \emph{Leymus} are among the more common grasses. Percent cover and species richness of understory are determined by site limitations. \emph{Pinus monophylla} and \emph{Juniperus osteosperma} may be present, especially in areas protected from fire (Neal 1988, LandFire 2007).


\subsubsection{Distribution}
This widespread system is common to the Basin and Range province. It ranges in elevation from 900 m to 2450+ m (3000 ft - 8000+ ft) and occurs on well-drained soils on foothills, terraces, slopes, and plateaus. It is found on deeper soils (LandFire 2007).

\subsection*{Disturbances}


\subsubsection{Wildfire}
Wildfires tend to be high mortality, stand-replacing fires that initiate a process of post-fire forest succession. High mortality fires kill large as well as small shrubs, and may kill many of the forbs and grasses as well, although below-ground organs of at least some individual shrubs and herbs survive and re-sprout. 

Replacement fires generally occur where shrub canopy exceeds 25\% cover, or where grass cover is greater than 15\% and shrub cover is greater than 20\%. Surface fires occur in areas dominated by grasses but are otherwise uncommon (LandFire 2007). \emph{A tridentata} does not sprout after burning but most of the other shrubs common to the type do (Neal 1988). For the last several decades, post-settlement conversion to \emph{Bromus tectorum} has become common and results in changes to fire frequency and vegetation dynamics. Extended periods of fire suppression or absence can lead to \emph{P. monophylla-J. osteosperma} encroachment and subsequent decline of other shrubs and herbaceous plants (LandFire 2007). 

Estimates of fire rotations are available from the LandFire project and a review paper (LandFire 2007, Van de Water and Safford 2011). The LandFire project’s published fire return intervals are based on a series of associated models created using the Vegetation Dynamics Development Tool (VDDT). In VDDT, fires are specified concurrently with the transition that follows them. For example, a replacement fire causes a transition to the early development stage. In the RMLands model, such fires are classified as high mortality. However, in VDDT mixed severity fires may cause a transition to early development, a transition to a more open seral stage, or no transition at all. In this case, we categorize the first example as a high mortality fire, and the second and third examples as a low mortality fire. Based on this approach, we calculated fire rotations and the probability of high mortality fire for each of the three SAGE seral stages (Table~\ref{tab:sagedesc_fire}). 

\subsubsection{Other Disturbance}
Other disturbances are not currently modeled, but may, depending on the seral stage affected and mortality levels, reset patches to early development, maintain existing seral stages, or shift/accelerate succession to a more open seral stage. 

\begin{table}[]
\small
\centering
\caption{Fire rotations (years) and probability of high versus low mortality fires. Values were derived from BpS model 0610800 (LandFire 2007), Van de Water and Safford (2011), and Safford (pers. comm. 2013).}
\label{tab:sagedesc_fire}
\begin{tabular}{@{}lcc@{}}
\toprule
\textbf{Condition}         & \multicolumn{1}{l}{\textbf{Fire Rotation}} & \multicolumn{1}{l}{\textbf{\begin{tabular}[c]{@{}l@{}}Proportion \\ High Mortality\end{tabular}}} \\ \midrule
Target                     & 115      & n/a       \\
Early Development - All    & 200      & 0         \\
Mid Development - Moderate & 125      & 1         \\
Late Development - Closed  & 100      & 0.9       \\ \bottomrule
\end{tabular}
\end{table}



\subsection*{Vegetation Seral Stages}
We recognize three separate seral stages for SAGE: Early Development (ED), Mid Development - Moderate Canopy Cover (MDM), and Late Development - Closed Canopy Cover (LDC) (Figure~\ref{sage_transmodel}). Our seral stages are an alternative to ``successional'' classes that imply a linear progression of states and tend not to incorporate disturbance. The seral stages identified here are derived from a combination of successional processes and anthropogenic and natural disturbance, and are intended to represent a composition and structural condition that can be arrived at from multiple other conditions described for that landcover type. Thus our seral stages incorporate age, size, canopy cover, and vegetation composition. In general, the delineation of stages has originated from the LandFire biophysical setting model descriptive of a given landcover type; however, seral stages are not necessarily identical to the classes identified in those models.

\begin{figure}[htbp]
\centering
\includegraphics[width=0.8\textwidth]{/Users/mmallek/Documents/Thesis/statetransmodel/StateTransitionModel/shrub.png}
\caption{State and Transition Model for Big Sagebrush. Each dark grey box represents one of the three seral stages for this landcover type. Three stages of development are represented: early, middle, and late. We describe the middle development stage as characterized by moderate canopy cover and the late development stage as characterized by closed canopy cover, but these are not hard and fast rules. Transitions between states/seral stages may occur as a result of high mortality fire, low mortality fire, or succession. Specific pathways for each are denoted by the appropriate color line and arrow: red lines relate to high mortality fire, orange lines relate to low mortality fire, and green lines relate to natural succession.} 
\label{sage_transmodel}
\end{figure}

\subsubsection{Early Development (ED)}

\paragraph{Description} \emph{A. tridentata} does not sprout after burning but most of the other shrubs common to the type do. Consequently, for as long as 20 years after fire the vegetative community may be dominated by \emph{Chrysothamnus}, \emph{Tetradymia}, and grasses. A very hot fire in a degraded site may result in a seral community dominated by annual grasses and forbs. Perennial bunchgrasses frequently survive fires and become dominant (Neal 1988). Canopy cover is less than 40\%, but shrub cover may be as little as 10\%. Fuel loading is discontinuous (LandFire 2007).

\paragraph{Succession Transition} In the absence of disturbance, patches in this seral stage will transition to MDM at 20 years. 

\paragraph{Wildfire Transition} High mortality wildfire is not modeled for this seral stage. Low mortality wildfire (100\% of fires in this seral stage) maintains the patch in the ED seral stage. 

\noindent\hrulefill


\subsubsection{Mid Development - Moderate Canopy Cover (MDM)}

\paragraph{Description} \emph{A. tridentata} usually reaches fairly stable dominance 10 to 20 years after disturbance, with or without an understory of perennial bunchgrass. \emph{A. tridentata} usually remains dominant indefinitely or until the next disturbance (Neal 1988). Shrub density is sufficient in old stands to carry the fire without fine fuels. Shrubs and herbaceous vegetation can be codominant. Generally, shrub cover averages 30\% (LandFire 2007).

\paragraph{Succession Transition} At 40 years without disturbance, patches in this seral stage will transition to LDC. 

\paragraph{Wildfire Transition} High mortality wildfire (90\% of fires in this seral stage) recycles the patch through the ED seral stage. Low mortality wildfire (10\%) maintains the patch in the MDM seral stage.

\noindent\hrulefill


\subsubsection{Late Development - Closed Canopy Cover (LDC)}

\paragraph{Description} Shrublands with some encroachment from \emph{P. monophylla} and \emph{J. osteosperma} possible. Wildfire has not occurred for at least 60 years. Tree species cover is highly variable. In the continued absence of disturbance, shrub cover will decline (LandFire 2007).

\paragraph{Succession Transition} In the absence of disturbance, patches in this seral stage will maintain. 

\paragraph{Wildfire Transition} High mortality wildfire (100\% of fires in this seral stage) recycles the patch through the ED seral stage. Low mortality wildfire is not modeled for this seral stage.

\noindent\hrulefill

\subsection*{Condition Classification}
Because seral stage classification was done through orthophoto analysis, no polygons are assigned to the LDC seral stage, which is actually not an \emph{Artemisia}-dominated seral stage. Polygons with a cover value (not Null) are assigned to the MDM stage. Polygons with a Null value for shrub cover are assigned to ED.



\clearpage
\subsection*{References}
\begin{hangparas}{.25in}{1} 
LandFire. ``Biophysical Setting Models.'' Biophysical Setting 0610800: Inter-Mountain Basins Big Sagebrush Shrubland. 2007. LANDFIRE Project, U.S. Department of Agriculture, Forest Service; U.S. Department of the Interior. \burl{http://www.landfire.gov/national_veg_models_op2.php}. Accessed 9 November 2012.

Neal, Donald L. ``Sagebrush (SGB).'' \emph{A Guide to Wildlife Habitats of California}, edited by Kenneth E. Mayer and William F. Laudenslayer. California Deparment of Fish and Game, 1988. \burl{http://www.dfg.ca.gov/biogeodata/cwhr/pdfs/SGB.pdf}. Accessed 4 December 2012.

Safford, Hugh. Regional Ecologist, USDA Forest Service. Personal communication, 15 August 2013.

Van de Water, Kip M. and Hugh D. Safford. ``A Summary of Fire Frequency Estimates for California Vegetation Before Euro-American Settlement.'' \emph{Fire Ecology} 7.3 (2011): 26-57. doi: 10.4996/fireecology.0703026.
\end{hangparas}


% !TEX root = master.tex
\newpage
\section{Black and Low Sagebrush (LSG)}
\label{lsg-description}

\subsection*{General Information}

\subsubsection{Cover Type Overview}

\textbf{Black and Low Sagebrush (LSG)}
\newline
\textbf{Crosswalks}
\begin{itemize}
	\item EVeg: Regional Dominance Type 1
	\begin{itemize}
		\item Low Sagebrush
		\item Black Sagebrush
	\end{itemize}

	\item LandFire BpS Model
	\begin{itemize}
		\item 0610790: Great Basin Xeric Mixed Sagebrush Shrubland
	\end{itemize}

	\item Presettlement Fire Regime Type
	\begin{itemize}
		\item Black and Low Sagebrush
	\end{itemize}
\end{itemize}

\noindent Reviewed by Michele Slaton, GIS Specialist, Inyo National Forest, USDA Forest Service

\subsubsection{Vegetation Description}
\paragraph{Black and Low Sagebrush (LSG)}	This landcover type is generally dominated by broad-leaved, evergreen shrubs of short stature, typically averaging about 15\% cover. Depending on site conditions, crowns may touch. Deciduous shrubs and small trees are sometimes sparsely scattered within this type. The ground cover of grasses and forbs is typically a sparse 5-15\% cover (Verner 1988). LSG may be dominated by either \emph{Artemisia arbuscula} or \emph{Artemisia nova}, often in association with \emph{Chrysothamnus viscidiflorus}, \emph{Purshia tridentata}, or \emph{Artemisia tridentata}; \emph{A. nova} is also commonly associated with \emph{Krascheninnikovia} and \emph{Ephedra}. \emph{Juniperus occidentalis} may be sparsely scattered in stands dominated by \emph{Artemisia arbuscula}, and \emph{Juniperus osteosperma} and \emph{Pinus monophylla} are sometimes scattered in stands dominated by \emph{A. nova}. A rich variety of forbs is usually present, including \emph{Eriogonum}, \emph{Erigeron}, \emph{Phlox}, \emph{Castilleja}, \emph{Sphaeralcea}, and \emph{Lupinus}. Common grasses include \emph{Poa}, \emph{Pseudoroegneria}, \emph{Elymus}, \emph{Stipa} and \emph{Festuca}. The abundance and distribution of associated plants is highly influenced by soils and precipitation (Verner 1988, LandFire 2007).

\subsubsection{Distribution}
Stands of \emph{A. arbuscula} are usually found on shallow soils with impaired drainage in the transition zone between the wetter bottom and open timber on the mountainsides. The type also occurs on terraces with hardpan or heavy clay soils. In mosaics formed with \emph{P. tridentata}, \emph{A. arbuscula} occurs on harsher sites with shallow, well-drained soils, while \emph{P. tridentata} occupies areas with deeper soils. Soils typically associated with stands of \emph{A. nova} are shallow, contain a high percentage of gravel, and are rich in mineral carbonates. It is prevalent on limestone soils (Verner 1988).

\emph{A. arbuscula} communities are generally restricted to elevated arid plains along the eastern flanks of the Sierra Nevada. \emph{A. nova} can occur in subalpine areas, at elevations above 2420 m (8000 ft). Stands dominated by \emph{A. arbuscula} range in elevation from 1210 to 2740 m (4000-9000 ft) (Verner 1988).


\subsection*{Disturbances}

\subsubsection{Wildfire}
Wildfires tend to be high mortality, stand-replacing fires that initiate a process of post-fire forest succession. High mortality fires kill large as well as small trees, and may kill many of the shrubs and herbs as well, although below-ground organs of at least some individual shrubs and herbs survive and re-sprout. 

\emph{A. nova} generally supports more fire than other dwarf sagebrushes. Stand-replacing fire is rare due to relatively low fuel loads and herbaceous cover. Bare ground acts as a micro-barrier to fire between low-statured shrubs. Stand-replacing fires can occur in this type when successive years of above average precipitation are followed by an average or dry year. Stand-replacing fires predominate in the late successional class where the herbaceous component has diminished or where trees dominate (LandFire 2007).

Although it is not included in this iteration of the model, scientists have noted that \emph{Bromus tectorum} has invaded most of these communities, altering successional pathways and disturbance regimes. It burns readily and is an early-season post-fire colonizer (Verner 1988).

Estimates of fire rotations are available from the LandFire project and a review paper (LandFire 2007, Van de Water and Safford 2011). The LandFire project’s published fire return intervals are based on a series of associated models created using the Vegetation Dynamics Development Tool (VDDT). In VDDT, fires are specified concurrently with the transition that follows them. For example, a replacement fire causes a transition to the early development stage. In the RMLands model, such fires are classified as high mortality. However, in VDDT mixed severity fires may cause a transition to early development, a transition to a more open seral stage, or no transition at all. In this case, we categorize the first example as a high mortality fire, and the second and third examples as a low mortality fire. Based on this approach, we calculated fire rotations and the probability of high mortality fire for each of the three LSG seral stages (Table~\ref{tab:lsgdesc_fire}). We computed the overall target fire rotation of 82 years based on values from Van de Water and Safford (2011). 




\begin{table}[!htbp]
\footnotesize
\centering
\caption{Fire rotation index values and probability of high severity fire (at least 75\% overstory tree mortality) probabilities. The seral stage that is most susceptible to fire (i.e., has the lowest predicted fire rotation) has a fire rotation index value of 1. Higher values correspond with lower likelihoods of experiencing wildfire. The values are relative only within an individual seral stage and should not be compared against other land cover types. Values were derived from VDDT model 0610790 (LandFire 2007) and Van de Water and Safford (2011). }
\label{tab:lsgdesc_fire}
\begin{tabular}{@{}lcc@{}}
\toprule
 \textbf{Seral Stage}    & \textbf{\begin{tabular}[c]{@{}c@{}}Fire Rotation \\ Index\end{tabular}} & \textbf{\begin{tabular}[c]{@{}c@{}}Probability of \\ High Severity Fire\end{tabular}} \\ \hline
Early Development - All    & 4.0     & 1      \\
Mid Development - Moderate & 1.0     & 1      \\
Late Development - Closed  & 2.4     & 0.31    \\ 
\emph{Target Fire Rotation}    			& \emph{82 years}  &   \\ 
\bottomrule
\end{tabular}
\end{table}

\subsubsection{Other Disturbance}
Other disturbances are not currently modeled, but may, depending on the seral stage affected and mortality levels, reset patches to early development, maintain existing seral stages, or shift/accelerate succession to a more open seral stage. 

\subsection*{Vegetation Seral Stages}
We recognize three separate seral stages for LSG: Early Development (ED), Mid Development - Moderate Canopy Cover (MDM), and Late Development - Closed Canopy Cover (LDC) (Figure~\ref{lsg_transmodel}). Our seral stages are an alternative to ``successional'' classes that imply a linear progression of states and tend not to incorporate disturbance. The seral stages identified here are derived from a combination of successional processes and anthropogenic and natural disturbance, and are intended to represent a composition and structural condition that can be arrived at from multiple other conditions described for that landcover type. Thus our seral stages incorporate age, size, canopy cover, and vegetation composition. In general, the delineation of stages has originated from the LandFire biophysical setting model descriptive of a given landcover type; however, seral stages are not necessarily identical to the classes identified in those models.


\begin{figure}[htbp]
\centering
\includegraphics[width=0.8\textwidth]{/Users/mmallek/Documents/Thesis/statetransmodel/StateTransitionModel/shrub.png}
\caption{State and Transition Model for Black and Low Sagebrush. Each dark grey box represents one of the three seral stages for this landcover type. Three stages of development are represented: early, middle, and late. We describe the middle development stage as characterized by moderate canopy cover and the late development stage as characterized by closed canopy cover, but these are not hard and fast rules. Transitions between states/seral stages may occur as a result of high mortality fire, low mortality fire, or succession. Specific pathways for each are denoted by the appropriate color line and arrow: red lines relate to high mortality fire, orange lines relate to low mortality fire, and green lines relate to natural succession.} 
\label{lsg_transmodel}
\end{figure}

\subsubsection{Early Development (ED)} 

\paragraph{Description} Early seral community dominated by herbaceous vegetation, including \emph{Poa}, \emph{Pseudoroegneria}, and \emph{Achnatherum}. Shrub canopy is less than 20\%. Fire-tolerant shrubs, such as \emph{Chrysothamnus} species are initial sprouters post-fire (LandFire 2007).

\paragraph{Succession Transition} In the absence of disturbance, patches in this seral stage will transition to MDM at 20 years. 

\paragraph{Wildfire Transition} High mortality wildfire (100\% of fires in this seral stage) recycles the patch through the ED seral stage. Low mortality wildfire is not modeled for this seral stage.

\noindent\hrulefill


\subsubsection{Mid Development - Moderate Canopy Cover (MDM)}

\paragraph{Description} Mid-seral community with a mixture of herbaceous and shrub vegetation. Vegetation present likely includes \emph{A. nova}, \emph{A. arbuscula}, \emph{Poa}, \emph{Achnatherum}, and \emph{Pseudoroegneria}.  Shrub cover often less than 25\% (LandFire 2007).

\paragraph{Succession Transition} After 120 years without high mortality disturbance, patches in this seral stage will transition to LDC. 

\paragraph{Wildfire Transition} High mortality wildfire (100\% of fires in this seral stage) recycles the patch through the ED seral stage. Low mortality wildfire is not modeled for this seral stage.

\noindent\hrulefill


\subsubsection{Late Development - Closed Canopy Cover (LDC)} 

\paragraph{Description} Late seral community with an increased presence of conifer trees (up to 40\% cover). The degree of tree canopy closure differs depending on whether it is an \emph{A. arbuscula} (closure likely under 15\%) or an \emph{A. nova} (closure up to 40\%) community. In \emph{A. arbuscula} communities a mixture of herbaceous and shrub vegetation with over 10\% shrub cover would still be present. In \emph{A. nova} communities the herbaceous and shrub component would be greatly reduced (less than 1\% cover). Vegetation present may also include \emph{Juniperus}, \emph{P. monophylla} and \emph{Achnatherum} (LandFire 2007).

\paragraph{Succession Transition} In the absence of disturbance, this class will maintain. 

\paragraph{Wildfire Transition} High mortality wildfire (31\% of fires in this seral stage) recycles the patch through the ED seral stage. Low mortality wildfire (69\%) maintains the LDO seral stage.

\noindent\hrulefill

\subsection*{Condition Classification}
Because seral stageification was done through orthophoto analysis, no polygons will be assigned to the LDC seral stage, which is actually not an \emph{Artemisia}-dominated seral stage. Only 3 polygons were assigned to LSG. Typical fields used to assign early-mid-late seral stage (overstory tree diameter) are null for shrubs. Cover is available. Polygons with cover less than 50\% were assigned to MDM and polygons with cover greater than 50\% were assigned to LDC.


\subsection*{References}

\begin{hangparas}{.25in}{1} 
\interlinepenalty=10000
LandFire. ``Biophysical Setting Models.'' Biophysical Setting 0610790: Great Basin Xeric Mixed Sagebrush Shrubland. 2007. LANDFIRE Project, U.S. Department of Agriculture, Forest Service; U.S. Department of the Interior. \burl{http://www.landfire.gov/national\_veg\_models\_op2.php}. Accessed 9 November 2012.

Van de Water, Kip M. and Hugh D. Safford. ``A Summary of Fire Frequency Estimates for California Vegetation Before Euro-American Settlement.'' \emph{Fire Ecology} 7.3 (2011): 26-57. doi: 10.4996/fireecology.0703026.

Verner, Jared. ``Low Sage (LSG).'' \emph{A Guide to Wildlife Habitats of California}, edited by Kenneth E. Mayer and William F. Laudenslayer. California Deparment of Fish and Game, 1988. \burl{http://www.dfg.ca.gov/biogeodata/cwhr/pdfs/SGB.pdf}. Accessed 4 December 2012.
\end{hangparas}



% !TEX root = master.tex
\newpage
\section{Curl-leaf Mountain Mahogany (CMM)}
\label{cmm-description}

\subsection*{General Information}

\subsubsection{Cover Type Overview}

\textbf{Curl-leaf Mountain Mahogany (CMM)}
\newline
Crosswalks
\begin{itemize}
	\item EVeg: Regional Dominance Type 1
	\begin{itemize}
		\item Curl-leaf Mountain Mahogany
	\end{itemize}

	\item LandFire BpS Model
	\begin{itemize}
		\item 0610620: Inter-Mountain Basin Curl-leaf Mountan Mahogany Woodland and Shrubland
	\end{itemize}

	\item Presettlement Fire Regime Type
	\begin{itemize}
		\item Curl-leaf Mountain Mahogany
	\end{itemize}
\end{itemize}

\noindent Reviewed by Becky Estes, Central Sierra Province Ecologist, USDA Forest Service

\subsubsection{Vegetation Description}
This landcover type is characterized by the dominance or co-dominance of \emph{Cercocarpus ledifolius}. Other shrubs such as \emph{Artemisia}, \emph{Arctostaphylos}, \emph{Ceanothus}, and \emph{Ephedra} may be present. \emph{C. ledifolius} is both a primary early successional colonizer rapidly invading bare mineral soils after disturbance and the dominant long-lived species. Depending on the effects of a given fire on the seed bank, in some cases it could take 10 years to recolonize. Where \emph{C. ledifolius} has reestablished quickly after fire, \emph{Chrysothamnus nauseosus} may codominate. Litter and shading by woody plants inhibits the establishment of \emph{C. ledifolius}, particularly in late seral stages where canopy cover is high. Reproduction often appears more dependent upon geographic variables (slope, aspect, and elevation) than biotic factors. \emph{Artemisia arbuscula} and \emph{Artemisia nova} are infrequently associated. \emph{Symphoricarpos}, \emph{Amelanchier}, and \emph{Ribes} are present on cooler, moister sites. \emph{Pinus monophylla}, \emph{Juniperus}, \emph{Pseudotsuga menziesii}, \emph{Abies magnifica}, \emph{Abies concolor}, and \emph{Pinus jeffreyi} may have sporadic presence at very low densities. In older stands the understory may consist largely of \emph{Leptodactylon pungens} (LandFire 2007, Gucker 2006).

\subsubsection{Distribution}
\emph{C. ledifolius} communities are usually found on upper slopes and ridges between 2130 and 3200 m (7000-10,500 ft), although northern stands may occur as low as 600 m (200 ft). It is more common on northwestern and northeastern aspects. Most stands occur on rocky, shallow soils and outcrops, with mature stand cover from 10-55\%. In the absence of fire, old stands may occur on somewhat deeper soils, with more than 55\% cover (LandFire 2007).

\subsection*{Disturbances}

\subsubsection{Wildfire}
Wildfires tend to be high mortality, stand-replacing fires that initiate a process of post-fire forest succession. High mortality fires kill large as well as small trees, and may kill many of the shrubs and herbs as well, although below-ground organs of at least some individual shrubs and herbs survive and re-sprout. 

\emph{C. ledifolius} is easily killed by fire and does not resprout. However, it is a primary early successional colonizer, rapidly invading bare mineral soils after disturbance. Fires are not common in early seral stages, when there is little fuel, except in chaparral-dominated stands. Stand-replacing fires are more common in mid-seral stands, where herbs and smaller shrubs provide ladder fuels. When surface fire is relatively common, stands will adopt a savanna-like woodland structure with an understory characterized by \emph{Ribes}, \emph{L. pungens}, and various grasses. Trees can become very old and will rarely show fire scars. In late, closed stands, the absence of herbs and small forbs makes fire uncommon, requiring extreme winds and drought conditions. However, stands that do burn often experience high mortality fire (LandFire 2007).

Estimates of fire rotations are available from the LandFire project and a review paper (LandFire 2007, Van de Water and Safford 2011). The LandFire project's published fire return intervals are based on a series of associated models created using the Vegetation Dynamics Development Tool (VDDT). In VDDT, fires are specified concurrently with the transition that follows them. For example, a replacement fire causes a transition to the early development stage. In the RMLands model, such fires are classified as high mortality. However, in VDDT mixed severity fires may cause a transition to early development, a transition to a more open seral stage, or no transition at all. In this case, we categorize the first example as a high mortality fire, and the second and third examples as a low mortality fire. Based on this approach, we calculated fire rotations and the probability of high mortality fire for each of the three CMM seral stages (Table~\ref{tab:cmmdesc_fire}). We computed the overall target fire rotation of 76 years based on values from Van de Water and Safford (2011). 




\begin{table}[!htbp]
\footnotesize
\centering
\caption{Fire rotation index values and probability of high severity fire (at least 75\% overstory tree mortality) probabilities. The seral stage that is most susceptible to fire (i.e., has the lowest predicted fire rotation) has a fire rotation index value of 1. Higher values correspond with lower likelihoods of experiencing wildfire. The values are relative only within an individual seral stage and should not be compared against other land cover types. Values were derived from VDDT model 0610790 (LandFire 2007) and Van de Water and Safford (2011). }
\label{tab:cmmdesc_fire}
\begin{tabular}{@{}lcc@{}}
\toprule
 \textbf{Seral Stage}    & \textbf{\begin{tabular}[c]{@{}c@{}}Fire Rotation \\ Index\end{tabular}} & \textbf{\begin{tabular}[c]{@{}c@{}}Probability of \\ High Severity Fire\end{tabular}} \\ \hline
Early (All)     		   & 4.8  & 0.17        \\
Mid--Moderate  			   & 1.0  & 0.67        \\
Late--Closed               & 28.8  & 1      \\ 
\emph{Target Fire Rotation}    			& \emph{76 years}  &   \\ 
\bottomrule
\end{tabular}
\end{table}
   			
\subsubsection{Other Disturbance}
Other disturbances are not currently modeled, but may, depending on the seral stage affected and mortality levels, reset patches to early development, maintain existing seral stages, or shift/accelerate succession to a more open seral stage. 

\subsection*{Vegetation Seral Stages}
We recognize three separate seral stages for CMM: Early Development (ED), Mid Development - Moderate Canopy Cover (MDM), and Late Development - Closed Canopy Cover (LDC) (Figure~\ref{cmm_transmodel}). Our seral stages are an alternative to ``successional'' classes that imply a linear progression of states and tend not to incorporate disturbance. The seral stages identified here are derived from a combination of successional processes and anthropogenic and natural disturbance, and are intended to represent a composition and structural condition that can be arrived at from multiple other conditions described for that landcover type. Thus our seral stages incorporate age, size, canopy cover, and vegetation composition. In general, the delineation of stages has originated from the LandFire biophysical setting model descriptive of a given landcover type; however, seral stages are not necessarily identical to the classes identified in those models.

\begin{figure}[hbt]
\centering
\includegraphics[width=0.8\textwidth]{/Users/mmallek/Documents/Thesis/statetransmodel/StateTransitionModel/shrub.png}
\caption{State and Transition Model for Curl-leaf Mountain Mahogany. Each dark grey box represents one of the three seral stages for this landcover type. Three stages of development are represented: early, middle, and late. We describe the middle development stage as characterized by moderate canopy cover and the late development stage as characterized by closed canopy cover, but these are not hard and fast rules. Transitions between states/seral stages may occur as a result of high mortality fire, low mortality fire, or succession. Specific pathways for each are denoted by the appropriate color line and arrow: red lines relate to high mortality fire, orange lines relate to low mortality fire, and green lines relate to natural succession.} 
\label{cmm_transmodel}
\end{figure}

\subsubsection{Early Development (ED)}

\paragraph{Description} \emph{C. ledifolius} seedlings rapidly invade bare mineral soils after fire. Litter and shading by woody plants inhibits establishment. Bunchgrasses and disturbance-tolerant forbs and resprouting shrubs, such as \emph{Symphoricarpos}, may be present. \emph{Ericameria} and \emph{Artemisia} seedlings are likely present. Vegetation composition will affect fire behavior, especially if chaparral species like \emph{Arctostaphylos} or \emph{Ceanothus} are present (LandFire 2007).

\paragraph{Succession Transition} In the absence of disturbance, patches in this seral stage will transition to MDM upon reaching 20 years of age. 

\paragraph{Wildfire Transition} High mortality wildfire (17\% of fires in this seral stage) recycles the patch through the ED seral stage. No transition occurs as a result of low mortality fire.

\noindent\hrulefill


\subsubsection{Mid Development - Moderate Canopy Cover (MDM)}

\paragraph{Description} \emph{C. ledifolius} may co-dominate with mature \emph{Artemisia}, \emph{Purshia}, \emph{Symphoricarpos}, or \emph{Ericameria}. Few \emph{C. ledifolius} seedlings are present. Canopy cover is variable (LandFire 2007).

\paragraph{Succession Transition} After 120 years in this stage, patches in this seral stage will transition to LDC.

\paragraph{Wildfire Transition} High mortality wildfire (67\% of fires in this seral stage) recycles the patch through the ED seral stage. No transition occurs as a result of low mortality fire.

\noindent\hrulefill


\subsubsection{Late Development - Closed Canopy Cover (LDC)}

\paragraph{Description} Moderate to high cover of large shrub- or tree-like \emph{C. ledifolius}. When low mortality fire is relatively frequent, late-successional \emph{C. ledifolius} may exhibit evidence of infrequent fire scars on older trees. Patches may consist of open savanna-like woodlands with an herbaceous-dominated understory. Other shrub species may be abundant, but decadent. When low mortality fire is absent, very few other shrubs are present, and herbaceous cover is low. Duff may be very deep, and scattered trees may occur. \emph{C. ledifolius} trees reach very old age in the absence of stand-replacing fire, potentially living over 1000 years (LandFire 2007).

\paragraph{Succession Transition} In the absence of disturbance, patches in this seral stage will remain in this seral stage. 

\paragraph{Wildfire Transition} High mortality wildfire (100\% of fires in this seral stage) recycles the patch through the ED seral stage.

\noindent\hrulefill

\subsection*{Condition Classification}
To create the initial cover and seral stage layer (2010), polygons were randomly assigned to seral stages based on a 20:10:70 distribution for early/mid/late development (based on an analysis of past fire in the project area). Random numbers between 0 and 1 were generated using numpy for Python and used to assign each CMM polygon to a seral stage.




\subsection*{References}
\begin{hangparas}{.25in}{1} 
\interlinepenalty=10000
Gucker, Corey L. ``Cercocarpus ledifolius'' \emph{Fire Effects Information System}, U.S. Department of Agriculture, Forest Service, Rocky Mountain Research Station, Fire Sciences Laboratory, 2006.  \burl{http://www.fs.fed.us/database/feis/} [Accessed 29 July 2013.]. 

LandFire. ``Biophysical Setting Models.'' Biophysical Setting 0610790: Great Basin Xeric Mixed Sagebrush Shrubland. 2007. LANDFIRE Project, U.S. Department of Agriculture, Forest Service; U.S. Department of the Interior. \burl{http://www.landfire.gov/national\_veg\_models\_op2.php}. Accessed 9 November 2012.

Van de Water, Kip M. and Hugh D. Safford. ``A Summary of Fire Frequency Estimates for California Vegetation Before Euro-American Settlement.'' \emph{Fire Ecology} 7.3 (2011): 26-57. doi: 10.4996/fireecology.0703026.

\end{hangparas}


% !TEX root = master.tex
\newpage
\section{Lodgepole Pine (LPN)}

\subsection*{General Information}

\subsubsection{Cover Type Overview}

\textbf{Lodgepole Pine (LPN)}
\newline
Crosswalks
\begin{itemize}
	\item EVeg: Regional Dominance Type 1
	\begin{itemize}
		\item Lodgepole Pine
	\end{itemize}

	\item LandFire BpS Model
	\begin{itemize}
		\item 0610581 Sierra Nevada Subalpine Lodgepole Pine Forest and Woodland - Wet
		\item 0610582 Sierra Nevada Subalpine Lodgepole Pine Forest and Woodland - Dry

	\end{itemize}

	\item Presettlement Fire Regime Type
	\begin{itemize}
		\item Lodgepole Pine
	\end{itemize}
\end{itemize}

\paragraph{Lodgepole Pine with Aspen (LPN-ASP)}
This type is created by overlaying the NRIS TERRA Inventory of Aspen on top of the EVeg layer. Where it intersects with LPN it is assigned to LPN-ASP.
\newline

\noindent Reviewed by Shana Gross, Ecologist, USDA Forest Service

\subsubsection{Vegetation Description}
\paragraph{Lodgepole Pine (LPN)} \emph{P. contorta} ssp. \emph{murrayana} is the overwhelming dominant within its forest community, mixing occasionally with \emph{Abies magnifica}, and with scattered \emph{Pinus jeffreyi}  and \emph{Pinus monticola}, and \emph{Tsuga mertensiana} at higher elevations (Fites-Kaufman et al. 2007). Mature Sierran stands often contain significant seedlings and saplings. Understory characteristics are influenced by proximity to meadow and stream margins. \emph{Arctostaphylos} and \emph{Ribes} are common shrubs. Stands associated with meadow edges and streams may have a rich herbaceous layer consisting of grasses, forbs, and sedges. Species associations are likely very location specific. Plants present may include but are not limited to \emph{Cassiope}, \emph{Vaccinium}, \emph{Phyllodoce}, \emph{Kalmia}, \emph{Ceanothus}, \emph{Chrysolepis}, and \emph{Carex}. Elsewhere, the understory may be virtually absent, consisting of scattered shrubs such as \emph{Quercus vaccinifolia}, and herbs like \emph{Antennaria}, \emph{Arabis}, \emph{Eriogonum}, and \emph{Gayophytum}. Fast-moving streams within the cover type are generally characterized by relatively dense populations of \emph{Salix} (Bartolome 1988, Fites-Kaufman et al. 2007, LandFire 2007a, LandFire 2007b).  

\paragraph{Lodgepole Pine with Aspen (LPN-ASP)}	When \emph{Populus tremuloides} co-occurs with LPN on the west side of the Sierran crest, it is typically found in smaller patches, often less than 2 ha (5 acres) in size. Mature stands in which \emph{P. tremuloides} are still dominant are usually relatively open. Average canopy closures range from 60 to 100 percent in young and intermediate-aged stands and from 25 to 60 percent in mature stands. The open nature of the stands results in substantial light penetration to the ground (Verner 1988).

\subsubsection{Distribution}
\paragraph{Lodgepole Pine (LPN)}	Open stands of \emph{P. contorta} ssp. \emph{murrayana}, which make up a widespread upper montane forest/woodland, tolerating both rocky soils and semisaturated meadow edges, in an elevational belt within and above the \emph{A. magnifica zone}. These forests, strongly dominated by \emph{P. contorta} ssp. \emph{murrayana}, generally occur at elevations of about 1,830 m to 2,400 m (6000 ft to 7875 ft) in the northern Sierra Nevada. Stands of \emph{P. contorta} ssp. \emph{murrayana} may reach much lower, however, with cold air drainage down glacial canyons (Fites-Kaufman et al. 2007, Anderson 1996). On infertile soils, \emph{P. contorta} ssp. \emph{murrayana} is often the only tree species that will grow (Lotan and Critchfield 1990).
More than any other Sierran conifer, \emph{P. contorta} ssp. \emph{murrayana} is relatively tolerant of poor soil aeration, and thus grows well around the margins of wet meadows and other moist areas. Many upper montane and subalpine meadows in the Sierra Nevada exhibit invasion of young \emph{P. contorta} ssp. \emph{murrayana} moving inward from their drier margins. It is not clear how much this process has been influenced by changes in fire frequency or grazing over the last 150 years (Fites-Kaufman et al. 2007).

\paragraph{Lodgepole Pine with Aspen (LPN-ASP)}		Sites supporting \emph{P. tremuloides} are associated with added soil moisture, i.e., azonal wet sites. These sites are found throughout the LPN zone, often close to streams, lakes, and meadows. Other sites include rock reservoirs, springs and seeps. Terrain can be simple to complex (LandFire 2007c). 


\subsection*{Disturbances}

\subsubsection{Wildfire}

\paragraph{Lodgepole Pine (LPN)} 	Wildfires tend to be high mortality, stand-replacing fires that initiate a process of post-fire forest succession. High mortality fires kill large as well as small trees, and may kill many of the shrubs and herbs as well, although below-ground organs of at least some individual shrubs and herbs survive and resprout. Low mortality fires tend to only kill small seedlings and depend on the herbaceous layer to carry fire.

Unlike the Rocky Mountain subspecies of \emph{P. contorta} (ssp. \emph{latifolia}), \emph{P. contorta} ssp. \emph{murrayana} does not have serotinous cones (Fites-Kaufman et al. 2007). Following high mortality fire, it initially establishes in even-aged stands, but small-scale disturbances and the ability of the subspecies to regenerate in the absence of fire promote uneven-aged structure (Cope 1993, Gross 2013).

High mortality fire occurs at long intervals. Mixed severity fire is related to fire behavior across the often moist areas where \emph{P. contorta} ssp. \emph{murrayana} is found. Surface fires are more common on drier sites, although in general sparse fuels limit fire ignition and spread. Most fires are small (less than 1 ha) but very large fires covering hundreds of hectares do occur (LandFire 2007a, LandFire 2007b). This is due in part to the high susceptibility to fire mortality by \emph{P. contorta} ssp. \emph{murrayana} because of its thin bark and shallower roots. Postfire conditions provide an ideal seedbed, and \emph{P. contorta} ssp. \emph{murrayana} is an early post-fire colonizer (Cope 1993).

\paragraph{Lodgepole Pine with Aspen (LPN-ASP)}	Sites supporting \emph{P. tremuloides} are maintained by stand-replacing disturbances that allow regeneration from below-ground suckers. Upland clones are impaired or suppressed by conifer ingrowth and overtopping and intensive grazing that inhibits growth. In a reference condition scenario, a few stands will advance toward conifer dominance, but in the current landscape scenario where fire has been reduced from reference conditions there are many more conifer-dominated mixed aspen stands (LandFire 2007c, Verner 1988). 

Estimates of fire rotations for these variants are available from the LandFire project and a few review papers. The LandFire project’s published fire return intervals are based on a series of associated models created using the Vegetation Dynamics Development Tool (VDDT). In VDDT, fires are specified concurrently with the transition that follows them. For example, a replacement fire causes a transition to the early development stage. In the RMLands model, such fires are classified as high mortality. However, in VDDT mixed severity fires may cause a transition to early development, a transition to a more open seral stage, or no transition at all. In this case, we categorize the first example as a high mortality fire, and the second and third examples as a low mortality fire. Based on this approach, we calculated fire rotations and the probability of high mortality fire for each of the LPN and LPN-ASP seral stages (Tables~\ref{tab:lpndesc_fire} and \ref{tab:lpnaspdesc_fire}). We computed overall target fire rotations based on values from Mallek et al. (2013) and Van de Water and Safford (2011). 



\begin{table}[]
\centering
\caption{Fire rotation (years) and proportion of high (versus low) mortality fires for Lodgepole Pine type. Values were derived from VDDT model 0610790 (LandFire 2007), Mallek et al. (2013), and Estes (personal communication). }
\label{tab:lpndesc_fire}
\begin{tabular}{@{}lcc@{}}
\toprule
\textbf{Condition}          & \textbf{Fire Rotation} & \multicolumn{1}{l}{\textbf{\begin{tabular}[c]{@{}l@{}}Proportion \\ High Mortality\end{tabular}}} \\ \midrule
Target                      & 52    & n/a        \\
Early Development - All     & 29    & 0.03       \\
Mid Development - Closed    & 59    & 0.41       \\
Mid Development - Moderate  & 27    & 0.15       \\
Mid Development - Open      & 18    & 0.07       \\
Late Development - Closed   & 37    & 0.26       \\
Late Development - Moderate & 24    & 0.13       \\
Late Development - Open     & 18    & 0.07       \\ \bottomrule
\end{tabular}
\end{table}

\begin{table}[]
\centering
\caption{Fire rotation (years) and proportion of high (versus low) mortality fires for Lodgepole Pine - Aspen type. Values were derived from VDDT model 0610790 (LandFire 2007) and Van de Water and Safford (pers. comm. 2013).}
\label{tab:lpnaspdesc_fire}
\begin{tabular}{@{}lcc@{}}
\toprule
\textbf{Condition}               & \textbf{Fire Rotation} & \multicolumn{1}{l}{\textbf{\begin{tabular}[c]{@{}l@{}}Proportion \\ High Mortality\end{tabular}}} \\ \midrule
Target                           & 52     & n/a        \\
Early Development - Aspen        & 29     & 0.03       \\
Mid Development - Aspen          & 59     & 0.41       \\
Mid Development - Aspen-Conifer  & 27     & 0.15       \\
Late Development - Conifer-Aspen & 24     & 0.13       \\
Late Development - Closed        & 37     & 0.26       \\ \bottomrule
\end{tabular}
\end{table}

\subsubsection{Other Disturbance}
Other disturbances are not currently modeled, but may, depending on the seral stage affected and mortality levels, reset patches to early development, maintain existing seral stages, or shift/accelerate succession to a more open seral stage. 

\subsection*{Vegetation Seral Stages}
We recognize seven separate seral stages for LPN: Early Development (ED), Mid Development - Open Canopy Cover (MDO), Mid Development - Moderate Canopy Cover, Mid Development - Closed Canopy Cover (MDC), Late Development - Open Canopy Cover (LDO), Late Development - Moderate Canopy Cover (LDM), and Late Development - Closed Canopy Cover (LDC) (Figure~\ref{transmodel_lpn}). The LPN-ASP variant is assigned to five seral stages: Early Development - Aspen (EDA), Mid Development - Aspen (MDA), Mid Development - Aspen with Conifer (MDAC), Late Development - Conifer with Aspen (LDCA), and Late Development - Closed Canopy Cover (LDC) (Figure~\ref{transmodel_lpn-asp}).

Our seral stages are an alternative to ``successional'' classes that imply a linear progression of states and tend not to incorporate disturbance. The seral stages identified here are derived from a combination of successional processes and anthropogenic and natural disturbance, and are intended to represent a composition and structural condition that can be arrived at from multiple other conditions described for that landcover type. Thus our seral stages incorporate age, size, canopy cover, and vegetation composition. In general, the delineation of stages has originated from the LandFire biophysical setting model descriptive of a given landcover type; however, seral stages are not necessarily identical to the classes identified in those models.


\begin{figure}[htbp]
\centering
\includegraphics[width=0.8\textwidth]{/Users/mmallek/Documents/Thesis/statetransmodel/StateTransitionModel/7class.png}
\caption{State and Transition Model for Lodgepole Pine Forest (not inclusive of the aspen variant). Each dark grey box represents one of the seven seral stages for this landcover type. Each column of boxes represents a stage of development: early, middle, and late. Each row of boxes represents a different level of canopy cover: closed (70-100\%), moderate (40-70\%), and open (0-40\%). Transitions between states/seral stages may occur as a result of high mortality fire, low mortality fire, or succession. Specific pathways for each are denoted by the appropriate color line and arrow: red lines relate to high mortality fire, orange lines relate to low mortality fire, and green lines relate to natural succession.} 
\label{transmodel_lpn}
\end{figure}

\begin{figure}[htbp]
\centering
\includegraphics[width=0.8\textwidth]{/Users/mmallek/Documents/Thesis/statetransmodel/StateTransitionModel/5class-asp.png}
\caption{State and Transition Model for Lodgepole Pine Forest - Aspen variant. Each dark grey box represents one of the seven seral stages for this landcover type. Each column of boxes represents a stage of development: early, middle, and late. Transitions between states/seral stages may occur as a result of high mortality fire, low mortality fire, or succession. Specific pathways for each are denoted by the appropriate color line and arrow: red lines relate to high mortality fire, orange lines relate to low mortality fire, and green lines relate to natural succession.} 
\label{transmodel_lpn-asp}
\end{figure}


\subsubsection{Lodgepole Pine}

\paragraph{Early Development (ED)}

\paragraph{Description} Grasses, forbs, low shrubs, and sparse to moderate cover of trees (primarily \emph{P. contorta} ssp. \emph{murrayana}) seedlings/saplings with an open canopy. This seral stage is characterized by the recruitment of a new cohort of early successional, shade-intolerant tree species into an open area created by a stand-replacing disturbance. 


A short period of herbaceous productivity precedes closure of the tree canopy on productive sites. The prolific seed output, establishment, and seedling growth of \emph{P. contorta} ssp. \emph{murrayana} makes the period of herbaceous production short (Bartolome 1988). \emph{P. contorta} ssp. \emph{murrayana} regeneration density ranges from moderate to dog hair thickets (LandFire 2007a).


\paragraph{Succession Transition} In the absence of disturbance, patches in this seral stage will begin transitioning to MDC at 10 years at a rate of 0.6 per time step. At 40 years, all patches will succeed. On average, patches remain in early development for 18 years.

\paragraph{Wildfire Transition} High mortality wildfire (100\% of fires in this seral stage) recycles the patch through the Early Development seral stage. No transition occurs as a result of low mortality fire.

\noindent\hrulefill


\paragraph{Mid Development - Open Canopy Cover (MDO)}

\paragraph{Description} Sparse ground cover of grasses, forbs, and shrubs. Mid-maturity \emph{P. contorta} ssp. \emph{murrayana} where surface fire or other disturbance has opened the stand. Canopy cover ranges from 10-50\% (LandFire 2007a).
Continued recruitment into stands produces overstocking and slow growth of the overcrowded trees. This overcrowding may make them susceptible to insects, although others have argued that the more vigorously growing trees are more likely to be attacked. Beetle infestation creates large quantities of fuel that increase the probability of wildfire (Bartolome 1988).


\paragraph{Succession Transition} Patches in this seral stage may stay in this seral stage under low mortality disturbance, but after 10 years without fire they begin transitioning to MDM at a rate of 0.8 per time step. Succession to LDO occurs once the patch has been in mid development for 50 years. The rate of succession per time step is 0.5. At 100 years, all stands will succeed to LDO. On average, patches remain in mid development for 54 years.

\paragraph{Wildfire Transition} High mortality wildfire (7\% of fires in this seral stage) recycles the patch through the Early Development seral stage. Low mortality wildfire (93\%) maintains the patch in MDO.

\noindent\hrulefill

\paragraph{Mid Development - Moderate Canopy Cover (MDM)}

\paragraph{Description} Sparse ground cover of grasses, forbs, and shrubs. Mid-maturity \emph{P. contorta} ssp. \emph{murrayana} where surface fire or other disturbance has opened the stand. Canopy cover ranges from 10-50\% (LandFire 2007a).

Continued recruitment into stands produces overstocking and slow growth of the overcrowded trees. This overcrowding may make them susceptible to insects, although others have argued that the more vigorously growing trees are more likely to be attacked. Beetle infestation creates large quantities of fuel that increase the probability of wildfire (Bartolome 1988).


\paragraph{Succession Transition} Patches in this seral stage may stay in this seral stage under low mortality disturbance, but after 10 years without fire they begin transitioning to MDC at a rate of 0.8 per time step. Succession to LDM occurs once the patch has been in mid development for 45 years. The rate of succession per time step is 0.55. At 90 years, all stands will succeed to LDM.

\paragraph{Wildfire Transition} High mortality wildfire (15\% of fires in this seral stage) recycles the patch through the Early Development seral stage. Low mortality wildfire (85\%) triggers a transition to MDO 68\% of the time; otherwise, it remains in MDM.

\noindent\hrulefill

\paragraph{Mid Development - Closed Canopy Cover (MDC)}

\paragraph{Description} Sparse ground cover of grasses, forbs, and shrubs; mid-maturity \emph{P. contorta} ssp. \emph{murrayana} undergoing intrinsic stand thinning. Considerable surface fuel from tree mortality from previous fire. Canopy cover is greater than 50\% (LandFire 2007a).

Continued recruitment into stands produces overstocking and slow growth of the overcrowded trees. This overcrowding may make them susceptible to insects, although others have argued that the more vigorously growing trees are more likely to be attacked. Beetle infestation creates large quantities of fuel that increase the probability of wildfire. (Bartolome 1988).


\paragraph{Succession Transition} After 40 years in a MD seral stage without a wildfire-triggered transition, patches in this seral stage will begin transitioning to LDC. The rate of succession per time step is 0.6. At 80 years, all patches succeed to LDC.

\paragraph{Wildfire Transition} High mortality wildfire (41\% of fires in this seral stage) recycles the patch through the Early Development seral stage. Low mortality wildfire (59\%) triggers a transition to MDM.

\noindent\hrulefill


\paragraph{Late Development - Open Canopy Cover (LDO)}

\paragraph{Description} Areas that have experienced one or more low severity understory fires that had reduced stand density or old stands that have not experienced fire but have been thinned by other processes (tree falls, etc.). Stands are uneven aged. Canopy cover ranges from 10-50\% (LandFire 2007a).

\paragraph{Succession Transition} Patches in this seral stage may maintain under low mortality disturbance, but after 25 years without fire, these patches succeed to LDM at a rate of 0.7 per timestep.

\paragraph{Wildfire Transition} High mortality wildfire (7\% of fires in this seral stage) recycles the patch through the Early Development seral stage. Low mortality wildfire (93\%) maintains the patch in LDO.

\noindent\hrulefill

\paragraph{Late Development - Moderate Canopy Cover (LDM)}

\paragraph{Description} Sparse ground cover of grasses, forbs, and shrubs. Mid-maturity \emph{P. contorta} ssp. \emph{murrayana} where surface fire or other disturbance has opened the stand. Canopy cover ranges from 10-50\% (LandFire 2007a).

Continued recruitment into stands produces overstocking and slow growth of the overcrowded trees. This overcrowding may make them susceptible to insects, although others have argued that the more vigorously growing trees are more likely to be attacked. Beetle infestation creates large quantities of fuel that increase the probability of wildfire (Bartolome 1988).


\paragraph{Succession Transition} Patches in this seral stage may stay in this seral stage under low mortality disturbance, but after 25 years without fire they begin transitioning to LDC at a rate of 0.7 per time step. 

\paragraph{Wildfire Transition} High mortality wildfire (13\% of fires in this seral stage) recycles the patch through the Early Development seral stage. Low mortality wildfire (87\%) triggers a transition to LDO 73\% of the time; otherwise, it remains in LDM.

\noindent\hrulefill

\paragraph{Late Development - Closed Canopy Cover (LDC)}

\paragraph{Description} Old \emph{P. contorta} ssp. \emph{murrayana} stands where fire has had minimal influence. Canopy cover exceeds 50\%.

\paragraph{Succession Transition} This class will maintain in the absence of disturbance.

\paragraph{Wildfire Transition} High mortality wildfire (26\% of fires in this seral stage) recycles the patch through the Early Development seral stage. Low mortality wildfire (73\%) triggers a transition to LDM.

\noindent\hrulefill
\noindent\hrulefill

\subsubsection{Aspen Variant}

\paragraph{Early Development - Aspen (ED-A)}

\paragraph{Description} Grasses, forbs, low shrubs, and sparse to moderate cover of tree seedlings/saplings (primarily \emph{P. tremuloides}) with an open canopy. This seral stage is characterized by the recruitment of a new cohort of early successional, shade-intolerant tree species into an open area created by a stand-replacing disturbance. 

Following disturbance, succession proceeds rapidly from an herbaceous layer to shrubs and trees, which invade together (Verner 1988). \emph{P. tremuloides} suckers over 6ft tall develop within about 10 years (LandFire 2007c). 


\paragraph{Succession Transition} Unless it burns, a patch in the early seral stage persists for 10 years, at which point it transitions to MD-A.

\paragraph{Wildfire Transition} High mortality wildfire (100\% of fires in this seral stage) recycles the patch through the Early Development seral stage. No transition occurs as a result of low mortality fire.

\noindent\hrulefill


\paragraph{Mid Development - Aspen (MD-A)}

\paragraph{Description} \emph{P. tremuloides} trees 5-16'' DBH. Canopy cover is highly variable, and can range from 40-100\%. These patches range in age from 10 to 110 years. Some understory conifers, predominantly \emph{P. contorta} ssp. \emph{murrayana}, are encroaching, but \emph{P. tremuloides} is still the dominant component of the stand (LandFire 2007c).

\paragraph{Succession Transition} MD-A persists for at least 50 years in the absence of fire, after which patches begin transitioning to MD-AC at a rate of 0.6 per timestep. After 100 years all remaining MD-A patches transition to MD-AC. 

\paragraph{Wildfire Transition} High mortality wildfire (41\% of fires in this seral stage) recycles the patch through the Early Development seral stage. No transition occurs as a result of low mortality fire.

\noindent\hrulefill

\paragraph{Mid Development - Aspen with Conifer (MD-AC)}

\paragraph{Description} These stands have been protected from fire since the last stand-replacing disturbance. \emph{P. tremuloides} trees are predominantly 16'' DBH and greater. Conifers (predominantly \emph{P. contorta} ssp. \emph{murrayana}) are present and becoming increasingly dominant over the \emph{P. tremuloides}. Conifers are pole to medium-sized, and conifer cover is at least 40\% (LandFire 2007c).

\paragraph{Succession Transition} MD-AC persists for 100 years in the absence of high mortality fire, after which patches transition to LDC. 

\paragraph{Wildfire Transition} High mortality wildfire (15\% of fires in this seral stage) returns the patch to ED-A. Low mortality wildfire (85\%) maintains the patch in MD-AC.

\noindent\hrulefill

\paragraph{Late Development - Closed (LDC)}

\paragraph{Description} Some \emph{P. tremuloides} continue to be present in the understory, but large\emph{ P. contorta} ssp. \emph{murrayana} are now the dominant tree species, having overtopped the \emph{P. tremuloides}. Smaller conifers are present in the midstory as well (LandFire 2007a). This seral stage is analogous to the LDC seral stage for the LPN variant.

\paragraph{Succession Transition} Patches in this seral stage will maintain in the absence of disturbance.

\paragraph{Wildfire Transition} High mortality wildfire (26\% of fires in this seral stage) will return the patch to ED-A. Low mortality wildfire (74\%) opens the stand up to LD-CA.

\noindent\hrulefill


\paragraph{Late Development - Conifer with Aspen (LD-CA)}

\paragraph{Description} If stands are sufficiently protected from fire such that conifer species overtop \emph{P. tremuloides} and become large, they may be able to withstand some fire that more sensitive \emph{P. tremuloides} cannot. When this occurs, it creates a patch characterized by late development conifers, such as \emph{P. contorta} ssp. \emph{murrayana}, and early seral \emph{P. tremuloides}. 

\paragraph{Succession Transition} LD-CA persists for 70 years in the absence of any fire, at which point patches transition to LDC. 

\paragraph{Wildfire Transition} High mortality wildfire (13\% of fires in this seral stage) returns the patch to ED-A. Low mortality wildfire (87\%) maintains the stand in LD-CA. 

\noindent\hrulefill

\newpage
\subsection*{Condition Classification}

\begin{table}[hbp]
\small
\centering
\caption{Classification of seral stage for LPN. Diameter at Breast Height (DBH) and Cover From Above (CFA) values taken from EVeg polygons. DBH categories are: null, 0-0.9'', 1-4.9'', 5-9.9'', 10-19.9'', 20-29.9'', 30''+. CFA categories are null, 0-10\%, 10-20\%, \dots , 90-100\%. Each row in the table below should be read with a boolean AND across each column.}
\label{lpn_classification}
\begin{tabular}{@{}lrrrrr@{}}
\toprule
\textbf{\begin{tabular}[l]{@{}l@{}}Cover \\ Condition\end{tabular}} & \textbf{\begin{tabular}[r]{@{}r@{}}Overstory Tree \\ Diameter 1 \\ (DBH)\end{tabular}} & \textbf{\begin{tabular}[r]{@{}r@{}}Overstory Tree \\ Diameter 2 \\ (DBH)\end{tabular}} & \textbf{\begin{tabular}[r]{@{}r@{}}Total Tree\\ CFA (\%)\end{tabular}} & \textbf{\begin{tabular}[r]{@{}r@{}}Conifer \\ CFA (\%)\end{tabular}} & \textbf{\begin{tabular}[r]{@{}r@{}}Hardwood \\ CFA (\%)\end{tabular}} \\ \midrule
Early All & 0-4.9'' & any & any & any & any \\
Mid Open & 5-9.9'' & any & 0-40 & any & any \\
Mid Moderate & 5-9.9'' & any & 40-70 & any & any \\
Mid Closed & 5-9.9'' & any & 70-100 & any & any \\
Late Open & 10''+ & any & 0-40 & any & any \\
Late Moderate & 10''+ & any & 40-70 & any & any \\
Late Closed & 10''+ & any & 70-100 & any & any \\ \bottomrule
\end{tabular}
\end{table}

LPN-ASP seral stages were assigned manually using NAIP 2010 Color IR imagery to assess seral stage.



\clearpage
\subsection*{References}
\begin{hangparas}{.25in}{1} 
Bartolome, James W. ``Lodgepole Pine (LPN).'' \emph{A Guide to Wildlife Habitats of California}, edited by Mayer, Kenneth E. and William F. Laudenslayer. California Deparment of Fish and Game. 1988. \burl{http://www.dfg.ca.gov/biogeodata/cwhr/pdfs/LPN.pdf}. Accessed 4 December 2012.

``CalVeg Zone 1.'' Vegetation Descriptions. \emph{Vegetation Classification and Mapping}.  11 December 2008. U.S. Forest Service. \burl{http://www.fs.usda.gov/Internet/FSE\_DOCUMENTS/fsbdev3\_046448.pdf}. Accessed 2 April 2013.

Cope, Amy B. 1993. ``Pinus contorta var. murrayana.'' In: Fire Effects Information System, [Online].  U.S. Department of Agriculture, Forest Service,  Rocky Mountain Research Station, Fire Sciences Laboratory (Producer).  \burl{http://www.fs.fed.us/database/feis/} [Accessed 4 December 2012].

Fites-Kaufman, Jo Ann, Phil Rundel, Nathan Stephenson, and Dave A. Wixelman. ``Montane and Subalpine Vegetation of the Sierra Nevada and Cascade Ranges.'' In \emph{Terrestrial Vegetation of California, 3rd Edition}, edited by Michael Barbour, Todd Keeler-Wolf, and Allan A. Schoenherr, 456-501. Berkeley and Los Angeles: University of California Press, 2007. 

Gross, Shana. Ecologist, USDA Forest Service. Personal communication, 3 July 2013.

Lotan, James E. and William B. Critchfield. ``Lodgepole Pine.'' Russell M. Burns and Barbara H. Honkala, tech. coords. Silvics of North America, vol 1. Conifers; Glossary. Agriculture handbook no.654. Washington, D.C.: U.S. Dept. of Agriculture, Forest Service, 1990. 

LandFire. ``Biophysical Setting Models.'' Biophysical Setting 0610581: Sierra Nevada Subalpine Lodgepole Pine Forest and Woodland. 2007a. LANDFIRE Project, U.S. Department of Agriculture, Forest Service; U.S. Department of the Interior. \burl{http://www.landfire.gov/national\_veg\_models\_op2.php}. Accessed 9 November 2012.

LandFire. ``Biophysical Setting Models.'' Biophysical Setting 0610582: Sierra Nevada Subalpine Lodgepole Pine Forest and Woodland. 2007b. LANDFIRE Project, U.S. Department of Agriculture, Forest Service; U.S. Department of the Interior. \burl{http://www.landfire.gov/national\_veg\_models\_op2.php}. Accessed 9 November 2012.

LandFire. ``Biophysical Setting Models.'' Biophysical Setting 0610610: Inter-Mountain Basins Aspen-Mixed Conifer Forest and Woodland. 2007c. LANDFIRE Project, U.S. Department of Agriculture, Forest Service; U.S. Department of the Interior. \burl{http://www.landfire.gov/national\_veg\_models\_op2.php}. Accessed 7 January 2013.

Safford, Hugh S. Regional Ecologist, USDA Forest Service. Personal communication, 5 May 2013.

Skinner, Carl N. and Chi-Ru Chang. ``Fire Regimes, Past and Present.'' \emph{Sierra Nevada Ecosystem Project: Final report to Congress, vol. II, Assessments and scientific basis for management options}. Davis: University of California, Centers for Water and Wildland Resources, 1996.

Van de Water, Kip M. and Hugh D. Safford. ``A Summary of Fire Frequency Estimates for California Vegetation Before Euro-American Settlement.'' \emph{Fire Ecology} 7.3 (2011): 26-57. doi: 10.4996/fireecology.0703026.

Verner, Jared. ``Aspen (ASP).'' \emph{A Guide to Wildlife Habitats of California}, edited by Kenneth E. Mayer and William F. Laudenslayer. California Deparment of Fish and Game, 1988. \burl{http://www.dfg.ca.gov/biogeodata/cwhr/pdfs/ASP.pdf}. Accessed 4 December 2012.

\end{hangparas}


% !TEX root = master.tex
\newpage
\section{Mixed Evergreen Forest (MEG)}
\label{meg-description}

\subsection*{General Information}

\subsubsection{Cover Type Overview}

\textbf{Mixed Evergreen Forest (MEG)}
\newline
Crosswalks
\begin{itemize}
	\item EVeg: Regional Dominance Type 1
	\begin{itemize}
		\item Interior Mixed Hardwood
		\item California Bay
		\item Canyon Live Oak
		\item Madrone
		\item Bigleaf Maple
		\item Interior Live Oak
		\item Montane Mixed Hardwood 
		\item Pacific Douglas Fir
		\item Tanoak
	\end{itemize}

	\item EVeg: Regional Dominance Type 2
	\begin{itemize}
		\item Tanoak (regardless of RD Type 1 value, and therefore inclusive of all potential Type 1 vegetation types)
	\end{itemize}

	\item LandFire BpS Model
	\begin{itemize}
		\item 0610430 Mediterranean California Mixed Evergreen Forest
	\end{itemize}

	\item Presettlement Fire Regime Type
	\begin{itemize}
		\item Mixed Evergreen Forest
	\end{itemize}
\end{itemize}

\paragraph{Mesic Modifier (MEG\_M)}
This type is created by intersecting a binary xeric/mesic layer with the existing vegetation layer. MEG cells that intersect with mesic cells are assigned to the mesic modifier.
\paragraph{Xeric Modifier (MEG\_X)}
This type is created by intersecting a binary xeric/mesic layer with the existing vegetation layer. MEG cells that intersect with xeric cells are assigned to the xeric modifier.
\paragraph{Ultramafic Modifier (MEG\_U)}
This type is created by intersecting an ultramafic soils/geology layer with the existing vegetation layer. Where ultramafic cells intersect with MEG they are assigned to the ultramafic modifier.



\noindent Reviewed by Kyle Merriam, Sierra-Cascade Province Ecologist, USDA Forest Service; Becky Estes, Central Sierra Province Ecologist, USDA Forest Service


\subsubsection{Vegetation Description}
\paragraph{Mixed Evergreen Forest (MEG)} 	This landcover type forms a complex mosaic of forest due to the geologic, topographic, and successional variation typical within its range. This type is characterized by a combination of coniferous and broadleaved trees. Characteristic trees include \emph{Pseudotsuga menziesii}, \emph{Quercus chrysolepis}, \emph{Notholithocarpus densiflorus},\footnote{Tan oak was known as \emph{Lithocarpus densiflorus} for over 90 years before botanists renamed it \emph{Notholithocarpus densiflorus} in 2008 (Manos et al. 2008). Some sources and database continue to use the old name and plant symbol.}  \emph{Arbutus menziesii}, \emph{Umbellularia californica}, and \emph{Chrysolepis chrysophylla}. Species composition is primarily determined by the environmental gradients of temperature and moisture availability. \emph{Quercus kelloggii} is found on drier sites on inland portion of the range. \emph{Pinus lambertiana} and \emph{Pinus ponderosa} can be present in this type. These stands tend to have dense or diverse shrub understories with \emph{Ceanothus}, \emph{Corylus}, Gaultheria, Morella, Rhododendron, Ribes, Rubus, Toxicodendron diversilobum, and Vaccinium. Grass species include Bromus, Festuca, and Hierochloe. Polystichum \emph{munitum} and \emph{Pteridium aquilinum} var. \emph{pubescens} sometimes grow abundantly. \emph{Carex} spp. are present in some places (LandFire 2007, McDonald 1988, Tappeiner 1990).

\begin{adjustwidth}{2cm}{}
\textbf{Mesic Modifier (MEG\_M)}
Deep mesic soils support aggregations that include a lower or midstory layer of dense, sclerophyllous, broad-leaved evergreen trees like \emph{N. densiflorus} and \emph{Arbutus menziesii}, with an irregular, often open, higher layer of tall needle-leaved evergreen trees, typically \emph{P. menziesii}. A small number of pole and sapling trees occur throughout stands. On wetter sites, shrub layers are well developed, often with 100\% cover. Cover of the herbaceous layer under the shrubs can be up to 10 percent. At higher elevations, the shrubs disappear and the herb layer is often 100\%. Diversity of tree size typically increases with stand age, along with tree spacing. Young stands have closely spaced and uniformly distributed trees, whereas older stands have a more patchy stem distribution. Snags and downed logs, an important structural component of this habitat, increase in density or volume with stand age (Raphael 1988). Potential additional conifer associates include \emph{Abies concolor}, \emph{Pinus lambertiana}, \emph{Calocedrus decurrens}, and \emph{Pinus ponderosa} (Tappeiner 1990). A large variety of shrubs, forbs, grasses, sedges, and ferns, along with \emph{N. densiflorus} sprouts, can become aggressive on burned or cutover areas. This is especially true in areas where high severity fires have locally eliminated conifer seed sources (Tappeiner 1990).

\medskip
\noindent \textbf{Xeric Modifier (MEG\_X)}
A pronounced hardwood tree layer is typical, with an infrequent and poorly developed shrub stratum, and a sparse herbaceous layer (McDonald 1988). Characteristic oaks include \emph{Q. chrysolepis}, \emph{Q. wislizeni}, \emph{Q. kelloggi}, and \emph{Quercus garryana}. \emph{ Q. chrysolepis} and \emph{Q. wislizeni} are the most common oaks in the project area. They may individually form almost pure stands on steep canyon slopes and rocky ridgetops throughout the Sierra Nevada, or co-occur. They have tremendously variable growth forms, ranging from shrubs with multiple trunks on rocky, steep slopes, to magnificently spreading tall trees on deeper soils in moister areas. Both are evergreen with dense canopies (Allen-Diaz et al. 2007). Tree spacing is close (3-4 m) on better sites, and wider (8-10 m) on poor sites. In general, snags and downed woody material are sparse. Lower elevation associates are \emph{Pinus sabiniana}, \emph{Pinus attenuata}, \emph{N. densiflorus}, \emph{A. menziesii}, \emph{Quercus wislizeni}, \emph{C. chrysophylla}, and scrubby \emph{U. californica} (McDonald 1988).

\medskip
\noindent \textbf{Ultramafic Modifier (MEG\_U)}
\emph{Notholithocarpus densiflorus} var. \emph{echinoides}, or dwarf tanoak, grows on ultramafic and other less productive sites (Estes 2013). It is unclear if the 2 varieties differ genetically or if the small stature of dwarf tanoak is due to unproductive site conditions. Ecology literature does not usually distinguish between the 2 infrataxa (Fryer 2008). However, its identification is pertinent to management decisions. While \emph{N. lithocarpus} is generally protected as an oak species, the dwarf variety may be classified as a shrub and therefore subject to treatment or removal. Typically, \emph{P. menziesii} attains less dominance and may replaced by open stands of various conifers, such as \emph{Pinus ponderosa}, \emph{Pinus sabiniana}, or \emph{Pinus jeffreyi}. Trees occur within a generally open grassland or shrubland. The shrub layer is likely to include \emph{Quercus vaccinifolia}, \emph{N. densiflorus}, \emph{U. californica}, \emph{Quercus breweri}, and \emph{Rhamnus}. Common grasses include \emph{Stipa}, \emph{Festuca}, and \emph{Danthonia} (LandFire 2007b, McDonald 1988, O'Geen et al. 2007, Raphael 1988). 

\end{adjustwidth}

\subsubsection{Distribution}
\paragraph{Mixed Evergreen Forest}		This highly variable cover type occurs in the Sierra Nevada on all aspects at elevations of 350 m (1150 ft) to over 1700 m (5575 ft) (LandFire 2007a). Soil depth classes range from shallow to deep. The large number of species in the type, both conifer and hardwood, allow it to occupy and persist in a wide range of environments. Good soils and poor, steep slopes and slight, frequently disturbed and pristine all are at least adequate habitats for one or more species (McDonald 1988).

A xeric-mesic gradient was developed based on four variables: 1) aspect, 2) potential evapotranspiration, 3) topographic wetness index, and 4) soil water storage. The variables were standardized by z-score such that higher values correspond to more mesic environments. Thus, potential evapotranspiration was inverted to maintain this balance. The four variables were combined with equal weights. This final variables was split into xeric vs. mesic, with xeric occupying the negative end of the range up to $\frac{1}{4}$ standard deviation below the mean (zero) and mesic occupying the remaining portion of the spectrum.

\begin{adjustwidth}{2cm}{}
\textbf{Mesic Modifier }
Soils are deep, well-drained, and loamy, sandy, or gravelly. Found in valleys, coves, ravines, along streams, and on north as well as east slopes. It typically occurs in areas that are cool and moist sites in areas where precipitation is highest most likely in the form of rain and snow.

\medskip
\noindent \textbf{Xeric Modifier}
Q. chrysolepis and associates are found on a wide range of slopes, especially those that are moderate to steep. Soils are for the most part rocky, alluvial, coarse textured, poorly developed, and well drained. 

\medskip
\noindent \textbf{Ultramafic Modifier} Ultramafics have been mapped at various spatial densities throughout the elevational range of the landcover type. Low to moderate elevations in ultramafic and serpentinized areas often produce soils low in essential minerals like calcium potassium, and nitrogen, and have excessive accumulations of heavy metals such as nickel and chromium. These sites vary widely in the degree of serpentinization and effects on their overlying plant communities (``CalVeg Zone 1'' 2011). Note, the terms ``ultramafic rock'' and ``serpentine'' are broad terms used to describe a number of different but related rock types, including serpentinite, peridotite, dunite, pyroxenite, talc and soapstone, among others (O'Geen et al. 2007). 

\end{adjustwidth}

%%%

\subsection*{Disturbances}
\subsubsection{Wildfire}

\paragraph{Mixed Evergreen Forest}		Fire is the dominant disturbance event. Wildfires are common and frequent; mortality depends on vegetation vulnerability and wildfire intensity. Low mortality fires kill small trees and may consume above-ground portions of small oaks, shrubs and herbs, but do not kill large trees or below-ground organs of most oaks, shrubs and herbs which promptly resprout. High mortality fires kill trees of all sizes and may kill many of the shrubs and herbs as well. However, high mortality fires typically kill only the above ground portions of the oaks, shrubs and herbs; consequently, most oaks, shrubs and herbs promptly resprout from surviving below ground organs.

The vast majority of fires occur in late summer or early fall and are associated with lightning storms. Native American burns locally increased the frequency and may have been extensive prior to 1850. However, research also suggests that fire frequencies actually increased after European settlement (Merriam, pers. comm. 2013). Fires in the past were often large in area due to the high number of ignition points associated with fire events, and created patches of varying age and species composition (LandFire 2007a). 

Hardwoods typically provide the greatest cover after fire due to root-crown sprouting. Depending upon fire severity many hardwoods may have epicormic sprouting well into the crown. Species composition, density and interspecific competition within stands contributes to multiple pathways following disturbance. If fire has been absent from an area for an extended period of time, some conifers may be able to establish and persist even with the return of frequent low severity fire. But, if low severity fire is frequent after a stand-replacing fire, conifers will be more or less excluded and hardwoods will dominate (LandFire 2007a).

Estimates of fire rotations for these variants are available from the LandFire project and a few review papers. The LandFire project’s published fire return intervals are based on a series of associated models created using the Vegetation Dynamics Development Tool (VDDT). In VDDT, fires are specified concurrently with the transition that follows them. For example, a replacement fire causes a transition to the early development stage. In the RMLands model, such fires are classified as high mortality. However, in VDDT mixed severity fires may cause a transition to early development, a transition to a more open seral stage, or no transition at all. In this case, we categorize the first example as a high mortality fire, and the second and third examples as a low mortality fire. Based on this approach, we calculated fire rotations and the probability of high mortality fire for each of the MEG seral stages across the three variants (Tables~\ref{tab:megmdesc_fire}--\ref{tab:megudesc_fire}). We computed overall target fire rotations based on expert input from Safford and Estes, values from Mallek et al. (2013), and Van de Water and Safford (2011). 

\begin{adjustwidth}{2cm}{}
\textbf{Mesic Modifier }
\emph{N. densiflorus} is adapted to ignite easily. In the lower montane zone of the Sierra Nevada where \emph{N. densiflorus} occurs, the historic fire regime was characterized by dormant season fires of mostly low to moderate severity (Tappeiner 1990). In stands with high \emph{N. densiflorus} cover, \emph{N. densiflorus} may dominate the stand for many years before conifers re-establish. Patchy, stand-replacement fires were most common on north-facing slopes and during extended droughts. \emph{N. densiflorus} seedlings and saplings are typically top-killed by even low severity surface fire. Large trees usually survive moderate-severity fire, bearing fire scars afterward. Even \emph{N. densiflorus} with thick bark (3-10 cm) typically sustain bole damage from fire. Relative to associated conifers, mature \emph{P. menziesii} is fairly resistant to surface fires. Crown fires cause extensive mortality (Tappeiner 1990).

\medskip
\noindent \textbf{Xeric Modifier} \emph{Q. chrysolepis} has loose, dead, flaky bark that catches fire readily and burns intensely. Occasional fire often changes a stand of \emph{Q. chrysolepis} to \emph{Q. wislizeni}-chaparral, but without fire for sufficient time, trees again develop. Where fire is frequent, this oak becomes scarce or even drops out of the montane hardwood community (McDonald 1988).

\medskip
\noindent \textbf{Ultramafic Modifier} Historically, these woodland types had frequent low-severity fire. However, now there is higher susceptibility to stand replacing fire because of fire exclusion.

\end{adjustwidth}

%%%


\begin{table}[!htbp]
\footnotesize
\centering
\caption{Fire rotation index values and probability of high severity fire (at least 75\% overstory tree mortality) probabilities for Mixed Evergreen Forest - Mesic. The seral stage that is most susceptible to fire (i.e., has the lowest predicted fire rotation) has a fire rotation index value of 1. Higher values correspond with lower likelihoods of experiencing wildfire. The values are relative only within an individual seral stage and should not be compared against other land cover types. Values were derived from VDDT model 0610790 (LandFire 2007a) and Safford and Estes (personal communication). }
\label{tab:megmdesc_fire}
\begin{tabular}{@{}lcc@{}}
\toprule
 \textbf{Seral Stage}    & \textbf{\begin{tabular}[c]{@{}c@{}}Fire Rotation \\ Index\end{tabular}} & \textbf{\begin{tabular}[c]{@{}c@{}}Probability of \\ High Severity Fire\end{tabular}} \\ \hline
Early (All)     		 & 3.9     & 1                             \\
Mid--Closed    			 & 2.7     & 0.11                          \\
Mid--Moderate  			 & 1.5    & 0.11                          \\
Mid--Open      			 & 1.0   & 0.11                          \\
Late--Closed   			 & 2.5   & 0.21                          \\
Late--Moderate 			 & 1.4   & 0.11                          \\
Late--Open     			 & 1.0   & 0.11   						\\  
\emph{Target Fire Rotation}    			& \emph{50 years}  &   \\ 
\bottomrule
\end{tabular}
\end{table}

\begin{table}[!htbp]
\footnotesize
\centering
\caption{Fire rotation index values and probability of high severity fire (at least 75\% overstory tree mortality) probabilities for Mixed Evergreen Forest - Xeric. The seral stage that is most susceptible to fire (i.e., has the lowest predicted fire rotation) has a fire rotation index value of 1. Higher values correspond with lower likelihoods of experiencing wildfire. The values are relative only within an individual seral stage and should not be compared against other land cover types. Values were derived from VDDT model 0610790 (LandFire 2007a), and Safford and Estes (personal communication). }
\label{tab:megxdesc_fire}
\begin{tabular}{@{}lcc@{}}
\toprule
 \textbf{Seral Stage}    & \textbf{\begin{tabular}[c]{@{}c@{}}Relative Susceptibility \\ to Fire\end{tabular}} & \textbf{\begin{tabular}[c]{@{}c@{}}Probability of \\ High Severity Fire\end{tabular}} \\ \hline
Early (All)     		 & 5.8            & 1       \\
Mid--Closed    			 & 2.7        		 & 0.10    \\
Mid--Moderate  			 & 1.5      		& 0.10    \\
Mid--Open      			 & 1.0   		 & 0.10    \\
Late--Closed   			 &  2.5           & 0.10    \\
Late--Moderate 			 &  1.4        & 0.10    \\
Late--Open     			 &  1.0     		& 0.03 	  \\  
\emph{Target Fire Rotation}    			& \emph{40 years}  &   \\ 
\bottomrule
\end{tabular}
\end{table}

\begin{table}[!htbp]
\footnotesize
\centering
\caption{Fire rotation index values and probability of high severity fire (at least 75\% overstory tree mortality) probabilities for Mixed Evergreen Forest - Ultramafic. The seral stage that is most susceptible to fire (i.e., has the lowest predicted fire rotation) has a fire rotation index value of 1. Higher values correspond with lower likelihoods of experiencing wildfire. The values are relative only within an individual seral stage and should not be compared against other land cover types. Values were derived from VDDT model 0711700 (LandFire 2007b), and Safford and Estes (personal communication). }
\label{tab:megudesc_fire}
\begin{tabular}{@{}lcc@{}}
\toprule
 \textbf{Seral Stage}    & \textbf{\begin{tabular}[c]{@{}c@{}}Relative Susceptibility \\ to Fire\end{tabular}} & \textbf{\begin{tabular}[c]{@{}c@{}}Probability of \\ High Severity Fire\end{tabular}} \\ \hline
Early (All)     		 & 3.9            & 1           \\
Mid--Closed    			 & 2.7           & 0.11        \\
Mid--Moderate  			 & 1.5           & 0.11        \\
Mid--Open      			 & 1.0           & 0.11        \\
Late--Closed   			 & 2.5            & 0.21        \\
Late--Moderate 			 & 1.4           & 0.11        \\
Late--Open     			 & 1.0           & 0.11   		\\ 
\emph{Target Fire Rotation}    			& \emph{50 years}  &   \\ 
\bottomrule
\end{tabular}
\end{table}

%%%

\subsubsection{Other Disturbance}
Other disturbances are not currently modeled, but may, depending on the seral stage affected and mortality levels, reset patches to early development, maintain existing seral stages, or shift/accelerate succession to a more open seral stage. All of the tree species associated with this vegetation type are susceptible to a wide variety of pathogens and insects (such as sudden oak death for \emph{N. densiflorus}, which is caused by the pathogen \emph{Phytophthora ramorum}).

\subsection*{Vegetation Seral Stages}
We recognize seven separate seral stages for MEG: Early Development (ED), Mid Development - Open Canopy Cover (MDO), Mid Development - Moderate Canopy Cover, Mid Development - Closed Canopy Cover (MDC), Late Development - Open Canopy Cover (LDO), Late Development - Moderate Canopy Cover (LDM), and Late Development - Closed Canopy Cover (LDC) (Figure~\ref{meg_transmodel}). Our seral stages are an alternative to ``successional'' classes that imply a linear progression of states and tend not to incorporate disturbance. The seral stages identified here are derived from a combination of successional processes and anthropogenic and natural disturbance, and are intended to represent a composition and structural condition that can be arrived at from multiple other conditions described for that landcover type. Thus our seral stages incorporate age, size, canopy cover, and vegetation composition. In general, the delineation of stages has originated from the LandFire biophysical setting model descriptive of a given landcover type; however, seral stages are not necessarily identical to the classes identified in those models.

\begin{figure}[htbp]
\centering
\includegraphics[width=0.8\textwidth]{/Users/mmallek/Documents/Thesis/statetransmodel/StateTransitionModel/7class.png}
\caption{State and Transition Model for Mixed Evergreen Forest. Each dark grey box represents one of the seven seral stages for this landcover type. Each column of boxes represents a stage of development: early, middle, and late. Each row of boxes represents a different level of canopy cover: closed (70-100\%), moderate (40-70\%), and open (0-40\%). Transitions between states/seral stages may occur as a result of high mortality fire, low mortality fire, or succession. Specific pathways for each are denoted by the appropriate color line and arrow: red lines relate to high mortality fire, orange lines relate to low mortality fire, and green lines relate to natural succession.} 
\label{meg_transmodel}
\end{figure}

\paragraph{Early Development (ED)}

\paragraph{Description} This seral stage is characterized by the diversity of species establishing and reestablishing into an open area created by a stand-replacing disturbance. 

\begin{adjustwidth}{2cm}{}
\textbf{Mesic Modifier } On mesic sites, abundant grasses, forbs, low shrubs, found under sparse to moderate cover of trees (primarily \emph{P. menziesii} and \emph{N. densiflorus}) seedlings/saplings with an open canopy. Seedling establishment of \emph{P. menziesii} following fire is dependent on the spacing and number of surviving seed trees. Seedling establishment following large stand-replacing fires may be slow if seed trees are killed over extensive areas. Or, if there are numerous, well-spaced surviving seed trees within the burned area, a new cohort of seedlings can quickly establish (Uchytil 1991). Nearly all \emph{N. densiflorus} burls sprout after fire, and survivorship is high. \emph{Q. chrysolepis}, if present, also sprouts readily, and shrubs such as \emph{Mahonia}, \emph{Gaultheria}, and \emph{Rhododendron} may be significant. Shrub growth from seed banks, e.g. \emph{Ceanothus integerrimus}, can also be high (LandFire 2007a). Thus, \emph{N. densiflorus} and other shrubs usually dominante the initial seral stage if \emph{P. menziesii} isn’t able to seed in quickly (Raphael 1988).

\medskip
\noindent \textbf{Xeric Modifier}  On xeric sites, grasses, forbs, low shrubs, and sparse cover of tree seedlings and saplings are found under an open canopy. Forest openings contain a dense cover of hardwood sprouts. Sprouting shrubs such as \emph{M. aquifolium}, \emph{Gaultheria shallon}, and \emph{Rhododendron} may be significant. Shrub growth from seed banks, e.g. \emph{Ceanothus integerrimus}, can also be high (LandFire 2007a). 


\medskip
\noindent \textbf{Ultramafic Modifier}  On ultramafic sites, \emph{P. menziesii} may be stunted and slow-growing, and \emph{N. densiflorus} var. \emph{echinoides} may be present. Grasses like \emph{Festuca}, \emph{Danthonia}, and \emph{Acnatherum}, or else chaparral shrubs establish. Scattered \emph{Pinus ponderosa}, \emph{Pinus sabiniana}, or \emph{Pinus jeffreyi} may also be present (LandFire 2007b).

\end{adjustwidth}

\paragraph{Succession Transition}
\begin{adjustwidth}{2cm}{}
\textbf{Mesic and Xeric Modifier } In the absence of disturbance, patches in this seral stage will begin transitioning to MDM at 20 years. The rate of succession per time step is 0.8. At 40 years, all patches will succeed. On average, patches remain in ED for 26 years.


\medskip
\noindent \textbf{Ultramafic Modifier} Succession may be delayed. Thus, in the absence of disturbance, patches in this seral stage will begin transitioning to MDM after 30 years and may be delayed in the ED seral stage for as long as 80 years. A patch in this seral stage succeeds at a rate of 0.4 per time step. On average, patches remain in ED for 43 years.

\end{adjustwidth}

\paragraph{Wildfire Transition} High mortality wildfire (100\% of fires in this seral stage) recycles the patch through the ED seral stage. Low mortality wildfire is not modeled for this seral stage.

\noindent\hrulefill


\paragraph{Mid Development - Open Canopy Cover (MDO)}

\paragraph{Description}
\begin{adjustwidth}{2cm}{}
\textbf{Mesic Modifier } Sparse ground cover of grasses, forbs, and shrubs; open tree canopy cover (primarily \emph{P. menziesii} and \emph{N. densiflorus}). Other \emph{Quercus} and \emph{Arctostaphylos} species may also be present. In this stage, hardwoods are dominant, but \emph{P. menziesii} and possibly other conifers are established or establishing under the predominantly \emph{N. densiflorus} canopy (LandFire 2007a, McDonald 1988). 


\medskip
\noindent \textbf{Xeric Modifier}  Sparse ground cover of grasses, forbs, and shrubs; open tree canopy cover, primarily hardwoods such as \emph{Q. chrysolepis} and \emph{Q. kelloggii}. Conifers such as \emph{P. menziesii} are present at low densities in emergent status. The shrub understory is still a significant presence (LandFire 2007a). 


\medskip
\noindent \textbf{Ultramafic Modifier}  Ultramafic sites are characterized by open \emph{P. menziesii}, \emph{Pinus ponderosa}, \emph{Pinus sabiniana}, or \emph{Pinus jeffreyi} stands with an understory comprised of \emph{N. densiflorus} var. \emph{echinoides} or \emph{Q. chrysolepis} as well as grasses, forbs, and shrubs (LandFire 2007b).

\end{adjustwidth}
\paragraph{Succession Transition}
\begin{adjustwidth}{2cm}{}
\textbf{Mesic and Xeric Modifier } Patches in this seral stage may stay in this seral stage under low mortality disturbance, but after 15 years without fire they begin transitioning to MDM at a rate of 0.8 per time step. After 20 years in a mid development stage, patches in this seral stage will begin transitioning to LDO. The rate of succession per time step is 0.8. At 40 years, all patches succeed. On average, patches remain in the mid development stage for 26 years.


\medskip
\noindent \textbf{Ultramafic Modifier}  Succession may be delayed. Thus, in the absence of low mortality disturbance, patches in the MDO seral stage will begin transitioning to MDM after 20 years in MDO at a rate of 0.7 per timestep. Patches in this seral stage will begin transitioning to LDO after 30 years in a mid development stage, and may be delayed in this stage for as long as 80 years. A patch in this seral stage succeeds at a rate of 0.4 per time step. On average, patches remain in the mid development stage for 43 years.

\end{adjustwidth}
\paragraph{Wildfire Transition}
\begin{adjustwidth}{2cm}{}
\textbf{Mesic Modifier } High mortality wildfire (11\% of fires in this seral stage) recycles the patch through the ED seral stage. Low mortality wildfire (89\%) does not effect a change in the MDO seral stage.


\medskip
\noindent \textbf{Xeric Modifier}  High mortality wildfire (10\% of fires in this seral stage) recycles the patch through the ED seral stage. Low mortality wildfire (90\%) does not effect a change in the MDO seral stage.


\medskip
\noindent \textbf{Ultramafic Modifier} High mortality wildfire (11\% of fires) recycles the patch through the ED seral stage. Low mortality wildfire (89\%) does not effect a change in the MDO seral stage.

\end{adjustwidth}

\noindent\hrulefill

\paragraph{Mid Development - Moderate Canopy Cover (MDM)}

\paragraph{Description}
\begin{adjustwidth}{2cm}{}
\textbf{Mesic Modifier } Sparse ground cover of grasses, forbs, and shrubs; moderate tree canopy cover (primarily \emph{P. menziesii} and \emph{ N. densiflorus}). Other \emph{Quercus} and \emph{Arctostaphylos} species may also be present. In this stage, hardwoods are dominant, but \emph{P. menziesii} and possibly other conifers are established or establishing under the predominantly \emph{N. densiflorus} canopy (LandFire 2007a, McDonald 1988). 


\medskip
\noindent \textbf{Xeric Modifier} Sparse ground cover of grasses, forbs, and shrubs; moderate tree canopy cover, primarily hardwoods such as \emph{Q. chrysolepis} and \emph{Q. kelloggii}. Conifers such as \emph{P. menziesii} are present at low densities in emergent status. The shrub understory is still a significant presence (LandFire 2007a). 


\medskip
\noindent \textbf{Ultramafic Modifier}  Ultramafic sites are characterized by open \emph{P. menziesii}, \emph{Pinus ponderosa}, \emph{Pinus sabiniana}, or \emph{Pinus jeffreyi} stands with an understory comprised of \emph{N. densiflorus} var. \emph{echinoides} or \emph{Q. chrysolepis} as well as grasses, forbs, and shrubs (LandFire 2007b).

\end{adjustwidth}
\paragraph{Succession Transition}
\begin{adjustwidth}{2cm}{}
\textbf{Mesic and Xeric Modifier } Patches in this seral stage may stay in this seral stage under low mortality disturbance, but after 15 years without fire they begin transitioning to MDC at a rate of 0.8 per time step. After 20 years in a mid development stage, patches in this seral stage will begin transitioning to LDM. The rate of succession per time step is 0.8. At 40 years, all patches succeed. On average, patches remain in the mid development stage for 26 years.


\medskip
\noindent \textbf{Ultramafic Modifier} Succession may be delayed. Thus, in the absence of low mortality disturbance, patches in the MDM seral stage will begin transitioning to MDC after 20 years in MDM at a rate of 0.7 per timestep. Patches in this seral stage will begin transitioning to LDM after 30 years in a mid development stage, and may be delayed in this stage for as long as 80 years. A patch in this seral stage succeeds at a rate of 0.4 per time step. On average, patches remain in the mid development stage for 43 years.

\end{adjustwidth}
\paragraph{Wildfire Transition}
\begin{adjustwidth}{2cm}{}
\textbf{Mesic Modifier } High mortality wildfire (11\% of fires in this seral stage) recycles the patch through the ED seral stage. Low mortality wildfire (89\%) triggers a transition to MDM 14\% of the time; otherwise, it remains in MDC.

\medskip
\noindent \textbf{Xeric Modifier} High mortality wildfire (10\% of fires in this seral stage) recycles the patch through the ED seral stage. Low mortality wildfire (90\%) triggers a transition to MDM 14\% of the time; otherwise, it remains in MDC.

\medskip
\noindent \textbf{Ultramafic Modifier} High mortality wildfire (11\% of fires) recycles the patch through the ED seral stage. Low mortality wildfire (89\%) triggers a transition to MDM 13\% of the time; otherwise, it remains in MDC.

\end{adjustwidth}
\noindent\hrulefill

\paragraph{Mid Development - Closed Canopy Cover (MDC)}

\paragraph{Description}
\begin{adjustwidth}{2cm}{}
\textbf{Mesic Modifier } Sparse ground cover of grasses, forbs, and shrubs; closed tree canopy cover (primarily \emph{P. menziesii} and \emph{N. densiflorus}). Other \emph{Quercus} and \emph{Arctostaphylos} species may also be present. In this stage, hardwoods are dominant, but \emph{P. menziesii} and possibly other conifers are established or establishing under the predominantly \emph{N. densiflorus} canopy (LandFire 2007a, McDonald 1988). 

\medskip
\noindent \textbf{Xeric Modifier} Sparse ground cover of grasses, forbs, and shrubs; closed tree canopy cover, primarily hardwoods such as \emph{Q. chrysolepis} and \emph{Q. kelloggii}. Conifers such as \emph{P. menziesii} are present at low densities in emergent status. The shrub understory is still a significant presence (LandFire 2007a). 

\medskip
\noindent \textbf{Ultramafic Modifier} Ultramafic sites are characterized by open \emph{P. menziesii}, \emph{Pinus ponderosa}, \emph{Pinus sabiniana}, or \emph{Pinus jeffreyi} stands with an understory comprised of \emph{N. densiflorus} var. \emph{echinoides} or \emph{Q. chrysolepis} as well as grasses, forbs, and shrubs (LandFire 2007b).

\end{adjustwidth}
\paragraph{Succession Transition}
\begin{adjustwidth}{2cm}{}
\textbf{Mesic and Xeric Modifier } After 20 years in a mid development stage, patches in this seral stage will begin transitioning to LDC. The rate of succession per time step is 0.8. At 40 years, all patches succeed. On average, patches remain in the mid development stage for 26 years.

\medskip
\noindent \textbf{Ultramafic Modifier} Succession may be delayed. Patches in this seral stage will begin transitioning to LDC after 30 years in a mid development stage, and may be delayed in this stage for as long as 80 years. A patch in this seral stage succeeds at a rate of 0.4 per time step. On average, patches remain in the mid development stage for 43 years.

\end{adjustwidth}
\paragraph{Wildfire Transition}
\begin{adjustwidth}{2cm}{}
\textbf{Mesic Modifier } High mortality wildfire (11\% of fires in this seral stage) recycles the patch through the ED seral stage. Low mortality wildfire (89\%) triggers a transition to MDM 22\% of the time; otherwise, it remains in MDC.

\medskip
\noindent \textbf{Xeric Modifier} High mortality wildfire (10\% of fires in this seral stage) recycles the patch through the ED seral stage. Low mortality wildfire (90\%) triggers a transition to MDM 20\% of the time; otherwise, it remains in MDC.

\medskip
\noindent \textbf{Ultramafic Modifier} High mortality wildfire (11\% of fires) recycles the patch through the ED seral stage. Low mortality wildfire (89\%) triggers a transition to MDM 22\% of the time; otherwise, it remains in MDC.

\end{adjustwidth}
\noindent\hrulefill


\paragraph{Late Development - Open Canopy Cover (LDO)}

\paragraph{Description}
\begin{adjustwidth}{2cm}{}
\textbf{Mesic Modifier } Overstory of large and very large trees, primarily \emph{P. menziesii}. Canopy cover less than 40\%. \emph{P. lambertiana} also occurs. \emph{N. densiflorus} is tolerant of both full sun and shade, and usually dominates the subcanopy at this stage. Co-dominance of the upper canopy with \emph{P. menziesii} is uncommon but possible after extended periods without disturbance (Uchytil 1991, LandFire 2007a). There is also some evidence that the senescence of late development \emph{N. densiflorus} may cause openings in the canopy and allow for continued \emph{P. menziesii} dominance (Estes pers. comm. 2013). \emph{Quercus} and \emph{Arctostaphylos} species may also be present in the sub-canopy (LandFire 2007a).

\medskip
\noindent \textbf{Xeric Modifier}  Overstory of large and very large trees, often with canopy cover less than 40\%. \emph{P. menziesii}, \emph{Q. chrysolepis}, and \emph{Arctostaphylos mewukka} may occur. Conifers are taller and larger than in MD and clearly form the upper canopy layer here. Shrubs persist in openings but those in shade are likely to begin senescing (LandFire 2007a). On ultramafic sites, large \emph{Pinus ponderosa} may additionally be present. Grass savannah persists on sites experiencing low intensity fire (with \emph{Festuca}, \emph{Achnatherum}, and \emph{Danthonia}). Where fire is less frequent, chaparral shrubland develops (with \emph{Arctostaphylos} and \emph{Quercus breweri}) (LandFire 2007b).

\medskip
\noindent \textbf{Ultramafic Modifier} On ultramafic sites, large \emph{Pinus ponderosa}, \emph{Pinus sabiniana}, or \emph{Pinus jeffreyi} may be present along with \emph{P. menziesii} and \emph{N. densiflorus} var. \emph{echinoides}. Grass savannah persists on sites experiencing low intensity fire (with \emph{Festuca}, \emph{Achnatherum}, and \emph{Danthonia}). Where fire is less frequent, chaparral shrubland develops (with \emph{Arctostaphylos} and \emph{Quercus breweri}) (LandFire 2007b).

\end{adjustwidth}
\paragraph{Succession Transition}
\begin{adjustwidth}{2cm}{}
\textbf{Mesic and Xeric Modifier } Patches in this seral stage may stay in this seral stage under low mortality disturbance, but after 15 years without fire they begin transitioning to LDM at a rate of 0.8 per time step. 

\medskip
\noindent \textbf{Ultramafic Modifier} Succession may be delayed. Thus, in the absence of low mortality disturbance, patches in the LDO seral stage will begin transitioning to LDM after 20 years in LDO at a rate of 0.7 per timestep. 

\end{adjustwidth}
\paragraph{Wildfire Transition}
\begin{adjustwidth}{2cm}{}
\textbf{Mesic Modifier } High mortality wildfire (11\% of fires in this seral stage) recycles the patch through the ED seral stage. Low mortality wildfire (89\%) does not effect a change in the MDO seral stage. 

\medskip
\noindent \textbf{Xeric Modifier} High mortality wildfire (3\% of fires in this seral stage) recycles the patch through the ED seral stage. Low mortality wildfire (97\%) does not effect a change in the MDO seral stage.

\medskip
\noindent \textbf{Ultramafic Modifier} High mortality wildfire (11\% of fires) recycles the patch through the ED seral stage. Low mortality wildfire (89\%) does not effect a change in the MDO seral stage.

\end{adjustwidth}
\noindent\hrulefill

\paragraph{Late Development - Moderate Canopy Cover (LDM)}

\paragraph{Description}
\begin{adjustwidth}{2cm}{}
\textbf{Mesic Modifier } Overstory of large and very large trees, primarily \emph{P. menziesii}. Canopy cover between 40\% and 60\%. \emph{P. lambertiana} also occurs. \emph{N. densiflorus} is tolerant of both full sun and shade, and usually dominates the subcanopy at this stage. Co-dominance of the upper canopy with \emph{P. menziesii} is uncommon but possible after extended periods without disturbance (Uchytil 1991, LandFire 2007a). There is also some evidence that the senescence of late development \emph{N. densiflorus} may cause openings in the canopy and allow for continued \emph{P. menziesii} dominance (Estes pers. comm. 2013). \emph{Quercus} and \emph{Arctostaphylos} species may also be present in the sub-canopy (LandFire 2007a).

\medskip
\noindent \textbf{Xeric Modifier} Overstory of large and very large trees, often with canopy cover between 40\% and 60\%. \emph{P. menziesii}, \emph{Q. chrysolepis}, and \emph{Arctostaphylos mewukka} may occur. Conifers are taller and larger than in MD and clearly form the upper canopy layer here. Shrubs persist in openings but those in shade are likely to begin senescing (LandFire 2007a). On ultramafic sites, large \emph{Pinus ponderosa} may additionally be present. Grass savannah persists on sites experiencing low intensity fire (with \emph{Festuca}, \emph{Achnatherum}, and \emph{Danthonia}). Where fire is less frequent, chaparral shrubland develops (with \emph{Arctostaphylos} and \emph{Quercus breweri}) (LandFire 2007b).

\medskip
\noindent \textbf{Ultramafic Modifier} On ultramafic sites, large \emph{Pinus ponderosa}, \emph{Pinus sabiniana}, or \emph{Pinus jeffreyi} may be present along with \emph{P. menziesii} and \emph{N. densiflorus} var. \emph{echinoides}. Grass savannah persists on sites experiencing low intensity fire (with \emph{Festuca}, \emph{Achnatherum}, and \emph{Danthonia}). Where fire is less frequent, chaparral shrubland develops (with \emph{Arctostaphylos} and \emph{Quercus breweri}) (LandFire 2007b).

\end{adjustwidth}
\paragraph{Succession Transition}
\begin{adjustwidth}{2cm}{}
\textbf{Mesic and Xeric Modifier } Patches in this seral stage may stay in this seral stage under low mortality disturbance, but after 15 years without fire they begin transitioning to LDC at a rate of 0.8 per time step. 

\medskip
\noindent \textbf{Ultramafic Modifier} Succession may be delayed. Thus, in the absence of low mortality disturbance, patches in the LDM seral stage will begin transitioning to LDC after 20 years in LDM at a rate of 0.7 per timestep. 

\end{adjustwidth}
\paragraph{Wildfire Transition}
\begin{adjustwidth}{2cm}{}
\textbf{Mesic Modifier } High mortality wildfire (11\% of fires in this seral stage) recycles the patch through the ED seral stage. Low mortality wildfire (89\%) triggers a transition to LDO 17\% of the time; otherwise, it remains in LDM.

\medskip
\noindent \textbf{Xeric Modifier} High mortality wildfire (10\% of fires in this seral stage) recycles the patch through the ED seral stage. Low mortality wildfire (90\%) triggers a transition to LDO 15\% of the time; otherwise, it remains in LDM.

\medskip
\noindent \textbf{Ultramafic Modifier}  High mortality wildfire (11\% of fires) recycles the patch through the ED seral stage. Low mortality wildfire (89\%) triggers a transition to LDO 17\% of the time; otherwise, it remains in LDM.

\end{adjustwidth}
\noindent\hrulefill

\paragraph{Late Development - Closed Canopy Cover (LDC)}

\paragraph{Description}
\begin{adjustwidth}{2cm}{}
\textbf{Mesic Modifier } Overstory of large and very large trees, primarily \emph{P. menziesii}. Canopy cover exceeds 70\%. \emph{P. lambertiana} also occurs. \emph{N. densiflorus} is tolerant of both full sun and shade, and usually dominates the subcanopy at this stage. Co-dominance of the upper canopy with \emph{P. menziesii} is uncommon but possible after extended periods without disturbance (Uchytil 1991, LandFire 2007a). There is also some evidence that the senescence of late development \emph{N. densiflorus} may cause openings in the canopy and allow for continued \emph{P. menziesii} dominance (Estes pers. comm. 2013). \emph{Quercus} and \emph{Arctostaphylos} species may also be present in the sub-canopy (LandFire 2007a).

\medskip
\noindent \textbf{Xeric Modifier} Overstory of large and very large trees, often with canopy cover over 70\%. \emph{P. menziesii}, \emph{Q. chrysolepis}, and \emph{Arctostaphylos mewukka} may occur. Conifers are taller and larger than in MD and clearly form the upper canopy layer here. Shrubs persist in openings but those in shade are likely to begin senescing (LandFire 2007a). On ultramafic sites, large \emph{Pinus ponderosa} may additionally be present. Grass savannah persists on sites experiencing low intensity fire (with \emph{Festuca}, \emph{Achnatherum}, and \emph{Danthonia}). Where fire is less frequent, chaparral shrubland develops (with \emph{Arctostaphylos} and \emph{Quercus breweri}) (LandFire 2007b).

\medskip
\noindent \textbf{Ultramafic Modifier} On ultramafic sites, large \emph{Pinus ponderosa}, \emph{Pinus sabiniana}, or \emph{Pinus jeffreyi} may be present along with \emph{P. menziesii} and \emph{N. densiflorus} var. \emph{echinoides}. Grass savannah persists on sites experiencing low intensity fire (with \emph{Festuca}, \emph{Achnatherum}, and \emph{Danthonia}). Where fire is less frequent, chaparral shrubland develops (with \emph{Arctostaphylos} and \emph{Quercus breweri}) (LandFire 2007b).

\end{adjustwidth}

\paragraph{Succession Transition}
\begin{adjustwidth}{2cm}{}
\textbf{Mesic, Xeric, and Ultramafic Modifier } In the absence of disturbance, patches in this seral stage will remain in this seral stage. 

\end{adjustwidth}

\paragraph{Wildfire Transition}
\begin{adjustwidth}{2cm}{}
\textbf{Mesic Modifier } High mortality wildfire (21\% of fires in this seral stage) recycles the patch through the ED seral stage. Low mortality wildfire (79\%) triggers a transition to LDM 26\% of the time; otherwise, it remains in LDC.

\medskip
\noindent \textbf{Xeric Modifier} High mortality wildfire (21\% of fires in this seral stage) recycles the patch through the ED seral stage. Low mortality wildfire (79\%) triggers a transition to LDM 24\% of the time; otherwise, it remains in LDC.

\medskip
\noindent \textbf{Ultramafic Modifier} High mortality wildfire (11\% of fires) recycles the patch through the ED seral stage. Low mortality wildfire (79\%) triggers a transition to LDM 26\% of the time; otherwise, it remains in LDC.

\end{adjustwidth}
\noindent\hrulefill

%\newpage

\subsection*{Seral Stage Classification}
\begin{table}[hbp]
\footnotesize
\centering
\caption{Classification of seral stage for MEG. Diameter at Breast Height (DBH) and Cover From Above (CFA) values taken from EVeg polygons. DBH categories are: null, 0-0.9'', 1-4.9'', 5-9.9'', 10-19.9'', 20-29.9'', 30''+. CFA categories are null, 0-10\%, 10-20\%, \dots , 90-100\%. Each row in the table below should be read with a boolean AND across each column.}
\label{meg_classification}
\begin{tabular}{@{}lrrrrr@{}}
\toprule
\textbf{\begin{tabular}[l]{@{}l@{}}Cover \\ Condition\end{tabular}} & \textbf{\begin{tabular}[r]{@{}r@{}}Overstory Tree \\ Diameter 1 \\ (DBH)\end{tabular}} & \textbf{\begin{tabular}[r]{@{}r@{}}Overstory Tree \\ Diameter 2 \\ (DBH)\end{tabular}} & \textbf{\begin{tabular}[r]{@{}r@{}}Total Tree\\ CFA (\%)\end{tabular}} & \textbf{\begin{tabular}[r]{@{}r@{}}Conifer \\ CFA (\%)\end{tabular}} & \textbf{\begin{tabular}[r]{@{}r@{}}Hardwood \\ CFA (\%)\end{tabular}} \\ \midrule
Early All        & 0-4.9''         & any & any    & any & any \\
Mid Open         & 5-19.9''        & any & 0-40   & any & any \\
Mid Moderate     & 5-19.9''        & any & 40-70  & any & any \\
Mid Closed       & 5-19.9''        & any & 70-100 & any & any \\
Late Open        & 20-40''+        & any & 0-40   & any & any \\
Late Moderate    & 20-40''+        & any & 40-70  & any & any \\
Late Closed      & 20-40''+        & any & 70-100 & any & any \\ \bottomrule
\end{tabular}
\end{table}


\clearpage
\subsection*{References}

\begin{hangparas}{.25in}{1} 
\interlinepenalty=10000
Allen-Diaz, Barbara, Richard Standiford, and Randall D. Jackson. ``Oak Woodlands and Forests.'' In \emph{Terrestrial Vegetation of California, 3rd Edition}, edited by Michael Barbour, Todd Keeler-Wolf, and Allan A. Schoenherr, 313-338. Berkeley and Los Angeles: University of California Press, 2007. 

``CalVeg Zone 1.'' Vegetation Descriptions. \emph{Vegetation Classification and Mapping}.  11 December 2008. U.S. Forest Service. \burl{http://www.fs.usda.gov/Internet/FSE\_DOCUMENTS/fsbdev3\_046448.pdf}. Accessed 2 April 2013.

Estes, Becky, Province Ecologist, USDA Forest Service. Personal communication, 15 August 2013 and 3 September 2013.

LandFire. ``Biophysical Setting Models.'' Biophysical Setting 0610430: Mediterranean California Mixed Evergreen Forest. 2007a. LANDFIRE Project, U.S. Department of Agriculture, Forest Service; U.S. Department of the Interior. \burl{http://www.landfire.gov/national\_veg\_models\_op2.php}. Accessed 9 November 2012.

LandFire. ``Biophysical Setting Models.'' Biophysical Setting 0711700: Klamath-Siskiyou Xeromorphic Serpentine Savanna and Chaparral. 2007b. LANDFIRE Project, U.S. Department of Agriculture, Forest Service; U.S. Department of the Interior. \burl{http://www.landfire.gov/national\_veg\_models\_op2.php}. Accessed 30 November 2012.

Mallek, Chris, Hugh Safford, Joshua Viers, and Jay Miller. ``Modern departures in fire severity and area vary by forest type, Sierra Nevada and southern Cascades, California, USA.'' Ecosphere 4.12 (2013): art153. doi: http://www.esajournals.org/doi/pdf/10.1890/ES13-00217.1. 

Manos, P. S., C. H. Cannon, and S. H. Oh. ``Phylogenetic relationships and taxonomic status of the paleoendemic Fagaceae of western North America: recognition of a new genus, Notholithocarpus.'' Madroño 55.3 (2008): 181-190. doi: 10.3120/0024-9637-55.3.181

McDonald, Philip M. ``Montane Hardwood (MHW).'' \emph{A Guide to Wildlife Habitats of California}, edited by Kenneth E. Mayer and William F. Laudenslayer. California Deparment of Fish and Game, 1988. \burl{http://www.dfg.ca.gov/biogeodata/cwhr/pdfs/MHW.pdf}. Accessed 4 December 2012.

Merriam, Kyle. Province Ecologist, USDA Forest Service. Personal communication, 9 July 2013.

O'Geen, Anthony T., Randy A. Dahlgren, and Daniel Sanchez-Mata. ``California Soils and Examples of Ultramafic Vegetation.'' In \emph{Terrestrial Vegetation of California, 3rd Edition}, edited by Michael Barbour, Todd Keeler-Wolf, and Allan A. Schoenherr, 71-106. Berkeley and Los Angeles: University of California Press, 2007. 

Raphael, Martin G. ``Douglas-Fir (DFR).'' \emph{A Guide to Wildlife Habitats of California}, edited by Kenneth E. Mayer and William F. Laudenslayer. California Deparment of Fish and Game, 1988. \burl{http://www.dfg.ca.gov/biogeodata/cwhr/pdfs/DFR.pdf}. Accessed 4 December 2012.

Safford, Hugh. Regional Ecologist, USDA Forest Service. Personal communication, 15 August 2013.

Skinner, Carl N. and Chi-Ru Chang. ``Fire Regimes, Past and Present.'' \emph{Sierra Nevada Ecosystem Project: Final report to Congress, vol. II, Assessments and scientific basis for management options}. Davis: University of California, Centers for Water and Wildland Resources, 1996.

Tappeiner, John C., Philip M. McDonald, Douglass F. Roy. ``Tanoak.'' Silvics of North America: 2. Hardwoods. Agriculture Handbook 654. Burns, Russell M., and Barbara H. Honkala, tech. cords. U.S. Department of Agriculture, Forest Service, 1990. \burl{http://www.na.fs.fed.us/spfo/pubs/silvics\_manual/volume\_2/quercus/chrysolepis.htm}. Accessed 7 December 2012.

Uchytil, Ronald J. ``Pseudotsuga menziesii var. menziesii''.  \emph{Fire Effects Information System}, U.S. Department of Agriculture, Forest Service,  Rocky Mountain Research Station, Fire Sciences Laboratory, 1991. \burl{http://www.fs.fed.us/database/feis/plants/tree/quekel/all.html}. Accessed 21 December 2012.

Van de Water, Kip M. and Hugh D. Safford. ``A Summary of Fire Frequency Estimates for California Vegetation Before Euro-American Settlement.'' \emph{Fire Ecology} 7.3 (2011): 26-57. doi: 10.4996/fireecology.0703026.
\end{hangparas}

% !TEX root = master.tex
\newpage
\section{Montane Riparian (MRIP)}
\label{mrip-description}

\subsection*{General Information}

\subsubsection{Cover Type Overview}

\textbf{Montane Riparian (MRIP)}
\newline
Crosswalks
\begin{itemize}
	\item EVeg: Regional Dominance Type 1
	\begin{itemize}
		\item Riparian Mixed Hardwood
		\item White Alder
		\item Willow
		\item Black Cottonwood
		\item Willow - Alder
		\item Mountain Alder
		\item Willow (Shrub)
	\end{itemize}

	\item LandFire BpS Model
	\begin{itemize}
		\item 0611520: California Montane Riparian Systems 
	\end{itemize}

	\item Presettlement Fire Regime Type
	\begin{itemize}
		\item N/A
	\end{itemize}
\end{itemize}

\noindent Reviewed by Sarah Sawyer, Assistant Pacific Southwest Regional Ecologist, USDA Forest Service

\subsubsection{Vegetation Description}
This system often occurs as a highly variable mosaic of multiple communities that are tree-dominated with a diverse shrub component. The variety of plant associations connected to this system reflect elevation, stream gradient, floodplain width, and flooding events. Usually, the montane riparian zone occurs as a narrow, often dense grove of broad-leaved, winter deciduous trees with a sparse understory. At high mountain elevations, there are usually more shrubs in the understory. At high elevations, the type may not be well developed or may occur in the shrub stage only (LandFire 2007, Grenfell 1988). Due to the methodology of assigning the landscape to particular landcover types, the montane riparian type is limited to those sites determined to be dominated by the species assemblages listed in the above crosswalk section. While we recognize that the riparian zone commonly includes areas near watercourses that are dominated by conifers and other trees, for the purposes of this model those sites have been sorted into the pertinent landcover type in accordance with the dominant vegetation observed. We do not have the capacity at this time to groundtruth or map riparian zones based on understory or midstory vegetation.

Characteristic species are many, including those from the following genera: \emph{Acer}, \emph{Alnus}, \emph{Cornus}, \emph{Populus}, \emph{Rhododendron}, and \emph{Salix}. These habitats can occur as \emph{Alnus} or \emph{Salix} stringers along streams of seeps. In other situations an overstory of \emph{Populus} and/or \emph{Alnus} may be present (Grenfell 1988). Other tree species may include \emph{Pseudotsuga menziesii}, \emph{Platanus racemosa}, and \emph{Quercus agrifolia}. At lower elevations, the riparian areas may contain \emph{Arbutus menziesii}, \emph{Lithocarpus densiflorus}, \emph{Umbellularia californica}, \emph{Cornus}, \emph{Acer} and \emph{Fraxinus}. \emph{Salix} species are common throughout, following a series of species as elevation increases (LandFire 2007).


\subsubsection{Distribution}
MRIP is associated with montane lakes, ponds, seeps, bogs and meadows as well as rivers, streams and springs. Water may be permanent or ephemeral. The transition between MRIP and adjacent non-riparian vegetation may be abrupt, especially where the topography is steep. Typically, this vegetation type occurs below 2440 m (8000 ft) (Grenfell 1988).

\subsection*{Disturbances}

\subsubsection{Wildfire}
Fire frequency is highly variable within the riparian zone. Factors that include but are not limited to topography, elevation, climate, dominant vegetation, and existing vegetation all affect fire frequency and intensity. Riparian zones are heavily influenced by the fire regime of adjacent landcover types and so are still susceptible to disturbance by wildfire, even frequent and high mortality fires. Streams also act as an inhibitor of fire spread, thus contributing to spatial and temporal diversity of landscapes beyond what their relative area would suggest (Grenfell 1988). 

In some forested riparian areas, pre-fire suppression fire return intervals were likely lower than adjacent uplands, while in others, fire frequency appears to have been comparable in riparian and upland areas. FRI values are shorter for riparian zones bordering narrow streams compared to zones around wider and deeper streams. In arid ecosystems, FRIs may be shorter than the surrounding areas in part because the increased productivity of these sites results in more fuels to carry fire. Lower elevation and adjacency to fire-tolerant vegetation also contribute to shorter FRIs for some riparian areas (Sawyer 2013).

Estimates of fire rotations are available from the LandFire project and a review paper (LandFire 2007, Van de Water and Safford (2011). The LandFire project’s published fire return intervals are based on a series of associated models created using the Vegetation Dynamics Development Tool (VDDT). In VDDT, fires are specified concurrently with the transition that follows them. For example, a replacement fire causes a transition to the early development stage. In the RMLands model, such fires are classified as high mortality. However, in VDDT mixed severity fires may cause a transition to early development, a transition to a more open seral stage, or no transition at all. In this case, we categorize the first example as a high mortality fire, and the second and third examples as a low mortality fire. Based on this approach, we calculated fire rotations and the probability of high mortality fire for each of the three MRIP seral stages (Table~\ref{tab:mripdesc_fire}). We computed the overall target fire rotation of 53 years based on values from Van de Water and Safford (2011). 




\begin{table}[!htbp]
\footnotesize
\centering
\caption{Fire rotation index values and probability of high severity fire (at least 75\% overstory tree mortality) probabilities. The seral stage that is most susceptible to fire (i.e., has the lowest predicted fire rotation) has a fire rotation index value of 1. Higher values correspond with lower likelihoods of experiencing wildfire. The values are relative only within an individual seral stage and should not be compared against other land cover types. Values were derived from VDDT model 0611520 (LandFire 2007) and Van de Water and Safford (2011). }
\label{tab:mripdesc_fire}
\begin{tabular}{@{}lcc@{}}
\toprule
 \textbf{Seral Stage}    & \textbf{\begin{tabular}[c]{@{}c@{}}Fire Rotation \\ Index\end{tabular}} & \textbf{\begin{tabular}[c]{@{}c@{}}Probability of \\ High Severity Fire\end{tabular}} \\ \hline
Early (All)       & 1.0  & 1        \\
Mid--Open   		& 1.0  & 0.5        \\
Late--Closed      & 1.0  & 0.5      \\ 
\emph{Target Fire Rotation}    			& \emph{53 years}  &   \\ 
\bottomrule
\end{tabular}
\end{table}

\subsubsection{Other Disturbance}
Other disturbances are not currently modeled, but may, depending on the seral stage affected and mortality levels, reset patches to early development, maintain existing stages, or shift/accelerate succession to a more open stage. 

\subsection*{Vegetation Seral Stages}
We recognize three separate seral stages for MRIP: Early Development (ED), Mid Development - Open Canopy Cover (MDO), and Late Development - Open Canopy Cover (LDO) (Figure~\ref{mrip_transmodel}). Our seral stages are an alternative to ``successional'' classes that imply a linear progression of states and tend not to incorporate disturbance. The seral stages identified here are derived from a combination of successional processes and anthropogenic and natural disturbance, and are intended to represent a composition and structural condition that can be arrived at from multiple other conditions described for that landcover type. Thus our seral stages incorporate age, size, canopy cover, and vegetation composition. In general, the delineation of stages has originated from the LandFire biophysical setting model descriptive of a given landcover type; however, seral stages are not necessarily identical to the classes identified in those models.

\begin{figure}[htbp]
\centering
\includegraphics[width=0.8\textwidth]{/Users/mmallek/Documents/Thesis/statetransmodel/StateTransitionModel/mrip.png}
\caption{State and Transition Model for Montane Riparian Forest. Each dark grey box represents one of the three seral stages for this landcover type. Three stages of development are represented: early, middle, and late. We describe the middle development stage as characterized by open canopy cover and the late development stage as characterized by closed canopy cover, but these are not hard and fast rules. Transitions between states/seral stages may occur as a result of high mortality fire, low mortality fire, or succession. Specific pathways for each are denoted by the appropriate color line and arrow: red lines relate to high mortality fire, orange lines relate to low mortality fire, and green lines relate to natural succession.} 
\label{mrip_transmodel}
\end{figure}



\paragraph{Early Development (ED)}

\paragraph{Description} Immediate post-disturbance responses are dependent on pre-burn vegetation composition. Typically tree dominated, but shrubs may co-dominate. \emph{Salix} and \emph{Alnus} are common, though overall composition is highly variable (LandFire 2007).

\paragraph{Succession Transition} In the absence of disturbance, patches in this seral stage will transition to MDO at 10 years.

\paragraph{Wildfire Transition} High mortality wildfire (100\% of fires in this seral stage) recycles the patch through the ED seral stage. Low mortality wildfire is not modeled for this seral stage.

\noindent\hrulefill


\paragraph{Mid Development - Open Canopy Cover (MDO)}

\paragraph{Description} Vegetation composition in this seral stage includes tall trees and shrubs. \emph{Salix}, \emph{Populus}, and \emph{Alnus} are common. Patches in MDO are more susceptible to fire than the early seral stage (LandFire 2007).

\paragraph{Succession Transition} After 20 years without a wildfire-triggered transition, patches in this seral stage will succeed to LDO.

\paragraph{Wildfire Transition} High mortality wildfire (50\% of fires in this seral stage) recycles the patch through the ED seral stage. Low mortality wildfire (50\%) does not effect a change in the MDO seral stage.

\noindent\hrulefill


\paragraph{Late Development - Open Canopy Cover (LDO)}

\paragraph{Description} This class represents the mature, large \emph{Populus}, \emph{Alnus}, etc. woodlands (LandFire 2007).

\paragraph{Succession Transition} In the absence of disturbance, patches in this seral stage will maintain, regardless of soil characteristics.

\paragraph{Wildfire Transition} High mortality wildfire (50\% of fires) recycles the patch through the ED seral stage. Low mortality wildfire (50\%) does not effect a change in the LDO seral stage.

\noindent\hrulefill





\subsection*{Seral Stage Classification}
\begin{table}[hbp]
\footnotesize
\centering
\caption{Classification of seral stage for MEG. Diameter at Breast Height (DBH) and Cover From Above (CFA) values taken from EVeg polygons. DBH categories are: null, 0-0.9'', 1-4.9'', 5-9.9'', 10-19.9'', 20-29.9'', 30''+. CFA categories are null, 0-10\%, 10-20\%, \dots , 90-100\%. Each row in the table below should be read with a boolean AND across each column.}
\label{mrip_classification}
\begin{tabular}{@{}lrrrrr@{}}
\toprule
\textbf{\begin{tabular}[l]{@{}l@{}}Cover \\ Condition\end{tabular}} & \textbf{\begin{tabular}[r]{@{}r@{}}Overstory Tree \\ Diameter 1 \\ (DBH)\end{tabular}} & \textbf{\begin{tabular}[r]{@{}r@{}}Overstory Tree \\ Diameter 2 \\ (DBH)\end{tabular}} & \textbf{\begin{tabular}[r]{@{}r@{}}Total Tree\\ CFA (\%)\end{tabular}} & \textbf{\begin{tabular}[r]{@{}r@{}}Conifer \\ CFA (\%)\end{tabular}} & \textbf{\begin{tabular}[r]{@{}r@{}}Hardwood \\ CFA (\%)\end{tabular}} \\ \midrule
Early            & Null           & any & any & any & any \\
Early            & 0-9.9''         & any & any & any & any \\
Mid Open         & 10-19.9''       & any & any & any & any \\
Late Open        & 20-30''+        & any & any & any & any \\ \bottomrule
\end{tabular}
\end{table}


\subsection*{References}
\begin{hangparas}{.25in}{1} 
\interlinepenalty=10000
Grenfell, Jr., William E. ``Montane Riparian (MRI).'' \emph{A Guide to Wildlife Habitats of California}, edited by Kenneth E. Mayer and William F. Laudenslayer. California Deparment of Fish and Game, 1988. \burl{http://www.dfg.ca.gov/biogeodata/cwhr/pdfs/MRI.pdf}. Accessed 4 December 2012.

LandFire. ``Biophysical Setting Models.'' Biophysical Setting 0611520: California Montane Riparian Systems. 2007. LANDFIRE Project, U.S. Department of Agriculture, Forest Service; U.S. Department of the Interior. \burl{http://www.landfire.gov/national_veg_models_op2.php}. Accessed 9 November 2012.

Sawyer, Sarah C. ``Natural Range of Variation of Non-Meadow Riparian Habitat in the Bioregional Assessment Area'' (unpublished paper, Ecology Group, Pacific Southwest Research Station, 2013).

Skinner, Carl N. and Chi-Ru Chang. ``Fire Regimes, Past and Present.'' \emph{Sierra Nevada Ecosystem Project: Final report to Congress, vol. II, Assessments and scientific basis for management options}. Davis: University of California, Centers for Water and Wildland Resources, 1996.

Van de Water, Kip and Malcom North. ``Fire history of coniferous riparian forests in the Sierra Nevada.'' \emph{Forest Ecology and Management} 260: 384-395. 2010.

\end{hangparas}





% !TEX root = master.tex
\newpage
\section{Oak Woodland (OAK)}
\label{oak-description}

\subsection*{General Information}

\subsubsection{Cover Type Overview}

\textbf{Oak Woodland (OAK)} 
\newline
\textbf{Crosswalks}
\begin{itemize}
	\item EVeg: Regional Dominance Type 1
	\begin{itemize}
		\item Gray Pine
		\item Blue Oak
		\item Valley Oak
	\end{itemize}

	\item LandFire BpS Model
	\begin{itemize}
		\item 0611140: California Lower Montane Blue Oak-Foothill Pine Woodland and Savanna
	\end{itemize}

	\item Presettlement Fire Regime Type
	\begin{itemize}
		\item Oak Woodland
	\end{itemize}
\end{itemize}

\noindent Reviewed by Becky Estes, Central Sierra Province Ecologist, USDA Forest Service

\subsubsection{Vegetation Description}
The Oak Woodland landcover type is characterized by savannas, woodlands, or forests of either monospecific or mixed stands of various oak species. \emph{Quercus douglasii}, \emph{Quercus lobata}, \emph{Quercus wislizenii}, and \emph{Quercus garryana} are the major dominants. In oak forests where mixtures of tree oak and conifer species exist \emph{Quercus kelloggii} and \emph{Quercus chrysolepis} occur along with \emph{Pinus sabiniana} (Allen-Diaz et al. 2007). 

Both \emph{Q. douglasii} and \emph{Q. lobata} are endemic to California. \emph{Q. lobata} are among the oldest and largest oaks in North America. Tree age can exceed 500 years. \emph{Q. douglasii} are relatively slow-growing, long-lived trees. On \emph{Q. douglasii-P. sabiniana} woodlands, \emph{P. sabiniana} is taller and dominates the overstory, but is shorter-lived (at approximately 80 years) than \emph{Q. douglasii} (150-250 years). \emph{Q. douglasii} is usually the more abundant of the two trees, but \emph{P. sabiniana} contributes as much basal area as \emph{Q. douglasii} (Allen-Diaz et al. 2007).

Typical vegetation is dominated by open oak savannah with relatively uniform mature trees at low densities (less than 40\% cover), where understory vegetation structure is a function of frequent surface fire that mediates woody plant development. In some instances and in some sites tree density will increase to 70\% or more, forming a relatively stable hardwood forest type subject to surface fires in the hardwood litter and rare stand replacement fire (LandFire 2007).

In riparian forests, associates include \emph{Platanus racemosa}, \emph{Juglans hindsii}, \emph{Acer negundo}, \emph{Populus fremontii}, \emph{Salix}, and \emph{Fraxinus latifolia}. In drier areas and open woodlands, shrubs usually clump together in open areas with full sun. Species may include \emph{Aesculus californica}, \emph{Ceanothus}, \emph{Arctostaphylos}, \emph{Rhamnus}, \emph{Toxicodendron diversilobum}, and \emph{Cercis occidentalis} (Allen-Diaz et al. 2007). The shrub layer is best developed along natural drainages, becoming insignificant in the uplands. Ground cover consists of a well-developed carpet of grasses and forbs (Ritter 1988b). Common forbs include \emph{Daucus}, \emph{Geranium}, \emph{Madia}, and \emph{Trifolium}. Most understory cover is created by annual grasses, including \emph{Bromus}, \emph{Lolium}, and \emph{Hordeum} (Allen-Diaz et al. 2007).

Oak recruitment is poor in many areas today, due to both natural and human causes. Many stands exist as groups of medium-to-large trees with few or no young oaks. There is concern that these woodlands may be slowly changing into savannas and grasslands as trees die and are not replaced. Mortality of oak saplings seems to be related to competition for moisture with grasses and forbs, wild and domestic animals feeding on acorns and seedlings, fire suppression, and flood control. Most recent work suggests that recruitment is limited not by reproduction, but by the establishment and survival of saplings (Allen-Diaz et al. 2007).


\subsubsection{Distribution}
Oak Woodland has a patchy distribution embedded in a matrix of agriculture, urban development, grasslands, riparian forests, and other conifer and oak woodland types. It occurs in a band along the western Sierra Nevada foothills, generally below 800 m (2642 feet) in elevation, although individual species described here are capable of surviving at higher elevations. In general, tree density is highest along natural drainages with deeper soils, and lower in uplands and on steeper slopes. The transition from savanna to woodland to forest is largely driven by soil, precipitation, and elevation (Allen-Diaz et al. 2007).

Soils in this type vary significantly, with different types conducive to the establishment of differing dominant tree species. \emph{Q. lobata} is best developed on deep, well-drained alluvial soils, usually in valley bottoms (Ritter 1988b). \emph{Q. wislizeni} becomes more abundant on steeper slopes, shallower soils, and at higher elevations. \emph{Q. douglasii} woodlands occur on a wide range of soils; however, they are often shallow, rocky, infertile, and well drained. The overstory ranges from sparsely scattered trees on poor sites to nearly closed canopies on good quality sites (Allen-Diaz et al. 2007, Ritter 1988a). \emph{Q. douglasii-P. sabiniana} woodlands are found on variety of generally well-drained parent materials, ranging from gravelly loam through stony clay loam. They occupy steeper, drier slopes with shallower and rockier soils than pure oak woodlands (Verner 1988). 


\subsection*{Disturbances}

\subsubsection{Wildfire}
An overstory dominated by deciduous hardwood species results in an herbaceous surface fuel complex dominating fuel/fire influences (LandFire 2007). Because of the long period of human habitation of oak woodlands, it is extremely difficult to define the ``natural'' fire regime. Lightning-caused fires certainly occurred in the past, but decades may pass between these events. Native Americans used fire in their stewardship of oak woodlands; however, it is difficult to document the frequency, intensity, and extent of burning by Native Americans. Some estimate the fire return interval (FRI) of that period to be around 25 years. The first European settlers continued to use fire as a management practice; burning intervals ranged from 8-15 years. Ranchers continued the practice through the 1950s, but since then fire suppression has emerged as the standard management policy (Allen-Diaz et al. 2007).  

The fire regime which produced this landcover type is thought to be frequent; mortality depends on vegetation vulnerability and wildfire intensity. Younger oaks are fire-sensitive and frequently killed by even low severity fires. However, they typically sprout post-disturbance. Older, decadent oaks are not likely to sprout after being damaged or killed by fire. Therefore, younger stands are more likely to regrow after fires and fire exclusion can have a significant effect on stand structure. \emph{P. sabiniana}’s regeneration is dependent on regeneration from seed, although it, too, is fire-adapted. It also grows faster than \emph{Q. douglasii} and is an important colonizer (Allen-Diaz et al. 2007). 

Estimates of fire rotations are available from the LandFire project and Mallek et al. (2013). The LandFire project’s published fire return intervals are based on a series of associated models created using the Vegetation Dynamics Development Tool (VDDT). In VDDT, fires are specified concurrently with the transition that follows them. For example, a replacement fire causes a transition to the early development stage. In the RMLands model, such fires are classified as high mortality. However, in VDDT mixed severity fires may cause a transition to early development, a transition to a more open seral stage, or no transition at all. In this case, we categorize the first example as a high mortality fire, and the second and third examples as a low mortality fire. Based on this approach, we calculated fire rotations and the probability of high mortality fire for each of the OAK seral stages (Table~\ref{tab:oakdesc_fire}). We computed the overall target fire rotation of 26 years based on values from Mallek et al. (2013). 




\begin{table}[!htbp]
\footnotesize
\centering
\caption{Fire rotation index values and probability of high severity fire (at least 75\% overstory tree mortality) probabilities. The seral stage that is most susceptible to fire (i.e., has the lowest predicted fire rotation) has a fire rotation index value of 1. Higher values correspond with lower likelihoods of experiencing wildfire. The values are relative only within an individual seral stage and should not be compared against other land cover types. Values were derived from BpS model 0610800 (LandFire 2007), Van de Water and Safford (2011), and Safford (pers. comm. 2013).}
\label{tab:oakdesc_fire}
\begin{tabular}{@{}lcc@{}}
\toprule
 \textbf{Seral Stage}    & \textbf{\begin{tabular}[c]{@{}c@{}}Fire Rotation \\ Index\end{tabular}} & \textbf{\begin{tabular}[c]{@{}c@{}}Probability of \\ High Severity Fire\end{tabular}} \\ \hline
Early (All)     		 & 1.3           & 0.01                          \\
Mid--Closed    			 & 1.5           & 0.07                          \\
Mid--Moderate  			 & 1.4           & 0.06                          \\
Mid--Open      			 & 1.2           & 0.05                          \\
Late--Closed   			 & 3.3           & 0.5                           \\
Late--Moderate 			 & 1.5           & 0.18                          \\
Late--Open     			 & 1.0             & 0.08       \\ 
\emph{Target Fire Rotation}    			& \emph{26 years}  &   \\ 
\bottomrule
\end{tabular}
\end{table}

\subsubsection{Other Disturbance}
Other disturbances are not currently modeled, but may, depending on the seral stage affected and mortality levels, reset patches to early development, maintain existing seral stages, or shift/accelerate succession to a more open condition. 

\subsection*{Vegetation Seral Stages}
We recognize seven separate seral stages for OAK: Early Development (ED), Mid Development - Open Canopy Cover (MDO), Mid Development - Moderate Canopy Cover, Mid Development - Closed Canopy Cover (MDC), Late Development - Open Canopy Cover (LDO), Late Development - Moderate Canopy Cover (LDM), and Late Development - Closed Canopy Cover (LDC) (Figure~\ref{oak_transmodel}). Our seral stages are an alternative to ``successional'' classes that imply a linear progression of states and tend not to incorporate disturbance. The seral stages identified here are derived from a combination of successional processes and anthropogenic and natural disturbance, and are intended to represent a composition and structural condition that can be arrived at from multiple other conditions described for that landcover type. Thus our seral stages incorporate age, size, canopy cover, and vegetation composition. In general, the delineation of stages has originated from the LandFire biophysical setting model descriptive of a given landcover type; however, seral stages are not necessarily identical to the classes identified in those models.


\begin{figure}[htbp]
\centering
\includegraphics[width=0.8\textwidth]{/Users/mmallek/Documents/Thesis/statetransmodel/StateTransitionModel/7class.png}
\caption{State and Transition Model for Oak Woodland. Each dark grey box represents one of the seven seral stages for this landcover type. Each column of boxes represents a stage of development: early, middle, and late. Each row of boxes represents a different level of canopy cover: closed (70-100\%), moderate (40-70\%), and open (0-40\%). Transitions between states/seral stages may occur as a result of high mortality fire, low mortality fire, or succession. Specific pathways for each are denoted by the appropriate color line and arrow: red lines relate to high mortality fire, orange lines relate to low mortality fire, and green lines relate to natural succession.} 
\label{oak_transmodel}
\end{figure}

\paragraph{Early Development (ED)}

\paragraph{Description} Post-replacement sapling/regeneration phase. Largely a function of either early seral remaining in early seral due to replacement fire, or due to less common late seral replacement fire. Re-establishment can occur from basal resprouting or sexual reproduction, depending on composition, growth form, and seed dynamics. Patch size likely ranges from very small gap recruitment to areas approximately 100 acres. May include \emph{Q. douglasii}, \emph{Q. chrysolepis}, \emph{Q. garryana}, \emph{P. sabiniana}, and a variety of shrubs (LandFire 2007).


\paragraph{Succession Transition} In the absence of disturbance, patches in this seral stage will begin transitioning to MDM at 20 years at a rate of 0.6 per time step. At 60 years in ED, all remaining patches transition to MDM. On average, patches remain in early development for 28 years.

\paragraph{Wildfire Transition} High mortality wildfire (1\% of fires in this seral stage) recycles the patch through the ED seral stage. No transition occurs as a result of low mortality fire. 

\noindent\hrulefill


\paragraph{Mid Development - Open Canopy Cover (MDO)}

\paragraph{Description} Intermediate phase, older than 20 years. Sparse new recruitment of cohorts occurs in the later stages of this seral stage, leading to an open canopy. Periodic surface fire is relatively common, but replacement fire rare due to low intensity fire type and resilience of typical species to top kill. Patch size is typically in the hundreds of acres. May include \emph{Q. douglasii}, \emph{Q. chrysolepis}, \emph{Q. garryana}, \emph{P. sabiniana}, and a variety of shrubs (LandFire 2007).

\paragraph{Succession Transition} In the absence of stand-replacing disturbance, patches in this seral stage will begin transitioning to MDM at 15 years at a rate of 0.7 per time step. Succession to LDO begins after 40 years in a mid development stage. The rate of succession per time step is 0.7. At 70 years in MDO, all remaining patches transition to LDO. On average, patches remain in mid development for 47 years.

\paragraph{Wildfire Transition} High mortality wildfire (5\% of fires in this seral stage) recycles the patch through the ED seral stage. Low mortality fire (95\%) maintains the MDO seral stage and allows for succession to LDO.

\noindent\hrulefill

\paragraph{Mid Development - Moderate Canopy Cover (MDM)}

\paragraph{Description} Intermediate phase, older than 20 years. Some new recruitment of cohorts occurs in the later stages of this seral stage, resulting in moderate canopy cover. Periodic surface fire is relatively common, but replacement fire rare due to low intensity fire type and resilience of typical species to top kill. Patch size is typically in the hundreds of acres. May include \emph{Q. douglasii}, \emph{Q. chrysolepis}, \emph{Q. garryana}, \emph{P. sabiniana}, and a variety of shrubs (LandFire 2007).

\paragraph{Succession Transition} In the absence of stand-replacing disturbance, patches in this seral stage will begin transitioning to MDC at 15 years at a rate of 0.7 per time step. Succession to LDM begins after 40 years in a mid development stage. The rate of succession per time step is 0.7. At 70 years in MDM, all remaining patches transition to LDM. On average, patches remain in mid development for 47 years.

\paragraph{Wildfire Transition} High mortality wildfire (6\% of fires in this seral stage) recycles the patch through the ED seral stage. Low mortality fire (94\%) maintains the MDM seral stage and allows for succession to LDM.

\noindent\hrulefill

\paragraph{Mid Development - Closed Canopy Cover (MDC)}

\paragraph{Description} Intermediate phase, older than 20 years. Significant new recruitment of cohorts occurs in the later stages of this seral stage, resulting in a closed canopy. Periodic surface fire is relatively common, but replacement fire rare due to low intensity fire type and resilience of typical species to top kill. Patch size is typically in the hundreds of acres. May include \emph{Q. douglasii}, \emph{Q. chrysolepis}, \emph{Q. garryana}, \emph{P. sabiniana}, and a variety of shrubs (LandFire 2007).

\paragraph{Succession Transition} Succession to LDC begins after 40 years in a mid development stage. The rate of succession per time step is 0.7. At 70 years in a mid development stage, all remaining patches transition to LDC. On average, patches remain in mid development for 47 years.

\paragraph{Wildfire Transition} High mortality wildfire (7\% of fires in this seral stage) recycles the patch through the ED seral stage. Low mortality fire (93\%) maintains the MDC seral stage and allows for succession to LDC.

\noindent\hrulefill


\paragraph{Late Development - Open Canopy Cover (LDO)}

\paragraph{Description} Open woodland with mature oak and conifer trees. This seral stage is highly stable, as most fire is frequent, low severity fire acting as a maintenance agent. Tree density and canopy cover increase over time to relatively stable conditions. In some cases woody encroachment and increased tree density occurs under missed fire cycles. If P. sabiniana occurs, it quickly becomes very large. Some replacement fire occurs initiating secondary succession in the ED seral stage. Patch size in the hundreds, to possibly thousands, of acres. Canopy cover ranges from 11-40\%. May include \emph{Q. douglasii, Q. chrysolepis, Q. garryana, P. sabiniana}, and a variety of shrubs (LandFire 2007).

\paragraph{Succession Transition} In the absence of disturbance, patches in this seral stage will begin transitioning to LDM after 15 years at a rate of 0.7 per time step. 

\paragraph{Wildfire Transition} High mortality wildfire (8\% of fires in this seral stage) recycles the patch through the ED seral stage. Low mortality fire (92\%) maintains the LDO seral stage.

\noindent\hrulefill

\paragraph{Late Development - Moderate Canopy Cover (LDM)}

\paragraph{Description} Woodland with mature oak and conifer trees. This seral stage is fairly stable, as fire tends to be frequent, low severity fire acting as a maintenance agent. Tree density and canopy cover are increasing over time due to missed fire cycles or high productivity. Periodic surface fire is relatively common, but replacement fire is uncommon due to low intensity fire type and resilience of typical species to top kill. If \emph{P. sabiniana} occurs, it quickly becomes very large. Patch size is  in the hundreds of acres. Canopy cover ranges from 40-70\%. May include \emph{Q. douglasii}, \emph{Q. chrysolepis}, \emph{Q. garryana}, \emph{P. sabiniana}, and a variety of shrubs (LandFire 2007).

\paragraph{Succession Transition} In the absence of disturbance, patches in this seral stage will begin transitioning to LDC after 15 years at a rate of 0.7 per time step. 

\paragraph{Wildfire Transition} High mortality wildfire (18\% of fires in this seral stage) recycles the patch through the ED seral stage. Low mortality fire (82\%) opens the patch up to LDO 14\% of the time; otherwise, the patch remains in LDM.

\noindent\hrulefill

\paragraph{Late Development - Closed Canopy Cover (LDC)}

\paragraph{Description} Late seral stage arising from a rare period of no fire in the LDM seral stage for at least 15 years, allowing woody understory encroachment and higher tree density. If \emph{P. sabiniana} occurs, it quickly becomes very large. Fire that does not effect a change in seral stage is rare; low mortality fire is the normal pathway back to late development, open seral stages, while high mortality results in a return to early seral conditions. Patch size is likely in the tens of acres. May include \emph{Q. douglasii}, \emph{Q. chrysolepis}, \emph{Q. garryana}, \emph{P. sabiniana}, and a variety of shrubs. If the closed seral stage persists for decades and \emph{P. sabiniana} is present, it can begin to shade out the oak trees (LandFire 2007).

\paragraph{Succession Transition} In the absence of disturbance, patches in this seral stage will maintain.

\paragraph{Wildfire Transition} High mortality wildfire (50\% of fires in this seral stage) recycles the patch through the ED seral stage. Low mortality fire (50\%) opens the patch up to LDM.

\noindent\hrulefill

%\clearpage
\subsection*{Seral Stage Classification}
\begin{table}[hbp]
\footnotesize
\centering
\caption{Classification of seral stage for OAK. Diameter at Breast Height (DBH) and Cover From Above (CFA) values taken from EVeg polygons. DBH categories are: null, 0-0.9'', 1-4.9'', 5-9.9'', 10-19.9'', 20-29.9'', 30''+. CFA categories are null, 0-10\%, 10-20\%, \dots , 90-100\%. Each row in the table below should be read with a boolean AND across each column.}
\label{oak_classification}
\begin{tabular}{@{}lrrrrr@{}}
\toprule
\textbf{\begin{tabular}[l]{@{}l@{}}Cover \\ Condition\end{tabular}} & \textbf{\begin{tabular}[r]{@{}r@{}}Overstory Tree \\ Diameter 1 \\ (DBH)\end{tabular}} & \textbf{\begin{tabular}[r]{@{}r@{}}Overstory Tree \\ Diameter 2 \\ (DBH)\end{tabular}} & \textbf{\begin{tabular}[r]{@{}r@{}}Total Tree\\ CFA (\%)\end{tabular}} & \textbf{\begin{tabular}[r]{@{}r@{}}Conifer \\ CFA (\%)\end{tabular}} & \textbf{\begin{tabular}[r]{@{}r@{}}Hardwood \\ CFA (\%)\end{tabular}} \\ \midrule
Early            & 0-4.9''         & any & any    & any    & any    \\
Mid Open         & 5-9.9''         & any & 0-40   & any    & any    \\
Mid Moderate     & 5-9.9''         & any & 40-70  & any    & any    \\
Mid Closed       & 5-9.9''         & any & 70-100 & any    & any    \\
Late Open        & 10''+           & any & 0-40   & any    & any    \\
Late Open        & 10''+           & any & null   & 0-40   & 0-40   \\
Late Moderate    & 10''+           & any & 40-70  & any    & any    \\
Late Moderate    & 10''+           & any & null   & 40-70  & 0-70   \\
Late Moderate    & 10''+           & any & null   & 0-70   & 40-70  \\
Late Closed      & 10''+           & any & 70-100 & any    & any    \\
Late Closed      & 10''+           & any & null   & 70-100 & any    \\
Late Closed      & 10''+           & any & null   & any    & 70-100 \\ \bottomrule
\end{tabular}
\end{table}


\clearpage

\subsection*{References}

\begin{hangparas}{.25in}{1} 
\interlinepenalty=10000
Allen-Diaz, Barbara, Richard Standiford, and Randall D. Jackson. ``Oak Woodlands and Forests.'' In \emph{Terrestrial Vegetation of California, 3rd Edition}, edited by Michael Barbour, Todd Keeler-Wolf, and Allan A. Schoenherr, 313-338. Berkeley and Los Angeles: University of California Press, 2007. 

``CalVeg Zone 1.'' Vegetation Descriptions. \emph{Vegetation Classification and Mapping}.  11 December 2008. U.S. Forest Service. \burl{http://www.fs.usda.gov/Internet/FSE_DOCUMENTS/fsbdev3_046448.pdf}. Accessed 2 April 2013.

LandFire. ``Biophysical Setting Models.'' Biophysical Setting 0611140: California Lower Montane Blue Oak-Foothill Pine Woodland and Savanna. 2007. LANDFIRE Project, U.S. Department of Agriculture, Forest Service; U.S. Department of the Interior. \burl{http://www.landfire.gov/national_veg_models_op2.php}. Accessed 9 November 2012.

Ritter, Lyman V. ``Blue Oak Woodland (BOW).'' \emph{A Guide to Wildlife Habitats of California}, edited by Kenneth E. Mayer and William F. Laudenslayer. California Deparment of Fish and Game, 1988a. \burl{http://www.dfg.ca.gov/biogeodata/cwhr/pdfs/BOW.pdf}. Accessed 4 December 2012.

Ritter, Lyman V. ``Valley Oak Woodland (VOW).'' \emph{A Guide to Wildlife Habitats of California}, edited by Kenneth E. Mayer and William F. Laudenslayer. California Deparment of Fish and Game, 1988b. \burl{http://www.dfg.ca.gov/biogeodata/cwhr/pdfs/VOW.pdf}. Accessed 4 December 2012.

Skinner, Carl N. and Chi-Ru Chang. ``Fire Regimes, Past and Present.'' \emph{Sierra Nevada Ecosystem Project: Final report to Congress, vol. II, Assessments and scientific basis for management options}. Davis: University of California, Centers for Water and Wildland Resources, 1996.

Van de Water, Kip M. and Hugh D. Safford. ``A Summary of Fire Frequency Estimates for California Vegetation Before Euro-American Settlement.'' \emph{Fire Ecology} 7.3 (2011): 26-57. doi: 10.4996/fireecology.0703026.

Verner, Jared. ``Blue Oak-Foothill Pine (BOP).'' \emph{A Guide to Wildlife Habitats of California}, edited by Kenneth E. Mayer and William F. Laudenslayer. California Deparment of Fish and Game, 1988. \burl{http://www.dfg.ca.gov/biogeodata/cwhr/pdfs/BOP.pdf}. Accessed 4 December 2012. 

\end{hangparas}


% !TEX root = master.tex
\newpage
\section{Oak-Conifer Forest and Woodland (OCFW)}
\label{ocfw-description}

\subsection*{General Information}

\subsubsection{Cover Type Overview}

\textbf{Oak-Conifer Forest and Woodland (OCFW)}
\newline
Crosswalks
\begin{itemize}
	\item East of the Sierra Crest
	\begin{itemize}
		\item Eveg: Regional Dominance Type 1
		\begin{itemize}
			\item Black Oak
			\item Eastside Pine
			\item Jeffrey Pine
			\item Ponderosa Pine
		\end{itemize}
		\emph{And}
		\item Eveg: Regional Dominance Type 2
		\begin{itemize}
			\item Black Oak
			\item Canyon Live Oak
			\item Madrone
			\item Montane Mixed Hardwood
			\item Scrub Oak
		\end{itemize}
	\end{itemize}

	\item West of the Sierra Crest
	\begin{itemize}
		\item Eveg: Regional Dominance Type 1
		\begin{itemize}
			\item Black Oak
			\item Eastside Pine
			\item Jeffrey Pine
			\item Ponderosa Pine
		\end{itemize}

		\item LandFire BpS Model
		\begin{itemize}
			\item 0610300 Mediterranean California Lower Montane Black Oak-Conifer Forest and Woodland
		\end{itemize}
		
		\item Presettlement Fire Regime Type
		\begin{itemize}
			\item Yellow Pine
		\end{itemize}
\end{itemize}
\end{itemize}

Modifiers
\begin{itemize}
	\item Ultramafic: This type is created by intersecting an ultramafic soils/geology layer with the existing vegetation layer. Where cells intersect with OCFW they are assigned to the ultramafic modifier.
\end{itemize}

\noindent Reviewed by Becky Estes, Central Sierra Province Ecologist, USDA Forest Service; Kyle Merriam, Sierra-Cascade Province Ecologist, USDA Forest Service


\subsubsection{Vegetation Description}
\textbf{Oak-Conifer Forest and Woodland (OCFW)} The Oak-Conifer Forest and Woodland landcover type is characterized by woodlands or forests of \emph{Pinus ponderosa} or \emph{Pinus jeffreyi} with one or more oaks, such as \emph{Quercus kelloggii}, \emph{Quercus garryana}, \emph{Quercus wislizeni}, or \emph{Quercus chrysolepis}. \emph{Pseudotsuga menziesii} and other conifer species are uncommon but may co-occur, especially after long-term fire suppression (LandFire 2007a). \emph{Pinus jeffreyi} tends to dominate on ultramafic sites (Fitzhugh 1988). In some areas, sites are dominated initially by oaks, which form a dense subcanopy. Eventually, and especially on locally mesic sites, conifers will form a persistent emergent canopy over the oak as a bi-layered canopy (LandFire 2007a). In other cases, characteristic species occur in a mosaic-like pattern with small pure stands of conifers interspersed with small stands of broad-leaved trees. Most of the broad-leaved trees are schlerophyllous evergreen, but winter-deciduous species also occur (Anderson 1988). The understory is composed of shrubs such as \emph{Arctostaphylos}, \emph{Ceanothus, Chamaebatia, Cornus, Eriodictyon, Garrya, Prunus, Rhamnus, Ribes,} and \emph{Toxicodendron diversilobum}. Grasses and forbs are diverse and include \emph{Bromus}, \emph{Melica}, \emph{Poa}, \emph{Elymus}, \emph{Carex}, \emph{Collinsia}, \emph{Saltugilia}, \emph{Iris}, \emph{Lupinus}, \emph{Streptanthus}, \emph{Viola}, and \emph{Pteridium aquilnum} (LandFire 2007a, Fitzhugh 1988).

\begin{adjustwidth}{2cm}{}

\textbf{Ultramafic Modifier (OCFW\_U)}  \emph{P. ponderosa} or \emph{P. jeffreyi} woodlands occur mainly on low-elevation ultramafics. They grow on strongly serpentinized soil, and are typically adjacent to the non-ultramafic form of the cover type. While \emph{P. ponderosa} or \emph{P. jeffreyi} dominates, it may be associated with \emph{Calocedrus decurrens, Pinus attentuata, Pinus lambertiana, P. sabiniana}, and \emph{Q. chysolepis} (O'Geen et al. 2007). \emph{Q. kelloggi} is rare on ultramafic soils (Fryer 2007). The shrub layer is dominated by \emph{Arctostaphylos, Ceanothus, Eriodictyon, Heteromeles}, and \emph{Pickeringia}. The herb layer is a mix of sparse perennials and many annual grasses and forbs (O'Geen et al. 2007). 

\end{adjustwidth}


\subsubsection{Distribution}
This type occurs in the valleys and lower slopes of mountainous terrain, on a variety of parent materials including granitics, metamorphic and Franciscan metasedimentary parent material and deep, well developed soils, although rocky soils are also possible. Slopes are generally steep and all aspects are included. In the northern Sierra Nevada the elevational range is 240 to 1800 m (800 to 5000 ft) (LandFire 2007a, Anderson 1988).

\begin{adjustwidth}{2cm}{}

\textbf{Ultramafic Modifier} Ultramafics have been mapped at various spatial densities throughout the elevational range of the OCFW landcover type. Low to moderate elevations in ultramafic and serpentinized areas often produce soils low in essential minerals like calcium potassium, and nitrogen, and have excessive accumulations of heavy metals such as nickel and chromium. These sites vary widely in the degree of serpentinization and effects on their overlying plant communities (``CalVeg Zone 1'' 2011). Note, the terms ``ultramafic rock'' and ``serpentine'' are broad terms used to describe a number of different but related rock types, including serpentinite, peridotite, dunite, pyroxenite, talc and soapstone, among others (O'Geen et al. 2007).


\end{adjustwidth}

\subsection*{Disturbances}

\subsubsection{Wildfire}
Wildfires are common and frequent; mortality depends on vegetation vulnerability and wildfire intensity. Low mortality fires kill small trees and consume above-ground portions of shrubs and herbs, but do not kill large trees or below-ground organs of most shrubs and herbs which promptly re-sprout. High mortality fires kill large as well as small trees, and may kill many of the shrubs and herbs as well. Fire kills the above-ground portions of the shrubs and herbs, but most shrubs and herbs promptly resprout from surviving below-ground organs. Wildfires may trigger transitions between seral stages.

OCFW sites are fire-adapted and had frequent, low severity surface fires prior to fire exclusion in the late nineteenth century. Historically, fire return intervals (FRIs) in \emph{P. ponderosa-Q. kelloggii} forests increased with increasing elevation in the Sierra Nevada, with a tendency towards shorter mean FRIs (5-15 years) on dry, west- and south-facing slopes and longer FRIs (15-25 years) on mesic, east- and north-facing slopes. Mid-elevation forests typically had mixed-severity fires that created patchy mosaics (Fryer 2007).

Estimates of fire rotations for these variants are available from the LandFire project and a few review papers. The LandFire project’s published fire return intervals are based on a series of associated models created using the Vegetation Dynamics Development Tool (VDDT). In VDDT, fires are specified concurrently with the transition that follows them. For example, a replacement fire causes a transition to the early development stage. In the RMLands model, such fires are classified as high mortality. However, in VDDT mixed severity fires may cause a transition to early development, a transition to a more open seral stage, or no transition at all. In this case, we categorize the first example as a high mortality fire, and the second and third examples as a low mortality fire. Based on this approach, we calculated fire rotations and the probability of high mortality fire for each of the OCFW seral stages (including the ultramafic modifier) (Tables~\ref{tab:ocfwdesc_fire} and \ref{tab:ocfwudesc_fire}). We computed overall target fire rotations based on expert input from Safford and Estes, and values from Mallek et al. (2013), and Van de Water and Safford (2011). 




\begin{table}[!htbp]
\footnotesize
\centering
\caption{Fire rotation index values and probability of high severity fire (at least 75\% overstory tree mortality) probabilities for Oak-Conifer Forest and Woodland. The seral stage that is most susceptible to fire (i.e., has the lowest predicted fire rotation) has a fire rotation index value of 1. Higher values correspond with lower likelihoods of experiencing wildfire. The values are relative only within an individual seral stage and should not be compared against other land cover types. Values were derived from VDDT model 0610300 (LandFire 2007a), Mallek et al. (2013), and Safford and Estes (personal communication). }
\label{tab:ocfwdesc_fire}
\begin{tabular}{@{}lcc@{}}
\toprule
 \textbf{Seral Stage}    & \textbf{\begin{tabular}[c]{@{}c@{}}Fire Rotation \\ Index\end{tabular}} & \textbf{\begin{tabular}[c]{@{}c@{}}Probability of \\ High Severity Fire\end{tabular}} \\ \hline
Early (All)     		 & 3.8            & 1                             \\
Mid--Closed    			 & 1.4            & 0.26                          \\
Mid--Moderate  			 & 1.2           & 0.14                          \\
Mid--Open      			 & 1.0           & 0.05                          \\
Late--Closed   			 & 1.9           & 0.20                          \\
Late--Moderate 			 & 1.3           & 0.08                          \\
Late--Open     			 & 1.0           & 0.01        \\ 
\emph{Target Fire Rotation}    			& \emph{21 years}  &   \\ 
\bottomrule
\end{tabular}
\end{table}

\begin{table}[!htbp]
\footnotesize
\centering
\caption{Fire rotation index values and probability of high severity fire (at least 75\% overstory tree mortality) probabilities for Oak-Conifer Forest and Woodland - Ultramafic. The seral stage that is most susceptible to fire (i.e., has the lowest predicted fire rotation) has a fire rotation index value of 1. Higher values correspond with lower likelihoods of experiencing wildfire. The values are relative only within an individual seral stage and should not be compared against other land cover types. Values were derived from VDDT model 0610210 (LandFire 2007b), Mallek et al. (2013), and Safford and Estes (personal communication).}
\label{tab:ocfwudesc_fire}
\begin{tabular}{@{}lcc@{}}
\toprule
 \textbf{Seral Stage}    & \textbf{\begin{tabular}[c]{@{}c@{}}Fire Rotation \\ Index\end{tabular}} & \textbf{\begin{tabular}[c]{@{}c@{}}Probability of \\ High Severity Fire\end{tabular}} \\ \hline
Early (All)     		 & 3.8            & 1                             \\
Mid--Closed    			 & 1.4           & 0.26                          \\
Mid--Moderate  			 & 1.2           & 0.14                          \\
Mid--Open      			 & 1.0           & 0.05                          \\
Late--Closed   			 & 1.9           & 0.20                          \\
Late--Moderate 			 & 1.3           & 0.08                          \\
Late--Open     			 & 1.0           & 0.01        \\ 
\emph{Target Fire Rotation}    			& \emph{21 years}  &   \\ 
\bottomrule
\end{tabular}
\end{table}

\subsubsection{Other Disturbance}

\subsection*{Vegetation Seral Stages}
We recognize seven separate seral stages for OCFW and OCFW\_U: Early Development (ED), Mid Development - Open Canopy Cover (MDO), Mid Development - Moderate Canopy Cover, Mid Development - Closed Canopy Cover (MDC), Late Development - Open Canopy Cover (LDO), Late Development - Moderate Canopy Cover (LDM), and Late Development - Closed Canopy Cover (LDC) (Figure~\ref{transmodel_ocfw}). Our seral stages are an alternative to ``successional'' classes that imply a linear progression of states and tend not to incorporate disturbance. The seral stages identified here are derived from a combination of successional processes and anthropogenic and natural disturbance, and are intended to represent a composition and structural condition that can be arrived at from multiple other conditions described for that landcover type. Thus our seral stages incorporate age, size, canopy cover, and vegetation composition. In general, the delineation of stages has originated from the LandFire biophysical setting model descriptive of a given landcover type; however, seral stages are not necessarily identical to the classes identified in those models.

\begin{figure}[htbp]
\centering
\includegraphics[width=0.8\textwidth]{/Users/mmallek/Documents/Thesis/statetransmodel/StateTransitionModel/7class.png}
\caption{State and Transition Model for Oak-Conifer Forest and Woodland. Each dark grey box represents one of the seven seral stages for this landcover type. Each column of boxes represents a stage of development: early, middle, and late. Each row of boxes represents a different level of canopy cover: closed (70-100\%), moderate (40-70\%), and open (0-40\%). Transitions between states/seral stages may occur as a result of high mortality fire, low mortality fire, or succession. Specific pathways for each are denoted by the appropriate color line and arrow: red lines relate to high mortality fire, orange lines relate to low mortality fire, and green lines relate to natural succession.} 
\label{transmodel_ocfw}
\end{figure}

\paragraph{Early Development (ED)} 


\paragraph{Description}
The early seral stage is the initial post-disturbance community dominated by coppicing oak sprouts (predominantly \emph{Q. kelloggi}, but potentially also \emph{Q. chrysolepis}). \emph{T. diversilobum} may be abundant. Bunchgrasses and associated forbs dominate understory. Localized native herbivory may maintain oak sprouts in ``shrub'' form for extended period. Vegetation may also include conifer seedling/saplings (LandFire 2007a).

On sites or areas that are dry or of low quality, significant pine regeneration may depend on concurrent disturbance of shrub species and a good pine seed crop with favorable weather. Thus, it may require 50-100 years for significant pine regeneration in the absence of intervention. Dense brush is typical in young stands and an herbaceous layer may develop on some sites. On drier sites, there is less tendency for succession toward shade-adapted species. As young, dense stands age and attain a closed canopy, they exclude most undergrowth. When other adapted conifers occur in moist pine stands of medium to high site quality, they may form a significant understory in about 20 years in the absence of fire (Fitzhugh 1988).

\paragraph{Succession Transition} In the absence of disturbance, patches in this seral stage will begin transitioning to a mid development seral stage at 20 years. The rate of succession per time step is 0.7. The transition may be to either MDC or MDO. The secondary rate of succession to MDO is 0.4 and to MDC is 0.6. At 50 years, all patches will have succeeded to either MDC or MDO. On average, patches remain in ED for 27 years.
\begin{adjustwidth}{2cm}{}
\medskip
\textbf{Ultramafic Modifier} Succession may be substantially delayed. Thus, in the absence of disturbance, patches in this seral stage will begin transitioning to MDO at 50 years and may be delayed in the ED seral stage for as long as 100 years. A patch in this seral stage succeeds at a rate of 0.2 per time step. 

\end{adjustwidth}
\paragraph{Wildfire Transition}
High mortality wildfire (100\% of fires in this seral stage) recycles the patch through the Early Development seral stage, regardless of soil type. Low mortality wildfire is not modeled for this seral stage.


\noindent\hrulefill


\paragraph{Mid Development - Open Canopy Cover (MDO)}

\paragraph{Description} The mid-seral, open seral stage has hardwoods dominating the canopy and may have sporadic conifer presence at low coverage levels. Oaks are pole-sized to very large. Bunchgrasses and shade-intolerant shrubs, most notably, will be prominent on the majority of sites. This seral stage is distinguished from MDM and MDC primarily by its reduced conifer presence (LandFire 2007a).

\paragraph{Succession Transition} Patches in this seral stage will maintain under low mortality disturbance, but after 15 years without fire they begin transitioning to MDM at a rate of 0.7 per timestep. At 150 years since transitioning to a mid development seral stage, succession to LDO occurs at a rate of 0.3 per timestep. All remaining patches transition at 230 years. 
\begin{adjustwidth}{2cm}{}
\medskip

\textbf{Ultramafic Modifier}  In the absence of low mortality disturbance, patches will begin transitioning to MDC at 30 years at a rate of 0.1 per timestep. At 200 years in the mid development seral stage, succession to LD occurs at a rate of 0.3 per timestep. All remaining patches transition at 280 years.

\end{adjustwidth}
\paragraph{Wildfire Transition}
High mortality wildfire (5\% of fires in this seral stage) recycles the patch through the ED seral stage. Low mortality wildfire (95\%) maintains the patch in MDO.

\noindent\hrulefill

\paragraph{Mid Development - Moderate Canopy Cover (MDM)}

\paragraph{Description} The mid-seral, moderate canopy cover seral stage may represent a drier, hardwood dominated site that has gone without fire for an extended period, or a mesic site supporting both oak and yellow pine species that has been opened up by fire. \emph{P. menziesii} may occur. Oaks are pole to medium sized with moderate crown closure. Conifers are generally medium to large, depending on stand age. Overall canopy cover ranges from 40-70\%. Sod-forming grasses and shade-tolerant shrubs will be prominent on the majority of sites. Species from more arid sites may be remnants of earlier, more open post-fire communities (LandFire 2007a).

\paragraph{Succession Transition} Patches in this seral stage may maintain under low mortality disturbance, but after 15 years without fire they begin transitioning to MDC at a rate of 0.7 per timestep. At 110 years since transitioning to a mid development seral stage, succession to LDO occurs at a rate of 0.3 per timestep. All remaining patches transition at 180 years.
\begin{adjustwidth}{2cm}{}

\medskip
\textbf{Ultramafic Modifier}  In the absence of low mortality disturbance, patches will begin transitioning to MDC at 30 years at a rate of 0.1 per timestep. At 130 years in the mid development seral stage, succession to LDM occurs at a rate of 0.2 per timestep. All remaining patches transition at 250 years.

\end{adjustwidth}
\paragraph{Wildfire Transition}
High mortality wildfire (14\% of fires in this seral stage) recycles the patch through the ED seral stage. Low mortality wildfire (86\%) triggers a transition to MDO 32\% of the time; otherwise the patch remains in MDC.

\noindent\hrulefill

\paragraph{Mid Development - Closed Canopy Cover (MDC)}

\paragraph{Description} The mid-seral, closed seral stage is representative of the more mesic end of the environmental gradient and supports a dense canopy of oak and \emph{P. ponderosa} and/or \emph{P. jeffreyi}. Occasional \emph{P. menziesii} may occur. Oaks are pole to medium sized with crown closure approaching 70\%. Conifers are generally medium to large, depending on stand age. Overall canopy cover is at least 50\%. Sod-forming grasses and shade-tolerant shrubs will be prominent on the majority of sites. Species from more arid sites may be remnants of earlier, more open post-fire communities (LandFire 2007a).

\paragraph{Succession Transition} In the absence of stand-replacing disturbance, patches in this seral stage will begin transitioning to LDC at 80 years in an mid development seral stage at a rate of 0.3 per time step. At 150 years, all remaining patches succeed to LDC.
\begin{adjustwidth}{2cm}{}

\medskip
\textbf{Ultramafic Modifier}  Transition to late seral seral stages may be substatially delayed. Thus, in the absence of stand-replacing disturbance, patches in this seral stage will begin transitioning to LDC after 80 years at a rate of 0.2 per time step and may be delayed in a mid development seral stage for up to 300 years.

\end{adjustwidth}
\paragraph{Wildfire Transition} High mortality wildfire (15\% of fires in this seral stage) recycles the patch through the ED seral stage. Low mortality wildfire (85\%) triggers a transition to MDM 60\% of the time; otherwise the patch remains in MDC.


\noindent\hrulefill


\paragraph{Late Development - Open Canopy Cover (LDO)}

\paragraph{Description} The late-seral seral stage occurs when stand-replacing fire has been excluded from a patch for an extended period of time. Oaks are being overtopped by conifers. Thus, in this seral stage, oaks comprise a smaller proportion of the stand. Oaks and conifers are mature and large (LandFire 2007a). In general, sites are relatively open (Estes 2013).

\paragraph{Succession Transition} Patches in this seral stage will maintain under low mortality disturbance, but after 15 years without fire they begin transitioning to LDM at a rate of 0.7 per timestep. 
\begin{adjustwidth}{2cm}{}

\medskip
\textbf{Ultramafic Modifier}  In the absence of disturbance, patches in LDO will maintain.

\end{adjustwidth}
\paragraph{Wildfire Transition}
High mortality wildfire (1\% of fires in this seral stage) recycles the patch through the ED seral stage. Low mortality wildfire (99\%) maintains the patch in LDO.

\noindent\hrulefill

\paragraph{Late Development - Moderate Canopy Cover (LDM)}

\paragraph{Description} The late-seral seral stage occurs when stand-replacing fire has been excluded from a patch for an extended period of time. Oaks are being overtopped by conifers, including shade-tolerant conifers such as \emph{P. menziesii}. Thus, in this seral stage, oaks and even pines comprise a smaller proportion of the stand. Oaks and conifers are mature and large (LandFire 2007a). 

\paragraph{Succession Transition} Patches in this seral stage will maintain under low mortality disturbance, but after 15 years without fire they begin transitioning to LDC at a rate of 0.7 per timestep.
\begin{adjustwidth}{2cm}{}

\medskip
\textbf{Ultramafic Modifier}  In the absence of disturbance, patches in LDM will maintain.

\end{adjustwidth}
\paragraph{Wildfire Transition} High mortality wildfire (8\% of fires in this seral stage) recycles the patch through the ED seral stage. Low mortality wildfire (92\%) triggers a transition to LDO 18\% of the time; otherwise the patch remains in LDC.

\noindent\hrulefill

\paragraph{Late Development - Closed Canopy Cover (LDC)}

\paragraph{Description} The late-seral seral stage occurs when stand-replacing fire has been excluded from a patch for an extended period of time. Oaks are being overtopped by conifers, especially shade-tolerant conifers such as \emph{P. menziesii}. Thus, in this seral stage, oaks and even pines comprise a smaller proportion of the stand (LandFire 2007a). 

\paragraph{Succession Transition} In the absence of transition-causing disturbance, patches in this seral stage will maintain, regardless of soil characteristics.

\paragraph{Wildfire Transition} High mortality wildfire (20\% of fires in this seral stage) recycles the patch through the ED seral stage. Low mortality wildfire (80\%) triggers a transition to LDM 58\% of the time; otherwise the patch remains in LDC.

\noindent\hrulefill


\newpage
\subsection*{Seral Stage Classification}
\begin{table}[!htbp]
\footnotesize
\centering
\caption{Classification of cover seral stage for OCFW, for early and mid development stages. Diameter at Breast Height (DBH) and Cover From Above (CFA) values taken from EVeg polygons. DBH categories are: null, 0-0.9'', 1-4.9'', 5-9.9'', 10-19.9'', 20-29.9'', 30''+. CFA categories are null, 0-10\%, 10-20\%, \dots , 90-100\%. Each row in the table below should be read with a boolean AND across each column of a row.}
\label{ocfw_classification}
\begin{tabular}{@{}lrrrrr@{}}
\toprule
\textbf{\begin{tabular}[l]{@{}l@{}}Cover \\ Condition\end{tabular}} & \textbf{\begin{tabular}[r]{@{}r@{}}Overstory Tree \\ Diameter 1 \\ (DBH)\end{tabular}} & \textbf{\begin{tabular}[r]{@{}r@{}}Overstory Tree \\ Diameter 2 \\ (DBH)\end{tabular}} & \textbf{\begin{tabular}[r]{@{}r@{}}Total Tree\\ CFA (\%)\end{tabular}} & \textbf{\begin{tabular}[r]{@{}r@{}}Conifer \\ CFA (\%)\end{tabular}} & \textbf{\begin{tabular}[r]{@{}r@{}}Hardwood \\ CFA (\%)\end{tabular}} \\ \midrule
Early All        & null           & null    & any    & any    & any    \\
Early All        & 0-4.9''         & 0-4.9''  & any    & any    & any    \\
Early All        & 0-4.9''         & null    & any    & any    & any    \\
Mid Open         & 0-4.9''         & 5-29.9'' & 0-40   & any    & any    \\
Mid Open         & 5-29.9''        & null    & 0-40   & any    & any    \\
Mid Open         & 5-29.9''        & null    & null   & 0-40   & null   \\
Mid Open         & 5-29.9''        & null    & null   & null   & 0-40   \\
Mid Open         & 5-29.9''        & null    & null   & 0-40   & 0-40   \\
Mid Open         & 5-29.9''        & 0-29.9'' & 0-40   & any    & any    \\
Mid Open         & 5-29.9''        & 0-29.9'' & null   & 0-40   & 0-40   \\
Mid Moderate     & 0-4.9''         & 5-29.9'' & 40-70  & any    & any    \\
Mid Moderate     & 5-29.9''        & null    & 40-70  & any    & any    \\
Mid Moderate     & 5-29.9''        & null    & null   & 40-70  & null   \\
Mid Moderate     & 5-29.9''        & null    & null   & null   & 40-70  \\
Mid Moderate     & 5-29.9''        & null    & null   & 40-70  & 0-70   \\
Mid Moderate     & 5-29.9''        & null    & null   & 0-70   & 40-70  \\
Mid Moderate     & 5-29.9''        & 0-29.9'' & 40-70  & any    & any    \\
Mid Moderate     & 5-29.9''        & 0-29.9'' & null   & 40-70  & 0-70   \\
Mid Moderate     & 5-29.9''        & 0-29.9'' & null   & 0-70   & 40-70  \\
Mid Closed       & 0-4.9''         & 5-29.9'' & 70-100 & any    & any    \\
Mid Closed       & 5-29.9''        & null    & 70-100 & any    & any    \\
Mid Closed       & 5-29.9''        & null    & null   & 70-100 & any    \\
Mid Closed       & 5-29.9''        & null    & null   & any    & 70-100 \\
Mid Closed       & 5-29.9''        & 0-29.9'' & 70-100 & any    & any    \\
Mid Closed       & 5-29.9''        & 0-29.9'' & null   & 70-100 & any    \\
Mid Closed       & 5-29.9''        & 0-29.9'' & null   & any    & 70-100 \\ \bottomrule
\end{tabular}
\end{table}

\begin{table}[!htbp]
\footnotesize
\centering
\caption{Classification of cover seral stage for OCFW, for late development stages. Diameter at Breast Height (DBH) and Cover From Above (CFA) values taken from EVeg polygons. DBH categories are: null, 0-0.9'', 1-4.9'', 5-9.9'', 10-19.9'', 20-29.9'', 30''+. CFA categories are null, 0-10\%, 10-20\%, \dots , 90-100\%. Each row in the table below should be read with a boolean AND across each column of a row.}
\label{ocfw_classification2}
\begin{tabular}{@{}lrrrrr@{}}
\toprule
\textbf{\begin{tabular}[l]{@{}l@{}}Cover \\ Condition\end{tabular}} & \textbf{\begin{tabular}[r]{@{}r@{}}Overstory Tree \\ Diameter 1 \\ (DBH)\end{tabular}} & \textbf{\begin{tabular}[r]{@{}r@{}}Overstory Tree \\ Diameter 2 \\ (DBH)\end{tabular}} & \textbf{\begin{tabular}[r]{@{}r@{}}Total Tree\\ CFA (\%)\end{tabular}} & \textbf{\begin{tabular}[r]{@{}r@{}}Conifer \\ CFA (\%)\end{tabular}} & \textbf{\begin{tabular}[r]{@{}r@{}}Hardwood \\ CFA (\%)\end{tabular}} \\ \midrule
Late Open        & 30''+           & any     & 0-40   & any    & any    \\
Late Open        & 30''+           & any     & null   & 0-40   & null   \\
Late Open        & 30''+           & any     & null   & null   & 0-40   \\
Late Open        & 30''+           & any     & null   & 0-40   & 0-40   \\
Late Open        & any            & 30''+    & 0-40   & any    & any    \\
Late Open        & any            & 30''+    & null   & 0-40   & null   \\
Late Open        & any            & 30''+    & null   & null   & 0-40   \\
Late Open        & any            & 30''+    & null   & 0-40   & 0-40   \\
Late Moderate    & 30''+           & any     & 40-70  & any    & any    \\
Late Moderate    & 30''+           & any     & null   & 40-70  & null   \\
Late Moderate    & 30''+           & any     & null   & null   & 40-70  \\
Late Moderate    & 30''+           & any     & null   & 40-70  & 0-70   \\
Late Moderate    & 30''+           & any     & null   & 0-70   & 40-70  \\
Late Moderate    & any            & 30''+    & 40-70  & any    & any    \\
Late Moderate    & any            & 30''+    & null   & 40-70  & null   \\
Late Moderate    & any            & 30''+    & null   & null   & 40-70  \\
Late Moderate    & any            & 30''+    & null   & 40-70  & 0-70   \\
Late Moderate    & any            & 30''+    & null   & 0-70   & 40-70  \\
Late Closed      & 30''+           & any     & 70-100 & any    & any    \\
Late Closed      & 30''+           & any     & null   & 70-100 & any    \\
Late Closed      & 30''+           & any     & null   & any    & 70-100 \\
Late Closed      & any            & 30''+    & 70-100 & any    & any    \\
Late Closed      & any            & 30''+    & null   & 70-100 & any    \\
Late Closed      & any            & 30''+    & null   & any    & 70-100 \\ \bottomrule
\end{tabular}
\end{table}


\clearpage

\subsection*{References}
\begin{hangparas}{.25in}{1} 
\interlinepenalty=10000
Anderson, Richard. ``Montane Hardwood-Conifer (MHC).'' \emph{A Guide to Wildlife Habitats of California}, edited by Kenneth E. Mayer and William F. Laudenslayer. California Deparment of Fish and Game, 1988. \burl{http://www.dfg.ca.gov/biogeodata/cwhr/pdfs/MHC.pdf}. Accessed 4 December 2012.

``CalVeg Zone 1.'' Vegetation Descriptions. \emph{Vegetation Classification and Mapping}.  11 December 2008. U.S. Forest Service. \burl{http://www.fs.usda.gov/Internet/FSE_DOCUMENTS/fsbdev3_046448.pdf}. Accessed 2 April 2013.
Estes, Becky L. Personal communication, 21 June 2013.

Fitzhugh, E. Lee. ``Ponderosa Pine (PPN).'' \emph{A Guide to Wildlife Habitats of California}, edited by Kenneth E. Mayer and William F. Laudenslayer. California Deparment of Fish and Game, 1988. \burl{http://www.dfg.ca.gov/biogeodata/cwhr/pdfs/PPN.pdf}. Accessed 4 December 2012.

Fryer, Janet L. ``Quercus kelloggii.'' \emph{Fire Effects Information System}, U.S. Department of Agriculture, Forest Service,  Rocky Mountain Research Station, Fire Sciences Laboratory, 2007. \burl{http://www.fs.fed.us/database/feis/plants/tree/quekel/all.html}. Accessed 21 December 2012.

LandFire. ``Biophysical Setting Models.'' Biophysical Setting 0610300: Mediterranean California Lower Montane Black Oak-Conifer Forest and Woodland. 2007a. LANDFIRE Project, U.S. Department of Agriculture, Forest Service; U.S. Department of the Interior. \burl{http://www.landfire.gov/national_veg_models_op2.php}. Accessed 9 November 2012.

LandFire. ``Biophysical Setting Models.'' Biophysical Setting 0610210: Klamath-Siskiyou Lower Montane Serpentine Mixed Conifer Woodland. 2007b. LANDFIRE Project, U.S. Department of Agriculture, Forest Service; U.S. Department of the Interior. \burl{http://www.landfire.gov/national_veg_models_op2.php}. Accessed 9 November 2012.

LandFire. ``Biophysical Setting Models.'' Biophysical Setting 0711700: Klamath-Siskiyou Xeromorphic Serpentine Savanna and Chaparral. 2007c. LANDFIRE Project, U.S. Department of Agriculture, Forest Service; U.S. Department of the Interior. \burl{http://www.landfire.gov/national_veg_models_op2.php}. Accessed 30 November 2012.

O'Geen, Anthony T., Randy A. Dahlgren, and Daniel Sanchez-Mata. ``California Soils and Examples of Ultramafic Vegetation.'' In \emph{Terrestrial Vegetation of California, 3rd Edition}, edited by Michael Barbour, Todd Keeler-Wolf, and Allan A. Schoenherr, 71-106. Berkeley and Los Angeles: University of California Press, 2007. 

Skinner, Carl N. and Chi-Ru Chang. ``Fire Regimes, Past and Present.'' \emph{Sierra Nevada Ecosystem Project: Final report to Congress, vol. II, Assessments and scientific basis for management options}. Davis: University of California, Centers for Water and Wildland Resources, 1996.

Van de Water, Kip M. and Hugh D. Safford. ``A Summary of Fire Frequency Estimates for California Vegetation Before Euro-American Settlement.'' \emph{Fire Ecology} 7.3 (2011): 26-57. doi: 10.4996/fireecology.0703026.

\end{hangparas}


% !TEX root = master.tex
\newpage
\section{Red Fir (RFR)}
\label{rfr-description}

\subsection*{General Information}

\subsubsection{Cover Type Overview}

\textbf{Red Fir (RFR)}
\newline
Crosswalks
\begin{itemize}
	\item EVeg: Regional Dominance Type 1
	\begin{itemize}
		\item Red Fir
	\end{itemize}

	\item Presettlement Fire Regime Type
	\begin{itemize}
		\item Red Fir
	\end{itemize}
\end{itemize}


\noindent Modifiers \\
\medskip
\noindent \textbf{Mesic Modifier } This type is created by intersecting a binary xeric/mesic layer with the existing vegetation layer. RFR cells that intersect with mesic cells are assigned to the mesic modifier.
\begin{itemize}
	\item LandFire BpS Model
	\begin{itemize}
		\item 0610321 Mediterranean California Red Fir Forest – Cascades
	\end{itemize}
\end{itemize}

\noindent \textbf{Xeric Modifier} This type is created by intersecting a binary xeric/mesic layer with the existing vegetation layer. RFR cells that intersect with xeric cells are assigned to the xeric modifier.
\begin{itemize}
	\item LandFire BpS Model
	\begin{itemize}
		\item 0610322 Mediterranean California Red Fir Forest – Southern Sierra
	\end{itemize}
\end{itemize}

\noindent \textbf{Ultramafic Modifier} This type is created by intersecting an ultramafic soils/geology layer with the existing vegetation layer. Where ultramafic cells intersect with RFR they are assigned to the ultramafic modifier.
\begin{itemize}
	\item LandFire BpS Model
	\begin{itemize}
		\item 0710220 Klamath-Siskiyou Upper Montane Serpentine Mixed Conifer Woodland
	\end{itemize}
\end{itemize}

\noindent \textbf{Red Fir with Aspen (RFR\_ASP)} This type is created by overlaying the NRIS TERRA Inventory of Aspen on top of the EVeg layer. Where it intersects with RFR it is assigned to RFR-ASP.

\noindent Reviewed by Marc Meyer, Southern Sierra Province Ecologist, USDA Forest Service

\subsubsection{Vegetation Description}
\textbf{Red Fir} The Red Fir landcover type is characterized by the presence of \emph{Abies magnifica}. Other conifer species such as \emph{Pinus monticola}, \emph{Pinus contorta} ssp. \emph{murrayana}, \emph{Tsuga mertensiana}, \emph{Abies concolor}, and \emph{Pinus jeffreyi} occur at varying densities (LandFire 2007a, LandFire 2007b). Mature \emph{A. magnifica} stands are frequently monotypic, with very few other plant species in any layer. Heavy shade and a thick layer of duff tends to inhibit understory vegetation, especially in dense stands (Barrett 1988). However, there are many open or patchy stands on less productive soils that are not monotypic, but rather codominant with other tree species. These sites may have substantial shrub cover (Meyer pers. comm.).

Stand-replacing disturbances such as lightning-caused fires, windthrows, insect outbreaks, and disease kill groups of trees (Barrett 1988). Stand structure is complex. Most current (fire-suppressed) \emph{A. magnifica} stands that were logged in the 19th century have an even-aged structure. In contrast, current unlogged and fire-suppressed stands have an uneven-aged or irregular age structure. Lastly, presettlement stands with an active fire regime had a relatively flat age-class structure that did not fit a classic even- or uneven-aged distribution (Meyer pers. comm. 2013). That is, frequent small-scale disturbance led to small patches of even-aged trees within the average ``stand,'' and most age classes in a given stand are represented by some of these small patches (Taylor and Halpern 1991). After fire, \emph{A. magnifica} seedlings may establish in canopy gaps, especially if they are small to moderate in size. \emph{P. contorta} ssp. \emph{murrayana}, as well as \emph{P. jeffreyi} and \emph{P. monticola}, may also function as post-fire pioneer species (Meyer pers. comm., Chappell and Agee 1996). On sites where these pioneering types occur under an \emph{A. magnifica} canopy, the \emph{A. magnifica} will dominate over the long-term (Cope 1993).

In openings resulting from tree mortality or logging, and under open stands on poor sites, many species may occur. Large shrubfields can dominate areas after severe fire, although conifers eventually will reclaim these sites. In some cases, particularly on xeric sites with significant shrub cover, reforestation can be effectively delayed for decades. \emph{Ribes}, \emph{Arctostaphylos}, and \emph{Ceanothus} are the most commonly found shrubs (Laacke 1990). Other associated shrubs include \emph{Symphoricarpos rotundifolius, Lonicera conjugialis}, and \emph{Quercus vaccinifolia} (Meyer pers. comm.). Associated herbaceous genera include \emph{Carex, Lupinus, Xerophyllum, Eucephalus, Pedicularis, Gayophytum, Pyrola} and \emph{Monardella} (Cope 1993).



\begin{adjustwidth}{2cm}{}
\noindent \textbf{Mesic Modifier } In addition to \emph{A. magnifica}, mesic regions within the RFR landcover type are associated with the presence of \emph{P. monticola} and \emph{P. contorta} ssp. \emph{murrayana}. \emph{T. mertensiana} may occur on northern aspects. \emph{A. concolor} is uncommon, except at lower elevations (LandFire 2007b).

\medskip
\noindent \textbf{Xeric Modifier}  These sites often include and are occasionally codominated by \emph{A. concolor}, \emph{P. jeffreyi}, and \emph{P. contorta} ssp. \emph{marayanna}, although other conifer species (e.g. \emph{P. lambertiana}) can also be present in lesser amounts at lower elevations. \emph{A. concolor} is more prevalent at lower elevations. \emph{P. jeffreyi} is more common on shallow soils or when disturbance is frequent. Shrubs and herbs generally contribute less than 30\% cover each. If shrub cover is higher, the shrubs are short or prostrate (LandFire 2007a).

\medskip
\noindent \textbf{Ultramafic Modifier} Ultramafic soils, support a number of endemic plant species. Slowly growing and often stunted \emph{P. contorta} ssp. \emph{murrayana} and \emph{P. jeffreyi} occur in combinations or in nearly pure open stands. \emph{A. magnifica} may be less dominant. Hardwoods are usually sparse, but shrubs such as \emph{Arctostaphylos}, \emph{Quercus}, \emph{Rhamnus, Lithocarpus, Rhododendron}, and \emph{Ceanothus} may occur on these sites. (``CalVeg Zone 1'' 2011)

\end{adjustwidth}

\noindent \textbf{Red Fir with Aspen} When \emph{Populus tremuloides} co-occurs with RFR on the west side of the Sierran crest, it is typically found in smaller patches, often less than 2 ha (5 acres) in size. This variant is not subject to the modifiers described above because it is only found on mesic sites with deeper soils. Mature stands in which \emph{P. tremuloides} are still dominant are usually relatively open. Average canopy closures range from 35-95\%. The open nature of the stands results in substantial light penetration to the ground (Meyer pers. comm., Verner 1988).



\subsubsection{Distribution}

\textbf{Red Fir} This cover type occupies the elevational band from about 1900 to 2750 m (6000 to 9000 ft). It is bounded and intergrades with Sierran Mixed Conifer at lower elevations. Geology is quite variable (Barrett 1988).

A xeric-mesic gradient was developed based on four variables: 1) aspect, 2) potential evapotranspiration, 3) topographic wetness index, and 4) soil water storage. The variables were standardized by z-score such that higher values correspond to more mesic environments. Thus, potential evapotranspiration was inverted to maintain this balance. The four variables were combined with equal weights. This final variables was split into xeric vs. mesic, with xeric occupying the negative end of the range up to $\frac{1}{4}$ standard deviation below the mean (zero) and mesic occupying the remaining portion of the spectrum.


\begin{adjustwidth}{2cm}{}
\medskip
\noindent \textbf{Mesic Modifier } These sites generally receive more moisture, either from precipitation, by virtue of being positioned on middle or lower slopes or drainage bottoms, or both. They may be adjacent to meadows or riparian areas. They are found at the highest elevations and north-facing aspects.

\medskip
\noindent \textbf{Xeric Modifier} These sites are typically drier and tend to occupy the lower portion of the RFR zone. They are also more likely to exist on south-facing aspects and steeper slopes.

\medskip
\noindent \textbf{Ultramafic Modifier} Ultramafic soils have been mapped at various spatial densities throughout the elevational range of the Red Fir landcover type. Low to moderate elevations in ultramafic and serpentinized areas often produce soils low in essential minerals such as calcium and magnesium or have excessive accumulations of heavy metals such as nickel and chromium. These sites vary widely in the degree of serpentization and effects on their overlying plant communities (``CalVeg Zone 1''). Note, the terms ``ultramafic rock'' and ``serpentine'' are broad terms used to describe a number of different but related rock types, including serpentinite, peridotite, dunite, pyroxenite, talc and soapstone, among others (O'Geen et al. 2007).

\end{adjustwidth}

\noindent \textbf{Red Fir with Aspen} Sites supporting \emph{P. tremuloides} are associated with added soil moisture, i.e., azonal wet sites. These sites are found throughout the RFR zone, often close to streams and lakes. Other sites include meadow edges, rock reservoirs, springs and seeps. Terrain can be simple to complex. At lower elevations, topographic conditions for this type tends toward positions resulting in relatively colder, wetter conditions within the prevailing climate, e.g., ravines, north slopes, wet depressions, etc. (LandFire 2007c). In general, these sites lie on lower slope positions, and are associated with slopes under 25\% (Potter 1998).

\subsection*{Disturbances}

\subsubsection{Wildfire}

\textbf{Red Fir} Fires in high-elevation \emph{A. magnifica} forests are generally not as intense as those in the Rocky Mountains and are typically less intense than those at lower elevations. Lesser annual fuel accumulation, less severe fire weather conditions, and compact and patchy fuels are all factors (Meyer pers. comm.). Still, fire has an important role in maintaining species diversity within these forests. Fire creates canopy openings by killing mature pioneer species such as \emph{P. contorta} ssp. \emph{murrayana} or \emph{P. jeffreyi} and some mature \emph{A. magnifica} (Cope 1993). 

Estimates of fire rotations for these variants are available from the LandFire project and a few review papers. The LandFire project’s published fire return intervals are based on a series of associated models created using the Vegetation Dynamics Development Tool (VDDT). In VDDT, fires are specified concurrently with the transition that follows them. For example, a replacement fire causes a transition to the early development stage. In the RMLands model, such fires are classified as high mortality. However, in VDDT mixed severity fires may cause a transition to early development, a transition to a more open seral stage, or no transition at all. In this case, we categorize the first example as a high mortality fire, and the second and third examples as a low mortality fire. Based on this approach, we calculated fire rotations and the probability of high mortality fire for each of the RFR seral stages across the three variants, as well as for the RFR\_ASP variant (Tables~\ref{tab:rfrmdesc_fire}--\ref{tab:rfr-aspdesc_fire}). We computed overall target fire rotations based on expert input from Safford and Estes, values from Mallek et al. (2013), and Van de Water and Safford (2011). 





\begin{adjustwidth}{2cm}{}
\medskip
\noindent \textbf{Mesic Modifier } Most fires occur during the late season during tree dormancy, fire complexity is moderate to high, and fire size averages 400 acres. It is very difficult to determine the replacement fire return interval. Replacement fire likely varies with slope position, and landscapes with greater topographic variation are likely to experience more stand replacement fires.

\medskip
\noindent \textbf{Xeric Modifier} Because of slow fuel accumulation rates, it is possible to have long gaps between surface fires in some seral stages. The discontinuous nature of the fuels limit extent of fires, and while fires may burn less often, they may burn at high severities. High intensity crown fires are uncommon.

\medskip
\noindent \textbf{Ultramafic Modifier} This type has a very limited distribution and consequently limited information for fire occurrence history. Low mortality fire is more common than high mortality fire. Most medium and high severity fire may actually occur on middle and upper slope positions.

\end{adjustwidth}

\noindent \textbf{Red Fir with Aspen} Sites supporting \emph{P. tremuloides} are maintained by stand-replacing disturbances that allow regeneration from below-ground suckers. Upland clones are impaired or suppressed by conifer ingrowth and overtopping and intensive grazing that inhibits growth. In a reference condition scenario, a few stands will advance toward conifer dominance. In the current landscape scenario, where fire has been reduced from reference conditions, there are many more conifer-dominated mixed aspen stands (LandFire 2007c, Verner 1988).


\begin{table}[]
\small
\centering
\caption{Fire rotation (years) and proportion of high (versus low) mortality fires for Red Fir – Mesic. Values were derived from VDDT model 0610322 (LandFire 2007b), Mallek et al. (2013), and Safford and Estes (personal communication).}
\label{tab:rfrmdesc_fire}
\begin{tabular}{@{}lcc@{}}
\toprule
\textbf{Condition}         & \multicolumn{1}{l}{\textbf{Fire Rotation}} & \multicolumn{1}{l}{\textbf{\begin{tabular}[c]{@{}l@{}}Proportion \\ High Mortality\end{tabular}}} \\ \midrule
Target                      & 60            & n/a                           \\
Early Development – All     & 58            & 1                             \\
Mid Development – Closed    & 55            & 0.35                          \\
Mid Development – Moderate  & 34            & 0.17                          \\
Mid Development – Open      & 25            & 0.09                          \\
Late Development – Closed   & 52            & 0.41                          \\
Late Development – Moderate & 32            & 0.16                          \\
Late Development – Open     & 23            & 0.05                  \\ \bottomrule
\end{tabular}
\end{table}

\begin{table}[]
\small
\centering
\caption{Fire rotation (years) and proportion of high (versus low) mortality fires for Red Fir – Xeric. Values were derived from VDDT model 0610321 (LandFire 2007a), and Safford and Estes (personal communication). }
\label{tab:rfrxdesc_fire}
\begin{tabular}{@{}lcc@{}}
\toprule
\textbf{Condition}         & \multicolumn{1}{l}{\textbf{Fire Rotation}} & \multicolumn{1}{l}{\textbf{\begin{tabular}[c]{@{}l@{}}Proportion \\ High Mortality\end{tabular}}} \\ \midrule
Target                      & 40            & n/a                           \\
Early Development – All     & 50            & 1                             \\
Mid Development – Closed    & 94            & 0.50                          \\
Mid Development – Moderate  & 65            & 0.25                          \\
Mid Development – Open      & 50            & 0.13                          \\
Late Development – Closed   & 74            & 0.38                          \\
Late Development – Moderate & 55            & 0.19                          \\
Late Development – Open     & 43            & 0.09                  \\ \bottomrule
\end{tabular}
\end{table}

\begin{table}[]
\small
\centering
\caption{Fire rotation (years) and proportion of high (versus low) mortality fires for Red Fir – Ultramafic. Values were derived from VDDT model 0610322 (LandFire 2007b), and Safford and Estes (personal communication). }
\label{tab:rfrudesc_fire}
\begin{tabular}{@{}lcc@{}}
\toprule
\textbf{Condition}         & \multicolumn{1}{l}{\textbf{Fire Rotation}} & \multicolumn{1}{l}{\textbf{\begin{tabular}[c]{@{}l@{}}Proportion \\ High Mortality\end{tabular}}} \\ \midrule
Target                      & 120           & n/a                           \\
Early Development – All     & 117           & 1                             \\
Mid Development – Closed    & 110           & 0.35                          \\
Mid Development – Moderate  & 69            & 0.17                          \\
Mid Development – Open      & 50            & 0.09                          \\
Late Development – Closed   & 104           & 0.41                          \\
Late Development – Moderate & 63            & 0.16                          \\
Late Development – Open     & 46            & 0.05                  \\ \bottomrule
\end{tabular}
\end{table}

\begin{table}[]
\small
\centering
\caption{Fire rotation (years) and proportion of high (versus low) mortality fires for Red Fir – Aspen type. Values were derived from VDDT model 0610610 (LandFire 2007) and Van de Water and Safford (pers. comm. 2013).}
\label{tab:rfr-aspdesc_fire}
\begin{tabular}{@{}lcc@{}}
\toprule
\textbf{Condition}         & \multicolumn{1}{l}{\textbf{Fire Rotation}} & \multicolumn{1}{l}{\textbf{\begin{tabular}[c]{@{}l@{}}Proportion \\ High Mortality\end{tabular}}} \\ \midrule
Target                           & 60            & n/a                           \\
Early Development – Aspen        & 58            & 0.03                          \\
Mid Development – Aspen          & 55            & 0.41                          \\
Mid Development – Aspen-Conifer  & 34            & 0.15                          \\
Late Development – Conifer-Aspen & 32            & 0.13                          \\
Late Development – Closed        & 52            & 0.26                  \\ \bottomrule
\end{tabular}
\end{table}

\subsubsection{Other Disturbance}
Other disturbances are not currently modeled, but may, depending on the seral stage affected and mortality levels, reset patches to early development, maintain existing seral stages, or shift/accelerate succession to a more open seral stage. 

\subsection*{Vegetation Seral Stages}
We recognize seven separate seral stages for RFR: Early Development (ED), Mid Development – Open Canopy Cover (MDO), Mid Development – Moderate Canopy Cover, Mid Development – Closed Canopy Cover (MDC), Late Development – Open Canopy Cover (LDO), Late Development – Moderate Canopy Cover (LDM), and Late Development – Closed Canopy Cover (LDC) (Figure~\ref{transmodel_rfr}). The RFR-ASP variant is also assigned to five seral stages: Early Development – Aspen (ED-A), Mid Development – Aspen (MD-A), Mid Development – Aspen with Conifer (MD-AC), Late Development Closed (LDC), and Late Development – Conifer with Aspen (LD-CA) (Figure~\ref{transmodel_rfr-asp}). 

Our seral stages are an alternative to ``successional'' classes that imply a linear progression of states and tend not to incorporate disturbance. The seral stages identified here are derived from a combination of successional processes and anthropogenic and natural disturbance, and are intended to represent a composition and structural condition that can be arrived at from multiple other conditions described for that landcover type. Thus our seral stages incorporate age, size, canopy cover, and vegetation composition. In general, the delineation of stages has originated from the LandFire biophysical setting model descriptive of a given landcover type; however, seral stages are not necessarily identical to the classes identified in those models.

\begin{figure}[htbp]
\centering
\includegraphics[width=0.8\textwidth]{/Users/mmallek/Documents/Thesis/statetransmodel/StateTransitionModel/7class.png}
\caption{State and Transition Model for Red Fir Forest (not inclusive of the aspen variant). Each dark grey box represents one of the seven seral stages for this landcover type. Each column of boxes represents a stage of development: early, middle, and late. Each row of boxes represents a different level of canopy cover: closed (70-100\%), moderate (40-70\%), and open (0-40\%). Transitions between states/seral stages may occur as a result of high mortality fire, low mortality fire, or succession. Specific pathways for each are denoted by the appropriate color line and arrow: red lines relate to high mortality fire, orange lines relate to low mortality fire, and green lines relate to natural succession.} 
\label{transmodel_rfr}
\end{figure}

\begin{figure}[htbp]
\centering
\includegraphics[width=0.8\textwidth]{/Users/mmallek/Documents/Thesis/statetransmodel/StateTransitionModel/5class-asp.png}
\caption{State and Transition Model for Red Fir Forest - Aspen variant. Each dark grey box represents one of the seven seral stages for this landcover type. Each column of boxes represents a stage of development: early, middle, and late. Transitions between states/seral stages may occur as a result of high mortality fire, low mortality fire, or succession. Specific pathways for each are denoted by the appropriate color line and arrow: red lines relate to high mortality fire, orange lines relate to low mortality fire, and green lines relate to natural succession.} 
\label{transmodel_rfr-asp}
\end{figure}

\subsection*{Red Fir}

\paragraph{Description}
\paragraph{Early Development (ED)} This seral stage is characterized by the recruitment of a new cohort of early successional, shade-intolerant tree species into an open area created by a stand-replacing disturbance. Conifer associates regenerate from seed. Occasionally, large brush fields may develop after hot wildfires and are dominated by \emph{Ceanothus, Arctostaphylos, Chrysolepsis}, or other shrub species for many years (Barrett 1988). On mesic sites, \emph{P. monticola} and \emph{P. contorta} ssp. \emph{murrayana} regenerate from seed. \emph{A. magnifica} comes in over time. Shrub cover is an important component; herb cover varies (LandFire 2007b). On xeric sites, there is regeneration of \emph{A. magnifica} and \emph{A. concolor}, perhaps \emph{P. jeffreyi} or \emph{P. lambertiana} from seed. Shrub and herb cover varies. (LandFire 2007a). Ultramafic sites will have similar species composition, especially at edges, but \emph{P. jeffreyi}, are relatively more common. Shrubs and herbs are sparse (O'Geen et al. 2007).

\paragraph{Succession Transition}

\begin{adjustwidth}{2cm}{}

\noindent \textbf{Mesic Modifier } In the absence of disturbance, patches in this seral stage will begin transitioning to MDC at age 30 at a rate of 0.6 per timestep. At 70 years, all stands will succeed to MDC. On average, patches remain in ED for 38 years.

\medskip
\noindent \textbf{Xeric Modifier}  Transition to mid development seral stages may be somewhat delayed. In the absence of disturbance, patches in this seral stage will begin transitioning to MDO at 50 years and may be delayed in the ED seral stage for as long as 150 years. A patch in this seral stage succeeds at a rate of 0.3 per timestep. On average, patches remain in ED for 67 years.

\medskip
\noindent \textbf{Ultramafic Modifier}  Transition to mid development seral stages may be substantially delayed. Thus, in the absence of disturbance, patches in this seral stage will begin transitioning to MDO after 80 years and may be delayed in the ED seral stage for as long as 150 years. A patch in this seral stage succeeds at a rate of 0.2 per timestep. On average, patches remain in ED for 105 years.

\end{adjustwidth}



\paragraph{Wildfire Transition} High mortality wildfire (100\% of fires in this seral stage) recycles the patch through the Early Development seral stage, regardless of soil type. Low mortality wildfire is not modeled for this seral stage.

\noindent\hrulefill


\paragraph{Mid Development – Open Canopy Cover (MDO)} 

\paragraph{Description} The pole/medium tree seral stage produces dense stands of young \emph{A. magnifica} that grow slowly with little mortality for many years (Barrett 1988). Cover of grasses, forms, and shrubs is on the decline as conifer canopy cover ranges from 10-40\%. \emph{A. magnifica} either is or is transitioning to become the dominant tree species. Canopy cover is less than 40\% (LandFire 2007a, LandFire 2007b).

On mesic sites, \emph{P. monticola} and \emph{P. contorta} ssp. \emph{murrayana} are present in varying amounts. Grasses, forbs, and shrubs are declining, although chaparral type shrubs, such as \emph{Arctostaphylos} or \emph{Chrysolepsis} can contribute to a dense understory. On xeric sites, \emph{A. concolor} and \emph{P. jeffreyi} are present in varying amounts, and shrub cover varies (LandFire 2007a, LandFire 2007b). Ultramafic sites will have similar species composition, especially at edges, but \emph{P. jeffreyi} is relatively more common (O'Geen et al. 2007).


\paragraph{Succession Transition}
\begin{adjustwidth}{2cm}{}

\noindent \textbf{Mesic Modifier } In the absence of low mortality disturbance, patches in the MDO seral stage will begin transitioning to MDM after 10 years at a rate of 0.22 per timestep. Succession to LDO takes place at 80 years since entering a middle development seral stage. 

\medskip
\noindent \textbf{Xeric Modifier} In the absence of low mortality disturbance, patches in the MDO seral stage will begin transitioning to MDM at 25 years at a rate of 0.2 per timestep. Succession to LDO takes place variably beginning at 80 years since transition to middle development at a rate of 0.6 per timestep. All patches succeed to a late seral stage by 100 years. On average, patches remain in MDM for 88 years.

\medskip
\noindent \textbf{Ultramafic Modifier} In the absence of low mortality disturbance, patches in the MDO seral stage will begin transitioning to MDM after 40 years at a rate of 0.1 per timestep. Succession to LDO takes place variably beginning at 120 years since transition to middle development at a rate of 0.3 per timestep, and all patches succeed by 180 years. On average, patches remain in ED for 117 years.

\end{adjustwidth}

\paragraph{Wildfire Transition}
\begin{adjustwidth}{2cm}{}
\noindent \textbf{Mesic Modifier } High mortality wildfire (9\% of fires in this seral stage) returns the patch to Early Development. Low mortality fire (91\%) maintains the MDO seral stage and allows for succession to LDO.

\medskip
\noindent \textbf{Xeric Modifier}  High mortality wildfire (13\% of fires in this seral stage) returns the patch to Early Development. Low mortality fire (87\%) maintains the MDO seral stage and allows for succession to LDO. 

\medskip
\noindent \textbf{Ultramafic Modifier}  High mortality wildfire (9\% of fires in this seral stage) returns the patch to Early Development. Low mortality fire (91\%) maintains the MDO seral stage and allows for succession to LDO.

\end{adjustwidth}

\noindent\hrulefill

\paragraph{Mid Development – Moderate Canopy Cover (MDM)}

\paragraph{Description} The pole/medium tree seral stage produces stands of young \emph{A. magnifica} with moderate canopy cover that grow slowly with little mortality for many years (Barrett 1988). Cover of grasses, forms, and shrubs is on the decline as conifer canopy cover exceeds 40\%. \emph{A. magnifica} either is or is transitioning to become the dominant tree species. On mesic sites, \emph{P. monticola} and \emph{P. contorta} ssp. \emph{murrayana} are present in varying amounts, while on xeric sites \emph{P. jeffreyi} and \emph{A. concolor} are associates (LandFire 2007a, LandFire 2007b). \emph{P. jeffreyi} is the most likely associate on ultramafic sites (O'Geen et al. 2007).

\paragraph{Succession Transition}
\begin{adjustwidth}{2cm}{}
\noindent \textbf{Mesic Modifier } In the absence of low mortality disturbance, patches in the MDM seral stage will begin transitioning to MDC after 10 years at a rate of 0.22 per timestep. Succession to LDM takes place at 80 years since entering a middle development seral stage. 

\medskip
\noindent \textbf{Xeric Modifier}  In the absence of low mortality disturbance, patches in the MDM seral stage will begin transitioning to MDC after 25 years at a rate of 0.2 per timestep. Succession to LDM begins at 80 years since entering a middle development stage at a rate of 0.65 per timestep. At 100 years after entering a middle development stage, all stands transition to LDM. 

\medskip
\noindent \textbf{Ultramafic Modifier} Transition to late seral seral stages may be substantially delayed. Thus, in the absence of low mortality disturbance, patches in the MDM seral stage will begin transitioning to MDC after 40 years at a rate of 0.1 per timestep. Succession to LDM begins at 100 years since entering a middle development stage at a rate of 0.3 per timestep. At 165 years after entering a middle development stage, all stands transition to LDM. 

\end{adjustwidth}

\paragraph{Wildfire Transition}
\begin{adjustwidth}{2cm}{}
\noindent \textbf{Mesic Modifier } High mortality wildfire (17\% of fires in this seral stage) returns the patch to ED. Low mortality wildfire (83\%) opens the stand up to MDO 13\% of the time; otherwise, the patch remains in MDC. 

\medskip
\noindent \textbf{Xeric Modifier}  High mortality wildfire (25\% of fires in this seral stage) returns the patch to ED. Low mortality wildfire (75\%) opens the stand up to MDO 13\% of the time; otherwise, the patch remains in MDC.

\medskip
\noindent \textbf{Ultramafic Modifier} High mortality wildfire (17\% of fires in this seral stage) returns the patch to ED. Low mortality wildfire (83\%) opens the stand up to MDO 13\% of the time; otherwise, the patch remains in MDC.

\end{adjustwidth}

\noindent\hrulefill

\paragraph{Mid Development – Closed Canopy Cover (MDC)}

\paragraph{Description} The pole/medium tree seral stage produces dense stands of young \emph{A. magnifica} that grow slowly with little mortality for many years (Barrett 1988). Cover of grasses, forms, and shrubs is on the decline as conifer canopy cover exceeds 40\%. \emph{A. magnifica} either is or is transitioning to become the dominant tree species. On mesic sites, \emph{P. monticola} and \emph{P. contorta} ssp. \emph{murrayana} are present in varying amounts, while on xeric sites \emph{P. jeffreyi} and \emph{A. concolor} are associates (LandFire 2007a, LandFire 2007b). \emph{P. jeffreyi} is the most likely associate on ultramafic sites (O'Geen et al. 2007).

\paragraph{Succession Transition}
\begin{adjustwidth}{2cm}{}
\noindent \textbf{Mesic Modifier } After 80 years in the mid development stage and in the absence of stand-replacing fire, all patches transition to LDC.

\medskip
\noindent \textbf{Xeric Modifier}  Transition to late seral seral stages may be delayed. Thus, in the absence of disturbance, patches in this seral stage will begin transitioning to LDC at 80 years in mid development at a rate of 0.7 per timestep and may be delayed in the MDC seral stage for up to 100 years.

\medskip
\noindent \textbf{Ultramafic Modifier} Transition to late seral seral stages may be substantially delayed. Thus, in the absence of disturbance, patches in this seral stage will begin transitioning to LDC at 80 years in the mid development stage at a rate of 0.3 per time step and may be delayed in the MDC seral stage for up to 150 years.

\end{adjustwidth}

\paragraph{Wildfire Transition}
\begin{adjustwidth}{2cm}{}
\noindent \textbf{Mesic Modifier } High mortality wildfire (35\% of fires in this seral stage) returns the patch to ED. Low mortality wildfire (65\%) opens the stand up to MDM 17\% of the time; otherwise, the patch remains in MDC. 

\medskip
\noindent \textbf{Xeric Modifier} High mortality wildfire (50\% of fires in this seral stage) returns the patch to ED. Low mortality wildfire (50\%) opens the stand up to MDM 17\% of the time; otherwise, the patch remains in MDC.

\medskip
\noindent \textbf{Ultramafic Modifier} High mortality wildfire (35\% of fires in this seral stage) returns the patch to ED. Low mortality wildfire (65\%) opens the stand up to MDM 17\% of the time; otherwise, the patch remains in MDC.

\end{adjustwidth}

\noindent\hrulefill


\paragraph{Late Development – Open Canopy Cover (LDO)}

\paragraph{Description} In the large tree seral stage, subdominant trees die and add to a growing layer of duff and downed woody material, and dominant trees continue to grow for several hundred years. \emph{A. magnifica} is the most common tree species. The understory of mature stands may be limited to less than 5\% cover (e.g. \emph{Chimaphila menziesii, Pyrola picta}). This seral stage develops when low mortality disturbance is fairly frequent; it persists as long as low mortality fires continue to occur periodically. \emph{Ceanothus} and \emph{Arctostaphylos} populate disturbance-generated gaps. Canopy cover is less than 40\% (LandFire 2007a, LandFire 2007b).

On mesic sites, \emph{P. monticola} and \emph{P. contorta} ssp. \emph{murrayana} may comprise up to 20\% of tree cover each. \emph{P. contorta} ssp. \emph{murrayana} acts as the pioneering conifer. On xeric sites, \emph{A. concolor} and \emph{P. jeffreyi} are the common associates and pioneer conifer species after disturbance (Barrett 1988, LandFire 2007a, LandFire 2007b). Ultramafic sites will have similar species composition, especially at edges, but \emph{P. jeffreyi} is relatively more common (O'Geen et al. 2007).


\paragraph{Succession Transition}
\begin{adjustwidth}{2cm}{}
\noindent \textbf{Mesic Modifier } In the presence of low mortality disturbance, patches in this seral stage can self-perpetuate, but after 10 years with no fire, patches in this seral stage will begin transitioning to LDM at a rate of 0.2 per timestep.

\medskip
\noindent \textbf{Xeric Modifier}  In the presence of low mortality disturbance, patches in this seral stage can self-perpetuate, but after 25 years with no fire, patches in this seral stage will begin transitioning to LDM at a rate of 0.2 per timestep.

\medskip
\noindent \textbf{Ultramafic Modifier} Patches occurring on ultramafic soils may succeed to LDM after 35 years with no fire, but the rate is just 0.2 per timestep.

\end{adjustwidth}

\paragraph{Wildfire Transition}
\begin{adjustwidth}{2cm}{}
\noindent \textbf{Mesic Modifier } High mortality wildfire (11\% of fires in this seral stage) returns the patch to early development. Low mortality wildfire (89\%) maintains LDO. 

\medskip
\noindent \textbf{Xeric Modifier} High mortality wildfire (3\% of fires in this seral stage) returns the patch to early development. Low mortality wildfire (97\%) maintains LDO. 

\medskip
\noindent \textbf{Ultramafic Modifier} High mortality wildfire (11\% of fires in this seral stage) returns the patch to early development. Low mortality wildfire (89\%) maintains LDO.

\end{adjustwidth}

\noindent\hrulefill

\paragraph{Late Development – Moderate Canopy Cover (LDM)}

\paragraph{Description} In the large tree seral stage, subdominant trees die and add to a growing layer of duff and downed woody material, and dominant trees continue to grow for several hundred years to heights of 40 m (130 ft). Overall conifer cover ranges from 40\% to 70\%. \emph{A. magnifica} is the most common tree species. The understory of mature stands is limited to less than 5 percent cover of shade tolerant forbs (e.g., \emph{Chimaphila menziesii, Pyrola picta}). 

On mesic sites, \emph{P. monticola} is the primary associate, with some \emph{P. contorta} ssp. \emph{murrayana} occurring in the understory. On xeric sites, \emph{A. magnifica} occurs in pure to mixed stands, and \emph{A. concolor} and \emph{P. jeffreyi} are the primary associates (Barrett 1988, LandFire 2007a, LandFire 2007b). Ultramafic sites will have similar species composition, especially at edges, but \emph{P. jeffreyi} is relatively more common. (O'Geen et al. 2007).


\paragraph{Succession Transition} In the absence of disturbance, patches in this seral stage will maintain, regardless of soil characteristics.

\paragraph{Wildfire Transition}
\begin{adjustwidth}{2cm}{}
\noindent \textbf{Mesic Modifier } High mortality wildfire (16\% of fires in this seral stage) will return the patch to Early Development. Low mortality wildfire (84\%) opens the stand up to LDO 15\% of the time; otherwise, the patch remains in LDM. 

\medskip
\noindent \textbf{Xeric Modifier} High mortality wildfire (19\% of fires in this seral stage) will return the patch to Early Development. Low mortality wildfire (81\%) opens the stand up to LDO 15\% of the time; otherwise, the patch remains in LDM. 

\medskip
\noindent \textbf{Ultramafic Modifier} High mortality wildfire (16\% of fires in this seral stage) will return the patch to Early Development. Low mortality wildfire (84\%) opens the stand up to LDO 15\% of the time; otherwise, the patch remains in LDM.

\end{adjustwidth}

\noindent\hrulefill

\paragraph{Late Development – Closed Canopy Cover (LDC)}

\paragraph{Description} In the large tree seral stage, subdominant trees die and add to a growing layer of duff and downed woody material, and dominant trees continue to grow for several hundred years to heights of 40 m (130 ft). Overall conifer cover exceeds 70\%. \emph{A. magnifica} is the most common tree species. The understory of mature stands is limited to less than 5 percent cover of shade tolerant forbs (e.g., \emph{Chimaphila menziesii, Pyrola picta}). 

On mesic sites, \emph{P. monticola} is the primary associate, with some \emph{P. contorta} ssp. \emph{murrayana} occuring in the understory. On xeric sites, \emph{A. magnifica} occurs in pure to mixed stands, and \emph{A. concolor} and \emph{P. jeffreyi} are the primary associates (Barrett 1988, LandFire 2007a, LandFire 2007b). Ultramafic sites will have similar species composition, especially at edges, but \emph{P. jeffreyi} is relatively more common (O'Geen et al. 2007).


\paragraph{Succession Transition} In the absence of disturbance, patches in this seral stage will maintain, regardless of soil characteristics.



\paragraph{Wildfire Transition}

\begin{adjustwidth}{2cm}{}
\noindent \textbf{Mesic Modifier } High mortality wildfire (41\% of fires in this seral stage) will return the patch to Early Development. Low mortality wildfire (59\%) opens the stand up to LDM 10\% of the time; otherwise, the patch remains in LDC. 

\medskip
\noindent \textbf{Xeric Modifier} High mortality wildfire (38\% of fires in this seral stage) will return the patch to Early Development. Low mortality wildfire (62\%) opens the stand up to LDM 10\% of the time; otherwise, the patch remains in LDC. 

\medskip
\noindent \textbf{Ultramafic Modifier} High mortality wildfire (41\% of fires in this seral stage) will return the patch to Early Development. Low mortality wildfire (59\%) opens the stand up to LDM 10\% of the time; otherwise, the patch remains in LDC.

\end{adjustwidth}

\noindent\hrulefill
\noindent\hrulefill

\subsubsection{Aspen Variant}

\paragraph{Early Development – Aspen (ED–A)}

\paragraph{Description} Grasses, forbs, low shrubs, and sparse to moderate cover of tree seedlings/saplings (primarily \emph{P. tremuloides}) with an open canopy. This seral stage is characterized by the recruitment of a new cohort of early successional, shade-intolerant tree species into an open area created by a stand-replacing disturbance. 

Following disturbance, succession proceeds rapidly from an herbaceous layer to shrubs and trees, which invade together (Barrett 1988). \emph{P. tremuloides} suckers over 6ft tall develop within about 10 years (LandFire 2007c). 



\paragraph{Succession Transition} Unless it burns, a patch in ED–A persists for 10 years, at which point it transitions to MD-A.

\paragraph{Wildfire Transition} High mortality wildfire (100\% of fires in this seral stage) recycles the patch through the ED–A seral stage. Low mortality wildfire is not modeled for this seral stage.

\noindent\hrulefill


\paragraph{Mid Development – Aspen (MD–A)}

\paragraph{Description} \emph{P. tremuloides} trees 5-16'' DBH. Canopy cover is highly variable, and can range from 40-100\%. These patches range in age from 10 to 110 years. Some understory conifers, including \emph{P. contorta} ssp. \emph{murrayana}, \emph{A. concolor}, and \emph{A. magnifica} are encroaching, but \emph{P. tremuloides} is still the dominant component of the stand (LandFire 2007c).

\paragraph{Succession Transition} MD-A persists for at least 50 years in the absence of fire, after which patches in this seral stage begin transitioning to MD-AC at a rate of 0.6 per timestep. At 100 years since entering MD-A, any remaining patches transition to MD-AC.

\paragraph{Wildfire Transition} High mortality wildfire (35\% of fires in this seral stage) recycles the patch through the ED-A seral stage. No transition occurs as a result of low mortality fire.

\noindent\hrulefill

\paragraph{Mid Development – Aspen with Conifer (MD–AC)}

\paragraph{Description} These stands have been protected from fire since the last stand-replacing disturbance. \emph{P. tremuloides} trees are predominantly 16'' DBH and greater. Conifers are present and overtopping the \emph{P. tremuloides}. \emph{A. concolor} is a typical conifer that is successional to \emph{P. tremuloides}, and is depicted here, but other conifers including \emph{P. ponderosa} and \emph{P. lambertiana} are also possible. Conifers are pole to medium-sized, and conifer cover is at least 40\% (LandFire 2007c).

\paragraph{Succession Transition} MD-AC persists for 100 years in the absence of fire, after which patches in this seral stage transition to LDC. 

\paragraph{Wildfire Transition} High mortality wildfire (17\% of fires in this seral stage) returns the patch to ED-A. Low mortality wildfire (83\%) maintains the patch in MD–AC.

\noindent\hrulefill

\paragraph{Late Development – Closed (LDC)}

\paragraph{Description} Some \emph{P. tremuloides} continue to be present in the understory, but large conifers are now the dominant tree species, having overtopped the \emph{P. tremuloides}. Smaller conifers are present in the midstory as well. Conifer species likely present include \emph{A. concolor, A. magnifica}, and \emph{P. contorta} ssp. \emph{murrayana}. (LandFire 2007a). This seral stage is analogous to the LDC seral stage for the RFR variant.

\paragraph{Succession Transition} In the absence of disturbance, patches in this seral stage will maintain, regardless of soil characteristics.

\paragraph{Wildfire Transition} High mortality wildfire (41\% of fires in this seral stage) will return the patch to ED–A. Low mortality wildfire (59\%) usually has little effect, although 10\% of the time it opens the stand up to LD-CA.

\noindent\hrulefill


\paragraph{Late Development – Conifer with Aspen (LD–CA)}

\paragraph{Description} If stands are sufficiently protected from fire such that conifer species overtop \emph{P. tremuloides} and become large, they may be able to withstand some fire that more sensitive \emph{P. tremuloides} cannot. When this occurs, it creates a patch characterized by late development conifers, such as \emph{A. concolor} or \emph{A. magnifica}, and early seral \emph{P. tremuloides}. 

\paragraph{Succession Transition} LD-CA persists for 70 years in the absence of any fire, after which patches transition to LDC. 

\paragraph{Wildfire Transition} High mortality wildfire (16\% of fires in this seral stage) returns the patch to ED-A. Low mortality wildfire (84\%) maintains the stand in LD-CA.

\noindent\hrulefill


\newpage

\subsection*{Seral Stage Classification}
\begin{table}[hbp]
\small
\centering
\caption{Classification of cover seral stage for RFR. Diameter at Breast Height (DBH) and Cover From Above (CFA) values taken from EVeg polygons. DBH categories are: null, 0-0.9'', 1-4.9'', 5-9.9'', 10-19.9'', 20-29.9'', 30''+. CFA categories are null, 0-10\%, 10-20\%, \dots , 90-100\%. Each row in the table below should be read with a boolean AND across each column of a row.}
\label{rfr_classification}
\begin{tabular}{@{}lrrrrr@{}}
\toprule
\textbf{\begin{tabular}[l]{@{}l@{}}Cover \\ Condition\end{tabular}} & \textbf{\begin{tabular}[r]{@{}r@{}}Overstory Tree \\ Diameter 1 \\ (DBH)\end{tabular}} & \textbf{\begin{tabular}[r]{@{}r@{}}Overstory Tree \\ Diameter 2 \\ (DBH)\end{tabular}} & \textbf{\begin{tabular}[r]{@{}r@{}}Total Tree\\ CFA (\%)\end{tabular}} & \textbf{\begin{tabular}[r]{@{}r@{}}Conifer \\ CFA (\%)\end{tabular}} & \textbf{\begin{tabular}[r]{@{}r@{}}Hardwood \\ CFA (\%)\end{tabular}} \\ \midrule
Early All        & null           & any & any    & any    & any  \\
Early All        & 0-4.9''         & any & any    & any    & any  \\
Mid Open         & 5-19.9''        & any & null   & null   & null \\
Mid Open         & 5-19.9''        & any & 0-40   & any    & any  \\
Mid Open         & 5-19.9''        & any & null   & 0-40   & null \\
Mid Moderate     & 5-19.9''        & any & 40-70  & any    & any  \\
Mid Moderate     & 5-19.9''        & any & null   & 40-70  & null \\
Mid Closed       & 5-19.9''        & any & 70-100 & any    & any  \\
Mid Closed       & 5-19.9''        & any & null   & 70-100 & any  \\
Late Open        & 20''+           & any & null   & null   & null \\
Late Open        & 20''+           & any & 0-40   & any    & any  \\
Late Open        & 20''+           & any & null   & 0-40   & null \\
Late Moderate    & 20''+           & any & 40-70  & any    & any  \\
Late Moderate    & 20''+           & any & null   & 40-70  & null \\
Late Closed      & 20''+           & any & 70-100 & any    & any  \\
Late Closed      & 20''+           & any & null   & 70-100 & any  \\ \bottomrule
\end{tabular}
\end{table}

RFR-ASP seral stages were assigned manually using NAIP 2010 Color IR imagery to assess seral stage.



\clearpage

\subsection*{References}

\begin{hangparas}{.25in}{1} 
\interlinepenalty=10000
Barrett, Reginald H. ``Red Fir (RFR).'' \emph{A Guide to Wildlife Habitats of California}, edited by Kenneth E. Mayer and William F. Laudenslayer. California Department of Fish and Game, 1988. \burl{http://www.dfg.ca.gov/biogeodata/cwhr/pdfs/RFR.pdf}. Accessed 4 December 2012.

``CalVeg Zone 1.'' Vegetation Descriptions. Vegetation Classification and Mapping.  11 December 2008. U.S. Forest Service. \burl{http://www.fs.usda.gov/Internet/FSE_DOCUMENTS/fsbdev3_046448.pdf}. Accessed 2 April 2013.

Chappell, Christopher B. and James K. Agee. ``Fire Severity and Tree Seedling Establishment in Abies Magnifica Forests, Southern Cascades, Oregon.'' \emph{Ecological Applications} 6.2 (1996): 628-640.

Cope, Amy B. ``Abies magnifica.'' \emph{Fire Effects Information System}, U.S. Department of Agriculture, Forest Service,  Rocky Mountain Research Station, Fire Sciences Laboratory, 1993. \burl{http://www.fs.fed.us/database/feis/} [Accessed 4 December 2012].

Estes, Becky. Central Sierra Province Ecologist, USDA Forest Service. 2013.

Laacke, Robert J. ``California Red Fir.'' Russell M. Burns and Barbara H. Honkala, tech. coords. \emph{Silvics of North America, vol 1. Conifers}; Glossary. Agriculture handbook no. 654. Washington, D.C.: U.S. Dept. of Agriculture, Forest Service, 1990. 

LandFire. ``Biophysical Setting Models.'' Biophysical Setting 0610321: Mediterranean California Red Fir Forest - Cascades. 2007a. LANDFIRE Project, U.S. Department of Agriculture, Forest Service; U.S. Department of the Interior. \burl{http://www.landfire.gov/national_veg_models_op2.php}. Accessed 9 November 2012.

LandFire. ``Biophysical Setting Models.'' Biophysical Setting 0610322: Mediterranean California Red Fir Forest – Southern Sierra. 2007b. LANDFIRE Project, U.S. Department of Agriculture, Forest Service; U.S. Department of the Interior. \burl{http://www.landfire.gov/national_veg_models_op2.php}. Accessed 9 November 2012.

LandFire. ``Biophysical Setting Models.'' Biophysical Setting 0610610: Inter-Mountain Basins Aspen-Mixed Conifer Forest and Woodland. 2007c. LANDFIRE Project, U.S. Department of Agriculture, Forest Service; U.S. Department of the Interior. \burl{http://www.landfire.gov/national_veg_models_op2.php}. Accessed 7 January 2013.

LandFire. ``Biophysical Setting Models.'' Biophysical Setting 0710320: Mediterranean California Red Fir Forest. 2007d. LANDFIRE Project, U.S. Department of Agriculture, Forest Service; U.S. Department of the Interior. \burl{http://www.landfire.gov/national_veg_models_op2.php}. Accessed 30 November 2012.

LandFire. ``Biophysical Setting Models.'' Biophysical Setting 0710220: Klamath-Siskiyou Upper Montane Serpentine Mixed Conifer Woodland. 2007e. LANDFIRE Project, U.S. Department of Agriculture, Forest Service; U.S. Department of the Interior. \burl{http://www.landfire.gov/national_veg_models_op2.php}. Accessed 30 November 2012.

Meyer, Marc D. Personal communication, 19 June 2013.

Meyer, Marc D. ``Natural Range of Variation of Red Fir Forests in the Bioregional Assessment Area'' (unpublished paper, Ecology Group, Pacific Southwest Research Station, 2013).

O'Geen, Anthony T., Randy A. Dahlgren, and Daniel Sanchez-Mata. ``California Soils and Examples of Ultramafic Vegetation'' In \emph{Terrestrial Vegetation of California, 3rd Edition}, edited by Michael Barbour, Todd Keeler-Wolf, and Allan A. Schoenherr, 71-106. Berkeley and Los Angeles: University of California Press, 2007. 

Potter, Donald A. ``Forested Communities of the Upper Montane in the Central and Southern Sierra Nevada.'' Gen. Tech. Rep. PSW-GTR-169. Albany, CA: Pacific Southwest Research Station, Forest Service, U.S. Department of Agriculture, 1998.

Safford, Hugh S. Personal communication, 5 May 2013.

Skinner, Carl N. and Chi-Ru Chang. ``Fire Regimes, Past and Present.'' \emph{Sierra Nevada Ecosystem Project: Final report to Congress, vol. II, Assessments and scientific basis for management options}. Davis: University of California, Centers for Water and Wildland Resources, 1996.

Taylor, Alan H. and Charles B. Halpern. ``The structure and dynamics of Abies magnifica forests in the southern Cascade Range, USA.'' \emph{Journal of Vegetation Science} 2 (1991): 189-200.

Van de Water, Kip M. and Hugh D. Safford. ``A Summary of Fire Frequency Estimates for California Vegetation Before Euro-American Settlement.'' \emph{Fire Ecology} 7.3 (2011): 26-57. doi: 10.4996/fireecology.0703026.

Verner, Jared. ``Aspen (ASP).'' \emph{A Guide to Wildlife Habitats of California}, edited by Kenneth E. Mayer and William F. Laudenslayer. California Deparment of Fish and Game, 1988. \burl{http://www.dfg.ca.gov/biogeodata/cwhr/pdfs/ASP.pdf}. Accessed 4 December 2012.

\end{hangparas}


% !TEX root = master.tex
\newpage
\section{Sierran Mixed Conifer (SMC)}

\subsection*{General Information}

\subsubsection{Cover Type Overview}

\textbf{Sierran Mixed Conifer (SMC)}
\newline
Crosswalks
\begin{itemize}
	\item EVeg: Regional Dominance Type 1
	\begin{itemize}
		\item Mixed Conifer - Fir
		\item Mixed Conifer - Pine
	\end{itemize}

	\item Presettlement Fire Regime Type
	\begin{itemize}
		\item Red Fir
	\end{itemize}
\end{itemize}


\noindent Modifiers \\
\medskip
\noindent \textbf{Mesic Modifier } This type is created by intersecting a binary xeric/mesic layer with the existing vegetation layer. SMC cells that intersect with mesic cells are assigned to the mesic modifier.
\begin{itemize}
	\item LandFire BpS Model
	\begin{itemize}
		\item 0610280 Mediterranean California Mesic Mixed Conifer Forest and Woodland
	\end{itemize}
		\item Presettlement Fire Regime Type: 
	\begin{itemize}
		\item Moist Mixed Conifer
	\end{itemize}
\end{itemize}

\noindent \textbf{Xeric Modifier} This type is created by intersecting a binary xeric/mesic layer with the existing vegetation layer. SMC cells that intersect with xeric cells are assigned to the xeric modifier.
\begin{itemize}
	\item LandFire BpS Model
	\begin{itemize}
		\item 0610270 Mediterranean California Dry-Mesic Mixed Conifer Forest and Woodland
	\end{itemize}
		\item Presettlement Fire Regime Type: 
	\begin{itemize}
		\item Dry Mixed Conifer
	\end{itemize}
\end{itemize}

\noindent \textbf{Ultramafic Modifier} This type is created by intersecting an ultramafic soils/geology layer with the existing vegetation layer. Where ultramafic cells intersect with SMC they are assigned to the ultramafic modifier.
\begin{itemize}
	\item LandFire BpS Model
	\begin{itemize}
		\item 0710220 Klamath-Siskiyou Upper Montane Serpentine Mixed Conifer Woodland
	\end{itemize}
	\item Presettlement Fire Regime Type: 
	\begin{itemize}
		\item N/A
	\end{itemize}
\end{itemize}

\noindent \textbf{Sierran Mixed Conifer with Aspen (SMC\_ASP)} This type is created by overlaying the NRIS TERRA Inventory of Aspen on top of the EVeg layer. Where it intersects with SMC it is assigned to SMC-ASP. \\



\noindent Reviewed by Hugh Safford, Regional Ecologist, USDA Forest Service; Becky Estes, Central Sierra Province Ecologist, USDA Forest Service


\subsubsection{Vegetation Description}

\textbf{Sierran Mixed Conifer (SMC)} The Sierran Mixed Conifer landcover type is typically composed of three or more conifers, sometimes mixed with hardwoods. In forests experiencing the natural fire regime, stand and landscape structure are both highly heterogeneous, and age structure is usually uneven. Past management (e.g. logging and fire suppression) and its effects on forest succession have resulted in greater structural homogeneity and a dramatic increase in the presence of shade tolerant/fire intolerant tree species. Old-growth stands where fire has been excluded are often multi-storied, with the overstory comprised of various species (often dominated by pines) and the understory dominated by \emph{Abies concolor} and \emph{Calocedrus decurrens}. In the absence of fire, forested stands can form closed, multilayered canopies with over 100\% overlapping cover. Such dense stands were probably relatively uncommon before settlement, and found in moist microsites, on north slopes, and at higher elevations. When openings occur, shrubs are common in the understory. Before Euroamerican settlement, this landcover type was dominated by open stand conditions and old forest, but today closed canopy conditions dominated by middle aged trees are more common. Even aged stands are also widespread (Allen 2005). 

Five conifers and one hardwood typify this landcover type: \emph{A. concolor, Pseudotsuga menziesii, Pinus ponderosa, Pinus lambertiana, C. decurrens}, and \emph{Quercus kelloggii}. \emph{A. concolor} tends to be the most ubiquitous species because it is the competitive dominant in this landcover type. It tolerates shade, reproduces prolifically in the absence of fire, and has the ability to survive long periods of overtopping in brush fields. \emph{P. menziesii} replaces white fir as the competitive dominant at lower elevations. \emph{P. ponderosa}, which was historically the dominant species in SMC forest, still dominates at lower elevations and on south slopes. Like \emph{P. lambertiana}, its densities have been much reduced by logging. \emph{Pinus jeffreyi} commonly replaces \emph{P. ponderosa} at high elevations, on cold sites, or on ultramafic soils. \emph{Abies magnifica} is a minor associate at the highest elevations, as are \emph{Pinus monticola} and \emph{Pinus contorta} ssp. \emph{murrayana}. \emph{P. lambertiana} is found throughout the landcover type, but its densities have been much reduced by selective logging and white pine blister rust. \emph{Q. kelloggii} is a common component in stands on warm, dry sites. It sprouts prolifically after fire, and although it does best on open sites, it is maintained under adverse conditions such as overtopping by conifers and thin soils (Allen 2005). In some locations, \emph{Populus tremuloides} is also a component of the stand and, when present, typically dominates during the early seral stages following disturbance.

\emph{Ceanothus, Arctostaphylos, Chrysolepis, Prunus, Ribes, Rosa}, and \emph{Chamaebatia} are common shrub genera in the understory (Allen 2005). Grasses and forbs are diverse but rarely contribute much cover, except where stand structure is open. 


\begin{adjustwidth}{2cm}{}
\medskip
\noindent \textbf{Mesic Modifier } The primary species associated with mesic sites are \emph{A. concolor, P. menziesii, C. decurrens}, and \emph{P. lambertiana}. \emph{P. contorta} ssp. \emph{murrayana} may also be associated with mesic forests at higher elevations. As elevations begin to increase, \emph{A. magnifica} becomes more prominent. \emph{Lithocarpus densiflora} is an indicator of lower elevation sites with high water availability, either from meteoric or surface water. Understory diversity is often low in these sites, as high canopy cover and tree density reduce solar incidence at the soil surface. Very often the ground is covered in thick litter and duff. Some shade tolerant shrub and herb species occur.

\medskip
\noindent \textbf{Xeric Modifier}  Xeric sites are characterized by the presence of shade intolerant/fire tolerant conifer species such as \emph{P. ponderosa}, \emph{P. jeffreyi}, and \emph{P. lambertiana}, as well as the occurrence of varying amounts of more shade tolerant species like \emph{A. concolor} and \emph{C. decurrens}  \emph{Q. kelloggii} is locally common. The pines normally are prominent on south and west facing slopes, \emph{A. concolor} and sometimes \emph{P. menziesii}  on north and east slopes, and \emph{C. decurrens} as a secondary component on all slopes. At lower elevations, \emph{Pinus sabiniana}, and \emph{Quercus chrysolepis} may become common associates. Understory shrubs include \emph{Ceanothus, Arctostaphylos, Chamaebatia}, and \emph{Artemisia} and \emph{Purshia} in dry, eastern sites.

\medskip
\noindent \textbf{Ultramafic Modifier} Ultramafic soils support a number of endemic plant species. Slowly growing and often stunted \emph{P. contorta} ssp. \emph{murrayana} and \emph{P. jeffreyi} occur in combinations or in nearly pure open stands. Other tree associates on ultramafics include \emph{P. menziesii}, \emph{C. decurrens}, and \emph{Pinus attenuata}. Hardwoods are usually sparse, but shrubs such as \emph{Arctostaphylos, Quercus, Rhamnus, Lithocarpus, Rhododendron}, and \emph{Ceanothus} may occur on these sites. Often, a dramatic landscape shift occurs across abrupt discontinuities between ultramafics and other rock types. For example, regional stands of dense conifer forests are replaced by stunted and open stands of other conifers, by chaparral or even by barrens on which woody vegetation is absent (``CalVeg Zone 1'' 2011).

\end{adjustwidth}

\medskip
\noindent \textbf{Sierran Mixed Conifer  with Aspen} When \emph{P. tremuloides} co-occurs with SMC on the west side of the crest, it is typically found in smaller patches, often less than 2 hectares (5 acres) in size. This variant is not subject to the modifiers described above because it is only found on mesic sites. Mature stands in which \emph{P. tremuloides} are still dominant are usually relatively open. Average canopy closures of stands in eastern California range from 60\% to 100\% in young and intermediate-aged stands and from 25\% to 60\% in mature stands. The open nature of the stands results in substantial light penetration to the ground (Verner 1988).



\subsubsection{Distribution}

\textbf{Sierran Mixed Conifer } SMC generally forms a vegetation band ranging from 500 to 2000 m (1500 to 6500 ft). It dominates the western middle elevation slopes of the Sierra Nevada. Soils supporting SMC are varied in depth and composition, and are derived primarily from Mesozoic granitic, Paleozoic metamorphic rocks, and Cenozoic volcanic rocks (Allen 2005). 

A xeric-mesic gradient was developed based on four variables: 1) aspect, 2) potential evapotranspiration, 3) topographic wetness index, and 4) soil water storage. The variables were standardized by z-score such that higher values correspond to more mesic environments. Thus, potential evapotranspiration was inverted to maintain this balance. The four variables were combined with equal weights. This final variables was split into xeric vs. mesic, with xeric occupying the negative end of the range up to $\frac{1}{4}$ standard deviation below the mean (zero) and mesic occupying the remaining portion of the spectrum.


\begin{adjustwidth}{2cm}{}
\textbf{Mesic Modifier } Generally found on favorable slopes, primarily north and east aspects throughout the geographic range, as well as along streams in drier areas. It is more common at higher elevations as compared to the xeric type (``CalVeg Zone 1'' 2011).

\medskip
\noindent \textbf{Xeric Modifier} Occurs on south and west-facing aspects (LandFire 2007b). At lower elevations patches may be found on north slopes. At higher elevations this landcover type most typically occurs on south, east and west aspects. 

\medskip
\noindent \textbf{Ultramafic Modifier} Ultramafics have been mapped at various spatial densities throughout the elevational range of the SMC landcover type. Low to moderate elevations in ultramafic and serpentinized areas often produce soils low in essential minerals like calcium potassium, and nitrogen, and have excessive accumulations of heavy metals such as nickel and chromium. These sites vary widely in the degree of serpentinization and effects on their overlying plant communities (``CalVeg Zone 1'' 2011). Note, the terms ``ultramafic rock'' and ``serpentine'' are broad terms used to describe a number of different but related rock types, including serpentinite, peridotite, dunite, pyroxenite, talc and soapstone, among others (O'Geen et al. 2007). 

\end{adjustwidth}

\medskip
\noindent \textbf{Sierran Mixed Conifer with Aspen} Sites supporting \emph{P. tremuloides} are usually associated with added soil moisture, i.e., azonal wet sites. These sites are found throughout the SMC zone, often close to streams and lakes. Other sites include meadow edges, rock reservoirs, springs and seeps. Terrain can be simple to complex. At lower elevations, topographic conditions for this type tends toward positions resulting in relatively colder, wetter conditions within the prevailing climate, e.g., ravines, north slopes, wet depressions, etc. (LandFire 2007c).

\subsection*{Disturbances}

\subsubsection{Wildfire}

\textbf{Sierran Mixed Conifer } Wildfires are common and frequent; mortality depends on vegetation vulnerability and wildfire intensity. Low mortality fires kill small trees and may consume above-ground portions of small oaks, shrubs and herbs, but do not kill large trees or below-ground organs of most oaks, shrubs and herbs which promptly resprout. High mortality fires kill trees of all sizes and may kill many of the shrubs and herbs as well. However, high mortality fires typically kill only the above ground portions of the oaks, shrubs and herbs; consequently, most oaks, shrubs and herbs promptly resprout from surviving below ground organs.

Data on fire return intervals (FRIs) are available from a few review papers. Mallek et al. (2013) calculated presettlement fire rotation for 7 major forest types in the Sierra Nevada. Skinner and Chang (1996) aggregated FRIs from the Sierra Nevada and separated pre-1850 data from overall data. Van de Water and Safford’s 2011 review paper aggregates hundreds of articles, conference proceedings, and LandFire data on fire return intervals, with an emphasis on Californian sources. We also include here data from the pertinent individual LandFire BpS models (2007a, 2007b, 2007c, 2007d).

Estimates of fire rotations for these variants are available from the LandFire project and a few review papers. The LandFire project’s published fire return intervals are based on a series of associated models created using the Vegetation Dynamics Development Tool (VDDT). In VDDT, fires are specified concurrently with the transition that follows them. For example, a replacement fire causes a transition to the early development stage. In the RMLands model, such fires are classified as high mortality. However, in VDDT mixed severity fires may cause a transition to early development, a transition to a more open seral stage, or no transition at all. In this case, we categorize the first example as a high mortality fire, and the second and third examples as a low mortality fire. Based on this approach, we calculated fire rotations and the probability of high mortality fire for each of the SMC seral stages across the three variants, as well as for the SMC\_ASP variant (Tables~\ref{tab:smcmdesc_fire}--\ref{tab:smc-aspdesc_fire}). We computed overall target fire rotations based on expert input from Safford and Estes, values from Mallek et al. (2013), and Van de Water and Safford (2011). 


\begin{adjustwidth}{2cm}{}
\textbf{Mesic Modifier } Low mortality fire is fairly frequent. Fire severity is typically positively correlated with slope position. 

\medskip
\noindent \textbf{Xeric Modifier} Fire of all severity levels is fairly common. This landcover type has one of the shortest fire rotations. 

\medskip
\noindent \textbf{Ultramafic Modifier} This type has a very limited distribution and consequently limited information for fire occurrence history. Low mortality fire is more common than high mortality fire. Most medium and high severity fire occurs on middle and upper slope positions.

\end{adjustwidth}

\medskip
\noindent \textbf{Sierran Mixed Conifer with Aspen} Sites supporting \emph{P. tremuloides} are maintained by stand-replacing disturbances that allow regeneration from below-ground suckers. Upland clones are impaired or suppressed by conifer ingrowth and overtopping and intensive grazing that inhibits growth. In a reference condition scenario, a few stands will advance toward conifer dominance, but in the current landscape scenario where fire has been reduced from reference conditions there are many more conifer-dominated mixed aspen stands (LandFire 2007c, Verner 1988).


\begin{table}[]
\small
\centering
\caption{Fire rotation (years) and proportion of high (versus low) mortality fires for Sierran Mixed Conifer - Mesic. Values were derived from Mallek et al. (2013) and VDDT model 0610280 (LandFire 2007a). }
\label{tab:smcmdesc_fire}
\begin{tabular}{@{}lcc@{}}
\toprule
\textbf{Condition}         & \multicolumn{1}{l}{\textbf{Fire Rotation}} & \multicolumn{1}{l}{\textbf{\begin{tabular}[c]{@{}l@{}}Proportion \\ High Mortality\end{tabular}}} \\ \midrule
Target                      & 29            & n/a                           \\
Early Development - All     & 44            & 1                             \\
Mid Development - Closed    & 19            & 0.23                          \\
Mid Development - Moderate  & 13            & 0.17                          \\
Mid Development - Open      & 10            & 0.14                          \\
Late Development - Closed   & 34            & 0.37                          \\
Late Development - Moderate & 13            & 0.14                          \\
Late Development - Open     & 8             & 0.08                     \\ \bottomrule
\end{tabular}
\end{table}

\begin{table}[]
\small
\centering
\caption{Fire rotation (years) and proportion of high (versus low) mortality fires for Sierran Mixed Conifer - Xeric. Values were derived from Mallek et al. (2013) and VDDT model 0610280 (LandFire 2007b).}
\label{tab:smcxdesc_fire}
\begin{tabular}{@{}lcc@{}}
\toprule
\textbf{Condition}         & \multicolumn{1}{l}{\textbf{Fire Rotation}} & \multicolumn{1}{l}{\textbf{\begin{tabular}[c]{@{}l@{}}Proportion \\ High Mortality\end{tabular}}} \\ \midrule
Target                      & 22            & n/a                           \\
Early Development - All     & 32            & 1                             \\
Mid Development - Closed    & 11            & 0.48                          \\
Mid Development - Moderate  & 10            & 0.26                          \\
Mid Development - Open      & 9             & 0.09                          \\
Late Development - Closed   & 16            & 0.25                          \\
Late Development - Moderate & 10            & 0.11                          \\
Late Development - Open     & 8             & 0.05                  \\ \bottomrule
\end{tabular}
\end{table}

\begin{table}[]
\small
\centering
\caption{Fire rotation (years) and proportion of high (versus low) mortality fires for Sierran Mixed Conifer - Ultramafic. Values were derived from Van de Water and Safford (2011), and Mallek et al. (2013) and VDDT model 071220 (LandFire 2007d). }
\label{tab:smcudesc_fire}
\begin{tabular}{@{}lcc@{}}
\toprule
\textbf{Condition}         & \multicolumn{1}{l}{\textbf{Fire Rotation}} & \multicolumn{1}{l}{\textbf{\begin{tabular}[c]{@{}l@{}}Proportion \\ High Mortality\end{tabular}}} \\ \midrule
Target                      & 60            & n/a                           \\
Early Development - All     & 89            & 1                             \\
Mid Development - Closed    & 39            & 0.23                          \\
Mid Development - Moderate  & 27            & 0.17                          \\
Mid Development - Open      & 21            & 0.14                          \\
Late Development - Closed   & 69            & 0.37                          \\
Late Development - Moderate & 27            & 0.14                          \\
Late Development - Open     & 16            & 0.08                  \\ \bottomrule
\end{tabular}
\end{table}

\begin{table}[]
\small
\centering
\caption{Fire rotation (years) and proportion of high (versus low) mortality fires for Sierran Mixed Conifer - Aspen type. Values were derived from VDDT models 0610280 and 0610610 (LandFire 2007a, LandFire 2007c) and Van de Water and Safford (2011). }
\label{tab:smc-aspdesc_fire}
\begin{tabular}{@{}lcc@{}}
\toprule
\textbf{Condition}         & \multicolumn{1}{l}{\textbf{Fire Rotation}} & \multicolumn{1}{l}{\textbf{\begin{tabular}[c]{@{}l@{}}Proportion \\ High Mortality\end{tabular}}} \\ \midrule
Target                           & 29            & n/a                           \\
Early Development - Aspen        & 44            & 1                             \\
Mid Development - Aspen          & 19            & 0.26                          \\
Mid Development - Aspen-Conifer  & 13            & 0.18                          \\
Late Development - Conifer-Aspen & 13            & 0.14                          \\
Late Development - Closed        & 34            & 0.37                  \\ \bottomrule
\end{tabular}
\end{table}

\subsubsection{Other Disturbance}
Other disturbances are not currently modeled, but may, depending on the seral stage affected and mortality levels, reset patches to early development, maintain existing seral stages, or shift/accelerate succession to a more open seral stage. All of the tree species associated with this vegetation type are susceptible to a wide variety of pathogens and insects. 

\subsection*{Vegetation Seral Stages}
We recognize seven separate seral stages for SMC: Early Development (ED), Mid Development - Open Canopy Cover (MDO), Mid Development - Moderate Canopy Cover, Mid Development - Closed Canopy Cover (MDC), Late Development - Open Canopy Cover (LDO), Late Development - Moderate Canopy Cover (LDM), and Late Development - Closed Canopy Cover (LDC) (Figure~\ref{transmodel_smc}). The SMC-ASP variant is also assigned to five seral stages: Early Development - Aspen (ED-A), Mid Development - Aspen (MD-A), Mid Development - Aspen with Conifer (MD-AC), Late Development Closed (LDC), and Late Development - Conifer with Aspen (LD-CA) (Figure~\ref{transmodel_smc-asp}).

Our seral stages are an alternative to ``successional'' classes that imply a linear progression of states and tend not to incorporate disturbance. The seral stages identified here are derived from a combination of successional processes and anthropogenic and natural disturbance, and are intended to represent a composition and structural condition that can be arrived at from multiple other conditions described for that landcover type. Thus our seral stages incorporate age, size, canopy cover, and vegetation composition. In general, the delineation of stages has originated from the LandFire biophysical setting model descriptive of a given landcover type; however, seral stages are not necessarily identical to the classes identified in those models.


\begin{figure}[hbp]
\centering
\includegraphics[width=0.8\textwidth]{/Users/mmallek/Documents/Thesis/statetransmodel/StateTransitionModel/7class.png}
\caption{State and Transition Model for Sierran Mixed Conifer Forest (not inclusive of the aspen variant). Each dark grey box represents one of the seven seral stages for this landcover type. Each column of boxes represents a stage of development: early, middle, and late. Each row of boxes represents a different level of canopy cover: closed (70-100\%), moderate (40-70\%), and open (0-40\%). Transitions between states/seral stages may occur as a result of high mortality fire, low mortality fire, or succession. Specific pathways for each are denoted by the appropriate color line and arrow: red lines relate to high mortality fire, orange lines relate to low mortality fire, and green lines relate to natural succession.} 
\label{transmodel_smc}
\end{figure}

\begin{figure}[htbp]
\centering
\includegraphics[width=0.8\textwidth]{/Users/mmallek/Documents/Thesis/statetransmodel/StateTransitionModel/5class-asp.png}
\caption{State and Transition Model for Sierran Mixed Conifer Forest - Aspen variant. Each dark grey box represents one of the seven seral stages for this landcover type. Each column of boxes represents a stage of development: early, middle, and late. Transitions between states/seral stages may occur as a result of high mortality fire, low mortality fire, or succession. Specific pathways for each are denoted by the appropriate color line and arrow: red lines relate to high mortality fire, orange lines relate to low mortality fire, and green lines relate to natural succession.} 
\label{transmodel_smc-asp}
\end{figure}

\subsection*{Sierran Mixed Conifer }

\paragraph{Description}
\paragraph{Early Development (ED)} This seral stage is characterized by the recruitment of a new cohort of early successional tree species into an open area created by a stand-replacing disturbance. After disturbance, succession proceeds from an ephemeral herb to perennial grass-herb community. This seral stage generally lasts only a few years before shifting to a shrub-seedling-sapling seral stage dominated by any of the following genera: \emph{Arctostaphylos, Ceanothus, Prunus, Ribes, and Chamaebatia}, as well as \emph{Q. vaccinifolia}. Tree seedlings/saplings typical of the cover type can be either high or low density depending on local environmental conditions and climate conditions following the disturbance. In some cases (e.g., favorable climate conditions develop following the stand-replacing disturbance and a good seed source), tree seedlings may develop a nearly continuous canopy and succeed relatively quickly to mid-development seral stages. In other cases, and more commonly on xeric or ultramafic sites, chaparral conditions may dominate and persist for long periods of time (LandFire 2007a, LandFire 2007b).

\paragraph{Succession Transition}

\begin{adjustwidth}{2cm}{}
\textbf{Mesic Modifier } In the absence of disturbance, patches in this seral stage will begin transitioning to MDC or MDO after 20 years at a rate of 0.8 per timestep. The transition to MDC is twice as likely as transition to MDO.  At 40 years, all remaining patches will succeed to either MDC or MDO. On average, patches remain in ED for 26 years.

\medskip
\noindent \textbf{Xeric Modifier}  Transition to the MD seral stages may be substantially delayed. Thus, in the absence of disturbance, patches in this seral stage will begin transitioning to MDO after 40 years and may be delayed in ED for as long as 80 years. During this period, succession occurs at a rate of 0.4 per timestep. On average, patches remain in ED for 53 years.

\medskip
\noindent \textbf{Ultramafic Modifier}  Transition to the MD seral stage may be substantially delayed. Thus, in the absence of disturbance, patches in this seral stage will begin transitioning to MDO after 80 years and may be delayed in ED for as long as 150 years. During this period, succession occurs at a rate of 0.2 per timestep. On average, patches remain in ED for 105 years.

\end{adjustwidth}



\paragraph{Wildfire Transition} High mortality wildfire (100\% of fires in this seral stage) recycles the patch through the Early Development seral stage, regardless of soil type. Low mortality wildfire is not modeled for this seral stage. 

\noindent\hrulefill


\paragraph{Mid Development - Open Canopy Cover (MDO)} 

\paragraph{Description} Heterogeneous ground cover of grasses, forbs, and shrubs. Trees present are pole to medium sized conifers with canopy cover less than 40\% (LandFire 2007a). Conifer species likely present include \emph{A. concolor, C. decurrens P. ponderosa, P. menziesii}, and \emph{P. lambertiana}. Pines predominate on xeric sites while firs predominate on mesic sites. \emph{Q. kelloggi} may occur as well, mostly on warmer slopes and where soils are less productive (LandFire 2007a). Ultramafic sites will have similar species composition, especially at edges, but \emph{P. jeffreyi} and \emph{C. decurrens} are relatively more common (O'Geen et al. 2007).

\paragraph{Succession Transition}
\begin{adjustwidth}{2cm}{}
\textbf{Mesic Modifier } In the absence of low mortality disturbance, patches in the MDO seral stage will begin transitioning to MDM after 15 years at a rate of 0.9 per timestep. Succession to LDO takes place variably after 100 years since entering a middle development seral stage, at a rate of 0.4 per timestep. All patches succeed by 150 years in MD.  On average (across all canopy cover seral stages), patches remain in mid development for 113 years. 

\medskip
\noindent \textbf{Xeric Modifier} In the absence of low mortality disturbance, patches in the MDO seral stage will begin transitioning to MDC after 84 years at a rate of 0.3 per timestep. Succession to LDO takes place variably beginning at 160 years since transition to middle development, at a rate of 0.4 per timestep. All patches succeed by 200 years. On average (across all canopy cover seral stages), patches remain in mid development for 173 years.

\medskip
\noindent \textbf{Ultramafic Modifier} In the absence of low mortality disturbance, patches in the MDO seral stage will begin transitioning to MDC after 40 years in MDO at a rate of 0.1 per timestep. Succession to LDO takes place variably beginning at 200 years since transition to middle development at a rate of 0.4 per timestep. All patches succeed by 260 years. On average (across all canopy cover seral stages), patches remain in mid development for 213 years.

\end{adjustwidth}

\paragraph{Wildfire Transition}
\begin{adjustwidth}{2cm}{}
\textbf{Mesic Modifier } High mortality wildfire (14\% of fires in this seral stage) returns the patch to Early Development. Low mortality fire (86\%) maintains the MDO seral stage and allows for succession to LDO. 

\medskip
\noindent \textbf{Xeric Modifier}  High mortality wildfire (9\% of fires in this seral stage) returns the patch to Early Development. Low mortality fire (91\%) maintains the MDO seral stage and allows for succession to LDO. 

\medskip
\noindent \textbf{Ultramafic Modifier}  High mortality wildfire (14\% of fires in this seral stage) returns the patch to Early Development. Low mortality fire (86\%) maintains the MDO seral stage and allows for succession to LDO.

\end{adjustwidth}

\noindent\hrulefill

\paragraph{Mid Development - Moderate Canopy Cover (MDM)}

\paragraph{Description} Sparse ground cover of grasses, forbs, and shrubs; moderate to dense cover of trees. Conifers are pole to medium-sized, with canopy cover from 40-70\%. Conifer species likely present include \emph{A. concolor, C. decurrens, P. ponderosa, P. menziesii}, and \emph{P. lambertiana}. \emph{Q. kelloggi} may occur as well, mostly on warmer slopes and where soils are less productive (LandFire 2007a, LandFire 2007b). Ultramafic sites will have similar species composition, especially at edges, but \emph{P. jeffreyi} and \emph{C. decurrens} are relatively more common (O'Geen et al. 2007).

\paragraph{Succession Transition}
\begin{adjustwidth}{2cm}{}
\textbf{Mesic Modifier } In the absence of low mortality disturbance, patches in the MDM seral stage will begin transitioning to MDC after 15 years at a rate of 0.9 per timestep. Patches in the MDM seral stage begin transitioning to LDM once the time since transition to a mid development seral stage is at least 100 years at a rate of 0.6 per timestep. All patches succeed by 150 years in mid development. On average (across all canopy cover seral stages), patches remain in mid development for 113 years.

\medskip
\noindent \textbf{Xeric Modifier}  Transition to late seral seral stages may be delayed. In the absence of low mortality disturbance, patches in the MDM seral stage will begin transitioning to MDC after 40 years at a rate of 0.3 per timestep. Patches in this seral stage will begin transitioning to LDC after 160 years in an MD seral stage at a rate of 0.4 per time step and may be delayed in the MDC seral stage for up to 200 years. On average (across all canopy cover seral stages), patches remain in mid development for 173 years. 

\medskip
\noindent \textbf{Ultramafic Modifier} Transition to late seral seral stages may be substantially delayed. Thus, in the absence of disturbance, patches in this seral stage will begin transitioning to MDC after 40 years at a rate of 0.1 per timestep. Patches in the MDM seral stage begin transitioning to LDM once the time since transition to a mid development seral stage is at least 200 years at a rate of 0.4 per timestep. All patches succeed by 260 years in mid development. On average (across all canopy cover seral stages), patches remain in mid development for 213 years.

\end{adjustwidth}

\paragraph{Wildfire Transition}
\begin{adjustwidth}{2cm}{}
\textbf{Mesic Modifier } High mortality wildfire (17\% of fires in this seral stage) returns the patch to ED. Low mortality wildfire (83\%) opens the stand up to MDO 36\% of the time; otherwise, the patch remains in MDM. 

\medskip
\noindent \textbf{Xeric Modifier}  High mortality wildfire (26\% of fires in this seral stage) returns the patch to ED. Low mortality wildfire (74\%) opens the stand up to MDO 32\% of the time; otherwise, the patch remains in MDM.

\medskip
\noindent \textbf{Ultramafic Modifier} High mortality wildfire (17\% of fires in this seral stage) returns the patch to ED. Low mortality wildfire (83\%) opens the stand up to MDO 36\% of the time; otherwise, the patch remains in MDM.

\end{adjustwidth}

\noindent\hrulefill

\paragraph{Mid Development - Closed Canopy Cover (MDC)}

\paragraph{Description} Sparse ground cover of grasses, forbs, and shrubs; moderate to dense cover of trees. Conifers are pole to medium-sized, with canopy cover from 70-100\%. Conifer species likely present include \emph{A. concolor, C. decurrens, P. ponderosa, P. menziesii}, and \emph{P. lambertiana}. \emph{Q. kelloggi} may occur as well, mostly on warmer slopes and where soils are less productive (LandFire 2007a, LandFire 2007b). Ultramafic sites will have similar species composition, especially at edges, but \emph{P. jeffreyi} and \emph{C. decurrens} are relatively more common (O'Geen et al. 2007).

\paragraph{Succession Transition}
\begin{adjustwidth}{2cm}{}
\textbf{Mesic Modifier }  Patches in the MDM seral stage begin transitioning to LDM once the time since transition to a mid development seral stage is at least 100 years in the absence of fire, at which point stands succeed to LDC at a rate of 0.4 per timestep. All patches succeed by 150 years in mid development. On average (across all canopy cover seral stages), patches remain in mid development for 113 years.

\medskip
\noindent \textbf{Xeric Modifier}  Transition to late seral seral stages may be delayed. Thus, in the absence of disturbance, patches in this seral stage will begin transitioning to LDC after 160 years in an mid development seral stage at a rate of 0.4 per time step and may be delayed in the mid development stage for up to 200 years. 

\medskip
\noindent \textbf{Ultramafic Modifier} Transition to late seral seral stages may be substantially delayed. Thus, in the absence of disturbance, patches in this seral stage will begin transitioning to LDC after 200 years in the mid development stage at a rate of 0.4 per time step and may be delayed in a mid development seral stage for up to 260 years.

\end{adjustwidth}

\paragraph{Wildfire Transition}
\begin{adjustwidth}{2cm}{}
\textbf{Mesic Modifier } High mortality wildfire (23\% of fires in this seral stage) returns the patch to ED. Low mortality wildfire (77\%) opens the stand up to MDM 53\% of the time; otherwise, the patch remains in MDC. 

\medskip
\noindent \textbf{Xeric Modifier} High mortality wildfire (48\% of fires in this seral stage) returns the patch to ED. Low mortality wildfire (52\%) opens the stand up to MDM 42\% of the time; otherwise, the patch remains in MDC.

\medskip
\noindent \textbf{Ultramafic Modifier} High mortality wildfire (23\% of fires in this seral stage) returns the patch to ED. Low mortality wildfire (77\%) opens the stand up to MDM 53\% of the time; otherwise, the patch remains in MDC.

\end{adjustwidth}

\noindent\hrulefill


\paragraph{Late Development - Open Canopy Cover (LDO)}

\paragraph{Description} Heterogenous ground cover of grasses, forbs, and low shrubs; low density (less than 40\% canopy cover) of large trees. Occurring in small to moderately-sized patches on southerly aspects and ridge tops. Upper canopy trees may be very large, but overall size classes vary with a patchy distribution and open canopy. This seral stage develops when low-mortality disturbance is fairly frequent; it persists as long as low-mortality fires continue to occur periodically. Conifer species likely present include \emph{A. concolor, C. decurrens, P. ponderosa, P. menziesii}, and \emph{P. lambertiana}. \emph{Q. kelloggi} may occur as well, mostly on warmer slopes and where soils are less productive (LandFire 2007a, LandFire 2007b). Ultramafic sites will have similar species composition, especially at edges, but \emph{P. jeffreyi} and \emph{C. decurrens} are relatively more common (O'Geen et al. 2007).


\paragraph{Succession Transition}
\begin{adjustwidth}{2cm}{}
\textbf{Mesic Modifier } In the presence of low mortality disturbance, patches in this seral stage can self-perpetuate, but after 15 years with no fire, these patches will begin transitioning to LDM at a rate of 0.9 per timestep.

\medskip
\noindent \textbf{Xeric Modifier}  Succession to LDM may occur after 20 years with no fire at a rate of 0.6 per timestep. 

\medskip
\noindent \textbf{Ultramafic Modifier} Patches occurring on ultramafic soils may succeed to LDC after 25 years with no fire at a rate of 0.2 per timestep.

\end{adjustwidth}

\paragraph{Wildfire Transition}
\begin{adjustwidth}{2cm}{}
\textbf{Mesic Modifier } High mortality wildfire (8\% of fires in this seral stage) returns the patch to early development. Low mortality wildfire (92\%) maintains LDO.

\medskip
\noindent \textbf{Xeric Modifier} High mortality wildfire (5\% of fires in this seral stage) returns the patch to early development. Low mortality wildfire (95\%) maintains LDO. 

\medskip
\noindent \textbf{Ultramafic Modifier} High mortality wildfire (8\% of fires in this seral stage) returns the patch to early development. Low mortality wildfire (92\%) maintains LDO.

\end{adjustwidth}

\noindent\hrulefill

\paragraph{Late Development - Moderate Canopy Cover (LDM)}

\paragraph{Description} Overstory of large and very large trees with canopy cover 40-70\%. Understory characterized by medium and smaller-sized shade-tolerant conifers (LandFire 2007a). Conifer species likely present include \emph{A. concolor, C. decurrens, P. ponderosa, P. menziesii}, and \emph{P. lambertiana}. \emph{Q. kelloggi} may occur as well, mostly on warmer slopes and where soils are less productive (LandFire 2007a, LandFire 2007b). Ultramafic sites will have similar species composition, especially at edges, but \emph{P. jeffreyi} and \emph{C. decurrens} are relatively more common (O'Geen et al. 2007).


\paragraph{Succession Transition} 
\begin{adjustwidth}{2cm}{}
\textbf{Mesic Modifier } In the presence of low mortality disturbance, patches in this seral stage can self-perpetuate, but after 15 years with no fire, these patches will begin transitioning to LDC at a rate of 0.9 per timestep.

\medskip
\noindent \textbf{Xeric Modifier} Succession to LDC may occur after 20 years with no fire at a rate of 0.6 per timestep. 

\medskip
\noindent \textbf{Ultramafic Modifier} Patches occurring on ultramafic soils may succeed to LDC after 25 years with no fire at a rate of 0.2 per timestep.

\end{adjustwidth}
\paragraph{Wildfire Transition}
\begin{adjustwidth}{2cm}{}
\textbf{Mesic Modifier } High mortality wildfire (14\% of fires in this seral stage) will return the patch to Early Development. Low mortality wildfire (86\%) usually has little effect, although 24\% of the time it opens the stand up to LDO. 

\medskip
\noindent \textbf{Xeric Modifier} High mortality wildfire (11\% of fires in this seral stage) will return the patch to Early Development. Low mortality wildfire (99\%) usually has little effect, although 30\% of the time it opens the stand up to LDO. 

\medskip
\noindent \textbf{Ultramafic Modifier} High mortality wildfire (14\% of fires in this seral stage) will return the patch to Early Development. Low mortality wildfire (86\%) usually has little effect, although 24\% of the time it opens the stand up to LDO. 

\end{adjustwidth}

\noindent\hrulefill

\paragraph{Late Development - Closed Canopy Cover (LDC)}

\paragraph{Description} Overstory of large and very large trees with canopy cover over 70\%. Understory characterized by medium and smaller-sized shade-tolerant conifers (LandFire 2007a). Conifer species likely present include \emph{A. concolor, C. decurrens, P. ponderosa, P. menziesii}, and \emph{P. lambertiana}. \emph{Q. kelloggi} may occur as well, mostly on warmer slopes and where soils are less productive (LandFire 2007a, LandFire 2007b). Ultramafic sites will have similar species composition, especially at edges, but \emph{P. jeffreyi} and \emph{C. decurrens} are relatively more common (O'Geen et al. 2007).

\paragraph{Succession Transition} In the absence of disturbance, patches in this seral stage will maintain, regardless of soil characteristics.

\paragraph{Wildfire Transition}

\begin{adjustwidth}{2cm}{}
\textbf{Mesic Modifier } High mortality wildfire (37\% of fires in this seral stage) will return the patch to Early Development. Low mortality wildfire (63\%) may have little effect, but 54\% of the time it opens the stand up to LDM. 

\medskip
\noindent \textbf{Xeric Modifier} High mortality wildfire (25\% of fires in this seral stage) will return the patch to Early Development. Low mortality wildfire (65.9\%) may have little effect, but 57\% of the time it opens the stand up to LDM. 

\medskip
\noindent \textbf{Ultramafic Modifier} High mortality wildfire (37\% of fires in this seral stage) will return the patch to Early Development. Low mortality wildfire (63\%) may have little effect, but 54\% of the time it opens the stand up to LDM.

\end{adjustwidth}

\noindent\hrulefill
\noindent\hrulefill

\subsubsection{Aspen Variant}

\paragraph{Early Development - Aspen (ED-A)}

\paragraph{Description} Grasses, forbs, low shrubs, and sparse to moderate cover of tree seedlings/saplings (primarily \emph{P. tremuloides}) with an open canopy. This seral stage is characterized by the recruitment of a new cohort of early successional, shade-intolerant tree species into an open area created by a stand-replacing disturbance.

Following disturbance, succession proceeds rapidly from an herbaceous layer to shrubs and trees, which invade together (Verner 1988). \emph{P. tremuloides} suckers over 6ft tall develop within about 10 years (LandFire 2007c). 



\paragraph{Succession Transition} Unless it burns, a patch in ED-A persists for 10 years, at which point it transitions to MD-A.

\paragraph{Wildfire Transition} High mortality wildfire (100\% of fires in this seral stage) recycles the patch through the ED-A seral stage. Low mortality wildfire is not modeled for this seral stage.

\noindent\hrulefill


\paragraph{Mid Development - Aspen (MD-A)}

\paragraph{Description} \emph{P. tremuloides} trees 5-16 in DBH. Canopy cover is highly variable, and can range from 40-100\%. These patches range in age from 10 to 110 years. Some understory conifers, including \emph{P. ponderosa}, \emph{P. lambertiana}, and \emph{A. concolor} are encroaching, but \emph{P. tremuloides} is still the dominant component of the stand (LandFire 2007c).

\paragraph{Succession Transition} Patches in the MD-A seral stage persist for at least 50 years in the absence of fire, after which stands begin transitioning to MD-AC at a rate of 0.6 per timestep. After 100 years since entering MD-A, any remaining patches transition to MD-AC. 

\paragraph{Wildfire Transition} High mortality wildfire (26\% of fires in this seral stage) recycles the patch through the ED-A seral stage. No transition occurs as a result of low mortality fire.

\noindent\hrulefill

\paragraph{Mid Development - Aspen with Conifer (MD-AC)}

\paragraph{Description} These stands have been protected from fire since the last stand-replacing disturbance. \emph{P. tremuloides} trees are predominantly 16in DBH and greater. Conifers are present and overtopping the \emph{P. tremuloides}. \emph{A. concolor} is a typical conifer that is successional to \emph{P. tremuloides}, and is depicted here, but other conifers including \emph{P. ponderosa} and \emph{P. lambertiana} are also possible. Conifers are pole to medium-sized, and conifer cover is at least 40\% (LandFire 2007c).

\paragraph{Succession Transition} Patches in the MD-AC seral stage persist for 100 years in the absence of high mortality fire, at which point which patches transition to LDC. 

\paragraph{Wildfire Transition} High mortality wildfire (18\% of fires in this seral stage) returns the patch to ED-A. Low mortality wildfire (82\%) maintains the patch in MD-AC.

\noindent\hrulefill

\paragraph{Late Development - Closed (LDC)}

\paragraph{Description} Some \emph{P. tremuloides} continue to be present in the understory, but large conifers are now the dominant tree species, having overtopped the \emph{P. tremuloides}. Smaller conifers are present in the midstory as well. Conifer species likely present include \emph{A. concolor, C. decurrens, P. ponderosa, P. menziesii}, and \emph{P. lambertiana}. (LandFire 2007a, LandFire 2007b, LandFire 2007c). This seral stage is analogous to the LDC seral stage for the SMC variant.

\paragraph{Succession Transition} In the absence of disturbance, patches in this seral stage will maintain, regardless of soil characteristics.

\paragraph{Wildfire Transition} High mortality wildfire (37\% of fires in this seral stage) will return the patch to ED-A. Low mortality wildfire (63\%) opens the stand up to LD-CA 54\% of the time.

\noindent\hrulefill


\paragraph{Late Development - Conifer with Aspen (LD-CA)}

\paragraph{Description} If stands are sufficiently protected from fire such that conifer species overtop \emph{P. tremuloides} and become large, they may be able to withstand some fire that more sensitive \emph{P. tremuloides} cannot. When this occurs, it creates a patch characterized by late development conifers, such as \emph{A. concolor, P. ponderosa}, or \emph{P. lambertiana}, and early seral \emph{P. tremuloides}. 

\paragraph{Succession Transition} Patches in the LD-CA seral stage persist for 70 years, at which time patches transition to LDC. 

\paragraph{Wildfire Transition} High mortality wildfire (14\% of fires in this seral stage) returns the patch to ED-A. Low mortality wildfire (86\%) maintains the stand in LD-CA. 

\noindent\hrulefill



\newpage
\subsection*{Seral Stage Classification}
\begin{table}[hbp]
\small
\centering
\caption{Classification of cover seral stage for SMC. Diameter at Breast Height (DBH) and Cover From Above (CFA) values taken from EVeg polygons. DBH categories are: null, 0-0.9'', 1-4.9'', 5-9.9'', 10-19.9'', 20-29.9'', 30''+. CFA categories are null, 0-10\%, 10-20\%, \dots , 90-100\%. Each row in the table below should be read with a boolean AND across each column of a row.}
\label{smc_classification}
\begin{tabular}{@{}lrrrrr@{}}
\toprule
\textbf{\begin{tabular}[l]{@{}l@{}}Cover \\ Condition\end{tabular}} & \textbf{\begin{tabular}[r]{@{}r@{}}Overstory Tree \\ Diameter 1 \\ (DBH)\end{tabular}} & \textbf{\begin{tabular}[r]{@{}r@{}}Overstory Tree \\ Diameter 2 \\ (DBH)\end{tabular}} & \textbf{\begin{tabular}[r]{@{}r@{}}Total Tree\\ CFA (\%)\end{tabular}} & \textbf{\begin{tabular}[r]{@{}r@{}}Conifer \\ CFA (\%)\end{tabular}} & \textbf{\begin{tabular}[r]{@{}r@{}}Hardwood \\ CFA (\%)\end{tabular}} \\ \midrule
Early All        & null           & any & any    & any    & any  \\
Early All        & 0-4.9''         & any & any    & any    & any  \\
Mid Open         & 5-19.9''        & any & null   & null   & null \\
Mid Open         & 5-19.9''        & any & 0-40   & any    & any  \\
Mid Open         & 5-19.9''        & any & null   & 0-40   & null \\
Mid Moderate     & 5-19.9''        & any & 40-70  & any    & any  \\
Mid Moderate     & 5-19.9''        & any & null   & 40-70  & null \\
Mid Closed       & 5-19.9''        & any & 70-100 & any    & any  \\
Mid Closed       & 5-19.9''        & any & null   & 70-100 & any  \\
Late Closed      & 20''+           & any & 70-100 & any    & any  \\
Late Closed      & 20''+           & any & null   & 70-100 & any  \\
Late Moderate    & 20''+           & any & 40-70  & any    & any  \\
Late Moderate    & 20''+           & any & null   & 40-70  & any  \\
Late Open        & 20''+           & any & null   & null   & null \\
Late Open        & 20''+           & any & 0-40   & any    & any  \\
Late Open        & 20''+           & any & null   & 0-40   & null  \\ \bottomrule
\end{tabular}
\end{table}

SMC-ASP seral stages were assigned manually using NAIP 2010 Color IR imagery to assess seral stage.



\clearpage

\subsection*{References}
\begin{hangparas}{.25in}{1} 
Allen, Barbara H. ``Sierran Mixed Conifer (SMC).'' \emph{A Guide to Wildlife Habitats of California}, edited by Kenneth E. Mayer and William F. Laudenslayer. California Deparment of Fish and Game, 1988, updated 2005. \burl{http://www.dfg.ca.gov/biogeodata/cwhr/pdfs/SMC.pdf}. Accessed 4 December 2012.

``CalVeg Zone 1.'' Vegetation Descriptions. Vegetation Classification and Mapping.  11 December 2008. U.S. Forest Service. \burl{http://www.fs.usda.gov/Internet/FSE_DOCUMENTS/fsbdev3_046448.pdf}. Accessed 2 April 2013.
Estes, Becky. Personal communication, 15 August 2013.

LandFire. ``Biophysical Setting Models.'' Biophysical Setting 0610280: Mediterranean California Mesic Mixed Conifer Forest and Woodland. 2007a. LANDFIRE Project, U.S. Department of Agriculture, Forest Service; U.S. Department of the Interior. \burl{http://www.landfire.gov/national_veg_models_op2.php}. Accessed 9 November 2012.

LandFire. ``Biophysical Setting Models.'' Biophysical Setting 0610270: Mediterranean California Dry-Mesic Mixed Conifer Forest and Woodland. 2007b. LANDFIRE Project, U.S. Department of Agriculture, Forest Service; U.S. Department of the Interior. \burl{http://www.landfire.gov/national_veg_models_op2.php}. Accessed 9 November 2012.

LandFire. ``Biophysical Setting Models.'' Biophysical Setting 0610610: Inter-Mountain Basins Aspen-Mixed Conifer Forest and Woodland. 2007c. LANDFIRE Project, U.S. Department of Agriculture, Forest Service; U.S. Department of the Interior. \burl{http://www.landfire.gov/national_veg_models_op2.php}. Accessed 7 January 2013.

LandFire. ``Biophysical Setting Models.'' Biophysical Setting 0710220: Klamath-Siskiyou Upper Montane Serpentine Mixed Conifer Woodland. 2007d. LANDFIRE Project, U.S. Department of Agriculture, Forest Service; U.S. Department of the Interior. \burl{http://www.landfire.gov/national_veg_models_op2.php}. Accessed 30 November 2012.

O'Geen, Anthony T., Randy A. Dahlgren, and Daniel Sanchez-Mata. ``California Soils and Examples of Ultramafic Vegetation.'' In \emph{Terrestrial Vegetation of California, 3rd Edition}, edited by Michael Barbour, Todd Keeler-Wolf, and Allan A. Schoenherr, 71-106. Berkeley and Los Angeles: University of California Press, 2007. 

Safford, Hugh S. Personal communication.

Skinner, Carl N. and Chi-Ru Chang. ``Fire Regimes, Past and Present.'' \emph{Sierra Nevada Ecosystem Project: Final report to Congress, vol. II, Assessments and scientific basis for management options}. Davis: University of California, Centers for Water and Wildland Resources, 1996.

Van de Water, Kip M. and Hugh D. Safford. ``A Summary of Fire Frequency Estimates for California Vegetation Before Euro-American Settlement.'' \emph{Fire Ecology} 7.3 (2011): 26-57. doi: 10.4996/fireecology.0703026.

Verner, Jared. ``Aspen (ASP).'' \emph{A Guide to Wildlife Habitats of California}, edited by Kenneth E. Mayer and William F. Laudenslayer. California Deparment of Fish and Game, 1988. \burl{http://www.dfg.ca.gov/biogeodata/cwhr/pdfs/ASP.pdf}. Accessed 4 December 2012.


\end{hangparas}


% !TEX root = master.tex
\newpage
\section{Subalpine Conifer (SCN)}
\label{scn-description}

\subsection*{General Information}

\subsubsection{Cover Type Overview}

\textbf{Subalpine Conifer (SCN)}
\newline
Crosswalks
\begin{itemize}
	\item EVeg: Regional Dominance Type 1
	\begin{itemize}
		\item Alpine Mixed Scrub
		\item Mountain Hemlock
		\item Subalpine Conifers
		\item Whitebark Pine
	\end{itemize}

	\item LandFire BpS Model
	\begin{itemize}
		\item Subalpine Conifer
	\end{itemize}

	\item Presettlement Fire Regime Type
	\begin{itemize}
		\item 0610330 Mediterranean California Subalpine Woodland
		\item 0610440 Northern California Mesic Subalpine Woodland
		\item 0610710 Sierra Nevada Alpine Dwarf-Shrubland
	\end{itemize}
\end{itemize}

\noindent \textbf{Subalpine Conifer with Aspen (SCN-ASP)}
This type is created by overlaying the NRIS TERRA Inventory of Aspen on top of the EVeg layer. Where it intersects with SCN it is assigned to SCN-ASP.
\newline

\noindent Reviewed by Marc Meyer, Southern Sierra Province Ecologist, USDA Forest Service

\subsubsection{Vegetation Description}
\textbf{Subalpine Conifer (SCN)} The SCN landscape is comprised of a mosaic of subalpine forests/woodlands, meadows, rock outcrops, and scrub vegetation types. These forests are open stands of conifers occurring on generally sandy soils or rocky slopes at elevations above the upper montane forest stands of \emph{Abies magnifica}. Stand densities are low. Many, but not all, species form shrubby krummholz forms of growth near their upper elevational limits (Fites-Kaufman 2007). 

\emph{Tsuga mertensiana} is often the most common tree species and mixes with \emph{P. contorta} ssp. \emph{murrayana, A. magnifica, Pinus monticola}, and \emph{Pinus albicaulis}. In some areas, \emph{P. contorta} ssp. \emph{murrayana} dominates post-disturbances stands. \emph{T. mertensiana} seedlings are relatively shade tolerant compared to other subalpine conifers and do well under closed canopy conditions. \emph{P. albicaulis} presence increases in the southern portion of the project area (Fites-Kaufman 2007, LandFire 2007a).

Treeline growth of multistemmed trees and shrubby krummholz growth of conifers varies with latitude in the Sierra Nevada. Treeline in the northern Sierra Nevada is dominated by \emph{P. albicaulis}, which frequently occurs with a krummholz form of growth near its upper limit. Several other species may also form krummholz growth forms, including \emph{Juniperus occidentalis}, \emph{Tsuga mertensiana}, \emph{P. contorta} ssp. \emph{murrayana}, and rarely \emph{Pinus jeffreyi} (Fites-Kaufman 2007). 

Although typically of minor importance, a shrub understory may include \emph{Arctostaphylos, Ribes, Phyllodoce, Vaccinium}, and \emph{Kalmia} can occur on moist sites. Herbs present may include \emph{Lupinus, Hieracium, Arabis, Aster}, and \emph{Erigeron}. \emph{Carex} and various grasses are also common (Verner and Purcell 1988, LandFire 2007a).

\medskip
\noindent \textbf{Subalpine Conifer with Aspen (SCN-ASP)} These are upland forests and woodlands dominated by \emph{Populus tremuloides} without a significant conifer component. Conifers may be present in these systems; however, these patches of \emph{P. tremuloides} are not typically successional to conifers. The understory structure may be complex with multiple shrub and herbaceous layers, or simple with just an herbaceous layer. The herbaceous layer may be dense or sparse, dominated by graminoids or forbs. Common shrubs include \emph{Acer, Amelanchier, Artemisia, Juniperus, Prunus, Rosa, Shepherdia, Symphoricarpos}, and the dwarf-shrubs \emph{Mahonia} and \emph{Vaccinium}. Common graminoids may include \emph{Bromus, Calamagrostis, Carex, Elymus, Festuca}, and \emph{Hesperostipa}. Associated forbs may include \emph{Achillea, Eucephalus, Delphinium, Geranium, Heracleum, Ligusticum, Lupinus, Osmorhiza, Pteridium, Rudbeckia, Thalictrum, Valeriana, Wyethia}, and many
others (LandFire 2007b).


\subsubsection{Distribution}
\textbf{Subalpine Conifer (SCN)} The elevational distribution of subalpine forest communities varies with latitude. In the northern Sierra Nevada, such stands begin around 2,450 m and extend up to treeline at 2,750 m to 3,100 m (9,000 ft to 11,000 ft). Both upper and lower limits of subalpine species distributions are driven by a variety of factors, including soil resources, water availability, and climatic limiting factors (Fites-Kaufman 2007).

These forests are characterized by a relatively short growing season with cool temperatures. With the exception of occasional summer thunderstorms, most precipitation falls as snow. Wet years with abundant snowfall can limit growth as these may produce late-lying snowfields that reduce the length of the growing season. Winds can be severe, particularly around exposed ridges. Such wind conditions may produce snow-free winter areas that lower soil temperatures and increase plant water stress (Fites-Kaufman 2007).

Because of the solid granite parent material, areas with deeper soil accumulation can become waterlogged for much of the year. For these reasons, the length of the growing season is a function of not only early season limitation due to low temperatures and snowfields, but also late season limitations due to drought. Studies of the dynamics of alterations of treeline elevation over the past several millennia have reinforced the significance of complex interactions of both temperature and seasonal water availability in determining such changes (Fites-Kaufman 2007). 


\textbf{Subalpine Conifer with Aspen (SCN-ASP)} Sites supporting \emph{P. tremuloides} are associated with added soil moisture, i.e., azonal wet sites. These sites are often close to streams, lakes, and meadows. Other sites include rock reservoirs, springs and seeps. Terrain can be simple to complex. At lower elevations, topographic conditions for this type tends toward positions resulting in relatively colder, wetter conditions within the prevailing climate, e.g., ravines, north slopes, wet depressions, etc. (LandFire 2007b). \emph{P. tremuloides} stands may also be associated with lateral or terminal moraine boulder material, talus-colluvium, rock falls, or lava flows. In addition, pure stands may be found in topographic positions where snow accumulates, mostly at higher north facing elevations, where snow presence means the growing season is too short to support conifers (Shepperd et al. 2006).

\subsection*{Disturbances}

\subsubsection{Wildfire}
\textbf{Subalpine Conifer (SCN)} Most of the subalpine areas of the Sierra Nevada were subjected to repeated glaciation during the Pleistocene, and thus have thin and poorly developed soils with little organic matter. The small amounts of litter accumulation and open stand structure of subalpine forests mean that fire is rare (Fites-Kaufman 2007). It is, however, the major disturbance event of this type (LandFire 2007a). Meyer’s 2013 review suggests that historic and current fire regimes in subalpine forests are normally climate-limited and dominated by surface fires with crown fires occurring occasionally.  

\textbf{Subalpine Conifer with Aspen (SCN-ASP)} Sites supporting \emph{P. tremuloides} are maintained by stand-replacing disturbances that allow regeneration from below-ground suckers. Replacement fire and ground fire are thought to have been common in stable \emph{P. tremuloides} stands historically. Because \emph{P. tremuloides} is associated with mesic conditions, it rarely burns during the normal lightning season. However, during years with little precipitation stands may be more susceptible to burning. Evidence from fire scars and historical studies show that past fires occurred mostly during the spring and fall. These are typically self-perpetuating stands (LandFire)

Estimates of fire rotations for these variants are available from the LandFire project and a few review papers. The LandFire project’s published fire return intervals are based on a series of associated models created using the Vegetation Dynamics Development Tool (VDDT). In VDDT, fires are specified concurrently with the transition that follows them. For example, a replacement fire causes a transition to the early development stage. In the RMLands model, such fires are classified as high mortality. However, in VDDT mixed severity fires may cause a transition to early development, a transition to a more open seral stage, or no transition at all. In this case, we categorize the first example as a high mortality fire, and the second and third examples as a low mortality fire. Based on this approach, we calculated fire rotations and the probability of high mortality fire for each of the SCN and SCN-ASP seral stage (Tables~\ref{tab:scndesc_fire} and \ref{tab:rfr-aspdesc_fire}). We computed overall target fire rotations based on values from Mallek et al. (2013) and Van de Water and Safford (2011) as well as consultations with Meyer, Safford, and Estes (personal communication). 





\begin{table}[!htbp]
\footnotesize
\centering
\caption{Fire rotation index values and probability of high severity fire (at least 75\% overstory tree mortality) probabilities for Subalpine Conifer. The seral stage that is most susceptible to fire (i.e., has the lowest predicted fire rotation) has a fire rotation index value of 1. Higher values correspond with lower likelihoods of experiencing wildfire. The values are relative only within an individual seral stage and should not be compared against other land cover types. Values were derived from VDDT model 0610440 (LandFire 2007), Mallek et al. (2013), and Estes, Safford, and Meyer (personal communication). }
\label{tab:scndesc_fire}
\begin{tabular}{@{}lcc@{}}
\toprule
 \textbf{Seral Stage}    & \textbf{\begin{tabular}[c]{@{}c@{}}Fire Rotation \\ Index\end{tabular}} & \textbf{\begin{tabular}[c]{@{}c@{}}Probability of \\ High Severity Fire\end{tabular}} \\ \hline
Early (All)     		 & 1.65           & 1           \\
Mid--Closed    			 & 1.10           & 0.67        \\
Mid--Moderate  			 & 1.05           & 0.63        \\
Mid--Open      			 & 1.00           & 0.61        \\
Late--Closed   			 & 1.10           & 0.67        \\
Late--Moderate 			 & 1.05           & 0.63        \\
Late--Open     			 & 1.00           & 0.61 	      \\ 
\emph{Target Fire Rotation}    			& \emph{296 years}  &   \\ 
\bottomrule
\end{tabular}
\end{table}

\begin{table}[!htbp]
\footnotesize
\centering
\caption{Fire rotation index values and probability of high severity fire (at least 75\% overstory tree mortality) probabilities for Subalpine Conifer - Aspen type. The seral stage that is most susceptible to fire (i.e., has the lowest predicted fire rotation) has a fire rotation index value of 1. Higher values correspond with lower likelihoods of experiencing wildfire. The values are relative only within an individual seral stage and should not be compared against other land cover types. Values were derived from VDDT model 0610110 (LandFire 2007), Van de Water and Safford (pers. comm. 2013), Safford, and Estes (personal communication).}
\label{tab:scnasp-desc_fire}
\begin{tabular}{@{}lcc@{}}
\toprule
 \textbf{Seral Stage}    & \textbf{\begin{tabular}[c]{@{}c@{}}Fire Rotation \\ Index\end{tabular}} & \textbf{\begin{tabular}[c]{@{}c@{}}Probability of \\ High Severity Fire\end{tabular}} \\ \hline
Early--Aspen        & 2.0           & 1         \\
Mid--Aspen          & 1.0           & 0.67      \\
Late--Conifer-Aspen & 1.3           & 0.63		   \\ 
\emph{Target Fire Rotation}    			& \emph{296 years}  &   \\ 
\bottomrule
\end{tabular}
\end{table}

\subsubsection{Other Disturbance}
Other disturbances are not currently modeled, but may, depending on the seral stage affected and mortality levels, reset patches to early development, maintain existing seral stage, or shift/accelerate succession to a more open seral stage. 

\subsection*{Vegetation Seral Stages}
We recognize seven separate seral stages for SCN: Early Development (ED), Mid Development - Open Canopy Cover (MDO), Mid Development - Moderate Canopy Cover, Mid Development - Closed Canopy Cover (MDC), Late Development - Open Canopy Cover (LDO), Late Development - Moderate Canopy Cover (LDM), and Late Development - Closed Canopy Cover (LDC) (Figure~\ref{transmodel_scn}). The SCN-ASP variant is assigned to three seral stages: Early Development - Aspen (ED-A), Mid Development - Aspen (MD-A), and Late Development - Conifer with Aspen (LD-CA) (Figure~\ref{transmodel_scn-asp}).

Our seral stages are an alternative to ``successional'' classes that imply a linear progression of states and tend not to incorporate disturbance. The seral stages identified here are derived from a combination of successional processes and anthropogenic and natural disturbance, and are intended to represent a composition and structural condition that can be arrived at from multiple other conditions described for that landcover type. Thus our seral stages incorporate age, size, canopy cover, and vegetation composition. In general, the delineation of stages has originated from the LandFire biophysical setting model descriptive of a given landcover type; however, seral stages are not necessarily identical to the classes identified in those models.


\begin{figure}[htbp]
\centering
\includegraphics[width=0.8\textwidth]{/Users/mmallek/Documents/Thesis/statetransmodel/StateTransitionModel/7class.png}
\caption{State and Transition Model for Subalpine Conifer Forest (not inclusive of the aspen variant). Each dark grey box represents one of the seven seral stage for this landcover type. Each column of boxes represents a stage of development: early, middle, and late. Each row of boxes represents a different level of canopy cover: closed (70-100\%), moderate (40-70\%), and open (0-40\%). Transitions between states/seral stage may occur as a result of high mortality fire, low mortality fire, or succession. Specific pathways for each are denoted by the appropriate color line and arrow: red lines relate to high mortality fire, orange lines relate to low mortality fire, and green lines relate to natural succession.} 
\label{transmodel_scn}
\end{figure}

\begin{figure}[htbp]
\centering
\includegraphics[width=0.8\textwidth]{/Users/mmallek/Documents/Thesis/statetransmodel/StateTransitionModel/3class-asp.png}
\caption{State and Transition Model for Subalpine Conifer Forest, Aspen variant. Each dark grey box represents one of the three seral stages for this landcover type. Three seral stages of development are represented: early, middle, and late. Transitions between states/seral stages may occur as a result of high mortality fire, low mortality fire, or succession. Specific pathways for each are denoted by the appropriate color line and arrow: red lines relate to high mortality fire, orange lines relate to low mortality fire, and green lines relate to natural succession.} 
\label{transmodel_scn-asp}
\end{figure}

\paragraph{Early Development (ED)}

\paragraph{Description} The first few years following stand-replacing wildfire are characterized by bare ground, herbs, shrubs, and varying densities of tree seedlings (presumably dependent on seed sources). Dominant species include coniferous tree seedlings, resprouting grasses and shrubs, and invading herbs. Shrubs include \emph{Ribes} spp. Herbs and grasses include \emph{Aster, Pedicularis, Hieracium, Arabis, Erigeron, Carex, Luzula}, and \emph{Poa} (LandFire 2007a).

\paragraph{Succession Transition} In the absence of disturbance, patches in this seral stage will begin transitioning to mid development after 20 years at a rate of 0.4 per time step. Transition to either MDC or MDO can occur, although transition to MDC occurs 90\% of the time. At 80 years, all patches will succeed. On average, patches remain in ED for 33 years.

\paragraph{Wildfire Transition} High mortality wildfire (100\% of fires) recycles the patch through the Early Development seral stage. Low mortality wildfire is not modeled for this seral stage.

\noindent\hrulefill


\paragraph{Mid Development - Open Canopy Cover (MDO)} 

\paragraph{Description} This seral stage represents delayed tree regeneration and long-term domination by shrubs and herbs. Shrubs include \emph{Ribes} spp. Herbs and grasses include \emph{Aster, Pedicularis, Hieracium, Arabis, Erigeron, Carex, Luzula}, and \emph{Poa}. Trees are represented by seedlings and saplings of \emph{T. mertensiana}, \emph{P. contorta} ssp. \emph{murrayana}, and other species (LandFire 2007a).

\paragraph{Succession Transition} Patches in this seral stage will maintain under low mortality disturbance. In the absence of low mortality disturbance, patches in the MDO seral stage will begin transitioning to MDM after 40 years at a rate of 0.3 per timestep. Succession to LDO takes place variably after 60 years since entering a middle development seral stage, at a rate of 0.45 per timestep. All patches succeed by 130 years in mid development.  On average (across all canopy cover seral stages), patches remain in mid development for 71 years.

\paragraph{Wildfire Transition} High mortality wildfire (61\% of fires) recycles the patch through the Early Development seral stage. Low mortality wildfire (39\%) maintains the patch in MDO.

\noindent\hrulefill

\paragraph{Mid Development - Moderate Canopy Cover (MDM)}

\paragraph{Description} This seral stage represents rapid regeneration by \emph{P. contorta} ssp. \emph{murrayana}, with additional conifers coming in, including \emph{T. mertensiana}, \emph{A. magnifica}, and \emph{P. monticola}. Shrubs include \emph{Ribes} spp. Herbs and grasses include \emph{Aster, Pedicularis, Hieracium, Arabis, Erigeron, Carex, Luzula}, and \emph{Poa}. (LandFire 2007a).

\paragraph{Succession Transition} In the absence of low mortality disturbance, patches in the MDM seral stage will begin transitioning to MDC after 40 years at a rate of 0.3 per timestep. Succession to LDM takes place variably after 60 years since entering a middle development seral stage, at a rate of 0.45 per timestep. All patches succeed by 130 years in mid development.  On average (across all canopy cover seral stages), patches remain in mid development for 71 years.
 
\paragraph{Wildfire Transition} High mortality wildfire (63\% of fires) recycles the patch through the Early Development seral stage. Low mortality wildfire (37\%) triggers a transition to MDO.

\noindent\hrulefill

\paragraph{Mid Development - Closed Canopy Cover (MDC)}

\paragraph{Description} This seral stage represents rapid regeneration by \emph{P. contorta} ssp. \emph{murrayana}, with additional conifers coming in, including \emph{T. mertensiana}, \emph{A. magnifica}, and \emph{P. monticola}. Shrubs include \emph{Ribes} spp. Herbs and grasses include \emph{Aster, Pedicularis, Hieracium, Arabis, Erigeron, Carex, Luzula}, and \emph{Poa}. (LandFire 2007a).

\paragraph{Succession Transition} After 60 years without a wildfire-triggered transition, patches in this seral stage will begin transitioning to LDC at a rate of 0.45 per time step. Succession to LDC may occur once the patch age since transition to the mid development stage is at least 60 years. After 130 years, all patches will succeed.

\paragraph{Wildfire Transition} High mortality wildfire (67\% of fires) recycles the patch through the Early Development seral stage. Low mortality wildfire (33\%) triggers a transition to MDM.

\noindent\hrulefill


\paragraph{Late Development - Open Canopy Cover (LDO)}

\paragraph{Description} This seral stage represents late-successional stands with large individuals (DBH greater than 20 in) of \emph{T. mertensiana} and other species. The open stand structure is maintained by mixed severity fire and insect-caused tree mortality (the latter not modeled at this time). Shrubs include \emph{Ribes} spp. Herbs and grasses include \emph{Aster, Pedicularis, Hieracium, Arabis, Erigeron, Carex, Luzula}, and \emph{Poa}. (LandFire 2007a).

\paragraph{Succession Transition} In the absence of any fire, succession to LDM begins at 40 years at a rate of 0.3 per timestep.

\paragraph{Wildfire Transition} High mortality wildfire (61\% of fires) recycles the patch through the Early Development seral stage. Low mortality wildfire (39\%) maintains the patch in LDO.

\noindent\hrulefill

\paragraph{Late Development - Moderate Canopy Cover (LDM)}

\paragraph{Description} This seral stage represents late-successional stands with large individuals (DBH greater than 20 in) of \emph{T. mertensiana} and other species, and advanced regeneration of \emph{T. mertensiana} and other shade tolerant species. The moderately open stand structure is generated by recent low mortality fire and insect-caused tree mortality (the latter not modeled at this time). Shrubs include \emph{Ribes} spp. Herbs and grasses include \emph{Aster, Pedicularis, Hieracium, Arabis, Erigeron, Carex, Luzula}, and \emph{Poa}. (LandFire 2007a).

\paragraph{Succession Transition} In the absence of any fire, succession to LDC begins at 40 years at a rate of 0.3 per timestep.

\paragraph{Wildfire Transition} High mortality wildfire (63\% of fires) recycles the patch through the Early Development seral stage. Low mortality wildfire (37\%) triggers a transition to LDO. 

\noindent\hrulefill

\paragraph{Late Development - Closed Canopy Cover (LDC)}

\paragraph{Description} This seral stage represents late-successional stands with large individuals (DBH greater than 20 in) of \emph{T. mertensiana} and other species, and advanced regeneration of \emph{T. mertensiana} and other shade tolerant species. Shrubs include \emph{Ribes} spp. Herbs and grasses include \emph{Aster, Pedicularis, Hieracium, Arabis, Erigeron, Carex, Luzula}, and \emph{Poa}. (LandFire 2007a).

\paragraph{Succession Transition} Patches in this seral stage will maintain in the absence of disturbance.

\paragraph{Wildfire Transition} High mortality wildfire (67\% of fires) recycles the patch through the Early Development seral stage. Low mortality wildfire (33\%) triggers a transition to LDM. 

\noindent\hrulefill
\noindent\hrulefill

\subsubsection{Aspen Variant}

\paragraph{Early Development - Aspen (ED-A)}

\paragraph{Description} Grasses, forbs, low shrubs, and sparse to moderate cover of tree seedlings/saplings (primarily \emph{P. tremuloides}) with an open canopy. This seral stage is characterized by the recruitment of a new cohort of early successional, shade-intolerant tree species into an open area created by a stand-replacing disturbance. Following disturbance, succession proceeds rapidly from an herbaceous layer to shrubs and trees, which invade together (Verner 1988). \emph{P. tremuloides} suckers over 6ft tall develop within about 10 years (LandFire 2007b). 


\paragraph{Succession Transition} Unless it burns, a patch in the early seral stage persists for 10 years, at which point it transitions to MD-A.

\paragraph{Wildfire Transition} High mortality wildfire (100\% of fires) recycles the patch through the ED-A seral stage. Low mortality wildfire is not modeled for this seral stage.

\noindent\hrulefill


\paragraph{Mid Development - Aspen (MD-A)}

\paragraph{Description} \emph{P. tremuloides} trees 5-16'' DBH. Canopy cover is highly variable, and can range from 40-100\%. These patches range in age from 10 to 110 years (LandFire 2007b).

\paragraph{Succession Transition} Patches in the MD-A seral stage persist for at least 80 years in the absence of fire, at which point they begin transitioning to LD-CA at a rate of 0.3 per timestep. After 200 years since entering MD-A, any remaining patches transition to LD-CA. 

\paragraph{Wildfire Transition} High mortality wildfire (67\% of fires in this seral stage) recycles the patch through the ED-A seral stage. No transition occurs as a result of low mortality fire (33\%).

\noindent\hrulefill



\paragraph{Late Development - Aspen with Conifer (LD-AC)}

\paragraph{Description} These stands have been protected from fire since the last stand-replacing disturbance. \emph{P. tremuloides} trees are predominantly 16'' DBH and greater. Conifers are encroaching and can eventually overtop the aspen (LandFire 2007).

\paragraph{Succession Transition} Patches in this seral stage will maintain in the absence of disturbance.

\paragraph{Wildfire Transition} High mortality wildfire (63\% of fires in this seral stage) returns the patch to ED-A. Low mortality wildfire (33\%) maintains the patch in LD-CA.

\noindent\hrulefill




\subsection*{Seral Stage Classification}
\begin{table}[hbp]
\footnotesize
\centering
\caption{Classification of cover seral stage for SCN. Diameter at Breast Height (DBH) and Cover From Above (CFA) values taken from EVeg polygons. DBH categories are: null, 0-0.9'', 1-4.9'', 5-9.9'', 10-19.9'', 20-29.9'', 30''+. CFA categories are null, 0-10\%, 10-20\%, \dots , 90-100\%. Each row in the table below should be read with a boolean AND across each column.}
\label{scn_classification}
\begin{tabular}{@{}lrrrrr@{}}
\toprule
\textbf{\begin{tabular}[l]{@{}l@{}}Cover \\ Condition\end{tabular}} & \textbf{\begin{tabular}[r]{@{}r@{}}Overstory Tree \\ Diameter 1 \\ (DBH)\end{tabular}} & \textbf{\begin{tabular}[r]{@{}r@{}}Overstory Tree \\ Diameter 2 \\ (DBH)\end{tabular}} & \textbf{\begin{tabular}[r]{@{}r@{}}Total Tree\\ CFA (\%)\end{tabular}} & \textbf{\begin{tabular}[r]{@{}r@{}}Conifer \\ CFA (\%)\end{tabular}} & \textbf{\begin{tabular}[r]{@{}r@{}}Hardwood \\ CFA (\%)\end{tabular}} \\ \midrule
Early All        & null           & any & any    & any    & any  \\
Early All        & 0-4.9''         & any & any    & any    & any  \\
Mid Open         & 5-19.9''        & any & null   & null   & null \\
Mid Open         & 5-19.9''        & any & 0-40   & any    & any  \\
Mid Open         & 5-19.9''        & any & null   & 0-40   & null \\
Mid Moderate     & 5-19.9''        & any & 40-70  & any    & any  \\
Mid Moderate     & 5-19.9''        & any & null   & 40-70  & null \\
Mid Closed       & 5-19.9''        & any & 70-100 & any    & any  \\
Mid Closed       & 5-19.9''        & any & null   & 70-100 & any  \\
Late Closed      & 20''+           & any & 70-100 & any    & any  \\
Late Closed      & 20''+           & any & null   & 70-100 & any  \\
Late Moderate    & 20''+           & any & 40-70  & any    & any  \\
Late Moderate    & 20''+           & any & null   & 40-70  & any  \\
Late Open        & 20''+           & any & null   & null   & null \\
Late Open        & 20''+           & any & 0-40   & any    & any  \\
Late Open        & 20''+           & any & null   & 0-40   & null \\ \bottomrule
\end{tabular}
\end{table}

SCN-ASP seral stages were assigned manually using NAIP 2010 Color IR imagery to assess seral stage.


\clearpage

\subsection*{References}

\begin{hangparas}{.25in}{1} 
\interlinepenalty=10000

Estes, Becky. Personal communication, 15 August 2013.

Fites-Kaufman, Jo Ann, Phil Rundel, Nathan Stephenson, and Dave A. Wixelman. ``Montane and Subalpine Vegetation of the Sierra Nevada and Cascade Ranges.'' In \emph{Terrestrial Vegetation of California, 3rd Edition}, edited by Michael Barbour, Todd Keeler-Wolf, and Allan A. Schoenherr, 456-501. Berkeley and Los Angeles: University of California Press, 2007. 

LandFire. ``Biophysical Setting Models.'' Biophysical Setting 0610440: Northern California Mesic Subalpine Woodland. 2007a. LANDFIRE Project, U.S. Department of Agriculture, Forest Service; U.S. Department of the Interior. \burl{http://www.landfire.gov/national_veg_models_op2.php}. Accessed 9 November 2012.

LandFire. ``Biophysical Setting Models.'' Biophysical Setting 0610610: Inter-Mountain Basins Aspen-Mixed Conifer Forest and Woodland. 2007b. LANDFIRE Project, U.S. Department of Agriculture, Forest Service; U.S. Department of the Interior. \burl{http://www.landfire.gov/national_veg_models_op2.php}. Accessed 7 January 2013.

Meyer, Marc D. ``Natural Range of Variation of Red Fir Forests in the Bioregional Assessment Area'' (unpublished paper, Ecology Group, Pacific Southwest Research Station, 2013).

Safford, Hugh S. Personal communication, 5 May 2013, 15 August 2013.

Shepperd, Wayn De, Paul C. Rogers, David Burton, and Dale L. Bartos. ``Ecology, Biodiversity, Management, and Restoration of Aspen in the Sierra Nevada.'' General Technical Report RMRS-GTR-178. Rocky Mountain Research Station, Forest Service, U.S. Department of Agriculture, 2006.

Van de Water, Kip M. and Safford, Hugh D. ``A Summary of Fire Frequency Estimates for California Vegetation Before Euro-American Settlement.'' \emph{Fire Ecology} 7.3 (2011): 26-57. doi: 10.4996/fireecology.0703026.

Verner, Jared. ``Aspen (ASP).'' \emph{A Guide to Wildlife Habitats of California}. 1988. Mayer, Kenneth E. and Laudenslayer, William F., eds. California Department of Fish and Game. \burl{http://www.dfg.ca.gov/biogeodata/cwhr/pdfs/ASP.pdf}. Accessed 4 December 2012.

\end{hangparas}


% !TEX root = master.tex
\newpage
\section{Western White Pine (WWP)}
\label{wwp-description}
\subsection*{General Information}

\subsubsection{Cover Type Overview}

\textbf{Western White Pine (WWP)}
\newline
\textbf{Crosswalks}
\begin{itemize}
	\item EVeg: Regional Dominance Type 1
	\begin{itemize}
		\item Western White Pine 
	\end{itemize}

	\item LandFire BpS Model
	\begin{itemize}
		\item 0711720: Sierran-Intermontane Desert Western White Pine-White Fir Woodland
	\end{itemize}

	\item Presettlement Fire Regime Type
	\begin{itemize}
		\item Western White Pine
	\end{itemize}
\end{itemize}

\noindent Reviewed by Becky Estes, Central Sierra Province Ecologist, USDA Forest Service

\subsubsection{Vegetation Description}
\emph{Pinus monticola} is locally abundant in subalpine habitats along the west slope of the Sierra Nevada, where it may occur in small pure stands. More commonly, it mixes with \emph{Pinus contorta} ssp. \emph{murrayana, Pinus jeffreyi, Tsuga mertensiana}, and \emph{Abies magnifica} (particularly on the west side of the Sierra crest) and \emph{Abies concolor} or \emph{Pinus ponderosa} (particularly on the east side) (Fites-Kaufman et al. 2007, LandFire 2007, Estes pers. comm. 2013).

This system tends to be more woodland than forest in character, and the undergrowth is more open and drier, with little shrub or herbaceous cover. Tree regeneration is less prolific than in other mixed-montane conifer systems of the Cascades, Sierras and California Coast Ranges (LandFire 2007). \emph{P. monticola} generally maintains a tree form of growth up nearly to treeline, where it is commonly replaced by other subalpine species on rocky ridges (Fites-Kaufman et al. 2007).

Understories are typically open, with moderately low shrub cover and diversity, and include \emph{Arctostaphylos, Chrysolepis, Ceanothus}, and \emph{Ribes}. Common herbaceous taxa include \emph{Arnica, Festuca, Poa, Carex, Pyrola}, and \emph{Hieracium}. In openings, \emph{Wyethia} can be abundant (LandFire 2007).


\subsubsection{Distribution}
With respect to the focal landscape within the northern Sierra Nevada, these forests and woodlands are found in the upper montane to subalpine zones, at elevations generally over 2000 m (6560 ft). 

It is found on all slopes and aspects, although it occurs more frequently on drier areas. This ecological system generally occurs on basalts, andesite, glacial till, basaltic rubble, colluvium, or volcanic ash-derived soils. These soils have characteristic features of good aeration and drainage, coarse textures, circumneutral to slightly acidic pH, an abundance of mineral material, rockiness, and periods of drought during the growing season. Climatically, this system occurs somewhat in the rain shadow of the Sierras and has a more continental regime, similar to the northern Great Basin (LandFire 2007).


\subsection*{Disturbances}
Most fires in this type are low mortality fires that allow large areas of the landscape to develop mature characteristics. Occasional severe fires are driven by weather extremes (LandFire 2007). Young trees are very susceptible to mortality from fire, but mature \emph{P. monticola} is moderately fire resistant. After a stand-replacing fire, \emph{P. monticola} will seed in from adjacent areas. After a cool to moderate fire that leaves a mosaic of mineral soil and duff, it will reoccupy the site from seed stored in the seed bank. Overall, \emph{P. monticola} is a fire-dependent, seral species. Fire suppression has resulted in decreased stocking levels, mostly due to the increase in White pine blister rust (\emph{Cronartium ribicola}). Periodic, stand-replacing fire or other disturbance is needed to remove competing conifers and allow \emph{P. monticola} to develop (Griffith 1992). 

Estimates of fire rotations for these variants are available from the LandFire project and a few review papers. The LandFire project’s published fire return intervals are based on a series of associated models created using the Vegetation Dynamics Development Tool (VDDT). In VDDT, fires are specified concurrently with the transition that follows them. For example, a replacement fire causes a transition to the early development stage. In the RMLands model, such fires are classified as high mortality. However, in VDDT mixed severity fires may cause a transition to early development, a transition to a more open seral stage, or no transition at all. In this case, we categorize the first example as a high mortality fire, and the second and third examples as a low mortality fire. Based on this approach, we calculated fire rotations and the probability of high mortality fire for each of the WWP and WWP-ASP seral stages (Table~\ref{tab:wwpdesc_fire}). We computed overall target fire rotations based on values from Van de Water and Safford (2011). 

\subsubsection{Wildfire}




\begin{table}[!htbp]
\footnotesize
\centering
\caption{Fire rotation index values and probability of high severity fire (at least 75\% overstory tree mortality) probabilities. The seral stage that is most susceptible to fire (i.e., has the lowest predicted fire rotation) has a fire rotation index value of 1. Higher values correspond with lower likelihoods of experiencing wildfire. The values are relative only within an individual seral stage and should not be compared against other land cover types. Values were derived from VDDT model 0711720 (LandFire 2007) and Van de Water and Safford (2011). }
\label{tab:wwpdesc_fire}
\begin{tabular}{@{}lcc@{}}
\toprule
 \textbf{Seral Stage}    & \textbf{\begin{tabular}[c]{@{}c@{}}Fire Rotation \\ Index\end{tabular}} & \textbf{\begin{tabular}[c]{@{}c@{}}Probability of \\ High Severity Fire\end{tabular}} \\ \hline
Early (All)     		 & 1.8            & 0.17         \\
Mid--Closed    			 & 1.8            & 0.18         \\
Mid--Moderate  			 & 1.3            & 0.12         \\
Mid--Open      			 & 1.0            & 0.09         \\
Late--Closed   			 & 1.8            & 0.17         \\
Late--Moderate 			 & 1.3            & 0.12         \\
Late--Open     			 & 1.0            & 0.09 	    \\ 
\emph{Target Fire Rotation}    			& \emph{88 years}  &   \\ 
\bottomrule
\end{tabular}
\end{table}

\subsubsection{Other Disturbance}
Other disturbances are not currently modeled, but may, depending on the seral stage affected and mortality levels, reset patches to early development, maintain existing seral stages, or shift/accelerate succession to a more open seral stage. 

\subsection*{Vegetation Seral Stages}
We recognize seven separate seral stages for WWP: Early Development (ED), Mid Development - Open Canopy Cover (MDO), Mid Development - Moderate Canopy Cover, Mid Development - Closed Canopy Cover (MDC), Late Development - Open Canopy Cover (LDO), Late Development - Moderate Canopy Cover (LDM), and Late Development - Closed Canopy Cover (LDC) (Figure~\ref{transmodel_wwp}). Our seral stages are an alternative to ``successional'' classes that imply a linear progression of states and tend not to incorporate disturbance. The seral stages identified here are derived from a combination of successional processes and anthropogenic and natural disturbance, and are intended to represent a composition and structural condition that can be arrived at from multiple other conditions described for that landcover type. Thus our seral stages incorporate age, size, canopy cover, and vegetation composition. In general, the delineation of stages has originated from the LandFire biophysical setting model descriptive of a given landcover type; however, seral stages are not necessarily identical to the classes identified in those models.


\begin{figure}[hbp]
\centering
\includegraphics[width=0.8\textwidth]{/Users/mmallek/Documents/Thesis/statetransmodel/StateTransitionModel/7class.png}
\caption{State and Transition Model for Western White Pine Forest. Each dark grey box represents one of the seven seral stages for this landcover type. Each column of boxes represents a stage of development: early, middle, and late. Each row of boxes represents a different level of canopy cover: closed (70-100\%), moderate (40-70\%), and open (0-40\%). Transitions between states/seral stages may occur as a result of high mortality fire, low mortality fire, or succession. Specific pathways for each are denoted by the appropriate color line and arrow: red lines relate to high mortality fire, orange lines relate to low mortality fire, and green lines relate to natural succession.} 
\label{transmodel_wwp}
\end{figure}

\paragraph{Early Development (ED)}

\paragraph{Description} Open stand of \emph{P. monticola}, \emph{A. magnifica}, as well as other tree seedlings mixed with grasses and shrubs. Early seral dominant species include Ceanothus and various grasses. A portion of these stands get into a shrub dominated stage that can persist for for a few decades (LandFire 2007, Estes 2013).

\paragraph{Succession Transition} In the absence of disturbance, patches in this seral stage will begin transitioning to an mid development seral stage after 30 years at a rate of 0.7 per time step. At 70 years, all remaining patches will succeed. The secondary rate of succession to MDO is 0.8, and to MDC is 0.2. On average, patches remain in early development for 43 years.

\paragraph{Wildfire Transition} High mortality wildfire (17\% of fires in this seral stage) recycles the patch through the Early Development seral stage. No transition occurs as a result of low mortality fire.

\noindent\hrulefill


\paragraph{Mid Development - Open Canopy Cover (MDO)}

\paragraph{Description} Open stand of early seral tree species. Heterogeneous ground cover of grasses, forbs, and shrubs. Trees present are pole to medium sized conifers with canopy cover less than 40\%. Conifer species likely present include \emph{P. monticola}, \emph{A. magnifica}, and \emph{P. jeffreyi} (LandFire 2007, Estes 2013).

\paragraph{Succession Transition} Patches in this seral stage will maintain under low mortality disturbance, but after 15 years without fire they begin transitioning to MDM at a rate of 0.8 per time step. Succession to LDO occurs once the patch has been in mid development for 70 years. The rate of succession per time step is 0.4. At 120 years, all remaining patches succeed. On average, patches remain in early development for 83 years.

\paragraph{Wildfire Transition} High mortality wildfire (9\% of fires in this seral stage) recycles the patch through the ED seral stage. Low mortality wildfire (91\%) maintains the patch in MDO.

\noindent\hrulefill

\paragraph{Mid Development - Moderate Canopy Cover (MDM)}

\paragraph{Description} Sparse ground cover of grasses, forbs, and shrubs; moderate to dense cover of trees. Conifers are pole to medium-sized, with canopy cover ranging from 40-70\%. Conifer species likely present include \emph{P. monticola}, \emph{A. magnifica}, and \emph{P. jeffreyi} (LandFire 2007, Estes 2013).

\paragraph{Succession Transition} Patches in this seral stage will maintain under low mortality disturbance, but after 15 years without fire they begin transitioning to MDC at a rate of 0.8 per time step. Succession to LDM occurs once the patch has been in mid development for 70 years. The rate of succession per time step is 0.4. At 120 years, all remaining patches succeed. On average, patches remain in early development for 83 years.

\paragraph{Wildfire Transition} High mortality wildfire (12\% of fires in this seral stage) recycles the patch through the ED seral stage. Low mortality wildfire (88\%) opens the patch up to MDO 40\% of the time; otherwise, the patch remains in MDM.

\noindent\hrulefill

\paragraph{Mid Development - Closed Canopy Cover (MDC)}

\paragraph{Description} Sparse ground cover of grasses, forbs, and shrubs; moderate to dense cover of trees. Conifers are pole to medium-sized, with canopy cover over 70\%. Forests of this type rarely, if ever, exceed 80\% canopy closure even in closed, dense seral stages. Conifer species likely present include \emph{P. monticola}, \emph{A. magnifica}, and \emph{P. jeffreyi} (LandFire 2007, Estes 2013).

\paragraph{Succession Transition} Succession to LDC occurs once the patch has been in mid development for 70 years. The rate of succession per time step is 0.4. At 120 years, all remaining patches succeed. On average, patches remain in early development for 83 years.

\paragraph{Wildfire Transition} High mortality wildfire (18\% of fires in this seral stage) recycles the patch through the Early Development seral stage. Low mortality wildfire (82\%) opens the patch up to MDM 80\% of the time; otherwise, the patch remains in MDC.

\noindent\hrulefill


\paragraph{Late Development - Open Canopy Cover (LDO)}

\paragraph{Description} Open stands of large trees, primarily \emph{P. monticola}, \emph{A. magnifica}, and \emph{P. jeffreyi}. Canopy cover is less than 40\% (LandFire 2007, Estes 2013).

\paragraph{Succession Transition} Patches in this seral stage will maintain under low mortality disturbance, but after 15 years without fire, these patches succeed to LDM at a rate of 0.8 per timestep.

\paragraph{Wildfire Transition} High mortality wildfire (9\% of fires in this seral stage) recycles the patch through the Early Development seral stage. Low mortality wildfire (91\%) maintains the patch in LDO.

\noindent\hrulefill

\paragraph{Late Development - Moderate Canopy Cover (LDM)}

\paragraph{Description} Closed stands of large trees, primarily \emph{P. monticola}, \emph{A. magnifica}, and \emph{P. jeffreyi}. Forests in this landcover type rarely exceed 80\% canopy closure even in closed, dense seral stages. Canopy cover exceeds 70\% (LandFire 2007, Estes 2013).

\paragraph{Succession Transition} Patches in this seral stage will maintain under low mortality disturbance, but after 15 years without fire, these patches succeed to LDC at a rate of 0.8 per timestep.

\paragraph{Wildfire Transition} High mortality wildfire (12\% of fires in this seral stage) recycles the patch through the ED seral stage. Low mortality wildfire (82\%) opens the patch up to LDO 40\% of the time; otherwise, the patch remains in LDC.

\noindent\hrulefill

\paragraph{Late Development - Closed Canopy Cover (LDC)}

\paragraph{Description} Closed stands of large trees, primarily \emph{P. monticola}, \emph{A. magnifica}, and \emph{P. jeffreyi}. Forests in this landcover type rarely exceed 80\% canopy closure even in closed, dense conditions. Canopy cover exceeds 40\% (LandFire 2007, Estes 2013).

\paragraph{Succession Transition} Patches in this seral stage will maintain in the absence of disturbance.

\paragraph{Wildfire Transition} High mortality wildfire (17\% of fires in this seral stage) recycles the patch through the ED seral stage. Low mortality wildfire (83\%) opens the patch up to LDM 80\% of the time; otherwise, the patch remains in LDC.

\noindent\hrulefill




\subsection*{Seral Stage Classification}
\begin{table}[hbp]
\footnotesize
\centering
\caption{Classification of seral stage for WWP. Diameter at Breast Height (DBH) and Cover From Above (CFA) values taken from EVeg polygons. DBH categories are: null, 0-0.9'', 1-4.9'', 5-9.9'', 10-19.9'', 20-29.9'', 30''+. CFA categories are null, 0-10\%, 10-20\%, \dots , 90-100\%. Each row in the table below should be read with a boolean AND across each column.}
\label{wwp_classification}
\begin{tabular}{@{}lrrrrr@{}}
\toprule
\textbf{\begin{tabular}[l]{@{}l@{}}Cover \\ Condition\end{tabular}} & \textbf{\begin{tabular}[r]{@{}r@{}}Overstory Tree \\ Diameter 1 \\ (DBH)\end{tabular}} & \textbf{\begin{tabular}[r]{@{}r@{}}Overstory Tree \\ Diameter 2 \\ (DBH)\end{tabular}} & \textbf{\begin{tabular}[r]{@{}r@{}}Total Tree\\ CFA (\%)\end{tabular}} & \textbf{\begin{tabular}[r]{@{}r@{}}Conifer \\ CFA (\%)\end{tabular}} & \textbf{\begin{tabular}[r]{@{}r@{}}Hardwood \\ CFA (\%)\end{tabular}} \\ \midrule
Early All        & 0-4.9''         & any & any    & any    & any \\
Mid Open         & 5-19.9''        & any & 0-40   & any    & any \\
Mid Moderate     & 5-19.9''        & any & 40-70  & any    & any \\
Mid Closed       & 5-19.9''        & any & null   & 70-100 & any \\
Mid Closed       & 5-19.9''        & any & 70-100 & any    & any \\
Late Open        & 20''+           & any & 0-40   & any    & any \\
Late Moderate    & 20''+           & any & 40-70  & any    & any \\
Late Closed      & 20''+           & any & 70-100 & any    & any \\ \bottomrule
\end{tabular}
\end{table}



\clearpage

\subsection*{References}

\begin{hangparas}{.25in}{1} 
\interlinepenalty=10000
Estes, Becky. Central Sierra Province Ecologist, USDA Forest Service. Personal communication, 3 September 2013.

Griffith, Randy Scott. 1993. ``Pinus monticola.'' In: \emph{Fire Effects Information System}, [Online].  U.S. Department of Agriculture, Forest Service,  Rocky Mountain Research Station, Fire Sciences Laboratory (Producer).  \burl{http://www.fs.fed.us/database/feis/}. [Accessed 4 December 2012].

Fites-Kaufman, Jo Ann, Phil Rundel, Nathan Stephenson, and Dave A. Wixelman. ``Montane and Subalpine Vegetation of the Sierra Nevada and Cascade Ranges.'' In \emph{Terrestrial Vegetation of California, 3rd Edition}, edited by Michael Barbour, Todd Keeler-Wolf, and Allan A. Schoenherr, 456-501. Berkeley and Los Angeles: University of California Press, 2007. 

LandFire. ``Biophysical Setting Models.'' Biophysical Setting 0711720: Sierran-Intermontane Desert Western White Pine-White Fir Woodland. 2007. LANDFIRE Project, U.S. Department of Agriculture, Forest Service; U.S. Department of the Interior. \burl{http://www.landfire.gov/national_veg_models_op2.php}. Accessed 30 November 2012.

Skinner, Carl N. and Chi-Ru Chang. ``Fire Regimes, Past and Present.'' \emph{Sierra Nevada Ecosystem Project: Final report to Congress, vol. II, Assessments and scientific basis for management options}. Davis: University of California, Centers for Water and Wildland Resources, 1996.

Van de Water, Kip M. and Hugh D. Safford. ``A Summary of Fire Frequency Estimates for California Vegetation Before Euro-American Settlement.'' \emph{Fire Ecology} 7.3 (2011): 26-57. doi: 10.4996/fireecology.0703026.

\end{hangparas}


% !TEX root = master.tex
\newpage
\section{Yellow Pine (YPN)}
\label{ypn-description}

\subsection*{General Information}

\subsubsection*{Cover Type Overview}

\textbf{Yellow Pine (YPN)}
\newline
\textbf{Crosswalks}
\begin{itemize}
	\item EVeg: Regional Dominance Type 1
	\begin{itemize}
		\item Eastside Pine
		\item Jeffrey Pine
		\item Ponderosa Pine
	\end{itemize}

	\item LandFire BpS Model
	\begin{itemize}
		\item Yellow Pine
	\end{itemize}

	\item Presettlement Fire Regime Type
	\begin{itemize}
		\item 0610310 California Montane Jeffrey Pine (-Ponderosa Pine) Woodland
	\end{itemize}

	\item Only occurs on the east side of the Sierra crest.
\end{itemize}

\noindent \textbf{Yellow Pine with Aspen (YPN-ASP)}
\newline
This type is created by overlaying the NRIS TERRA Inventory of Aspen on top of the EVeg layer. Where it intersects with YPN it is assigned to YPN-ASP.

\noindent Reviewed by Hugh Safford, Regional Ecologist, USDA Forest Service

\subsubsection*{Vegetation Description}
\textbf{Yellow Pine (YPN)}	This landcover type is characterized by yellow pine species such as \emph{Pinus ponderosa} or \emph{Pinus jeffreyi} that occur on the east side of the Sierra crest (LandFire 2007a). Relatively pure stands of yellow pine may occur, or they may mix with other tree species including \emph{Abies concolor, Juniperus occidentalis, Pinus contorta} ssp. \emph{murrayana}, and \emph{Quercus kelloggi} (Fites-Kaufman et al. 2007, Fitzhugh 1988). Their understory may include both montane forest and Great Basin shrubs, including but not limited to \emph{Ceanothus, Arctostaphylos, Symphoricarpos, Artemisia tridentata, Purshia tridentata, Ericameria nauseosa, Cercocarpus}, and \emph{Holodiscus}. Herbaceous plants and grasses may include \emph{Wyethia, Balsamorhiza sagittata, Festuca, Calamagrostis}, and \emph{Elymus} (LandFire 2007a, Fitzhugh 1988).

Without disturbance, except for naturally occurring fire, a mosaic of uneven-aged patches develops, with open spaces and dense sapling stands (Safford 2013). \emph{Q. kelloggi} or \emph{Juniperus occidentalis} may form an understory, but pure stands of pine also are found. An open stand of low shrubs, and a grassy herb layer are typical. Crowns of pines are open, allowing light, wind and rain to penetrate, whereas other associated trees provide more dense foliage (Fitzhugh 1988).

\textbf{Yellow Pine with Aspen (YPN-ASP)} These are upland forests and woodlands dominated by \emph{Populus tremuloides} without a significant conifer component, often termed ``stable aspen.'' The understory structure may be complex with multiple shrub and herbaceous layers, or simple with just an herbaceous layer. The herbaceous layer may be dense or sparse, dominated by graminoids or forbs. Common shrubs include \emph{Acer, Amelanchier, Artemisia, Juniperus, Prunus, Rosa, Shepherdia, Symphoricarpos}, and the dwarf-shrubs \emph{Mahonia} and \emph{Vaccinium}. Common graminoids may include \emph{Bromus, Calamagrostis, Carex, Elymus, Festuca}, and \emph{Hesperostipa}. Associated forbs may include \emph{Achillea, Eucephalus, Delphinium, Geranium, Heracleum, Ligusticum, Lupinus, Osmorhiza, Pteridium, Rudbeckia, Thalictrum, Valeriana, Wyethia}, and many others (LandFire 2007b).


\subsubsection*{Distribution}
\textbf{Yellow Pine}	This landcover type occurs on all aspects from about 1200 m to 1980 m (4000-6500 ft) in elevation, east of the Sierra Nevada crest (Fitzhugh 1988). It is usually found on volcanic and granitic substrates, in shallow soils with a frigid soil temperature regime (LandFire 2007a).

\textbf{Yellow Pine with Aspen}	Sites supporting \emph{P. tremuloides} are associated with added soil moisture, i.e., azonal wet sites. These sites are often close to streams, lakes, and meadows. Other sites include rock reservoirs, springs and seeps. Terrain can be simple to complex. At lower elevations, topographic conditions for this type tends toward positions resulting in relatively colder, wetter conditions within the prevailing climate, e.g., ravines, north slopes, wet depressions, etc. (LandFire 2007b). \emph{P. tremuloides} stands may also be associated with lateral or terminal moraine boulder material, talus-colluvium, rock falls, or lava flows. In addition, pure stands may be found in topographic positions where snow accumulates, mostly at higher north facing elevations, where snow presence means the growing season is too short to support conifers (Shepperd et al. 2006). 


\subsection*{Disturbances}

\subsubsection*{Wildfire}
\textbf{Yellow Pine} Wildfires are common and frequent; mortality depends on vegetation vulnerability and wildfire intensity. Low mortality fires kill small trees and consume above-ground portions of shrubs and herbs, but do not kill large trees or below-ground organs of most shrubs and herbs which promptly re-sprout. High mortality fires kill large as well as small trees, and may kill many of the shrubs and herbs as well. Fire kills the above-ground portions of the shrubs and herbs, but most shrubs and herbs resprout from surviving below-ground organs. Wildfires may trigger transitions between developmental seral stages.

The relatively long needles of yellow pines and relatively open structure of theses stands make for dry surface and ground fuels that burn readily. Thus, fires in these stands burn more frequently than those in adjacent forests (Fites-Kaufman et al. 2007). In fact, fire is an integral part of the ecology of yellow pines. Fire has allowed yellow pines to dominate sites where it is the potential climax as well as sites where it would otherwise be seral to more shade-tolerant tree species. \emph{P. ponderosa} and \emph{P. jeffreyi} have evolved with a thick bark and open crown structure that allows them to survive most fires. Mature trees will self-prune, leaving a smooth bole which reduces aerial fire spread. Also, fire creates favorable seedbeds for seedling establishment (Habeck 1992). 

\medskip
\noindent \textbf{Yellow Pine with Aspen}	Sites supporting \emph{P. tremuloides} are maintained by stand-replacing disturbances that allow regeneration from below-ground suckers. Replacement fire and ground fire are thought to have been common in stable \emph{P. tremuloides} stands historically. Because \emph{P. tremuloides} is associated with mesic conditions, it rarely burns during the normal lightning season. However, during years with little precipitation stands may be more susceptible to burning. Evidence from fire scars and historical studies show that past fires occurred mostly during the spring and fall. These are typically self-perpetuating stands (LandFire 2007b)

Van de Water and Safford (2011) found a mean fire return interval of 19 years, median of 20 years, mean min interval of 10 years and mean max of 90 years for Aspen. The LandFire model for northern Sierra Nevada ``stable aspen'' predicts a mean FRI of 31 years. Replacement FRI has a mean of 68 years with a range of 50-300 years, while mixed severity FRI has a mean of 57 years with a range of 20-60 years, and low severity fire is not modeled (LandFire 2007b). We recalculated these numbers using seral stage-specific information and using only high and low mortality fire categories, which resulted in an interval of 38 years for high mortality fire, 111 years for low mortality fire, and 29 years for any fire.

Estimates of fire rotations for these variants are available from the LandFire project and a few review papers. The LandFire project’s published fire return intervals are based on a series of associated models created using the Vegetation Dynamics Development Tool (VDDT). In VDDT, fires are specified concurrently with the transition that follows them. For example, a replacement fire causes a transition to the early development stage. In the RMLands model, such fires are classified as high mortality. However, in VDDT mixed severity fires may cause a transition to early development, a transition to a more open seral stage, or no transition at all. In this case, we categorize the first example as a high mortality fire, and the second and third examples as a low mortality fire. Based on this approach, we calculated fire rotations and the probability of high mortality fire for each of the YPN and YPN-ASP seral stages (Tables~\ref{tab:ypndesc_fire} and \ref{tab:ypn-aspdesc_fire}). We computed overall target fire rotations based on values from Mallek et al. (2013) and Van de Water and Safford (2011). 




\begin{table}[!htbp]
\footnotesize
\centering
\caption{Fire rotation index values and probability of high severity fire (at least 75\% overstory tree mortality) probabilities for Yellow Pine. The seral stage that is most susceptible to fire (i.e., has the lowest predicted fire rotation) has a fire rotation index value of 1. Higher values correspond with lower susceptibility to wildfire. The values are relative only within an individual seral stage and should not be compared against other land cover types. Values were derived from VDDT model 0610581 (LandFire 2007), Mallek et al. (2013), and Safford (personal communication). }
\label{tab:ypndesc_fire}
\begin{tabular}{@{}lcc@{}}
\toprule
 \textbf{Seral Stage}    & \textbf{\begin{tabular}[c]{@{}c@{}}Fire Rotation \\ Index\end{tabular}} & \textbf{\begin{tabular}[c]{@{}c@{}}Probability of \\ High Severity Fire\end{tabular}} \\ \hline
Early (All)     		 & 3.8            & 1           \\
Mid--Closed    			 & 1.4            & 0.26        \\
Mid--Moderate  			 & 1.2            & 0.14        \\
Mid--Open      			 & 1.0             & 0.05        \\
Late--Closed   			 & 1.9            & 0.20        \\
Late--Moderate 			 & 1.3             & 0.08        \\
Late--Open     			 & 1.0             & 0.01     	\\ 
\emph{Target Fire Rotation}    			& \emph{21 years}  &   \\ 
\bottomrule
\end{tabular}
\end{table}

\begin{table}[!htbp]
\footnotesize
\centering
\caption{Fire rotation index values and probability of high severity fire (at least 75\% overstory tree mortality) probabilities for Yellow Pine - Aspen type. The seral stage that is most susceptible to fire (i.e., has the lowest predicted fire rotation) has a fire rotation index value of 1. Higher values correspond with lower susceptibility to wildfire. The values are relative only within an individual seral stage and should not be compared against other land cover types. Values were derived from VDDT model 0610110 (LandFire 2007) and Safford (personal communication).}
\label{tab:ypn-aspdesc_fire}
\begin{tabular}{@{}lcc@{}}
\toprule
 \textbf{Seral Stage}    & \textbf{\begin{tabular}[c]{@{}c@{}}Fire Rotation \\ Index\end{tabular}} & \textbf{\begin{tabular}[c]{@{}c@{}}Probability of \\ High Severity Fire\end{tabular}} \\ \hline
Early--Aspen        & 2.9            & 1           \\
Mid--Aspen          & 1.1            & 0.26        \\
Late--Conifer-Aspen & 1.0            & 0.08  		 \\ 
\emph{Target Fire Rotation}    			& \emph{21 years}  &   \\ 
\bottomrule
\end{tabular}
\end{table}

\subsubsection*{Other Disturbance}
Other disturbances are not currently modeled, but may, depending on the seral stage affected and mortality levels, reset patches to early development, maintain existing seral stages, or shift/accelerate succession to a more open seral stage. 

\subsection*{Vegetation Seral Stages}
We recognize seven separate seral stages for YPN: Early Development (ED), Mid Development - Open Canopy Cover (MDO), Mid Development - Moderate Canopy Cover, Mid Development - Closed Canopy Cover (MDC), Late Development - Open Canopy Cover (LDO), Late Development - Moderate Canopy Cover (LDM), and Late Development - Closed Canopy Cover (LDC) (Figure~\ref{transmodel_ypn}).  The YPN -ASP variant is assigned to three seral stages: Early Development - Aspen (ED-A), Mid Development - Aspen (MD-A), and Late Development - Conifer with Aspen (LD-CA) (Figure~\ref{transmodel_ypn-asp}).

Our seral stages are an alternative to ``successional'' classes that imply a linear progression of states and tend not to incorporate disturbance. The seral stages identified here are derived from a combination of successional processes and anthropogenic and natural disturbance, and are intended to represent a composition and structural condition that can be arrived at from multiple other conditions described for that landcover type. Thus our seral stages incorporate age, size, canopy cover, and vegetation composition. In general, the delineation of stages has originated from the LandFire biophysical setting model descriptive of a given landcover type; however, seral stages are not necessarily identical to the classes identified in those models.

\begin{figure}[htbp]
\centering
\includegraphics[width=0.8\textwidth]{/Users/mmallek/Documents/Thesis/statetransmodel/StateTransitionModel/7class.png}
\caption{State and Transition Model for Yellow Pine Forest and Woodland (not inclusive of the aspen variant). Each dark grey box represents one of the seven seral stages for this landcover type. Each column of boxes represents a stage of development: early, middle, and late. Each row of boxes represents a different level of canopy cover: closed (70-100\%), moderate (40-70\%), and open (0-40\%). Transitions between states/seral stages may occur as a result of high mortality fire, low mortality fire, or succession. Specific pathways for each are denoted by the appropriate color line and arrow: red lines relate to high mortality fire, orange lines relate to low mortality fire, and green lines relate to natural succession.} 
\label{transmodel_ypn}
\end{figure}

\begin{figure}[htbp]
\centering
\includegraphics[width=0.8\textwidth]{/Users/mmallek/Documents/Thesis/statetransmodel/StateTransitionModel/3class-asp.png}
\caption{State and Transition Model for Yellow Pine Forest and Woodland, Aspen variant. Each dark grey box represents one of the three seral stages for this landcover type. Three seral stages of development are represented: early, middle, and late. Transitions between states/seral stages may occur as a result of high mortality fire, low mortality fire, or succession. Specific pathways for each are denoted by the appropriate color line and arrow: red lines relate to high mortality fire, orange lines relate to low mortality fire, and green lines relate to natural succession.} 
\label{transmodel_ypn-asp}
\end{figure}

\paragraph*{Early Development (ED)}

\paragraph*{Description} Grasses, forbs, low shrubs, and sparse to moderate cover of trees (primarily \emph{P. ponderosa} or \emph{P. jeffreyi}) seedlings/saplings with an open canopy. This seral stage is characterized by the recruitment of a new cohort of early successional, shade-intolerant tree species into an open area created by a stand-replacing disturbance. Following such disturbance, some sites are dominated by dense shrub stands composed of \emph{P. tridentata}, \emph{Arctostaphylos}, and/or \emph{Ceanothus}, depending on location. Other postfire sites are more open and dominated by dense pine seedlings, bunchgrasses and forbs.

\paragraph*{Succession Transition} In the absence of disturbance, patches in this seral stage will begin transitioning to MDC or MDO after 40 years at a rate of 0.7 per timestep. The transition to MDO is twice as likely as transition to MDC.  At 80 years, all remaining patches will succeed to either MDC or MDO. 

\paragraph*{Wildfire Transition} High mortality wildfire (100\% of fires in this seral stage) recycles the patch through the Early Development seral stage. Low mortality wildfire is not modeled for this seral stage. 

\noindent\hrulefill


\paragraph*{Mid Development - Open Canopy Cover (MDO)}

\paragraph*{Description} Open mid-development forest with diverse herbaceous understory and scattered woody shrubs. Conifers, primarily \emph{P. ponderosa} or \emph{P. jeffreyi}, are medium sized. Herbs and other species gradually decline as growing trees begin to shade understory. Maintained by frequent burning. Canopy cover is less than 40\% (LandFire 2007a). 

\paragraph*{Succession Transition} Patches in this seral stage will maintain under low mortality disturbance, but after 20 years without fire they begin transitioning to MDM at a rate of 0.8 per time step. Succession to LDO occurs once the patch has been in mid development for 170 years. The rate of succession per time step is 0.4. After 230 years, all patches will have succeeded.

\paragraph*{Wildfire Transition} High mortality wildfire (5\% of fires in this seral stage) recycles the patch through the Early Development seral stage. Low mortality wildfire (95\%) maintains the patch in MDO.

\noindent\hrulefill

\paragraph*{Mid Development - Moderate Canopy Cover (MDM)}

\paragraph*{Description} Mid-development forest with moderate canopy cover. Somewhat ``overstocked'' pole to large pole size trees, primarily \emph{P. ponderosa} or \emph{P. jeffreyi}, susceptible to stagnation. Marginal understory associated with limited site resources. Develops where fire frequency is too low to thin small trees. Canopy cover is 40-70\% (LandFire 2007a).

\paragraph*{Succession Transition} Patches in this seral stage may maintain under low mortality disturbance, but after 20 years without fire they begin transitioning to MDC at a rate of 0.8 per time step. At 130 years since succession to a mid development seral stage, these patches will begin transitioning to LDC. The rate of succession per time step is 0.3. After 230 years, all patches will have succeeded.

\paragraph*{Wildfire Transition} High mortality wildfire (14\% of fires in this seral stage) recycles the patch through the Early Development seral stage. Low mortality wildfire (86\%) opens the stand up to MDO 32\% of the time; otherwise, the patch remains in MDC.

\noindent\hrulefill

\paragraph*{Mid Development - Closed Canopy Cover (MDC)}

\paragraph*{Description} Dense mid-development forest. ``Overstocked'' pole to large pole size trees, primarily \emph{P. ponderosa} or \emph{P. jeffreyi}, susceptible to stagnation. Marginal understory associated with limited site resources. Develops where fire frequency is too low to thin small trees. Canopy cover is over 70\% (LandFire 2007a).

\paragraph*{Succession Transition} At 100 years since succession to a mid development seral stage, these patches will begin transitioning to LDC. The rate of succession per time step is 0.2. After 200 years, all patches will have succeeded.

\paragraph*{Wildfire Transition} High mortality wildfire (26\% of fires in this seral stage) recycles the patch through the Early Development seral stage. Low mortality wildfire (74\%) opens the stand up to MDM 60\% of the time; otherwise, the patch remains in MDC.

\noindent\hrulefill


\paragraph*{Late Development - Open Canopy Cover (LDO)}

\paragraph*{Description} Open late-development forest with large and very large trees, primarily \emph{P. ponderosa} or \emph{P. jeffreyi}. Trees grow in often widely spaced clumps and the understory is open and often diverse. Surface fuels are limited due to frequent burning. Canopy cover is less than 40\% (LandFire 2007a, Safford 2013).

\paragraph*{Succession Transition} Patches in this seral stage will maintain under low mortality disturbance, but after 25 years without fire, these patches succeed to LDM at a rate of 0.7 per timestep.

\paragraph*{Wildfire Transition} High mortality wildfire (1\% of fires in this seral stage) recycles the patch through the Early Development seral stage. Low mortality wildfire (99\%) maintains the patch in LDO.

\noindent\hrulefill

\paragraph*{Late Development - Moderate Canopy Cover (LDM)}

\paragraph*{Description} Open late-development forest with large and very large trees, primarily \emph{P. ponderosa} or \emph{P. jeffreyi}. Trees grow in often widely spaced clumps, although they are becoming more dense, and the understory is fairly open and often diverse. Surface fuels are accumulating. Canopy cover is 40-70\% (LandFire 2007a, Safford 2013).

\paragraph*{Succession Transition} Patches in this seral stage may maintain under low mortality disturbance, but after 25 years without fire, these patches succeed to LDC at a rate of 0.7 per timestep.

\paragraph*{Wildfire Transition} High mortality wildfire (8\% of fires in this seral stage) recycles the patch through the Early Development seral stage. Low mortality wildfire (92\%) opens the stand up to LDO 18\% of the time; otherwise, the patch remains in LDM.

\noindent\hrulefill

\paragraph*{Late Development - Closed Canopy Cover (LDC)}

\paragraph*{Description} Dense late-development forest, primarily \emph{P. ponderosa} or \emph{P. jeffreyi} with large and very large trees, sometimes with significant within-stand mortality. Substantial surface fuel accumulation and ladder fuels. Canopy cover exceeds 70\% (LandFire 2007a).

\paragraph*{Succession Transition} Patches in this seral stage will maintain in the absence of disturbance.

\paragraph*{Wildfire Transition} High mortality wildfire (20\% of fires in this seral stage) recycles the patch through the Early Development seral stage. Low mortality wildfire (80\%) opens the stand up to LDM 58\% of the time; otherwise, the patch remains in LDC.

\noindent\hrulefill
\noindent\hrulefill

\subsubsection*{Aspen Variant}

\paragraph*{Early Development - Aspen (ED-A)}

\paragraph*{Description} Grasses, forbs, low shrubs, and sparse to moderate cover of tree seedlings/saplings (primarily \emph{P. tremuloides}) with an open canopy. This seral stage is characterized by the recruitment of a new cohort of early successional, shade-intolerant tree species into an open area created by a stand-replacing disturbance. 

Following disturbance, succession proceeds rapidly from an herbaceous layer to shrubs and trees, which invade together (Verner 1988). \emph{P. tremuloides} suckers over 6ft tall develop within about 10 years (LandFire 2007b). 

\paragraph*{Succession Transition} Unless it burns, a patch in the early seral stage persists for 10 years, at which point it transitions to MD-A.

\paragraph*{Wildfire Transition} High mortality wildfire (100\% of fires in this seral stage) recycles the patch through the ED-A seral stage. Low mortality wildfire is not modeled for this seral stage.

\noindent\hrulefill


\paragraph*{Mid Development - Aspen (MD-A)}

\paragraph*{Description} \emph{P. tremuloides} trees 5-16'' DBH. Canopy cover is highly variable, and can range from 40-100\%. These patches range in age from 10 to 110 years. (LandFire 2007b).

\paragraph*{Succession Transition} Patches in the MD-A seral stage persist for at least 80 years in the absence of any fire, after which they begin transitioning to LD-CA at a rate of 0.6 per timestep. After 130 years without fire all remaining MD-A patches transition to LD-CA. 

\paragraph*{Wildfire Transition} High mortality wildfire (26\% of fires in this seral stage) recycles the patch through the ED-A seral stage. No transition occurs as a result of low mortality fire (74\%).

\noindent\hrulefill


\paragraph*{Late Development - Aspen with Conifer (LD-AC)}

\paragraph*{Description} If stands are sufficiently protected from fire such that conifer species overtop \emph{P. tremuloides} and become large, they may be able to withstand some fire that more sensitive \emph{P. tremuloides} cannot. When this occurs, it creates a patch characterized by late development conifers, such as \emph{P. contorta} ssp. \emph{murrayana}, and early seral \emph{P. tremuloides}. 

\paragraph*{Succession Transition} LD-CA persists for 70 years in the absence of any fire, at which point patches transition to LDC. 

\paragraph*{Wildfire Transition} High mortality wildfire (13\% of fires in this seral stage) returns the patch to ED-A. Low mortality wildfire (87\%) maintains the stand in LD-CA. 

\noindent\hrulefill




\subsection*{Seral Stage Classification}
\begin{table}[hbp]
\footnotesize
\centering
\caption{Classification of seral stage for YPN. Diameter at Breast Height (DBH) and Cover From Above (CFA) values taken from EVeg polygons. DBH categories are: null, 0-0.9'', 1-4.9'', 5-9.9'', 10-19.9'', 20-29.9'', 30''+. CFA categories are null, 0-10\%, 10-20\%, \dots , 90-100\%. Each row in the table below should be read with a boolean AND across each column.}
\label{ypn_classification}
\begin{tabular}{@{}lrrrrr@{}}
\toprule
\textbf{\begin{tabular}[l]{@{}l@{}}Cover \\ Condition\end{tabular}} & \textbf{\begin{tabular}[r]{@{}r@{}}Overstory Tree \\ Diameter 1 \\ (DBH)\end{tabular}} & \textbf{\begin{tabular}[r]{@{}r@{}}Overstory Tree \\ Diameter 2 \\ (DBH)\end{tabular}} & \textbf{\begin{tabular}[r]{@{}r@{}}Total Tree\\ CFA (\%)\end{tabular}} & \textbf{\begin{tabular}[r]{@{}r@{}}Conifer \\ CFA (\%)\end{tabular}} & \textbf{\begin{tabular}[r]{@{}r@{}}Hardwood \\ CFA (\%)\end{tabular}} \\ \midrule
Early All        & 0-4.9''         & any & any    & any & any \\
Mid Open         & 5-19.9''        & any & 0-40   & any & any \\
Mid Moderate     & 5-19.9''        & any & 40-70  & any & any \\
Mid Closed       & 5-19.9''        & any & 70-100 & any & any \\
Late Open        & 20-40''+        & any & 0-40   & any & any \\
Late Moderate    & 20-40''+        & any & 40-70  & any & any \\
Late Closed      & 20-40''+        & any & 70-100 & any & any \\ \bottomrule
\end{tabular}
\end{table}

YPN-ASP seral stages were assigned manually using NAIP 2010 Color IR imagery to assess seral stage.



\clearpage

\subsection*{References}

\begin{hangparas}{.25in}{1} 
\interlinepenalty=10000

Fitzhugh, E. Lee. ``Eastside Pine (EPN).'' \emph{A Guide to Wildlife Habitats of California}, edited by Kenneth E. Mayer and William F. Laudenslayer. California Deparment of Fish and Game, 1988. \burl{http://www.dfg.ca.gov/biogeodata/cwhr/pdfs/EPN.pdf}. Accessed 4 December 2012.

Habeck, R. J. ``Pinus ponderosa var. ponderosa.'' \emph{Fire Effects Information System}, U.S. Department of Agriculture, Forest Service,  Rocky Mountain Research Station, Fire Sciences Laboratory, 1992. \burl{http://www.fs.fed.us/database/feis/plants/tree/quekel/all.html}. Accessed 21 December 2012.

LandFire. ``Biophysical Setting Models.'' Biophysical Setting 0610310: California Montane Jeffrey Pine (-Ponderosa Pine) Woodland. 2007a. LANDFIRE Project, U.S. Department of Agriculture, Forest Service; U.S. Department of the Interior. \burl{http://www.landfire.gov/national_veg_models_op2.php}. Accessed 9 November 2012.

LandFire. ``Biophysical Setting Models.'' Biophysical Setting 0610110: Rocky Mountain Aspen Forest and Woodland. 2007b. LANDFIRE Project, U.S. Department of Agriculture, Forest Service; U.S. Department of the Interior. \burl{http://www.landfire.gov/national_veg_models_op2.php}. Accessed 7 January 2013.

Safford, Hugh S. Personal communications, 5 May 2013, 26 July 2013, 15 August 2013.

Shepperd, Wayn De, Paul C. Rogers, David Burton, and Dale L. Bartos. ``Ecology, Biodiversity, Management, and Restoration of Aspen in the Sierra Nevada.'' General Technical Report RMRS-GTR-178. Rocky Mountain Research Station, Forest Service, U.S. Department of Agriculture, 2006.

Skinner, Carl N. and Chi-Ru Chang. ``Fire Regimes, Past and Present.'' \emph{Sierra Nevada Ecosystem Project: Final report to Congress, vol. II, Assessments and scientific basis for management options}. Davis: University of California, Centers for Water and Wildland Resources, 1996.

Van de Water, Kip M. and Hugh D. Safford. ``A Summary of Fire Frequency Estimates for California Vegetation Before Euro-American Settlement.'' \emph{Fire Ecology} 7.3 (2011): 26-57. doi: 10.4996/fireecology.0703026.

Verner, Jared. ``Aspen (ASP).'' \emph{A Guide to Wildlife Habitats of California}, edited by Kenneth E. Mayer and William F. Laudenslayer. California Deparment of Fish and Game, 1988. \burl{http://www.dfg.ca.gov/biogeodata/cwhr/pdfs/ASP.pdf}. Accessed 4 December 2012.

\end{hangparas}



