% !TEX root = master.tex

\chapter{Conclusion}

Changes to fire-dependent forest ecosystems due to human land use and fire suppression have been a major focus of research in California's northern Sierra Nevada mountains. Changes to the global climate are projected to manifest locally as increases in temperature and precipitation, and as changes in the frequency and duration of drought. Forest Service land managers are directed to inform their decisions with an analysis of the range of variability of ecological processes at multiple scales. Based on these broad themes, I conducted a study of the historical and potential future ranges of variability in wildfire and forest succession in the Yuba River watershed study area.

To explore two different time periods, neither of which can be observed at present, I used simulations to generate many potential historical and future trajectories of wildfires and vegetation response. The landscape disturbance and sucession model software \textsc{RMLands} was used for both cases. To generate the future climate trajectories, I used Palmer Drought Severity Index values calculated by \citet{Cook2014}, which are based on the Representative Concentration Pathway RCP8.5 projections. Model design, parameter selection, and parameter values were informed by a synthesis of empirical and expert knowledge on the disturbance and succession processes characteristic of the pre-settlement period in the ecoregion containing the Yuba River watershed. I described the range of variability for the historical period of 1550-1850 and for the future period of 2010-2100, both qualitatively and quantitatively. My results represent one of the only attempts to quantify and describe both a historical and future range of variability for part of the Sierra Nevada using a landscape disturbance and sucession model.

Results for both the simulated historical and simulated future periods included a wildfire regime characterized by more frequent and extensive fires. Under the future climate trajectories, I observed a small increase in total area burned and a more substantial increase in area burned at high mortality as with increasing climate parameter values in each simulation.. I also compared the current landscape with both ranges of variability, focusing on the Sierran Mixed Conifer - Mesic and Sierran Mixed Conifer - Xeric cover types. Overall, the current study area composition and configuration departs from the historical range of variability not only at the landscape scale, but also at the cover type and seral stage level. I also find that today's landscape is outside the future range of variability, and that this departure increases with increasingly warm and dry climate conditions.

Based on these results, I recommend that managers implement more aggressive restoration efforts that consider both seral stage distributions as well as patch characteritics such as shape complexity and size. Because most future decisions will involve balancing trade-offs between different species, including humans, it will be important to communicate the reasoning behind individual decisions, and critical to articulate the variable effects of fire. The results from these studies can be used to inform goals and individual forest management activities, and provide a mechanism to articulate the contribution of small projects to the larger context of the landscape or Forest. If new vegetation maps are produced in the future, they could be compared to the identified ranges of variability to measure departure after several years or decades of vegetation treatments.


%%%%%%%%%%%% hrv and frv against each other
\section{Historical versus Future Ranges of Variability}
Analysis of the FRV simulations focused on comparing the results from different climate trajectories to one another and to the current conditions. This was done to focus on the FRV study as a separate research effort, but the HRV and FRV are linked by the underlying methodology used in both. The two range of variability analyses used the same parameters, but the FRV study used climate parameters from the pre-European settlement period of 1550--1850, as a reference period \citep{Safford2013}. The combination of results depicting the relationship between the current landscape, HRV, and FRV could be used as one tool for identifying and prioritizing management strategies that would promote resilient forests \citep{Keane2009}. In a preliminary comparison of the seral stage distribution results from the HRV and FRV simulations, I observed that current conditions are generally departed from both the simulated ranges of variability (see Appendix~\ref{app:futurecovcond}). Often the HRV and FRV are comparable, suggesting that management must change in order to restore forests. In these cases, restoration toward the conditions represented by the range of variability analysis results should be evaluated for practicality, with specific implementation being done at a site-specific basis using additional local data to inform specific management actions.

%%%%%%%%%%%% place in the current canon

\section{Comparison to Other Studies}

My results agree qualitatively with several other studies that describe a historical fire regime in the Sierran Nevada characterized by patchy, commonly present but rarely large, high severity fire, within a matrix of mainly lower severity fires \citep{Keeley2000,Hessburg2005,Collins2010,Baker2014}. Comparisons to recent research that attempts to describe the proportion of high severity fire present historically \citep{Mallek2013}, or to the overall forest structure present prior to European settlement \citep{Stephens2015,Baker2014} is difficult because those studies represent an inferred snapshot, as opposed to the range of variability offered in this study. Generally, my results fall in-between other research on various measures. For example, the under 10\% high mortality fire suggested by \citet{Mallek2013} seems far too low for a landscape that included open forest and chaparral stands, but the 39\% implied by \citet{Baker2014} is fairly high for a mean or median value. As fire should lead to particular vegetation patterns, my study offers the benefit of including both as directly related to one another, rather than relying on separate sets of studies. I found large differences between mesic and xeric historical mixed conifer forest structure, which complicates any comparison to the \citet{Baker2014} results, which collapse these, although my results agree with \citet{Baker2014} and \citet{Collins2010} in that the historical forest was much more complex and heterogenous in structure than the present-day forest. The challenge in comparing my results to those of other studies is amplified significantly for the future climate study. No studies currently exist with enough similarities to provide a meaningful comparison.

Despite some technically different results, I largely agree with and came to similar conclusions to existing discussions on how to restore and manage northern Sierran forests. Restoration targets should be defined as a range of variability similar to what I have outined, rather than a fixed value or proportion \citep{Collins2011}. Because climate change is likely to lead to more area burning at high severity, the need to generate early successional forest through management is small, but the need to creatively and carefully manage such patch types will become more and more important in the decades ahead \citep{Collins2010,Littell2012}. Restoration work and planning should take place within the context of an adaptive management framework that promotes reevaluation of success and provides flexibility to managers to respond to actual conditions, while ensuring decisions are made within a landscape-level context.


%%%%%%%%%%%%% from ch2 discussion
%%% current literature



%So a lot of this seems to come down to the scale of analysis and whether fires are described as a unit or as component patches. Only a few people are focusing on the implications, both directions, of forests where a particular fire frequency or severity scenario exists, or where a particular seral stage distribution eixsts, and whether the predicted values for each make sense. One thing my study brings to the table is a much longer period of analysis, which forces a long-term outlook for both vegetation and fire, and can get us away from describing chaparral fire regimes one way and ponderosa fire regimes another way, while ignoring the fact that a transition between those cover types and therefore those fire regimes is part of the system as a whole. It almost feels like the classification is too demanding. Everyone knows and discusses the heterogenous nature of mixed conifer forests, but in trying to analyze them they are simplified too often in ways that erase their heterogeny, which then inhibits the ability of managers to make decisions about how to treat forests at small scales. I think all the researchers are overplaying the applicability of their analyses by just one step. The goal of these studies is to say something about a huge region, but perhaps that is not the role, and we should focus more on describing the region so that management can place actions in that larger context, while not trying to mimic too closely the regional makeup. It is the desire to classify an entire fire or entire regime as 'low severity' when the truth is so much more complicated, that leads us to conclusions about the overall amount of that kind of fire on the landscape and the kind of fire that 'should' be on the landscape and the kind of fire that the public can expect to see on the landscape.




% don't include recommendations


%%%%%%%%%%%%% future research
\section{Areas of Future Research}
Several areas of future research needs emerged over the course of this study. First, the probability of high mortality fire at the seral stage level is fairly sensitive. However, it was difficult to find empirical data that measured high mortality fire occurrence. Where it can be inferred or recorded from historical data such as post-fire chaparral patches or dense stands of saplings, no knowledge exists about the seral stage that preceded the chaparral. What information is available is much more recent (within the past several decades) and is therefore affected by the legacy of timber harvest, fire suppression, and ongoing climate change. Further research measuring the likelihood of high mortality fire under different ecosystem types, topographic position, and drought severity could be used to improve this model.

Second, further model validation and assessment work could be conducted in the future as more data become available. For example, the model predictions may be re-evaluated in several years. Over the long-term, the availability of detailed, spatially explicit vegetation management data and vegetation maps can be utilized to examine the efficacy of \textsc{RMLands} for simulating future landscape dynamics can be evaluated through hindcasting, and make recommendations to changes in the model. In addition, it is possible that climate change will lead to changes in fire behavior over the long term. Further research should be done to determine how fire regimes are changing and whether or not these shifts are occuring within existing or novel ecological communities. Finally, an obvious next step is to simulate both the historical and future time periods after incorporating vegetation management scenarios into the model, in order to evaluate the impact of different land management practices on the wildfire regime and vegetation pattern.

Finally, U.S. Forest Service scientists are actively researching insect outbreaks that affect western forests, including those in the Sierra Nevada \citep{Liebhold2011}. Since I began work on this study in the fall of 2012, the drought in California has facilitated a large increase in bark beetle populations and the tree mortality from them \citep{Fimrite2016}. Aerial detection surveys over the study area show pine beetle there is currently less severe than in the southern Sierra \citep{Moore2015a,Moore2015b}. This is due to the fact that the drought is much worse in that area, implying that if drought becomes more frequent and severe under climate change, the northern Sierra could also be facing more severe pine beetle outbreaks \citep{Moore2015a,Moore2015b,Fimrite2016}. Unlike wildfire, which is a physical process that is fundamentally the same everywhere, even though its effects are incredibly diverse, insects are a biological agent. Thus the life history characteristics of insects, the method individual species use to invade trees and reproduce, and a tree's response to this invasion result in a much more complex disturbance ecology than that of wildfire \citep{Bentz2010}. To examine even the effects of bark beetles, a specific type of insects, is a complex undertaking \citep{Fettig2007}. The effects of climate change, especially increased temperatures and decreased precipitation, may enhance the invasion potential for some insect species in some locations, but may also inhibit it \citep{Logan2003,Bentz2010}. If the influence of insect disturbance increases in the study area during this century, there are sure to be interactive effects with wildfire that would potentially reduce the predictive power of my results \citep{Ferrell1996}.


%*** maybe it would be better to model fire severity on a continuous scale? not sure how you'd implement though.


