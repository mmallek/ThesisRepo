% !TEX root = master.tex

\chapter{Introduction} %Becky says this is the lit review. OK that some things might be methods more typically.
\label{CH1}
\setcitestyle{notesep={;},aysep={}}

\newcommand{\source}[1]{%
  \nobreak\parbox[t]{\linewidth}{\raggedleft #1}% Placing a quote source
}%
\todo{want source on next line down, also do I want to make it only on right half of page or something?}
\begin{quote}
\emph{“In California, vegetation is the meeting place of fire and ecosystems. The plants are the fuel and fire is the driver of vegetation change. Fire and vegetation are often so interactive that they can scarcely be considered separately from each other.”} \\
\medskip
  \source{%
    ---M.G. Barbour, B. Pavlik, F. Drysdale, and S. Lindstrom, 1993 \\ (quoted in \citet{Sugihara2006}) \\
    %
  }
\end{quote}



%\section{Thesis organization and project objectives}

The overall purpose of this project was to quantify the historical range of variability (HRV) in landscape structure in the Yuba River watershed on the Tahoe National Forest and evaluate the relative effects of alternative future climate scenarios on landscape structure. My first objective was to synthesize the empirical and expert knowledge on disturbance and succession processes characteristic of the pre-settlement period in the ecoregion containing the Yuba River watershed. The introduction to this thesis includes a broad introduction to the study area and the motivation for this study. I include a description of the physical and biological geography of the study area and how range of variability analyses are relevant to forest and restoration planning efforts. I introduce the concept of ``range of variability'' and outline a methodology for describing the historical and potential future ranges of variability in the disturbance regime and landscape pattern.

My second objective was to quantify the historical range of variability in the disturbance regime, landscape composition, and landscape configuration in the watershed. In Chapter 2, I provide a detailed presentation of the methodology used in developing the historical range of variability. This includes the development of the input spatial data layers, selection of values for model parameterization, model calibration and execution, and the suite of tools used to conduct the analysis. Results are presented by focusing first on the disturbance regime and second on the vegetation response. A more focused analysis of the two most extensive cover types, Sierran Mixed Conifer - Mesic and Sierran Mixed Conifer - Xeric, which are key components of Sierra Nevada forests and thus a major focus for resource managers \citep{North2010}, are included in the main body of this chapter. However, analysis and interpretation for seven additional cover types is included in Appendix~\ref{app:full-results}. 

My third objective was to quantify a future range of variability by varying the climate parameter value used in the model based on values representing a suite of alternative climate trajectories, while maintaining all other parameters and model methodology used in the HRV analysis. This study comprises Chapter 3, and has been written in the format of a journal article to facilitate eventual submission. Results and analysis focus on the disturbance regime for the landscape as a whole, the seral stage distribution of two key cover types (Sierran Mixed Conifer - Mesic and Sierran Mixed Conifer - Xeric), and the early seral patch configuration for those two cover types. The emphasis is on comparing the future scenario results to the current condition. 

My fourth and final objective was to interpret the results, develop management recommendations, and consider management implications. The dicussion sections of both Chapter 2 and Chapter 3 include my overall assessment of the landscape under the historical and future ranges of variability, a comparison to the current conditions, and a set of management recommendations. 


%%%%%%%%%%%%%%%%%%%%%%%%%%%%%%%%%%%%%%%%%%%%%%%%%%%%%%%%%%%%%%%%%%%%%%%%%%%%%%%%%%%
\section{Study Area}

\subsection{Physical Geography}

\begin{figure}[!htbp]
\includegraphics[width=\textwidth]{/Users/mmallek/Tahoe/Report3/images/studyarea.png}
\caption{The Sierra Nevada Ecoregion is outlined in red. The study area (outlined in black) is located in the northern extent of the Sierra Nevada on the Tahoe National Forest, comprising the Yuba River watershed.}
\label{projectarea}
\end{figure}

The Sierra Nevada is a major North American mountain range and ecological region, located east of California's Central Valley and extending from Fredonyer Pass in the north to southern Kern County in the south. Much of the Sierra Nevada is reserved as federally-held public land, managed by the U.S. Forest Service, the Bureau of Land Management, and the National Park Service. The Plumas and Tahoe National Forests are located in the northern portion of the Sierra Nevada. The study area (see Figure~\ref{projectarea}) is located on the northern part of the Tahoe National Forest, on the Yuba River and Sierraville Ranger Districts, and comprises about 181,550 ha. It is defined by a set of three HUC-5 watersheds\footnote{HUCs are Hydrologic Unit Codes that refer to a nested system of watersheds in the United States, defined by the United States Geological Survey. The HUC-2 watershed scale includes the largest defined watersheds, which are then subdivided down to a smallest level of HUC-16. HUCs are commonly used by agencies like the Forest Service to organize land management.}, the Upper North Yuba River, the Middle Yuba River, and the Lower North Yuba River, all of which are collectively referred to in this document as the Upper Yuba River watershed. 

The topography of the study area consists of rugged mountains incised by two major and a few minor river drainages. Elevation ranges from about 350--2500 m. The area receives 30--260 cm of precipitation annually, most of which falls as snow in the middle to upper elevations \citep{Storer1963}. Like the rest of the Sierra Nevada, the study area has a Mediterranean climate, in which summer drought typically persists from May to September. This annual drought translates into increased importance for the development of a significant snowpack during the winter months, since snowmelt runoff is a key source of soil moisture during the late spring and summer months \citep{Minnich2007,Skinner1996}. In the Sierra Nevada, the heaviest precipitation occurs to the east and north of the San Francisco Bay area \citep{VanWag2006}; my study area is within this region. Datasets of the 30-year normal precipitation at 800 m resolution for the northern Sierra (obtained from the Oregon State PRISM for a side project), illustrate that particularly high amounts of precipitation falls across the middle elevations of the study area compared to the larger region \citep{PRISMClimateGroup2004}. This increased moisture contributes to the occurrence of exceptionally productive patches of forest \citep[][ Alan Doerr, personal communication]{Littell2012}.




%%%%%%%%%%%%%%%%%%%%%%%%%%%%%%%%%%%%%%%%%%%%%%%%%%%%%%%%%%%%%%%%%%%%%%%%%%%%%%%%%%%

\paragraph*{Current Management Context}

The arrival of Europeans in the 1850s sparked a transformation of the landscape as the new settlers harvested timber, extracted gold using hydraulic mining techniques, and suppressed wildfires at ever-increasing scales \citep{Storer1963} (Figure~\ref{figs:historicalphotos}). Today, forestry, mining, grazing, and dozens of recreational activities, including hunting, mountain biking, and hiking, all take place under the multiple-use mandate of the Tahoe National Forest. Grazing allotments also exist within the study area. In addition, 
%231,368 hectares inside of the project area + bufferhave non-Forest Service ownership
%125,637 hectares inside project area only are FS
%project area is 181,550 so 
%percent of project area not in USFS is 0.3079757643
about 30\% of the land inside the study area is not held by the U.S. Forest Service. Many of these lands were privately owned, often by timber companies, before the original forest reserve was created. Land within the reserve was occasionally given to other public or private entities, leading to a ``checkerboard'' pattern of public versus private ownership persists today (Figure~\ref{ownership}). Mining of gold and other minerals also continues. These economic activities affect and interact with ongoing vegetation succession and disturbance processes in the area \citep{USDAForestService2014}.

\begin{figure}[!htbp]
  \centering
  \subfloat[][]{
    \centering
		\includegraphics[height=2in]{/Users/mmallek/Documents/Thesis/Seminar/oldsugarpinerailroad.jpg}
    %\label{fig:covermap}
  } 
  \subfloat[][]{
    \includegraphics[height=2in]{/Users/mmallek/Documents/Thesis/Seminar/oldsheepgrazing.jpg}
   	% \label{fig:conditionmap}
   } \\
  \subfloat[][]{
    \includegraphics[height=2in]{/Users/mmallek/Documents/Thesis/Seminar/historicalfiresuppression.jpg}
    %\label{fig:conditionmap}
   }
     \subfloat[][]{
    \includegraphics[height=2in]{/Users/mmallek/Documents/Thesis/Seminar/retardantUSFS.jpg}
    %\label{fig:conditionmap}
    }
   \caption{(a) Harvested  sugar pine trees awaiting transport by rail to markets. Timber harvest historically focused on the most valuable trees, which in this area tended to be fire resistant species such as sugar pine and ponderosa pine. (b) Hundreds of thousands of sheep were grazed on mountain meadows. Grazing removes fine fuels, altering the fire regime by reducing the frequency and extent of wildfire. (c) Fire suppression to protect natural resources, especially timber, was implemented soon after the arrival of Europeans. (d) After World War II, fire suppression technology became more advanced, and most fires were quickly put out.} 
\label{figs:historicalphotos}
\end{figure}



\begin{figure}[!htbp]
\centering
\includegraphics[width=0.4\textheight]{/Users/mmallek/Tahoe/Report3/images/ownership_resized.png}
\caption{Map of National Forest lands and lands held by other entities (including private, industry, and other public land). Forest lands are in green, with other ownership in tan. The boundary of the study area is in black; this image shows the 10 km buffer.} 
\label{ownership}
\end{figure}

In the western Sierra Nevada, foothill communities and lower elevation oak-conifer woodlands have experience a loss of species diversity, fragmentation, and outright habitat conversion due to the overlap with private lands and population growth. Middle elevation forests were and are more affected by mining and forestry; most easily accessible trees were probably cut before national forests were established \citep{SNEP1996}. The wildfire regime has been significantly altered in hardwoods, yellow pine, and mixed conifer forests \citep{Merriam2013,Safford2013}, and much less so in red fir and subalpine forests \citep{Meyer2013,Meyer2013a}. However, other human activities since the late 1800s have altered the structure of western Sierra Nevada forests, most notably to simplify it in several ways, including a decrease in species, multi-story canopies, and snags \citep{SNEP1996}. These activities are related mainly to timber harvest and to the extensive network of roads constructed to support timber harvest, fire control, and recreation. This simplification of landscape structure may have a negative impact on wildlife and potentially lead to a loss of biodiversity in forests \citep{Thompson2003,Manley2004,Hunter2011}. The inherent heterogeneity of both abiotic and biotic characteristics of the Sierra Nevada complicates efforts to measure the effects of human-caused fragmentation, as Sierran forests tend to be somewhat patchy even in the absence of human alterations \citep{Franklin1996}. In the Pacific Northwest, ``old-growth'' connotes very large blocks of uniformly very old trees. However, in the Sierra Nevada ``old-growth'' indicates not only the presence of very large and old trees, but also a complex, patchy, ``messy'' forest of varying age classes, species, fuel quantities, and vegetation structure \citep{SNEP1996}.

Although many uses of the forest led to changes in vegetation structure and composition, logging and wildfire suppression in combination have altered the historical fire regime and vegetation patterns most significantly \citep{Storer1963,Stephens2015,Knapp2013,Hessburg2005}. These activities persist today. Clearcutting, shelterwood, salvage cutting, and plantation management have been major components of timber management on the Forest \citep{USDAForestService2014}. Between 1988 and 2002, timber sales in the Sierra Nevada as a whole dropped drastically, but on the Tahoe National Forests timber sale levels have fluctuated both up and down (although annual sawtimber sold has decreased similarly to other Sierran Forests) \citep{USDAForestService2004}. Although very large fires have burned in the Sierra Nevada recently (e.g., the 2013 Rim Fire), few large fires have impacted the Tahoe National Forest in the last 100 years \citep{USDAForestService1990}. In 1960, approximately 40,500 ha burned on the Forest. The low total burned acreage is despite fairly high fire starts (both human- and lightning-caused), indicating that suppression efforts have been very successful \citep{USDAForestService1990,calfire2012,usgs-fire-data2012}. The 1999 Pendola Fire burned a total of 4,735 ha. The final fire perimeter included a total of 1,565 ha on the Tahoe and Plumas National Forests \citep{Shaw2009,USDOJ2009}. Post-fire analysis of the burn in the Tahoe National Forest around the Bullards Bar Reservoir and west of Camptonville, CA quantified the total area burned at high severity at 70\%, prompting the need for restoration actions \citep{USDAForestService1999}. 




%\clearpage

%%%%%%%%%%%%%%%%%%%%%%%%%%%%%%%%%%%%%%%%%%%%%%%%%%%%%%%%%%%%%%%%%%%%%%%%%%%%%%%%%%%
\subsection{Disturbance Regime}
\paragraph*{Fire in the Sierra Nevada}
In the Sierra Nevada, cycles of fire and vegetation recovery occur variably over large extents, as well as over long periods of time. Ongoing disturbance results in heterogeneity in vegetation composition and configuration, which can be captured by various statistical metrics \citep{Monica2008}. Prior to European settlement, wildfire was the major source of disturbance in Sierran forests, shaping the composition and configuration of vegetation communities. Fires were primarily lightning-caused, although indigenous peoples are thought to have set fires for vegetation management, especially in the lower elevations \citep{Anderson1996}. 

\begin{figure}[!htbp]
  \centering
  \subfloat[][]{
    \centering
		\includegraphics[height=2in]{/Users/mmallek/Documents/Thesis/Seminar/lowseverityfire-AlanTaylor.jpg}
    \label{fig:lowseverity}
  } 
  \subfloat[][]{
    \includegraphics[height=2in]{/Users/mmallek/Documents/Thesis/Seminar/smokywoodsUSFS.jpg}
    \label{fig:highseverity}
   } 
   \caption{(a) Low severity fires have short flame heights, staying near the ground. Fire primarily consumes fuels and understory vegetation. Photo by Alan Taylor. (b) 2013 Rim Fire burning near a highway. Fire consumes fuels and both understory and overstory vegetation. Photo from USFS Region 5.} 
\label{figs:fireseverity}
\end{figure}

In general, regardless of vegetation type, fires during the pre-settlement period were thought to burn primarily at low intensities (Figure~\ref{fig:lowseverity}). High severity fire (over 75\% overstory canopy mortality) was uncommon \citep{Skinner1996, SNEP1996a,Mallek2013,Stephens2015}. Under this disturbance regime, stand-replacing fire initiated early development conditions on the landscape, but since they were uncommon, most fires only affected the understory by removing fuels \citep{Skinner1996, SNEP1996a,Mallek2013}. In some cases, individual trees or groups of tree could ``torch'' and burn down during a fire that otherwise consumed ground fuels. This is sometimes referred to as a ``moderate severity'' fire \citep{Beaty2001}. In these instances, fire thinned the forest. Trees left alive after fire were more widely spaced. Thus the overall age of the stand, defined as the age of the overstory trees, would be unaffected, but succession would be reset to an extent among the understory vegetation. Fires of moderate severity, that result in more open overstory canopy cover, were more prevalent in more xeric environments, including south-facing slopes and ridges \citep{Mallek2013,Safford2014,SNEP1996a,SNEP1996}. Where fires did not recur frequently or only occurred at very low severity levels, succession processes such as infill or overstory growth led to gradual closing of the overstory tree canopy. It should be noted that closed canopy forests could be even-aged or multi-aged; the term ``closed'' alone does not distinguish fine-scale heterogeneity at a finer-scale than the 30 m pixel used in my model (as explained in Chapter~\ref{sec:hrvmethods}). For most of the cover types in the study area, high severity fire rates were low, allowing stands to succeed into late development and old growth conditions with a variety of canopy structures \citep{Mallek2013,Safford2014,SNEP1996a,SNEP1996}. 


Fire rotations, defined as the time it takes to burn an area equivalent to the total area under study \citep{Agee1993}, have been calculated for the major cover types in the study area \citep{Mallek2013}. Wildfires were frequent, with a mean rotation as short as 20 years in \emph{Pinus ponderosa} (Ponderosa pine)-dominated forests. Wetter mixed conifer areas are predicted to have had a mean fire rotation of 30 years. Fire rotation is thought to increase gradually with elevation. For example, mesic \emph{Abies magnifica} (Red Fir) forests, which exist around 2,000 feet higher in elevation than \emph{P.~ponderosa} forests, had a mean fire rotation of 60 years \citep{Mallek2013}. Thus, under historical conditions, fire rotations increased with increased moisture and elevation. Mean fire rotations do not tell the full story, however, as the variance around a mean fire rotation could be remarkable, with some parts of the forest experiencing fire frequently, while other escaped fire for long periods. 

\begin{wrapfigure}{O}{0.5\textwidth} % use a capital R to allow figure to float
\includegraphics[width=0.5\textwidth]{/Users/mmallek/Documents/Thesis/Seminar/aerialpostfireUSFS.jpg}
\caption{Because fire suppression creates homogeneous forest stands that extend across large areas, when fires escape control they can grow large very quickly, and burn with high intensity over large areas. As a result, the consequence of altering fire regimes has been that instead of fires occurring mostly as low severity with patches of high severity, the opposite take place. During the historical period, the amount of high versus low severity fire in this image probably would have been inverted.} 
\label{fig:bigfirephoto}
\end{wrapfigure} 

Since then, fire suppression, logging, grazing, and mining have all interacted to alter the historical fire regime and vegetation patterns \citep{Stephens2015,Knapp2013}. For the xeric forest types within this landscape, frequent fires (predominantly low and mixed severity) were the norm \citep{Hessburg2005}. After large-scale fire suppression became the norm in the second half of the 19th century, less fire-tolerant species (such as \emph{Pseudotsuga menziesii} (Douglas fir) and \emph{Abies concolor} (white fir)) have come to dominate areas where they were once only one part of the vegetation community. These species have increased in both density and volume. Grazing and development made fires less common by altering or removing the fine fuels that carried fire. Timber harvest, especially of fire-tolerant species such as \emph{P.~ponderosa} and \emph{Pinus lambertiana} (sugar pine), accelerated the increased cover of species such as \emph{A. concolor}. Finally, fire suppression allowed the buildup of medium size fuels and ladder fuels, which promotes larger and hotter fires when they do occur (Figure~\ref{fig:highseverity}). Moreover, the lack of natural fires has meant that variation in fuel loading has decreased, which allows large fires to spread over very large areas \citep{Hessburg2005,Beaty2007,Meyer2008} (Figures~\ref{fig:bigfirephoto}). Consequently, recent research has focused on what options are available to try and prevent the occurrence of large, highly severe fires, and what management actions are appropriate after such events (\citep{Stephens2013,North2010}. Managing for and within the range of variability for a landscape is one potential solution.


%\clearpage

%%%%%%%%%%%%%%%%%%%%%%%%%%%%%%%%%%%%%%%%%%%%%%%%%%%%%%%%%%%%%%%%%%%%%%%%%%%%%%%%%%%
\subsection{Ecology of vegetation systems}
Vegetation in the study area is tremendously diverse and changes slowly along an elevational gradient and in response to local changes in drainage, aspect, and soil structure. Grasslands, chaparral, oak woodlands, mixed conifer forests, and subalpine forests are all found within the study area (Figure~\ref{fig:majorvegtypes}). Many species exhibit fire-adapted traits, such as resprouting from roots after a fire, fire-induced germination, or thick bark \citep{VanWag2006}. In collaboration with USDA Forest Service staff, a system of land cover and seral stage classification based on the LandFire National Vegetation Dynamics Models \citeyearpar{Landfire2007}) and Presettlement Fire Regimes from \citet{VandeWater2011} was developed. I then crosswalked Forest Service corporate spatial data based on the Northern Sierra \textsc{CalVeg} classification to each cover type \cite{USDAForestService2008}. I also considered information from \emph{A Guide to Wildlife Habitats of California}, popularly known as the ``Wildlife Habitat Relationship (WHR)'' cover types \citep{WHR1988}. 

There are 13 major vegetation types, several of which include one or more of the following variants: mesic, xeric, ultramafic, and aspen. In total there are 31 defined cover types. In order to realistically simulate fires within the study area, I buffered it by 10 km in all directions. This allowed space for fires to move in and out of the focal landscape. A few of the 31 cover types only occur within this buffer, but they are still fully defined within the model. I completed analysis on the nine (of 31) cover types that extend across at least 1000 ha of the study area because I had the most confidence in results for cover types at least that well represented. Within this thesis, I focus most of my reported results and discussion on the two most prevalent forest cover types, Sierra Mixed Conifer - Mesic and Sierran Mixed Conifer - Xeric, because these results are statistically reliable and a deeper understanding of the historical and potential future ranges of variability of the mixed conifer belt will be the most useful to managers. Results and analysis for the other cover types were included in reports to the Tahoe National Forest and are included in Appendix \ref{app:full-results}. In this introduction I review the cover types following the ecological zone groupings from \citet{VanWag2006}. Figure~\ref{fig:majorvegtypes} depicts the major vegetation types that undergo succession, plus the six static land cover types, within the study area. Figure~\ref{fig:ecologicalzonebands} illustrates the elevational distribution of ecological zones in the Sierra Nevada more broadly.


\begin{figure}[!htbp]
\centering
\includegraphics[width=\textwidth]{/Users/mmallek/Documents/Thesis/maps/majorvegtypes_shade.pdf}
\caption{The 13 major vegetation types that undergo succession, plus the 6 static land cover types. This map does not distinguish between xeric, mesic, ultramafic, and aspen variants. See Figure~\ref{fig:covermap} for more detail. The black inner boundary is defined by subwatersheds of the Yuba River. The outer boundary is the result of a buffer used in the simulation. In the study area, elevation increases from west to east, and the landscape is incised by three main river canyons, two of which lead to the Pendola reservoir in the southwest part of the map.
}
\label{fig:majorvegtypes}
\end{figure}

\begin{figure}[!htbp]
\centering
\includegraphics[height=0.3\textheight]{/Users/mmallek/Documents/Thesis/Plots/ImagefromFireTextbook.pdf}
\caption{Area of ecological zones by 500 m elevation bands. The elevational distribution of ecological zones is evident as area cover by each zone increases and then decreases as elevation increases. Aspect, soil quality, moisture availability, and elevation all contribute to the exact distribution of individual tree species within these broad vegetation types. Figure from ``Sierra Nevada Ecoregion,'' by Van Wagtendonk and Fites-Kaufman, Chapter 12 in \emph{Fire in California Ecosystems}, p. 269.}
\label{fig:ecologicalzonebands}
\end{figure}

\paragraph*{Foothill shrubland and woodland} This ecological zone lies directly adjacent and to the west of the study area. A small part of the buffer used around the study area includes this zone, which is represented by the Oak Woodland cover type. This type is characterized by savannas, woodlands, or forests of either monospecific or mixed stands of various oak species. \emph{Quercus douglasii}, \emph{Quercus lobata}, \emph{Quercus wislizenii}, and \emph{Quercus garryana} are the major dominants. (Appendix~\ref{oak-description}). 

\paragraph*{Lower Montane forest} This ecological zone includes Oak-Conifer Forests and Woodlands and Mixed Evergreen forests at lower elevations, developing into Sierran Mixed Conifer forests with increasing elevation. All three of these cover types are typified by a combination of both coniferous and broadleaved trees. 

Mixed Evergreen is characterized by dense stands of \emph{Notholithocarpus densiflorus} (Tanoak) and \emph{Arbutus menziesii} complemented by \emph{Pseudotsuga menziesii} on more mesic soils. In xeric sites, conifers are less common, and a hardwood tree layer composed of evergreen oaks such as \emph{Quercus chrysolepis}, \emph{Quercus wislizeni}, \emph{Quercus kelloggi}, and \emph{Quercus garryana} instead dominates. Fires are fairly common in this cover type, but the vegetation quickly recovers, and many of the hardwoods resprout after fire (Appendix~\ref{meg-description}). 

Oak-Conifer Forests and Woodlands are characterized by the conifers \emph{Pinus ponderosa} or \emph{Pinus jeffreyi}, with one or more oaks, such as \emph{Quercus kelloggii}, \emph{Quercus garryana}, \emph{Quercus wislizeni}, or \emph{Quercus chrysolepsis}. Historically, low severity fires were extremely common. Fire is integral to the ecology of the yellow pines and oaks that characterize this system, and this cover type is one of the most altered by fire suppression (Appendix~\ref{ocfw-description}).

Sierran Mixed Conifer forests (Figure~\ref{fig:smctrees}) are characterized by five conifers and one hardwood: \emph{Abies concolor, Pseudotsuga menziesii, P.~ponderosa, P.~lambertiana, Calocedrus decurrens}, and \emph{Quercus kelloggii}. At least three conifers are typically present in any given stand. All of these species can be found in either cover type, but some are more closely associated with either the mesic or xeric variant. The characteristic species of the mesic type, \emph{A.~concolor} and \emph{P.~menziesii}, are less adapted to fire. Species characteristic of the xeric type, \emph{P.~ponderosa}, \emph{P.~lambertiana}, plus \emph{Q.~kelloggii}, are more fire-adapted. \emph{C.~decurrens} is found in both subtypes, but is very rarely dominant. The distrbution of these species is normally an outcome of the variation in the frequency and intensity of wildfire under natural conditions, although alteration of these conditions can affect ther distrbution. \emph{A.~concolor} tends to be the most ubiquitous species, especially on north-facing slopes. \emph{P.~ponderosa} was historically the dominant species, under the previous frequent low severity fire regime. It is still the most prevalent on south slopes and is present continuously from the Oak-Conifer Forest and Woodland belt below it in elevation (Appendix~\ref{smc-description}).


\begin{figure}[!htbp]
\centering
\includegraphics[width=\textwidth]{/Users/mmallek/Documents/Thesis/Seminar/mixedconiferslide.png}
\caption{The Sierran Mixed Conifer mesic and xeric cover types dominate the study landscape, extending across 61\% of it. They are characterized by five conifer and one deciduous oak species. The characteristic species of the mesic type, \emph{Abies concolor} (white fir) and \emph{Pseudotsuga menziesii} (Douglas fir), are less fire-tolerant. Species characteristic of the xeric type, \emph{Pinus ponderosa} (ponderosa pine), \emph{Pinus lambertiana} (sugar pine), plus \emph{Quercus kelloggii} (black oak), are more fire-adapted. \emph{Calocedrus decurrens} (incense cedar) is found in both subtypes, but is very rarely dominant.}
\label{fig:smctrees}
\end{figure}


\paragraph*{Upper Montane forest} The Upper Montane zone is defined by the presence of Red Fir forests \citep{Potter1998}. The Red Fir cover type is dominated by \emph{Abies magnifica}, but other species do co-occur. On mesic sites, \emph{Pinus monticola} and \emph{Pinus contorta} ssp. \emph{murrayana} are also found, while on xeric sites \emph{Abies concolor} and \emph{Pinus jeffreyi} are more common. Red Fir forests are extremely resilient to disturbance. Wildfires were less common in these forests than those of the Lower Montane zone, and stands were characterized by complex patches of even-aged trees within a single stand arising from localized disturbance events. The boundaries between Sierran Mixed Conifer and Red Fir are fuzzy, such that the types overlap with one another at their boundaries. Similarly, Red Fir forests of the Upper Montane zone blend into the lodgepole and subalpine conifers of the Subalpine forest zone (Appendix~\ref{rfr-description}).


\paragraph*{Subalpine forest} I defined two cover types within the subalpine forest. Lodgepole Pine occurs along the lower elevation portion of the zone, usually in wetter soils, such as along meadow edge, although it is also found on rocky soils. Unlike the subspecies of Lodgepole pine found in the Rocky Mountains, \emph{Pinus contorta} ssp.\@ \emph{murrayana} does not have serotinous cones. Wildfires in Lodgepole Pine tend to be high severity and recur at long intervals (Appendix~\ref{lpn-description}).% Fites-Kaufman et al 2007, Landfire). 

Above it, Subalpine Conifer is found, consisting of a mosaic of forest, woodland, meadow, and scrub vegetation. Trees often grow as krummholz forms at the highest elevations. \emph{Tsuga mertensiana} is often the most common tree species and mixes with \emph{Pinus contorta} ssp.\@ \emph{murrayana, Abies magnifica, Pinus monticola}, and \emph{Pinus albicaulis}. Wildfire is very rare in this cover type (Appendix~\ref{scn-description}). 

Western White Pine sometimes occurs in sufficiently continuous patches to be classified as its own type, separate from the Subalpine Conifer group. Typified by \emph{Pinus monticola}, species from the Subalpine Conifer and Red Fir cover types sometimes co-occur as well. This cover type tends to occur on drier soils. Most fires are low severity fires that promote the development of late successional forests (Appendix~\ref{wwp-description}).


\paragraph*{Alpine meadow and shrubland} All lands within the study area are below climatic treeline, and I did not identify alpine meadows and shrublands that are not appropriately considered an early successional stage of Subalpine Conifer.


\paragraph*{Eastside forest and woodland} Although the study area does not include any lands east of the Sierra crest, the buffer around the study area does, and I classified and described the Yellow Pine cover type to capture most of this vegetation community. It is characterized by \emph{Pinus ponderosa} or \emph{Pinus jeffreyi}, but other conifers and oaks, as well as \emph{Juniperus occidentalis} may occur. Under historical conditions, wildfires were extremely common and were almost always low severity; fire is integral to the ecology of yellow pines (Appendix~\ref{ypn-description}). 
%
Oak-Conifer Forests and Woodlands, as described above, are also present (Appendix~\ref{ocfw-description}). 
%
Also east of the crest are two shrub community types: Big Sagebrush (Appendix~\ref{sage-description}), typified by \emph{Artemisia tridentata} and Black and Low Sagebrush (Appendix~\ref{lsg-description}), typified by \emph{Artemisia arbuscula} or \emph{Artemisia nova}. 


\paragraph*{Other cover types and variants} Some cover types not listed above can be found in any zone. I classified riparian vegetation into a Montane Riparian cover type (Appendix~\ref{mrip-description}). 
%
Curl-leaf Mountain Mahogany is most common in the Upper Montane zone and above, but may occur in the Lower Montane zone as well. Typified by \emph{Cercocarpus ledifolius}, it is also a shrub cover type (Appendix~\ref{cmm-description}). 
%
Several of the cover types listed here include ultramafic (a rock type that includes serpentine) variants, which are characterized by discontinuous fuel cover of grasses and low shrubs. The most common tree species to grow on ultramafic sites is \emph{Pinus jeffreyi}. Many endemic species grow on ultramafic soils, which are high in metal concentrations and fairly unproductive (\citealt{OGeen2007}, Appendices~\ref{smc-description}, \ref{ocfw-description}, \ref{rfr-description}, \ref{meg-description}).
%
Finally, I included aspen variants to several cover types. Lodgepole Pine, Red Fir, and Sierran Mixed Conifer all included variants that were seral to conifer forests in the absence of fire to maintain the aspen stand (Appendices~\ref{smc-description}, \ref{rfr-description}, \ref{lpn-description}). Subalpine Conifer and Yellow Pine included variants considered ``stable'' aspen (Appendices~\ref{scn-description}, \ref{ypn-description}). In all cases, the total area classified to aspen is very small. 

%%%%%%%%%%%%%%%%%%%%%%%%%%%%%%%%%%%%%%%%%%%%%%%%%%%%%%%%%%%%%%%%%%%%%%%%%%%%%%%%%%%

\section{Forest Planning}

With the emergence of ecosystem management as an organizing framework for natural resources management in the early 1990s, the need to recognize ecosystems as dynamic and constantly-changing became well accepted, and calls to manage forests sustainably became common \citep{Christensen1996}. Just as timber harvest and fire suppression were included in Forest Service policy, so now is guidance on how to restore forests to more resilient conditions. Within the context of forest and land management planning, the restoration of ecosystems to their pre-European settlement states was incorporated as a goal or desired future condition into various plans, including the Sierra Nevada Ecosystem Project \citep{SNEP1996a}. 

Each National Forest in the country has a Forest Plan, which guides the management of all resources on a National Forest. By 2000, the U.S. Forest Service's formal Planning Rule, which provides direction on the development of these plans, explicitly called for the agency to estimate and describe the range of variability under natural disturbance regimes, and manage for those characteristics (36 CFR \textsection 219 2000). Decisions were to be grounded in the context of ``maintain[ing] or restor[ing] ecological conditions that are similar to the biological and physical range of expected variability'' (36 CFR \textsection 219.4 2000). The need to consider the natural range of variability was maintained through various amendments to the rule, and is still present in the new 2012 rule, finalized in early 2015: ``Plan decisions affecting ecosystem diversity must provide for maintenance or restoration of the characteristics of ecosystem composition and structure within the range of variability that would be expected to occur under natural disturbance regimes of the current climatic period'' (36 CFR \textsection 219 2012). Thus, the Planning Rule instructs land managers to restore and maintain ecosystems characteristic of the conditions that would be expected to occur in the absence of modern humans, and recognizes that ecosystems are dynamic and variable over time.





%%%%%%%%%%%%%%%%%%%%%%%%%%%%%%%%%%%%%%%%%%%%%%%%%%%%%%%%%%%%%%%%%%%%%%%%%%%%%%%%%%%
\section{Range of Variability Analysis}

Historic range of variability (HRV) analysis is a useful paradigm in landscape planning. HRV analysis is intended to help conceptualize the mechanisms behind large-scale ecosystem functions and provide a basis from which to make predictions about how a given ecosystem will react to disturbances in the future \citep{Nonaka2005,Landres1999}. Methods for quantifying the natural range of variability for a diversity of landscapes in the United States augmented the development of research focused on this task \citep{Landres1999}. Of these, simulation of the historical dynamics became fairly popular. By 2004, some 45 landscape fire and succession models alone had been developed \citep{Keane2004}. Many of these, such as \textsc{landis} \citep{He1999}, \textsc{zelig-l} \citep{Miller1999}, \textsc{safe-forests} \cite{Sessions1997} and \textsc{landsum} \citep{Keane2012} are still in use today. Landscape fire and succession models are used to create spatially-explicit simulations of both of these key forest processes, typically outputting a set of GIS layers for each timestep of the model that can then be analyzed to quantify trajectories and patterns in the disturbance regime, seral stage composition, and landscape configuration over time \citep{Keane2004}. A component of many landscape fire and succession models are state and transition models, which form a framework for defining the fundamental vegetation communities and the probabilities over time for transitions from one state to another \citep{Stringham2003,Blankenship2015}.

Although empirical data may sometimes be available on some variables affecting HRV, the time scales and broad spatial extents under study make simulations a logical choice, allowing researchers to incorporate all parameters of interest and ultimately derive a meaningful quantification \citep{Swetnam1999,Mladenoff1999}. Range of variability analyses have been conducted using literature searches exclusively, including within the Sierra Nevada \citep[e.g.,][]{Safford2013}. Results of such analyses depend on the assumption that an aggregation of many small studies is sufficient to address long-term, large-scale questions, and require researchers to accept many unknowns about research methodologies. At the same time, in landscapes severely impacted by European settlement, such as those of the northern Sierra Nevada, it is impossible to observe trajectories in which fire suppression has not influenced landscape structure \citep{Keane2012}. In the absence of consistent and complete data, simulations can be used to incorporate the data that do exist and generate new datasets of otherwise unobservable landscape trajectories. From these new datasets, statistical analyses can be used to describe the landscape quantitatively, and subsequently make inferences about the HRV of an area, as well as compare current conditions to the HRV (Figure~\ref{fig:hrvplot}). The Rocky Mountains and Oregon Coast Range have been the focus of several simulated HRV studies \citep{Keane1996,Tinker2003,McGarigal2005,Nonaka2005,Blankenship2015}, while the only one conducted in the Sierra Nevada took place in Sequoia National Park, in the southern Sierra \citep{Miller1999}. My study area is in the northern Sierra, and is different in its land use history, vegetation, and disturbance regime. Consequently, my study is the first major HRV analysis in the northern Sierra Nevada. 

\begin{figure}[!htbp]
\centering
\includegraphics[width=\textwidth]{/Users/mmallek/Documents/Thesis/Seminar/HRVgraphic.png}
\caption{The range of variability is the dynamic change over time of a given attribute of an ecosystem under study. It represents a resilient prior state because these ecosystems developed and were maintained by a particular set of disturbance regimes and climatic conditions. It is therefore a reasonable target for restoration and maintenance efforts. This figure shows a generic range of variability plot. Any statistical measure derived from the outcome of my simulations may be evaluated under the range of variability framework. The range of variability can be defined for a particular case in various ways: by the green box, by the red horizontal lines, by minimum and maximum all historical data (arrow), etc.
}
\label{fig:hrvplot}
\end{figure}

This thesis includes an analysis of the simulated historical range of variability for the Upper Yuba River Watershed and an analysis of disturbance and succession trends associated with future climate scenarios. I define the HRV as the variation in disturbance processes and landscape composition and configuration over the 300 years prior to European settlement. A primary motivation for this study was to aid in planning on the Tahoe National Forest. The quantitative assessment of the HRV of this landscape provides managers with a statistical, ecosystem-level analysis of the disturbance and succession processes that characterize this portion of the northern Sierra Nevada. Quantification of HRV provides managers with a neutral assessment of the current departure from HRV, which they can use to prioritize certain vegetation types, disturbance processes, or their intersection for restoration or maintenance. Because the simulation captures landscape changes over hundreds of years, far longer than the planning cycle, the results allow managers to ground near-term plans and expectations within a larger context. 

In planning restoration efforts, it is reasonable to use as a reference the last known period during which a dynamic but resilient landscape existed \citep{Swetnam1999}. The arrival of European settlers to the Sierra Nevada led to sweeping ecological changes that now have greatly altered many Sierran landscapes through fire suppression, grazing, road building, timber cutting, recreation, and other activities \citep{Storer1963,Stephens2015,Knapp2013,Hessburg2005}. The period prior to European settlement, then, is a suitable reference condition against which I can compare current landscape structure and dynamics. Moreover, it is frequently used in the western United States as the historical reference period for restoration planning \citep{Safford2013}. It is thought to represent a resilient prior state during which ecosystems developed and were maintained by a particular combination of disturbance regimes and climatic conditions \citep{VandeWater2011,Meyer2013}. The period is also several times the length of rotation periods identified for well-understood cover types within the study area. Finally, it is a time frame for which I have sufficient information to have confidence in model results. 

I am mindful of the fact that this reference period overlaps the ``Little Ice Age,'' which may temper the utility of the results as specific management targets, but does not diminish their usefulness in other ways \citep{Minnich2007,Safford2013}. The chosen reference period was a not time of stasis, climatically, ecologically, or culturally. The oscillation of the Palmer Drought Severity Index, a measure of climate variability in terms of precipitation and temperature over time, illustrates this (see Chapter~\ref{subsubsec:distparams} for a detailed discussion). In addition, multi-year droughts and El Ni\~no/La Ni\~na events also occurred over this time frame \citep{Minnich2007}. Ecologically, my historical period occurred during a very long-term (on the scale of millennia) shift to a warmer and drier climate, with an associated shift toward species more tolerant of such conditions, such as yellow pine species, and away from species like white fir, which prefer more mesic conditions. A slow shift toward more frequent fire occurred in conjunction with the warming and drying climate \citep{Safford2013}. At the same time, several Native American tribes were living throughout the study area during the reference period. Debate is ongoing among scientists and researchers as to the extent to which those peoples managed vegetation through setting fires \citep{Anderson1996}. As a result, fire history data includes evidence from both lightning-caused and human-caused fires, and the historical record does not always distinguish them from one another. And, regardless of how fires begin, they burn and impact forest vegetation. I lack an empirical basis for excluding some fires from the record, and so decided to include all the available data.

Comparing the current landscape to the chosen reference period should not imply that a proper management goal would be to recreate all of the ecological conditions and dynamics of this period. Such a goal may not be possible, nor potentially desirable in light of ongoing climate change, ecological shifts, and social realities. However, using the chosen reference period provides an opportunity to compare current conditions to a baseline set of data on ecosystem conditions (composition, configuration, and disturbance processes) and `` develop an idea of trend over time and idea of the level of departure of altered ecosystems from their `natural' state'' \citep{Safford2013}. The results presented here will complement the Natural Range of Variability assessments compiled by the Forest Service's Pacific Southwest Region Ecology group \citep{Safford2013,Merriam2013,Meyer2013a,Meyer2013,Estes2013,Estes2013a,Gross2013}. An understanding of natural landscape structures and variability during this reference period also provides a basis for forest management policies and associated actions that seek to mimic natural disturbance patterns \citep{Romme2000,Buse2002}. 






%%%%%%%%%%%%%%%%%%%%%%%%%%%%%%%%%%%%%%%%%%%%%%%%%%%%%%%%%%%%%%%%%%%%%%%%%%%%%%%%%%%\subsection{Modeling Framework}
\section{Modeling Framework}
\label{sec:modelframework}

To simulate the historic range of variability during the reference period in the study area, I used a modified version of the Rocky Mountain Landscape Simulator (\textsc{RMLands}), a spatially-explicit, stochastic, landscape-level disturbance and succession model capable of simulating fine-grained processes over large spatial and long temporal extents \citep{McGarigal2001}. It is grid-based and simulates fire on landscapes in a spatially explicit and realistic manner. State transitions are simulated at the 30 m pixel scale. As a result, I did not assign fires as a whole to a \emph{low}, \emph{mixed}, or \emph{high severity} status. Instead, I focused on defining conditions under which transitions among potential states within a given cover type occur or not, and subsequently labeling the effect of fire as high mortality (corresponding to high severity) or low mortality (encompassing all other severity levels) at the grid cell level. An individual fire event is nearly always composed of a mix of high and low mortality fire effects. Transitions may also take place in the absence of fire due to natural succession \citep{McGarigal2012}. Outputs from the model are readable by the landscape pattern analysis software \textsc{Fragstats} \citep{Fragstats2012}, which facilitates the landscape configuration analysis.

\textsc{RMLands} was originally developed to simulate the historical range of variability of forests in southwestern Colorado. Reports on the historical range of variability were completed in 2005 for the San Juan National Forest and the Uncompaghre Plateau \citep{McGarigal2005,McGarigal2005a}. \textsc{RMLands} has also been used to simulate wildfire and vegetation succession on the Lolo National Forest in Montana \citep{Cushman2011}. I worked closely with Kevin McGarigal and Eduard Ene to adapt the software for use in the Sierra Nevada. I then used the modified software to prepare an HRV analysis for part of the Tahoe National Forest in California. This work has been a collaborative effort between the University of Massachusetts, Amherst, staff on the Tahoe National Forest in California, and ecologists from the USFS Region 5 Ecology program. In addition to developing cover types and seral stage definitions for Sierran vegetation that were compatible with the \textsc{RMLands} modeling framework, I also modified the level at which it handles susceptibility and mortality, and introduced a new parameter, the Topographic Position Index. It is worth noting that I, like many researchers using simulations to conduct HRV analyses, parameterized RMLands to simulate passive management of fire and vegetation \citep{Wimberly2002,Nonaka2005,McGarigal2012}. That is, I did not simulate vegetation treatments, nor did I attempt to emulate additional fire suppression efforts.






%%%%%%%%%%%%%%%%%%%%%%%%%%%%%%%%%%%%%%%%%%%%%%%%%%%%%%%%%%%%%%%%%%%%%%%%%%%%%%%%%%%




\section{Climate Change and Future Ranges of Variability}

In addition to the HRV analysis, the need to explore and understand the ramifications of climate change on the disturbance regime and its relationship to the forest has become well recognized. As important as it is to understand the dynamics characteristic of the historical period, the future climate will differ from the historic climate. 
%
\begin{figure}[!htbp]
\centering
\includegraphics[width=\textwidth]{/Users/mmallek/Documents/Thesis/Seminar/snowpack_comparison.png}
\caption{The Sierra Nevada March snowpack levels, as seen from a NASA satellite. The top image is from March 2010, the last year with average winter snowfall in the region. The second image is from March 2015. The red circle surrounds my study area. Lower snowpack levels mean less water over less time in mountain rivers, causing moisture stress in forest plants and increasing their susceptibility to wildfire.
}
\label{fig:satellitesnowpack}
\end{figure}

Recent warming and drying trends, and the current drought (Figure~\ref{fig:satellitesnowpack}), have already influenced a more frequent and proportionally more severe fire regime in western forests in general and the Sierra Nevada in particular \citep{McKenzie2004,Westerling2011,Miller2012}. These trends are anticipated to continue under warmer and drier climate change scenarios \citep{Westerling2008,Dale2001}.  Changes have also been reported in the elevation of fires in the Sierra Nevada, increasing the potential for upward shifts of the elevational range occupied by species and vegetation assemblages \citep{Schwartz2015}. In addition, concern about the impact of changes to precipitation and temperature anticipated under climate change in the northern Sierra on local disturbance regimes, and subsequently, seral stage distribution and patch configuration has motivated analysis that consider not only at the current and historical conditions, but also future conditions \citep{Fule2008,North2012}. 

Range of variability analyses that offer a complementary analysis of future scenarios under climate change are rare (but see \cite{Keane2008} and \cite{Duveneck2014}). Where the focus of management efforts has in the past been restoration, current policy emphasizes using adaptive strategies to ensure resilient ecosystems \citep{Stephens2010}. By simulating a range of potential future climate scenarios, I generate data to use in evaluating trends in landscape pattern related to trends projected under climate change, and place the current landscape in that context. Moreover, I use this additional information to consider which restoration strategies are likely to promote resilient forests and make sense ecologically for the area under study \citep{Duncan2010}. 

The range of potential future climate scenarios used to parameterize the model in this study come from models initialized using the set of parameters for Representative Concentration Pathway (RCP) 8.5. The RCP scenarios are those currently used in climate change research. They replaced the previous set of scenarios, known as SRES after the Special Report on Emissions Scenarios that detailed them \citep{VanVuuren2011a}. They were developed in an effort that overhauled the IPCC and climate change research communities' approach to developing and using climate scenarios \citep{Moss2008}. A key difference between the approach used to build the SRES scenarios and that used to build the RCP scenarios was the shift from a sequential to a parallel approach (Figure~\ref{fig:scenarioapproach}). In SRES, emission and socioeconomic scenarios were designed first, and the outputs used to model radiative forcing, then climate projections, and finally impacts. Conversely, in the current parallel approach the first step is to design RCPs that correspond to levels of radiative forcing, and then allow climate modeling and emissions and socioeconomic scenario building to happen concurrently based on the same starting set of assumptions. The outputs from both components of the parallel process are then used to analyze impacts \citep{Moss2010}. 

The new process has clear advantages, including the fact that by not prescribing the mechanisms that lead to particular RCPs, flexibility is provided to examine a huge range of different factors and how they combine that can lead to a particular outcome \citep{VanVuuren2011}. However, it can be slightly more confusing to think about RCPs on their own, since they are not really intended to be used on their own, but rather as the shared basis for analyzing possible future outcomes \citep{VanVuuren2011}. The RCP outcomes are standardized in that they are based on (named after) the predicted conditions in 2100 \citep{VanVuuren2011a}. They are also spatially explicit, and all RCP scenarios are based on the same geographic locations \citep{VanVuuren2011a}. Finally, they are actually trajectories defined by their end point, such that they explicitly include a sequence of data points from the starting conditions to the final conditions \citep{VanVuuren2011}. I use data built from the trajectory to 2100 to parameterize \textsc{RMLands} for simulating disturbance and succession into the future.

\begin{figure}[!htbp]
\includegraphics[width=\textwidth]{/Users/mmallek/Documents/Thesis/Plots/rcp-sequence.png}
\caption{Differences between the SRES scenario development and use (sequential approach) and the current RCP scenario development and use (parallel appraoch). Figure from \citet{Moss2008}.}
\label{fig:scenarioapproach}
\end{figure}

RCP8.5 includes no specific climate mitigation target, unlike the other three RCP scenarios in use \citep{Riahi2011}. As a result, it is considered a reference, or baseline scenario, in which greenhouse gas emission and concentrations increase over time without leveling out \citep{Riahi2011}. A literature review during the RCP development process designated radiative forcing in 2100 of 8.5 W/m$^2$ as the high end of plausible futures that had been modeled \citep{VanVuuren2011}. The corresponding concentration of $> \sim1370 \text{ CO}_2$ -eq in 2100, compared to 375 $\text{CO}_2$ -eq in 2005. The 66\% range for the variable of temperature increase above pre-industrial levels under the RCP8.5 scenario is 4.0\textdegree -- 6.1\textdegree C \citep{Rogelj2012}. Since the development of RCP8.5 as a scenario, narratives illustrating one potential set of socioeconomic and political conditions have been developed, one of which I include here:
%
\begin{quote} 
The scenario’s storyline describes a heterogeneous world with continuously increasing global population, resulting in a global population of 12 billion by 2100. Per capita income growth is slow and both internationally as well as regionally there is only little convergence between high and low income countries...The slow economic development also implies little progress in terms of efficiency. Combined with the high population growth, this leads to high energy demands...the future energy system moves toward coal-intensive technology choices with high GHG emissions...agricultural productivity increases to feed a steadily increasing population...Compared to the scenario literature RCP8.5 depicts thus a relatively conservative business as usual case with low income, high population and high energy demand due to only modest improvements in energy intensity \citep{Riahi2011}.
\end{quote}
%
Thus the RCP8.5 scenario, developed in 2008 is intended to serve as the upper boundary for changes to the climate and associated consequences by 2100 \citep{Moss2008}. The extent to which it can still be considered an upper bound in 2015 is outside the scope of this thesis. Regardless, the results presented in Chapter~\ref{ch:FRV} are explicitly tied to the RCP8.5 scenario. I present and discuss them within this context.


As part of my investigation into future landscape trends, I observed a dramatic shift in the proportion of both xeric and mesic mixed conifer forests in the Early Development seral stage. I then focused my analysis of patch configuration on the Early Development stage of these two cover types. Early successional habitats are not a major focus of forest ecology research, in part because they are seen as an intermediate phase that for some forest users is kept short \citep{Swanson2011}. However, they are a critical component of all systems, functioning as a major contributor to biodiversity and supporting a range of species' habitat needs \citep{Chang1995,Hutto2008,Swanson2011}. The Sierra Nevada Framework, last updated in 2007, identifies management indicator species that use openings and early seral habitat \citep{USDAForestService2004,USDAForestService2007}. Recent trends of increasing wildfire extent and severity mean that managers face more decisions about when and how to manage post-fire early successional habitat \citep{Stephens2013,Dellasala2014}. My model results will provide insight into the spatial configuration of early successional forests under a natural fire regime for the intensively used mixed conifer zone. These results may be used when designing restoration efforts using both prescribed fire and mechanical harvest techniques.

%Had a note in here about adding more about early seral, but not sure there's a lot more to say. Don't want to talk a lot about range shifts because I can't model it anyway.




