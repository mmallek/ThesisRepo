% !TEX root = master.tex

\chapter{Introduction} %Becky says this is the lit review. OK that some things might be methods more typically.
\setcitestyle{notesep={;},aysep={}}

\section{Project Area}

\subsection{Physical Geography}

\begin{figure}[!htbp]
\includegraphics[width=\textwidth]{/Users/mmallek/Tahoe/Report3/images/studyarea.png}
\caption{The Sierra Nevada Ecoregion is outlined in brown. The project landscape (outlined in black) is located in the northern extent of the Sierra Nevada on the Tahoe National Forest, comprising the Yuba River watershed.}
\label{projectarea}
\end{figure}

The Sierra Nevada is a major North American mountain range and ecological region, located east of California's Central Valley and extending from Fredonyer Pass in the north to southern Kern County in the south. Much of the Sierra Nevada is reserved as federally-held public land, managed by the U.S. Forest Service, Bureau of Land Management, and the National Park Service. The Plumas and Tahoe National Forests are located in the northern portion of the Sierra Nevada. The project landscape (see Figure~\ref{projectarea}) is located on the northern part of the Tahoe National Forest, on the Yuba River and Sierraville Ranger Districts, and comprises about 181,550 hectares. It is composed of a set of three HUC-5 watersheds, the Upper North Yuba River, the Middle Yuba River, and the Lower North Yuba River, collectively referred to in this document as the Upper Yuba River watershed. HUCs are Hydrologic Unit Codes that refer to a nested system of watersheds in the United States, defined by the United States Geological Survey. The HUC-2 watershed scale includes the largest defined watersheds, which are then subdivided down to a smallest level of HUC-16. HUCs are commomnly used by agencies like the Forest Service to organize land management.

The topography of the project landscape consists of rugged mountains incised by two major and a few minor river drainages. Elevation ranges from about 350 to 2500 meters. The area receives 30--260 cm of precipitation annually, most of which falls as snow in the middle to upper elevations \citep{Storer1963}. A summer drought typically persists from May to September, increasing the importance of developing a significant snowpack during the winter months, since snowmelt runoff is a key source of soil moisture during the late spring and summer months \citep{Minnich2007}. In the Sierra Nevada, the heaviest precipitation occurs to the east and north of the San Francisco Bay area; our study area is within this region (Fire Ch 12). We downloaded 800 m resolution 30-year normal precipitation datasets for the northern Sierra from the Oregon State PRISM project for a side project, which illustrate the particularly high amounts of precipitation that falls across middle elevations of the study area compared to the larger region (PRISM Climate Group). This increased amount contributes to the occurrence of patches of exceptionally productive forests \citep[][ Alan Doerr, personal communication]{Littell2012}.

\paragraph{Current Management Context}

The arrival of Europeans in the 1850s sparked a transformation of this landscape as people harvested timber, extracted gold using hydraulic mining techniques, and suppressed wildfires \citep{Storer1963}. Forestry, mining, grazing, and dozens of recreational activities, including hunting, mountain biking, and hiking are all important uses taking place on the Tahoe National Forest. Fifteen allotments exist within the project area for both cattle and sheep grazing. In addition, 231,368 hectares inside of the project area have non-Forest Service ownership. Many of these lands were privately held, often by timber companies, before the Forest was created. In addition, many public lands were given to the Central Pacific Railroad in the late 19th century, and this ``checkerboard'' ownership pattern persists today (Figure~\ref{ownership}). Mining of gold and other minerals also continues. These activities affect and interact with ongoing vegetation succession and disturbance processes in the area \citep{USDAForestService2014}.

\begin{figure}[!htbp]
\centering
\includegraphics[width=0.4\textheight]{/Users/mmallek/Tahoe/Report3/images/ownership_resized.png}
\caption{Map of National Forest lands and lands held by other entities (including private, industry, and other public land). Forest lands are in green, with other ownership in tan. The boundary of the study area is in black; this image shows the 10 km buffer.} 
\label{ownership}
\end{figure}

Although many uses of the forest led to changes in vegetation structure and composition, logging and wildfire suppression in combination have altered the historical fire regime and vegetation patterns most significantly. The Tahoe National Forest has active timber and fire management programs. Clearcutting, shelterwood, salvage cutting, and plantation management have been major components of timber management on the Forest. Between 1988 and 2002, timber sales in the Sierra Nevada dropped drastically, but on the Tahoe National Forests timber sale levels have fluctuated both up and down (although annual sawtimber sold has decreased similarly to other Sierran Forests) \citep{USDAForestService2004}. Although very large fires have burned in the Sierra Nevada recently (e.g. the 2013 Rim Fire), few large fires have impacted the Tahoe National Forest in the last 100 years \citep{USDAForestService1990}. In 1960, approximately 100,000 acres burned on the Forest. The low total burned acreage is despite fairly high fire starts (both human- and lightning-caused), indicating that suppression efforts have been very successful \citep{USDAForestService1990}.
% B wanted to know where that burn was, but I just got that out of the Forest Plan and it didn't say.

\subsection{Disturbance Regime}
\paragraph{Fire in the Sierra Nevada}
In the Sierra Nevada, cycles of fire and vegetation recovery occur variably over large extents, as well as over long periods of time. Ongoing disturbance results in increased heterogeneity, which can be captured by various metrics used to describe vegetation composition and configuration \citep{Monica2008}. Prior to European settlement, wildfire was the major source of disturbance in Sierran forests, shaping the composition and configuration of vegetation communities. Fires were primarily lightning-caused, although indigenous peoples are thought to have set fires for vegetation management, especially in the lower elevations \citep{Safford2013}. In general, fire was frequent, with a mean rotation as short as 20 years in Ponderosa Pine-dominated forests. Wetter mixed conifer areas are predicted to have had a mean fire rotation of 30 years. Fire rotation is thought to increase gradually with elevation. For example, mesic Red Fir forests, which exist around 2,000 feet higher in elevation than Ponderosa Pine forests, had a mean fire rotation of 60 years \citep{Mallek2013}. Thus fire rotations increased with increased moisture and elevation. Variance around a mean fire rotation can be remarkable, as some parts of the forest experience fire much more frequently, while other escape fire for long periods. 

In general, regardless of vegetation type, fires during the pre-settlement period were thought to burn primarily at low intensities. High mortality (over 70\% overstory canopy mortality) was uncommon. Under this disturbance regime, stand-replacing fire initiated early development conditions on the landscape, but since they were uncommon, most fires only affected the understory by removing fuels. In some cases moderate overstory mortality opened forest canopies without resetting stand development, especially in more xeric parts of the forest \citep{Mallek2013,Safford2014,SNEP1996,SNEP1996a}. An absence of fire over many decades led to gradual closing of the canopy due to succession\todo{becky says I skim over this and should give it more attention}. The rarity of high mortality patches allowed large forest stands to succeed into late development and old growth conditions \citep{SNEP1996,Mallek2013,Safford2014,SNEP1996a}. 

From Becky, in response to previous sentence: ``This isn't necessarily the case.  Those areas that were succeeded to old growth were areas that wouldn't have burned as regularly (lower slopes, canyon bottoms).  With frequent fire the surrounding landscapes were maintained in a low fuel condition so fires wouldn't have spread into them.  With fire suppression, the density of these forests have increased across the board and this has resulted in altered fire spread and intensity patterns.''

Since then, fire suppression, logging, grazing, and mining have all interacted to alter the historical fire regime and vegetation patterns \citep{Stephens2015,Knapp2013}. For the xeric forest types within this landscape, frequent fires (usually having low mortality) were the norm. After large-scale fire suppression became the norm in the second half of the 19th century, less fire-tolerant species (such as Douglas fir and white fir) have come to dominate areas where they were once a minor part of the vegetation community. Grazing and development made fires less common by altering or removing the fine fuels that carried fire. Timber harvest, especially of fire-tolerant species such as ponderosa and sugar pines, accelerated the increased cover of species such as white fir. Finally, fire suppression allowed the buildup of medium size fuels and ladder fuels, which promotes larger and hotter fires when they do occur. Moreover, the lack of natural fires has meant that variation in fuel loading has decreased, which allows large fires to spread over very large areas \citep{Hessburg2005,Beaty2007,Meyer2008}.

%%% HRV Methods
Many species exhibit fire-adapted traits, such as resprouting from roots after a fire (e.g. tanoak), fire-induced germination (e.g. manzanita), or thick bark (e.g. ponderosa pine).  


\subsection{Ecology of vegetation systems}
Vegetation is tremendously diverse and changes slowly along an elevational gradient and in response to local changes in drainage, aspect, and soil structure. Grasslands, chaparral, oak woodlands, mixed conifer forests, and subalpine forests are all found within the study area. In collaboration with USDA Forest Service staff, we developed a system of land cover and seral stage classification based on LandFire (\burl{http://www.landfire.org} and \citet{VandeWater2011}'s Presettlement Fire Regimes, and crosswalked Forest Service corporate spatial data based on the Northern Sierra \textsc{CalVeg} classification to the 31 cover types. We also considered information from \emph{A Guide to Wildlife Habitats of California}, popularly known as the ``Wildlife Habitat Relationship (WHR)'' cover types. There are actually 13 major vegetation types, several of which include one or more of the following variants: mesic, xeric, ultramafic, and aspen. We completed analysis on the nine (of 31) cover types that extend across at least 1000 hectares because we had the most confidence in results for cover types at least that well represented. Within this thesis, we focus our reported results and discussion on the two most prevalent cover types, mesic mixed conifer forests and xeric mixed conifer forests, because the results are reliable and a deeper understanding of the historical range of variability of mixed conifer zone will be the most useful to managers. In this introduction we review the cover types following the ecological zone groupings from \citet{VanWag2006}. 

\paragraph{Foothill shrubland and woodland} This ecological zone lies directly adjacent and to the west of the study area. A small part of the buffer we used around the study area includes this zone, which is represented by the Oak Woodland cover type. This type is characterized by savannas, woodlands, or forests of either monospecific or mixed stands of various oak species. \emph{Quercus douglasii}, \emph{Quercus lobata}, \emph{Quercus wislizenii}, and \emph{Quercus garryana} are the major dominants. (Appendix~\ref{oak-description}).

\paragraph{Lower montane forest} This ecological zone includes Oak-Conifer Forests and Woodlands and Mixed Evergreen forests at lower elevations, developing into Sierran Mixed Conifer forests with increasing elevation. All three of these cover types are typified by a combination of both coniferous and broadleaved trees. 

Mixed Evergreen is characterized by dense stands of \emph{Notholithocarpus densiflorus} (Tanoak) and \emph{Arbutus menziesii} complemented by \emph{Pseudotsuga menziesii} on more mesic soils. In xeric sites, conifers are less common, and a hardwood tree layer composed of evergreen oaks such as \emph{Quercus chrysolepis}, \emph{Quercus wislizeni}, \emph{Quercus kelloggi}, and \emph{Quercus garryana} instead dominates. Fires are fairly common in this cover type, but the vegetation quickly recovers, and many of the hardwoods resprout after fire (Appendix~\ref{meg-description}). 

Oak-Conifer Forests and Woodlands are characterized by the conifers \emph{Pinus ponderosa} or \emph{Pinus jeffreyi}, with one or more oaks, with one or more oaks, such as \emph{Quercus kelloggii}, \emph{Quercus garryana}, \emph{Quercus wislizeni}, or \emph{Quercus chrysolepsis}. Historically, low severity fires were extremely common. Fire is integral to the ecology of yellow pines, and this cover type is one of the most altered by fire suppression (Appendix~\ref{ocfw-description}).

Sierran Mixed Conifer forests are characterized by five conifers and one hardwood: \emph{Abies concolor, Pseudotsuga menziesii, Pinus ponderosa, Pinus lambertiana, Calocedrus decurrens}, and \emph{Quercus kelloggii}. At least three conifers are typically present in any given stand. \emph{A. concolor} tends to be the most ubiquitous species, especially on north-facing slopes. \emph{Pinus ponderosa} was historically the dominant species, under the previous frequent low severity fire regime. It is still the most prevalent on south slopes and is present continuously from the Oak-Conifer Forest and Woodland zone below it in elevation (Appendix~\ref{smc-description}).


\paragraph{Upper montane forest} The upper montane zone is defined by the presence of Red Fir forests \citep{Potter1998}. The Red Fir cover type is dominated by \emph{Abies magnifica}, but other species do co-occur. On mesic sites, \emph{Pinus monticola} and \emph{Pinus contorta} ssp. \emph{murrayana} are also found, while on xeric sites \emph{Abies concolor} and \emph{Pinus jeffreyi} are more common. Red Fir forests are extremely resilient to disturbance. Wildfires were less common in these forests than those of the Lower montane zone, and stands were characterized by complex patches of even-aged trees within a single stand arising from localized disturbance events (Appendix~\ref{rfr-description}). The boundaries between Sierran Mixed Conifer and Red Fir are fuzzy, and the types intergrade with one another. Similarly, Red Fir forests of the Upper montane zone blend into the lodgepole and subalpine conifers of the Subalpine forest zone.


\paragraph{Subalpine forest} We defined two cover types within the subalpine forest. Lodgepole Pine occurs along the lower elevation portion of the zone, usually in wetter soils, such as along meadow edge, although it is also found on rocky soils. Unlike the subspecies of Lodgepole pine found in the Rocky Mountains, \emph{Pinus contorta} ssp. \emph{murrayana} does not have serotinous cones. Wildfires in Lodgepole Pine tend to be high severity and recur at long intervals ((Appendix~\ref{lpn-description}).% Fites-Kaufman et al 2007, Landfire). 

Above it, Subalpine Conifer is found, consisting of a mosaic of forest, woodland, meadow, and scrub vegetation. Trees often grow as krummholz forms at the highest elevations. \emph{Tsuga mertensiana} is often the most common tree species and mixes with \emph{Pinus contorta} ssp. \emph{murrayana, Abies magnifica, Pinus monticola}, and \emph{Pinus albicaulis}. Wildfire is very rare in this cover type (Appendix~\ref{scn-description}). 

Western White Pine sometimes occurs in sufficiently continuous patches to be classified as its own type, separate from the Subalpine Conifer group. Typified by \emph{Pinus monticola}, species from the Subalpine Conifer and Red Fir cover types sometimes co-occur as well. This cover type tends to occur on drier soils. Most fires are low severity fires that promote the development of late successional forests (Appendix~\ref{wwp-description}).


\paragraph{Alpine meadow and shrubland} All lands within the study area are below climatic treeline, and we did not identify alpine meadows and shrublands that are not appropriately considered an early successional stage of Subalpine Conifer.


\paragraph{Eastside forest and woodland} Although the project area does not include any lands east of the Sierra crest, the buffer around the study area did, and we classified and described the Yellow Pine cover type to capture most of this vegetation community. It is characterized by \emph{Pinus ponderosa} or \emph{Pinus jeffreyi}, but other conifers and oaks, as well as \emph{Juniperus occidentalis} may occur. Under historical conditions, wildfires were extremely common and were almost always low severity; fire is integral to the ecology of yellow pines (Appendix~\ref{ypn-description}). 
%
Oak-Conifer Forests and Woodlands, as described above, are also present (Appendix~\ref{ocfw-description}). 
%
Also east of the crest are two shrub community types: Big Sagebrush (Appendix~\ref{sage-description}), typified by \emph{Artemisia tridentata} and Black and Low Sagebrush (Appendix~\ref{lsg-description}), typified by \emph{Artemisia arbuscula} or \emph{Artemisia nova}. 


\paragraph{Other cover types and variants} Some cover types not listed above can be found in any zone. We classified riparian vegetation into a Montane Riparian cover type (Appendix~\ref{mrip-description}). 
%
Curl-leaf Mountain Mahogany is most common in the Upper montane zone and above, but may occur in the Lower montane zone as well. Typified by \emph{Cercocarpus ledifolius}, it is also a shrub cover type (Appendix~\ref{cmm-description}). 
%
Several of the cover types listed here include ultramafic (a rock type that includes serpentine) variants, which are characterized by discontinuous fuel cover of grasses and low shrubs. The most common tree species to grow on ultramafic sites is \emph{Pinus jeffreyi}. Many endemic species grow on ultramafic soils, which are high in metal concentrations and fairly unproductive \citep{OGeen2007}.
%
Finally, we included aspen variants to several cover types. Lodgepole Pine, Red Fir, and Sierran Mixed Conifer all included variants that were seral to conifer forests in the absence of fire to maintain the aspen stand. Subalpine Conifer and Yellow Pine included variants considered ``stable'' aspen. In all cases, the total area classified to aspen is very small. We neither analyze nor report results for the aspen variants in this thesis.


\section{Range of Variability Analysis}
HISTORICAL
Historic range of variability (HRV) is a useful tool in landscape planning \citep{Nonaka2005}. HRV analysis is intended to help conceptualize the mechanisms behind large-scale ecosystem functions and provide a basis from which to make predictions about how a given ecosystem will react to disturbances in the future \citep{Nonaka2005,Landres1999}. U.S. Forest Service managers and planners are directed to ground decisions in the context of ``maintain[ing] or restor[ing] ecological conditions that are similar to the biological and physical range of expected variability'' (36 CFR 219.4). On the Tahoe National Forest, an opportunity to model HRV and future management scenarios to inform restoration planning arose after the 1999 Pendola Fire, which burned 3,000 acres on the Tahoe National Forest. This work is a collaborative effort between the University of Massachusetts -- Amherst, staff on the Tahoe National Forest in California, and ecologists from the USFS Region 5 Ecology program.\todo{add more about Forest Planning}

Although empirical data may sometimes be available on some variables affecting HRV, the time scales and broad spatial extents under study require simulation in order to incorporate all parameters of interest and therefore derive a meaningful quantification  \citep{Swetnam1999,Mladenoff1999}. Range of variability analyses have been conducted using literature searches exclusively, including within the Sierra Nevada \citep{Safford2013}. Results of such analyses depend on the assumption that an aggregation of many small studies is sufficient to address long-term, large-scale questions, and require researchers to accept many unknowns about research methodologies. Moreover, in landscapes severely impacted by European settlement, such as those of the northern Sierra Nevada, we can never observe trajectories in which fire suppression is not part of the equation \citep{Keane2012}. In the absence of consistent and complete data, simulations can be used to incorporate the data that does exist and generate new datasets of otherwise unobservable landscape trajectories. From these new datasets, statistical analyses can be used to describe the landscape quantitatively, and subsequently make inferences about the HRV of an area, as well as compare current conditions to the HRV.

This thesis includes an analysis of the historical range of variability for the Upper Yuba River Watershed and an analysis of disturbance and succession trends associated with future climate scenarios. We define the HRV as the variance in disturbance processes and landscape composition and configuration over the 300 years prior to European settlement. A primary motivation for this study was to aid in planning on the Tahoe National Forest. The quantitative assessment of the HRV of this landscape provides managers with a statistical, ecosystem-level analysis of the disturbance and succession processes that characterize this portion of the northern Sierra Nevada. Quantification of HRV provides managers with a neutral assessment of the current departure from HRV, which they can use to prioritize certain vegetation types, disturbance processes, or their intersection for restoration or maintenance. Because the simulation captures landscape changes over hundreds of years, far longer than the planning cycle, the results allow managers to ground near-term plans and expectations within a larger context. 

In restoration planning efforts, it is logical to look back to the last known period during which a dynamic but resilient landscape existed. The arrival of European settlers to the Sierra Nevada led to sweeping ecological changes that now have greatly altered many Sierran landscapes -- through fire suppression, grazing, road-building, timber cutting, recreation, and other activities \citep{Storer1963,Stephens2015,Knapp2013,Hessburg2005}.  The period prior to European settlement, then, is a suitable reference condition against which we can compare current landscape structure and dynamics. Moreover, it is frequently used in the western United States as the historical reference period for restoration planning \citep{VandeWater2011,Safford2013,Meyer2013}. The period is also up to several times the length of rotation periods identified for well-understood cover types within the project area. Finally, it is a timeframe for which we have sufficient information to have confidence in model results. 

We are mindful of the fact that this reference period overlaps the ``Little Ice Age,'' which may temper the utility of the results as specific management targets, but does not diminish their usefulness in other ways \citep{Minnich2007,Safford2013}. The chosen reference period was a not time of stasis, climatically, ecologically, or culturally. The oscillation of the Palmer Drought Severity Index, a measure of climate variability in terms of precipitation and temperature, over time illustrates this (see Chapter~\ref{subsubsec:distparams} for a detailed discussion). Multi-year droughts and El Ni\~no/La Ni\~na events also occurred over this time frame \citep{Minnich2007}. Ecologically, our historical period occurred during a very long-term (on the scale of thousands of years) shift to a warmer and drier climate, with an associated shift toward species more tolerant of such conditions, such as yellow pine species, and away from species like white fir, which prefer more mesic conditions. A slow shift toward more frequent fire was also taking place \citep{Safford2013}. Culturally, several Native American tribes were living throughout the project area during the reference period. Debate continues among scientists and researchers as to the extent to which those peoples managed vegetation through setting fires \citep{Safford2013}\todo{others?}. Because we lack empirical evidence to distinguish between lightning-caused and human-caused fires during the reference period, we decided not to exclude any fire frequency or rotation data on the basis of not being reflective of ``natural'' conditions. 

We emphasize that our choice of reference periods does not suggest that it should be our goal in management to recreate all of the ecological conditions and dynamics of this period. Such a goal may not be possible, nor potentially desireable in light of current climate change, ecological shifts, and social realities. However, the reference period proposed will allow us to compare current conditions to a baseline set of data on ecosystem conditions (composition, configuration, and disturbance processes) ``to develop an idea of trend over time and idea of the level of depature of altered ecosystems from their ``natural'' state'' \citep{Safford2013}. The results presented here will complement the Natural Range of Variability assessments compiled by the Forest Service's Pacific Southwest Region Ecology group \citep[e.g.,][]{Safford2013,Merriam2013,Meyer2013a,Meyer2013,Estes2013,Estes2013a,Gross2013}. An understanding of natural landscape structures and variability during this reference period also provides a basis for forest management policies and associated actions that seek to mimic natural disturbance patterns \citep{Romme2000,Buse2002}. 

%%%%%%%%%%%%%%%%%%%%%%%%%%%%%%%%%%%%%%%%%%%%%%%%%%%%%%%%%%%%%%%%%%%%%%%%%%%%%%%%%%
FUTURE

Methods for quantifying the natural range of variability for a diversity of landscapes in the United States augmented the development of research focused on this task \citep{Landres1999}. Of these, simulation of the historical dynamics became fairly popular. By 2004, some 45 landscape fire and succession models alone had been developed \citep{Keane2004}. Many of these, such as \textsc{landis} \citep{He1999}, \textsc{zelig-l} \citep{Miller1999}, \textsc{safe-forests} \cite{Sessions1997} and \textsc{landsum} \citep{Keane2012} are still in use today. Landscape fire and succession models are used to create spatially-explicit simulations of both of these key forest processes, typically outputting a set of GIS layers for each timestep of the model that can then be analyzed to quantify trajectories and patterns in the disturbance regime, seral stage composition, and landscape configuration over time \citep{Keane2004}. Many range of variability analyses in the United States focus on the historical range of variability (HRV) of an area. The Rocky Mountains and Oregon Coast Range in particular have been the focus of several HRV studies, while only one has been conducted in the Sierra Nevada \citep{Miller1999}, which took place in Sequoia National Park in the southern Sierra. 

While HRV studies can play an important role in informing natural range of variability, the need to explore and understand the ramifications of climate change on the disturbance regime and forest structure is also critical. As informative as an understanding of the landscape characteristics of the historical period may be, the simple fact is that future climate will differ from the historic climate. Fires have become more common and proportionally more severe in the last few decades, and and this is anticipated to continue under warmer and drier climate change scenarios \citep{McKenzie2004,Westerling2007,Dale2001}. Where the focus of management efforts had been restoration in the past, now adaptation to ensure resilient ecosystems is the primary objective of managers \citep{Stephens2010}. By simulating a range of potential future climate scenarios, can evaluate trends related to trends projected under climate change, and place the current landscape in that context. Moreover, an examination of landscape composition and configuration under potential climate change scenarios is important because it can provide additional information about what restoration strategies are likely to remain resilient and make sense ecologically for the area under study. 

\subsection{Modeling Framework}
\textsc{RMLands} has been used previously to assess the HRV on the San Juan National Forest \citep{Mcgarigal2012} and the Uncompahgre Plateau \citep{Romme2009} in Colorado, as well as the Lolo National Forest in Montana \citep{Cushman2011}. As a continuation of the Montana study, which adapted \textsc{RMLands} to use data from the LandFire project (\burl{http://www.landfire.gov}), we further adapted the software for use in the Sierra Nevada in order to prepare an HRV analysis for part of the Tahoe National Forest in California. Although many range of variability analyses have been completed, research that offers a complementary analysis of future scenarios under climate change are rare (but see \cite{Keane2008} and \cite{Duveneck2014}). Quantifying and describing a ``Future Range of Variability'' (FRV) can inform how realistic restoration toward a historical condition may be \citep{Duncan2010}.


\section{Forest Planning}

With the emergence of ecosystem management in the early 1990s, the need to recognize ecosystems as dynamic and constantly-changing became well accepted, and calls to manage forests sustainably became common \citep{Christensen1996}. Within the context of forest and land management planning, the restoration of ecosystems to their pre-European settlement states was incorporated as a goal or desired future condition into various plans, including the Sierra Nevada Ecosystem Project \cite{SNEP1996a}. By 2000, the U.S. Forest Service's formal Planning Rule explicitly called for the agency to estimate and describe the range of variability under natural disturbance regimes, and manage for those characteristics (36 CFR \textsection 219 2000). The need to consider the natural range of variability was maintained through various amendments to the rule, and is still present in the new 2012 rule, finalized in early 2015 (36 CFR \textsection 219 2012).

\section{Methodological Limitations}
Because this study relied on the use of computer models, we note here some limitations that should be understood before applying the results in a management context. First, while it is important to recognize the many advantages of models, it is critical to understand that models are abstract and simplified representations of reality. \textsc{RMLands}, in particular, simulates wildfires, but does not simulate all of the disturbance processes or all of the complex interactions among them that characterize real landscapes. Ultimately, the results of a model are constrained by the quality of input data, which are not perfect. For example, the vegetation cover layer is subject to human interpretation errors and objective classification errors, and is further limited by the spatial resolution of the grid. The most appropriate use of the results is therefore to help identify the most influential factors driving landscape change, implications of our simulated disturbance and succession regime, and areas where further research is needed to delineate key parameters.

Second, it is important to realize that \textsc{RMLands} requires substantial parameterization before it can be applied to a particular landscape. To the extent possible, we have utilized local empirical data. However, we also drew on relevant scientific studies, often from other geographic locations, and relied heavily on expert opinion when scientific studies and local empirical data were not available. Our estimate of the HRV is subject to change as new scientific understanding or better data become available.

Third, the Sierra Nevada vegetation is extremely diverse and complex in its spatial arrangement and scale of mapping. In this thesis we limit our results to an evaluation of the full landscape and of xeric and mesic mixed conifer forests, which together comprise 63\% of the study area. Results for the next seven most extensive types (xeric and mesic mixed evergreen forests, xeric and mesic red fir forests, oak-conifer forests and woodlands, ultramafic oak-conifer forests and woodlands, and ultramafic mixed conifer forests) are included in the appendices. In general, our confidence in the results decline as the extent of a cover type declines, because the results are statistical and large samples are needed. We do not provide results for cover types that extend across less than 1000 ha of the study area.

Fourth, this thesis (and \textsc{RMLands}) focuses on the effects of one major natural disturbance: fire. Other kinds of natural disturbances also occur, including insects and disease, wind-throw, wild ungulate and beaver herbivory, avalanches, and other forms of soil movement, but the impacts of these other disturbances tend to be localized in time or space and have far less impact on vegetation patterns over broad spatial and temporal scales than does fire.

\section{Climate - FRV scenarios}
In addition to the HRV analysis, the need to explore and understand the ramifications of climate change on the disturbance regime and its relationship to the forest has become more well recognized. As important as it is to understand the characteristics of the historical period, the future climate will differ from the historic climate. Fires have become more common and proportionally more severe in the last few decades, and and this is anticipated to continue under warmer and drier climate change scenarios \citep{McKenzie2004,Westerling2008,Miller2012}. Where the focus of management efforts had been restoration in the past, now adaptation to ensure resilient ecosystems is the primary objective of managers \citep{Stephens2010}. By simulating both the historical and a range of potential future climate scenarios, we are able to isolate the effect of climate on the disturbance regime, and evaluate the difference between results generated under the historical versus future scenarios, as well as place the current landscape within the context of both. This provides a more complete picture of the range of potential variation in the landscape than either an HRV or future climate analaysis alone.

Concern about the impact of changes to precipitation and temperature anticipated under climate change in the northern Sierra on local disturbance regimes, and subsequently, seral stage distribution and patch configuration has motivated analysis that consider not only at the current and historical conditions, but also future conditions\citep{Fule2008,North2012}. 

Recent warming and drying trends have already influenced a more frequent and proportionally more severe fire regime in western forests in general and the Sierra Nevada in particular \citep{McKenzie2004,Westerling2011,Miller2012}. These trends are anticipated to continue under warmer and drier climate change scenarios \citep{Westerling2008}. Changes have also been reported in the elevation of fires in the Sierra Nevada, increasing the potential for range shifts upslope \citep{Schwartz2015}. Where the focus of management efforts had been restoration in the past, now adaptation to ensure resilient ecosystems is the primary objective of managers \citep{Stephens2010}. 

* more papers on fire, seral stages, early seral, type shifts

\section{Approach of this model}


\section{Project objectives}

The overall purpose of this project is to quantify the historical range of variability in landsca pe structure in the Yuba River watershed on the Tahoe National Forest and evaluate the relative effects of  alternative future land management scenarios on landscape structure. The specific objectives are as follows:
\begin{enumerate}
	\item Synthesize the empirical and expert knowledge on disturbance and succession processes characteristic of the pre-settlement period in the ecoregion containing the Yuba River watershed.
	\item Based on the synthesis above, simulate landscape dynamics using the Rocky Mountain Landscape Simulator (\textsc{RMLands}).
	\item Based on the simulation above, quantify the HRV in the disturbance regime, landscape composition, and landscape configuration of our focal ecoregion.
	\item Using a suite of future climate scenarios, simulate landscape dynamics and quantify the FRV in the disturbance regime, landscape composition, and landscape configuration of our focal ecoregion.
	\item Interpret the results and provide management recommendations where appropriate.
\end{enumerate}

In this study, our objectives were to evaluate the effect of climate change on the wildfire regime and landscape composition and configuration for the Yuba River waterhsed on the Tahoe National Forest. We held all model parameters constant except the climate parameter, incorporating Palmer Drought Severity Index (PDSI) values from a suite of seven climate trajectories developed by the National Center for Atmospheric Research (USA) and the Canadian Centre for Climate Modelling and Analysis to the year 2090 \citep{Cook2014}. We used \textsc{Fragstats} software and R to analyze outputs and report the 90\% range of variability for simulated future metrics. Ultimately, we evaluate our results for a series of simulations for these future scenarios to the current conditions. 


\section{Thesis organization}
We begin with a detailed presentation of the methodology behind this project for the historic range of variability and the future management scenarios. This includes the development of the input spatial data layers, selection of values for model parameterization, model calibration and execution, and the suite of tools used to conduct the analysis. Next we present the results for the HRV and the future scenarios, focusing first on the disturbance regime and second on the vegetation response. We focus on these results by analysis method, then in the subsequent chapter, Analysis by Cover Type, discuss the results in more detail for each of the cover types independently. The next chapter, Discussion, includes an description of the scope and limitations of our simulation and its results, as well as an overall assessment of the landscape under the HRV, as compared to the current conditions. We include management implications of this study. Finally, we present the results of the FRV analysis.


