% !TEX root = master.tex

\section{Discussion and Management Implications}
\label{sec:hrvdiscussion}




%%%%%%%%%%%%%%%%%%%%%%%%%%%%%%%%%%%%%%%%%%%%%%%%%%%%%%%%%%%%%%%%%%%%%%%%%%%%%
%%%%%%%%%%%%%%%%%%%%%%%%%%%%%%%%%%%%%%%%%%%%%%%%%%%%%%%%%%%%%%%%%%%%%%%%%%%%%


\subsection{Overall Landscape Assessment}

One objective of the thesis was to quantify the HRV in the disturbance regime, landscape composition, and landscape configuration of our focal ecoregion. We did this both at the landscape and cover type scale, and begin with an overall assessment of the landscape-level historical conditions.

As shown in Chapter~\ref{sec:hrvresults}, fire size was weakly but positively associated with the climate parameter (bigger fires under droughtier conditions). However, it is also influenced by vegetation susceptibility and the specified distribution of disturbance size. For this reason, large areas burned in relatively ``wet'' timesteps. Fires during the simulated historical period burned far more frequently and across larger extents than at any time since record keeping began within the Yuba River watershed study area. Most individual fires were small and predominantly low severity. However, large fires did occur, which disproportionately shaped patch configuration on the landscape and directly affected the seral stage composition of cover types. Because the disturbance regime results were essentially as specified in the model, we focus attention on the composition and configuration results in this section.
% come up with better word for "droughtier"

% rewrote 2015-09-20
With respect to the seral stage distribution, a few patterns emerged across the nine focal cover types. (Full results for these cover types are in Tables~\ref{tab:covcond1} to \ref{tab:covcond3}.) Overall, the study area departs from the HRV not only at the landscape scale, but also at the cover type and seral stage level. Full departure from the simulated HRV is the norm for seral stages within the nine focal cover types. In just 4 of 63 potential combinations, the current proportion of a given seral stage is well within the HRV (\textsc{ocfw\_u} Early, \textsc{rfr\_x} Late--Closed, \textsc{smc\_m} Early, and \textsc{smc\_u} Mid--Moderate). Another 8 cover-seral stage types were moderately departed from the HRV (between either the 5th and 25th, or the 75th and 95th percentiles). Mesic and xeric mixed evergreen forests, as well as xeric mixed conifer forests, had no seral stages within the HRV.

%rewrote this 2015-09-20
Interestingly, early seral conditions were less common during the simulation than on the current landscape for the mixed evergreen and ultramafic mixed forests, but more common for the xeric mixed conifer forests; for the other types, however, the proportion of early seral is within the HRV. Both mixed evergreen types and both oak-conifer types have a smaller proportion of forest in late development now than during the simulation. Late--Open conditions were far more common during the HRV. For all cover types except for mesic mixed evergreen forests, Late--Open conditions were far more common during the HRV, and the current landscape is fully departed from the HRV. The Mid--Open was more common for xeric and ultramafic mixed conifer forests during the simulation than today, while Mid-Closed was dominant during the HRV for mesic mixed conifer forests, but now is not. Finally, closed canopy conditions were much more dominant in mesic red fir forests than on the present landscape; these stages are also completely outside the HRV.

%-- Closed canopy ocfw dominant now, but not during HRV.  (didn't use)
%-- Closed conditions were much more dominant in mesic red fir forests.
%-- Mid open more common for smcx and smcm
%-- Mid closed dominant during hrv for Smcm, but not now
%-- Mixed evergreen forests and oak-conifer forests and woodlands generally have less area in late development now than during the simulated HRV.
%-- Early seral was much more prevalent in SMCX and somewhat more prevalent in RFRX, but consistent with the other cover types.
%-- Early more common for smcx
%-- Late open more common for all cover types except rfrm, always outside of hrv
%-- Mixed evergreen and xeric mixed conifer had no seral stages within the HRV.
%-- 12 of 63 cover type - seral stage combinations with the HRV
%-- Only OCFWU-Early, RFRX-LDC, SMCM-Early, SMCU-MDM, were within interquartile range.
%-- Being completely outside the simulated HRV is the norm for seral stages with the focal nine areas. In many cases, the current proportion of a given seral stage is within the HRV, but this generally means either between the 5th and 25th percentiles, or between the 75th and 95th percentiles. In just four cases (\textsc{ocfw\_u} early, \textsc{rfr\_x} late closed, \textsc{smc\_m} early, and \textsc{smc\_u} mid moderate) was the current proportion within the 25th-75th interquartile range.

An analysis of \textsc{fragstats} metrics at the landscape scale also yields insight into the historical period. First, we note that during the HRV, the landscape was composed of larger and more extensive patches, as illustrated by Figure~\ref{fig:fragland}. This trend was heavily influenced by the presence of wildfires on the landscape, as high mortality fires in particular created large areas of Early development vegetation (Figure~\ref{fig:patchmaps1-early}). However, we also observed large patches in the other seral stages, which were more likely to form long or convoluted patches that were nonetheless extensive (Figure~\ref{fig:patchmaps1-mid}).

% figure redone 2015-09-20
\begin{figure}[!htbp]
  \centering
  \subfloat[][]{
    \centering
    \includegraphics[width=0.45\textwidth]{/Users/mmallek/Documents/Thesis/maps/hrv-largepatch2410.pdf}
    \label{fig:patchmaps1-early}
    }%
  \subfloat[][]{
    \includegraphics[width=0.45\textwidth]{/Users/mmallek/Documents/Thesis/maps/hrv-largepatch2430.pdf}
    \label{fig:patchmaps1-mid}
    }
  \caption{(a) A large patch, highlighted with dark grey, of Sierran Mixed Conifer - Mesic in Early Development. The patch is 5,780 hectares, one of the second-largest patch during this timestep. (b) Two patches of Sierran Mixed Conifer - Mesic in Late--Closed. These are two of the largest patches during that timestep, at 10,930 hectares and 4,750 hectares.} 
  \label{fig:patchmaps1}
\end{figure}


% big 2622 12,600

In addition, we observe increased dominance by certain cover-seral stage types, as illustrated by smaller values for the Simpson's Evenness Index during the HRV (Figure~\ref{fig:fragland_siei}). For example, within the Sierran Mixed Conifer - Xeric cover type, Early Development and Mid--Open were much more widespread during the simulated HRV than in the current landscape (Figure~\ref{fig:patchmaps2}). Because Sierran Mixed Conifer - Xeric is so widespread, this shift would directly influence Simpson's Evenness, lowering its value.

% figure redone 2015-09-20
\begin{figure}[!htbp]
  \centering
  \subfloat[][]{
    \centering
    \includegraphics[width=0.45\textwidth]{/Users/mmallek/Documents/Thesis/maps/hrv-dominance-ts0.pdf}
    \label{fig:patchmaps2-ts0}
    }%
  \subfloat[][]{
    \includegraphics[width=0.45\textwidth]{/Users/mmallek/Documents/Thesis/maps/hrv-dominance-26.pdf}
    \label{fig:patchmaps2-ts570}
    }
  \caption{Cover-Seral stage map focused on patches from Sierran Mixed Conifer - Xeric, showing increased dominance by certain cover-seral stage types during the HRV. (a) The current landscape. (b) The same region of the map during a randomly selected timestep after the equilibration period. Note the contrast between the two maps with respect to the seral stages and size of individual patches.} 
  \label{fig:patchmaps2}
\end{figure}


Third, we find that patches on the landscape were more aggregated at the cell-level during HRV, which is illustrated by the Contagion metric. In general, patches have low levels of both dispersion and interspersion. Of course, there are many ``edgy'' areas on the landscape, but this metric indicates that across the full landscape aggregation is more typical, particularly in comparison to the current landscape. Again, the homogeneity of post-fire early development stands likely aids in increasing the contagion value (Figure~\ref{fig:patchmaps3}). 

% figure redone 2015-09-20
\begin{figure}[!htbp]
  \centering
  \subfloat[][]{
    \centering
    \includegraphics[width=0.45\textwidth]{/Users/mmallek/Documents/Thesis/maps/hrv-agg-ts0.pdf}
    \label{fig:patchmaps3-ts0}
    }%
  \subfloat[][]{
    \includegraphics[width=0.45\textwidth]{/Users/mmallek/Documents/Thesis/maps/hrv-agg-24.pdf}
    \label{fig:patchmaps3-ts615}
    }
  \caption{Cover-Seral stage map focused on patches from Sierran Mixed Conifer - Mesic. (a) The current landscape. (b) The same region of the map during a randomly selected timestep after the equilibration period. Note the contrast between the two maps with respect to the contagion (at the cell level).} 
  \label{fig:patchmaps3} %year 1075 is timestep 215; year 675 is timestep 135
\end{figure}

However, despite having a higher edge-to-area ratio, and being more geometrically complex, patches during the simulated HRV still show an increase in core area over the present landscape (Figure~\ref{fig:fragland_core}). This indicates that the large patches that contain core area are sufficiently large to surpass the relatively high amount of core area on the present landscape. We expected that the current landscape might have large amounts of core area because current patches, bordered by roads and human-designed treatment unit boundaries, are likely to create simple shapes. Since this is an area-weighted measure, we conclude that it is the presence of many large patches that are large enough to contain significant core area that resulted in a simulated historical range of variability that does not overlap the current landscape. The results for the shape complexity metric confirm this analysis (Figure~\ref{fig:fragland_shape}). Especially among the largest patches on the landscape, convoluted shapes are common. Again, this is not to say that large and simple shapes do not occur---they do---but in comparison to the current landscape, complex shapes were characteristic of the simulated HRV (Figure~\ref{fig:patchmaps4}).

% figure redone 2015-09-20
\begin{figure}[!htbp]
  \centering
  \subfloat[][]{
    \centering
    \includegraphics[width=0.45\textwidth]{/Users/mmallek/Documents/Thesis/maps/hrv-complexpatch-2622.pdf}
    \label{fig:patchmaps4-ts555}
    }%
  \subfloat[][]{
    \includegraphics[width=0.45\textwidth]{/Users/mmallek/Documents/Thesis/maps/hrv-simplepatch2422.pdf}
    \label{fig:patchmaps4-ts690}
    }
  \caption{Cover-Seral stage map focused on patches from Sierran Mixed Conifer. (a) A large patch of Sierran Mixed Conifer - Xeric in the mid development open seral stage illustrates how very large patches (this one is 12,600 hectares) may have relatively small amounts of core for their shape, yet accumulate a lot of core because of their overall size. (b) Simpler shapes do exist, such as this patch of Sierran Mixed Conifer - Mesic in the mid development open seral stage, which has a lot of core area. However, they are often much smaller (this one is 2,320 hectares).}
  \label{fig:patchmaps4}
\end{figure}

Examining some of the class-level results for similar \textsc{Fragstats} metrics as for the landscape as a whole, we see consistencies and some interesting diversions (Figures~\ref{fig:smcm_areaam}--\ref{fig:smcx_clumpy}. In general, for the area-weighted metrics, if higher proportions of the landscape are occupied by a given seral stage, that cover-seral stage combination is likely to have high values for a given metric. Again, results for area-weighted mean patch area and mean core area are consistent. Early development patches are typically characterized by more complex shapes that are less aggregated during the HRV, as compared to the current landscape. This is due to the fact early development patches during the simulation are created by fires that are allowed to burn naturally, rather than by vegetation treatments. In addition, seral stages that became rare during the HRV tend to be smaller, aggregated and to have simple shapes. Larger shapes have a greater potential to have complicated shapes and to not be aggregated. This result may reflect the nature of the metrics at the class level as much as the seral stage structure. Overall, the class-level metrics reflect the interplay between wildfire and succession, and can best be used when considering management alternatives at the class level. Otherwise, the landscape-level interpretation is more appropriate when speaking generally. 


\subsection{Model evaluation and Sensitivity Analysis}
Ideally, the model would be evaluated using a hindcasting strategy and a sensitivity analysis on the final parameter set. Unfortunately, at the time of this study \textsc{RMLands} functionality for simulating vegetation treatments in the Sierra Nevada was still under development. Moreover, the existing descriptions of past vegetation treatments are not sufficiently detailed to truly test our model. A rigorous sensitivity analysis was not completed due to very long run times and large amount of data associated with using the model. However, we have a few observations about the relative sensitivity of some parameters based on our experience calibrating. First, the ignition parameter is quite sensitive; changing it by a few interval values changes all the outcomes. Presumably this happens because increasing the number of potential fire starts increases the odds of a fire initiating on a susceptible cell, although a formal evaluation of its effect is outside the scope of this project. In comparison, the fire return index is relatively insensitive; we often modified it by over an order of magnitude in order to effect a small change in the rotation outcome. 
%B says this is a future research topic, not a limitation. but I think most models should be validated and think Jack and Brad would agree. so not going to change this.

During model development we learned that the probability of high mortality fire at the seral stage level is relatively sensitive, which is logical because the conversion of forest to early seral conditions directly impacts most of the metrics by which we evaluate landscape structure and composition. Probabilities of high mortality fire at the seral stage level are extremely difficult to find in the literature because no record can be taken from a tree completely consumed in a fire. Only a few researchers have attempted to infer high severity fire based on factors such as later reports of dense young conifer or shrub cover \citep{Baker2014}. Most studies of stand-replacing fire have relied on satellite imagery to confirm ``stand-replacement'' \citep[e.g.,][]{Collins2010,Mallek2013}. Even if legacy trees\footnote{trees that are much older than the overall stand and that are presumed to have been left standing after a prior stand-replacing disturbance} existed and could be sampled, it would be difficult to determine when and over what extent past stand-replacing fires burned because of low sample sizes associated with legacy trees, as well as other factors such as post-fire drought \citep{Minnich2000,Baker2014}. Even the studies that derive percent high severity based on imagery produce overall cover type estimates, rather than estimates based on seral stage, which are needed to improve the model.

% becky said to move this from the results section, not sure how it'll fit in. Maybe as a limitation of the model or explanation of caveats?
In the previous sections, we describe the simulated disturbance regime in terms of its spatial extent and distribution, frequency, and temporal variability, for the landscape as a whole. Variations among vegetation types are described below and in Appendix~\ref{app:covtype_analysis}. Some results were used to evaluate whether the model was correctly calibrated; specifically, fire rotation values were used in model calibration, as described in Chapter~\ref{sec:hrvmethods}. These rotation values are also an outcome of the model, and are therefore reported here. We used rotation values as the calibration target because targets were available from the literature and because fire rotation is a fundamental measurement that \textsc{RMLands} was designed to capture. In addition, using rotation ties calibration to a parameter that is relatable to Forest Service staff and that can be used as a target by managers in various programs. In addition to the disturbance regime, this chapter includes results for the seral stage dynamics and analysis of the landscape configuration metrics from \textsc{Fragstats}.

\clearpage
\subsection{Management Implications}
%"Drawing comparisons between past and current fire frequencies can assist resource managers in prioritizing areas for ecological restoration, fuels reduction, certain fire or habitat management practices, and other activities. " (Hugh)

The primary goal of this study was to use our knowledge and understanding of vegetation dynamics and disturbance processes to simulate changes to landscape composition and configuration over time under a historical reference framework. We then compared those dynamics to observations of current conditions, and assessed the departure from the historic range of variability. Our landscape-level assessment is that both composition and configuration deviate substantially from the HRV. In general, the current landscape is dominated by Mid and Late Development forest and lacks the fire-dependent open canopy stand conditions (Mid-Open, Late-Open) and spatial heterogeneity in vegetation that were maintained by natural disturbances during the reference period. The departure from the HRV is probably due in large part to land management practices, especially fire suppression an timber management, over the last 150 years \citep{Safford2014,Stephens2007}. Substantial changes to the ecology and landscape function in western forest landscapes have been documented not only in the Sierra Nevada, but also in the Cascades and Rocky Mountains \citep{Hessburg2005,Baker2012,Baker2014,Mallek2013,Agee1993} . However, no other studies have been completed focusing on the Tahoe National Forest that incorporate spatially explicit disturbance and succession modeling in combination with analyses of landscape structure.

In the western Sierra Nevada, foothill communities and lower elevation oak-conifer woodlands have experience a loss of species diversity, fragmentation, and outright habitat conversion due to the overlap with private lands and population growth. Middle elevation forests were and are more affected by mining and forestry; most easily accessible trees were probably cut before national forests were established \citep{SNEP1996}. The wildfire regime has been significantly altered in hardwoods, yellow pine, and mixed conifer forests \citep{Merriam2013,Safford2013}, and much less so in red fir and subalpine forests \citep{Meyer2013,Meyer2013a}. However, other human activities since the late 1800s have altered the structure of western Sierra Nevada forests, most notably to simplify it in several ways, including a decrease in species, multi-story canopies, and snags \citep{SNEP1996}. These activities are related mainly to timber harvest and to the extensive network of roads constructed to support timber harvest, fire control, and recreation. It has been suggested that this simplification of landscape structure may have a negative impact on wildlife and potentially lead to a loss of biodiversity in forests \citep{Thompson2003,Manley2004,Hunter2011}. Isolating the effect of fragmentation in our study landscape is made more difficult due to the inherent heterogeneity of Sierran landscapes---a consequence of steep natural gradients in elevation, topography, and substrate---and forests in this region tend to be somewhat patchy even in the absence of human alterations \citep{Franklin1996}. In the Pacific Northwest, ``old-growth'' connotes very large blocks of uniformly very old trees. However, in the Sierra Nevada ``old-growth'' indicates not only the presence of very large and old trees, but also a complex, patchy, ``messy'' forest of varying age classes, species, fuel quantities, and vegetation structure \citep{SNEP1996}.


Management actions on the Tahoe National Forest likely directly altered the landscape in a few specific ways. First, fire suppression facilitates the closure of forest canopies, leading to more closed and less open canopy cover across all cover types. Second, it indirectly reduces the amount of the Early seral stage, particularly in areas not accessible or appropriate for timber harvest. Third, it contributes to a reduction in patch complexity over time because the irregular borders of a natural fire would not become irregular borders of cover-seral stage patches. Timber management could also have promoted departure from the HRV. Boundaries of timber sales tend to be linear, rather than irregular, affecting patch complexity. Fine-scale heterogeneity was also reduced as a result of managing forests to be composed primarily of the most valuable timber species, as well as easily accessible and harvested, rather than complexly structured to provide habitat for a range of plants and animals. While substantial, thinning and harvest operations affected only a small fraction of the amount of forest that probably burned annually during the historical period, so the likelihood that timber management activities compensated for the loss of stand opening and stand initiation effects from fire is low. Active replanting after cutting or after fire would also reduce the average area in the Early stage at any given moment %\citep{Dellasala2014}
. In addition to these impacts, the Tahoe National Forest has been managed to promote the development of late successional, old growth forest habitat, in order to provide habitat for species dependent on it, like the spotted owl \citep{USDAForestService2004a}. This also pushes the landscape into an older, more closed condition, since both fire suppression and a lack of timber harvest affect the character of these focal old growth areas \citep{Franklin1996}.


%Based on our results, it might be tempting for managers to reach the simple conclusion that the landscape is less fragmented today than during the reference period. For example, today's landscape is simpler (lower \textsc{shape} values), contains less core area (based on \textsc{core\_am}), and has less contrast between patch types (\textsc{econ\_am}) (Figures~\ref{fig:fragland} and \ref{fig:patchmaps1}--\ref{fig:patchmaps4}). However, this conclusion is not as straightforward as it might seem. Fragmentation is a landscape-level process in which a specific habitat is progressively sub-divided into smaller, geometrically altered, and more isolated fragments as a result of both natural and human activities. This process involves changes in landscape composition, structure, and function at many scales and occurs on a backdrop of a natural patch mosaic created by vegetation transitions, both those mediated by and independent from natural disturbances \citep{McGarigal1995}. The scale at which fragmentation occurs is at the level of a specific habitat type; it is the habitat, rather than the landscape, that becomes fragmented. In this study we evaluated the spatial pattern---and by implication, the fragmentation---of many different patch types (defined by unique combinations of cover type and condition class). Certaintly, some of these patch types are less fragmented in the current landscape than they were under the simulated HRV\todo{B: What is train of thought? Like overview of configuration and why not simple to implement at management level}. 


In addition, we evaluated vegetation patterns in the current landscape after excluding roads (i.e., we removed roads from the land cover map by filling in those areas with the nearest alternate cover type). Figure \ref{fig:roadcovermap} shows the cover type layer with roads overlaid. We did this in order to be consistent with our simulation of landscape structure changes during the reference period. However, the significant and extensive impact of roads on Sierran ecosystems is well documented in the literature \citep{Karr2004,Trombulak2000,Gucinski2001,Theobald2011}.  In particular, roads are linear landscape features that can create high-contrast edges and bisect patches. They may cause greater fragmentation of habitats than the direct loss of habitat from associated land use activities \citep{Gucinski2001,Tinker1998,Mcgarigal2001}. Given the large amount of roads within the project area and their disproportionate influence on landscape structure and function, it is important to consider the likely impact of roads on configuration metrics. In many cases, at least at the local scale, including roads would probably exacerbate the difference between the current landscape and the HRV.
%
\begin{figure}[!htbp]
  \centering
  \includegraphics[height=.4\textwidth]{/Users/mmallek/Tahoe/Report2/images/roads_cover.png}
  \caption{The core project area superimposed with the cover type map and all roads in black. There are a two designated roadless areas, but in general roads are common throughout the watershed. The closeup area is just northeast of the Pendola reservoir.} 
  \label{fig:roadcovermap}
\end{figure}
%


%"Drawing comparisons between past and current fire frequencies can assist resource managers in prioritizing areas for ecological restoration, fuels reduction, certain fire or habitat management practices, and other activities. " (Hugh)

Our results imply that any attempt to restore the landscape structure to a composition and configuration that aligns with the historic range of variability described in this thesis would likely be a difficult and long-term undertaking. Our model equilibration period is the length of the interval between the starting condition and structure (in this case, of the current landscape) and beginning of a stable range of variation. It is a function of not only how far outside the stable range of variation the current landscape is, but also the speed at which disturbance and succession processes interact to affect a change in the landscape trajectory. Thus, we can infer that if management activities were designed to emulate natural disturbance processes, then it would take a length of time equal to the equilibration period to return the landscape to its HRV. The equilibration period for our simulation was 40 timesteps, or 200 years. Given that forest planning horizons are on timescales of 5--30 years, and the fact that climate change is predicted to have measurable impacts on both wildfires and vegetation community succession and structure over the next several decades, an effort to ``restore'' the current landscape to the exact conditions aligned with our simulated HRV may not be practical \citep{Millar1999,Millar2014}. Moreover, the extent and intensity of disturbance required to emulate the natural disturbance regime is significant, and a simple restoration of the historical fire regimes would not be possible with public input and environmental consultation. Certainly, social, economic, and political challenges exist. However, as several ecologists have pointed out, our results can still be used to guide the prioritization of areas for restoration, for designing fire and fuels management projects, or describing desired future conditions \citep{Safford2013,Keeley2000}.



\subsection{Individual Cover Type Recomendations}
Our study methodology makes it possible to single out separate cover types and seral stages for analysis, and in this section we provide some interpretation of those results that is targeted toward providing recommendations for specific strategies that could be used to push landscape structure toward the HRV, and other recommendations of management strategies to avoid. Because in reality cover types are interspersed with one another, managers will need to consider the recommendations for individual cover types within the context of the vegetation actually present within a given management unit.

%\subsubsection{Mixed Evergreen - Mesic}
%Mesic mixed evergreen forests in our simulation burned with a fire rotation of 55 years under historical conditions. Low mortality fire was much more common than high mortality fire, although both are necessary to create stand structure and composition similar to that observed during the simulation. Late development stands (predominantly closed canopy) of mixed evergreen were dominant during the simulation, especially compared to the current landscape. The current landscape also contains more area in the Early Development condition (Figure \ref{fig:covcondbar_megm}. 

%Based on these observations, we recommend management strategies that imitate low-mortality disturbance; that is, removing less than 70\% of the existing top-level canopy cover. Such a strategy would provide a mechanism for the forest to age into the late development conditions. It is not necessary for all stands of mesic mixed evergreen forests to receive the same treatment on the same schedule. In certain parts of the forest, fires would have recurred more frequently than the 55 year rotation (see Figure \ref{fig:preturn_megm}). The point-specific fire return interval for low mortality fires within this cover type ranged from around 20 years to 130 years. Managers who are charged with focusing fuels reduction on certain areas within the forest could set goals of carrying out vegetation treatments somewhat more frequently in certain parts of the forest and less frequently in other parts, thereby also contributing to the overall more complex landscape pattern observed during the simulated HRV. 

%With respect to the landscape structure metrics (computed using \textsc{Fragstats}), we highlight the values for the Late--Closed and --Moderate conditions, since they dominated during the HRV simulation. We observed that present-day patches of mesic mixed evergreen forests in these conditions were smaller, more clumped, less geometrically complex, and contained less core area than during the simulated HRV. Having less core area and being more clumped may seem contradictory, but this outcome is likely due to the smaller size of the current patches. Restoration of these forests to patches that reflect a more natural succession process may be challenging for managers, given practical needs like using roads and riparian buffers as the edges of treatment units. It may not be practical to perform mechanical treatments over large areas within this cover type. However, when conducting treatments using prescription fires, creative solutions should be sought to generate more complex edges and to complete burns over sufficiently large areas that larger core areas can be generated.

%Because the most commonly occuring condition classes for mesic mixed forests were actually late development closed and moderate canopy cover, it may also be more helpful for managers to think about this cover type not as receiving direct treatment, but as being influenced by other treatments targeting cover types such as oak-conifer forests and woodlands or sierran mixed conifer forests. When mesic mixed evergreen forests border or serve as the edge of treatment units targeting other vegetation types, managers can evaluate the patches of mesic mixed evergreen forests present at the time of treatment and consider whether to expand the management unit to treat them, to try and achieve an irregular edge to increase patch complexity, or to exclude them from treatment in order to promote development into the late development conditions.                                                                                                                                                                                                                                                                                                                                                                                                                                                    
\subsubsection{Sierran Mixed Conifer - Mesic}
Mesic mixed conifer forests are thought to have burned with a fire rotation of 27 years under historical conditions. Low mortality fire was much more common than high mortality fire, but both were necessary to create stand structure and composition similar to that observed during the simulation. 

All seven seral stages were well represented during the simulated historical period. The most common condition (highest median value) during the simulation was Mid--Closed. The proportion of mesic mixed conifer on the current landscape is close to the median values for Early, Mid--Open, and Late--Moderate. There is much more Mid--Moderate and Late--Closed, and much less Mid--Closed and Late--Open, now compared to the simulated HRV. Based on these observations, in the short term we recommend maintaining existing open stands of mesic mixed conifer though prescribed fire and other understory vegetation treatments. Because patch sizes are much smaller now on average, when identifying stands to push towards, for example, Mid--Closed or Late--Open, we recommend looking for treatment units near or adjacent to existing patches of the target seral stage, in order to create large patches in that type. Overall, closed canopy forest across development classes is near the median values for closed canopy seral stages during the simulated HRV. We do not believe more closed canopy forest is needed to bring the landscape toward HRV conditions. That said, it probably makes sense to work to create conditions for younger closed canopy forests, so that all the closed canopy stands on the forest do not belong to the oldest age class.

In the medium term, we recommend the restoration of fire wherever practicable. Fire was quite common on the historical landscape and was the main driver of the complexity of patches we see in the simulated landscape. The point-specific fire return interval for low mortality fire ranged greatly, from 18 years to around 100 years. Managers can transfer this variability into flexibility when planning and executing vegetation treatments and wildfire response. In fact, the spatial variability of fire is critical for creating spatial variability of forests and plants observed as outputs of the simulation. Managers who are charged with focusing fuels reduction on certain areas within the forest could set goals of carrying out vegetation treatments somewhat more frequently in certain parts of the forest and less frequently in other parts, thereby also contributing to the overall more complex landscape pattern observed during the simulated HRV. 

With respect to the landscape structure metrics (computed using \textsc{Fragstats}), we observed that present-day patches of mesic mixed conifer forests in these seral stages were smaller, more fragmented, less geometrically complex, and contained less core area than during the simulated HRV. Restoration of these forests to patches that reflect a more natural succession process may be challenging for managers, given practical needs like using roads and riparian buffers as the edges of treatment units. It may not be practical to perform mechanical treatments over large areas within this cover type. However, when conducting treatments using prescription fires, creative solutions should be sought to generate more complex edges and to complete burns over sufficiently large areas that large core areas are a byproduct of the treatment.
       
\subsubsection{Sierran Mixed Conifer - Xeric}
Xeric mixed conifer forests are thought to have burned with a fire rotation of 23 years under historical conditions. Low mortality fire was more common than high mortality fire, but both were necessary to create stand structure and composition similar to that observed during the simulation. 

In contrast to the results for Sierran Mixed Conifer - Mesic, we observed a consolidation of the area in this cover type into the Early, Mid--Open, and Late--Open seral stages. Mid--Closed, Mid--Moderate, and Late--Closed were present in very low amounts, while Late--Moderate was well established but not dominant. This cover type is completely departed from the HRV. Based on these observations, in the short term we recommend maintaining existing early development and open stands of xeric mixed conifer through prescribed fire and other understory vegetation treatments. Harvest treatments should focus on areas now in closed canopy conditions. Because patch sizes are much smaller now on average, when identifying stands to push towards and open canopy structure, we recommend looking for treatment units near or adjacent to existing open patches in order to create larger patches overall. Of course, it is also important for fuels specialists, foresters, and biologists to work together to include fine-scale structural complexity in these large patches of open canopy stands.

In the medium term, we recommend the restoration of fire wherever practicable. Fire was quite common on the historical landscape and was the main driver of the complexity of patches we see in the simulated landscape. The point-specific fire return interval for low mortality fire ranged greatly, from 17 years to around 80 years. Managers can transfer this variability into flexibility when planning and executing vegetation treatments and wildfire response. In fact, the spatial variability of fire is critical for creating spatial variability of forests and plants observed as outputs of the simulation. Managers who are charged with focusing fuels reduction on certain areas within the forest could set goals of carrying out vegetation treatments somewhat more frequently in certain parts of the forest and less frequently in other parts, thereby also contributing to the overall more complex landscape pattern observed during the simulated HRV. 

With respect to the landscape structure metrics (computed using \textsc{Fragstats}), we cannot make generalizations about all of the seral stages, because some were nearly absent during the simulated historical period. Instead we focus on the most common stages: Early, Mid--Open, and Late--Open. Present day patches of these seral stages in xeric mixed conifer forests are smaller, more fragmented, less geometrically complex, and contain less core area than during the simulated HRV. Restoration of these forests to patches that reflect a more natural succession process may be challenging for managers, given practical needs like using roads and riparian buffers as the edges of treatment units. It may not be practical to perform mechanical treatments over large areas within this cover type. However, when conducting treatments using prescription fires, creative solutions should be sought to generate more complex edges and to complete burns over sufficiently large areas that large core areas are a byproduct of the treatment.










