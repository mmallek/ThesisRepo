% !TEX root = master.tex

\section{Discussion and Management Implications}
\label{sec:hrvdiscussion}




%%%%%%%%%%%%%%%%%%%%%%%%%%%%%%%%%%%%%%%%%%%%%%%%%%%%%%%%%%%%%%%%%%%%%%%%%%%%%
%%%%%%%%%%%%%%%%%%%%%%%%%%%%%%%%%%%%%%%%%%%%%%%%%%%%%%%%%%%%%%%%%%%%%%%%%%%%%


\subsection{Overall Landscape Assessment}

Fires during the simulated historical period burned far more frequently and across larger extents than at any time since record keeping began (according to available fire history data \citep{calfire2012,usgs-fire-data2012}) within the Yuba River watershed study area (see Tables~\ref{tab:darea_atleast} and \ref{tab:darea}, and Figure~\ref{fig:distid_median} and \ref{fig:distid_mean}). Most of the simulated fires were small and predominantly low mortality. However, large fires did occur, which indirectly affected patch configuration and directly affected the seral stage composition of cover types. Five-year periods (timesteps) in which around half the study area burned were very rare, indicating that occasional extremely widespread fire is a characteristic of the fire regime in this study area. Fire size was weakly but positively associated with the climate parameter (bigger fires when drought conditions were more severe). However, fire size is also influenced by vegetation susceptibility and the specified distribution of disturbance size. For this reason, large areas burned in relatively ``wet'' timesteps. 

The resulting wildfire regime, though dramatic, reflects the historical period prior to modern fire management. In the model, fires are regulated only by local vegetation, topography, barriers, and weather (as represented by stochastically generated potential fire sizes). Although this can and does result in some very large fires, extending across tens of thousands of hectares, more typically I observed fairly small fires. This agrees with empirical data showing that when fire is frequent, it tends to be low severity and is often small in extent \citep{Kilgore1979,Taylor2012}. The large fire sizes observed reflect the fact that the simulation covers a very large spatial extent, and thus results in many more acres of fire burning per timestep than typically burn today \citep{calfire2012,usgs-fire-data2012}. Regardless, because I had such a high degree of control over the fire regime, the particular disturbance regime results primarily confirm that the model was functioning appropriately and as designed. 

% rewrote 2015-09-20
With respect to the seral stage distribution, a few patterns emerged (Full results for the nine focal cover types can be found in Appendix~\ref{sec:indiv_cov_results}.) Overall, the current study area composition departs from the HRV not only at the landscape scale, but also at the cover type and seral stage level. Each seral stage of the xeric mixed conifer type is completely departed from the simulated HRV. Mesic mixed conifer forests are currently less departed overall. Three of the seral stages (Mid--Open, Late--Closed, and Late--Moderate) have distributions that indicate that the current landscape is moderately departed within the HRV, and the current proportion of Early Development mesic mixed conifer forest is close to the median value observed over the course of the simulation. However, the current proportion of the landscape covered by the three remaining seral stages is completely departed from the simulated historical distributions.

%rewrote again to remove non-smc types on 2016-01-14
In both the mesic and xeric mixed conifer types, as well as the other cover types studied closely, Late--Open conditions were far more common during the HRV than on the current landscape. In addition, early succesional conditions were more common during the simulated historical period than on the current landscape for xeric mixed conifer forests. This increase in the proportion of the study area belonging to Early Development and open canopy forests is directly connected to the frequent and extensive pattern of wildfires burning in the HRV scenario (Figure~\ref{fig:darea_smcx}, Tables~\ref{tab:darea_smcx} and \ref{tab:darea_atleast_smcx}). Mesic mixed conifer forests experienced somewhat less fire than the xeric type, but wildfires were still extensive (Figure~\ref{fig:darea_smcm} and Tables~\ref{tab:darea_smcm} and \ref{tab:darea_atleast_smcm}). The large amount of Mid--Closed in this cover type may be due to a combination of its relatively low susceptibility, relatively short early period (averaging 26 years), and the fact that Early Development is twice as likely to succeed to Mid--Closed as to Mid--Open (Appendix~\ref{smc-description}).


%rewrote this 2015-09-20
%Interestingly, early seral conditions were less common during the simulation than on the current landscape for the mixed evergreen and ultramafic mixed forests, but more common for the xeric mixed conifer forests; for the other types, however, the proportion of early seral is within the HRV. Both mixed evergreen types and both oak-conifer types have a smaller proportion of forest in late development now than during the simulation. Late--Open conditions were far more common during the HRV. For all cover types except for mesic mixed evergreen forests, Late--Open conditions were far more common during the HRV, and the current landscape is fully departed from the HRV. The Mid--Open was more common for xeric and ultramafic mixed conifer forests during the simulation than today, while Mid--Closed was dominant during the HRV for mesic mixed conifer forests, but now is not. Finally, closed canopy conditions were much more dominant in mesic red fir forests than on the present landscape; these stages are also completely outside the HRV.

%-- Closed canopy ocfw dominant now, but not during HRV.  (didn't use)
%-- Closed conditions were much more dominant in mesic red fir forests.
%-- Mid open more common for smcx and smcm
%-- Mid closed dominant during hrv for Smcm, but not now
%-- Mixed evergreen forests and oak-conifer forests and woodlands generally have less area in late development now than during the simulated HRV.
%-- Early seral was much more prevalent in SMCX and somewhat more prevalent in RFRX, but consistent with the other cover types.
%-- Early more common for smcx
%-- Late open more common for all cover types except rfrm, always outside of hrv
%-- Mixed evergreen and xeric mixed conifer had no seral stages within the HRV.
%-- 12 of 63 cover type - seral stage combinations with the HRV
%-- Only OCFWU-Early, RFRX-LDC, SMCM-Early, SMCU-MDM, were within interquartile range.
%-- Being completely outside the simulated HRV is the norm for seral stages with the focal nine areas. In many cases, the current proportion of a given seral stage is within the HRV, but this generally means either between the 5th and 25th percentiles, or between the 75th and 95th percentiles. In just four cases (\textsc{ocfw\_u} early, \textsc{rfr\_x} late closed, \textsc{smc\_m} early, and \textsc{smc\_u} mid moderate) was the current proportion within the 25th-75th interquartile range.


An analysis of configuration metrics at the landscape scale yields insight into the historical period. First, I note that compared to the present conditions, during the HRV the study area was composed of larger and more extensive patches, as illustrated by Figure~\ref{fig:fragland1}. This trend was heavily influenced by the presence of wildfires on the landscape, as high mortality fire in particular created large areas of Early Development vegetation (Figure~\ref{fig:patchmaps1-early}). However, I also observed large patches in the other seral stages, which were more likely to form long or convoluted patches that were nonetheless extensive (Figure~\ref{fig:patchmaps1-mid}).

% figure redone 2015-09-20
\begin{figure}[!htbp]
  \centering
  \subfloat[][]{
    \centering
    \includegraphics[width=0.45\textwidth]{/Users/mmallek/Documents/Thesis/maps/hrv-largepatch2410.pdf}
    \label{fig:patchmaps1-early}
    }%
  \subfloat[][]{
    \includegraphics[width=0.45\textwidth]{/Users/mmallek/Documents/Thesis/maps/hrv-largepatch2430.pdf}
    \label{fig:patchmaps1-mid}
    }
  \caption{(a) A large patch, bordered in dark grey, of Sierran Mixed Conifer - Mesic in Early Development. The patch is 5,780 hectares, one of the second-largest patch during this timestep. (b) Two patches of Sierran Mixed Conifer - Mesic in Late--Closed. These are two of the largest patches during that timestep, at 10,930 hectares and 4,750 hectares.} 
  \label{fig:patchmaps1}
\end{figure}


% big 2622 12,600

In addition, I observed increased dominance by certain cover-seral stage types during the HRV, which is likely what led to smaller values for the Simpson's Evenness Index (Figure~\ref{fig:fragland_siei}). For example, within the Sierran Mixed Conifer - Xeric cover type, Early Development and Mid--Open were much more widespread during the simulated HRV than in the current landscape (Figure~\ref{fig:patchmaps2}). Because Sierran Mixed Conifer - Xeric is so widespread, this shift would directly influence Simpson's Evenness, lowering its value.

% figure redone 2015-09-20
\begin{figure}[!htbp]
  \centering
  \subfloat[][]{
    \centering
    \includegraphics[width=0.45\textwidth]{/Users/mmallek/Documents/Thesis/maps/hrv-dominance-ts0.pdf}
    \label{fig:patchmaps2-ts0}
    }%
  \subfloat[][]{
    \includegraphics[width=0.45\textwidth]{/Users/mmallek/Documents/Thesis/maps/hrv-dominance-26.pdf}
    \label{fig:patchmaps2-ts570}
    }
  \caption{Cover type-Seral stage map focused on patches from Sierran Mixed Conifer - Xeric, showing increased dominance by certain cover type-seral stage types during the HRV. (a) The current landscape. (b) The same region of the map during a randomly selected timestep after the equilibration period. Note the contrast between the two maps with respect to the seral stages and size of individual patches.} 
  \label{fig:patchmaps2}
\end{figure}


Third, I find that patches on the landscape were more aggregated at the cell-level during HRV, which is illustrated by the Contagion metric (Figure~\ref{fig:fragland_contag}). In general, patches have low levels of both dispersion and interspersion. Of course, there are many ``edgy'' areas on the landscape, but this metric indicates that across the full landscape aggregation is more typical, particularly in comparison to the current landscape. Again, the homogeneity of post-fire early successional stands likely aids in increasing the contagion value (Figure~\ref{fig:patchmaps3}). 

% figure redone 2015-09-20
\begin{figure}[!htbp]
  \centering
  \subfloat[][]{
    \centering
    \includegraphics[width=0.45\textwidth]{/Users/mmallek/Documents/Thesis/maps/hrv-agg-ts0.pdf}
    \label{fig:patchmaps3-ts0}
    }%
  \subfloat[][]{
    \includegraphics[width=0.45\textwidth]{/Users/mmallek/Documents/Thesis/maps/hrv-agg-24.pdf}
    \label{fig:patchmaps3-ts615}
    }
  \caption{Cover type-Seral stage map focused on patches from Sierran Mixed Conifer - Mesic. (a) The current landscape. (b) The same region of the map during a randomly selected timestep after the equilibration period. Note the contrast between the two maps with respect to contagion (at the cell level).} 
  \label{fig:patchmaps3} %year 1075 is timestep 215; year 675 is timestep 135
\end{figure}

Despite having a higher edge-to-area ratio, and being more geometrically complex, patches during the simulated HRV still show an increase in core area over the present landscape (Figure~\ref{fig:fragland_core}). This indicates that the large patches that contain core area are sufficiently large to surpass the relatively high amount of core area on the present landscape. I expected that the current landscape might have large amounts of core area because human-designed management unit boundaries are likely to create simple shapes, especially when they incorporate linear features such as roads or streams. Since this is an area-weighted measure, I conclude that it is the presence of many large patches that are large enough to contain significant core area that resulted in a simulated historical range of variability that does not overlap the current landscape. The results for the shape complexity metric confirm this analysis (Figure~\ref{fig:fragland_shape}). Especially among the largest patches on the landscape, convoluted shapes are common. Again, this is not to say that large and simple shapes do not occur---they do---but in comparison to the current landscape, complex shapes were characteristic of the simulated HRV (Figure~\ref{fig:patchmaps4}).

% figure redone 2015-09-20
\begin{figure}[!htbp]
  \centering
  \subfloat[][]{
    \centering
    \includegraphics[width=0.45\textwidth]{/Users/mmallek/Documents/Thesis/maps/hrv-complexpatch-2622.pdf}
    \label{fig:patchmaps4-ts555}
    }%
  \subfloat[][]{
    \includegraphics[width=0.45\textwidth]{/Users/mmallek/Documents/Thesis/maps/hrv-simplepatch2422.pdf}
    \label{fig:patchmaps4-ts690}
    }
  \caption{Cover type-Seral stage map focused on patches from Sierran Mixed Conifer. (a) A large patch of Sierran Mixed Conifer - Xeric in the mid development open seral stage illustrates how very large patches (this one is 12,600 hectares) may have relatively small amounts of core for their shape, yet accumulate a lot of core area because of their overall size. (b) Simpler shapes do exist, such as this patch of Sierran Mixed Conifer - Mesic in the mid development open seral stage, which has a lot of core area. However, they are often much smaller (this one is 2,320 hectares).}
  \label{fig:patchmaps4}
\end{figure}

Examining some of the class-level results for similar configuration metrics, I observed many consistencies and a few interesting diversions (Figures~\ref{fig:smcm_areaam}--\ref{fig:smcx_clumpy}). In general, for the area-weighted metrics, if higher proportions of the landscape are occupied by a given seral stage, that cover type-seral stage combination is likely to have high values for a given metric. Again, results for area-weighted mean patch area and mean core area are consistent. Early Development patches are typically characterized by more complex shapes that are less aggregated during the HRV, as compared to the current landscape. This is due to the fact that during the simulation early successional patches are created by fires that are allowed to burn naturally, rather than by vegetation treatments with predetermined, linear boundaries. In addition, seral stages that became rare during the HRV tend to be smaller, aggregated and to have simple shapes. Larger shapes have a greater potential to have complicated shapes and to be disaggregated. This result may reflect the nature of the metrics at the class level as much as the seral stage structure. Overall, the class-level metrics reflect the interplay between wildfire and succession, and can best be used when considering management alternatives at the class level. However, the landscape-level interpretation is the most suitable for evaluating the study area and the potential impact of management actions, including restoration. 

\clearpage

\subsection{Comparison to Other Studies}
Substantial changes to the ecology and landscape function in western forest landscapes have been documented not only in the Sierra Nevada, but also in the Cascades and Rocky Mountains \citep{Agee1993,Hessburg2005,Baker2012,Mallek2013,Baker2014}. Several recent studies have attempted to characterize the historical fire regime and vegetation patterns based on empirical data and inferences derived from it. My results agree with some of these empirically-derived results, and diverge from others.
%
\citet{Collins2010} evaluated the spatial fire effects of two wildfires that burned in a part of Yosemite National Park that has allowed wildfire for the past 35 years. The purpose of the study was to better describe ``mixed severity'' fires. They found that patches of stand-replacing fire occurred in a range of sizes, skewed towards smaller patches, within a broader area that burned at lower severity.
%
\citet{Collins2011} and \citet{Stephens2015} compared data from a 1911 timber inventory to forest conditions from 2005--2007. The resurveyed plots were characterized by higher canopy cover and tree densities than in the 1911 survey. They also inferred that historically, fires burned at varying densities, and that large high severity patches were extremely rare \citep{Collins2011,Stephens2015}.
%
\citet{Baker2014} conducted an extensive review of the 1865--1885 General Land Office surveys, using the quantitative and qualitative data therein to reconstruct historical forest structure and fire in Sierran mixed conifer forests. They concluded that evidence of high severity fire was present across 31--39\% of the Sierra, a point disputed by \citet{Fule2014} based most strongly on the assumption by the \citet{Baker2014} study that the presence of dense stands of small trees were evidence of high severity fire. 
%
My study is unique in its incorporation of spatially explicit disturbance and succession modeling in combination with analyses of landscape structure; the studies above did not include any simulations. In addition, my study focused on the Tahoe National Forest specifically, an area often not included in other studies, perhaps in part due to its history of active management and use, which makes field-based studies difficult to use for making inferences about historical systems.

I first compare the results from my simulations to those reported in other papers. As described in Section \ref{subsec:hrvmodelparam}, in this study, high severity fire is defined as fire that kills 75\% or more of the overstory canopy cover. This threshold has been established for many years within the literature \citep{Agee1993,Agee2007,Miller2009,Baker2014}. Recently, some researchers have argued that 95\% is a better threshold for defining high severity fire, which is also sometimes referred to as a stand-replacing event \citep{Mallek2013,Fule2014}. However, this is a newer methodology, and unless highlighted by authors, I have assumed that mentions of high severity fire refer to the older 75\% and above threshold.

The proportion of high mortality fire in the simulated HRV falls at the high end of those presented in the literature, and falls closest to the 39\% calculated by \citet{Baker2014}, which includes, as does my study, chaparral fields as part of the mixed conifer landscape. \citet{Collins2010}, in a study of two recent fires, found that the proportion of high severity fire was 15\%, which is similar to the proportions of high severity fire in the open and moderate seral stages parameterized in the model. Both studies assessed the landscape at a particular point in time, thus producing estimates of high severity fire that reflect a snapshot. My range of variability analysis results included similar outcomes to these two studies, within my observed range of variability. In this sense my results agree with both studies, and more importantly, highlight the importance of examining the range of variability instead of only a snapshot, which can overly focus attention on precise statistical measures.

In contrast, the assertion by \citet{Collins2011} that high severity fire was nearly absent in mixed conifer forests during the historical period is not supported by my results (though it should be noted that my parameterization did not reflect an assumption of very rare high severity fire). The proportions of high severity fire of less than 10\% put forward by \citet{Mallek2013} also contrast sharply with my results. This may be due in part to an apparent southern bias in the \citet{Mallek2013} analysis: many of the reference sites and literature used to support this number are taken from the southern Sierra Nevada or Baja California. There are many differences between the northern and southern Sierra, from precipitation patterns, to elevation, to the presence of natural fire breaks, that should lead to differences in the fire regimes \citep{Stephens2015}. The relatively high amounts of precipitation and high productivity of the Tahoe National Forest in particular also contribute to its fire regime, and may account for some of the observed difference \citep{PRISMClimateGroup2004,VanWag2006,Littell2012}. One of the main points made by \citet{Baker2014} is that an \emph{exclusively} low severity fire regime does not accurately characterize historical Sierran mixed conifer forests. In my view, the 6--8\% high severity fire reported by \citet{Mallek2013} is essentially an exclusive low severity regime. During early phases of model development for this study, project partners and I considered the consequences of using probabilities of high severity fire that small, and concluded that using such values would lead to unrealistic outcomes. Therefore, higher probabilities of high severity fire were used in model parameterization. These probabilities were sourced from \citet{Landfire2007} and differ by seral stage. Probabilities used in model parameterizations at the seral stage level variously match the values proposed by \citet{Collins2010}, \citet{Mallek2013}, and \citet{Baker2014}.

As can be inferred from the range of variability in the size of Early Development patches (Figures~\ref{fig:fragclass_smcm} and \ref{fig:fragclass_smcx}), the size of individual patches of high severity fire within larger fire perimeters ranged widely. Furthermore, these high severity patches are a critical component of fire in these forests because of their impacts to vegetation \citep{Collins2010}. This is in agreement with the description of high severity fire as occurring in a heterogeneous and complex manner \citep{Keeley2000,Hessburg2005,Collins2010,Baker2014}. \citet{Stephens2015} argue that high severity fire is rare, but their study was focused on the extreme southern portion of the Sierra Nevada, and the results may not be as pertinent to the northern Sierran mixed conifer forests that occur in my study area. In addition, the authors highlight the coarse resolution of their methodology as a limitation of their study. \citet{Stephens2015} acknowledge that in a forest with such low incidences of high severity fire, most trees would be 200--400 years old, and mortality would come from stressors other than fire. The highly complex forests that most researchers agree typify the northern Sierra \citep{Chang1995} could not have come from an exclusively low severity wildfire regime unless some other disturbance was in fact the dominant structuring force.

\citet{Fule2014} raise several issues with the \citet{Baker2014} methodology that they assert undermine conclusions drawn from them. First, \citet{Fule2014} claim that the Baker papers \citep{Baker2012,Baker2014} conclude that ``present-day large, high-severity fires are not distinguishable from historical patterns.'' It is important to distinguish patterns from aggregations. The claim in \citet{Baker2012} is that the proportion of high severity fire has not changed, not that its pattern is the same. \citet{Fule2014} also take issue with the use of tree size to infer stand age, and with it evidence of past disturbance. They support this claim with references to several studies of various forest types that demonstrated the issues with assuming tree size perfectly predicts stand age. However, associating stand age with tree size is a heuristic used in many studies \citep{WHR1988,Landfire2007,USDAForestService2009}, including this work; in the absence of dendrochronology, I am unaware of other methods of inferring age from timber surveys.

Critiques from \citet{Fule2014} about fire severity and comparing Monitoring Trends in Burn Severity (MTBS, \burl{mtbs.gov}), are also critiques of all historical studies. They appear to argue that a comparison of recent fire severity to historical fire severity cannot be done using the MTBS data, since no historical study could hope to conduct a severity assessment that would be analogous to MTBS. \citet{Fule2014} also assert that a 75\% overstory mortality threshold for designating high severity fire is inappropriate, yet this is the definition used in the seminal work by \citet{Agee1993} in his fire ecology study, as well as that employed when creating the Vegetation Dynamics models used in \citet{Landfire2007}. While this critique may be valid, the use of this threshold by \citet{Baker2014} is not unique in the field, and is not reason enough to dismiss his results. 

Finally, \citet{Fule2014} do not mention the incorporation of chaparral as an indicator of high severity fire. The \citep{Baker2014} methodology uses chaparral existence as evidence of past fire, as does mine. Incorporating chaparral into a cover type map as an early stage of forest by definition increases the area in an early successional state, as well as the area assumed to have been affected by high severity fire. Reburning of chaparral also usually results in high severity (high mortality in my model, see Appendix~\ref{app:covertypedesc}) fire. The similarity of my results to those reported by \citet{Baker2014} are likely related to our similar treatment of chaparral.

I observed high mortality fire rotations (113 years and 72 years for mesic and xeric mixed conifer forests, respectively) that are much shorter than the 281 year rotation calculated by \citet{Baker2014} for mixed conifer forests in the northern Sierra Nevada. Before making inferences about the implications of the fire rotations found by myself or \citet{Baker2014}, other researchers should consider viewing the seral stage distribution dynamics, and potentially the output maps, from simulations like mine. Based on my results, I would predict that fire rotations even longer than the 281 years proposed by Baker would yield a mixed conifer forest almost entirely composed of old-growth forest, at least in this study area, which conflicts with the understanding of Sierran forests as being extremely patchy \citep{Franklin1996,SNEP1996}. 

It is difficult to place my modeling results within the context of some empirical studies and review papers due to large differences in classification strategies. These differences apply both to the fire regime and to vegetation communities. For example, it is very common to classify entire fires as low, moderate, or high severity, or as mixed, and then to apply that to entire fire regimes \citep{Chang1995,Beaty2008,Collins2010,Kane2013}. The result of this is a progressive dilution of the complex behavior of fire and its effects on vegetation. In the context of certain academic research this is probably not an issue, but it can be problematic when this generalization is used to motivate specific restoration strategies or communications with the public about fire \citep{Little2008,Sanjayan2012}. Moreover, the definition of these severity levels is subjective and breaks down in a range of variability framework like mine. For example, in \citet{Chang1995}, mixed conifer is assigned to a short-interval, low-intensity regime, while red fir is assigned to a variable-interval, variable-intensity regime. Over a several-hundred year historical period, the mixed conifer forests in my simulations were characterized by fire regimes occurring at variable intervals and with variable intensities. Thus two supposedly distinct fire regimes become indistinguishable at sufficiently long timescales.
%
A second difference is that vegetation classifications in other studies are often more precise and numerous, especially in empirical studies. For example, the conception of vegetation classes presented by \citet{Skinner1996} includes chaparral as its own type, and subdivides some of the cover types I used in the simulations. I cannot downscale my results below the cover type level, so it is difficult to evaluate how my results confirm or refute those presented by researchers working with more narrowly defined vegetation classes.

Recent literature makes fewer claims about the vegetation pattern of historical forests, perhaps because empirical data only provide part of the picture. As \citet{Collins2010} point out, high severity fire patches become early successional patches. When fires are allowed to burn naturally, forests have more heterogeneous structure, which is especially manifest as high patch complexity \citep{Collins2010}. \citep{Baker2014} asserted that 23\% of the historical mixed conifer forest was open canopy. Some argue that this is an underestimate \citep{Fule2014}. I am unable to fully understand how this number was generated. Even leaving out the Early Development seral stage, which typically has low overstory tree canopy cover, during the simulated HRV I recorded a median total of Mid--Open and Late--Open forest at 20\% for the mesic mixed conifer type, and in the xeric type this proportion reaches nearly 50\% (Tables~\ref{tab:covcond_smcm} and \ref{tab:covcond_smcx}). The average across the landscape would thus have been well above 23\%. The median proportion of Early Development forest was 15\% in the mesic type and 35\% in the xeric type. I am unable to connect how \citet{Baker2014} could record 39\% high severity fire evidence and only 23\% open canopy forest, unless his definition of open is dependent only on density and not on canopy, in which case a comparison to my results cannot be made. As stated above, the claims by \citet{Mallek2013} as described above should logically lead to a different proportion of the landscape in Late Development than observed in my simulations. Despite some important differences between my results and those found by other researchers, the findings of at least \citet{Baker2012} and \citet{Collins2010}, together with mine, present strong evidence of an extremely heterogeneous forest structure that diverges markedly from current actual or planned conditions. 

I also reviewed the forest management recommendations made by the authors of recent studies. In general, I agree with most contemporary findings. As proposed by \citet{Collins2011}, restoration to a range of variability characteristic of the historic period should not be a hard and fast rule; the real purpose of restoration is to develop a forest resilient to disturbance. Restoration toward the HRV makes sense only in the context of an adaptive management framework that promotes reevaluation of success and provides flexibility for managers to change course if warranted. It is also critical that restoration targets be defined as a range of variation, rather than a fixed target based on a mean or median value \citep{Collins2011} In an early paper, \citet{Collins2010} suggest creating patches of early successional vegetation communities that mimic configuration of historical patches, within the bounds of what is socially and legally acceptable. I strongly agree with this need, although given recent large and high severity fires, it may be safest and most cost-effective to take advantage of existing early successional patches and plan other restoration around them, rather than generate them intentionally. Findings from my study should provide additional motivation to follow the adaptation options developed for the Tahoe National Forest as described in \citet{Littell2012}, including a focus on increasing diversity at both large and fine scales, implementing treatment on larger management units, and using large-scale disturbance as an opportunity to practice adaptive management. I agree with \citet{Baker2012} that only mimics low severity disturbance will not restore forests to the HRV. Fire can never be completely controlled. However, by implementing restoration that restores the spatial range of variability in landscape structure by incorporating temporal ranges of variability in burning, managers may be able to increase their ability to control fire in specific places where the values at risk necessitate fire management efforts.


\clearpage
\subsection{Management Implications}

% becky said delete the intro sentence, move limitaitons elsewhere
%The results of this study include a detailed quantitative and qualitative description of the historic reference period. Using these results I can next assess the extent to which human actions have led to ecological changes, such that landscapes and their functions are outside of their historic range of variability \citep{Landres1999,Swetnam1999}. Considering the alignment or gap between past and current fire frequencies is a potential basis for developing priorizations of forest management activities, including ecological restoration, fuels reduction, and habitat management \citep{Fule2008,Safford2014}. 

%"Drawing comparisons between past and current fire frequencies can assist resource managers in prioritizing areas for ecological restoration, fuels reduction, certain fire or habitat management practices, and other activities. " (Hugh)


In this study I leveraged current scientific knowledge and understanding of vegetation dynamics and disturbance processes to simulate changes to landscape composition and configuration over time under a historical reference framework. I then compared those dynamics to observations of current conditions, and assessed the departure from the historic range of variability. My landscape-level conclusions are that both composition and configuration deviate substantially from the HRV. In general, the current landscape is dominated by Mid and Late Development forest and lacks the fire-dependent open canopy stand conditions (Mid-Open, Late-Open) and spatial heterogeneity in vegetation that were maintained by natural disturbances during the reference period. Departure is probably due in large part to land management practices, especially fire suppression and timber management, over the last 150 years \citep{Stephens2007,Safford2014}. 

The observed departure from the HRV of the current landscape can be explained in large part by management actions on the Tahoe National Forest that directly altered the landscape. First, fire suppression facilitated the closure of forest canopies, leading to more closed and less open canopy cover across forests of varying species composition \citep{Beaty2007}. Second, it indirectly reduced the amount of the Early Development seral stage, particularly in the xeric mixed conifer forest (as evidenced by dominance of Late Development forest in the 2010 cover types map \citep{USDAForestService2009}). Third, I infer that it contributed to a reduction in patch complexity over time because without irregular borders of a natural fire, irregular borders of cover type-seral stage patches may not develop. Timber management could also have promoted departure from the HRV. Boundaries of timber sales tend to be linear, rather than irregular, affecting patch complexity \citep{USDAForestService2012}. Fine-scale heterogeneity would also be reduced in forests managed primarily to increase the proportional representation of the most valuable timber species, with structures designed to be easily accessible and harvested, rather than managed for complex structures that would provide habitat for a range of plants and animals \citep{Franklin2002}. 

While substantial, thinning and harvest operations affected only a small fraction of the amount of forest that probably burned annually during the historical period, so the likelihood that timber management activities compensated for the loss of stand opening and stand initiation effects from fire is low \citep{Cushman2011,USDAForestService2012}. Active replanting after cutting or after fire would also reduce the average area in the Early Development stage at any given moment because it would reduce the time spent in the Early Development stage, and because planting and managing trees would accelerate the time to Mid Development \citep{Dellasala2014}. In addition to these impacts, the Tahoe National Forest has been managed to promote the development of late successional, old growth forest habitat, in order to provide habitat for species dependent on it, like the spotted owl \citep{USDAForestService2004a}. This also pushes the landscape into an older, more closed condition, since both fire suppression and a lack of timber harvest affect the character of these old growth areas \citep{Franklin1996}.

%Based on our results, it might be tempting for managers to reach the simple conclusion that the landscape is less fragmented today than during the reference period. For example, today's landscape is simpler (lower \textsc{shape} values), contains less core area (based on \textsc{core\_am}), and has less contrast between patch types (\textsc{econ\_am}) (Figures~\ref{fig:fragland} and \ref{fig:patchmaps1}--\ref{fig:patchmaps4}). However, this conclusion is not as straightforward as it might seem. Fragmentation is a landscape-level process in which a specific habitat is progressively sub-divided into smaller, geometrically altered, and more isolated fragments as a result of both natural and human activities. This process involves changes in landscape composition, structure, and function at many scales and occurs on a backdrop of a natural patch mosaic created by vegetation transitions, both those mediated by and independent from natural disturbances \citep{McGarigal1995}. The scale at which fragmentation occurs is at the level of a specific habitat type; it is the habitat, rather than the landscape, that becomes fragmented. In this study we evaluated the spatial pattern---and by implication, the fragmentation---of many different patch types (defined by unique combinations of cover type and condition class). Certaintly, some of these patch types are less fragmented in the current landscape than they were under the simulated HRV\todo{B: What is train of thought? Like overview of configuration and why not simple to implement at management level}. 



%"Drawing comparisons between past and current fire frequencies can assist resource managers in prioritizing areas for ecological restoration, fuels reduction, certain fire or habitat management practices, and other activities. " (Hugh)

My results imply that, even if climate change were discounted, using the simulated historical range of variability as a restoration target would be challenging and take many decades to implement, a claim also made by \citet{Collins2011}. The model equilibration used in this study (see Chapter~\ref{sec:hrvmethods} for more details) was 200 years long. Even though the seral stage distribution of some cover types equilibrated in less than that time, given that forest planning horizons are on timescales of 5--30 years, additional analyses would need to be completed to determine reasonable benchmarks for landscape change at those shorter timescales \citep{Millar1999,Millar2014}. It is true that since I simulated natural disturbance processes, accelerated management could reduce the time it takes to restore the HRV characteristics to the study area. However, at the same time, there are existing social, economic, and political challenges that would simultaneously retard progress. The extent and intensity of disturbance required to emulate the natural disturbance regime is significant, and a simple restoration of the historical fire regimes would not be possible with public input and environmental consultation. For these reasons a more practical use of these results are to guide the prioritization of areas for restoration, design fire and fuels management projects, or describe desired future conditions \citep{Keeley2000,Fule2008,Safford2014}.

\subsection{Limitations}
Several limitations apply to my results. First, my results are based on an initial cover type map that does not include roads because roads were not part of the historical landscape. I excluded roads by overwriting the land cover type of ``road'' with the nearest alternative cover type. Figure \ref{fig:roadcovermap} shows the cover type layer with roads overlaid. However, extensive research shows that roads have significant impacts on Sierran ecosystems \citep{Trombulak2000,Gucinski2001,Karr2004,Theobald2011}. The primary impact of roads is that they break up patches (the unit of analysis when using \textsc{Fragstats}) and are associated with high-contrast seral stages adjacent to one another. Some researchers have found that the impacts from roads in terms of patch characteristics may be greater than that due to management because roads occur so extensively across the study area \citep{Tinker1998,Gucinski2001,Mcgarigal2001}. Given the large amount of roads within the study area and their disproportionate influence on landscape structure and function, it is important to consider the likely impact of roads on configuration metrics. In many cases, at least at the local scale, having included roads would probably have led to results in which the difference between the current landscape and the HRV was exacerbated.
%
\begin{figure}[!htbp]
  \centering
  \includegraphics[height=.4\textwidth]{/Users/mmallek/Tahoe/Report2/images/roads_cover.png}
  \caption{The core study area superimposed with the cover type map and all roads in black. There are a two designated roadless areas, but in general roads are common throughout the watershed. The closeup area is just northeast of the Pendola reservoir.} 
  \label{fig:roadcovermap}
\end{figure}
%

In addition, results are constrained by certain limitations related to the nature of this study as one that simulates landscape-level dynamics. For example, the vegetation cover layer is subject to human interpretation errors and objective classification errors, and is further limited by the spatial resolution of the grid. Still, my results can be used to help identify the most influential factors driving landscape change, the implications of the simulated disturbance and succession regime, and areas where further research is needed to delineate key parameters. The estimate of the HRV described here could change if new scientific understanding or better data that would affect model parameterization becomes available.

An issue brought up by \citet{Fule2014} is that the \citep{Baker2014} does not incorporate other disturbances into his assumptions about what processes produced the observed forest structure. This limitation also applies to this study, which only explicitly incorporates wildfire. It is therefore likely that my study underestimates total disturbance, as fire regime parameters were based off empirical data that only incorporated specific fire evidence.


\subsection{Using results at various scales}
This study was designed to address questions on the landscape level, and therefore the simulations and analysis were conducted at that scale. When considering smaller-scale subregions of the study area, managers should carefully review results through a comparative or relative lens. In general, the landscape-level statistical results are inappropriate for use as the template for a project-level target. Instead, the landscape-level results should guide the development of project-level targets. In addition, the success of a project should be measured not by whether the results of the project mirror the results of this study, but by whether the project contributed to a particular landscape-level shift set as a goal using those landscape-level results. This is a subtle but important distinction in how to use the outputs of these simulations. 

To further illustrate, consider an area identified as being appropriate for fire restoration. The composition of that area cannot be assumed to match that of the overall landscape. Many variables will affect the composition, including environmental (topographic position) and social (wildland-urban interface) considerations. Furthermore, my results rely on the generation and analysis of a large quantity of data. Even in the absence of such variation, if the scale of analysis is reduced from 180,000 ha to a few thousand, the statistical validity of the results and their implications deteriorates. As discussed, the need for statistically viable results led me to use only results from cover types at least 1000 ha in extent, and focused my analysis on the two dominant cover types, which extend across an area many times that minimum.

With respect to planning, my results are best used to provide a broad-scale context to smaller-scaled projects. As an example, consider the seral stage distribution of xeric mixed conifer forests. Across the full study area, about 27\% to 39\% of xeric mixed conifer forest is in the Late--Open stage. Based on this, the Forest may set this range as a management target. If the current proportion on the forest is 15\%, an individual project may seek to maintain or create additional acres in that seral stage. However, these results cannot tell managers how much of the cover type in that stage should exist within that project boundary, which could also be 0\% or 100\%. The key is for managers to place individual projects in a landscape context, and understand the contributions different parts of the forest have on the overall landscape pattern (in this example, the overall seral stage distribution).

\textsc{RMLands} has been optimized to perform well statistically when used to examine landscape change over large extents and long time periods. While spatially explicit in its dynamics, it is a simulation, producing many potential outcomes, and is not intended to predict specific outcomes at specific places and points in time. Rather, the data produced are aggregated in order to develop an understanding of the area as a whole. Because this model should not be used to predict an actual fire rotation value for a particular point on the landscape, it should also not be used to predict the effect of a particular vegetation management strategy \emph{at a particular point on the landscape.} Other forest and fire simulation models are designed for this purpose. \textsc{RMLands} is designed to produce outputs that facilitate measuring the effect of vegetation management implemented across space and time, which in turn enables assessing and understanding the potential impacts of such actions at scales larger than is practical or possible to measure experimentally.  

Although not conducted for this study, there are additional ways to use the results of this study that maintain a landscape-level focus. For example, a future project could conduct a comparable analysis on another similar sized and ecologically analogous area. The results from the two landscapes could then be compared, and the results used to prioritize work based on the relative degree of departure from the HRV of a particular landscape in its composition or configuration. Another option would be to use the average canopy cover map (Figure~\ref{fig:averagecc}) to make the spatial aspect of the range of variability explicit, since the boxplots as presented summarize both over space and time. A comparison of the average canopy structure to the current canopy structure could serve as a starting point from which to prioritize survey and monitoring work. The field work, aimed at ground-truthing the results, would then feed into an analysis to determine where to implement management designed to shift the overall landscape toward a particular state. Such a strategy could work as long as decision-makers do not try to impose the median condition for individual cells during the simulation on all cells within the study area. Ultimately, spatial information should inform targets, not \emph{be} the target.

\subsection{Recommendations for Sierran Mixed Conifer - Mesic and - Xeric}
My study methodology makes it possible to single out separate cover types and seral stages for analysis. In this section I interpret those results and identify recommendations for specific strategies that could be used to push landscape structure toward the HRV, and other recommendations of management strategies to avoid. Because in reality cover types are interspersed with one another, managers will need to consider the recommendations for individual cover types within the context of the vegetation actually present within a given management unit. In some cases, the motivation for a given vegetation treatment may focus on the cover type involved, as well as the seral stage. However, because cover types in the real world blend into one another (as opposed to being separated by stark lines such as those in cover type maps), treatment unit boundaries are likely to include different cover types near those boundaries. The treatment prescribed may have been designed mostly for one cover type, but will affect adjacent cover types. In these situations, managers should have the flexibility to consider what landscape-level objectives can be achieved through adjusting the treatment boundaries to add more of or exclude adjacent cover types.

The observed fire rotation for mesic mixed conifer forests during the simulated historical period was 27 years, while the observed fire rotation for xeric mixed conifer forests was 23 years. In both, low mortality fire was much more common than high mortality fire. However, fires having both outcomes are necessary to create stand structure and composition similar to that observed during the simulation. 

In mesic mixed conifer forests, all seven seral stages were well represented during the simulated historical period. The most common seral stage (highest median value) during the simulation was Mid--Closed. The proportion of mesic mixed conifer on the current landscape is close to the median values for Early, Mid--Open, and Late--Moderate. There is much more Mid--Moderate and Late--Closed, and much less Mid--Closed and Late--Open, in the 2010 landscape compared to the simulated HRV. Based on these observations, in the short term I recommend maintaining existing open stands of mesic mixed conifer though prescribed fire and other understory vegetation treatments. Because patch sizes are much smaller now on average, when identifying stands to push towards, for example, Mid--Closed or Late--Open, I recommend looking for treatment units near or adjacent to existing patches of the target seral stage, in order to create large patches in that type. Overall, the relative proportion of closed canopy forest across development classes is near the median values for closed canopy seral stages during the simulated HRV. I do not believe more closed canopy forest is needed to bring the landscape toward HRV conditions. That said, it probably makes sense to work to create conditions for younger closed canopy forests, so that all the closed canopy stands on the forest do not belong to the oldest age class.

In contrast to the results for Sierran Mixed Conifer - Mesic, I observed a consolidation of the area in Sierran Mixed Conifer - Xeric into the Early, Mid--Open, and Late--Open seral stages. Mid--Closed, Mid--Moderate, and Late--Closed were present in very low amounts, while Late--Moderate was well established but not dominant. Based on these observations, in the short term I recommend maintaining existing Early Development, Mid--Open, and Late--Open stands of xeric mixed conifer through prescribed fire and other understory vegetation treatments. Harvest treatments should focus on areas now in closed canopy conditions. As with the mesic mixed conifer forests, because patch sizes are much smaller now on average, when identifying stands to push towards and open canopy structure, I recommend looking for treatment units near or adjacent to existing open patches in order to create larger patches overall. Of course, it is also important for fuels specialists, foresters, and biologists to work together to include fine-scale structural complexity in these large patches of open canopy stands.
 
In the medium term, for both cover types I recommend the restoration of fire wherever practicable. Fire was quite common on the simulated historical landscape and was probably the main driver of the complexity of patches I see in the simulated landscape, which is in agreement with empirically and modeled estimates of historical forest structure \citep{Franklin2002,Nonaka2005,Mallek2013}. The point-specific fire rotation for low mortality fire ranged greatly for both mesic mixed conifer forests (which ranged from 18 years to about 100 years) and xeric mixed conifer forests (which ranged from 17 years to about 80 years). Managers can transfer this variability into flexibility when planning and executing vegetation treatments and wildfire response (Figure~\ref{fig:pfire_comp_HCRD}). In fact, the spatial variability of fire was instrumental in creating the spatial variability of forests and plants observed as outputs of the simulation. Managers who are charged with focusing fuels reduction on certain areas within the forest could set goals of carrying out vegetation treatments somewhat more frequently in these parts of the forest and less frequently elsewhere, such as in spotted owl activity centers, thereby also contributing to the overall more complex landscape pattern observed during the simulated HRV. Because fuels reductions along roads offers benefits that include enhancing the ability of the road to serve as a barrier to fire spread and increasing the safety moving through forested areas during wildfire incidents, among others, they may be priority restoration sites and early targets for promoting open canopy conditions (Figure~\ref{fig:pfire_comp_EDNF}). 

\begin{figure}[!htbp]
  \centering
  \subfloat[][]{
    \centering
    \includegraphics[width=0.5\textwidth]{/Users/mmallek/Documents/Thesis/Seminar/pfirebefore.jpg}
    }%
  \subfloat[][]{
    \includegraphics[width=0.5\textwidth]{/Users/mmallek/Documents/Thesis/Seminar/pfireafter.jpg}
    }
  \caption{Before and after a prescribed fire treatment on the Hat Creek Ranger District, Lassen National Forest \citep{SNAMNphoto}. The outcome of this burn was primarily low mortality, although some under and middlestory trees were killed as a result of the fire.} 
  \label{fig:pfire_comp_HCRD}
\end{figure}

\begin{figure}[!htbp]
  \centering
  \subfloat[][]{
    \centering
    \includegraphics[width=0.5\textwidth]{/Users/mmallek/Documents/Thesis/Seminar/EDNF_mech_after.jpg}
    }%
  \subfloat[][]{
    \includegraphics[width=0.5\textwidth]{/Users/mmallek/Documents/Thesis/Seminar/EDNF_mech_before.jpg}
    }
  \caption{Before and after a mechanized fuels treatment on the Eldorado National Forest \citep{Winford2015}. The timing of mechanized treatments is less restricted than prescribed fires, which can only take place under certain weather and fuels conditions. When mechanized treatments are available, fuels managers have more flexibility in selecting treatment options. A mechanized treatment like the one pictured here could be followed by a prescribed burn.} 
  \label{fig:pfire_comp_EDNF}
\end{figure}


With respect to the landscape structure metrics that characterize the mixed conifer forest type, results differed between the mesic and xeric variants. In mesic mixed conifer forests, I did not find consistent patterns in how or if the current landscape, at the seral stage level for this cover type, departs from the simulated historical results. This reflects the fact that the mesic mixed conifer type is inherently complex and not necessarily dominated by a particular average patch size or level of geometric complexity. I suggest that the finding of larger, less fragmented, and more geometrically complex patches of Early Development be used to guide the planning and execution of post-fire management of areas that burn at high severity. Promoting patches of Early Development, especially in conjunction with restoration practices that encourage specific patch configuration, would be one way to manage and guide forests toward the HRV.

Some of the seral stages for the xeric mixed conifer forest type were nearly absent during the simulated historical period. As a result, I cannot make generalizations about them and will instead focus on the most common stages: Early, Mid--Open, and Late--Open. Patches of these seral stages in xeric mixed conifer forests in 2010 were smaller, more fragmented, less geometrically complex, and contain less core area than during the simulated HRV. 
%
Restoration of these forests to patches that reflect a more natural succession process may be challenging for managers, given practical needs like using roads and riparian buffers as the edges of treatment units. It may not be practical to perform mechanical treatments over large areas within this cover type. However, when conducting treatments using prescription fires, creative solutions should be sought to generate more complex edges and to complete burns over sufficiently large areas to create large core areas as a byproduct of the treatment.











