% !TEX root = master.tex

\section{Discussion and Management Implications}
\label{sec:hrvdiscussion}




%%%%%%%%%%%%%%%%%%%%%%%%%%%%%%%%%%%%%%%%%%%%%%%%%%%%%%%%%%%%%%%%%%%%%%%%%%%%%
%%%%%%%%%%%%%%%%%%%%%%%%%%%%%%%%%%%%%%%%%%%%%%%%%%%%%%%%%%%%%%%%%%%%%%%%%%%%%


\subsection{Overall Landscape Assessment}

One objective of the thesis was to quantify the historical range of variability (HRV) in the disturbance regime, landscape composition, and landscape configuration of the study area. I completed analyses both at the landscape and cover type scale, and begin with an overall assessment of the landscape-level historical conditions.

As shown in Chapter~\ref{sec:hrvresults}, fire size was weakly but positively associated with the climate parameter (bigger fires when drought conditions were more severe). However, it is also influenced by vegetation susceptibility and the specified distribution of disturbance size. For this reason, large areas burned in relatively ``wet'' timesteps. 

Fires during the simulated historical period burned far more frequently and across larger extents than at any time since record keeping began (according to available fire history data \citep{calfire2012,usgs-fire-data2012}) within the Yuba River watershed study area (see Tables~\ref{tab:darea_atleast} and \ref{tab:darea}, and Figure~\ref{fig:distid_median} and \ref{fig:distid_mean}). Most individual fires were small and predominantly low severity. However, large fires did occur, which disproportionately shaped patch configuration on the landscape and directly affected the seral stage composition of cover types. Moreover, although more than 50\% of the study area burned very rarely, it did occur several times during the simulated HRV, indicating that occasional extremely widespread fire is a characteristic of the fire regime in this study area. The results also reflect a lack of any modern fire management. Thus, in the model, all fires burn unchecked, regulated only by local vegetation, topography, barriers, and the vagaries of weather (as represented by stochastically generated potential fire sizes). Although this can and does result in some very large fires, extending across tens of thousands of hectares, more typically I observed fairly small fires. In nature, as in the model, many lightning-caused fires burn themselves out without consuming much fuel. Regardless, because I had such a high degree of control over the fire regime, the bulk of the landscape assessment focuses on the composition and configuration results in this section. The particular disturbance regime results primarily confirm that the model was functioning appropriately and as designed.
% come up with better word for "droughtier"

% rewrote 2015-09-20
With respect to the seral stage distribution, a few patterns emerged (Full results for the nine focal cover types can be found in Appendix~\ref{sec:indiv_cov_results}.) Overall, the current study area composition departs from the HRV not only at the landscape scale, but also at the cover type and seral stage level. Complete departure from the simulated HRV is the norm for each seral stage of the xeric mixed conifer type. Mesic mixed conifer forests are less departed overall. Three of the seral stages (Mid--Open, Late--Closed, and Late--Moderate) have distributions that indicate that the current landscape is moderately departed from the HRV, and the current proportion of Early Development mesic mixed conifer forest is close to the median value observed over the course of the simulation. However, the current proportion of the landscape covered by the three remaining seral stages falls completely outside the simulated historical distributions.

%rewrote again to remove non-smc types on 2016-01-14
In both the mesic and xeric mixed conifer types, as well as the other cover types studied closely, Late--Open conditions were far more common during the HRV than on the current landscape. Interestingly, early seral conditions were more common during the simulated historical period than on the current landscape for xeric mixed conifer forests. This increase in the proportion of the study area belonging to early development and open canopy forests is directly connected to the frequent and extensive pattern of wildfires burning in the HRV scenario (Figure~\ref{fig:darea_smcx}, Tables~\ref{tab:darea_smcx} and \ref{tab:darea_atleast_smcx}). Mesic mixed conifer forests experienced somewhat less fire than the xeric type---the Mid--Closed seral stage actually dominated during the simulation---but wildfires were still extensive (Figure~\ref{fig:darea_smcm} and Tables~\ref{tab:darea_smcm} and \ref{tab:darea_atleast_smcm}). The large amount of Mid--Closed may be due to a combination of its relatively low susceptibility, relatively short early period (averaging 26 years), and the fact that Early Development is twice as likely to succeed to Mid--Closed as to Mid--Open (Appendix~\ref{smc-description}).


%rewrote this 2015-09-20
%Interestingly, early seral conditions were less common during the simulation than on the current landscape for the mixed evergreen and ultramafic mixed forests, but more common for the xeric mixed conifer forests; for the other types, however, the proportion of early seral is within the HRV. Both mixed evergreen types and both oak-conifer types have a smaller proportion of forest in late development now than during the simulation. Late--Open conditions were far more common during the HRV. For all cover types except for mesic mixed evergreen forests, Late--Open conditions were far more common during the HRV, and the current landscape is fully departed from the HRV. The Mid--Open was more common for xeric and ultramafic mixed conifer forests during the simulation than today, while Mid--Closed was dominant during the HRV for mesic mixed conifer forests, but now is not. Finally, closed canopy conditions were much more dominant in mesic red fir forests than on the present landscape; these stages are also completely outside the HRV.

%-- Closed canopy ocfw dominant now, but not during HRV.  (didn't use)
%-- Closed conditions were much more dominant in mesic red fir forests.
%-- Mid open more common for smcx and smcm
%-- Mid closed dominant during hrv for Smcm, but not now
%-- Mixed evergreen forests and oak-conifer forests and woodlands generally have less area in late development now than during the simulated HRV.
%-- Early seral was much more prevalent in SMCX and somewhat more prevalent in RFRX, but consistent with the other cover types.
%-- Early more common for smcx
%-- Late open more common for all cover types except rfrm, always outside of hrv
%-- Mixed evergreen and xeric mixed conifer had no seral stages within the HRV.
%-- 12 of 63 cover type - seral stage combinations with the HRV
%-- Only OCFWU-Early, RFRX-LDC, SMCM-Early, SMCU-MDM, were within interquartile range.
%-- Being completely outside the simulated HRV is the norm for seral stages with the focal nine areas. In many cases, the current proportion of a given seral stage is within the HRV, but this generally means either between the 5th and 25th percentiles, or between the 75th and 95th percentiles. In just four cases (\textsc{ocfw\_u} early, \textsc{rfr\_x} late closed, \textsc{smc\_m} early, and \textsc{smc\_u} mid moderate) was the current proportion within the 25th-75th interquartile range.


An analysis of \textsc{Fragstats} metrics at the landscape scale yields insight into the historical period. First, I note that compared to the present conditions, during the HRV the study area was composed of larger and more extensive patches, as illustrated by Figure~\ref{fig:fragland1}. This trend was heavily influenced by the presence of wildfires on the landscape, as high mortality fires in particular created large areas of Early development vegetation (Figure~\ref{fig:patchmaps1-early}). However, I also observed large patches in the other seral stages, which were more likely to form long or convoluted patches that were nonetheless extensive (Figure~\ref{fig:patchmaps1-mid}).

% figure redone 2015-09-20
\begin{figure}[!htbp]
  \centering
  \subfloat[][]{
    \centering
    \includegraphics[width=0.45\textwidth]{/Users/mmallek/Documents/Thesis/maps/hrv-largepatch2410.pdf}
    \label{fig:patchmaps1-early}
    }%
  \subfloat[][]{
    \includegraphics[width=0.45\textwidth]{/Users/mmallek/Documents/Thesis/maps/hrv-largepatch2430.pdf}
    \label{fig:patchmaps1-mid}
    }
  \caption{(a) A large patch, highlighted with dark grey, of Sierran Mixed Conifer - Mesic in Early Development. The patch is 5,780 hectares, one of the second-largest patch during this timestep. (b) Two patches of Sierran Mixed Conifer - Mesic in Late--Closed. These are two of the largest patches during that timestep, at 10,930 hectares and 4,750 hectares.} 
  \label{fig:patchmaps1}
\end{figure}


% big 2622 12,600

In addition, I observed increased dominance by certain cover-seral stage types during the HRV, which is likely what led to smaller values for the Simpson's Evenness Index (Figure~\ref{fig:fragland_siei}). For example, within the Sierran Mixed Conifer - Xeric cover type, Early Development and Mid--Open were much more widespread during the simulated HRV than in the current landscape (Figure~\ref{fig:patchmaps2}). Because Sierran Mixed Conifer - Xeric is so widespread, this shift would directly influence Simpson's Evenness, lowering its value.

% figure redone 2015-09-20
\begin{figure}[!htbp]
  \centering
  \subfloat[][]{
    \centering
    \includegraphics[width=0.45\textwidth]{/Users/mmallek/Documents/Thesis/maps/hrv-dominance-ts0.pdf}
    \label{fig:patchmaps2-ts0}
    }%
  \subfloat[][]{
    \includegraphics[width=0.45\textwidth]{/Users/mmallek/Documents/Thesis/maps/hrv-dominance-26.pdf}
    \label{fig:patchmaps2-ts570}
    }
  \caption{Cover-Seral stage map focused on patches from Sierran Mixed Conifer - Xeric, showing increased dominance by certain cover-seral stage types during the HRV. (a) The current landscape. (b) The same region of the map during a randomly selected timestep after the equilibration period. Note the contrast between the two maps with respect to the seral stages and size of individual patches.} 
  \label{fig:patchmaps2}
\end{figure}


Third, I find that patches on the landscape were more aggregated at the cell-level during HRV, which is illustrated by the Contagion metric (Figure~\ref{fig:fragland_contag}). In general, patches have low levels of both dispersion and interspersion. Of course, there are many ``edgy'' areas on the landscape, but this metric indicates that across the full landscape aggregation is more typical, particularly in comparison to the current landscape. Again, the homogeneity of post-fire early successional stands likely aids in increasing the contagion value (Figure~\ref{fig:patchmaps3}). 

% figure redone 2015-09-20
\begin{figure}[!htbp]
  \centering
  \subfloat[][]{
    \centering
    \includegraphics[width=0.45\textwidth]{/Users/mmallek/Documents/Thesis/maps/hrv-agg-ts0.pdf}
    \label{fig:patchmaps3-ts0}
    }%
  \subfloat[][]{
    \includegraphics[width=0.45\textwidth]{/Users/mmallek/Documents/Thesis/maps/hrv-agg-24.pdf}
    \label{fig:patchmaps3-ts615}
    }
  \caption{Cover-Seral stage map focused on patches from Sierran Mixed Conifer - Mesic. (a) The current landscape. (b) The same region of the map during a randomly selected timestep after the equilibration period. Note the contrast between the two maps with respect to the contagion (at the cell level).} 
  \label{fig:patchmaps3} %year 1075 is timestep 215; year 675 is timestep 135
\end{figure}

Despite having a higher edge-to-area ratio, and being more geometrically complex, patches during the simulated HRV still show an increase in core area over the present landscape (Figure~\ref{fig:fragland_core}). This indicates that the large patches that contain core area are sufficiently large to surpass the relatively high amount of core area on the present landscape. I expected that the current landscape might have large amounts of core area because human-designed management unit boundaries are likely to create simple shapes, especially when they incorporate linear features such as raods or streams. Since this is an area-weighted measure, I conclude that it is the presence of many large patches that are large enough to contain significant core area that resulted in a simulated historical range of variability that does not overlap the current landscape. The results for the shape complexity metric confirm this analysis (Figure~\ref{fig:fragland_shape}). Especially among the largest patches on the landscape, convoluted shapes are common. Again, this is not to say that large and simple shapes do not occur---they do---but in comparison to the current landscape, complex shapes were characteristic of the simulated HRV (Figure~\ref{fig:patchmaps4}).

% figure redone 2015-09-20
\begin{figure}[!htbp]
  \centering
  \subfloat[][]{
    \centering
    \includegraphics[width=0.45\textwidth]{/Users/mmallek/Documents/Thesis/maps/hrv-complexpatch-2622.pdf}
    \label{fig:patchmaps4-ts555}
    }%
  \subfloat[][]{
    \includegraphics[width=0.45\textwidth]{/Users/mmallek/Documents/Thesis/maps/hrv-simplepatch2422.pdf}
    \label{fig:patchmaps4-ts690}
    }
  \caption{Cover-Seral stage map focused on patches from Sierran Mixed Conifer. (a) A large patch of Sierran Mixed Conifer - Xeric in the mid development open seral stage illustrates how very large patches (this one is 12,600 hectares) may have relatively small amounts of core for their shape, yet accumulate a lot of core because of their overall size. (b) Simpler shapes do exist, such as this patch of Sierran Mixed Conifer - Mesic in the mid development open seral stage, which has a lot of core area. However, they are often much smaller (this one is 2,320 hectares).}
  \label{fig:patchmaps4}
\end{figure}

Examining some of the class-level results for similar \textsc{Fragstats} metrics as for the landscape as a whole, I observed many consistencies and a few interesting diversions (Figures~\ref{fig:smcm_areaam}--\ref{fig:smcx_clumpy}). In general, for the area-weighted metrics, if higher proportions of the landscape are occupied by a given seral stage, that cover-seral stage combination is likely to have high values for a given metric. Again, results for area-weighted mean patch area and mean core area are consistent. Early Development patches are typically characterized by more complex shapes that are less aggregated during the HRV, as compared to the current landscape. This is due to the fact that during the simulation early successional patches are created by fires that are allowed to burn naturally, rather than by vegetation treatments with predetermined, linear boundaries. In addition, seral stages that became rare during the HRV tend to be smaller, aggregated and to have simple shapes. Larger shapes have a greater potential to have complicated shapes and to be disaggregated. This result may reflect the nature of the metrics at the class level as much as the seral stage structure. Overall, the class-level metrics reflect the interplay between wildfire and succession, and can best be used when considering management alternatives at the class level. However, the landscape-level interpretation is the most suitable for evaluating the study area and the potential impact of management actions, including restoration. 

\clearpage



\clearpage
\subsection{Management Implications}

A major outcome of my study is to produce a better quantitative understanding of this historic reference period. With it, I can next assess the extent to which human actions have led to ecological changes, such that landscapes and their functions are outside of their historic range of variability \citep{Landres1999,Swetnam1999}. 
%
Considering the alignment or gap between past and current fire frequencies is a potential basis for developing priorizations of forest management activities, including ecological restoration, fuels reduction, and habitat management. 
%
My results are constrained by certain limitations related to the nature of this study as one that simulates landscape-level dynamics. For example, the vegetation cover layer is subject to human interpretation errors and objective classification errors, and is further limited by the spatial resolution of the grid. Still, my results can be used to help identify the most influential factors driving landscape change, the implications of the simulated disturbance and succession regime, and areas where further research is needed to delineate key parameters. The estimate of the HRV described here could change if new scientific understanding or better data that would affect model parameterization becomes available.

%"Drawing comparisons between past and current fire frequencies can assist resource managers in prioritizing areas for ecological restoration, fuels reduction, certain fire or habitat management practices, and other activities. " (Hugh)


A primary objective of this project was to use existing knowledge and understanding of vegetation dynamics and disturbance processes to simulate changes to landscape composition and configuration over time under a historical reference framework. I then compared those dynamics to observations of current conditions, and assessed the departure from the historic range of variability. My landscape-level assessment is that both composition and configuration deviate substantially from the HRV. In general, the current landscape is dominated by Mid and Late Development forest and lacks the fire-dependent open canopy stand conditions (Mid-Open, Late-Open) and spatial heterogeneity in vegetation that were maintained by natural disturbances during the reference period. The departure from the HRV is probably due in large part to land management practices, especially fire suppression and timber management, over the last 150 years \citep{Safford2014,Stephens2007}. Substantial changes to the ecology and landscape function in western forest landscapes have been documented not only in the Sierra Nevada, but also in the Cascades and Rocky Mountains \citep{Hessburg2005,Baker2012,Baker2014,Mallek2013,Agee1993}. However, no other studies have been completed that focus on the Tahoe National Forest and incorporate spatially explicit disturbance and succession modeling in combination with analyses of landscape structure.

In the western Sierra Nevada, foothill communities and lower elevation oak-conifer woodlands have experience a loss of species diversity, fragmentation, and outright habitat conversion due to the overlap with private lands and population growth. Middle elevation forests were and are more affected by mining and forestry; most easily accessible trees were probably cut before national forests were established \citep{SNEP1996}. The wildfire regime has been significantly altered in hardwoods, yellow pine, and mixed conifer forests \citep{Merriam2013,Safford2013}, and much less so in red fir and subalpine forests \citep{Meyer2013,Meyer2013a}. However, other human activities since the late 1800s have altered the structure of western Sierra Nevada forests, most notably to simplify it in several ways, including a decrease in species, multi-story canopies, and snags \citep{SNEP1996}. These activities are related mainly to timber harvest and to the extensive network of roads constructed to support timber harvest, fire control, and recreation. It has been suggested that this simplification of landscape structure may have a negative impact on wildlife and potentially lead to a loss of biodiversity in forests \citep{Thompson2003,Manley2004,Hunter2011}. Isolating the effect of fragmentation in mystudy landscape is made more difficult due to the inherent heterogeneity of Sierran landscapes---a consequence of steep natural gradients in elevation, topography, and substrate---and forests in this region tend to be somewhat patchy even in the absence of human alterations \citep{Franklin1996}. In the Pacific Northwest, ``old-growth'' connotes very large blocks of uniformly very old trees. However, in the Sierra Nevada ``old-growth'' indicates not only the presence of very large and old trees, but also a complex, patchy, ``messy'' forest of varying age classes, species, fuel quantities, and vegetation structure \citep{SNEP1996}.


Management actions on the Tahoe National Forest likely directly altered the landscape in a few specific ways. First, fire suppression facilitates the closure of forest canopies, leading to more closed and less open canopy cover across all cover types. Second, it indirectly reduces the amount of the Early seral stage, particularly in areas not accessible or appropriate for timber harvest. Third, it contributes to a reduction in patch complexity over time because the irregular borders of a natural fire would not become irregular borders of cover-seral stage patches. Timber management could also have promoted departure from the HRV. Boundaries of timber sales tend to be linear, rather than irregular, affecting patch complexity. Fine-scale heterogeneity was also reduced as a result of managing forests to be composed primarily of the most valuable timber species, as well as easily accessible and harvested, rather than complexly structured to provide habitat for a range of plants and animals. While substantial, thinning and harvest operations affected only a small fraction of the amount of forest that probably burned annually during the historical period, so the likelihood that timber management activities compensated for the loss of stand opening and stand initiation effects from fire is low. Active replanting after cutting or after fire would also reduce the average area in the Early stage at any given moment %\citep{Dellasala2014}
. In addition to these impacts, the Tahoe National Forest has been managed to promote the development of late successional, old growth forest habitat, in order to provide habitat for species dependent on it, like the spotted owl \citep{USDAForestService2004a}. This also pushes the landscape into an older, more closed condition, since both fire suppression and a lack of timber harvest affect the character of these focal old growth areas \citep{Franklin1996}.


%Based on our results, it might be tempting for managers to reach the simple conclusion that the landscape is less fragmented today than during the reference period. For example, today's landscape is simpler (lower \textsc{shape} values), contains less core area (based on \textsc{core\_am}), and has less contrast between patch types (\textsc{econ\_am}) (Figures~\ref{fig:fragland} and \ref{fig:patchmaps1}--\ref{fig:patchmaps4}). However, this conclusion is not as straightforward as it might seem. Fragmentation is a landscape-level process in which a specific habitat is progressively sub-divided into smaller, geometrically altered, and more isolated fragments as a result of both natural and human activities. This process involves changes in landscape composition, structure, and function at many scales and occurs on a backdrop of a natural patch mosaic created by vegetation transitions, both those mediated by and independent from natural disturbances \citep{McGarigal1995}. The scale at which fragmentation occurs is at the level of a specific habitat type; it is the habitat, rather than the landscape, that becomes fragmented. In this study we evaluated the spatial pattern---and by implication, the fragmentation---of many different patch types (defined by unique combinations of cover type and condition class). Certaintly, some of these patch types are less fragmented in the current landscape than they were under the simulated HRV\todo{B: What is train of thought? Like overview of configuration and why not simple to implement at management level}. 


In addition, I evaluated vegetation patterns in the current landscape after excluding roads (i.e., I removed roads from the land cover map by filling in those areas with the nearest alternate cover type). Figure \ref{fig:roadcovermap} shows the cover type layer with roads overlaid. I did this in order to be consistent with the simulation of landscape structure changes during the reference period. However, the significant and extensive impact of roads on Sierran ecosystems is well documented in the literature \citep{Karr2004,Trombulak2000,Gucinski2001,Theobald2011}.  In particular, roads are linear landscape features that can create high-contrast edges and bisect patches. They may cause greater fragmentation of habitats than the direct loss of habitat from associated land use activities \citep{Gucinski2001,Tinker1998,Mcgarigal2001}. Given the large amount of roads within the study area and their disproportionate influence on landscape structure and function, it is important to consider the likely impact of roads on configuration metrics. In many cases, at least at the local scale, including roads would probably exacerbate the difference between the current landscape and the HRV.
%
\begin{figure}[!htbp]
  \centering
  \includegraphics[height=.4\textwidth]{/Users/mmallek/Tahoe/Report2/images/roads_cover.png}
  \caption{The core study area superimposed with the cover type map and all roads in black. There are a two designated roadless areas, but in general roads are common throughout the watershed. The closeup area is just northeast of the Pendola reservoir.} 
  \label{fig:roadcovermap}
\end{figure}
%


%"Drawing comparisons between past and current fire frequencies can assist resource managers in prioritizing areas for ecological restoration, fuels reduction, certain fire or habitat management practices, and other activities. " (Hugh)

My results imply that any attempt to restore the landscape structure to a composition and configuration that aligns with the historic range of variability described in this thesis would likely be a difficult and long-term undertaking. The model equilibration period is the length of the interval between the starting condition and structure (in this case, of the current landscape) and beginning of a stable range of variation. It is a function of not only how far outside the stable range of variation the current landscape is, but also the speed at which disturbance and succession processes interact to affect a change in the landscape trajectory. Thus, I can infer that if management activities were designed to emulate natural disturbance processes, then it would take a length of time equal to the equilibration period to return the landscape to its HRV. The equilibration period for the simulation was 40 timesteps, or 200 years. Given that forest planning horizons are on timescales of 5--30 years, and the fact that climate change is predicted to have measurable impacts on both wildfires and vegetation community succession and structure over the next several decades, an effort to ``restore'' the current landscape to the exact conditions aligned with the simulated HRV may not be practical \citep{Millar1999,Millar2014}. Moreover, the extent and intensity of disturbance required to emulate the natural disturbance regime is significant, and a simple restoration of the historical fire regimes would not be possible with public input and environmental consultation. Certainly, social, economic, and political challenges exist. However, as several ecologists have pointed out, my results can still be used to guide the prioritization of areas for restoration, for designing fire and fuels management projects, or describing desired future conditions \citep{Safford2013,Keeley2000}.


\subsection{Using results for management at various scales}
It is important to understand that one limitation of this study is that it was not designed to address questions below the landscape level. While it may be tempting for managers to view subregions of the study area through the lens of the results, it would not be appropriate to set management targets to achieve, for example, proportions of cover type seral stages at a local scale such as a project analysis area that mirror those that characterize the FRV. Any management targets set using the results of this study should be measured at the full landscape level of this study. In addition, my results are organized by non-flexible boundaries such as the watershed and the area assigned to each cover type. Consequently it would not make sense to target a certain proportion of the landscape to be a particular cover type, since that was not modeled in the simulation. Furthermore, the results rely on the generation and analysis of a large quantity of data. When the scale of analysis is reduced, so is the quantity of data produced by the model, and with it, my confidence in the statistical validity of the results and their implications. 

To be more specific, I propose a simple example. The Tahoe National Forest could draw a boundary over an area identified as being appropriate for fire restoration. It would not be appropriate to assume that because across the full study area, late development, open canopy xeric mixed conifer forests ranged from about 25\% to 35\% of the total xeric mixed conifer forest, that within the newly identified project area, similar proportions should prevail. My study does not speak to an appropriate management target for any area smaller than the full study area. That said, managers could draw up a target increase or decrease in extent of a particular seral stage for xeric mixed conifer forests at the project level. This project-level information could then be used to consider how implementing the project would move the landscape as a whole toward or away from the FRV. 

Put another way, these results are no more applicable at the project level than the model should be considered a fire behavior model. It has been optimized to perform well statistically when used to examine landscape change over large extents and long time periods. While spatially explicit in its dynamics, it is a simulation, producing many potential outcomes, and is not intended to predict specific outcomes at specific places and points in time. Rather, the data produced are aggregated in order to develop an understanding of the area as a whole. Because this model should not be used to predict an actual fire rotation value for a particular point on the landscape, it should also not be used to predict the effect of a particular vegetation management strategy \emph{at a particular point on the landscape.} Other forest and fire simulation models are designed for this purpose. \textsc{RMLands} is designed to measure the effect of vegetation management implemented across space and time, to understand the potential impacts of such actions at scales larger than is practical or possible to assess experimentally. 

A more appropriate use of the results would be to run a similar analysis on another similar sized and ecologically similar area and compare the two, potentially using the results to prioritize work based on the degree of departure from an HRV of a particular landscape in its composition or configuration. Another option would be to use the average canopy cover map (Figure~\ref{fig:averagecc}) to make the spatial aspect of the range of variability explicit, since the boxplots as presented summarize both over space and time. A comparison of the average canopy structure to the current canopy structure could serve as a starting point from which to prioritize survey and monitoring work aimed at determining the best places to implement management designed to shift the overall landscape toward a particular state.

\subsection{Recommendations for Sierran Mixed Conifer - Mesic and - Xeric}
My study methodology makes it possible to single out separate cover types and seral stages for analysis, and in this section I provide some interpretation of those results that is targeted toward providing recommendations for specific strategies that could be used to push landscape structure toward the HRV, and other recommendations of management strategies to avoid. Because in reality cover types are interspersed with one another, managers will need to consider the recommendations for individual cover types within the context of the vegetation actually present within a given management unit. In some cases, it may be helpful for managers to think about a particular cover type not as receiving direct treatment, but as being influenced by other nearby treatments targeting other cover types. When differing cover types border or serve as the edge of treatment units targeting other vegetation types, managers can evaluate the patches of neighboring forests present at the time of treatment and consider whether to expand the management unit to treat them, to try and achieve an irregular edge to increase patch complexity, or to exclude them from treatment in order to promote development into the late development conditions.                                                                                                                                           The observed fire rotation for mesic mixed conifer forests during the simulated historical period was 27 years, while the observed fire rotation for xeric mixed conifer forests was 23 years. In both, low mortality fire was much more common than high mortality fire. Fires of both outcomes are necessary to create stand structure and composition similar to that observed during the simulation. 

In mesic mixed conifer forests, all seven seral stages were well represented during the simulated historical period. The most common condition (highest median value) during the simulation was Mid--Closed. The proportion of mesic mixed conifer on the current landscape is close to the median values for Early, Mid--Open, and Late--Moderate. There is much more Mid--Moderate and Late--Closed, and much less Mid--Closed and Late--Open, now compared to the simulated HRV. Based on these observations, in the short term I recommend maintaining existing open stands of mesic mixed conifer though prescribed fire and other understory vegetation treatments. Because patch sizes are much smaller now on average, when identifying stands to push towards, for example, Mid--Closed or Late--Open, I recommend looking for treatment units near or adjacent to existing patches of the target seral stage, in order to create large patches in that type. Overall, closed canopy forest across development classes is near the median values for closed canopy seral stages during the simulated HRV. I do not believe more closed canopy forest is needed to bring the landscape toward HRV conditions. That said, it probably makes sense to work to create conditions for younger closed canopy forests, so that all the closed canopy stands on the forest do not belong to the oldest age class.

In contrast to the results for Sierran Mixed Conifer - Mesic, I observed a consolidation of the area in Sierran Mixed Conifer - Xeric into the Early, Mid--Open, and Late--Open seral stages. Mid--Closed, Mid--Moderate, and Late--Closed were present in very low amounts, while Late--Moderate was well established but not dominant. This cover type is completely departed from the HRV. Based on these observations, in the short term I recommend maintaining existing early development and open stands of xeric mixed conifer through prescribed fire and other understory vegetation treatments. Harvest treatments should focus on areas now in closed canopy conditions. Because patch sizes are much smaller now on average, when identifying stands to push towards and open canopy structure, I recommend looking for treatment units near or adjacent to existing open patches in order to create larger patches overall. Of course, it is also important for fuels specialists, foresters, and biologists to work together to include fine-scale structural complexity in these large patches of open canopy stands.
 
In the medium term, I recommend the restoration of fire wherever practicable. Fire was quite common on the historical landscape and was the main driver of the complexity of patches I see in the simulated landscape. The point-specific fire rotation for low mortality fire ranged greatly for both mesic mixed conifer forests (which ranged from 18 years to about 100 years) and xeric mixed conifer forests (which ranged from 17 years to about 80 years). Managers can transfer this variability into flexibility when planning and executing vegetation treatments and wildfire response (Figure~\ref{fig:pfire_comp_HCRD}). In fact, the spatial variability of fire is critical for creating spatial variability of forests and plants observed as outputs of the simulation. Managers who are charged with focusing fuels reduction on certain areas within the forest could set goals of carrying out vegetation treatments somewhat more frequently in certain parts of the forest and less frequently in other parts, thereby also contributing to the overall more complex landscape pattern observed during the simulated HRV. Because fuels reductions along roads offers benefits that include enhancing the ability of the road to serve as a barrier to fire spread and increasing the safety moving through forested areas during wildfire incidents, among others, they may be priority restoration sites (Figure~). 

\begin{figure}[!htbp]
  \centering
  \subfloat[][]{
    \centering
    \includegraphics[width=0.5\textwidth]{/Users/mmallek/Documents/Thesis/Seminar/pfirebefore.jpg}
    }%
  \subfloat[][]{
    \includegraphics[width=0.5\textwidth]{/Users/mmallek/Documents/Thesis/Seminar/pfireafter.jpg}
    }
  \caption{Before and after a prescribed fire treatment on the Hat Creek Ranger District, Lassen National Forest. The outcome of this burn was primarily low mortality, although some under and middlestory trees were killed as a result of the fire. Images from the Sierra Nevada Avian Monitoring Information Network.} 
  \label{fig:pfire_comp_HCRD}
\end{figure}

\begin{figure}[!htbp]
  \centering
  \subfloat[][]{
    \centering
    \includegraphics[width=0.5\textwidth]{/Users/mmallek/Documents/Thesis/Seminar/EDNF_mech_after.jpg}
    }%
  \subfloat[][]{
    \includegraphics[width=0.5\textwidth]{/Users/mmallek/Documents/Thesis/Seminar/EDNF_mech_before.jpg}
    }
  \caption{Before and after a mechanized fuels treatment on the El Dorado National Forest. The timing of mechanized treatments is less restricted than prescribed fires, which can only take place under certain weather and fuels conditions. When mechanized treatments are available, fuels managers have more flexibility in selecting treatment options. A mechanized treatment like the one pictured here could be followed by a prescribed burn.} 
  \label{fig:pfire_comp_EDNF}
\end{figure}


With respect to the landscape structure metrics (computed using \textsc{Fragstats}) that characterize the mesic mixed conifer forest type, I did not find consistent patterns in how or if the current landscape, at the seral stage level for this cover type, departed from the simulated historical results. This reflects the fact that the mesic mixed conifer type is inherently complex and not necessarily dominated by a particular average patch size or level of geometric complexity. I suggest that the finding of larger, less fragmented, and more geometrically complex patches of Early Development be used to guide the planning and execution of restoration treatments that result in early development patches. Creating such early development patches will promote the eventual development of middle development patches with similar properties. Some of the seral stages for the xeric mixed conifer forest type ere nearly absent during the simulated historical period. As a result, I cannot make generalizations about them and will instead focus on the most common stages: Early, Mid--Open, and Late--Open. Present day patches of these seral stages in xeric mixed conifer forests are smaller, more fragmented, less geometrically complex, and contain less core area than during the simulated HRV. 

Restoration of these forests to patches that reflect a more natural succession process may be challenging for managers, given practical needs like using roads and riparian buffers as the edges of treatment units. It may not be practical to perform mechanical treatments over large areas within this cover type. However, when conducting treatments using prescription fires, creative solutions should be sought to generate more complex edges and to complete burns over sufficiently large areas that large core areas are a byproduct of the treatment.











