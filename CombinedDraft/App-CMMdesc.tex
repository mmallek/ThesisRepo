% !TEX root = master.tex
\newpage
\section{Curl-leaf Mountain Mahogany (CMM)}
\label{cmm-description}

\subsection*{General Information}

\subsubsection{Cover Type Overview}

\textbf{Curl-leaf Mountain Mahogany (CMM)}
\newline
Crosswalks
\begin{itemize}
	\item EVeg: Regional Dominance Type 1
	\begin{itemize}
		\item Curl-leaf Mountain Mahogany
	\end{itemize}

	\item LandFire BpS Model
	\begin{itemize}
		\item 0610620: Inter-Mountain Basin Curl-leaf Mountan Mahogany Woodland and Shrubland
	\end{itemize}

	\item Presettlement Fire Regime Type
	\begin{itemize}
		\item Curl-leaf Mountain Mahogany
	\end{itemize}
\end{itemize}

\noindent Reviewed by Becky Estes, Central Sierra Province Ecologist, USDA Forest Service

\subsubsection{Vegetation Description}
This landcover type is characterized by the dominance or co-dominance of \emph{Cercocarpus ledifolius}. Other shrubs such as \emph{Artemisia}, \emph{Arctostaphylos}, \emph{Ceanothus}, and \emph{Ephedra} may be present. \emph{C. ledifolius} is both a primary early successional colonizer rapidly invading bare mineral soils after disturbance and the dominant long-lived species. Depending on the effects of a given fire on the seed bank, in some cases it could take 10 years to recolonize. Where \emph{C. ledifolius} has reestablished quickly after fire, \emph{Chrysothamnus nauseosus} may codominate. Litter and shading by woody plants inhibits the establishment of \emph{C. ledifolius}, particularly in late seral stages where canopy cover is high. Reproduction often appears more dependent upon geographic variables (slope, aspect, and elevation) than biotic factors. \emph{Artemisia arbuscula} and \emph{Artemisia nova} are infrequently associated. \emph{Symphoricarpos}, \emph{Amelanchier}, and \emph{Ribes} are present on cooler, moister sites. \emph{Pinus monophylla}, \emph{Juniperus}, \emph{Pseudotsuga menziesii}, \emph{Abies magnifica}, \emph{Abies concolor}, and \emph{Pinus jeffreyi} may have sporadic presence at very low densities. In older stands the understory may consist largely of \emph{Leptodactylon pungens} (LandFire 2007, Gucker 2006).

\subsubsection{Distribution}
\emph{C. ledifolius} communities are usually found on upper slopes and ridges between 2130 and 3200 m (7000-10,500 ft), although northern stands may occur as low as 600 m (200 ft). It is more common on northwestern and northeastern aspects. Most stands occur on rocky, shallow soils and outcrops, with mature stand cover from 10-55\%. In the absence of fire, old stands may occur on somewhat deeper soils, with more than 55\% cover (LandFire 2007).

\subsection*{Disturbances}

\subsubsection{Wildfire}
Wildfires tend to be high mortality, stand-replacing fires that initiate a process of post-fire forest succession. High mortality fires kill large as well as small trees, and may kill many of the shrubs and herbs as well, although below-ground organs of at least some individual shrubs and herbs survive and re-sprout. 

\emph{C. ledifolius} is easily killed by fire and does not resprout. However, it is a primary early successional colonizer, rapidly invading bare mineral soils after disturbance. Fires are not common in early seral stages, when there is little fuel, except in chaparral-dominated stands. Stand-replacing fires are more common in mid-seral stands, where herbs and smaller shrubs provide ladder fuels. When surface fire is relatively common, stands will adopt a savanna-like woodland structure with an understory characterized by \emph{Ribes}, \emph{L. pungens}, and various grasses. Trees can become very old and will rarely show fire scars. In late, closed stands, the absence of herbs and small forbs makes fire uncommon, requiring extreme winds and drought conditions. However, stands that do burn often experience high mortality fire (LandFire 2007).

Estimates of fire rotations are available from the LandFire project and a review paper (LandFire 2007, Van de Water and Safford 2011). The LandFire project's published fire return intervals are based on a series of associated models created using the Vegetation Dynamics Development Tool (VDDT). In VDDT, fires are specified concurrently with the transition that follows them. For example, a replacement fire causes a transition to the early development stage. In the RMLands model, such fires are classified as high mortality. However, in VDDT mixed severity fires may cause a transition to early development, a transition to a more open seral stage, or no transition at all. In this case, we categorize the first example as a high mortality fire, and the second and third examples as a low mortality fire. Based on this approach, we calculated fire rotations and the probability of high mortality fire for each of the three CMM seral stages (Table~\ref{tab:cmmdesc_fire}). We computed the overall target fire rotation of 76 years based on values from Van de Water and Safford (2011). 




\begin{table}[!htbp]
\footnotesize
\centering
\caption{Fire rotation index values and probability of high severity fire (at least 75\% overstory tree mortality) probabilities. The seral stage that is most susceptible to fire (i.e., has the lowest predicted fire rotation) has a fire rotation index value of 1. Higher values correspond with lower likelihoods of experiencing wildfire. The values are relative only within an individual seral stage and should not be compared against other land cover types. Values were derived from VDDT model 0610790 (LandFire 2007) and Van de Water and Safford (2011). }
\label{tab:cmmdesc_fire}
\begin{tabular}{@{}lcc@{}}
\toprule
 \textbf{Seral Stage}    & \textbf{\begin{tabular}[c]{@{}c@{}}Fire Rotation \\ Index\end{tabular}} & \textbf{\begin{tabular}[c]{@{}c@{}}Probability of \\ High Severity Fire\end{tabular}} \\ \hline
Early (All)     		   & 4.8  & 0.17        \\
Mid--Moderate  			   & 1.0  & 0.67        \\
Late--Closed               & 28.8  & 1      \\ 
\emph{Target Fire Rotation}    			& \emph{76 years}  &   \\ 
\bottomrule
\end{tabular}
\end{table}
   			
\subsubsection{Other Disturbance}
Other disturbances are not currently modeled, but may, depending on the seral stage affected and mortality levels, reset patches to early development, maintain existing seral stages, or shift/accelerate succession to a more open seral stage. 

\subsection*{Vegetation Seral Stages}
We recognize three separate seral stages for CMM: Early Development (ED), Mid Development - Moderate Canopy Cover (MDM), and Late Development - Closed Canopy Cover (LDC) (Figure~\ref{cmm_transmodel}). Our seral stages are an alternative to ``successional'' classes that imply a linear progression of states and tend not to incorporate disturbance. The seral stages identified here are derived from a combination of successional processes and anthropogenic and natural disturbance, and are intended to represent a composition and structural condition that can be arrived at from multiple other conditions described for that landcover type. Thus our seral stages incorporate age, size, canopy cover, and vegetation composition. In general, the delineation of stages has originated from the LandFire biophysical setting model descriptive of a given landcover type; however, seral stages are not necessarily identical to the classes identified in those models.

\begin{figure}[hbt]
\centering
\includegraphics[width=0.8\textwidth]{/Users/mmallek/Documents/Thesis/statetransmodel/StateTransitionModel/shrub.png}
\caption{State and Transition Model for Curl-leaf Mountain Mahogany. Each dark grey box represents one of the three seral stages for this landcover type. Three stages of development are represented: early, middle, and late. We describe the middle development stage as characterized by moderate canopy cover and the late development stage as characterized by closed canopy cover, but these are not hard and fast rules. Transitions between states/seral stages may occur as a result of high mortality fire, low mortality fire, or succession. Specific pathways for each are denoted by the appropriate color line and arrow: red lines relate to high mortality fire, orange lines relate to low mortality fire, and green lines relate to natural succession.} 
\label{cmm_transmodel}
\end{figure}

\subsubsection{Early Development (ED)}

\paragraph{Description} \emph{C. ledifolius} seedlings rapidly invade bare mineral soils after fire. Litter and shading by woody plants inhibits establishment. Bunchgrasses and disturbance-tolerant forbs and resprouting shrubs, such as \emph{Symphoricarpos}, may be present. \emph{Ericameria} and \emph{Artemisia} seedlings are likely present. Vegetation composition will affect fire behavior, especially if chaparral species like \emph{Arctostaphylos} or \emph{Ceanothus} are present (LandFire 2007).

\paragraph{Succession Transition} In the absence of disturbance, patches in this seral stage will transition to MDM upon reaching 20 years of age. 

\paragraph{Wildfire Transition} High mortality wildfire (17\% of fires in this seral stage) recycles the patch through the ED seral stage. No transition occurs as a result of low mortality fire.

\noindent\hrulefill


\subsubsection{Mid Development - Moderate Canopy Cover (MDM)}

\paragraph{Description} \emph{C. ledifolius} may co-dominate with mature \emph{Artemisia}, \emph{Purshia}, \emph{Symphoricarpos}, or \emph{Ericameria}. Few \emph{C. ledifolius} seedlings are present. Canopy cover is variable (LandFire 2007).

\paragraph{Succession Transition} After 120 years in this stage, patches in this seral stage will transition to LDC.

\paragraph{Wildfire Transition} High mortality wildfire (67\% of fires in this seral stage) recycles the patch through the ED seral stage. No transition occurs as a result of low mortality fire.

\noindent\hrulefill


\subsubsection{Late Development - Closed Canopy Cover (LDC)}

\paragraph{Description} Moderate to high cover of large shrub- or tree-like \emph{C. ledifolius}. When low mortality fire is relatively frequent, late-successional \emph{C. ledifolius} may exhibit evidence of infrequent fire scars on older trees. Patches may consist of open savanna-like woodlands with an herbaceous-dominated understory. Other shrub species may be abundant, but decadent. When low mortality fire is absent, very few other shrubs are present, and herbaceous cover is low. Duff may be very deep, and scattered trees may occur. \emph{C. ledifolius} trees reach very old age in the absence of stand-replacing fire, potentially living over 1000 years (LandFire 2007).

\paragraph{Succession Transition} In the absence of disturbance, patches in this seral stage will remain in this seral stage. 

\paragraph{Wildfire Transition} High mortality wildfire (100\% of fires in this seral stage) recycles the patch through the ED seral stage.

\noindent\hrulefill

\subsection*{Condition Classification}
To create the initial cover and seral stage layer (2010), polygons were randomly assigned to seral stages based on a 20:10:70 distribution for early/mid/late development (based on an analysis of past fire in the project area). Random numbers between 0 and 1 were generated using numpy for Python and used to assign each CMM polygon to a seral stage.




\subsection*{References}
\begin{hangparas}{.25in}{1} 
\interlinepenalty=10000
Gucker, Corey L. ``Cercocarpus ledifolius'' \emph{Fire Effects Information System}, U.S. Department of Agriculture, Forest Service, Rocky Mountain Research Station, Fire Sciences Laboratory, 2006.  \burl{http://www.fs.fed.us/database/feis/} [Accessed 29 July 2013.]. 

LandFire. ``Biophysical Setting Models.'' Biophysical Setting 0610790: Great Basin Xeric Mixed Sagebrush Shrubland. 2007. LANDFIRE Project, U.S. Department of Agriculture, Forest Service; U.S. Department of the Interior. \burl{http://www.landfire.gov/national\_veg\_models\_op2.php}. Accessed 9 November 2012.

Van de Water, Kip M. and Hugh D. Safford. ``A Summary of Fire Frequency Estimates for California Vegetation Before Euro-American Settlement.'' \emph{Fire Ecology} 7.3 (2011): 26-57. doi: 10.4996/fireecology.0703026.

\end{hangparas}

