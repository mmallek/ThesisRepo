% !TEX root = master.tex

\chapter{Future Range of Variability}
\label{ch:FRV}

\section{Abstract}
In the Sierra Nevada, cycles of fire and vegetation recovery occur variably over large extents, as well as over long periods of time. The U.S. Forest Service's 2012 Planning Rule explicitly calls for the agency to estimate and describe the range of variability under natural disturbance regimes, and to manage for those characteristics. Recent warming and drying trends have already influenced a more frequent and proportionally more severe fire regime in the Sierra Nevada. These trends are anticipated to continue under warmer and drier climate change scenarios. I used \textsc{RMLands}, a spatially-explicit, stochastic, landscape-level disturbance and succession model, capable of simulating fine-grained processes over large spatial and long temporal extents, to evaluate trends in landscape composition and configuration under a Rabatnge of potential future climate scenarios. The study area includes the Yuba River watershed on the Tahoe National Forest, in northern California. The results show a positive relationship between burned area, high mortality fire, and drought severity (climate). I also find that today's landscape tends to depart from future ranges of variability, and that this departure is greater under increasingly warm and dry climate conditions. Based on these findings, I recommend implementing restoration efforts more aggressively and utilizing mitigation measures where the consequences of restoring natural fire regimes are socially unacceptable.



%%%%%%%%%%%%%%%%%%%%%%%%%%%%%%%%%%%%%%%%%%%%%%%%%%%%%%%%%%%%%%%%%%%%%%%%%%%%%%%%%%%%%%%%%%%%%%
%%%%%%%%%%%%%%%%%%%%%%%%%%%%%%%%%%%%%%%%%%%%%%%%%%%%%%%%%%%%%%%%%%%%%%%%%%%%%%%%%%%%%%%%%%%%%%
%%%%%%%%%%%%%%%%%%%%%%%%%%%%%%%%%%%%%%%%%%%%%%%%%%%%%%%%%%%%%%%%%%%%%%%%%%%%%%%%%%%%%%%%%%%%%%
\section{Introduction}

In the Sierra Nevada, cycles of fire and vegetation recovery occur variably over large extents, as well as over long periods of time. Ongoing disturbance results in heterogeneity in vegetation composition and configuration, which can be captured by various statistical metrics \citep{Monica2008}. Prior to European settlement, wildfire was the major source of disturbance in Sierran forests, shaping the composition and configuration of vegetation communities \citep{SNEP1996a}. Fires were primarily lightning-caused, although indigenous peoples are thought to have set fires for vegetation management, especially in the lower elevations \citep{Anderson1996}. Since then, fire suppression, logging, grazing, and mining have all interacted to alter the historical fire regime and vegetation patterns \citep{Knapp2013,Stephens2015}. 

In general, regardless of vegetation type, fires during the pre-settlement period were thought to burn primarily at low intensities \citep{Skinner1996}. High severity (over 75\% overstory canopy mortality) was uncommon \citep{Mallek2013,Stephens2015}. Most fires only affected the understory, removing downed fuels and shorter vegetation \citep{Stephens2015}. Where fires did not recur frequently, succession processes such as infill or overstory growth led to the gradual closing of the overstory tree canopy \citep{SNEP1996a}. For most of the forest cover types in the study area, high severity fire rates were low, allowing stands to succeed into late development and old growth conditions with a variety of canopy structures \citep{McKenzie2004,Safford2014}.

After large-scale fire suppression became the norm in the second half of the 19th century, less fire-tolerant species (such as \emph{Pseudotsuga menziesii} (Douglas fir) and \emph{Abies concolor} (white fir)) came to dominate areas where they had been a minor part of the vegetation community \citep{Beaty2007,Stephens2015}. Grazing and development made fires less common by altering or removing the fine fuels that carried fire \citep{Hessburg2005}. Timber harvest, especially of fire-tolerant species such as \emph{Pinus ponderosa} (ponderosa pine) and \emph{Pinus lambertiana} (sugar pine), accelerated the increased cover of species such as \emph{A. concolor} \citep{Collins2011}. Moreover, fire suppression allowed the buildup of medium size fuels and ladder fuels, promoting larger and more severe fires when they did occur \citep{Mallek2013}. Finally, the lack of natural fires led to decreased variation in fuel loading, facilitating the spread of fire over very large areas \citep{Beaty2007,Meyer2008}.

An analysis by \citep{Mallek2014} of the Eldorado and Tahoe National Forest meteorological history found that temperatures have increased over the last century, primarily manifesting as higher minimum temperatures. Changes in precipitation is occurring along elevational bands, with lower elevations experiencing a reduction in annual precipitation, while higher elevations experience an increase \citep{Mallek2014}. In addition, more precipitation is falling as rain as opposed to snow, and snow persists for less of the spring season \citep{Mallek2014}. Concern about the broad influence that climate change will have on local disturbance regimes, and subsequently, on seral stage distributions and patch configurations, motivates this study \citep{Fule2008,North2012}. Several researchers have shown that a more frequent and proportionally more severe fire regime in western forests in general and the Sierra Nevada in particular \citep{McKenzie2004,Westerling2011,Miller2012} is related to climate change. These trends are anticipated to continue under warmer and drier climate change scenarios \citep{Dale2001,Cook2004,Westerling2006, Westerling2008}. Upward shifts in the elevation of fires have also been reported in the Sierra Nevada\citep{Schwartz2015}. When fires occur at high severity, resetting succession, the potential for upward shifts of the elevational range occupied by species and vegetation assemblages increases \citep{Schwartz2015}. 

With the emergence of ecosystem management in the early 1990s, the need to recognize ecosystems as dynamic and constantly changing became well accepted, and calls to manage forests sustainably became common \citep{Christensen1996}. Within the context of forest and land management planning, the restoration of ecosystems to their pre-European settlement states was incorporated as a goal (or desired future condition) into various plans, including the Sierra Nevada Ecosystem Project \citep{SNEP1996a}. The U.S. Forest Service's 2000 Planning Rule explicitly called for the agency to estimate and describe the range of variability under natural disturbance regimes, and manage for those characteristics (36 CFR \textsection 219 2000). The need to consider the natural range of variability was maintained through various amendments to the rule, and is still present in the 2012 Planning Rule, finalized in early 2015 (36 CFR \textsection 219 2012). While the focus of management efforts has in the past been restoration, current policy emphasizes using adaptive strategies to ensure resilient ecosystems \citep{Stephens2010}.

Historical range of variability (HRV) analysis is intended to help conceptualize the mechanisms behind large-scale ecosystem functions and provide a basis from which to make predictions about how a given ecosystem will react to disturbances in the future \citep{Landres1999,Nonaka2005}. The development of methods for quantifying the natural range of variability for a diversity of landscapes in the United States augmented and facilitated the development of research focused on this task \citep{Landres1999}. By 2004, over 45 landscape fire and succession models alone had been developed, many of which were used to simulate historical dynamics \citep{Keane2004}. These models can be used to create spatially-explicit simulations of both of these key forest processes, typically outputting a set of GIS layers for each timestep of the model. These outputs can then be analyzed to quantify trajectories and patterns in the disturbance regime, seral stage composition, and landscape configuration over time \citep{Keane2004}. 

Many range of variability analyses in the United States focus on the HRV of an area. The Rocky Mountains and Oregon Coast Range in particular have been the focus of several HRV studies, while the only one that has been conducted in the Sierra Nevada took place in the southern Sierra \citep{Miller1999}. Range of variability analyses that offer a complementary analysis of future scenarios under climate change are rare (but see \cite{Keane2008} and \cite{Duveneck2014}). By simulating a range of potential future climate scenarios, I was able to generate data to use in evaluating trends in landscape pattern related to trends projected under climate change, and placed the current landscape in that context. Moreover, I use this additional information to consider which restoration strategies are likely to promote resilient forests and make sense ecologically for the area under study \citep{Duncan2010}.

\textsc{RMLands} has been used previously to assess the HRV on the San Juan National Forest and the Uncompahgre Plateau in Colorado \citep{McGarigal2005,McGarigal2005a,Romme2009}, as well as the Lolo National Forest in Montana \citep{Cushman2011}. Following the Montana study, which adapted \textsc{RMLands} to use data from the LandFire project \citep{Landfire2007}, I further adapted the software for use in the Sierra Nevada in order to prepare an HRV analysis for part of the Tahoe National Forest in California. In this study I quantify and describe a ``future range of variability'' (FRV) that can inform restoration and planning under a changing climate \citep{Fule2008,Duncan2010}. 




\paragraph*{Objectives}
In this study, my objectives were to simulate wildfires and vegetation succession under a set of potential future climate scenarios, compare the results to the current landscape across metrics describing landscape composition and configuration, and assess management implications. To do this, I simulated forest fires and succession using \textsc{RMLands}, holding all model parameters except the climate parameter constant. The climate parameter incorporated Palmer Drought Severity Index (PDSI) values from a suite of seven climate trajectories developed by the National Center for Atmospheric Research (USA) and the Canadian Centre for Climate Modelling and Analysis to the year 2100 \citep{Cook2014}. I used \textsc{Fragstats} and R software to analyze outputs and report the 90\% range of variability for simulated future metrics \citep{Fragstats2012,RCoreTeam2013}. Finally, I compared the simulated future landscape results to current conditions, drawing conclusions and assessing their implications for restoration and forest planning.





%%%%%%%%%%%%%%%%%%%%%%%%%%%%%%%%%%%%%%%%%%%%%%%%%%%%%%%%%%%%%%%%%%%%%%%%%%%%%%%%%%%%%%%%%%%%%%%%%%%%%%%%%%%%%%%%%%%%%%%%%%%%%%%%%%%%%%%%%%%%%%%%%%%%%%%%%%%%%%%%%%%%%%%%%%%%%%%%%%%%%%%%%%%%%%%%%%%%%%%%%%%%%%%%%%%%%%%%%%%%%%%%%%%%%%%%%%%%%%%%%%%%%%%%%%%%%%%%%%%%%%%%%%%%%%%%%%%%%%%%%%%%%%%%%%%%%%%%%%%%%%%%%%%%%%%%%%%%%%%%%%%%%%%%%%%%%%%%%%%%%%%%%%%%%%%%%%%%%%%%%%%%%%%%%%%%%%%%%%%%%%%%%%%%%%%%%%%%%%%%%%%%%%%%%%%%%%%%%%%%%%%%%%%%%%%%%%%%%%%%%%%%%%%%%%%%%%%%%%%%%%%%%%%%%%%%%%%%%%%%%%%%%%%%%%%%%%%%%%%%%%%%%%%%%%%%%%%%%%%%%%%%%%%%%%%%%%%%%%%%%%%%%%%%%%%%%%%%%%%%%%%%%%%%%%%%%%%%%%%%%%%%%%%%%%%%%%%%%%%%%%%%%%%%%%%%%%%%%%%%%%%%%%%%%%%%%%%%%%%%%%%%%%%%%%%%%%%%%%

\section{Methods}

\subsection{Study area}
The study area (see Figure~\ref{projectarea-ch3}) is located in the Yuba River and Sierraville Ranger Districts on the northern part of the Tahoe National Forest, and comprises about 181,500 ha. The topography of the study area consists of rugged mountains incised by two major and a few minor river drainages. Elevation ranges from about 350 m to 2500 m. Like the rest of the Sierra Nevada, the study area has a Mediterranean climate, in which summer drought typically persists from May to September \citep{Minnich2007,Skinner1996}. This annual summer drought is complemented by the development of a significant snowpack during the winter months. The area receives 30 cm to 260 cm of precipitation annually, most of which falls as snow in the middle to upper elevations \citep{Storer1963}. Within the Sierra Nevada, the heaviest precipitation occurs to the east and north of the San Francisco Bay area \citep{VanWag2006}. Datasets of the 30-year normal precipitation at 800 m resolution for the northern Sierra, illustrate that particularly high amounts of precipitation fall across the middle elevations of the study area compared to the larger region \citep{PRISMClimateGroup2004}. This increased moisture contributes to the occurrence of exceptionally productive patches of forest \citep{Littell2012}. Vegetation is tremendously diverse and changes slowly along an elevational gradient and in response to local changes in drainage, aspect, and soil structure. Vegetation communities include grasslands, chaparral, oak woodlands, mixed conifer forests, red fir forests, and subalpine forests.

% brad said to make study area more obvious for non-US readers; will probably have to redo plot for publication but this is ok for now I think
\begin{figure}
\centering
\includegraphics[width=.8\textwidth]{/Users/mmallek/Tahoe/Report3/images/studyarea.png}
\caption{The Sierra Nevada Ecoregion is outlined in brown. The study area (outlined in black) is located in the northern Sierra Nevada on the Tahoe National Forest, and follows the boundaries of the Yuba River watershed.}
\label{projectarea-ch3}
\end{figure}

Sierran Mixed Conifer - Xeric and Sierran Mixed Conifer - Mesic forests are the two most prevalent mapped cover types within the study area, together comprising 63\% of the landscape. Historically, fire recurred frequently in these cover types. They are characterized by five conifers and one hardwood: \emph{A.~concolor, P.~menziesii, P.~ponderosa, P.~lambertiana, Calocedrus decurrens}, and \emph{Quercus kelloggii}. At least three are typically present in any given stand \citep{Landfire2007}. All of these species can be found in either cover type, but some are more closely associated with either the mesic or xeric variant. The characteristic species of the mesic type, \emph{A.~concolor} and \emph{P.~menziesii}, are less adapted to fire, and mesic forests tend to have longer fire rotations \citep{Mallek2013}. Species characteristic of the xeric type, \emph{P.~ponderosa}, \emph{P.~lambertiana}, plus \emph{Q.~kelloggii}, are more fire-adapted \citep{Landfire2007}. Mean rotations could be as short as 20 years in \emph{P. ponderosa}-dominated forests \citep{Mallek2013}. \emph{C.~decurrens} is found in both subtypes, but is very rarely dominant. The distribution of these species is normally an outcome of the variation in the frequency and severity of wildfire under natural conditions, although alteration of these conditions can affect their distribution. Moderate severity fire that results in more open overstory canopy cover was more prevalent in more xeric environments, including south-facing slopes and ridges \citep{SNEP1996a,SNEP1996,Mallek2013,Safford2014}. \emph{A.~concolor} was frequently the most ubiquitous species, especially on north-facing slopes. \emph{P.~ponderosa} was historically the dominant species, and is strongly associated with a frequent, low severity fire regime \citep{WHR1988,Landfire2007}. During the historical period, fire rotations generally had a positive relationship with both moisture and elevation, and could vary widely around observed mean rotation periods \citep{Mallek2013}. 



\subsection{Simulations}

\textsc{RMLands} is a spatially-explicit, stochastic, landscape-level disturbance and succession model capable of simulating fine-grained processes over large spatial and long temporal extents \citep{McGarigal2005}. It is part of a class of models known as landscape disturbance (often fire) and succession models, many of which are still in use today (such as \textsc{landis} \citep{He1999}, \textsc{zelig-l} \citep{Miller1999}, \textsc{safe-forests} \citep{Sessions1997} and \textsc{landsum} \citep{Keane2012}). \textsc{RMLands} is grid-based and simulates fire on landscapes in a spatially explicit and realistic manner, in that fire perimeters resemble those that occur naturally \citep{McGarigal2005,McGarigal2005a}. A component of many landscape fire and succession models are state and transition models, which provide a framework for defining the fundamental vegetation communities and the probability over time of completing a transition from one defined set of vegetative community characteristics to another \citep{Stringham2003,Blankenship2015}. State transitions in \textsc{RMLands} are simulated at the 30 m pixel scale \citep{Cushman2011}. Transitions may take place in response to fire or in the absence of it (natural succession) \citep{McGarigal2012}. Outputs from the model are readable by the landscape pattern analysis software \textsc{Fragstats}, which facilitates the landscape configuration analysis \citep{Fragstats2012}.

In \textsc{RMLands}, fires spread probabilistically based on the susceptibility of an individual cell. It does not contain a mechanistic fire model and fuels are not directly incorporated into fire spread. In addition, I did not assign fires as a whole to a \emph{low}, \emph{mixed}, or \emph{high severity} status. Some fire ecologists combine fire attributes such as flame length and fire size into their interpretation of the relative \emph{severity} of a particular fire \citep{Agee1993}. Ecologists working at other scales and not working with models often describe \emph{mixed severity} regimes, in which the area within a fire perimeter burns primarily at low to moderate levels, but contains some patches that burn at high severity and thus experience stand-replacement \citep{Collins2010,Kane2013}. In other words, mixed severity fires burn in a mosaic pattern \citep{Beaty2001}. If I were to adopt that definition, nearly all fires would be classified as \emph{mixed severity} due to the resolution at which fire mortality is defined (the 30 m grid cell), rendering this perspective moot for the study. Instead, I focused on defining conditions under which transitions among potential states within a given cover type occur or not. 

In this application of \textsc{RMLands}, high mortality fire always creates early successional conditions within both mesic and xeric mixed conifer forests. High mortality fire, defined as fire in which over 75\% of the overstory canopy is killed, resets the successional pathway \citep{Agee1993}. Either chaparral or trees may establish after a cell experiences this level of fire during the simulations. Chaparral, a community that includes \emph{Arctostaphylos, Ceanothus}, and \emph{Chrysolepis} species, is often treated in vegetation classifications as its own cover type \citep{USDAForestService2008,VandeWater2011}. This is in part due to the fact that chaparral establishment tends to inhibit the establishment and growth of conifer species, thus delaying succession to (mid-development) forest \citep{Landfire2007}. When studies occur along short temporal scales, it may be more meaningful to categorize these two vegetation communities as separate. However, this study is focused on long temporal scales, and within the study area chaparral will eventually succeed to trees. Consequently, both communities are considered indicative of the Early Development seral stage. However, the model does not specify which community occurs in any given cell assigned to the Early Development seral stage.

\textsc{RMLands} evaluates and classifies fire by its effects on individual cells. First, it evaluates whether a cell burned. Next, all burned cells are evaluated probabilistically to determine if the fire in a given grid cell is high severity or low severity. If a cell burns at high severity, it is classified to a high mortality outcome and transitions to the Early Development seral stage. If it burns at low severity (which encompasses all other theoretical severity levels), then it is immediately evaluated and either remains in the current seral stage or transitions to a more open seral stage. An individual fire is nearly always composed of a mix of cells assigned to high and low mortality outcomes. Recently, some researchers have differed on whether 75\% or 95\% overstory tree mortality is a more appropriate cutoff point for defining a ``stand-replacing'' event \citep{Mallek2013,Fule2014}. In this thesis, I use 75\% as the cutoff, which is widely used in the literature \citep{Agee1993,Agee2007,Miller2009,Baker2014}. Also, I use \emph{fire severity} to refer to actual fire effects and empirical observations, and \emph{fire mortality} to refer to fire effects as specified in or output by the model.

%%%%
%%%%


In collaboration with project partners, I developed a nested land cover and seral stage classification based on Presettlement Fire Regimes \citep{VandeWater2011} and the LandFire Vegetation Dynamics models \citep{Landfire2007}. The resulting set of 31 land cover types was crosswalked to the U.S. Forest Service Existing Vegetation spatial dataset \citep{USDAForestService2009} using the Northern Sierra \textsc{calveg} descriptions \citep{USDAForestService2008}. I also used U.S. Forest Service corporate spatial data to develop input layers for the biological (e.g., seral stage, age) and physical environments (e.g., elevation). State and transition models for the 25 cover types that undergo succession were based on the Vegetation Dynamics Development Tool models associated with the LandFire project \citep{Landfire2007}, and refined with input from project partners to capture subtle changes in succession and transition at the scale of the study area. In general, model parameters were developed using meta-analyses published in the literature. For example, transition probabilities were calculated using the Vegetation Dynamics models \citep{Landfire2007}, several fire rotation calibration parameters were taken from \citet{Mallek2013}, and wind direction information was obtained from six area weather stations.\footnote{Weather station names: Rice Canyon, Saddleback, Downieville, White Cloud, Emigrant Gap, and Blue Canyon. Historical data was downloaded from \burl{http://www.wunderground.com}.} I used landscape conditions as of 2010 as the starting point for all simulations and as the ``current'' conditions for comparison to the FRV results.

Although \textsc{RMLands} is a process-based model with parameters sourced from the literature, project partners and I had greater confidence in some parameters than others, especially regarding their function within the \textsc{RMLands} framework. Consequently, I calibrated the model parameters by iteratively adjusting certain low-confidence parameters to optimize the output values for parameters known with high confidence. Specifically, I manipulated an ignition calibration coefficient (number of attempted fire starts) and a fire rotation index (mean return interval value for the Weibull distribution), and measured calibration success based on conformity to empirically derived rotation values at the cover type level. Fire rotation index values were multiplied by a single factor across all seral stages of a given cover type; that is, cover types were modified as groups but the index ratios within them were held constant.

The calibration target was defined as $\pm 10$\% of the original target rotations for the nine most prevalent cover types in the study area (Sierran Mixed Conifer - Mesic, Xeric, and Ultramafic variants; Red Fir - Mesic and Xeric variants; Oak-Conifer Forest and Woodland - standard and Ultramafic variants; Mixed Evergreen - Mesic and Xeric variants). I focused on these nine types because they all extend across more than 1,000 ha, and are thus statistically stable from simulation to simulation. Target values were based on published empirical values and refined with input from project partners \citep{Landfire2007,Mallek2013}. I selected rotation as the calibration target because empirical values were available from the literature and because fire rotation is a fundamental measurement that \textsc{RMLands} was designed to capture. In addition, using rotation ties calibration to a parameter that is relatable to U.S. Forest Service staff and that can be used as a target by managers in various programs.

In a historical range of variability analysis, the simulation length is designed to capture multiple full disturbance cycles and the consequent effects on vegetation. This is possible when the climate parameter oscillates around a mean, but not when a clear trend exists, such as that observed due to ongoing climate change. Therefore, instead of running the model for a long time and recycling the climate parameter values, I simulated the period from 2010--2100 repeatedly: 100 times for each of seven climate parameter sequences. No true model equilibration was done for the future simulations, but I elected to include only the final five timesteps (25 years) of the results in order to achieve some distance from the starting (current) conditions, maintain focus on the ending trajectory of climate and the range of variation possible in the landscape at the end of the 21st century, yet still capture the variability inherent to the PDSI-based climate parameter (Figure \ref{fig:pdsi-final5}).



%%%% Brad's comment: Is it possible to validate this model in some way? In particular, how do we know that varying the climate multiplier affects fires in a realistic way? Could you hindcast to see how well the PDSI for past years matches the historical fire pattern?

%%%% Notes in response: not easily. some language in hrv that addresses this. we don't know, beyond my assumption that the model treats it realistically in the first place, which is based on Kevin's word. You could hindcast, maybe. But I'm not expecting a super strong relationship. Don't know exactly how to deal with this comment but it feels important.
%%%% More notes: HRV article has a brief comparison of climate vs. disturbed area. Weak relationship. We expect cascading results affecting landscape structure and composition as well as changes to fires. Some non-standard fire behavior may occur but no real way to test this because we don't know what fires will "look" like in the future.

\subsection{Climate Parameter and PDSI}

The climate parameter values used in this study were derived from potential future climate models initialized using the set of parameters for Representative Concentration Pathway (RCP) 8.5. RCP8.5 includes no specific climate mitigation target, unlike the other three RCP scenarios in use \citep{Riahi2011}. As a result, it is considered a reference, or baseline scenario, in which greenhouse gas emission and concentrations increase over time without leveling out \citep{Riahi2011}. A literature review during the RCP development process designated radiative forcing in 2100 of 8.5 W/m$^2$ as the high end of plausible futures that had been modeled \citep{VanVuuren2011}. The corresponding $\text{CO}_2$ concentration is \textgreater $\sim$1370 $\text{CO}_2$ -eq in 2100, compared to 375 $\text{CO}_2$ -eq in 2005. The 66\% range of temperature increase above pre-industrial levels under the RCP8.5 scenario is 4.0\textdegree C--6.1\textdegree C \citep{Rogelj2012}. Since the development of RCP8.5 as a scenario, storylines have been produced that describe how such a scenario could come about. The main storyline released for RCP8.5 describes a world in which the human population rises to 12 billion by 2100. Little progress in energy efficiency and the food demands of the increasing population lead to high energy demands, which are met by coal-intensive technology choices \citep{Riahi2011}. Data from the RCP8.5 scenario was used to model climate variables like temperature and precipitation, which \citet{Cook2014} then used to project trajectories of drought severity to 2100.

A climate variable unique to each timestep in the model is the key parameter that varies across the scenarios in this study (Figure~\ref{fig:pdsi_future}). The Palmer Drought Severity Index (PDSI), a commonly used tool to assess drought in the western United States, forms the basis of this parameter \citep{Cooketal2004,Cook2004}. The PDSI incorporates precipitation and temperature within a water balance model, and is appropriate for use at scales the size of my study area \citep{HeimJr2002}. I used PDSI data from 2010 to 2099 calculated by \citet{Cook2014} to generate the climate parameter sequence for input to \textsc{RMLands}. The \citet{Cook2014} modeled PDSI values were fit to observed PDSI values from 1900 to the present, and used the same methodology as the North American Climate Atlas \citep{Cook2004}. Project partners analyzed the suite of climate models for which \citet{Cook2014} had calculated PDSI, and selected the \textsc{ccsm4} model from the National Center for Atmospheric Research and the \textsc{gfdl-esm2m} model from the NOAA Geophysical Fluid Dynamics Laboratory for use in this study. The \textsc{ccsm4} model projects warmer temperatures and similar precipitation levels in northern California, relative to the past several hundred years, while the \textsc{gfdl-esm2m} model projects hotter and drier weather \citep{Weiss2013}.

Six PDSI sequences based on the \textsc{ccsm4} model were available, so I treated each run as a separate scenario. To generate climate parameters from the PDSI sequences, I calculated the inverse Euclidean distance-weighted mean of PDSI values at 21 points surrounding the centroid of the study area. I then rescaled the results around the mean and standard deviation of the set of PDSI values representative of the 300 years prior to European settlement, during which the climate was more stable, and characterized by dynamic equilibrium rather than a trend \citep{Diamond1969,Cook2004}. This new set of values was the climate parameter trajectory used in my study. Because the parameter is a multiplier within the model, it was necessary to use 1 as the neutral value. Thus climate parameter values less than 1 reduce susceptibility, likelihood of fire initiation, and fire spread, while climate parameter values greater than 1 increase these properties. Each of the seven total runs followed a unique pattern and trend (Figure~\ref{fig:pdsi_future}). I present results in order of increasing median value for the climate parameter during the simulations to facilitate interpretation (Figure~\ref{pdsi-boxplots}). %The \textsc{ccsm-1} model has a distribution of climate parameter values similar to that of the historical period, with a median near 1.


\begin{figure}[!htbp]
\centering
  \subfloat[][]{
    \centering
	\includegraphics[height=0.2\textheight]{/Users/mmallek/Documents/Thesis/Plots/pdsi/futureclimate_wlm.png}
    \label{fig:pdsi-lm}
  }%
  \subfloat[][]{
  	\centering
	\includegraphics[height=0.20\textheight]{/Users/mmallek/Documents/Thesis/Plots/pdsi/future_last5timesteps.png}
	\label{fig:pdsi-final5}
	}
    \caption{(a) Climate parameter trajectory for the 18 timesteps used in the simulations for the six scenarios from the \textsc{ccsm4} model and the single scenario from the \textsc{gfdl-esm2m} model. Solid lines connect the climate parameter values for each timestep, and the dashed line represents a fitted linear regression to the data. (b) Zoom on the final five timesteps (without regression lines) for better visualization of variability in each scenario. The climate parameter in \textsc{RMLands} is based on the Palmer Drought Severity Index. Models in the legend appear in descending order from least to greatest mean value for the full time series of the simulation. In a historical range of variability analysis, the PDSI values average to 1 over a several hundred year period.}
\label{fig:pdsi_future}

\end{figure}

\begin{figure}[!htbp]
\centering
\includegraphics[width=\textwidth]{/Users/mmallek/Documents/Thesis/Seminar/frv-climparam-slide.png}
\caption{Boxplots of climate parameter value for the six runs of the \textsc{ccsm4} model and the single run from the \textsc{gfdl-esm2m} model. The climate parameter in \textsc{RMLands} is based on the Palmer Drought Severity Index. Models are arranged left-to-right from least to greatest median value for the full time series of the simulation, after the HRV. Boxplot whiskers extend from the $5^{\text{th}}-95^{\text{th}}$ range of variability for each model.}
\label{pdsi-boxplots}
\end{figure}


\subsection{Evaluating Future Range of Variability}
\label{sec:evalFRV}
I evaluated the range of variability in the wildfire regime across the future scenarios by calculating and comparing the median area burned on the landscape for each scenario across severity levels for the full landscape and the two most prevalent cover types in the study area, Sierran Mixed Conifer - Mesic and Sierran Mixed Conifer - Xeric (Figure~\ref{fig:dareacomp}). %\todo{compared to what we'd expect based on proportional difference?}
I also computed the fire rotation, defined as the time it takes to burn an area equivalent to the area under study \citep{Agee1993}, for those two cover types, and compared the simulated future scenario results to the empirically derived historical fire rotation (Figure~\ref{fig:frotation}). 
%


%\subsection*{Landscape composition}
The evaluation of landscape composition compares the seral stage distribution for mesic and xeric mixed conifer forests across future scenarios and to the current distribution. I report the 90\% range of variability and assess departure for each seral stage of these two focal cover types.\footnote{Both cover types include seven seral stages: Early Development, Mid--Closed, Mid--Moderate, Mid--Open, Late--Closed, Late--Moderate, and Late--Open. The first term refers to the developmental stage and the second to the proportion of canopy cover (breakpoints for canopy cover are at 40\% and 70\%).}


%
%\subsection*{Landscape configuration}
I used \textsc{Fragstats} software \citep{Fragstats2012} to conduct the spatial pattern analysis and assess differences in landscape configuration between the current landscape and the future range(s) of variability. However, several of the metrics are redundant with one another, and so I focused on a subset of four metrics to simplify interpretation: area-weighted mean patch area, area-weighted mean core area, clumpiness index, and area-weighted mean shape index. I report the 90\% range of variability using boxplots. %and for SMC?

To assess landscape departure, I compared the current landscape to the FRV, and classified departure based on the following standards: if the current landscape metric value falls within the $25^{\text{th}}-75^{\text{th}}$ percentile range (the box in my boxplots), it is considered within the FRV. If it falls within the $5^{\text{th}}-25^{\text{th}}$ percentile range or the $75^{\text{th}}-95^{\text{th}}$ percentile range (the whiskers in my boxplots), it is moderately departed. If it falls outside that range, it is completely departed.

%  (\textsc{area\_am}) (\textsc{core\_am}) (\textsc{contag}) (\textsc{gyrate\_am}) (\textsc{shape\_am}) (\textsc{siei})


\subsection{Methodological Limitations} 
Some limitations of this work should be clarified before applying the results in a management context. \textsc{RMLands} simulates wildfires, but there are many other disturbance processes that exist at varying scales that are not simulated here, including insects and disease, wind-throw, wild ungulate and beaver herbivory, avalanches, and other forms of soil movement. The complex interactions among them that characterize real landscapes are also, as a result, omitted from consideration. %
The cover type and seral stage data used as inputs were the best available, but are not perfect because of both human and computer errors in classification, and because they combine three separate classification efforts \citep{USDAForestService2009}. Improvements to the existing vegetation layer, such as the LiDAR maps currently in development, would improve the model by offering a standardized and more accurate classification of existing vegetation for the full study area. Also, if new data became available that would alter the model parameterization, then I would expect the estimate of the future range of variability to change. 

In addition, I did not simulate changes in the spatial configuration of land cover types. That is, I did not model elevational shifts in forest type location or type conversions after fire or drought events. Such changes are likely in the next 100 years and have already occurred in the region \citep{Bachelet2001}. Because of this, my results do not predict a specific future outcome, and should be used in conjunction with other studies, especially those focused on range shifts, to anticipate future vegetation patterns and manage accordingly. 

Finally, the climate parameter used in \textsc{RMLands} is a proxy for climate, not a direct measure of it. It incorporates two key pieces of climate data, temperature and precipitation, that closely relate to wildfire as a disturbance process \citep{Cushman2011}. It is based on the PDSI \citep{HeimJr2002}, but I removed some of the variability and extremes present in the raw PDSI in order to make it compatible with the model. Specifically, I collapsed the projected PDSI data into 5-year summer averages and rescaled it \citep{Cushman2011}. In addition, I utilized a single RCP, 8.5, which is the scenario projecting the most significant change in climate, after \citet{Cook2014}. It represents a ``business as usual'' storyline, and thus may not accurately predict the change to the future range of variability if significant action is taken to reduce greenhouse gas outputs and slow the rate of climate change in the next few decades \citep{Riahi2011}. Further, the ``business as usual'' scenario is \emph{not} a worst-case scenario. Events such as methane releases from the ocean and arctic lakes, thawing permafrost, and more, larger tundra fires could affect the rate of parameters associated with climate change such as temperature, as well as the eventual sum of greenhouse gases in the atmosphere \citep{Racine2004,Higuera2008,Schuur2009,Shakova2010,Hu2015}.






%%%%%%%%%%%%%%%%%%%%%%%%%%%%%%%%%%%%%%%%%%%%%%%%%%%%%%%%%%%%%%%%%%%%%%%%%%%%%%%%%%%%%%%%%%%%%%%%%%%%%%%%%%%%%%%%%%%%%%%%%%%%%%%%%%%%%%%%%%%%%%%%%%%%%%%%%%%%%%%%%%%%%%%%%%%%%%%%%%%%%%%%%%%%%%%%%%%%%%%%%%%%%%%%%%%%%%%%%%%%%%%%%%%%%%%%%%%%%%%%%%%%%%%%%%%%%%%%%%%%%%%%%%%%%%%%%%%%%%%%%%%%%%%%%%%%%%%%%%%%%%%%%%%%%%%%%%%%%%%%%%%%%%%%%%%%%%%%%%%%%%%%%%%%%%%%%%%%%%%%%%%%%%%%%%%%%%%%%%%%%%%%%%%%%%%%%%%%%%%%%%%%%%%%%%%%%%%%%%%%%%%%%%%%%%%%%%%%%%%%%%%%%%%%%%%%%%%%%%%%%%%%%%%%%%%%%%%%%%%%%%%%%%%%%%%%%%%%%%%%%%%%%%%%%%%%%%%%%%%%%%%%%%%%%%%%%%%%%%%%%%%%%%%%%%%%%%%%%%%%%%%%%%%%%%%%%%%%%%%%%%%%%%%%%%%%%%%%%%%%%%%%%%%%%%%%%%%%%%%%%%%%%%%%%%%%%%%%%%%%%%%%%%%%%%%%%%%%%%
\section{Results}
%* Looks like the absolute range of variability is a lot wider in the future than during the HRV!  \\
%* In some cases the landscape is out of HRV but within most of the FRVs

% brad says figures > table. precise results not necessary
% becky found this confusing
\subsection{Natural fire regime}

I analyzed the wildfire disturbance regime in terms of its effect on the full study area, on mesic mixed conifer forests, and on xeric mixed conifer forests. 
%
The median area of land burned by wildfire during simulations of seven alternative future climate trajectories was generally greater when trajectories included higher climate parameter values (Figure~\ref{fig:dareacomp}). In general, this trend was most pronounced for the measurement of median area burned at high mortality. The median area burned at low mortality did not follow a trend with respect to the climate trajectory used. I also observed a decrease in the ratio of low to high mortality fire for the full study area and the two focal mixed conifer forest cover types. This occurred as area burned at high mortality increased relative to the area burned at low mortality. Compared to the full landscape, this was slightly less conspicuous in mesic mixed conifer forests. However, the results in xeric mixed conifer forests are more striking: the \textsc{gfdl-esm2m} scenario, which features the highest climate parameter values, produced an equal proportion of area burned at high versus low mortality (Figure~\ref{fig:dareacomp_smcx}.

% I think Becky crossed this out because it's discussion, not results?
% In general, this trend was strongest for area burned at high mortality, which drove the increase in total area burned, as differences in area burned at low mortality did not appear to be ecologically significant. These observations hold for the full study area as well as the mixed conifer forests alone. More striking is the fact that 

% (\textsc{smc\_}) (\textsc{smc\_x})

%make these plots to the same scale
\begin{figure}[!htbp]
  \centering
    \subfloat[][]{
	\centering
	\includegraphics[width=0.5\textwidth]{/Users/mmallek/Documents/Thesis/Plots/darea/darea-allfmodels-combo.png}
	\label{fig:darea_modelcomp}
	} \\
  \subfloat[][]{
    \centering
    \includegraphics[width=0.5\textwidth]{/Users/mmallek/Documents/Thesis/Plots/darea/darea-allfmodels-smcm.png}
    \label{fig:dareacomp_smcm}
  } 
  \subfloat[][]{
  \centering
    \includegraphics[width=0.5\textwidth]{/Users/mmallek/Documents/Thesis/Plots/darea/darea-allfmodels-smcx.png}
    \label{fig:dareacomp_smcx}
  }
    \caption{Barplots showing the proportion burned at low mortality, high mortality, and any mortality fire, across the seven future climate scenarios. From left to right, scenarios are presented in order of increasing median climate parameter value. (a) full study landscape, (b) Sierran Mixed Conifer - Mesic, and (c) Sierran Mixed Conifer - Xeric}
  \label{fig:dareacomp}
\end{figure}

% Fire rotation
I also calculated the fire rotation for each scenario, and plotted these results across all scenarios (Figure~\ref{fig:frotation}). The range of median rotation values across the seven future climate scenarios always encompassed the empirically-derived historical rotation values. Although there is considerable variability, rotation values for high mortality events are the lowest during scenarios with high climate parameter values. Changes to the low mortality rotations are slight, but generally seem to increase proportionately to decreases in high mortality rotations, such that the ``any mortality'' rotation decline is modest.



\begin{figure}
\centering
\includegraphics[width=0.6\textwidth]{/Users/mmallek/Documents/Thesis/Plots/rotation/rotation_all.png}
\caption{Fire rotation values across scenarios for the full extent of the study area (red), Sierran Mixed Conifer - Mesic (green), and Sierran Mixed Conifer - Xeric (blue), during the last five timesteps of the simulations. The ``historical'' values represent fire rotations during the pre-European settlement period, and were used in initial model calibration. Point shapes correspond to different mortality levels from fire: low mortality (circles), high mortality (squares), and overall mortality (both high and low combined, triangles). Connecting lines have been included to aid in visualizing the trajectories for cover types and mortality values across scenarios.}
\label{fig:frotation}
\end{figure}



\subsection{Landscape Pattern}

\paragraph*{Seral Stage Distribution}
The landscape pattern analysis focuses first on changes to the seral stage distribution of mesic and xeric mixed conifer forests. Evidence of both high mortality fire, which always triggers a transition to Early Development, and low mortality fire, which can thin a stand and cause a transition to a more open canopy condition, is visible in examining the output grids.

In order to determine whether the results were an artifact of the initial condition, I also examined the trajectory of all seral stages for both mesic and xeric mixed conifer forests (Figures~\ref{fig:median_trajectory_smcm} and \ref{fig:median_trajectory_smcx}). I observed an increase in the proportion of Early Development in both focal cover types. This increase was followed by dynamic, shifting proportions of all seral stages in the resulting timesteps. The change in the first few timesteps was often (but not always) the largest shift in a given seral stage's proportion, but it was almost never larger than the shift over the remainder of the simulation. Shifts in xeric forests were steeper and converged more across climate scenarios by the end of the simulation than in the mesic forests.


\begin{figure}[htbp]
 \captionsetup[subfigure]{labelformat=empty}
  \centering
  \subfloat[][]{
    \centering
    \includegraphics[height=0.33\textwidth]{/Users/mmallek/Documents/Thesis/Plots/seralstage-trajectory-medians/2410-trajectory-median-legend.png}
  }\\%
  %\qquad
  \subfloat[][]{
    \includegraphics[width=0.33\textwidth]{/Users/mmallek/Documents/Thesis/Plots/seralstage-trajectory-medians/2420-trajectory-median-title.png}
  } 
    \subfloat[][]{
    \centering
    \includegraphics[width=0.33\textwidth]{/Users/mmallek/Documents/Thesis/Plots/seralstage-trajectory-medians/2421-trajectory-median-title.png}
  }%
  %\qquad
  \subfloat[][]{
    \includegraphics[width=0.33\textwidth]{/Users/mmallek/Documents/Thesis/Plots/seralstage-trajectory-medians/2422-trajectory-median-title.png}
  } \\
    \subfloat[][]{
    \centering
    \includegraphics[width=0.33\textwidth]{/Users/mmallek/Documents/Thesis/Plots/seralstage-trajectory-medians/2430-trajectory-median-title.png}
  }%
      \subfloat[][]{
    \centering
    \includegraphics[width=0.33\textwidth]{/Users/mmallek/Documents/Thesis/Plots/seralstage-trajectory-medians/2431-trajectory-median-title.png}
  } 
  \subfloat[][]{
    \includegraphics[width=0.33\textwidth]{/Users/mmallek/Documents/Thesis/Plots/seralstage-trajectory-medians/2432-trajectory-median-title.png}
  }
    \caption{Median trajectory across all climate scenarios for each Sierran Mixed Conifer - Mesic seral stage. Each climate scenario is shown in a different color. After a noisy beginning, a trend emerges for most seral stages by the final timesteps of the simulation.}
  \label{fig:median_trajectory_smcm}
\end{figure} 

\begin{figure}[htbp]
 \captionsetup[subfigure]{labelformat=empty}
  \centering
  \subfloat[][]{
    \centering
    \includegraphics[height=0.33\textwidth]{/Users/mmallek/Documents/Thesis/Plots/seralstage-trajectory-medians/2610-trajectory-median-legend.png}
  }\\%
  %\qquad
  \subfloat[][]{
    \includegraphics[width=0.33\textwidth]{/Users/mmallek/Documents/Thesis/Plots/seralstage-trajectory-medians/2620-trajectory-median-title.png}
  } 
    \subfloat[][]{
    \centering
    \includegraphics[width=0.33\textwidth]{/Users/mmallek/Documents/Thesis/Plots/seralstage-trajectory-medians/2621-trajectory-median-title.png}
  }%
  %\qquad
  \subfloat[][]{
    \includegraphics[width=0.33\textwidth]{/Users/mmallek/Documents/Thesis/Plots/seralstage-trajectory-medians/2622-trajectory-median-title.png}
  } \\
    \subfloat[][]{
    \centering
    \includegraphics[width=0.33\textwidth]{/Users/mmallek/Documents/Thesis/Plots/seralstage-trajectory-medians/2630-trajectory-median-title.png}
  }%
      \subfloat[][]{
    \centering
    \includegraphics[width=0.33\textwidth]{/Users/mmallek/Documents/Thesis/Plots/seralstage-trajectory-medians/2631-trajectory-median-title.png}
  } 
  \subfloat[][]{
    \includegraphics[width=0.33\textwidth]{/Users/mmallek/Documents/Thesis/Plots/seralstage-trajectory-medians/2632-trajectory-median-title.png}
  }
    \caption{Median trajectory across all climate scenarios for each Sierran Mixed Conifer - Xeric seral stage. Each climate scenario is shown in a different color. After a noisy beginning, a trend emerges for most seral stages by the final timesteps of the simulation.}
  \label{fig:median_trajectory_smcx}
\end{figure} %smcx

%I observe clear trends in three seral stages across both cover types (Figures~\ref{fig:covcond_smcm}-\ref{fig:covcond_smcx}). The proportion of Early Development increased in both with increasing climate parameter values, while Late--Closed and Late--Moderate both decreased. Surprisingly, as the climate parameter increased, the proportion of Mid--Open in the mesic mixed conifer forest increased, while in the xeric mixed conifer the proportion of Mid--Open decreased. In general, the proportion of the current landscape in each seral stage differed substantially from the future ranges of variability. Across all the seral stages, the proportion of each cover type in the Early Development stage increased most dramatically. I focus the configuration metrics analysis, then, on the Early Development stage of Sierran Mixed Conifer - Mesic and Sierran Mixed Conifer - Xeric.

%or could write as focused on one cover type at a time. 
I also evaluated the degree to which the current seral stage distribution for each cover type departs from the simulated FRV. 
%
In the mesic mixed conifer cover type, the current seral stage distribution is within the projected FRV for scenarios with low to moderate climate parameter values, but departs from the FRV for scenarios with high climate parameter values (Figure~\ref{fig:covcond_smcm}. The large increase in the proportion of the landscape in the Early Development and Mid--Open stages comes at the expense of all the Late Development stages.
%
In the xeric mixed conifer type, the results were even more dramatic. As Figure~\ref{fig:covcond_smcx} shows, I observed substantially greater proportion of Early Development in scenarios with high climate parameter, again at the expense of other seral stages. After Early Development, the Mid-Open and Late--Open stages were the next most common, and were fairly prevalent on the landscape. Overall, I observed complete departure of the current landscape from most seral stages across the future scenarios. The exceptions to this pattern were the Mid--Open and Late--Moderate seral stages. For the Mid--Open seral stage, the current landscape is completely departed from the FRV for all but the \textsc{gfdl-esm2m} scenario. In that case, the current landscape is moderately departed within the FRV. In the Late--Moderate seral stage, the current proportion was within the FRV for scenarios with low to moderate climate parameter values, and moderately departed within the FRV for scenarios with moderate to high climate parameter values. 
%
Across all the seral stages, the proportion of each cover type in the Early Development stage increased most dramatically. The configuration metrics analysis focuses on the characteristics of Early Development patches of Sierran Mixed Conifer - Mesic and Sierran Mixed Conifer - Xeric.

\begin{figure}[htbp]
 \captionsetup[subfigure]{labelformat=empty}
  \centering
  \subfloat[][]{
    \centering
    \includegraphics[width=0.5\textwidth]{/Users/mmallek/Documents/Thesis/Plots/covcond-byscenario/2410-boxplots.png}
  }%
  %\qquad
  \subfloat[][]{
    \includegraphics[width=0.5\textwidth]{/Users/mmallek/Documents/Thesis/Plots/covcond-byscenario/2420-boxplots.png}
  } \\
    \subfloat[][]{
    \centering
    \includegraphics[width=0.5\textwidth]{/Users/mmallek/Documents/Thesis/Plots/covcond-byscenario/2421-boxplots.png}
  }%
  %\qquad
  \subfloat[][]{
    \includegraphics[width=0.5\textwidth]{/Users/mmallek/Documents/Thesis/Plots/covcond-byscenario/2422-boxplots.png}
  } \\
    \subfloat[][]{
    \centering
    \includegraphics[width=0.5\textwidth]{/Users/mmallek/Documents/Thesis/Plots/covcond-byscenario/2430-boxplots.png}
  }%
  %\qquad
    \subfloat[][]{
    \centering
    \includegraphics[width=0.5\textwidth]{/Users/mmallek/Documents/Thesis/Plots/covcond-byscenario/2431-boxplots.png}
  } \\
  \subfloat[][]{
    \includegraphics[width=0.5\textwidth]{/Users/mmallek/Documents/Thesis/Plots/covcond-byscenario/2432-boxplots.png}
  }
    %\qquad
  %\subfloat[][]{
  %  \includegraphics[width=0.5\textwidth]{/Users/mmallek/Documents/Thesis/Plots/covcond-frvhrv/SMCM-frvhrv-boxplots.png}
  %}
  \caption{Boxplots illustrating the simulated future range of variability across each climate trajectory for Sierran Mixed Conifer - Mesic. The dashed black horizontal line represents the current condition. Boxplot whiskers extend from the $5^{\text{th}} - 95^{\text{th}}$ range of variability for each model. Climate models appear left-to-right in order of increasing median climate parameter value. See Section \ref{sec:evalFRV} for a detailed explanation of how to interpret departure based on these plots.}
  \label{fig:covcond_smcm}
\end{figure} %smcm

\begin{figure}[htbp]
 \captionsetup[subfigure]{labelformat=empty}
  \centering
  \subfloat[][]{
    \centering
    \includegraphics[width=0.5\textwidth]{/Users/mmallek/Documents/Thesis/Plots/covcond-byscenario/2610-boxplots.png}
  }%
  %\qquad
  \subfloat[][]{
    \includegraphics[width=0.5\textwidth]{/Users/mmallek/Documents/Thesis/Plots/covcond-byscenario/2620-boxplots.png}
  } \\
    \subfloat[][]{
    \centering
    \includegraphics[width=0.5\textwidth]{/Users/mmallek/Documents/Thesis/Plots/covcond-byscenario/2621-boxplots.png}
  }%
  %\qquad
  \subfloat[][]{
    \includegraphics[width=0.5\textwidth]{/Users/mmallek/Documents/Thesis/Plots/covcond-byscenario/2622-boxplots.png}
  } \\
    \subfloat[][]{
    \centering
    \includegraphics[width=0.5\textwidth]{/Users/mmallek/Documents/Thesis/Plots/covcond-byscenario/2630-boxplots.png}
  }%
      \subfloat[][]{
    \centering
    \includegraphics[width=0.5\textwidth]{/Users/mmallek/Documents/Thesis/Plots/covcond-byscenario/2631-boxplots.png}
  } \\
  \subfloat[][]{
    \includegraphics[width=0.5\textwidth]{/Users/mmallek/Documents/Thesis/Plots/covcond-byscenario/2632-boxplots.png}
  }
    %\qquad
  %\subfloat[][]{
  %  \includegraphics[width=0.5\textwidth]{/Users/mmallek/Documents/Thesis/Plots/covcond-frvhrv/SMCX-frvhrv-boxplots.png}
  %}
    \caption{Boxplots illustrating the simulated future range of variability across each climate trajectory for Sierran Mixed Conifer - Xeric. The dashed black horizontal line represents the current condition. Boxplot whiskers extend from the $5^{\text{th}} - 95^{\text{th}}$ range of variability for each model. Climate models appear left-to-right in order of increasing median climate parameter value. See Section \ref{sec:evalFRV} for a detailed explanation of how to interpret departure based on these plots.}
  \label{fig:covcond_smcx}
\end{figure} %smcx



% debparture categories
%not departed - boxes completely overlap/contain each other
%slightly departed - medians don't overlap the boxes, but boxes overlap
%moderately departed - box overlaps whiskers
%highly departed - only whiskers overlap
%completely departed - no overlap of full rv

% include any others?
% what about total edge?

\paragraph*{Early Development Patch Configuration}
The current landscape is completely departed from the FRV across all tested climate scenarios (Figures~\ref{fig:fragclass-smcm} and \ref{fig:fragclass-smcx}). The exception to this observation is that the mean patch size and mean core area size results in the xeric mixed conifer forests overlapped the whiskers of the boxplot results in all scenarios except the \textsc{ccsm3} and \textsc{gfdl-esm2m}, indicating moderate departure. In no case did I find that the current landscape was within the range of variability of the simulated future scenarios. In both mesic and xeric mixed conifer forests, mean patch size, mean core area size, and mean shape index (area-weighted, in all cases), increased with increasing climate parameter values. The trend is stronger in xeric forests, with the biggest difference apparent in the results for the \textsc{gfdl-esm2m} scenario. In the case of both mesic and xeric variants, the level of fragmentation, as measured by the Clumpiness Index, is completely departed from the current landscape. However, no trend appears with respect to the climate parameter. Thus, patches of early successional seral mixed conifer forests were larger, contained more core area, featured more complex shapes, and were less fragmented than patches on the present-day landscape.


\begin{figure}[!htbp]
 \captionsetup[subfigure]{labelformat=empty}
  \centering
  \subfloat[][]{
    \centering
    \includegraphics[width=0.5\textwidth]{/Users/mmallek/Documents/Thesis/Plots/fragclass-smcmetrics/SMC_M_EARLY_ALL_AREA_AM_boxplots.png}
    \label{fig:boxplot-class-smcm-areaam}
  }%
  %\qquad
  \subfloat[][]{
    \includegraphics[width=0.5\textwidth]{/Users/mmallek/Documents/Thesis/Plots/fragclass-smcmetrics/SMC_M_EARLY_ALL_CLUMPY_boxplots.png}
    \label{fig:boxplot-class-smcm-contag}
  } \\
    \subfloat[][]{
    \includegraphics[width=0.5\textwidth]{/Users/mmallek/Documents/Thesis/Plots/fragclass-smcmetrics/SMC_M_EARLY_ALL_CORE_AM_boxplots.png}
    \label{fig:boxplot-class-smcm-coream}
  }
    %\qquad
    \subfloat[][]{
    \includegraphics[width=0.5\textwidth]{/Users/mmallek/Documents/Thesis/Plots/fragclass-smcmetrics/SMC_M_EARLY_ALL_SHAPE_AM_boxplots.png}
    \label{fig:boxplot-class-smcm-shapeam}
} %\\
  %  \subfloat[][]{
  %  \includegraphics[width=0.5\textwidth]{/Users/mmallek/Documents/Thesis/Plots/fragclass-smcmetrics/SMC_M_EARLY_ALL_ECON_AM_boxplots.png}
  %  \label{fig:boxplot-class-smcm-econam}
  %}
    %\qquad
  %  \subfloat[][]{
  %  \includegraphics[width=0.5\textwidth]{/Users/mmallek/Documents/Thesis/Plots/fragclass-smcmetrics/SMC_M_EARLY_ALL_AI_boxplots.png}
  %  \label{fig:boxplot-class-smcm-ai}
  %}
 \caption{Boxplots illustrating the simulated future range of variability in Sierran Mixed Conifer - Mesic, Early Development, across future climate trajectories. The dashed black bar represents the current condition. Boxplot whiskers extend from the $5^{\text{th}} - 95^{\text{th}}$ percentile range of variability for each model. See Section \ref{sec:evalFRV} for a detailed explanation of how to interpret departure based on these plots.}
  \label{fig:fragclass-smcm}
\end{figure} %fragland

\begin{figure}[!htbp]
 \captionsetup[subfigure]{labelformat=empty}
  \centering
  \subfloat[][]{
    \centering
    \includegraphics[width=0.5\textwidth]{/Users/mmallek/Documents/Thesis/Plots/fragclass-smcmetrics/SMC_X_EARLY_ALL_AREA_AM_boxplots.png}
    \label{fig:boxplot-class-smcx-areaam}
  }%
  %\qquad
  \subfloat[][]{
    \includegraphics[width=0.5\textwidth]{/Users/mmallek/Documents/Thesis/Plots/fragclass-smcmetrics/SMC_X_EARLY_ALL_CLUMPY_boxplots.png}
    \label{fig:boxplot-class-smcx-contag}
  } \\
    \subfloat[][]{
    \includegraphics[width=0.5\textwidth]{/Users/mmallek/Documents/Thesis/Plots/fragclass-smcmetrics/SMC_X_EARLY_ALL_CORE_AM_boxplots.png}
    \label{fig:boxplot-class-smcx-coream}
  }
    %\qquad
    \subfloat[][]{
    \includegraphics[width=0.5\textwidth]{/Users/mmallek/Documents/Thesis/Plots/fragclass-smcmetrics/SMC_X_EARLY_ALL_SHAPE_AM_boxplots.png}
    \label{fig:boxplot-class-smcx-shapeam}
} %\\
  %  \subfloat[][]{
  %  \includegraphics[width=0.5\textwidth]{/Users/mmallek/Documents/Thesis/Plots/fragclasssmcmetrics/SMC_X_EARLY_ALL_ECON_AM_boxplots.png}
  %  \label{fig:boxplot-class-smcx-econam}
  %}
    %\qquad
  %  \subfloat[][]{
  %  \includegraphics[width=0.5\textwidth]{/Users/mmallek/Documents/Thesis/Plots/fragclasssmcmetrics/SMC_X_EARLY_ALL_AI_boxplots.png}
  %  \label{fig:boxplot-class-smcx-ai}
  %}
    \caption{Boxplots illustrating the simulated future range of variability in Sierran Mixed Conifer - Xeric, Early Development, across future climate trajectories. The dashed black bar represents the current condition. Boxplot whiskers extend from the $5^{\text{th}} - 95^{\text{th}}$ percentile range of variability foreach model. See Section \ref{sec:evalFRV} for a detailed explanation of how to interpret departure based on these plots.}
  \label{fig:fragclass-smcx}
\end{figure} %fragland




\clearpage



%The degree to which seral stages or fragstats metrics are departed is covered in the narrative part of the results, but I didn't include a table (would be redundant, which I have been scolded for by everyone on the committee). I



%%%%%%%%%%%%%%%%%%%%%%%%%%%%%%%%%%%%%%%%%%%%%%%%%%%%%%%%%%%%%%%%%%%%%%%%%%%%%%%%%%%%%%%%%%%%%%
%%%%%%%%%%%%%%%%%%%%%%%%%%%%%%%%%%%%%%%%%%%%%%%%%%%%%%%%%%%%%%%%%%%%%%%%%%%%%%%%%%%%%%%%%%%%%%
%%%%%%%%%%%%%%%%%%%%%%%%%%%%%%%%%%%%%%%%%%%%%%%%%%%%%%%%%%%%%%%%%%%%%%%%%%%%%%%%%%%%%%%%%%%%%%
%%%%%%%%%%%%%%%%%%%%%%%%%%%%%%%%%%%%%%%%%%%%%%%%%%%%%%%%%%%%%%%%%%%%%%%%%%%%%%%%%%%%%%%%%%%%%%
%%%%%%%%%%%%%%%%%%%%%%%%%%%%%%%%%%%%%%%%%%%%%%%%%%%%%%%%%%%%%%%%%%%%%%%%%%%%%%%%%%%%%%%%%%%%%%
%%%%%%%%%%%%%%%%%%%%%%%%%%%%%%%%%%%%%%%%%%%%%%%%%%%%%%%%%%%%%%%%%%%%%%%%%%%%%%%%%%%%%%%%%%%%%%
%%%%%%%%%%%%%%%%%%%%%%%%%%%%%%%%%%%%%%%%%%%%%%%%%%%%%%%%%%%%%%%%%%%%%%%%%%%%%%%%%%%%%%%%%%%%%%


\section{Discussion}

\subsection{Future landscape dynamics and comparison to current conditions}

The overall effect of a warming and drying climate in the simulated future climate scenarios was to alter the wildfire regime in the study landscape, which had both direct and indirect effects on the forest, at scales ranging a few to hundreds of thousands of acres. Examining the seral stage distribution created by wildfire provides a link between the abstract concept of fire frequency and forest pattern and structure. As a complement to the seral stage distribution analysis, I analyzed the patch configuration of individual seral stages. The Early Development stage of both mesic and xeric mixed conifer forests stood out for its shift, over increasingly warm and dry climate trajectories, to the dominant seral stage in both cover types. Although I analyzed and reviewed the patch metrics for the Middle and Late Development stages, my focus in this chapter is on the Early Development stage alone.

Early successional habitats are not a major focus of forest ecology research, in part because they are often seen as a temporally short, intermediate phase \citep{Swanson2011}. Forest management activities frequently occur in the early successional habitats, such as reforestation after fires or after timber harvest \citep{Stephens2010}. They are a critical component of forest ecosystems, functioning as a major contributor to biodiversity and supporting a range of species' habitat needs \citep{Chang1995,Hutto2008,Swanson2011}. Recent trends of increasing wildfire extent and severity mean that managers face more decisions about when and how to manage post-fire early successional habitat \citep{Stephens2013,Dellasala2014}. I use the model results to describe the range of variability of the spatial configuration of early successional forests under a natural fire regime for the mixed conifer zone. My findings are relevant to restoration efforts using both prescribed fire and mechanical harvest techniques. 

Scenarios with high climate parameter values were associated with a slight to moderate increase in average area burned during the simulated future period. Although on its own a change this small might indicate that northern Sierra Nevada forests are resilient to climate change and have stable outcomes, the average burned area results subsume an important result that high mortality fire increased relative to low mortality fire, especially  in scenarios with high climate parameter values (Figure~\ref{fig:dareacomp}). The area burned at low mortality was generally stable, and the fluctuations did not appear likely to be ecologically significant. Consequently, the increase in high mortality fire drove the observed increase in total area burned. This finding was more pronounced for the xeric mixed conifer forests than for the mesic mixed conifer forests or the landscape as a whole, which was surprising because I expected the xeric forests to be resilient to an increase in high mortality fire because of the ubiquity of low mortality fire. I predicted that a climate-driven increase in fire in general would lead to more open conditions, which did occur, and that since high severity fire is so rare as a baseline for xeric mixed conifer forests (the probability of high severity fire for is 0.09 for Mid--Open and 0.05 for Mid--Closed), that even an increase in the likelihood of a fire being high severity because of the effect of higher climate parameters would not translate into a substantial increase in area burned at high mortality. Instead, I did observe such an increase, which was marked by little change in the area burned at low mortality. The ratio of low to high mortality fire shifted from about 2.6 for the \textsc{ccsm1} model, which included the lowest climate parameter values, to about 1.0 for the \textsc{gfdl-esm2m} model, which included the highest climate parameter values. My analysis of fire rotation confirms and further illustrates this (Figure~\ref{fig:frotation}), highlighting the slight decrease in low mortality fire observed in results for the climate parameter trajectories with larger values. 

This result foreshadowed dramatic changes in the seral stage distribution, especially for xeric forests. In reviewing the seral stage trajectories, I observed sharp jumps in the first few timesteps in most of the cover type-seral stage plots (Figures~\ref{fig:median_trajectory_smcm} and \ref{fig:median_trajectory_smcx}). These abrupt shifts are likely an artifact of the initial conditions, as they can be traced directly to the starting conditions. However, these are the actual vegetation conditions from 2010 information. Large fires have been recorded elsewhere in the Sierra Nevada in the past five years, and there is a trend in the last 30 years of increasing fire \citep{Miller2012,Lydersen2014}, suggesting that this model result is not far-fetched simply because it projects a lot of fire in the short-term. I also suggest that the results for the xeric mixed conifer forests can be tied specifically to the increase in high mortality fire observed in that cover type. These results thus imply a future with much more high mortality fire than has been the norm in recent decades, with particular consequences for seral stage distributions.

Scenarios with high climate parameter values yielded a high proportion of land in the Early Development seral stage. In addition, open canopies became more prevalent and closed canopies became less prevalent, which can also be attributed to an increase in fire. In the mesic mixed conifer forests, this decline was a clear trend following increasing sets of climate parameter values, and losses of Late--Closed forest were correlated with gains in Early Development conditions. This implies that the decline in the proportion of late successional conditions in mesic mixed conifer forests is likely due to the increased amount of fires that result in high mortality, which precludes many cells from succeeding to a Late Development stage without experiencing a high mortality fire. The fact that the current proportion of Late--Open present on the current landscape is completely departed from all of the simulated FRVs may be due to the fact that fire exclusion has already reduced the amount of Late--Open on the landscape. Across future climate scenarios, seral stages other than Early Development, Mid--Open, and Late--Open were virtually absent from the xeric mixed conifer forests. More fire, and especially more fire with high mortality effects, would logically produce such a seral stage distribution. Again, cells may have been so frequently affected by fire that succession to closed or even moderate canopy conditions became statistically unlikely.

With that said, because more land in the Early Development seral stage is projected for the future (even if fire suppression policies continue), examining patch characteristics in a range of variability framework can provide insights into how to manage it \citep{McGarigal2005,McGarigal2005a}. The increased extent and severity of fire during the simulated future periods affected configuration metrics at the seral stage level. I focused on the Early Development stage in mesic and xeric mixed conifer forests to examine dependence on the climate parameter and because the U.S. Forest Service is actively developing management practices for early successional habitats. 

The questions driving the configuration analysis center around the questions of how big, how complex, and how fragmented Early Development patches were, and how much core area they contained, during the simulated future period. In most cases the current condition is fully departed from the 90\% range of variability observed in the simulated future scenarios. Compared to the current landscape, Early Development patches during the simulated future period were larger, had larger core areas, were less fragmented, were more irregularly shaped, and had less edge contrast. This was true regardless of the specific climate trajectory. Early Development xeric mixed conifer forests exhibited a stronger trend than the Early Development mesic mixed conifer forests. The effect of the climate parameter was more subtle on the configuration metrics than on the seral stage distribution.

These results imply that current trends of increased amounts of high severity fire \citep{Miller2012,Mallek2013} are related to climate, and may be difficult to reverse. I observed a loss of structural diversity (compared to the current conditions) within the xeric mixed conifer forests, which transformed into a distribution composed almost entirely of Early Development, Mid--Open, and Late--Open forest. Mesic forests contained more structural complexity, but a large increase in Early Development comes at the loss of the Late--Closed and Late--Moderate stages, which may be problematic because this cover type is a major source of late successional, closed canopy forest for the study area.

\subsection{Influence of the Initial Conditions}
Because the initial conditions for both forest types included a large amount of land in the Late Development and Mid--Closed seral stages, which are generally parameterized to be more susceptible to fire than other seral stages, it seemed plausible that these initial conditions could have a disproportionate impact on the trajectory and the range of variability, as displayed in Figures~\ref{fig:covcond_smcm} and \ref{fig:covcond_smcx}. This could occur if the higher susceptibility resulted in high rates of wildfire in addition to high rates of high mortality from wildfire, leading to a large amount of land in the Early Development stage that would then dominate the cover type composition for the remainder of the simulation. However, Figures~\ref{fig:median_trajectory_smcm} and \ref{fig:median_trajectory_smcx}, which depict the median values across all runs of each climate model scenario, and for each seral stage of mesic and xeric mixed conifer forests, demonstrate that this is not the case. While the first few timesteps include a very sharp shift in the proportion of each seral stage on the landscape, stasis does not follow. A single narrative cannot capture the dynamics across all seral stages of the two focal cover types. The landscape is resilient to the punch of the first few timesteps, settling into a an oscillating trend by the end of the simulation across all the climate scenarios. In neither cover type does it appear that the shifting proportions of Early Development in the first few timesteps directly cause or predict the proportions of Early Development in the last five timesteps.

It is true that the proportion of Early Development increases in the first several timesteps. However, in mesic mixed conifer the proportion dips back down, and appears to be responding more to the climate parameter than the initial conditions, particularly by timestep 14, which is the point at which I begin using the output landscape grids for analysis. In xeric mixed conifer, the initial increase continues throughout the length of the simulation, again indicating that the driver behind this increase is related primarily to the climate parameter set and related feedback, rather than the initial conditions. In the same way, if the preceding hypothesis were true, I would expect to see a fairly stable distribution once the first few timesteps had passed. Instead, for most seral stages, including the Late--Closed stage, I observe a clear trend. This trend appears to reflect that of the climate parameter, implying that climate is more influential than the initial conditions. Furthermore, my analysis and conclusions are focused on the end of the simulation. I did this in part to avoid incorporating artifacts related to initial conditions. The plots in Figures~\ref{fig:median_trajectory_smcm} and \ref{fig:median_trajectory_smcx} show that the shift in seral stage proportions over time follow trends, rather than oscillate in equilibrium, and thus the short length of the simulation (compared to a multi-century HRV) is not a main factor in the decline of middle and late successional forest observed in the simulations.


%Because this study relied on the use of computer models, the most appropriate use of the results is to help identify the most influential factors driving landscape change, implications of our simulated disturbance and succession regime, and areas where further research is needed to delineate key parameters. 

%***From powerpoint
%Keeping in mind that they are the outcome of modeling passive management of wildfire and forests. Generally, I observed more fire and more high mortality outcomes from fire, which increases the proportion of early seral and open canopy forest. Other seral stages decline. The resulting early seral patches are larger and have more complex shapes.

%In particular, in the disturbance regime, I observed that: Wildfires were frequent in all simulations, but I still recorded large fires with high proportions of high severity. This makes sense given the known, and specified relationship between drought and wildfire mortality. Thus, frequent fire alone does not necessarily provide increased resistance to high severity fire under more frequent or intense drought in the future. 

%(Seral Stage Distribution) Wide range of future variability in each scenario. Indicates that stability and predictability not likely to be characteristics of future forest and wildfire patterns and that our ability to predict when and where large fires will occur will be constrained by large uncertainty. Need to manage public, agency expectations

%(Early seral patch Configuration) My results were consistent among scenarios, showing the current landscape departs from the RV. Indicates that allowing fires to burn naturally is bigger factor than climate





\subsection{Management Implications}

\subsubsection{Using results at various scales}
This study was designed to address questions on the landscape level, and therefore the simulations and analysis were conducted at that scale. When considering smaller-scale subregions of the study area, managers should carefully review results through a comparative or relative lens. In general, the landscape-level statistical results are inappropriate for use as the template for a project-level target. Instead, the landscape-level results should guide the development of project-level targets. In addition, the success of a project should be measured not by whether the results of the project mirror the results of this study, but by whether the project contributed to a particular landscape-level shift set as a goal using those landscape-level results. This is a subtle but important distinction in how to use the outputs of these simulations. 

To further illustrate, consider an area identified as being appropriate for fire restoration. The composition of that area cannot be assumed to match that of the overall landscape. Many variables will affect the composition, including environmental (topographic position) and social (wildland-urban interface) considerations. Furthermore, my results rely on the generation and analysis of a large quantity of data. Even in the absence of such variation, if the scale of analysis is reduced from 180,000 ha to a few thousand, the statistical validity of the results and their implications deteriorates. As discussed, the need for statistically viable results led me to use only results from cover types at least 1000 ha in extent, and focused my analysis on the two dominant cover types, which extend across an area many times that minimum.

With respect to planning, my results are best used to provide a broad-scale context to smaller-scaled projects. As an example, consider the seral stage distribution of xeric mixed conifer forests. Across the full study area, about 27\% to 39\% of xeric mixed conifer forest is in the Late--Open stage. Based on this, the Forest may set this range as a management target. If the current proportion on the forest is 15\%, an individual project may seek to maintain or create additional acres in that seral stage. However, these results cannot tell managers how much of the cover type in that stage should exist within that project boundary, which could also be 0\% or 100\%. The key is for managers to place individual projects in a landscape context, and understand the contributions different parts of the forest have on the overall landscape pattern (in this example, the overall seral stage distribution).

\textsc{RMLands} has been optimized to perform well statistically when used to examine landscape change over large extents and long time periods. While spatially explicit in its dynamics, it is a simulation, producing many potential outcomes, and is not intended to predict specific outcomes at specific places and points in time. Rather, the data produced are aggregated in order to develop an understanding of the area as a whole. Because this model should not be used to predict an actual fire rotation value for a particular point on the landscape, it should also not be used to predict the effect of a particular vegetation management strategy \emph{at a particular point on the landscape.} Other forest and fire simulation models are designed for this purpose. \textsc{RMLands} is designed to produce outputs that facilitate measuring the effect of vegetation management implemented across space and time, which in turn enables assessing and understanding the potential impacts of such actions at scales larger than is practical or possible to measure experimentally.  


\subsubsection{Implications for Restoration}

Restoration toward a resilient, somewhat stable ecology is often an important goal of resource managers. The comparisons made here consider the difference between the current and future disturbance regimes and landscape patterns, given a scenario in which natural fire regimes were allowed to occur. Since letting all fires burn naturally is not a realistic option, I note that the results do not provide a simple roadmap for restoration. However, they provide some insights into what landscape patterns may be resilient with climate change. In addition, the results indicate whether restoration toward the current wildfire disturbance regime, seral stage composition, or patch configuration appears likely to succeed under climate change, or whether the future is likely to significantly diverge from the present. 

One clear result of this study is that more fire occurs under natural conditions and future climate scenarios than under current and presumed historical conditions. The implication is therefore that a shift towards more fire and especially more high severity fire is likely. Under a suppression-based fire management strategy, there is no indication that future fires would be less severe than those modeled here. I did not observe a plateau in landscape change by 2100, indicating that a stable ecology in terms of fire is not likely in the near-term. If restoration processes take many decades, stability should not be expected to return until restoration is completed, at the earliest. Ongoing trends in climate change will inhibit efforts to oppose analogous trends in forest disturbances. The wide observed range of variability further strengthens my conclusion that managers will have difficulty in maintaining a stable and predictable relationship between fire and the landscape. Large uncertainties in projections will be the norm for some time under a changing climate. 

In addition, given that both the frequency and extent of high mortality fire under increasingly severe climate conditions, I conclude that restoring frequent fire alone should not be assumed to eliminate the risk of high severity fire under more frequent or intense drought in the future (although it may reduce it). Moreover, my results were consistent among scenarios, indicating that allowing fires to burn naturally may be at least as large, if not larger, a factor than climate change on its own. These results point to a need for restoration of more fires that extend across much larger areas, many of which include patches of high severity fire that reset part of the burned area to early successional conditions. Such restoration is unlikely to be politically or socially popular \citep{Stephens2010,Stephens2013}. As a result, public expectations and the capabilities of agencies will need to be carefully managed and articulated \citep{Keeley2000}. 

Given my findings that seral stage distributions and patch configuration metrics are changing in response to changing climate, it is important to recognize that species will be affected differently based on their particular habitat requirements. Thus, using the future range of variability results as a restoration target has implications for wildlife species that may not be socially or legally acceptable. Species that rely on early successional conditions or open canopies will gain habitat in the future, due to the increase in the prevalence of these conditions. However, the associated reduction in late successional conditions, especially the more closed canopies currently characteristic of mesic mixed conifer forests, will likely have negative ramifications for species dependent on that structural context, such as spotted owl and fisher \citep{SNEP1996b}. Consequently, as managers and decision-makers consider which areas to restore and to what conditions, they will encounter real, unavoidable tradeoffs.

Opportunities to restore forests to patch configurations observed in the simulation may arise when designing timber harvest or fuels treatment projects. However, I caution that patch configurations under natural regimes may be challenging for managers to re-create, especially given constraints surrounding the use of fire. For example, it is common to use linear features such as roads and streams as boundaries for treatment.\footnote{I examined fuels treatment actions using fire and all forest management activities entered in the Forest ACtivity Tracking System (FACTS) geospatial database from 1995--2012. About 72\% of all activities, and 80\% of fuels treatment activities that used fire took place within 50 m of a road. This percentage increases to 91\% and 100\%, respectively, when the buffer is set at 250 m from a road.} However, such strategies may inhibit managers' ability to achieve patch configurations similar to those that would occur naturally.
%3248/4489, 4090/4489 - all forest activities
%432/538 - fire activities

As just mentioned, my results can be used to identify and prioritize management strategies. Since the U.S. Forest Service is directed to manage within the natural range of variability, it is important to define what the natural range of variability is, as well as the desired condition, and recognize the extent to which the two overlap. Some of the trends observed in this analysis, such as increased proportions of Early Development, will likely occur without active management, assuming existing trends of increasing fire severity continue. Others, such as shorter fire rotations or increased patch complexity, will not occur without changes to fire suppression strategies and related fuels management efforts. Once desired conditions are defined, a formal comparison of those conditions to those described as the set of potential future ranges of variability from this study is warranted. A value judgement should not be assigned to whether or not restoration achieves a seral stage distribution compatible with the simulated FRV. Rather, the restoration of the natural processes (occurrence and effects) of wildfire should be a high-level goal. Future seral stage distributions are one of several metrics that can be used to evaluate whether efforts to restore wildfire are successful. Whether and how the desired future conditions align with the FRV will affect what kinds of management actions are taken. Based soley on my results, a proactive approach to restoration would steer xeric forests toward more open conditions now, which I argue would benefit both Tahoe National Forest ecosystems and the public.

One limitation of the model is that it did not predict a longer burning season, which is an anticipated byproduct of climate change \citep{Westerling2008,Stephens2013}. It may be possible to counteract some of the trend toward increasing fire severity by applying more low severity prescribed burning during the shoulder seasons and winter. This might dampen the effects of warmer and drier conditions \citep{Conard2003}. Variabile density treatments designed to create fine-scale heterogeneity could also ameliorate some of the effects of more frequent fire \citep{Stephens2010,Knapp2012,North2012a}. The vulnerability of closed canopy forest should be an impetus to focus initial restoration and prescribed burning efforts in areas that will enhance managers' ability to protect those values. Of course, increased amounts of fire will pose challenges in a landscape with complex ownership patterns \citep{Stephens2013}. Excellent coordination at levels beyond the Tahoe National Forest or the U.S. Forest Service will be necessary to address this successfully. Fortunately, federal land management agencies, the California Department of Forestry and Fire Protection, and local governments are working together more and more to coordinate fire management across political boundaries \citep{InteragencyAgreement2013}.

Assessments of the effects of post-fire treatments of various kinds need to be continued and expanded in a systematic and rigorous way. Stand initiation patterns of the past, and the meteorological condition under which they took place, should be evaluated to understand whether similar conditions are likely in the future. Some researchers have called for special management of early successional habitat. For example, \citet{Dellasala2014} call for not managing post-disturbance early successional vegetation. \citet{Swanson2011} suggest mapping and managing early successional communities as a unique cover type, implying that they should be maintained at pre-identified locations. However, given the high probability of more early successional habitat being created by fire and the realities of federal budgets and personnel needs, it seems unlikely that special attention will be necessary to ensure that some newly created patches of early successional habitat are allowed to succeed naturally. In addition, mapping and managing a transient cover type does not fit within most forest management frameworks, including ours; I have asserted that early successional forest is a seral stage within a cover type. Practicing no management at all under climate change is risky, given the very real threat of invasive species, and the fact that current species assemblages may need help to re-establish \citep{Stephens2010}. That said, the large amount of early successional habitat projected to occur on the landscape also means that managers will have options when deciding where to implement restoration efforts that are designed to speed up succession or reduce susceptibility to subsequent burns; it is here that more research on the effects of Burned Area Emergency Response and restoration treatments would be most useful. Additionally, ongoing research on specific large fires, such as the 2013 Rim Fire, should yield insights into how shifts in the fire regime may change spread and susceptibility patterns, which could in turn be used to update and improve the \textsc{RMLands} parameterization \citep{Lydersen2014}.




\subsubsection{Implications for Planning}
Many National Forests will undergo Forest Plan revisions in the next decade. The 2012 Planning Rule instructs managers to manage for resilient conditions within a natural range of variability. While my results may be used as one potential range of variability, they are also the outcome of a model that simplified several aspects of the landscape, including ownership patterns and tolerable amounts of fire. For example, I did not model varying levels of fire suppression effort. Assuming the U.S. Forest Service and other stakeholders in the area do not view the amount of wildfire I simulated as tolerable, they will need to adopt various strategies to try and reduce the likelihood of frequent large fires, as well as the negative impacts of suppression. 

Overall, the results of my study strongly indicate that more time, personnel, and funding will be needed in the future to fight fire, and subsequently repair and restore the damage from firefighting efforts and the flames themselves. I assert that there is a critical need to define desired future conditions for forests, and compare them to what would naturally develop under the simulated conditions, which are intended to portray the future range of variability for the landscape under study. I believe it is likely that there will be a gap between the projected conditions and the desired future conditions. A ``no-management'' strategy of fire and forests is not compatible with human settlement in the region: even before European settlement, fires were used by native peoples to manage vegetation and fire behavior \citep{Anderson1996}. However, an analysis comparing my results to desired future conditions would provide information on the feasibility of restoration efforts, and provide one anchor for conversations about the future of forest and fire management in the area.



%permanent firebreaks
Managers and planners should also consider the values at risk from fire or fire exclusion and prepare to mitigate them. This need extends to ecological values, such as old growth, closed canopy forest, and social values, such as infrastructure and development. Mitigation of this risk probably requires some form of active management. I outlined some methods for restoring fire above. In addition, stakeholders could choose to develop fire breaks in advance of wildfires, for example by clearing vegetation along roads \citep{Conard2003}. However, imposing such ``unnatural'' features has other ecological tradeoffs, especially from increased fragmentation and reduced patch shape complexity \citep{Trombulak2000}. The effect of such a strategy could be evaluated using \textsc{RMLands}, either by increasing the ability of roads to function as barriers to fire spread, or by creating a new cover type for firebreaks and assigning it low susceptibility values. In order to completely evaluate such a strategy it would also be necessary to rerun the simulation with the currently maintained large roads parameterized as barriers to smaller fires.

%mimic old-growth
If more frequent fire does reduce the quantity of old growth forests in the area, mechanical techniques in combination with prescribed fire could be used to mimic the structural complexity of old growth in much younger stands \citep{Franklin2002}. Because it is so productive, this portion of the Tahoe National Forest may be a good place to experiment with this strategy; trees grow large more quickly here than in other locations in the mixed conifer belt \citep{PRISMClimateGroup2004,Littell2012}.

%type conversions
The risks of type conversions are also pertinent to managers, and potentially one of the biggest risks of an increase in high severity fire as a result of climate change \citep{Stephens2013,Mallek2014}. While not explicitly explored in this study, it is predicted that cover type shifts and conversions are more likely to follow stand-replacing disturbances \citep{Stephens2013}. This risk will increase with climate change, and the additional increase in stand-replacing events suggested by the model indicates an interaction between climate change and high mortality fire that should be taken into account by managers planning restoration after fires, especially when selecting what species to plant or encourage \citep{Fule2008,Schwartz2015}. 

% range shifts
Active management of forest resources may also be needed to address climate-induced range shifts of cover types \citep{Keane2009}. Such shifts are already happening, and high severity fires, since they reset the vegetation to early successional conditions, provide additional opportunities for these shifts to occur by facilitating the progression of vegetation through alternate stages of succession \citep{Bachelet2001}. Climate change may lead simply to upward or northerly movement of species ranges’, but could also lead to the establishment of nonnative, invasive species \citep{McKenzie2004}. Managers can choose to try to influence vegetation communities through replanting prior vegetation, engaging in assisted migration, or controlling undesirable species. 
% last word
Unfortunately, there are no obvious or easy answers here. However, results from this study can be used as basis for discussion between agencies, scientists, and the public, both to understand the potential future of wildfire and its effects in the northern Sierra Nevada, and to anchor descriptions of desired future conditions and desired management strategies for use in future forest planning and management.






