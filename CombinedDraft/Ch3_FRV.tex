% !TEX root = master.tex

\chapter{Future Range of Variability}
\label{ch:FRV}

\section{Abstract}
In the Sierra Nevada, cycles of fire and vegetation recovery occur variably over large extents, as well as over long periods of time. The U.S. Forest Service's 2012 Planning Rule explicitly calls for the agency to estimate and describe the range of variability under natural disturbance regimes, and manage for those characteristics. Recent warming and drying trends have already influenced a more frequent and proportionally more severe fire regime in the Sierra Nevada. These trends are anticipated to continue under warmer and drier climate change scenarios. I used \textsc{RMLands}, a spatially-explicit, stochastic, landscape-level disturbance and succession model capable of simulating fine-grained processes over large spatial and long temporal extents, to evaluate trends in landscape composition and configuration under a range of potential future climate scenarios. My results show increasing burned area and increasing high severity fire with increasing climate values. I also found that today's landscape is outside the future range of variability, and that this departure increases with increasingly warm and dry climate conditions. Based on these findings, I suggest more aggressive restoration efforts and implementation of mitigation measures where the consequences of changing fire regimes are socially unacceptable.



%%%%%%%%%%%%%%%%%%%%%%%%%%%%%%%%%%%%%%%%%%%%%%%%%%%%%%%%%%%%%%%%%%%%%%%%%%%%%%%%%%%%%%%%%%%%%%
%%%%%%%%%%%%%%%%%%%%%%%%%%%%%%%%%%%%%%%%%%%%%%%%%%%%%%%%%%%%%%%%%%%%%%%%%%%%%%%%%%%%%%%%%%%%%%
%%%%%%%%%%%%%%%%%%%%%%%%%%%%%%%%%%%%%%%%%%%%%%%%%%%%%%%%%%%%%%%%%%%%%%%%%%%%%%%%%%%%%%%%%%%%%%
\section{Introduction}
\todo{intro is a bit too fluid. need to clearly define sections and make sure sections only contain what's in the bullets. may be best to outline the intro and then do some revising.}
\subsection{Fire in the Sierra Nevada}
In the Sierra Nevada, cycles of fire and vegetation recovery occur variably over large extents, as well as over long periods of time. Ongoing disturbance results in increased heterogeneity captured by various metrics used to describe vegetation composition and configuration \citep{Monica2008}. Prior to European settlement, wildfire was the major source of disturbance in Sierran forests, shaping the composition and configuration of vegetation communities \citep{SNEP1996a}. Fires were primarily lightning-caused, although indigenous peoples are thought to have set fires for vegetation management, especially in the lower elevations \citep{Anderson1996}. In general, fire was frequent, with a mean rotation as short as 20 years in Ponderosa pine (\emph{Pinus ponderosa})-dominated forests. Fire rotations generally increase with increased moisture and elevation. Variance around a mean fire rotation can be remarkable, as some parts of the forest experience fire much more frequently, while others escape fire for long periods \citep{Mallek2013}. Under this disturbance regime, high severity fire was uncommon, but was necessary to initiate early development conditions on the landscape. The predominant effect of most fires was of low mortality among overstory trees, which tended to either have no significant effect on the overstory or to open forest canopies, the latter effect being more prevalent in more xeric parts of the forest \citep{Skinner1996,Safford2014,SNEP1996,SNEP1996a}.

Since then, fire suppression, logging, grazing, and mining have all interacted to alter the historical fire regime and vegetation patterns \citep{Stephens2015,Knapp2013}. After large-scale fire suppression became the norm in the second half of the 19th century, less fire-tolerant species (such as Douglas fir (\emph{Pseudotsuga menziesii}) and white fir (\emph{Abies concolor})) have come to dominate areas where they were once a minor part of the vegetation community. Grazing and development made fires less common by altering or removing the fine fuels that carried fire. Timber harvest, especially of fire-tolerant species such as ponderosa and sugar pines, accelerated the increased cover of species such as white fir. Finally, fire suppression allowed the buildup of medium size fuels and ladder fuels, which promotes larger more severe fires when they do occur. Moreover, the lack of natural fires has meant that variation in fuel loading has decreased, which allows large fires to spread over very large areas \citep{Hessburg2005,Beaty2007,Meyer2008}.



\subsection{Forest Planning}
With the emergence of ecosystem management in the early 1990s, the need to recognize ecosystems as dynamic and constantly-changing became well accepted, and calls to manage forests sustainably became common \citep{Christensen1996}. Within the context of forest and land management planning, the restoration of ecosystems to their pre-European settlement states was incorporated as a goal or desired future condition into various plans, including the Sierra Nevada Ecosystem Project \cite{SNEP1996a}. The U.S. Forest Service's 2000 Planning Rule explicitly called for the agency to estimate and describe the range of variability under natural disturbance regimes, and manage for those characteristics (36 CFR \textsection 219 2000). The need to consider the natural range of variability was maintained through various amendments to the rule, and is still present in the new 2012 rule, finalized in early 2015 (36 CFR \textsection 219 2012).



\subsection{Range of Variability Analysis}
Historic range of variability (HRV) analysis is a useful tool in landscape planning. HRV analysis is intended to help conceptualize the mechanisms behind large-scale ecosystem functions and provide a basis from which to make predictions about how a given ecosystem will react to disturbances in the future \citep{Nonaka2005,Landres1999}. Methods for quantifying the natural range of variability for a diversity of landscapes in the United States augmented the development of research focused on this task \citep{Landres1999}. Of these, simulation of the historical dynamics became fairly popular. By 2004, some 45 landscape fire and succession models alone had been developed \citep{Keane2004}. Many of these, such as \textsc{landis} \citep{He1999}, \textsc{zelig-l} \citep{Miller1999}, \textsc{safe-forests} \cite{Sessions1997} and \textsc{landsum} \citep{Keane2012} are still in use today. Landscape fire and succession models are used to create spatially-explicit simulations of both of these key forest processes, typically outputting a set of GIS layers for each timestep of the model that can then be analyzed to quantify trajectories and patterns in the disturbance regime, seral stage composition, and landscape configuration over time \citep{Keane2004}. A component of many landscape fire and succession models are state and transition models, which are as much as anything a framework for defining the fundamental vegetation communities and the probability over time of completing a transition from one state to another \citep{Stringham2003,Blankenship2015}.

%Many range of variability analyses in the United States focus on the historical range of variability (HRV) of an area. The Rocky Mountains and Oregon Coast Range in particular have been the focus of several HRV studies, while only one has been conducted in the Sierra Nevada \citep{Miller1999}, which took place in Sequoia National Park in the southern Sierra.

\subsection{Climate and range of variability}

While HRV studies can play an important role in informing natural range of variability, the need to explore and understand the ramifications of climate change on the disturbance regime and forest structure is also critical. Concern about the impact of changes to precipitation and temperature anticipated under climate change in the northern Sierra on local disturbance regimes, and subsequently, seral stage distribution and patch configuration has motivated analysis that consider not only at the current and historical conditions, but also future conditions\citep{Fule2008,North2012}.
%
Recent warming and drying trends have already influenced a more frequent and proportionally more severe fire regime in western forests in general and the Sierra Nevada in particular \citep{McKenzie2004,Westerling2011,Miller2012}. These trends are anticipated to continue under warmer and drier climate change scenarios \citep{Westerling2008,Dale2001}. Changes have also been reported in the elevation of fires in the Sierra Nevada, increasing the potential for range shifts upslope \citep{Schwartz2015}. Where the focus of management efforts had been restoration in the past, now adaptation to ensure resilient ecosystems is the primary objective of managers \citep{Stephens2010}.

Range of variability analyses that offer a complementary analysis of future scenarios under climate change are rare (but see \cite{Keane2008} and \cite{Duveneck2014}). By simulating a range of potential future climate scenarios, we can evaluate trends in the wildfire regime, landscape composition, and landscape configuration that are related to warming and drying trends projected under climate change, and place the current landscape in that context. Moreover, an examination of landscape composition and configuration under potential climate change scenarios is important because it can provide additional information about what restoration strategies are likely to remain resilient and make sense ecologically for the area under study \citep{Duncan2010}.

Early successional habitats are not a major focus of forest ecology research, in part because they are seen as an intermediate phase that is ideally short \citep{Swanson2011}. However, they are a critical component of all systems, providing the most biodiversity of all seral stages and supporting a range of species' habitat needs \citep{Chang1995,Hutto2008,Swanson2011}. Recent trends in wildfire extent and severity mean that managers are faced with decisions about when and how to manage post-fire early successional habitat. In addition, the Sierra Nevada Forest Plan identifies at least seven management indicator species that require openings and early seral habitat \citep{USDAForestService2004}.\footnote{Calliope hummingbird, Townsend's warbler, Wild turkey, dusky-footed woodrat, elk, mule deer, western skink are listed for openings and early seral habitat. Other species are listed separately as depending on meadows, chaparral, and grasslands.} Our model results will provide insight into the spatial configuration of early successional forests under a natural fire regime for the intensively used mixed conifer zone. These results may also be helpful in planning restoration efforts using both prescribed fire and mechanical harvest techniques.
%%%

\textsc{RMLands} has been used previously to assess the HRV on the San Juan National Forest and the Uncompahgre Plateau in Colorado \citep{McGarigal2005,McGarigal2005a,Romme2009}, as well as the Lolo National Forest in Montana \citep{Cushman2011}. Following the Montana study, which adapted \textsc{RMLands} to use data from the LandFire project (\burl{http://www.landfire.gov}), we further adapted the software for use in the Sierra Nevada in order to prepare an HRV analysis for part of the Tahoe National Forest in California. In this paper we quantify and describe a ``Future Range of Variability'' (FRV) that can inform restoration and planning under a changing climate \citep{Duncan2010}.


\subsection{Objectives}
In this study, our objectives were to evaluate the effect of climate change on the wildfire regime and landscape composition and configuration for the Yuba River waterhsed on the Tahoe National Forest. To do this, we simulated forest fires and succession using \textsc{RMLands}, holding all model parameters except the climate parameter constant. The climate parameter incorporated Palmer Drought Severity Index (PDSI) values from a suite of seven climate trajectories developed by the National Center for Atmospheric Research (USA) and the Canadian Centre for Climate Modelling and Analysis to the year 2090 \citep{Cook2014}. We used \textsc{Fragstats} software and R to analyze outputs and report the 90\% range of variability for simulated future metrics. Ultimately, we evaluate our results for a series of simulations for these future scenarios in comparison to the current conditions, and assess implications for restoration and forest planning.

















%%%%%%%%%%%%%%%%%%%%%%%%%%%%%%%%%%%%%%%%%%%%%%%%%%%%%%%%%%%%%%%%%%%%%%%%%%%%%%%%%%%%%%%%%%%%%%%%%%%%%%%%%%%%%%%%%%%%%%%%%%%%%%%%%%%%%%%%%%%%%%%%%%%%%%%%%%%%%%%%%%%%%%%%%%%%%%%%%%%%%%%%%%%%%%%%%%%%%%%%%%%%%%%%%%%%%%%%%%%%%%%%%%%%%%%%%%%%%%%%%%%%%%%%%%%%%%%%%%%%%%%%%%%%%%%%%%%%%%%%%%%%%%%%%%%%%%%%%%%%%%%%%%%%%%%%%%%%%%%%%%%%%%%%%%%%%%%%%%%%%%%%%%%%%%%%%%%%%%%%%%%%%%%%%%%%%%%%%%%%%%%%%%%%%%%%%%%%%%%%%%%%%%%%%%%%%%%%%%%%%%%%%%%%%%%%%%%%%%%%%%%%%%%%%%%%%%%%%%%%%%%%%%%%%%%%%%%%%%%%%%%%%%%%%%%%%%%%%%%%%%%%%%%%%%%%%%%%%%%%%%%%%%%%%%%%%%%%%%%%%%%%%%%%%%%%%%%%%%%%%%%%%%%%%%%%%%%%%%%%%%%%%%%%%%%%%%%%%%%%%%%%%%%%%%%%%%%%%%%%%%%%%%%%%%%%%%%%%%%%%%%%%%%%%%%%%%%%%%

\section{Methods}

\subsection*{Study area}
The study area (see Figure~\ref{projectarea-ch3}) is located on the northern part of the Tahoe National Forest, on the Yuba River and Sierraville Ranger Districts, and comprises about 181,550 hectares. The topography of the study area consists of rugged mountains incised by two major and a few minor river drainages. Elevation ranges from about 350 m to 2500 m. The area receives 30 cm to 260 cm of precipitation annually, most of which falls as snow in the middle to upper elevations \citep{Storer1963}. Like the rest of the Sierra, the study area has a Mediterranean climate, in which summer drought typically persists from May to September. This increases the importance of developing a significant snowpack during the winter months, since snowmelt runoff is a key source of soil moisture during the late spring and summer months \citep{Minnich2007,Skinner1996}. Datasets of the 30-year normal precipitation at 800 m resolution for the northern Sierra (obtained from the Oregon State PRISM for a side project), illustrate that particularly high amounts of precipitation falls across the middle elevations of the study area compared to the larger region \citep{PRISMClimateGroup2004}. This increased moisture contributes to the occurrence of exceptionally productive patches of forest \citep{Littell2012}. Vegetation is tremendously diverse and changes slowly along an elevational gradient and in response to local changes in drainage, aspect, and soil structure. Grasslands, chaparral, oak woodlands, mixed conifer forests, and subalpine forests are all found within the study area.

% brad said to make study area more obvious for non-US readers; will probably have to redo plot for publication but this is ok for now I think
\begin{figure}
\centering
\includegraphics[width=.8\textwidth]{/Users/mmallek/Tahoe/Report3/images/studyarea.png}
\caption{The Sierra Nevada Ecoregion is outlined in brown. The study area (outlined in green) is located in the northern extent of the Sierra Nevada on the Tahoe National Forest, comprising the Yuba River watershed.}
\label{projectarea-ch3}
\end{figure}

Xeric and mesic Sierran mixed conifer forests are the two most prevalent cover types within the study area, together comprising 63\% of the landscape. These forests are characterized by five conifers and one hardwood: \emph{Abies concolor, Pseudotsuga menziesii, Pinus ponderosa, Pinus lambertiana, Calocedrus decurrens}, and \emph{Quercus kelloggii}. At least three conifers are typically present in any given stand. \emph{A.~concolor} tends to be the most ubiquitous species, especially on north-facing slopes. \emph{Pinus ponderosa} was historically the dominant species, under the previous frequent low severity fire regime. It is still the most prevalent species on south-facing slopes and is present continuously from the Oak-Conifer Forest and Woodland zone below it in elevation (Appendix~\ref{smc-description}).

In my model, early seral conditions within both mesic and xeric mixed conifer forests are always created by high mortality fire, and either chaparral or trees may after a cell experiences fire. Chaparral, a community that includes \emph{Arctostaphylos, Ceanothus}, and \emph{Chrysolepis} species, is often treated as its own cover type. This is in part due to the fact that chaparral establishment tends to inhibit the establishment and growth of conifer species, thus delaying succession process to (mid-development) forest. When studies occur along short temporal scales, it is often more meaningful to categorize these two vegetation communities as separate. However, my study is focused on long temporal scales, and within the study area chaparral will eventually succeed to trees. Consequently, both communities are considered early development.\todo{is this enough info about Early?} 

\subsection{RMLands}

\textsc{RMLands} is a spatially-explicit, stochastic, landscape-level disturbance and succession model capable of simulating fine-grained processes over large spatial and long temporal extents \citep{McGarigal2005}. It is grid-based and simulates fire on landscapes in a spatially explicit and realistic manner, in that fire perimeters resemble those that occur naturally. State transitions are simulated at the 30 m pixel scale. Transitions may take place in response to fire or in the absence of it (natural succession) \citep{McGarigal2012}. Outputs from the model are readable by the landscape pattern analysis software \textsc{Fragstats} \citep{Fragstats2012}, which facilitates the landscape configuration analysis.

In \textsc{RMLands}, fires spread probabilistically based on the susceptibility of an individual cell. It does not contain a mechanistic fire model and fuels are not directly incorporated into fire spread. In addition, we do not classify individual fires as a whole to a ``low,'' ``mixed,'' or ``high'' severity status. Some fire ecologists combine fire attributes such as flame length and fire size into their interpretation of the relative ``severity'' of a particular fire \citep{Agee1993}.   Ecologists working at other scales and not working with models often describe ``mixed severity'' regimes \citep[e.g.,][]{Kane2013}, which \citet{Collins2010} define as ``stand-replacing patches within a matrix of low to moderate fire-induced effects.'' Because at the 30 m cell size of our model, nearly all fires would be classified as ``mixed severity'' by the prior definition, it becomes moot. Instead, we evaluate and classify fire by its effects on individual cells. First, we evaluate whether a cell burned. Next, all burned cells are evaluated probabilistically and assigned either a high severity (``high mortality'') outcome or low mortality outcome. If a cell burns at high severity, then it transitions to the Early seral stage. Recently, some researchers have differed on whether 75\% or 95\% overstory tree mortality is a more appropriate cutoff point for defining a ``stand-replacing'' event \citep{Fule2014,Mallek2013}. In this paper, we use 75\% as our cutoff, which is widely accepted in the literature \citep{Miller2009,Baker2014,Agee1993,Agee2007}.


In collaboration with USDA Forest Service staff, we developed a system of land cover and seral stage classification based on LandFire (\burl{http://www.landfire.org}) and \citet{VandeWater2011}'s Presettlement Fire Regimes, and crosswalked Forest Service corporate spatial data based on the Northern Sierra \textsc{calveg} classification to the 31 cover types. We also used Forest Service corporate spatial data to develop layers for seral stage and age, plus physical environment layers (e.g., elevation). State and transition models for 31 land cover types were developed based on the VDDT models associated with the LandFire project, and refined with input from local experts to capture subtle changes in succession and transition at the project scale. In general model parameters were developed using meta-analyses published in the literature. For example, LandFire data was used to calculate transition probabilities, several fire rotation calibration parameters were taken from \citet{Mallek2013}, and wind direction information was obtained from several area weather stations. We used landscape conditions as of 2010 as the starting point for all simulations and as the ``current'' conditions for comparison with the HRV results.

Although \textsc{RMLands} is a process-based model with parameters sourced from the literature, our team had greater confidence in some parameters than others, especially as to how they function within the \textsc{RMLands} framework. Consequently, we calibrated our model parameters by iteratively adjusting certain low-confidence parameters to optimize the output values for parameters known with high confidence. Specifically, we manipulated an ignition calibration coefficient and a fire return index (similar in concept to a fire return interval), and measured calibration success based on conformity to pre-specified rotation values at the cover type level. Fire return index values were multiplied by a single factor across all seral stages of a given cover type; cover types were modified as groups but the index ratios within them were held constant.

We set our calibration target as rotation values for the nine most prevalent cover types within 10\% of their original target rotations (Sierran Mixed Conifer - Mesic, Xeric, and Ultramafic variants; Red Fir - Mesic and Xeric variants; Oak-Conifer Forest and Woodland - standard and Ultramafic variants; Mixed Evergreen - Mesic and Xeric variants). We focused on these nine types because they all extend across more than 1,000 ha, and are thus statistically stable from simulation to simulation. Target values were based on published empirical values and refined with input from local experts. We chose rotation as the calibration target because targets were available from the literature and because fire rotation is a fundamental measurement that \textsc{RMLands} was designed to capture. In addition, using rotation ties calibration to a parameter that is relatable to Forest Service staff and that can be used as a target by managers in various programs.

In a historical range of variability analysis, the model is typically run for a very long time to capture full disturbance cycles and the consequent effects on vegetation. This is possible when the climate parameter oscillates around a mean, but not when a clear trend exists, such as in the case of climate change. Therefore instead of running the model for a long time, we ran it 100 times for each climate parameter sequence. No true model equilibration was done for the future simulations, but we elected to include only the final 5 timesteps (25 years) of the results in order to achieve some distance from the starting (current) conditions, maintain focus on the ending trajectory of climate and the range of variation possible in the landscape at the end of the 21st century, yet still capture the variability inherent to the PDSI-based climate parameter (Figure \ref{fig:pdsi-final5}).



%%%% Brad's comment: Is it possible to validate this model in some way? In particular, how do we know that varying the climate multiplier affects fires in a realistic way? Could you hindcast to see how well the PDSI for past years matches the historical fire pattern?

%%%% Notes in response: not easily. some language in hrv that addresses this. we don't know, beyond my assumption that the model treats it realistically in the first place, which is based on Kevin's word. You could hindcast, maybe. But I'm not expecting a super strong relationship. Don't know exactly how to deal with this comment but it feels important.
%%%% More notes: HRV article has a brief comparison of climate vs. disturbed area. Weak relationship. We expect cascading results affecting landscape structure and composition as well as changes to fires. Some non-standard fire behavior may occur but no real way to test this because we don't know what fires will "look" like in the future.

\subsection{Climate Parameter and PDSI}

The range of potential future climate scenarios used to parameterize the model in this study come from models initialized using the set of parameters for Representative Concentration Pathway (RCP) 8.5. RCP8.5 includes no specific climate mitigation target, unlike the other three RCP scenarios in use \citep{Riahi2011}. As a result, it is considered a reference, or baseline scenario, in which greenhouse gas emission and concentrations increase over time without leveling out \citep{Riahi2011}. A literature review during the RCP development process designated radiative forcing in 2100 of 8.5 W/m$^2$ as the high end of plausible futures that had been modeled \citep{VanVuuren2011}. The corresponding concentration of >\textasciitilde 1370 $\text{CO}_2$ -eq in 2100, compared to 375 $\text{CO}_2$ -eq in 2005. The 66\% range of temperature increase above pre-industrial levels under the RCP8.5 scenario is 4.0\textdegree -- 6.1\textdegree \citep{Rogelj2012}. Since the development of RCP8.5 as a scenario, storylines have been developed that describe how such a scenario could come about. In the case of RCP8.5, human populations continue to increase, rising to 12 billion by 2100. Little progress in energy efficiency and the food demands of the increasing population lead to high energy demands, which are met by coal-intensive technology choices \citep{Riahi2011}. Data from the RCP8.5 scenario was used to model climate variables like temperature and precipitation, which \citet{Cook2014} used to predict trajectories of drought severity to 2100.

A climate variable unique to each timestep in the model is the key parameter that varies across the scenarios in this study. The Palmer Drought Severity Index (PDSI) forms the basis of this parameter, a commonly used tool to assess drought in the western United States \citep{Cook2004}. The PDSI is appropriate for use in local scales like this section of the Tahoe National Forest, and evaluates precipitation and temperature within a water balance model \citep{HeimJr2002}. I used PDSI data from 2010 to 2099 from \citet{Cook2014} to generate the climate parameter for all future-based simulations. \citet{Cook2014} calculations were fit to PDSI values from 1900 to the present, and used the same methodology as the North American Climate Atlas \citep{Cook2004}. Project partners analyzed the suite of climate models for which \citet{Cook2014} had calculated PDSI, and selected the \textsc{ccsm4} from the National Center for Atmospheric Research and the \textsc{gfdl-esm2m} from the NOAA Geophysical Fluid Dynamics Laboratory. The \textsc{ccsm4} model projects warmer tempeartures and similar precipitation levels to the past several hundred years, while the \textsc{gfdl-esm2m} model projects hotter and drier weather.

Six PDSI sequences based on the \textsc{ccsm4} model were available, so I treated each run as a separate scenario. To generate climate parameters from the PDSI sequences, I calculated the inverse Euclidean distance-weighted mean of PDSI values at 21 points surrounding the centroid of the study area. I then rescaled the results around the mean and standard deviation of the PDSI values representative of the 300 years prior to European settlement, during which the climate was more stable, characterized by dynamic equilibrium rather than a trend. I used 1 as the neutral value so that the parameter could be used as a multiplier within the model. That is, climate parameter values less than 1 reduce susceptibility, fire starts, and spread, while greater climate parameter values increase these properties. Each of the seven total runs followed a unique pattern and trend (Figure \ref{fig:pdsi_future}). I present results in order of increasing median value for the climate parameter during our simulations to facilitate interpretation (Figure \ref{pdsi-boxplots}). The \textsc{ccsm-1} model is similarly distributed to the historical period, with a median near 1.


\begin{figure}[!htbp]
\centering
  \subfloat[][]{
    \centering
	\includegraphics[height=0.25\textheight]{/Users/mmallek/Documents/Thesis/Plots/pdsi/futureclimate_wlm.png}
    \label{fig:pdsi-lm}
  }%
  \subfloat[][]{
  	\centering
	\includegraphics[height=0.25\textheight]{/Users/mmallek/Documents/Thesis/Plots/pdsi/future_last5timesteps.png}
	\label{fig:pdsi-final5}
	}
    \caption{(a) Climate parameter trajectory for 18 timesteps used in our simulations for the 6 scenarios from the \textsc{ccsm4} model and the single scenario from the \textsc{gfdl-esm2m} model. Solid lines connect the climate parameter values for each timestep, and the dashed line represents a fitted linear regression to the data. (b) Zoom on the final five timesteps (without regression lines) for better visualization of variability in each scenario. The climate parameter in \textsc{RMLands} is based on the Palmer Drought Severity Index. Models in the legend appear in descending order from least to greatest mean value for the full time series of the simulation. In a historical range of variability analysis, the PDSI values would be centered around a mean value of 1.0.}
\label{fig:pdsi_future}

\end{figure}

\begin{figure}[!htbp]
\centering
\includegraphics[width=0.7\textwidth]{/Users/mmallek/Documents/Thesis/Seminar/frv-climparam-slide.png}
\caption{Boxplots of climate parameter value for the six runs of the \textsc{ccsm4} model and the single run from the \textsc{gfdl-esm2m} model. The climate parameter in \textsc{RMLands} is based on the Palmer Drought Severity Index. Models are arranged left-to-right from least to greatest mean value for the full time series of the simulation, after the HRV. Boxplot whiskers extend from the $5^{\text{th}}-95^{\text{th}}$ range of variability for each model.}
\label{pdsi-boxplots}
\end{figure}


\subsection*{Evaluating Future Range of Variability}
We evaluate the fire regime across the future scenarios by evaluating the median disturbed area for each across severity levels. %\todo{compared to what we'd expect based on proportional difference?}
We also compute the fire rotation for the two most prevalent cover types in the study area, Sierran Mixed Conifer - Mesic and Sierran Mixed Conifer - Xeric, and compare the simulated future scenario results to the historical fire rotation (Figure~\ref{fig:frotation}). Fire rotation is defined as the time it takes to burn an area equivalent to the area under study (either for a particular cover type or the landscape as a whole) \citep{Agee1993}.%
%


%\subsection*{Landscape composition}
My evaluation of landscape composition compares the seral stage distribution for mesic and xeric mixed conifer forests across future scenarios and against the current distribution. We report the 90\% range of variability and assess departure for these two focal cover types across their seral stages. Both cover types include seven seral stages: Early, Middle--Closed, Middle--Moderate, Middle--Open, Late--Closed, Late--Moderate, and Late--Open. The first term refers chiefly to the developmental stage and the second to the level of canopy cover (breakpoints for canopy cover are at 40\% and 70\%).


%
%\subsection*{Landscape configuration}
I used \textsc{Fragstats} computer software \citep{Fragstats2012} to conduct the spatial pattern analysis and assess differences in landscape configuration between the current landscape and the future range(s) of variability. However, several of the metrics are redundant with one another, and so I focus on a subset of four metrics to simplify interpretation: Area-weighted mean patch area, area-weighted mean core area, clumpiness index, area-weighted mean shape index. We report the $5^{\text{th}}-95^{\text{th}}$ percentile range of variability using boxplots. %and for SMC?

To assess landscape composition and configuration, I compare the current landscape to the FRV, and report departure based on the following standards. If the current landscape metric value falls within the $25^{\text{th}}$ to $75^{\text{th}}$ percentile range (the box in our boxplots), it is considered not departed. If it falls within the $5^{\text{th}}$ to $25^{\text{th}}$ percentile range or the $5^{\text{th}}$ to $95^{\text{th}}$ percentile range (the whiskers in our boxplots), it is moderately departed. If it falls outside that range, it is completely departed.

%  (\textsc{area\_am}) (\textsc{core\_am}) (\textsc{contag}) (\textsc{gyrate\_am}) (\textsc{shape\_am}) (\textsc{siei})


\subsection*{Methodological Limitations}\todo{copy some stuff from HRV chapter here.}
I\todo{Brad says to beef this up (no further specifics).} acknowledge some limitations that should be understood before applying the results in a management context. \textsc{RMLands} simulates wildfires, but does not simulate all of the disturbance processes or all of the complex interactions among them that characterize real landscapes. My input data was the best available, but is not perfect. Improvements to the existing vegetation layer, such as the LiDAR maps currently in development, would improve the model by offering a standardized and more accurate picture of the existing vegetation. In addition, I do not simulate changes in the spatial configuration of land cover types. That is, we do not model upward movement of forest types or type conversions after fire or drought events. Many studies indicate that such change will occur in the next 100 years and even has occurred in many places \citep{Bachelet2001}. Because of this, my results do not specifically represent a likely future outcome, and should be used in conjunction with other studies, especially on range shifts, to predict future vegetation patterns and manage accordingly. 

We note that the climate parameter is a proxy for climate, not a direct measure of it. We utilize a single RCP, 8.5, which is the scenario projecting the most significant change in climate, after \citet{Cook2014}. In generating the climate parameter, we collapsed the projected PDSI data into 5-year summer averages and rescaled it, thus removing some of the variability and extremes present in the raw PDSI.












%%%%%%%%%%%%%%%%%%%%%%%%%%%%%%%%%%%%%%%%%%%%%%%%%%%%%%%%%%%%%%%%%%%%%%%%%%%%%%%%%%%%%%%%%%%%%%%%%%%%%%%%%%%%%%%%%%%%%%%%%%%%%%%%%%%%%%%%%%%%%%%%%%%%%%%%%%%%%%%%%%%%%%%%%%%%%%%%%%%%%%%%%%%%%%%%%%%%%%%%%%%%%%%%%%%%%%%%%%%%%%%%%%%%%%%%%%%%%%%%%%%%%%%%%%%%%%%%%%%%%%%%%%%%%%%%%%%%%%%%%%%%%%%%%%%%%%%%%%%%%%%%%%%%%%%%%%%%%%%%%%%%%%%%%%%%%%%%%%%%%%%%%%%%%%%%%%%%%%%%%%%%%%%%%%%%%%%%%%%%%%%%%%%%%%%%%%%%%%%%%%%%%%%%%%%%%%%%%%%%%%%%%%%%%%%%%%%%%%%%%%%%%%%%%%%%%%%%%%%%%%%%%%%%%%%%%%%%%%%%%%%%%%%%%%%%%%%%%%%%%%%%%%%%%%%%%%%%%%%%%%%%%%%%%%%%%%%%%%%%%%%%%%%%%%%%%%%%%%%%%%%%%%%%%%%%%%%%%%%%%%%%%%%%%%%%%%%%%%%%%%%%%%%%%%%%%%%%%%%%%%%%%%%%%%%%%%%%%%%%%%%%%%%%%%%%%%%%%%
\section{Results}
%* Looks like the absolute range of variability is a lot wider in the future than during the HRV!  \\
%* In some cases the landscape is out of HRV but within most of the FRVs

% brad says figures > table. precise results not necessary
% becky found this confusing
\subsection*{Natural fire regime}

We analyzed the wildfire disturbance regime in terms of its effect on the full study area, mesic mixed conifer forests, and xeric mixed conifer forests.

The median area of land burned by wildfire during simulations of seven alternative future climate trajectories generally increased as the climate parameter increased (Figure~\ref{fig:dareacomp}). In general, this trend was strongest for area burned at high mortality, which drove the increase in total area burnes, as differences in area burned at low mortality did not appear to be ecologically significant. These observations hold for the full study area as well as the mixed conifer forests alone. More striking is the fact that the area burned at high mortality increased relative to low mortality. Compared to the full landscape, this was slightly less conspicuous for mesic mixed conifer forests. However, in xeric mixed conifer forests, the \textsc{gfdl-esm2m} scenario resulted in similar extents of high versus low mortality.

% (\textsc{smc\_}) (\textsc{smc\_x})

%make these plots to the same scale
\begin{figure}[!htbp]
  \centering
    \subfloat[][]{
	\centering
	\includegraphics[width=0.4\textwidth]{/Users/mmallek/Documents/Thesis/Plots/darea/darea-allfmodels.png}
	\label{fig:darea_modelcomp}
	} \\
  \subfloat[][]{
    \centering
    \includegraphics[width=0.4\textwidth]{/Users/mmallek/Documents/Thesis/Plots/darea/darea-allfmodels-smcm.png}
    \label{fig:dareacomp_smcm}
  } 
  \subfloat[][]{
  \centering
    \includegraphics[width=0.4\textwidth]{/Users/mmallek/Documents/Thesis/Plots/darea/darea-allfmodels-smcx.png}
    \label{fig:dareacomp_smcx}
  }
    \caption{Barplots showing (a) proportion of the full study landscape, (b) proportion of Sierran Mixed Conifer - Mesic, and (c) proportion of Sierran Mixed Conifer - Xeric burned for three mortality levels and across the historical simulation and the seven future climate scenarios. Future scenarios presented in order of increasing median value for climate parameter. From left to right, scenarios are presented in order of increasing median climate parameter value.}
  \label{fig:dareacomp}
\end{figure}

% Fire rotation
We also calculated the fire rotation for each scenario, and plot these results across all scenarios (Figure~\ref{fig:frotation}. The historical rotation values for both mixed conifer forest types are always within the range of rotations from the seven future climate scenarios. Although there is considerable variability in the results for all scenarios, as the climate parameter values increase, rotation values decrease for high mortality events. As with the disturbed area, changes to the low mortality rotations are slight, but seem to increase when high mortality rotation decreases more, such that the ``any mortality'' values decline is modest.



\begin{figure}
\centering
\includegraphics[width=0.6\textwidth]{/Users/mmallek/Documents/Thesis/Plots/rotation/rotation_all.png}
\caption{Fire rotation values across scenarios for the full extent of the study area (red), Sierran Mixed Conifer - Mesic (green), and Sierran Mixed Conifer - Xeric (blue), during the last five timesteps of the simulations. The ``Historical'' values are the target values representing the pre-European settlement period fire rotations, which were used in initial model calibration. Point shapes correspond to different mortality levels from fire: low mortality (circles), high mortality (squares), and overall mortality (both high and low combined, triangles). Connecting lines have been included to aid in finding cover and mortality values across scenarios.}
\label{fig:frotation}
\end{figure}



\subsection*{Landscape Pattern}

\paragraph{Seral Stage Distribution}
Our landscape pattern analysis focuses first on changes to the seral stage distribution of mesic and xeric mixed conifer forests. Evidence of both high mortality fire, which triggers a transition to the early seral stage for all cover types, and low mortality fire, which can thin a stand and cause a transition to a more open canopy condition (within the middle or late developmental stages), are visible in examining the output grids.

In order to determine whether the results were an artifact of the initial condition, I also examined the trajectory of all seral stages for both mesic and xeric mixed conifer forests (Figures~\ref{fig:median_trajectory_smcm} and \ref{fig:median_trajectory_smcx}). Because the initial condition for both forest types included a large amount of land in the late development and closed canopy types, which are generally parameterized to be more susceptible to fire than other seral stages, it seemed plausible that these initial conditions could have a disproportionate impact on the trajectory and the range of variability, as displayed in Figures~\ref{fig:covcond_smcm} and \ref{fig:covcond_smcx}. This could occur if the higher susceptibility resulted in high rates of wildfire in addition to high rates of high mortality from wildfire. In particular, this could generate a large amount of land in the Early Development stage that would then dominate the cover type composition for the remainder of the simulation. However, Figures~\ref{fig:median_trajectory_smcm} and \ref{fig:median_trajectory_smcx}, which depict the median values across all runs of each climate model scenario, and for each seral stage of mesic and xeric mixed conifer forests, demonstrate that this is not in fact the case. 

As can be seen in the trajectory plots of the early successional stage, the anticipated increase does occur in both forest types. However, in mesic mixed conifer the proportion dips back down, and appears to be responding more to the climate parameter than the initial conditions, particularly by timestep 14, which is the point at which I begin using the output landscape grids for analysis. In xeric mixed conifer, the initial increase continues throughout the length of the simulation, again indicating that the driver behind this increase is related primarily to the climate setting and related feedback, rather than the initial conditions. In the same way, if the above hypothesis were true, I would expect to see a fairly stable distribution once the first few timesteps had passed. Instead, for most seral stages, include late successional, closed canopy stages, I observe a clear trend. This trend suggests the climate parameter trend, and appears stronger than the influence of the initial conditions. Furthermore, my analysis and conclusions are focused on the end of the simulation. I did this in part to avoid incorporating artifacts related to initial conditions. These plots show that the shift in seral stage proportions over time follow trends, rather than oscillate in equilibrium, and thus the short length of the simulation (compared to a multi-century HRV) is not a main factor in the decline of middle and late successional forest observed in the simulations.

\begin{figure}[htbp]
 \captionsetup[subfigure]{labelformat=empty}
  \centering
  \subfloat[][]{
    \centering
    \includegraphics[height=0.33\textwidth]{/Users/mmallek/Documents/Thesis/Plots/seralstage-trajectory-medians/2410-trajectory-median-legend.png}
  }\\%
  %\qquad
  \subfloat[][]{
    \includegraphics[width=0.33\textwidth]{/Users/mmallek/Documents/Thesis/Plots/seralstage-trajectory-medians/2420-trajectory-median-title.png}
  } 
    \subfloat[][]{
    \centering
    \includegraphics[width=0.33\textwidth]{/Users/mmallek/Documents/Thesis/Plots/seralstage-trajectory-medians/2421-trajectory-median-title.png}
  }%
  %\qquad
  \subfloat[][]{
    \includegraphics[width=0.33\textwidth]{/Users/mmallek/Documents/Thesis/Plots/seralstage-trajectory-medians/2422-trajectory-median-title.png}
  } \\
    \subfloat[][]{
    \centering
    \includegraphics[width=0.33\textwidth]{/Users/mmallek/Documents/Thesis/Plots/seralstage-trajectory-medians/2430-trajectory-median-title.png}
  }%
      \subfloat[][]{
    \centering
    \includegraphics[width=0.33\textwidth]{/Users/mmallek/Documents/Thesis/Plots/seralstage-trajectory-medians/2431-trajectory-median-title.png}
  } 
  \subfloat[][]{
    \includegraphics[width=0.33\textwidth]{/Users/mmallek/Documents/Thesis/Plots/seralstage-trajectory-medians/2432-trajectory-median-title.png}
  }
    \caption{Median trajectory across all climate scenarios and seral stages, for mesic mixed conifer forests. Each climate scenario is shown in a different color. After a noisy beginning, a trend emerges for most seral stages by the final timesteps of the simulation.}
  \label{fig:median_trajectory_smcm}
\end{figure} 

\begin{figure}[htbp]
 \captionsetup[subfigure]{labelformat=empty}
  \centering
  \subfloat[][]{
    \centering
    \includegraphics[height=0.33\textwidth]{/Users/mmallek/Documents/Thesis/Plots/seralstage-trajectory-medians/2610-trajectory-median-legend.png}
  }\\%
  %\qquad
  \subfloat[][]{
    \includegraphics[width=0.33\textwidth]{/Users/mmallek/Documents/Thesis/Plots/seralstage-trajectory-medians/2620-trajectory-median-title.png}
  } 
    \subfloat[][]{
    \centering
    \includegraphics[width=0.33\textwidth]{/Users/mmallek/Documents/Thesis/Plots/seralstage-trajectory-medians/2621-trajectory-median-title.png}
  }%
  %\qquad
  \subfloat[][]{
    \includegraphics[width=0.33\textwidth]{/Users/mmallek/Documents/Thesis/Plots/seralstage-trajectory-medians/2622-trajectory-median-title.png}
  } \\
    \subfloat[][]{
    \centering
    \includegraphics[width=0.33\textwidth]{/Users/mmallek/Documents/Thesis/Plots/seralstage-trajectory-medians/2630-trajectory-median-title.png}
  }%
      \subfloat[][]{
    \centering
    \includegraphics[width=0.33\textwidth]{/Users/mmallek/Documents/Thesis/Plots/seralstage-trajectory-medians/2631-trajectory-median-title.png}
  } 
  \subfloat[][]{
    \includegraphics[width=0.33\textwidth]{/Users/mmallek/Documents/Thesis/Plots/seralstage-trajectory-medians/2632-trajectory-median-title.png}
  }
    \caption{Median trajectory across all climate scenarios and seral stages, for xeric mixed conifer forests. Each climate scenario is shown in a different color. After a noisy beginning, a trend emerges for most seral stages by the final timesteps of the simulation.}
  \label{fig:median_trajectory_smcx}
\end{figure} %smcx

We observe clear trends in three seral stages across both cover types (Figures~\ref{fig:covcond_smcm}-\ref{fig:covcond_smcx}). The proportion of Early Development increased in both with increasing climate parameter values, while Late--Closed and Late--Moderate both decreased. Surprisingly, as the climate parameter increased, the proportion of Middle--Open in the mesic mixed conifer forest increased, while in the xeric mixed conifer the proportion of Middle--Open decreased. In general, the proportion of the current landscape in each seral stage differed substantially from the future ranges of variability. Across all the seral stages, the proportion of each cover type in the early seral stage increased most dramatically. We focus our configuration metrics analysis, then, on the Early Development stage of Sierran Mixed Conifer - Mesic and Sierran Mixed Conifer - Xeric.

%or could write as focused on one cover type at a time. 
In the mesic mixed conifer cover type, as the climate parameter increased, the current seral stage distribution shifted from falling within the projected future range of variability to falling outside it. The larger increase in the proportion of the landscape in the Early and Middle--Open stages comes at the expense of all the Late Development stages, which decline with increasing climate parameter ranges.

In the xeric mixed conifer type, the results were even more dramatic. As Figure~ shows, I observed a substantial increase in the proportion of early, while all other stages showed a declining trend. After Early, the Middle-Open and Late--Open stages were the next most common, and were fairly prevalent on the landscape. Thus I observed complete departure of the current landscape from each seral stage, across all scenarios.

\begin{figure}[htbp]
 \captionsetup[subfigure]{labelformat=empty}
  \centering
  \subfloat[][]{
    \centering
    \includegraphics[width=0.5\textwidth]{/Users/mmallek/Documents/Thesis/Plots/covcond-byscenario/2410-boxplots.png}
  }%
  %\qquad
  \subfloat[][]{
    \includegraphics[width=0.5\textwidth]{/Users/mmallek/Documents/Thesis/Plots/covcond-byscenario/2420-boxplots.png}
  } \\
    \subfloat[][]{
    \centering
    \includegraphics[width=0.5\textwidth]{/Users/mmallek/Documents/Thesis/Plots/covcond-byscenario/2421-boxplots.png}
  }%
  %\qquad
  \subfloat[][]{
    \includegraphics[width=0.5\textwidth]{/Users/mmallek/Documents/Thesis/Plots/covcond-byscenario/2422-boxplots.png}
  } \\
    \subfloat[][]{
    \centering
    \includegraphics[width=0.5\textwidth]{/Users/mmallek/Documents/Thesis/Plots/covcond-byscenario/2430-boxplots.png}
  }%
  %\qquad
    \subfloat[][]{
    \centering
    \includegraphics[width=0.5\textwidth]{/Users/mmallek/Documents/Thesis/Plots/covcond-byscenario/2431-boxplots.png}
  } \\
  \subfloat[][]{
    \includegraphics[width=0.5\textwidth]{/Users/mmallek/Documents/Thesis/Plots/covcond-byscenario/2432-boxplots.png}
  }
    %\qquad
  %\subfloat[][]{
  %  \includegraphics[width=0.5\textwidth]{/Users/mmallek/Documents/Thesis/Plots/covcond-frvhrv/SMCM-frvhrv-boxplots.png}
  %}
  \caption{Boxplots illustrating the range of variability across future climate trajectories for Sierran Mixed Conifer - Mesic. The dashed black horizontal line represents the current condition. Boxplot whiskers extend from the $5^{\text{th}} - 95^{\text{th}}$ range of variability for each model. Climate models appear left-to-right in order of increasing median climate parameter value.}
  \label{fig:covcond_smcm}
\end{figure} %smcm

\begin{figure}[htbp]
 \captionsetup[subfigure]{labelformat=empty}
  \centering
  \subfloat[][]{
    \centering
    \includegraphics[width=0.5\textwidth]{/Users/mmallek/Documents/Thesis/Plots/covcond-byscenario/2610-boxplots.png}
  }%
  %\qquad
  \subfloat[][]{
    \includegraphics[width=0.5\textwidth]{/Users/mmallek/Documents/Thesis/Plots/covcond-byscenario/2620-boxplots.png}
  } \\
    \subfloat[][]{
    \centering
    \includegraphics[width=0.5\textwidth]{/Users/mmallek/Documents/Thesis/Plots/covcond-byscenario/2621-boxplots.png}
  }%
  %\qquad
  \subfloat[][]{
    \includegraphics[width=0.5\textwidth]{/Users/mmallek/Documents/Thesis/Plots/covcond-byscenario/2622-boxplots.png}
  } \\
    \subfloat[][]{
    \centering
    \includegraphics[width=0.5\textwidth]{/Users/mmallek/Documents/Thesis/Plots/covcond-byscenario/2630-boxplots.png}
  }%
      \subfloat[][]{
    \centering
    \includegraphics[width=0.5\textwidth]{/Users/mmallek/Documents/Thesis/Plots/covcond-byscenario/2631-boxplots.png}
  } \\
  \subfloat[][]{
    \includegraphics[width=0.5\textwidth]{/Users/mmallek/Documents/Thesis/Plots/covcond-byscenario/2632-boxplots.png}
  }
    %\qquad
  %\subfloat[][]{
  %  \includegraphics[width=0.5\textwidth]{/Users/mmallek/Documents/Thesis/Plots/covcond-frvhrv/SMCX-frvhrv-boxplots.png}
  %}
    \caption{Boxplots illustrating the range of variability across future climate trajectories for Sierran Mixed Conifer - Xeric. The dashed black horizontal line represents the current condition. Boxplot whiskers extend from the $5^{\text{th}} - 95^{\text{th}}$ range of variability for each model. Climate models appear left-to-right in order of increasing median climate parameter value.}
  \label{fig:covcond_smcx}
\end{figure} %smcx



% debparture categories
%not departed - boxes completely overlap/contain each other
%slightly departed - medians don't overlap the boxes, but boxes overlap
%moderately departed - box overlaps whiskers
%highly departed - only whiskers overlap
%completely departed - no overlap of full rv

% include any others?
% what about total edge?

\paragraph{Early Seral Patch Configuration}
Results for the configuration metrics associated with the Early Development seral stage of both Sierran Mixed Conifer - Mesic and Sierran Mixed Conifer - Xeric indicate that the current landscape is completely departed from the future ranges of variability across climate scenarios (Figures~\ref{fig:fragclass-smcm} and \ref{fig:fragclass-smcx}. The exception to this observation is that the mean patch size and mean core area size results in the xeric mixed conifer forests overlapped the whiskers of the boxplot results in all scenarios except the \textsc{ccsm3} and \textsc{gfdl-esm2m}, indicating moderate departure. In no case did we find that the current landscape was within the range of variability of the simulated future scenarios. In both mesic and xeric mixed conifer forests, mean patch size, mean core area size, and mean shape index (area-weighted, in all cases), increased with increasing climate parameter values. The trend is stronger in xeric forests, with the biggest difference apparent in the results for the \textsc{gfdl-esm2m} scenario. Again in the case of both mesic and xeric variants, the level of fragmentation is completely departed from the current landscape. However, no trend appears with respect to the climate parameter. Thus, patches of early seral mixed conifer forests were large, contained more core area, featured more complex shapes, and were less fragmented than patches on the present-day landscape.


\begin{figure}[!htbp]
 \captionsetup[subfigure]{labelformat=empty}
  \centering
  \subfloat[][]{
    \centering
    \includegraphics[width=0.5\textwidth]{/Users/mmallek/Documents/Thesis/Plots/fragclass-smcmetrics/SMC_M_EARLY_ALL_AREA_AM_boxplots.png}
    \label{fig:boxplot-class-smcm-areaam}
  }%
  %\qquad
  \subfloat[][]{
    \includegraphics[width=0.5\textwidth]{/Users/mmallek/Documents/Thesis/Plots/fragclass-smcmetrics/SMC_M_EARLY_ALL_CLUMPY_boxplots.png}
    \label{fig:boxplot-class-smcm-contag}
  } \\
    \subfloat[][]{
    \includegraphics[width=0.5\textwidth]{/Users/mmallek/Documents/Thesis/Plots/fragclass-smcmetrics/SMC_M_EARLY_ALL_CORE_AM_boxplots.png}
    \label{fig:boxplot-class-smcm-coream}
  }
    %\qquad
    \subfloat[][]{
    \includegraphics[width=0.5\textwidth]{/Users/mmallek/Documents/Thesis/Plots/fragclass-smcmetrics/SMC_M_EARLY_ALL_SHAPE_AM_boxplots.png}
    \label{fig:boxplot-class-smcm-shapeam}
} %\\
  %  \subfloat[][]{
  %  \includegraphics[width=0.5\textwidth]{/Users/mmallek/Documents/Thesis/Plots/fragclass-smcmetrics/SMC_M_EARLY_ALL_ECON_AM_boxplots.png}
  %  \label{fig:boxplot-class-smcm-econam}
  %}
    %\qquad
  %  \subfloat[][]{
  %  \includegraphics[width=0.5\textwidth]{/Users/mmallek/Documents/Thesis/Plots/fragclass-smcmetrics/SMC_M_EARLY_ALL_AI_boxplots.png}
  %  \label{fig:boxplot-class-smcm-ai}
  %}
 \caption{Boxplots illustrating the range of variability in Sierran Mixed Conifer - Mesic, Early Development across future climate trajectories. The dashed black bar represents the current condition. Boxplot whiskers extend from the $5^{\text{th}} - 95^{\text{th}}$ percentile range of variability for each model.}
  \label{fig:fragclass-smcm}
\end{figure} %fragland

\begin{figure}[!htbp]
 \captionsetup[subfigure]{labelformat=empty}
  \centering
  \subfloat[][]{
    \centering
    \includegraphics[width=0.5\textwidth]{/Users/mmallek/Documents/Thesis/Plots/fragclass-smcmetrics/SMC_X_EARLY_ALL_AREA_AM_boxplots.png}
    \label{fig:boxplot-class-smcx-areaam}
  }%
  %\qquad
  \subfloat[][]{
    \includegraphics[width=0.5\textwidth]{/Users/mmallek/Documents/Thesis/Plots/fragclass-smcmetrics/SMC_X_EARLY_ALL_CLUMPY_boxplots.png}
    \label{fig:boxplot-class-smcx-contag}
  } \\
    \subfloat[][]{
    \includegraphics[width=0.5\textwidth]{/Users/mmallek/Documents/Thesis/Plots/fragclass-smcmetrics/SMC_X_EARLY_ALL_CORE_AM_boxplots.png}
    \label{fig:boxplot-class-smcx-coream}
  }
    %\qquad
    \subfloat[][]{
    \includegraphics[width=0.5\textwidth]{/Users/mmallek/Documents/Thesis/Plots/fragclass-smcmetrics/SMC_X_EARLY_ALL_SHAPE_AM_boxplots.png}
    \label{fig:boxplot-class-smcx-shapeam}
} %\\
  %  \subfloat[][]{
  %  \includegraphics[width=0.5\textwidth]{/Users/mmallek/Documents/Thesis/Plots/fragclasssmcmetrics/SMC_X_EARLY_ALL_ECON_AM_boxplots.png}
  %  \label{fig:boxplot-class-smcx-econam}
  %}
    %\qquad
  %  \subfloat[][]{
  %  \includegraphics[width=0.5\textwidth]{/Users/mmallek/Documents/Thesis/Plots/fragclasssmcmetrics/SMC_X_EARLY_ALL_AI_boxplots.png}
  %  \label{fig:boxplot-class-smcx-ai}
  %}
    \caption{Boxplots illustrating the range of variability in Sierran Mixed Conifer - Xeric, EarlyDevelopment across future climate trajectories. The dashed black bar represents the current condition.Boxplot whiskers extend from the $5^{\text{th}} - 95^{\text{th}}$ percentile range of variability foreach model.}
  \label{fig:fragclass-smcx}
\end{figure} %fragland




\clearpage







%%%%%%%%%%%%%%%%%%%%%%%%%%%%%%%%%%%%%%%%%%%%%%%%%%%%%%%%%%%%%%%%%%%%%%%%%%%%%%%%%%%%%%%%%%%%%%
%%%%%%%%%%%%%%%%%%%%%%%%%%%%%%%%%%%%%%%%%%%%%%%%%%%%%%%%%%%%%%%%%%%%%%%%%%%%%%%%%%%%%%%%%%%%%%
%%%%%%%%%%%%%%%%%%%%%%%%%%%%%%%%%%%%%%%%%%%%%%%%%%%%%%%%%%%%%%%%%%%%%%%%%%%%%%%%%%%%%%%%%%%%%%
%%%%%%%%%%%%%%%%%%%%%%%%%%%%%%%%%%%%%%%%%%%%%%%%%%%%%%%%%%%%%%%%%%%%%%%%%%%%%%%%%%%%%%%%%%%%%%
%%%%%%%%%%%%%%%%%%%%%%%%%%%%%%%%%%%%%%%%%%%%%%%%%%%%%%%%%%%%%%%%%%%%%%%%%%%%%%%%%%%%%%%%%%%%%%
%%%%%%%%%%%%%%%%%%%%%%%%%%%%%%%%%%%%%%%%%%%%%%%%%%%%%%%%%%%%%%%%%%%%%%%%%%%%%%%%%%%%%%%%%%%%%%
%%%%%%%%%%%%%%%%%%%%%%%%%%%%%%%%%%%%%%%%%%%%%%%%%%%%%%%%%%%%%%%%%%%%%%%%%%%%%%%%%%%%%%%%%%%%%%


\section{Discussion}


\subsection{Future landscape dynamics of forests in the Yuba River watershed and comparison to current conditions}

I observed a slight to moderate increase in total area burned per timestep with increasing climate parameter sets. Although on its own this might indicate that northern Sierra Nevada forests are resilient to climate change and have stable outcomes, the total burned area results subsume the important finding that high severity fire dramatically increased relative to low severity fire, especially as the climate parameter increased (Figure~\ref{fig:dareacomp}). This finding was more pronounced for the xeric mixed conifer forests than for the mesic mixed conifer forests or the landscape as a whole, which was surprising because I expected the xeric forests to be resilient to an increase in high severity fire because of the ubiquity of low severity fire. Specifically, I found that results for xeric mixed conifer forests from the \textsc{ccsm1} model, which is similar to presettlement conditions, were a ratio of low to high mortality fire of about 2.6; results from the \textsc{gfdl-esm2m} model were a ratio of about 1.0. Our analysis of fire rotation confirms and illustrates this (Figure~\ref{fig:frotation}), making more clear the slight decrease in low mortality fire evident in results for the more extreme climate parameter sets.

From this result I expected to observe changes in the seral stage distribution, especially for xeric forests. I did, and they were dramatic. The trend of increasing fire and the trends in seral stage distribution have sharp jumps in the first few timesteps in most of the cover-seral stage plots (Figures~\ref{fig:median_trajectory_smcm} and \ref{fig:median_trajectory_smcx}). This can be considered an artifact of the initial conditions, as it is a result that can be traced directly to them. However, I also point out that the starting conditions are those from 2010. Large fires have been recorded elsewhere in the Sierra Nevada in the past five years, and there is a trend in the last 30 years of increasing fire \citep{Miller2012}, suggesting that this model result is not far-fetched simply because it projects a lot of fire in the short-term.

The proportion of early seral was strongly correlated with an increase in the climate parameter. In addition, open canopies became more prevalent and closed became less prevalent, which in our model can be attributed to an increase in fire. In the mesic mixed conifer forests, this decline was a clear trend following increasing climate parameter values, and losses of Late--Closed forest were correlated with gains in Early Development conditions. This implies that the decline in prevalence of late successional conditions in mesic mixed conifer forests is likely due to the increased amount of fires that result in high mortality, which precludes many cells from succeeding to the Late stage without experiencing that kind of fire effect. The complete departure of Late--Open across all scenarios compared to the current conditions may be due to the fact that fire exclusion has already reduced the amount of Late--Open on the landscape. In the xeric mixed conifer forests, seral stages other than Early Development, Mid--Open, and Late--Open, were virtually absent. This makes sense because more fire, and especially more fire with high mortality effects, would be likely to produce such a seral stage distribution. Again, cells would be so frequently affected by fire that succession to closed canopy conditions becomes statistically unlikely.

%In my model, early seral conditions are always created by high mortality fire, and either chaparral or trees may after a cell experiences fire. Chaparral is often treated as its own cover type. This is in part due to the fact that chaparral establishment tends to delay the succession process to forest. When studies occur along short temporal scales, it is often more meaningful to categorize these two vegetation communities as separate. However, my study is focused on long temporal scales, and within the study area chaparral will eventually succeed to trees. Consequently, both communities are considered early development. 
With that said, because more early seral is projected in the future, even under fire suppression, examining patch behavior in a range of variability framework can provide insights into how to manage it. I observed an effect of the increased amount of fire, in both extent and severity, on configuration metrics at the seral stage level. I focused on the Early Development stage in mesic and xeric mixed conifer forests because it experienced such a dramatic increase as the climate parameter increased and because the Forest Service is actively developing management practices for early seral habitats. 

I focused my configuration analysis on asking how big, how complex, and how fragmented early seral patches were, as well as how much core area they contained, during the simulated period. In general the results diverge from the current conditions; in most cases the current condition is fully departed from the 90\% range of variability in the future scenarios. Compared to the current landscape, early seral patches under our simulated future scenarios were larger, had larger core areas, were less fragmented, were more irregularly shaped, and had less edge contrast. We observe that the early xeric forests exhibit a stronger trend than the early mesic. Increasing the climate parameter has a more subtle effect on the configuration metrics compared to the seral stage distribution.

These results imply that current trends of increased amounts of high mortality fire relative to low mortality fire are related to climate, and may be difficult to reverse. We observed a loss of structural diversity (compared to current conditions) within the xeric forests, which shifted to a distribution composed almost entirely of Early Development or Open canopy cover forest. Mesic forests contained more structural complexity, but a large increase in Early Development comes at the loss of the Late--Closed and Late--Moderate stages, which is problematic because this cover type is a major source of late successional, closed canopy forest for the study area.

%Because this study relied on the use of computer models, the most appropriate use of the results is to help identify the most influential factors driving landscape change, implications of our simulated disturbance and succession regime, and areas where further research is needed to delineate key parameters. 

%***From powerpoint
%Keeping in mind that they are the outcome of modeling passive management of wildfire and forests. Generally, I observed more fire and more high mortality outcomes from fire, which increases the proportion of early seral and open canopy forest. Other seral stages decline. The resulting early seral patches are larger and have more complex shapes.

%In particular, in the disturbance regime, I observed that: Wildfires were frequent in all simulations, but I still recorded large fires with high proportions of high severity. This makes sense given the known, and specified relationship between drought and wildfire mortality. Thus, frequent fire alone does not necessarily provide increased resistance to high severity fire under more frequent or intense drought in the future. 

%(Seral Stage Distribution) Wide range of future variability in each scenario. Indicates that stability and predictability not likely to be characteristics of future forest and wildfire patterns and that our ability to predict when and where large fires will occur will be constrained by large uncertainty. Need to manage public, agency expectations

%(Early seral patch Configuration) My results were consistent among scenarios, showing the current landscape departs from the RV. Indicates that allowing fires to burn naturally is bigger factor than climate



\subsection{Historic range of variability}
Prior to this study we completed a historical range of variability (HRV) analysis for the same study area (Chapter~\ref{ch:hrv}). The HRV analysis used the same parameter set as the FRV, but with climate parameters from the pre-European settlement period of 1550--1850, which is an appropriate reference period \citep{Safford2013}. The combination of results depicting the relationship between the current landscape, HRV, and FRV provides a mechanism for land managers to identify and prioritize management strategies for promoting resilient forests. In comparing results from the HRV analysis to the FRV analysis, we note that in general current conditions are departed from both the HRV and the FRV results. Often the HRV and FRV are comparable. In these cases, restoration toward the conditions represented by the range of variability analysis results should be evaluated for practicality, with specific implementation being done at a site-specific basis using additional local data to inform specific management actions.

\subsection{Insect Disturbance}
Forest Service scientists are actively researching insect outbreaks that affect western forests, including those in the Sierra Nevada \citep{Liebhold2011}. Unlike wildfire, which is a physical process that is fundamentally the same everywhere, even though its effects are incredibly diverse, insects are a biological agent. Thus the life history characteristics of insects, the method individual species use to invade trees and reproduce, and a tree's response to this invasion result in a much more complex disturbance ecology than that of wildfire. To examine even the effects of bark beetles, a specific type of insects, is a complex undertaking \citep{Fettig2007}. The effects of climate change, especially increased temperatures and decreased precipitation, may enhance the invasion potential for some insect species in some locations, but may also inhibit it \citep{Logan2003,Bentz2010}. If the influence of insect disturbance increases in the study area during this century, there are sure to be interactive effects with wildfire that would influence reality and potentially reduce the predictive power of my results \citep{Ferrell1996}.


\subsection{Using results for management at various scales}
It is important to understand that one limitation of this study is that it was not designed to address questions below the landscape level. While it may be tempting for managers to view subregions of the study area through the lens of the results, it would not be appropriate to set management targets to achieve, for example, proportions of cover type seral stages at a local scale such as a project analysis area that mirror those that characterize the FRV. Any management targets set using the results of this study should be measured at the full landscape level of this study. In addition, my results are organized by non-flexible boundaries such as the watershed and the area assigned to each cover type. Consequently it would not make sense to target a certain proportion of the landscape to be a particular cover type, since that was not modeled in the simulation. Furthermore, the results rely on the generation and analysis of a large quantity of data. When the scale of analysis is reduced, so is the quantity of data produced by the model, and with it, my confidence in the statistical validity of the results and their implications. 

To be more specific, I propose a simple example. The Tahoe National Forest could draw a boundary over an area identified as being appropriate for fire restoration. It would not be appropriate to assume that because across the full study area, late development, open canopy xeric mixed conifer forests ranged from about 25\% to 35\% of the total xeric mixed conifer forest, that within the newly identified project area, similar proportions should prevail. My study does not speak to an appropriate management target for any area smaller than the full study area. That said, managers could draw up a target increase or decrease in extent of a particular seral stage for xeric mixed conifer forests at the project level. This project-level information could then be used to consider how implementing the project would move the landscape as a whole toward or away from the FRV. 

Put another way, these results are no more applicable at the project level than the model should be considered a fire behavior model. It has been optimized to perform well statistically when used to examine landscape change over large extents and long time periods. While spatially explicit in its dynamics, it is a simulation, producing many potential outcomes, and is not intended to predict specific outcomes at specific places and points in time. Rather, the data produced are aggregated in order to develop an understanding of the area as a whole. Because this model should not be used to predict an actual fire rotation value for a particular point on the landscape, it should also not be used to predict the effect of a particular vegetation management strategy \emph{at a particular point on the landscape.} Other forest and fire simulation models are designed for this purpose. \textsc{RMLands} is designed to measure the effect of vegetation management implemented across space and time, to understand the potential impacts of such actions at scales larger than is practical or possible to assess experimentally. A more appropriate use of the results would be to run a similar analysis on another similar sized and ecologically similar area and compare the two. As stated earlier (maybe?), it can also be rerun using different management strategies - this study only explored the no management alternative, but different levels of fire suppression or the addition of vegetation treatments - could also yield insights not otherwise obtainable through local-scale modeling.

\subsection{Implications for Restoration}

Restoration toward a resilient, somewhat stable ecology is often an important goal of resource managers. The comparisons made here consider the difference between the current and future disturbance regimes and landscape patterns, given a scenario in which natural fire regimes were allowed to occur. Since letting all fires burn naturally is not practicable, we note that the results do not provide a simple roadmap for restoration. However, they should provide insights into what landscape patterns may be resilient with climate change. In addition, the results indicate when restoration toward a historical regime, composition, or configuration appears likely to succeed under climate change, or whether the future is likely to significantly diverge from the past. 

One obvious result of this study is that more fire occurs under natural conditions and future climate scenarios than under current and presumed historic conditions. The implications from our study are that a shift towards more fire and especially more high severity fire is likely. Under a suppression-based strategy, there is no reason to believe that future fires would be less severe than those modeled here. I did not observe a plateau in landscape change by 2100, indicating that a stable ecology in terms of fire is not likely in the near-term. If restoration processes take many decades, stability should not be expected to return until restoration is completed, at the earliest. Ongoing trends in climate change will inhibit efforts to oppose analogous trends in forest disturbances. The wide observed range of variability further strengthens my conclusion that managers will have difficulty in maintaining a stable and predictable relationship between fire and the landscape. Large uncertainties in projections will be the norm for some time under a changing climate. 

In addition, given that both the frequency and incidence of high mortality increased under increasingly severe climate conditions, I conclude that restoring frequent fire alone should not be assumed to eliminate the risk of high severity fire (although it ought to reduce it) under more frequent or intense drought in the future. Moreover, my results were consistent among scenarios, indicating that allowing fires to burn naturally may be at least a large, if not larger, a factor than climate change on its own. These results point to a need for restoration of more fires that extend across much larger areas, many of which reset part of the burned area to an early seral condition. Such restoration is unlikely to be politically or socially popular. As a result, public expectations and the capabilities of agencies will need to be carefully managed and articulated. 

Given my findings that seral stage distributions, and patch configuration metrics, are changing in response to changing climate, it is important to recognize that species will be affected differently based on their particular habitat requirements. Thus, using the future range of variability results as a restoration target has implications for wildlife species that may not be socially or legally acceptable. Species that rely on early seral conditions or open canopies will gain habitat in the future, thanks to the increase in the prevalence of these conditions. However, the associated reduction in late development conditions, especially the more closed canopies currently characteristic of mesic mixed conifer forests, will likely have negative ramifications for species dependent on that structural context, such as spotted owl and fisher. Consequently, as managers and decision-makers designate which areas to restore and to what conditions, they will encounter real, unavoidable tradeoffs.

Opportunities to restore forests to patch configurations observed in the simulation may arise when designing timber harvest or fuels treatment projects. However, I caution that patch configurations under natural regimes may be challenging to re-create manually, especially given constraints surrounding the use of fire. For example, it is common to use linear features such as roads and streams as boundaries for treatment. However, such strategies may inhibit our ability to can achieve patch configurations similar to those that would occur naturally.

One limitation of the model is that it did not predict a longer burning season. It may be possible to counteract some of the high severity fire occurrence by injecting more low severity fire through prescribed burning during the shoulder seasons and winter. This could dampen the effects of warmer and drier conditions. Variability density treatments designed to create fine-scale heterogeneity could also ameliorate some of the effects of more frequent fire \citep{Stephens2010,Knapp2012,North2012a}. The vulnerability of closed canopy forest should be an impetus to focus initial restoration and prescribed burning efforts in areas where these values can be protected. Of course, increased amounts of fire will pose challenges in a landscape with complex ownership patterns \citep{Stephens2013}. Excellent coordination at levels beyond the Tahoe National Forest or the Forest Service will be necessary to address this successfully. Federal land management agencies, the California Department of Forestry and Fire Protection (``CalFire''), and local governments are working together more and more to coordinate fire suppression of large wildfires that cross political boundaries.

As just mentioned, my results can be used to identify and prioritize management strategies. Since the Forest Service is directed to manage within the natural range of variability, it is important to define what the natural range of variability is, as well as the desired condition, and recognize what is possible. Some of the trends observed in this analysis, such as increased proportions of Early Development, will likely occur without active management. Others will not occur without changes to fire suppression strategies and related fuels management efforts. Once desired conditions are defined, a comparison to the FRV from this study is warranted. However, this is not to say that achieving a seral stage distribution within the FRV is itself a good goal. Rather, the restoration of the natural process (occurrence and effects) of wildfire is a high-level goal, and future seral stage distrbutions are one of several metrics that can be used to evaluate whether efforts to restore wildfire are successful. Whether or not the desired condition aligns with the FRV will affect what kinds of management actions are taken. Based soley on my results, a proactive approach to restoration would steer xeric forests toward more open conditions now. This should benefit both the ecosystem and the public.

Assessments of the effects of post-fire treatments of various kinds need to be continued and expanded in a systematic and rigorous way. Stand initiation patterns of the past, and the meteorological condition under which they took place, should be evaluated to understand whether similar conditions are likely in the future. Some researchers have called for special management of early seral habitat. For example, \citet{Dellasala2014} call for not managing post-disturbance early seral vegetation. \citet{Swanson2011} suggest mapping and managing early seral communities as a unique cover type. However, given the high probability of more early successional habitat being created by fire and the realities of federal budgets and personnel needs, it seems unlikely that special attention is needed to ensure that some newly created patches of early seral habitat are allowed to succeed naturally. In addition, mapping and managing a transient cover type does not fit within most forest management frameworks, including ours; we assert that early seral forest is a condition within a cover type. Practicing no management at all under climate change is risky, given that current species assemblages may need help to re-establish, and the very real threat of invasive species \citep{Stephens2010}. That said, the large amount of early seral habitat projected to occur on the landscape also means that managers will have options when deciding where to implement restoration efforts that are designed to speed up succession or reduce susceptibility to subsequent burns; it is here that more research on the response to Burned Area Emergency Response and restoration treatments would be most useful.



\subsection{Implications for Planning}
Many National Forests will undergo Forest Plan revisions in the next decade. The 2012 Planning Rule instructs managers to manage for resilient conditions within a natural range of variability. While our results may be used as one potential range of variability, the fact is that our results are also the outcome of a model that simplified several aspects of the landscape, including ownership patterns and tolerable amounts of fire. For example, we did not model varying levels of fire suppression effort. Assuming the Forest Service and other stakeholders in the area do not view the amount of wildfire we simulated as tolerable, they will need to adopt various strategies to try and reduce the likelihood of frequent large fires, as well as the negative impacts of suppression. 

Overall, the results of my study strongly indicate that more time, personnel, and funding will be needed in the future to fight fire, and subsequently repair and restore the damage from firefighting efforts and the flames themselves. I assert that there is a critical need to define desired future conditions for forests, and compare them to what would naturally develop under the simulated conditions, which are intended to portray the future range of variability for the landscape under stuy. I believe it is likely that there will be a gap between the projected conditions and the desired future conditions. Clearly, a ``no-management'' strategy of fire and forests is not compatible with human settlement in the region: before European settlement, fires were used by native peoples to manage vegetation and fire behavior. However, an analysis comparing my results to desired future conditions would provide information on the feasibility of restoration efforts, and provide one anchor for conversations about the future of forest and fire management in the area.

%permanent firebreaks
Managers and planners should also consider the values at risk from fire or fire exclusion and prepare to mitigate them. This need extends to ecological values, such as old growth, closed canopy forest, and social values, such as infrastructure and development. Mitigation of this risk probably requires some form of active management. We outlined some methods for restoring disturbance above. In addition, stakeholders could choose to develop fire breaks in advance of wildfires, for example by clearing vegetation along roads \citep{Conard2003}. However, imposing such ``unnatural'' features has other ecological tradeoffs, especially from increased fragmentation \citep{Trombulak2000}. Such a strategy could be evaluating using our model, either by increasing the ability of roads to function as barriers to fire spread, or by creating a new landcover class for firebreaks and assigning it lowered susceptibility values. It would also be necessary to rerun the simulation with roads in their current form (as barriers) to completely evaluate such a strategy.

%mimic old-growth
On a different note, if more frequent fire does reduce the quantity of old growth forests in the area, mechanical techniques in combination with prescribed fire could be used to mimic the structural complexity of old growth in much younger stands \citep{Franklin2002}. Because it is so productive and trees here grow so quickly, this portion of the Tahoe National Forest may be a good place to experiment with this strategy; trees grow large more quickly here than in other locations in the mixed conifer belt \citep{PRISMClimateGroup2004,Littell2012}.

%type conversions
The risks of type conversions are also pertinent to managers, and potentially one of the biggest risks of an increase in high severity fire as a result of climate change. While not explicitly explored in this study, it is predicted that cover type shifts and conversions are more likely to follow stand-replacing disturbances \citep{Stephens2013}. This risk will increase with climate change, and the additional increase in stand-replacing events suggested by the model indicates an interaction between climate change and high mortality fire that should be taken into account by managers planning restoration after fires, especially when selecting what species to plant or encourage \citep{Fule2008,Schwartz2015}. Active management of forest resources may also be needed to address climate-induced range shifts of cover types. Such shifts are already happening, and high severity fires, since they reset the vegetation to early seral conditions, provide additional opportunities for these shifts to occur by facilitating the progression of vegetation through alternate stages of succession. Changes to vegetation under climate change may be as simple as the upward or northerly movement of species ranges’, but they could also include the establishment of nonnative, invasive species. Managers can choose to try to influence vegetation communities through replanting prior vegetation, engaging in assisted migration, or controlling undesirable species. Unfortunately, there are no obvious or easy answers here. More research on specific large fires, such as the 2013 Rim Fire, should yield insights into how shifts in the fire regime may change spread and susceptibility patterns, which could in turn be used to update and improve our \textsc{RMLands} parameterization \citep{Lydersen2014}.

% range shifts
Active management of forest resources may also be needed to address climate-induced range shifts of cover types. Such shifts are already happening, and high severity fires, since they reset the vegetation to early seral conditions, provide additional opportunities for these shifts to occur by facilitating the progression of vegetation through alternate stages of succession. Changes to vegetation under climate change may be as simple as the upward or northerly movement of species ranges’, but they could also include the establishment of nonnative, invasive species. Managers can choose to try to influence vegetation communities through replanting prior vegetation, engaging in assisted migration, or controlling undesirable species. Unfortunately, there are no obvious or easy answers here.

















