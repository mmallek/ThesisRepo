% !TEX root = master.tex

\chapter{Future Range of Variability}
\label{ch:FRV}

%\section{Abstract}
%In the Sierra Nevada, cycles of fire and vegetation recovery occur variably over large extents, as well as over long periods of time. The U.S. Forest Service's 2012 Planning Rule explicitly calls for the agency to estimate and describe the range of variability under natural disturbance regimes, and manage for those characteristics. Recent warming and drying trends have already influenced a more frequent and proportionally more severe fire regime in the Sierra Nevada. These trends are anticipated to continue under warmer and drier climate change scenarios. We used \textsc{RMLands}, a spatially-explicit, stochastic, landscape-level disturbance and succession model capable of simulating fine-grained processes over large spatial and long temporal extents, to evaluate trends in landscape composition and configuration under a range of potential future climate scenarios. Our results show increasing burned area and increasing high severity fire as with increasing climate values. We also found that today's landscape is outside the future range of variability, and that this departure increases with increasingly warm and dry climate conditions. Based on these findings, we suggest more aggressive restoration efforts and implementation of mitigation measures where the consequences of changing fire regimes are socially unacceptable.



%%%%%%%%%%%%%%%%%%%%%%%%%%%%%%%%%%%%%%%%%%%%%%%%%%%%%%%%%%%%%%%%%%%%%%%%%%%%%%%%%%%%%%%%%%%%%%
%%%%%%%%%%%%%%%%%%%%%%%%%%%%%%%%%%%%%%%%%%%%%%%%%%%%%%%%%%%%%%%%%%%%%%%%%%%%%%%%%%%%%%%%%%%%%%
%%%%%%%%%%%%%%%%%%%%%%%%%%%%%%%%%%%%%%%%%%%%%%%%%%%%%%%%%%%%%%%%%%%%%%%%%%%%%%%%%%%%%%%%%%%%%%
\section{Introduction}

\subsection{Fire in the Sierra Nevada}
In the Sierra Nevada, cycles of fire and vegetation recovery occur variably over large extents, as well as over long periods of time. Ongoing disturbance results in increased heterogeneity captured by various metrics used to describe vegetation composition and configuration \citep{Monica2008}. Prior to European settlement, wildfire was the major source of disturbance in Sierran forests, shaping the composition and configuration of vegetation communities \citep{SNEP1996a}. Fires were primarily lightning-caused, although indigenous peoples are thought to have set fires for vegetation management, especially in the lower elevations \citep{Anderson1996}. In general, fire was frequent, with a mean rotation as short as 20 years in Ponderosa pine (\emph{Pinus ponderosa})-dominated forests. Fire rotations generally increase with increased moisture and elevation. Variance around a mean fire rotation can be remarkable, as some parts of the forest experience fire much more frequently, while others escape fire for long periods \citep{Mallek2013}. Under this disturbance regime, high severity fire was uncommon, but was necessary to initiate early development conditions on the landscape. The predominant effect of most fires was of low overstory tree mortality, which tended to have no effect on the overstory or to open forest canopies, the latter effect more prevalent in more xeric parts of the forest \citep{Skinner1996,Safford2014,SNEP1996,SNEP1996a}.

Since then, fire suppression, logging, grazing, and mining have all interacted to alter the historical fire regime and vegetation patterns \citep{Stephens2015,Knapp2013}. After large-scale fire suppression became the norm in the second half of the 19th century, less fire-tolerant species (such as Douglas fir (\emph{Pseudotsuga menziesii}) and white fir (\emph{Abies concolor})) have come to dominate areas where they were once a minor part of the vegetation community. Grazing and development made fires less common by altering or removing the fine fuels that carried fire. Timber harvest, especially of fire-tolerant species such as ponderosa and sugar pines, accelerated the increased cover of species such as white fir. Finally, fire suppression allowed the buildup of medium size fuels and ladder fuels, which promotes larger more severe fires when they do occur. Moreover, the lack of natural fires has meant that variation in fuel loading has decreased, which allows large fires to spread over very large areas \citep{Hessburg2005,Beaty2007,Meyer2008}.



\subsection{Forest Planning}
With the emergence of ecosystem management in the early 1990s, the need to recognize ecosystems as dynamic and constantly-changing became well accepted, and calls to manage forests sustainably became common \citep{Christensen1996}. Within the context of forest and land management planning, the restoration of ecosystems to their pre-European settlement states was incorporated as a goal or desired future condition into various plans, including the Sierra Nevada Ecosystem Project \cite{SNEP1996a}. The U.S. Forest Service's 2000 Planning Rule explicitly called for the agency to estimate and describe the range of variability under natural disturbance regimes, and manage for those characteristics (36 CFR \textsection 219 2000). The need to consider the natural range of variability was maintained through various amendments to the rule, and is still present in the new 2012 rule, finalized in early 2015 (36 CFR \textsection 219 2012).



\subsection{Range of Variability Analysis}
Historic range of variability (HRV) analysis is a useful tool in landscape planning. HRV analysis is intended to help conceptualize the mechanisms behind large-scale ecosystem functions and provide a basis from which to make predictions about how a given ecosystem will react to disturbances in the future \citep{Nonaka2005,Landres1999}. Methods for quantifying the natural range of variability for a diversity of landscapes in the United States augmented the development of research focused on this task \citep{Landres1999}. Of these, simulation of the historical dynamics became fairly popular. By 2004, some 45 landscape fire and succession models alone had been developed \citep{Keane2004}. Many of these, such as \textsc{landis} \citep{He1999}, \textsc{zelig-l} \citep{Miller1999}, \textsc{safe-forests} \cite{Sessions1997} and \textsc{landsum} \citep{Keane2012} are still in use today. Landscape fire and succession models are used to create spatially-explicit simulations of both of these key forest processes, typically outputting a set of GIS layers for each timestep of the model that can then be analyzed to quantify trajectories and patterns in the disturbance regime, seral stage composition, and landscape configuration over time \citep{Keane2004}. A component of many landscape fire and succession models are state and transition models, which are as much as anything a framework for defining the fundamental vegetation communities and the probability over time of completing a transition from one state to another \citep{Stringham2003,Blankenship2015}.

%Many range of variability analyses in the United States focus on the historical range of variability (HRV) of an area. The Rocky Mountains and Oregon Coast Range in particular have been the focus of several HRV studies, while only one has been conducted in the Sierra Nevada \citep{Miller1999}, which took place in Sequoia National Park in the southern Sierra.

While HRV studies can play an important role in informing natural range of variability, the need to explore and understand the ramifications of climate change on the disturbance regime and forest structure is also critical. Concern about the impact of changes to precipitation and temperature anticipated under climate change in the northern Sierra on local disturbance regimes, and subsequently, seral stage distribution and patch configuration has motivated analysis that consider not only at the current and historical conditions, but also future conditions\citep{Fule2008,North2012}.
%
Recent warming and drying trends have already influenced a more frequent and proportionally more severe fire regime in western forests in general and the Sierra Nevada in particular \citep{McKenzie2004,Westerling2011,Miller2012}. These trends are anticipated to continue under warmer and drier climate change scenarios \citep{Westerling2008,Dale2001}. Changes have also been reported in the elevation of fires in the Sierra Nevada, increasing the potential for range shifts upslope \citep{Schwartz2015}. Where the focus of management efforts had been restoration in the past, now adaptation to ensure resilient ecosystems is the primary objective of managers \citep{Stephens2010}.

Range of variability analyses that offer a complementary analysis of future scenarios under climate change are rare (but see \cite{Keane2008} and \cite{Duveneck2014}). By simulating a range of potential future climate scenarios, we can evaluate trends related to trends projected under climate change, and place the current landscape in that context. Moreover, an examination of landscape composition and configuration under potential climate change scenarios is important because it can provide additional information about what restoration strategies are likely to remain resilient and make sense ecologically for the area under study \citep{Duncan2010}.

Early successional habitats are not a major focus of forest ecology research, in part because they are seen as an intermediate phase that is ideally short \citep{Swanson2011}. However, they are a critical component of all systems, providing the most biodiversity of all seral stages and supporting a range of species' habitat needs \citep{Chang1995,Hutto2008,Swanson2011}. Recent trends in wildfire extent and severity mean that managers are faced with decisions about when and how to manage post-fire early successional habitat. In addition, the Sierra Nevada Forest Plan identifies at least eight management indicator species that require openings and early seral habitat. Our model results will provide insight into the spatial configuration of early successional forests under a natural fire regime for the intensively used mixed conifer zone. These results may also be helpful in planning restoration efforts using both prescribed fire and mechanical harvest techniques.
%%%

\textsc{RMLands} has been used previously to assess the HRV on the San Juan National Forest and the Uncompahgre Plateau in Colorado \citep{McGarigal2005,McGarigal2005a,Romme2009}, as well as the Lolo National Forest in Montana \citep{Cushman2011}. Following the Montana study, which adapted \textsc{RMLands} to use data from the LandFire project (\burl{http://www.landfire.gov}), we further adapted the software for use in the Sierra Nevada in order to prepare an HRV analysis for part of the Tahoe National Forest in California. In this paper we quantify and describe a ``Future Range of Variability'' (FRV) that can inform restoration and planning under a changing climate \citep{Duncan2010}.


\subsection{Objectives}
In this study, our objectives were to evaluate the effect of climate change on the wildfire regime and landscape composition and configuration for the Yuba River waterhsed on the Tahoe National Forest. To do this, we simulated forest fires and succession using \textsc{RMLands}, holding all model parameters except the climate parameter constant. The climate parameter incorporated Palmer Drought Severity Index (PDSI) values from a suite of seven climate trajectories developed by the National Center for Atmospheric Research (USA) and the Canadian Centre for Climate Modelling and Analysis to the year 2090 \citep{Cook2014}. We used \textsc{Fragstats} software and R to analyze outputs and report the 90\% range of variability for simulated future metrics. Ultimately, we evaluate our results for a series of simulations for these future scenarios in comparison to the current conditions, and assess implications for restoration and forest planning.

















%%%%%%%%%%%%%%%%%%%%%%%%%%%%%%%%%%%%%%%%%%%%%%%%%%%%%%%%%%%%%%%%%%%%%%%%%%%%%%%%%%%%%%%%%%%%%%%%%%%%%%%%%%%%%%%%%%%%%%%%%%%%%%%%%%%%%%%%%%%%%%%%%%%%%%%%%%%%%%%%%%%%%%%%%%%%%%%%%%%%%%%%%%%%%%%%%%%%%%%%%%%%%%%%%%%%%%%%%%%%%%%%%%%%%%%%%%%%%%%%%%%%%%%%%%%%%%%%%%%%%%%%%%%%%%%%%%%%%%%%%%%%%%%%%%%%%%%%%%%%%%%%%%%%%%%%%%%%%%%%%%%%%%%%%%%%%%%%%%%%%%%%%%%%%%%%%%%%%%%%%%%%%%%%%%%%%%%%%%%%%%%%%%%%%%%%%%%%%%%%%%%%%%%%%%%%%%%%%%%%%%%%%%%%%%%%%%%%%%%%%%%%%%%%%%%%%%%%%%%%%%%%%%%%%%%%%%%%%%%%%%%%%%%%%%%%%%%%%%%%%%%%%%%%%%%%%%%%%%%%%%%%%%%%%%%%%%%%%%%%%%%%%%%%%%%%%%%%%%%%%%%%%%%%%%%%%%%%%%%%%%%%%%%%%%%%%%%%%%%%%%%%%%%%%%%%%%%%%%%%%%%%%%%%%%%%%%%%%%%%%%%%%%%%%%%%%%%%%%

\section{Methods}

\subsection*{Study area}
The project landscape (see Figure~\ref{projectarea-ch3}) is located on the northern part of the Tahoe National Forest, on the Yuba River and Sierraville Ranger Districts, and comprises about 181,550 hectares. The topography of the project landscape consists of rugged mountains incised by two major and a few minor river drainages. Elevation ranges from about 350 m to 2500 m. The area receives 30 cm to 260 cm of precipitation annually, most of which falls as snow in the middle to upper elevations \citep{Storer1963}. Some areas in the mid-elevation band receive high precipitation compared to the region, resulting in patches of exceptionally productive forest \citep{Littell2012}. Vegetation is tremendously diverse and changes slowly along an elevational gradient and in response to local changes in drainage, aspect, and soil structure. Grasslands, chaparral, oak woodlands, mixed conifer forests, and subalpine forests are all found within the study area.

% brad said to make study area more obvious for non-US readers; will probably have to redo plot for publication but this is ok for now I think
\begin{figure}
\centering
\includegraphics[width=.8\textwidth]{/Users/mmallek/Tahoe/Report3/images/studyarea.png}
\caption{The Sierra Nevada Ecoregion is outlined in brown. The project landscape (outlined in green) is located in the northern extent of the Sierra Nevada on the Tahoe National Forest, comprising the Yuba River watershed.}
\label{projectarea-ch3}
\end{figure}

Xeric and mesic Sierran mixed conifer forests are the two most prevalent cover types within the study area, together comprising 63\% of the landscape. These forests are characterized by five conifers and one hardwood: \emph{Abies concolor, Pseudotsuga menziesii, Pinus ponderosa, Pinus lambertiana, Calocedrus decurrens}, and \emph{Quercus kelloggii}. At least three conifers are typically present in any given stand. \emph{A.~concolor} tends to be the most ubiquitous species, especially on north-facing slopes. \emph{Pinus ponderosa} was historically the dominant species, under the previous frequent low severity fire regime. It is still the most prevalent species on south-facing slopes and is present continuously from the Oak-Conifer Forest and Woodland zone below it in elevation (Appendix~\ref{smc-description}).


\subsection{RMLands}

\textsc{RMLands} is a spatially-explicit, stochastic, landscape-level disturbance and succession model capable of simulating fine-grained processes over large spatial and long temporal extents \citep{McGarigal2005}. It is grid-based and simulates fire on landscapes in a spatially explicit and realistic manner. State transitions are simulated at the 30 m pixel scale. Transitions may take place in response to fire or in the absence of it (natural succession) \citep{McGarigal2012}. Outputs from the model are readable by the landscape pattern analysis software \textsc{Fragstats} \citep{Fragstats2012}, which facilitates the landscape configuration analysis.

In \textsc{RMLands}, fires spread probabilistically based on the susceptibility of an individual cell. It does not contain a fire model and fuels are not directly incorporated into fire spread. In addition, we do not classify individual fires as a whole to a ``low,'' ``mixed,'' or ``high'' severity status. Some fire ecologists combine fire attributes such as flame length and fire size into their interpretation of the relative ``severity'' of a particular fire \citep{Agee1993}.   Ecologists working at other scales and not working with models often describe ``mixed severity'' regimes \citep[e.g.,][]{Kane2013}, which \citet{Collins2010} define as ``stand-replacing patches within a matrix of low to moderate fire-induced effects.'' Because at the 30 m cell size of our model, nearly all fires would be classified as ``mixed severity'' by the prior definition, it becomes moot. Instead, we evaluate and classify fire by its effects on individual cells. First, we evaluate whether a cell burned. Next, all burned cells are evaluated probabilistically and assigned to a high severity (``high mortality'') outcome or not. All non-high mortality outcomes are considered low mortality. If a cell burns at high severity, then it transitions to the Early seral stage. Recently, some researchers have differed on whether 75\% or 95\% overstory tree mortality is a more appropriate cutoff point for defining a ``stand-replacing'' event \citep{Fule2014,Mallek2013}. In this paper, we use 75\% as our cutoff, which is widely accepted in the literature \citep{Miller2009,Baker2014,Agee1993,Agee2007}.


In collaboration with USDA Forest Service staff, we developed a system of land cover and seral stage classification based on LandFire (\burl{http://www.landfire.org}) and \citet{VandeWater2011}'s Presettlement Fire Regimes, and crosswalked Forest Service corporate spatial data based on the Northern Sierra \textsc{CalVeg} classification to the 31 cover types. We also used Forest Service corporate spatial data to develop layers for seral stage and age, plus physical environment layers (e.g., elevation). State and transition models for 31 land cover types were developed based on the VDDT models associated with the LandFire project, and refined with input from local experts to capture subtle changes in succession and transition at the project scale. In general model parameters were developed using meta-analyses published in the literature. For example, LandFire data was used to calculate transition probabilities, several fire rotation calibration parameters were taken from \citet{Mallek2013}, and wind direction information was obtained from several area weather stations. We used landscape conditions as of 2010 as the starting point for all simulations and as the ``current'' conditions for comparison with the HRV results.

Although \textsc{RMLands} is a process-based model with parameters sourced from the literature, our team had greater confidence in some parameters than others, especially as to how they function within the \textsc{RMLands} framework. Consequently, we calibrated our model parameters by iteratively adjusting certain low-confidence parameters to optimize the output values for parameters known with high confidence. Specifically, we manipulated an ignition calibration coefficient and a fire return index (similar in concept to a fire return interval), and measured calibration success based on conformity to pre-specified rotation values at the cover type level. Fire return index values were multiplied by a single factor across all seral stages of a given cover type; cover types were modified as groups but the index ratios within them were held constant.

We set our calibration target as rotation values for the nine most prevalent cover types within 10\% of their original target rotations (Sierran Mixed Conifer - Mesic, Xeric, and Ultramafic variants; Red Fir - Mesic and Xeric variants; Oak-Conifer Forest and Woodland - standard and Ultramafic variants; Mixed Evergreen - Mesic and Xeric variants). We focused on these nine types because they all extend across more than 1,000 ha, and are thus statistically stable from simulation to simulation. Target values were based on published empirical values and refined with input from local experts. We chose rotation as the calibration target because targets were available from the literature and because fire rotation is a fundamental measurement that \textsc{RMLands} was designed to capture. In addition, using rotation ties calibration to a parameter that is relatable to Forest Service staff and that can be used as a target by managers in various programs.

In a historical range of variability analysis, the model is typically run for a very long time to capture full disturbance cycles and the consequent effects on vegetation. This is possible when the climate parameter oscillates around a mean, but not when a clear trend exists, such as in the case of climate change. Therefore instead of running the model for a long time, we ran it 100 times for each climate parameter sequence. No true model equilibration was done for the future simulations, but we elected to include only the final 5 timesteps (25 years) of the results in order to achieve some distance from the starting (current) conditions, maintain focus on the ending trajectory of climate and the range of variation possible in the landscape at the end of the 21st century, yet still capture the variability inherent to the PDSI-based climate parameter (Figure \ref{fig:pdsi-final5}).



%%%% Brad's comment: Is it possible to validate this model in some way? In particular, how do we know that varying the climate multiplier affects fires in a realistic way? Could you hindcast to see how well the PDSI for past years matches the historical fire pattern?

%%%% Notes in response: not easily. some language in hrv that addresses this. we don't know, beyond my assumption that the model treats it realistically in the first place, which is based on Kevin's word. You could hindcast, maybe. But I'm not expecting a super strong relationship. Don't know exactly how to deal with this comment but it feels important.
%%%% More notes: HRV article has a brief comparison of climate vs. disturbed area. Weak relationship. We expect cascading results affecting landscape structure and composition as well as changes to fires. Some non-standard fire behavior may occur but no real way to test this because we don't know what fires will "look" like in the future.

\subsection{Climate Parameter and PDSI}
A climate variable unique to each timestep in the model is the key parameter that varies across the scenarios in this study. The Palmer Drought Severity Index (PDSI) forms the basis of this parameter, a commonly used tool to assess drought in the western United States \citep{Cook2004}. The PDSI is appropriate for use in local scales like ours, and evaluates precipitation and temperature within a water balance model \citep{HeimJr2002}. We used PDSI data from 2010 to 2099 from \citet{Cook2014} based on the RCP 8.5 model outputs to generate the climate parameter for all future-based simulations. Their calculations were fit to PDSI values from 1900 to the present, and used the same methodology as the North American Climate Atlas \citep{Cook2004}. Project partners analyzed the suite of climate models for which \citet{Cook2014} had calculated PDSI, and selected the \textsc{ccsm4} from the National Center for Atmospheric Research and the \textsc{gfdl-esm2m} from the NOAA Geophysical Fluid Dynamics Laboratory. The \textsc{ccsm4} model proejcts warmer tempeartures and similar precipitation levels to the past several hundred years, while the \textsc{gfdl-esm2m} model projects hotter and drier weather.

Six PDSI sequences based on the \textsc{ccsm4} model were available, so we treated each run as a separate scenario. To generate climate parameters from the PDSI sequences, we calculated the inverse Euclidean distance-weighted mean of PDSI values at 21 points surrounding the centroid of the project area. We then rescaled the results around the mean and standard deviation of the PDSI values representative of the 300 years prior to European settlement, during which the climate was more stable, characterized by dynamic equilibrium rather than a trend. We used 1 as the neutral value so that the parameter could be used as a multiplier within the model. That is, climate parameter values less than 1 reduce susceptibility, fire starts, and spread, while greater climate parameter values increase these properties. Each of the seven total runs followed a unique pattern and trend (Figure \ref{fig:pdsi_future}). We present results in order of increasing median value for the climate parameter during our simulations to facilitate interpretation (Figure \ref{pdsi-boxplots}). The \textsc{ccsm-1} model is similarly distributed to the historical period, with a median near 1.


\begin{figure}[!htbp]
\centering
  \subfloat[][]{
    \centering
	\includegraphics[height=0.25\textheight]{/Users/mmallek/Documents/Thesis/Plots/pdsi/futureclimate_wlm.png}
    \label{fig:pdsi-lm}
  }%
  \subfloat[][]{
  	\centering
	\includegraphics[height=0.25\textheight]{/Users/mmallek/Documents/Thesis/Plots/pdsi/future_last5timesteps.png}
	\label{fig:pdsi-final5}
	}
    \caption{(a) Climate parameter trajectory for 18 timesteps used in our simulations for the 6 scenarios from the \textsc{ccsm4} model and the single scenario from the \textsc{gfdl-esm2m} model. Solid lines connect the climate parameter values for each timestep, and the dashed line represents a fitted linear regression to the data. (b) Zoom on the final five timesteps (without regression lines) for better visualization of variability in each scenario. The climate parameter in \textsc{RMLands} is based on the Palmer Drought Severity Index. Models in the legend appear in descending order from least to greatest mean value for the full time series of the simulation. In a historical range of variability analysis, the PDSI values would be centered around a mean value of 1.0.}
\label{fig:pdsi_future}

\end{figure}

\begin{figure}[!htbp]
\centering
\includegraphics[width=0.7\textwidth]{/Users/mmallek/Documents/Thesis/Plots/pdsi/pdsiboxplots_ordered_nohrv.png}
\caption{Boxplots of climate parameter value for the six runs of the \textsc{ccsm4} model and the single run from the \textsc{gfdl-esm2m} model. The climate parameter in \textsc{RMLands} is based on the Palmer Drought Severity Index. Models are arranged left-to-right from least to greatest mean value for the full time series of the simulation, after the HRV. Boxplot whiskers extend from the $5^{\text{th}}-95^{\text{th}}$ range of variability for each model.}
\label{pdsi-boxplots}
\end{figure}


\subsection*{Evaluating Future Range of Variability}
We evaluate the fire regime across the future scenarios by evaluating the median disturbed area for each across severity levels. %\todo{compared to what we'd expect based on proportional difference?}
We also compute the fire rotation for the two most prevalent cover types in the study area, Sierran Mixed Conifer - Mesic and Sierran Mixed Conifer - Xeric, and compare the simulated future scenario results to the historical fire rotation (Figure~\ref{fig:frotation}). Fire rotation is defined as the time it takes to burn an area equivalent to the area under study (either for a particular cover type or the landscape as a whole) \citep{Agee1993}.%
%


%\subsection*{Landscape composition}
Our evaluation of landscape composition compares the seral stage distribution for mesic and xeric mixed conifer forests across future scenarios and against the current distribution. We report the 90\% range of variability and assess departure for these two focal cover types across their seral stages. Both cover types include seven seral stages: Early, Middle--Closed, Middle--Moderate, Middle--Open, Late--Closed, Late--Moderate, and Late--Open. The first term refers chiefly to the developmental stage and the second to the level of canopy cover (breakpoints for canopy cover are at 40\% and 70\%).


%
%\subsection*{Landscape configuration}
We used \textsc{Fragstats} computer software \citep{Fragstats2012} to conduct the spatial pattern analysis and assess differences in landscape configuration between the current landscape, the historical range of variability, and the future range(s) of variability. However, several of the metrics are redundant with one another, and so we focus on a subset of four metrics to simplify interpretation: Area-weighted mean patch area, area-weighted mean core area, clumpiness index, area-weighted mean shape index. We report the $5^{\text{th}}-95^{\text{th}}$ percentile range of variability using boxplots. %and for SMC?

To assess landscape composition and configuration, we compare the current landscape to the FRV, and report departure based on the following standards. If the current landscape metric value falls within the $25^{\text{th}}$ to $75^{\text{th}}$ percentile range (the box in our boxplots), it is considered not departed. If it falls within the $5^{\text{th}}$ to $25^{\text{th}}$ percentile range or the $5^{\text{th}}$ to $95^{\text{th}}$ percentile range (the whiskers in our boxplots), it is moderately departed. If it falls outside that range, it is completely departed. And if it falls to its death, it is dearly departed.\todo{eventually will have to take this out}

%  (\textsc{area\_am}) (\textsc{core\_am}) (\textsc{contag}) (\textsc{gyrate\_am}) (\textsc{shape\_am}) (\textsc{siei})


\subsection*{Methodological Limitations}
We acknowledge some limitations that should be understood before applying the results in a management context. \textsc{RMLands} simulates wildfires, but does not simulate all of the disturbance processes or all of the complex interactions among them that characterize real landscapes. Our input data was the best available, but is not perfect. We do not simulate changes in the spatial configuration of land cover types. That is, we do not model upward movement of forest types or type conversions after fire or drought events. Many studies indicate that such change will occur in the next 100 years and even has occurred in many places \citep{Bachelet2001}.

We note that the climate parameter is a proxy for climate, not a direct measure of it. We utilize a single RCP, 8.5, which is the scenario projecting the most significant change in climate, after \citet{Cook2014}. In generating the climate parameter, we collapsed the projected PDSI data into 5-year summer averages and rescaled it, thus removing some of the variability and extremes present in the raw PDSI.












%%%%%%%%%%%%%%%%%%%%%%%%%%%%%%%%%%%%%%%%%%%%%%%%%%%%%%%%%%%%%%%%%%%%%%%%%%%%%%%%%%%%%%%%%%%%%%%%%%%%%%%%%%%%%%%%%%%%%%%%%%%%%%%%%%%%%%%%%%%%%%%%%%%%%%%%%%%%%%%%%%%%%%%%%%%%%%%%%%%%%%%%%%%%%%%%%%%%%%%%%%%%%%%%%%%%%%%%%%%%%%%%%%%%%%%%%%%%%%%%%%%%%%%%%%%%%%%%%%%%%%%%%%%%%%%%%%%%%%%%%%%%%%%%%%%%%%%%%%%%%%%%%%%%%%%%%%%%%%%%%%%%%%%%%%%%%%%%%%%%%%%%%%%%%%%%%%%%%%%%%%%%%%%%%%%%%%%%%%%%%%%%%%%%%%%%%%%%%%%%%%%%%%%%%%%%%%%%%%%%%%%%%%%%%%%%%%%%%%%%%%%%%%%%%%%%%%%%%%%%%%%%%%%%%%%%%%%%%%%%%%%%%%%%%%%%%%%%%%%%%%%%%%%%%%%%%%%%%%%%%%%%%%%%%%%%%%%%%%%%%%%%%%%%%%%%%%%%%%%%%%%%%%%%%%%%%%%%%%%%%%%%%%%%%%%%%%%%%%%%%%%%%%%%%%%%%%%%%%%%%%%%%%%%%%%%%%%%%%%%%%%%%%%%%%%%%%%%%%
\section{Results}
%* Looks like the absolute range of variability is a lot wider in the future than during the HRV!  \\
%* In some cases the landscape is out of HRV but within most of the FRVs

% brad says figures > table. precise results not necessary
% becky found this confusing
\subsection*{Natural fire regime}

We analyzed the wildfire disturbance regime in terms of its effect on the full study area, mesic mixed conifer forests, and xeric mixed conifer forests.

The median area of land burned by wildfire during simulations of seven alternative future climate trajectories generally increased as the climate parameter increased (Figure~\ref{fig:dareacomp}). In general, this trend was strongest for area burned at high mortality, which drove the increase in total area burnes, as differences in area burned at low mortality did not appear to be ecologically significant. These observations hold for the full study area as well as the mixed conifer forests alone. More striking is the fact that the area burned at high mortality increased relative to low mortality. Compared to the full landscape, this was slightly less conspicuous for mesic mixed conifer forests. However, in xeric mixed conifer forests, the \textsc{gfdl-esm2m} scenario resulted in similar extents of high versus low mortality.

% (\textsc{smc\_}) (\textsc{smc\_x})

%make these plots to the same scale
\begin{figure}[!htbp]
  \centering
    \subfloat[][]{
	\centering
	\includegraphics[width=0.4\textheight]{/Users/mmallek/Documents/Thesis/Plots/darea/darea-allfmodels.png}
	\label{fig:darea_modelcomp}
	}

  \subfloat[][]{
    \centering
    \includegraphics[width=0.4\textheight]{/Users/mmallek/Documents/Thesis/Plots/darea/darea-allfmodels-smcm.png}
    \label{fig:dareacomp_smcm}
  }

  \subfloat[][]{
  \centering
    \includegraphics[width=0.4\textheight]{/Users/mmallek/Documents/Thesis/Plots/darea/darea-allfmodels-smcx.png}
    \label{fig:dareacomp_smcx}
  }
    \caption{Barplots showing (a) proportion of the full study landscape, (b) proportion of Sierran Mixed Conifer - Mesic, and (c) proportion of Sierran Mixed Conifer - Xeric burned for three mortality levels and across the historical simulation and the seven future climate scenarios. Future scenarios presented in order of increasing median value for climate parameter. Dotted lines in the HRV section represent the median value for each mortality level after combining all seven future scenario results. From left to right, scenarios are presented in order of increasing median climate parameter value.}
  \label{fig:dareacomp}
\end{figure}

% Fire rotation
We also calculated the fire rotation for each scenario, and plot these results across all scenarios (Figure~\ref{fig:frotation}. The historical rotation values for both mixed conifer forest types are always within the range of rotations from the seven future climate scenarios. Although there is considerable variability in the results for all scenarios, as the climate parameter values increase, rotation values decrease for high mortality events. As with the disturbed area, changes to the low mortality rotations are slight, but seem to increase when high mortality rotation decreases more, such that the ``any mortality'' values decline is modest.



\begin{figure}
\centering
\includegraphics[width=\textwidth]{/Users/mmallek/Documents/Thesis/Plots/rotation/rotation_all.png}
\caption{Fire rotation values across scenarios for Sierran Mixed Conifer - Mesic, Sierran Mixed Conifer - Xeric, and the full extent of the study area during the last five timesteps of the simulations, as well as the values for the historical range of variability. Different colors denote different extents (the full landscape, specific cover types). Different point shapes correspond to different mortality levels. Connecting lines have been included to aid in finding cover and mortality values across scenarios.}
\label{fig:frotation}
\end{figure}



\subsection*{Landscape Pattern}

\paragraph{Seral Stage Distribution}
Our landscape pattern analysis focuses first on changes to the seral stage distribution of mesic and xeric mixed conifer forests. Evidence of both high mortality fire, which triggers a transition to the early seral stage for all cover types, and low mortality fire, which can thin a stand and cause a transition to a more open canopy condition (within the middle or late developmental stages), are visible in examining the output grids.

We observe clear trends in three seral stages across both cover types (Figures~\ref{fig:covcond_smcm}-\ref{fig:covcond_smcx}). Early Development increased in both, while Late--Closed and Late--Moderate both decreased. Surprisingly, in the mesic mixed conifer the proportion of Middle--Open increased as the climate parameter increased, while in the xeric mixed conifer the proportion of Middle--Open decreased as the climate parameter increased. In general, the proportion of the current landscape in each seral stage differed substantially from the future ranges of variability. Across all the seral stages, the proportion of each cover type in the early seral stage increased most dramatically. We focus our configuration metrics analysis, then, on the Early Development stage of Sierran Mixed Conifer - Mesic and Sierran Mixed Conifer - Xeric.



\begin{figure}[htbp]
 \captionsetup[subfigure]{labelformat=empty}
  \centering
  \subfloat[][]{
    \centering
    \includegraphics[width=0.5\textwidth]{/Users/mmallek/Documents/Thesis/Plots/covcond-byscenario/2410-boxplots.png}
  }%
  %\qquad
  \subfloat[][]{
    \includegraphics[width=0.5\textwidth]{/Users/mmallek/Documents/Thesis/Plots/covcond-byscenario/2420-boxplots.png}
  } \\
    \subfloat[][]{
    \centering
    \includegraphics[width=0.5\textwidth]{/Users/mmallek/Documents/Thesis/Plots/covcond-byscenario/2421-boxplots.png}
  }%
  %\qquad
  \subfloat[][]{
    \includegraphics[width=0.5\textwidth]{/Users/mmallek/Documents/Thesis/Plots/covcond-byscenario/2422-boxplots.png}
  } \\
    \subfloat[][]{
    \centering
    \includegraphics[width=0.5\textwidth]{/Users/mmallek/Documents/Thesis/Plots/covcond-byscenario/2430-boxplots.png}
  }%
  %\qquad
    \subfloat[][]{
    \centering
    \includegraphics[width=0.5\textwidth]{/Users/mmallek/Documents/Thesis/Plots/covcond-byscenario/2431-boxplots.png}
  } \\
  \subfloat[][]{
    \includegraphics[width=0.5\textwidth]{/Users/mmallek/Documents/Thesis/Plots/covcond-byscenario/2432-boxplots.png}
  }
    %\qquad
  %\subfloat[][]{
  %  \includegraphics[width=0.5\textwidth]{/Users/mmallek/Documents/Thesis/Plots/covcond-frvhrv/SMCM-frvhrv-boxplots.png}
  %}
  \caption{Boxplots illustrating the range of variability across future climate trajectories for Sierran Mixed Conifer - Mesic. The horizontal blue line represents the current condition. Boxplot whiskers extend from the $5^{\text{th}} - 95^{\text{th}}$ range of variability for each model. Climate models appear left-to-right in order of increasing median climate parameter value.}
  \label{fig:covcond_smcm}
\end{figure} %smcm

\begin{figure}[htbp]
 \captionsetup[subfigure]{labelformat=empty}
  \centering
  \subfloat[][]{
    \centering
    \includegraphics[width=0.5\textwidth]{/Users/mmallek/Documents/Thesis/Plots/covcond-byscenario/2610-boxplots.png}
  }%
  %\qquad
  \subfloat[][]{
    \includegraphics[width=0.5\textwidth]{/Users/mmallek/Documents/Thesis/Plots/covcond-byscenario/2620-boxplots.png}
  } \\
    \subfloat[][]{
    \centering
    \includegraphics[width=0.5\textwidth]{/Users/mmallek/Documents/Thesis/Plots/covcond-byscenario/2621-boxplots.png}
  }%
  %\qquad
  \subfloat[][]{
    \includegraphics[width=0.5\textwidth]{/Users/mmallek/Documents/Thesis/Plots/covcond-byscenario/2622-boxplots.png}
  } \\
    \subfloat[][]{
    \centering
    \includegraphics[width=0.5\textwidth]{/Users/mmallek/Documents/Thesis/Plots/covcond-byscenario/2630-boxplots.png}
  }%
      \subfloat[][]{
    \centering
    \includegraphics[width=0.5\textwidth]{/Users/mmallek/Documents/Thesis/Plots/covcond-byscenario/2631-boxplots.png}
  } \\
  \subfloat[][]{
    \includegraphics[width=0.5\textwidth]{/Users/mmallek/Documents/Thesis/Plots/covcond-byscenario/2632-boxplots.png}
  }
    %\qquad
  %\subfloat[][]{
  %  \includegraphics[width=0.5\textwidth]{/Users/mmallek/Documents/Thesis/Plots/covcond-frvhrv/SMCX-frvhrv-boxplots.png}
  %}
    \caption{Boxplots illustrating the range of variability across future climate trajectories for Sierran Mixed Conifer - Xeric. The horizontal blue line represents the current condition. Boxplot whiskers extend from the $5^{\text{th}} - 95^{\text{th}}$ range of variability for each model. Climate models appear left-to-right in order of increasing median climate parameter value.}
  \label{fig:covcond_smcx}
\end{figure} %smcx

% departure categories
%not departed - boxes completely overlap/contain each other
%slightly departed - medians don't overlap the boxes, but boxes overlap
%moderately departed - box overlaps whiskers
%highly departed - only whiskers overlap
%completely departed - no overlap of full rv

% include any others?
% what about total edge?

	\begin{figure}[!htbp]
	 \captionsetup[subfigure]{labelformat=empty}
	  \centering
	  \subfloat[][]{
	    \centering
	    \includegraphics[width=0.5\textwidth]{/Users/mmallek/Documents/Thesis/Plots/fragclass-smcmetrics/SMC_M_EARLY_ALL_AREA_AM_boxplots.png}
	    \label{fig:boxplot-class-smcm-areaam}
	  }%
	  %\qquad
	  \subfloat[][]{
	    \includegraphics[width=0.5\textwidth]{/Users/mmallek/Documents/Thesis/Plots/fragclass-smcmetrics/SMC_M_EARLY_ALL_CLUMPY_boxplots.png}
	    \label{fig:boxplot-class-smcm-contag}
	  } \\
	    \subfloat[][]{
	    \includegraphics[width=0.5\textwidth]{/Users/mmallek/Documents/Thesis/Plots/fragclass-smcmetrics/SMC_M_EARLY_ALL_CORE_AM_boxplots.png}
	    \label{fig:boxplot-class-smcm-coream}
	  }
	    %\qquad
	    \subfloat[][]{
	    \includegraphics[width=0.5\textwidth]{/Users/mmallek/Documents/Thesis/Plots/fragclass-smcmetrics/SMC_M_EARLY_ALL_SHAPE_AM_boxplots.png}
	    \label{fig:boxplot-class-smcm-shapeam}
	} %\\
	  %  \subfloat[][]{
	  %  \includegraphics[width=0.5\textwidth]{/Users/mmallek/Documents/Thesis/Plots/fragclass-smcmetrics/SMC_M_EARLY_ALL_ECON_AM_boxplots.png}
	  %  \label{fig:boxplot-class-smcm-econam}
	  %}
	    %\qquad
	  %  \subfloat[][]{
	  %  \includegraphics[width=0.5\textwidth]{/Users/mmallek/Documents/Thesis/Plots/fragclass-smcmetrics/SMC_M_EARLY_ALL_AI_boxplots.png}
	  %  \label{fig:boxplot-class-smcm-ai}
	  %}
    \caption{Boxplots illustrating the range of variability in Sierran Mixed Conifer - Mesic, Early Development across future climate trajectories. The dashed black bar represents the current condition. Boxplot whiskers extend from the $5^{\text{th}} - 95^{\text{th}}$ percentile range of variability for each model.}
	  \label{fig:fragclass-smcm}
	\end{figure} %fragland

	\begin{figure}[!htbp]
	 \captionsetup[subfigure]{labelformat=empty}
	  \centering
	  \subfloat[][]{
	    \centering
	    \includegraphics[width=0.5\textwidth]{/Users/mmallek/Documents/Thesis/Plots/fragclass-smcmetrics/SMC_X_EARLY_ALL_AREA_AM_boxplots.png}
	    \label{fig:boxplot-class-smcx-areaam}
	  }%
	  %\qquad
	  \subfloat[][]{
	    \includegraphics[width=0.5\textwidth]{/Users/mmallek/Documents/Thesis/Plots/fragclass-smcmetrics/SMC_X_EARLY_ALL_CLUMPY_boxplots.png}
	    \label{fig:boxplot-class-smcx-contag}
	  } \\
	    \subfloat[][]{
	    \includegraphics[width=0.5\textwidth]{/Users/mmallek/Documents/Thesis/Plots/fragclass-smcmetrics/SMC_X_EARLY_ALL_CORE_AM_boxplots.png}
	    \label{fig:boxplot-class-smcx-coream}
	  }
	    %\qquad
	    \subfloat[][]{
	    \includegraphics[width=0.5\textwidth]{/Users/mmallek/Documents/Thesis/Plots/fragclass-smcmetrics/SMC_X_EARLY_ALL_SHAPE_AM_boxplots.png}
	    \label{fig:boxplot-class-smcx-shapeam}
	} %\\
	  %  \subfloat[][]{
	  %  \includegraphics[width=0.5\textwidth]{/Users/mmallek/Documents/Thesis/Plots/fragclass-smcmetrics/SMC_X_EARLY_ALL_ECON_AM_boxplots.png}
	  %  \label{fig:boxplot-class-smcx-econam}
	  %}
	    %\qquad
	  %  \subfloat[][]{
	  %  \includegraphics[width=0.5\textwidth]{/Users/mmallek/Documents/Thesis/Plots/fragclass-smcmetrics/SMC_X_EARLY_ALL_AI_boxplots.png}
	  %  \label{fig:boxplot-class-smcx-ai}
	  %}
	    \caption{Boxplots illustrating the range of variability in Sierran Mixed Conifer - Xeric, Early Development across future climate trajectories. The dashed black bar represents the current condition. Boxplot whiskers extend from the $5^{\text{th}} - 95^{\text{th}}$ percentile range of variability for each model.}
	  \label{fig:fragclass-smcx}
	\end{figure} %fragland




\clearpage








%%%%%%%%%%%%%%%%%%%%%%%%%%%%%%%%%%%%%%%%%%%%%%%%%%%%%%%%%%%%%%%%%%%%%%%%%%%%%%%%%%%%%%%%%%%%%%
%%%%%%%%%%%%%%%%%%%%%%%%%%%%%%%%%%%%%%%%%%%%%%%%%%%%%%%%%%%%%%%%%%%%%%%%%%%%%%%%%%%%%%%%%%%%%%
%%%%%%%%%%%%%%%%%%%%%%%%%%%%%%%%%%%%%%%%%%%%%%%%%%%%%%%%%%%%%%%%%%%%%%%%%%%%%%%%%%%%%%%%%%%%%%
%%%%%%%%%%%%%%%%%%%%%%%%%%%%%%%%%%%%%%%%%%%%%%%%%%%%%%%%%%%%%%%%%%%%%%%%%%%%%%%%%%%%%%%%%%%%%%
%%%%%%%%%%%%%%%%%%%%%%%%%%%%%%%%%%%%%%%%%%%%%%%%%%%%%%%%%%%%%%%%%%%%%%%%%%%%%%%%%%%%%%%%%%%%%%
%%%%%%%%%%%%%%%%%%%%%%%%%%%%%%%%%%%%%%%%%%%%%%%%%%%%%%%%%%%%%%%%%%%%%%%%%%%%%%%%%%%%%%%%%%%%%%
%%%%%%%%%%%%%%%%%%%%%%%%%%%%%%%%%%%%%%%%%%%%%%%%%%%%%%%%%%%%%%%%%%%%%%%%%%%%%%%%%%%%%%%%%%%%%%


\section{Discussion}

\subsection{Future landscape dynamics of forests in the Yuba River watershed and comparison to current conditions}
We observed a slight to moderate increase in total area burned per timestep with increasing climate parameter sets. Although on its own this might indicate that northern Sierra Nevada forests are resilient to climate change and have stable outcomes, the total burned area results subsume the important finding that high severity fire dramatically increased relative to low severity fire, especially as the climate parameter increased. This finding was more pronounced for the xeric mixed conifer forests than for the mesic mixed conifer forests or the landscape as a whole, which was surprising because we expected the xeric forests to be resilient to an increase in high severity fire because of the ubiquity of low severity fire. Specifically, we found that results for xeric mixed conifer forests from the \textsc{ccsm1} model, which is similar to presettlement conditions, were a ratio of low to high mortality fire of about 2.6; results from the \textsc{gfdl-esm2m} model were a ratio of about 1.0. Our analysis of fire rotation confirms and illustrates this, making more clear the slight decrease in low mortality fire evident in results for the more extreme climate parameter sets.

From this result we expected to observe changes in the seral stage distribution, especially for xeric forests. We did, and they were dramatic. The proportion of early seral was strongly correlated with an increase in the climate parameter. In addition, open canopies became more prevalent and closed became less prevalent, which in our model can be attributed to an increase in fire. In the mesic mixed conifer forests, this decline was a clear trend following increasing climate parameter values, and losses of Late--Closed forest were correlated with gains in Early Development conditions. In the xeric mixed conifer forests, seral stages other than Early Development, Mid--Open, and Late--Open, were virtually absent.

The increased amount of fire also affects configuration metrics at the seral stage level. We focused on the Early Development stage in mesic and xeric mixed conifer forests because it experienced such a dramatic increase as the climate parameter increased and because the Forest Service is actively developing management practices for early seral habitats. In general the results differ from the current conditions; in most cases the current condition is fully departed from the 90\% range of variability in the future scenarios. Compared to the current landscape, early seral patches under our simulated future scenarios were larger, had larger core areas, were less fragmented, were more irregularly shaped, and had less edge contrast. We observe that the early xeric forests exhibit a stronger trend than the early mesic. Increasing the climate parameter has a more subtle effect on the configuration metrics compared to the seral stage distribution.

Our results imply that current trends of increased amounts of high mortality fire relative to low mortality fire are related to climate, and may be difficult to reverse. We observed a loss of structural diversity within the xeric forests, which shifted to a distribution composed almost entirely of Early Development or Open canopy cover forest. Mesic forests contained more structural complexity, but a large increase in Early Development comes at the loss of the Late--Closed and Late--Moderate stages, which is problematic because this cover type is a major source of late successional, closed canopy forest for the study area.

Because this study relied on the use of computer models, the most appropriate use of the results is to help identify the most influential factors driving landscape change, implications of our simulated disturbance and succession regime, and areas where further research is needed to delineate key parameters. The comparisons made here consider the difference between the current and future disturbance regimes and landscape patterns, given a scenario in which natural fire regimes were allowed to occur. Since letting all fires burn naturally is not practicable, we note that the results do not provide a simple roadmap for restoration. However, they should provide insights into what landscape patterns may be resilient with climate change. In addition, the results indicate when restoration toward a historical regime, composition, or configuration appears likely to succeed under climate change, or whether the future is likely to significantly diverge from the past.

\subsection{Historic range of variability}
Prior to this study we completed a historical range of variability (HRV) analysis for the same project area (Chapter~\ref{ch:hrv}). The HRV analysis used the same parameter set as the FRV, but with climate parameters from the pre-European settlement period of 1550--1850, which is an appropriate reference period \citep{Safford2013}. The combination of results depicting the relationship between the current landscape, HRV, and FRV provides a mechanism for land managers to identify and prioritize management strategies for promoting resilient forests. In comparing results from the HRV analysis to the FRV analysis, we note that in general current conditions are departed from both the HRV and the FRV results. Often the HRV and FRV are comparable. In these cases, restoration toward the conditions represented by the range of variability analysis results should be evaluated for practicality, with specific implementation being done at a site-specific basis using additional local data to inform specific management actions.

\subsection{Implications for Restoration}

Restoration toward a resilient, somewhat stable ecology is often an important goal of resource managers. One obvious result of this study is that more fire occurs under natural conditions and future climate scenarios than under current and presumed historic conditions. The implications from our study are that a shift towards more fire and especially more high severity fire is likely.

A wide range of variability means that managers will have difficulty in maintaining a stable and predictable relationship between fire and the landscape. Public expectations and the capabilities of agencies will need to be carefully managed from a social perspective.

Species that rely on early seral conditions or open canopies will gain habitat in the future, thanks to the increase in the prevalence of these conditions. However, the associated reduction in late development conditions, especially the more closed canopies currently characteristic of mesic mixed conifer forests, will likely have negative ramifications for species dependent on that structural context, such as spotted owl and fisher.

One limitation of our model is that it did not predict a longer burning season. It may be possible to counteract some of the high severity fire occurrence by injecting more low severity fire through prescribed burning during the shoulder seasons and winter. This could dampen the effects of warmer and drier conditions. Variability density treatments designed to create fine-scale heterogeneity could also ameliorate some of the effects of more frequent fire \citep{Stephens2010,Knapp2012,North2012a}. The vulnerability of closed canopy forest should be an impetus to focus initial restoration and prescribed burning efforts in areas where these values can be protected. Of course, increased amounts of fire will pose challenges in a landscape with complex ownership patterns \citep{Stephens2013}. Excellent coordination at levels beyond the Tahoe National Forest or the Forest Service will be necessary to address this successfully. Federal land management agencies, the California Department of Forestry and Fire Protection (``CalFire''), and local governments are working together more and more to coordinate fire suppression of large wildfires that cross political boundaries.

As just mentioned, our results can be used to identify and prioritize management strategies. Since the Forest Service is directed to manage within natural range of variability, it is important to define what the natural range of variability is, as well as the desired condition, and recognize what is possible. Some of the trends observed in this analysis, such as increased proportions of Early Development, will likely occur without active management. Others will not occur without changes to fire suppression strategies and related fuels management efforts. Once desired conditions are defined, a comparison to the FRV from this study is warranted. Whether or not the desired condition aligns with the FRV will affect what kinds of management actions are taken.

Assessments of the effects of post-fire treatments of various kinds need to be continued and expanded in a systematic and rigorous way. Stand initiation patterns of the past, and the meteorological condition under which they took place, should be evaluated to understand whether similar conditions are likely in the future. Some researchers have called for special management of early seral habitat. For example \citet{Dellasala2014} call for not managing post-disturbance early seral vegetation. \citet{Swanson2011} suggest mapping and managing early seral communities as a unique cover type. However, given the high probability of more early successional habitat being created by fire and the realities of federal budgets and personnel needs, it seems unlikely that special attention is needed to ensure that some newly created patches of early seral habitat are allowed to succeed naturally. In addition, mapping and managing a transient cover type does not fit within most forest management frameworks, including ours; we assert that early seral forest is a condition within a cover type. Practicing no management at all under climate change is risky, given that current species assemblages may need help to re-establish, and the very real threat of invasive species \citep{Stephens2010}. That said, the large amount of early seral habitat projected to occur on the landscape also means that managers will have options when deciding where to implement restoration efforts that are designed to speed up succession or reduce susceptibility to subsequent burns; it is here that more research on the response to Burned Area Emergency Response and restoration treatments would be most useful.

\subsection{Implications for Planning}
Many National Forests will undergo Forest Plan revisions in the next decade. The 2012 Planning Rule instructs managers to manage for resilient conditions within a natural range of variability. While our results may be used as one potential range of variability, the fact is that our results are also the outcome of a model that simplified several aspects of the landscape, including ownership patterns and tolerable amounts of fire. For example, we did not model varying levels of fire suppression effort. Assuming the Forest Service and other stakeholders in the area do not view the amount of wildfire we simulated as tolerable, they will need to adopt various strategies to try and reduce the likelihood of frequent large fires, as well as the negative impacts of suppression. We outlined some managed methods for restoring disturbance above. In addition, stakeholders could choose to develop fire breaks in advance of wildfires, for example by clearing vegetation along roads \citep{Conard2003}. However, this has other ecological tradeoffs, especially from increased fragmentation \citep{Trombulak2000}. On a different note, if more frequent fire does reduce the quantity of old growth forests in the area, mechanical techniques in combination with prescribed fire could be used to mimic the structural complexity of old growth in much younger stands \citep{Franklin2002}. Because it is so productive and trees here grow so quickly, this portion of the Tahoe National Forest may be a good place to experiment with this strategy; trees grow large more quickly here than in other locations in the mixed conifer belt. 
% ask becky how to cite last sentence

The risks of type conversions are also pertinent to managers, and potentially one of the biggest risks of an increase in high severity fire as a result of climate change. While not explicitly explored in this study, it is predicted that cover type shifts and conversions are more likely to follow stand-replacing disturbances \citep{Stephens2013}. This risk will increase with climate change, and the additional increase in stand-replacing events suggested by the model indicates an interaction between climate change and high mortality fire that should be taken into account by managers planning restoration after fires, especially when selecting what species to plant or encourage \citep{Fule2008,Schwartz2015}. More research on specific large fires, such as the 2013 Rim Fire, should yield insights into how shifts in the fire regime may change spread and susceptibility patterns, which could in turn be used to update and improve our \textsc{RMLands} parameterization \citep{Lydersen2014}.

