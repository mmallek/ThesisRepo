% !TEX root = master.tex
\newpage
\section{Oak Woodland (OAK)}

\subsection*{General Information}

\subsubsection{Cover Type Overview}

\textbf{Oak Woodland (OAK)} 
\newline
Crosswalks
\begin{itemize}
	\item EVeg: Regional Dominance Type 1
	\begin{itemize}
		\item Gray Pine
		\item Blue Oak
		\item Valley Oak
	\end{itemize}

	\item LandFire BpS Model
	\begin{itemize}
		\item 0611140: California Lower Montane Blue Oak-Foothill Pine Woodland and Savanna
	\end{itemize}

	\item Presettlement Fire Regime Type
	\begin{itemize}
		\item Oak Woodland
	\end{itemize}
\end{itemize}

\noindent Reviewed by Becky Estes, Central Sierra Province Ecologist, USDA Forest Service

\subsubsection{Vegetation Description}
The Oak Woodland landcover type is characterized by savannas, woodlands, or forests of either monospecific or mixed stands of various oak species. \emph{Quercus douglasii}, \emph{Quercus lobata}, \emph{Quercus wislizenii}, and \emph{Quercus garryana} are the major dominants. In oak forests where mixtures of tree oak and conifer species exist \emph{Quercus kelloggii} and \emph{Quercus chrysolepis} occur along with \emph{Pinus sabiniana} (Allen-Diaz et al. 2007). 

Both \emph{Q. douglasii} and \emph{Q. lobata} are endemic to California. \emph{Q. lobata} are among the oldest and largest oaks in North America. Tree age can exceed 500 years. \emph{Q. douglasii} are relatively slow-growing, long-lived trees. On \emph{Q. douglasii-P. sabiniana} woodlands, \emph{P. sabiniana} is taller and dominates the overstory, but is shorter-lived (at approximately 80 years) than \emph{Q. douglasii} (150-250 years). \emph{Q. douglasii} is usually the more abundant of the two trees, but \emph{P. sabiniana} contributes as much basal area as \emph{Q. douglasii} (Allen-Diaz et al. 2007).

Typical vegetation is dominated by open oak savannah with relatively uniform mature trees at low densities (less than 40\% cover), where understory vegetation structure is a function of frequent surface fire that mediates woody plant development. In some instances and in some sites tree density will increase to 70\% or more, forming a relatively stable hardwood forest type subject to surface fires in the hardwood litter and rare stand replacement fire (LandFire 2007).

In riparian forests, associates include \emph{Platanus racemosa}, \emph{Juglans hindsii}, \emph{Acer negundo}, \emph{Populus fremontii}, \emph{Salix}, and \emph{Fraxinus latifolia}. In drier areas and open woodlands, shrubs usually clump together in open areas with full sun. Species may include \emph{Aesculus californica}, \emph{Ceanothus}, \emph{Arctostaphylos}, \emph{Rhamnus}, \emph{Toxicodendron diversilobum}, and \emph{Cercis occidentalis} (Allen-Diaz et al. 2007). The shrub layer is best developed along natural drainages, becoming insignificant in the uplands. Ground cover consists of a well-developed carpet of grasses and forbs (Ritter 1988b). Common forbs include \emph{Daucus}, \emph{Geranium}, \emph{Madia}, and \emph{Trifolium}. Most understory cover is created by annual grasses, including \emph{Bromus}, \emph{Lolium}, and \emph{Hordeum} (Allen-Diaz et al. 2007).

Oak recruitment is poor in many areas today, due to both natural and human causes. Many stands exist as groups of medium-to-large trees with few or no young oaks. There is concern that these woodlands may be slowly changing into savannas and grasslands as trees die and are not replaced. Mortality of oak saplings seems to be related to competition for moisture with grasses and forbs, wild and domestic animals feeding on acorns and seedlings, fire suppression, and flood control. Most recent work suggests that recruitment is limited not by reproduction, but by the establishment and survival of saplings (Allen-Diaz et al. 2007).


\subsubsection{Distribution}
Oak Woodland has a patchy distribution embedded in a matrix of agriculture, urban development, grasslands, riparian forests, and other conifer and oak woodland types. It occurs in a band along the western Sierra Nevada foothills, generally below 800 m (2642 feet) in elevation, although individual species described here are capable of surviving at higher elevations. In general, tree density is highest along natural drainages with deeper soils, and lower in uplands and on steeper slopes. The transition from savanna to woodland to forest is largely driven by soil, precipitation, and elevation (Allen-Diaz et al. 2007).

Soils in this type vary significantly, with different types conducive to the establishment of differing dominant tree species. \emph{Q. lobata} is best developed on deep, well-drained alluvial soils, usually in valley bottoms (Ritter 1988b). \emph{Q. wislizeni} becomes more abundant on steeper slopes, shallower soils, and at higher elevations. \emph{Q. douglasii} woodlands occur on a wide range of soils; however, they are often shallow, rocky, infertile, and well drained. The overstory ranges from sparsely scattered trees on poor sites to nearly closed canopies on good quality sites (Allen-Diaz et al. 2007, Ritter 1988a). \emph{Q. douglasii-P. sabiniana} woodlands are found on variety of generally well-drained parent materials, ranging from gravelly loam through stony clay loam. They occupy steeper, drier slopes with shallower and rockier soils than pure oak woodlands (Verner 1988). 


\subsection*{Disturbances}

\subsubsection{Wildfire}
An overstory dominated by deciduous hardwood species results in an herbaceous surface fuel complex dominating fuel/fire influences (LandFire 2007). Because of the long period of human habitation of oak woodlands, it is extremely difficult to define the ``natural'' fire regime. Lightning-caused fires certainly occurred in the past, but decades may pass between these events. Native Americans used fire in their stewardship of oak woodlands; however, it is difficult to document the frequency, intensity, and extent of burning by Native Americans. Some estimate the fire return interval (FRI) of that period to be around 25 years. The first European settlers continued to use fire as a management practice; burning intervals ranged from 8-15 years. Ranchers continued the practice through the 1950s, but since then fire suppression has emerged as the standard management policy (Allen-Diaz et al. 2007).  

The fire regime which produced this landcover type is thought to be frequent; mortality depends on vegetation vulnerability and wildfire intensity. Younger oaks are fire-sensitive and frequently killed by even low severity fires. However, they typically sprout post-disturbance. Older, decadent oaks are not likely to sprout after being damaged or killed by fire. Therefore, younger stands are more likely to regrow after fires and fire exclusion can have a significant effect on stand structure. \emph{P. sabiniana}’s regeneration is dependent on regeneration from seed, although it, too, is fire-adapted. It also grows faster than \emph{Q. douglasii} and is an important colonizer (Allen-Diaz et al. 2007). 

Estimates of fire rotations are available from the LandFire project and Mallek et al. (2013). The LandFire project’s published fire return intervals are based on a series of associated models created using the Vegetation Dynamics Development Tool (VDDT). In VDDT, fires are specified concurrently with the transition that follows them. For example, a replacement fire causes a transition to the early development stage. In the RMLands model, such fires are classified as high mortality. However, in VDDT mixed severity fires may cause a transition to early development, a transition to a more open seral stage, or no transition at all. In this case, we categorize the first example as a high mortality fire, and the second and third examples as a low mortality fire. Based on this approach, we calculated fire rotations and the probability of high mortality fire for each of the OAK seral stages (Table~\ref{tab:oakdesc_fire}). We computed the overall target fire rotation of 26 years based on values from Mallek et al. (2013). 




\begin{table}[]
\small
\centering
\caption{Fire rotations (years) and probability of high versus low mortality fires. Values were derived from BpS model 0610800 (LandFire 2007), Van de Water and Safford (2011), and Safford (pers. comm. 2013).}
\label{tab:oakdesc_fire}
\begin{tabular}{@{}lcc@{}}
\toprule
\textbf{Condition}         & \multicolumn{1}{l}{\textbf{Fire Rotation}} & \multicolumn{1}{l}{\textbf{\begin{tabular}[c]{@{}l@{}}Proportion \\ High Mortality\end{tabular}}} \\ \midrule
Target                      & 26            & n/a                           \\
Early Development - All     & 10            & 0.01                          \\
Mid Development - Closed    & 12            & 0.07                          \\
Mid Development - Moderate  & 10            & 0.06                          \\
Mid Development - Open      & 10            & 0.05                          \\
Late Development - Closed   & 25            & 0.5                           \\
Late Development - Moderate & 12            & 0.18                          \\
Late Development - Open     & 8             & 0.08       \\ \bottomrule
\end{tabular}
\end{table}

\subsubsection{Other Disturbance}
Other disturbances are not currently modeled, but may, depending on the seral stage affected and mortality levels, reset patches to early development, maintain existing seral stages, or shift/accelerate succession to a more open condition. 

\subsection*{Vegetation Seral Stages}
We recognize seven separate seral stages for OAK: Early Development (ED), Mid Development - Open Canopy Cover (MDO), Mid Development - Moderate Canopy Cover, Mid Development - Closed Canopy Cover (MDC), Late Development - Open Canopy Cover (LDO), Late Development - Moderate Canopy Cover (LDM), and Late Development - Closed Canopy Cover (LDC) (Figure~\ref{oak_transmodel}). Our seral stages are an alternative to ``successional'' classes that imply a linear progression of states and tend not to incorporate disturbance. The seral stages identified here are derived from a combination of successional processes and anthropogenic and natural disturbance, and are intended to represent a composition and structural condition that can be arrived at from multiple other conditions described for that landcover type. Thus our seral stages incorporate age, size, canopy cover, and vegetation composition. In general, the delineation of stages has originated from the LandFire biophysical setting model descriptive of a given landcover type; however, seral stages are not necessarily identical to the classes identified in those models.


\begin{figure}[htbp]
\centering
\includegraphics[width=0.8\textwidth]{/Users/mmallek/Documents/Thesis/statetransmodel/StateTransitionModel/7class.png}
\caption{State and Transition Model for Oak Woodland. Each dark grey box represents one of the seven seral stages for this landcover type. Each column of boxes represents a stage of development: early, middle, and late. Each row of boxes represents a different level of canopy cover: closed (70-100\%), moderate (40-70\%), and open (0-40\%). Transitions between states/seral stages may occur as a result of high mortality fire, low mortality fire, or succession. Specific pathways for each are denoted by the appropriate color line and arrow: red lines relate to high mortality fire, orange lines relate to low mortality fire, and green lines relate to natural succession.} 
\label{oak_transmodel}
\end{figure}

\paragraph{Early Development (ED)}

\paragraph{Description} Post-replacement sapling/regeneration phase. Largely a function of either early seral remaining in early seral due to replacement fire, or due to less common late seral replacement fire. Re-establishment can occur from basal resprouting or sexual reproduction, depending on composition, growth form, and seed dynamics. Patch size likely ranges from very small gap recruitment to areas approximately 100 acres. May include \emph{Q. douglasii}, \emph{Q. chrysolepis}, \emph{Q. garryana}, \emph{P. sabiniana}, and a variety of shrubs (LandFire 2007).


\paragraph{Succession Transition} In the absence of disturbance, patches in this seral stage will begin transitioning to MDM at 20 years at a rate of 0.6 per time step. At 60 years in ED, all remaining patches transition to MDM. On average, patches remain in early development for 28 years.

\paragraph{Wildfire Transition} High mortality wildfire (1\% of fires in this seral stage) recycles the patch through the ED seral stage. No transition occurs as a result of low mortality fire. 

\noindent\hrulefill


\paragraph{Mid Development - Open Canopy Cover (MDO)}

\paragraph{Description} Intermediate phase, older than 20 years. Sparse new recruitment of cohorts occurs in the later stages of this seral stage, leading to an open canopy. Periodic surface fire is relatively common, but replacement fire rare due to low intensity fire type and resilience of typical species to top kill. Patch size is typically in the hundreds of acres. May include \emph{Q. douglasii}, \emph{Q. chrysolepis}, \emph{Q. garryana}, \emph{P. sabiniana}, and a variety of shrubs (LandFire 2007).

\paragraph{Succession Transition} In the absence of stand-replacing disturbance, patches in this seral stage will begin transitioning to MDM at 15 years at a rate of 0.7 per time step. Succession to LDO begins after 40 years in a mid development stage. The rate of succession per time step is 0.7. At 70 years in MDO, all remaining patches transition to LDO. On average, patches remain in mid development for 47 years.

\paragraph{Wildfire Transition} High mortality wildfire (5\% of fires in this seral stage) recycles the patch through the ED seral stage. Low mortality fire (95\%) maintains the MDO seral stage and allows for succession to LDO.

\noindent\hrulefill

\paragraph{Mid Development - Moderate Canopy Cover (MDM)}

\paragraph{Description} Intermediate phase, older than 20 years. Some new recruitment of cohorts occurs in the later stages of this seral stage, resulting in moderate canopy cover. Periodic surface fire is relatively common, but replacement fire rare due to low intensity fire type and resilience of typical species to top kill. Patch size is typically in the hundreds of acres. May include \emph{Q. douglasii}, \emph{Q. chrysolepis}, \emph{Q. garryana}, \emph{P. sabiniana}, and a variety of shrubs (LandFire 2007).

\paragraph{Succession Transition} In the absence of stand-replacing disturbance, patches in this seral stage will begin transitioning to MDC at 15 years at a rate of 0.7 per time step. Succession to LDM begins after 40 years in a mid development stage. The rate of succession per time step is 0.7. At 70 years in MDM, all remaining patches transition to LDM. On average, patches remain in mid development for 47 years.

\paragraph{Wildfire Transition} High mortality wildfire (6\% of fires in this seral stage) recycles the patch through the ED seral stage. Low mortality fire (94\%) maintains the MDM seral stage and allows for succession to LDM.

\noindent\hrulefill

\paragraph{Mid Development - Closed Canopy Cover (MDC)}

\paragraph{Description} Intermediate phase, older than 20 years. Significant new recruitment of cohorts occurs in the later stages of this seral stage, resulting in a closed canopy. Periodic surface fire is relatively common, but replacement fire rare due to low intensity fire type and resilience of typical species to top kill. Patch size is typically in the hundreds of acres. May include \emph{Q. douglasii}, \emph{Q. chrysolepis}, \emph{Q. garryana}, \emph{P. sabiniana}, and a variety of shrubs (LandFire 2007).

\paragraph{Succession Transition} Succession to LDC begins after 40 years in a mid development stage. The rate of succession per time step is 0.7. At 70 years in a mid development stage, all remaining patches transition to LDC. On average, patches remain in mid development for 47 years.

\paragraph{Wildfire Transition} High mortality wildfire (7\% of fires in this seral stage) recycles the patch through the ED seral stage. Low mortality fire (93\%) maintains the MDC seral stage and allows for succession to LDC.

\noindent\hrulefill


\paragraph{Late Development - Open Canopy Cover (LDO)}

\paragraph{Description} Open woodland with mature oak and conifer trees. This seral stage is highly stable, as most fire is frequent, low severity fire acting as a maintenance agent. Tree density and canopy cover increase over time to relatively stable conditions. In some cases woody encroachment and increased tree density occurs under missed fire cycles. If P. sabiniana occurs, it quickly becomes very large. Some replacement fire occurs initiating secondary succession in the ED seral stage. Patch size in the hundreds, to possibly thousands, of acres. Canopy cover ranges from 11-40\%. May include \emph{Q. douglasii, Q. chrysolepis, Q. garryana, P. sabiniana}, and a variety of shrubs (LandFire 2007).

\paragraph{Succession Transition} In the absence of disturbance, patches in this seral stage will begin transitioning to LDM after 15 years at a rate of 0.7 per time step. 

\paragraph{Wildfire Transition} High mortality wildfire (8\% of fires in this seral stage) recycles the patch through the ED seral stage. Low mortality fire (92\%) maintains the LDO seral stage.

\noindent\hrulefill

\paragraph{Late Development - Moderate Canopy Cover (LDM)}

\paragraph{Description} Woodland with mature oak and conifer trees. This seral stage is fairly stable, as fire tends to be frequent, low severity fire acting as a maintenance agent. Tree density and canopy cover are increasing over time due to missed fire cycles or high productivity. Periodic surface fire is relatively common, but replacement fire is uncommon due to low intensity fire type and resilience of typical species to top kill. If \emph{P. sabiniana} occurs, it quickly becomes very large. Patch size is  in the hundreds of acres. Canopy cover ranges from 40-70\%. May include \emph{Q. douglasii}, \emph{Q. chrysolepis}, \emph{Q. garryana}, \emph{P. sabiniana}, and a variety of shrubs (LandFire 2007).

\paragraph{Succession Transition} In the absence of disturbance, patches in this seral stage will begin transitioning to LDC after 15 years at a rate of 0.7 per time step. 

\paragraph{Wildfire Transition} High mortality wildfire (18\% of fires in this seral stage) recycles the patch through the ED seral stage. Low mortality fire (82\%) opens the patch up to LDO 14\% of the time; otherwise, the patch remains in LDM.

\noindent\hrulefill

\paragraph{Late Development - Closed Canopy Cover (LDC)}

\paragraph{Description} Late seral stage arising from a rare period of no fire in the LDM seral stage for at least 15 years, allowing woody understory encroachment and higher tree density. If \emph{P. sabiniana} occurs, it quickly becomes very large. Fire that does not effect a change in seral stage is rare; low mortality fire is the normal pathway back to late development, open seral stages, while high mortality results in a return to early seral conditions. Patch size is likely in the tens of acres. May include \emph{Q. douglasii}, \emph{Q. chrysolepis}, \emph{Q. garryana}, \emph{P. sabiniana}, and a variety of shrubs. If the closed seral stage persists for decades and \emph{P. sabiniana} is present, it can begin to shade out the oak trees (LandFire 2007).

\paragraph{Succession Transition} In the absence of disturbance, patches in this seral stage will maintain.

\paragraph{Wildfire Transition} High mortality wildfire (50\% of fires in this seral stage) recycles the patch through the ED seral stage. Low mortality fire (50\%) opens the patch up to LDM.

\noindent\hrulefill

\clearpage
\subsection*{Seral Stage Classification}
\begin{table}[hbp]
\small
\centering
\caption{Classification of seral stage for OAK. Diameter at Breast Height (DBH) and Cover From Above (CFA) values taken from EVeg polygons. DBH categories are: null, 0-0.9'', 1-4.9'', 5-9.9'', 10-19.9'', 20-29.9'', 30''+. CFA categories are null, 0-10\%, 10-20\%, \dots , 90-100\%. Each row in the table below should be read with a boolean AND across each column.}
\label{oak_classification}
\begin{tabular}{@{}lrrrrr@{}}
\toprule
\textbf{\begin{tabular}[l]{@{}l@{}}Cover \\ Condition\end{tabular}} & \textbf{\begin{tabular}[r]{@{}r@{}}Overstory Tree \\ Diameter 1 \\ (DBH)\end{tabular}} & \textbf{\begin{tabular}[r]{@{}r@{}}Overstory Tree \\ Diameter 2 \\ (DBH)\end{tabular}} & \textbf{\begin{tabular}[r]{@{}r@{}}Total Tree\\ CFA (\%)\end{tabular}} & \textbf{\begin{tabular}[r]{@{}r@{}}Conifer \\ CFA (\%)\end{tabular}} & \textbf{\begin{tabular}[r]{@{}r@{}}Hardwood \\ CFA (\%)\end{tabular}} \\ \midrule
Early            & 0-4.9''         & any & any    & any    & any    \\
Mid Open         & 5-9.9''         & any & 0-40   & any    & any    \\
Mid Moderate     & 5-9.9''         & any & 40-70  & any    & any    \\
Mid Closed       & 5-9.9''         & any & 70-100 & any    & any    \\
Late Open        & 10''+           & any & 0-40   & any    & any    \\
Late Open        & 10''+           & any & null   & 0-40   & 0-40   \\
Late Moderate    & 10''+           & any & 40-70  & any    & any    \\
Late Moderate    & 10''+           & any & null   & 40-70  & 0-70   \\
Late Moderate    & 10''+           & any & null   & 0-70   & 40-70  \\
Late Closed      & 10''+           & any & 70-100 & any    & any    \\
Late Closed      & 10''+           & any & null   & 70-100 & any    \\
Late Closed      & 10''+           & any & null   & any    & 70-100 \\ \bottomrule
\end{tabular}
\end{table}


\clearpage

\subsection*{References}
\begin{hangparas}{.25in}{1} 
Allen-Diaz, Barbara, Richard Standiford, and Randall D. Jackson. ``Oak Woodlands and Forests.'' In \emph{Terrestrial Vegetation of California, 3rd Edition}, edited by Michael Barbour, Todd Keeler-Wolf, and Allan A. Schoenherr, 313-338. Berkeley and Los Angeles: University of California Press, 2007. 

``CalVeg Zone 1.'' Vegetation Descriptions. \emph{Vegetation Classification and Mapping}.  11 December 2008. U.S. Forest Service. \burl{http://www.fs.usda.gov/Internet/FSE_DOCUMENTS/fsbdev3_046448.pdf}. Accessed 2 April 2013.

LandFire. ``Biophysical Setting Models.'' Biophysical Setting 0611140: California Lower Montane Blue Oak-Foothill Pine Woodland and Savanna. 2007. LANDFIRE Project, U.S. Department of Agriculture, Forest Service; U.S. Department of the Interior. \burl{http://www.landfire.gov/national_veg_models_op2.php}. Accessed 9 November 2012.

Ritter, Lyman V. ``Blue Oak Woodland (BOW).'' \emph{A Guide to Wildlife Habitats of California}, edited by Kenneth E. Mayer and William F. Laudenslayer. California Deparment of Fish and Game, 1988a. \burl{http://www.dfg.ca.gov/biogeodata/cwhr/pdfs/BOW.pdf}. Accessed 4 December 2012.

Ritter, Lyman V. ``Valley Oak Woodland (VOW).'' \emph{A Guide to Wildlife Habitats of California}, edited by Kenneth E. Mayer and William F. Laudenslayer. California Deparment of Fish and Game, 1988b. \burl{http://www.dfg.ca.gov/biogeodata/cwhr/pdfs/VOW.pdf}. Accessed 4 December 2012.

Skinner, Carl N. and Chi-Ru Chang. ``Fire Regimes, Past and Present.'' \emph{Sierra Nevada Ecosystem Project: Final report to Congress, vol. II, Assessments and scientific basis for management options}. Davis: University of California, Centers for Water and Wildland Resources, 1996.

Van de Water, Kip M. and Hugh D. Safford. ``A Summary of Fire Frequency Estimates for California Vegetation Before Euro-American Settlement.'' \emph{Fire Ecology} 7.3 (2011): 26-57. doi: 10.4996/fireecology.0703026.

Verner, Jared. ``Blue Oak-Foothill Pine (BOP).'' \emph{A Guide to Wildlife Habitats of California}, edited by Kenneth E. Mayer and William F. Laudenslayer. California Deparment of Fish and Game, 1988. \burl{http://www.dfg.ca.gov/biogeodata/cwhr/pdfs/BOP.pdf}. Accessed 4 December 2012. 

\end{hangparas}

