% !TEX root = master.tex
\newpage
\section{Mixed Evergreen Forest (MEG)}

\subsection*{General Information}

\subsubsection{Cover Type Overview}

\textbf{Mixed Evergreen Forest (MEG)}
\newline
Crosswalks
\begin{itemize}
	\item EVeg: Regional Dominance Type 1
	\begin{itemize}
		\item Interior Mixed Hardwood
		\item California Bay
		\item Canyon Live Oak
		\item Madrone
		\item Bigleaf Maple
		\item Interior Live Oak
		\item Montane Mixed Hardwood 
		\item Pacific Douglas Fir
		\item Tanoak
	\end{itemize}

	\item EVeg: Regional Dominance Type 2
	\begin{itemize}
		\item Tanoak (regardless of RD Type 1 value, and therefore inclusive of all potential Type 1 vegetation types)
	\end{itemize}

	\item LandFire BpS Model
	\begin{itemize}
		\item 0610430 Mediterranean California Mixed Evergreen Forest
	\end{itemize}

	\item Presettlement Fire Regime Type
	\begin{itemize}
		\item Mixed Evergreen Forest
	\end{itemize}
\end{itemize}

\paragraph{Mesic Modifier (MEG\_M)}
This type is created by intersecting a binary xeric/mesic layer with the existing vegetation layer. MEG cells that intersect with mesic cells are assigned to the mesic modifier.
\paragraph{Xeric Modifier (MEG\_X)}
This type is created by intersecting a binary xeric/mesic layer with the existing vegetation layer. MEG cells that intersect with xeric cells are assigned to the xeric modifier.
\paragraph{Ultramafic Modifier (MEG\_U)}
This type is created by intersecting an ultramafic soils/geology layer with the existing vegetation layer. Where ultramafic cells intersect with MEG they are assigned to the ultramafic modifier.



\noindent Reviewed by Kyle Merriam, Sierra-Cascade Province Ecologist, USDA Forest Service; Becky Estes, Central Sierra Province Ecologist, USDA Forest Service


\subsubsection{Vegetation Description}
\paragraph{Mixed Evergreen Forest (MEG)} 	This landcover type forms a complex mosaic of forest due to the geologic, topographic, and successional variation typical within its range. This type is characterized by a combination of coniferous and broadleaved trees. Characteristic trees include \emph{Pseudotsuga menziesii}, \emph{Quercus chrysolepis}, \emph{Notholithocarpus densiflorus},\footnote{Tan oak was known as \emph{Lithocarpus densiflorus} for over 90 years before botanists renamed it \emph{Notholithocarpus densiflorus} in 2008 (Manos et al. 2008). Some sources and database continue to use the old name and plant symbol.}  \emph{Arbutus menziesii}, \emph{Umbellularia californica}, and \emph{Chrysolepis chrysophylla}. Species composition is primarily determined by the environmental gradients of temperature and moisture availability. \emph{Quercus kelloggii} is found on drier sites on inland portion of the range. \emph{Pinus lambertiana} and \emph{Pinus ponderosa} can be present in this type. These stands tend to have dense or diverse shrub understories with \emph{Ceanothus}, \emph{Corylus}, Gaultheria, Morella, Rhododendron, Ribes, Rubus, Toxicodendron diversilobum, and Vaccinium. Grass species include Bromus, Festuca, and Hierochloe. Polystichum \emph{munitum} and \emph{Pteridium aquilinum} var. \emph{pubescens} sometimes grow abundantly. \emph{Carex} spp. are present in some places (LandFire 2007, McDonald 1988, Tappeiner 1990).

\begin{adjustwidth}{2cm}{}
\textbf{Mesic Modifier (MEG\_M)}
Deep mesic soils support aggregations that include a lower or midstory layer of dense, sclerophyllous, broad-leaved evergreen trees like \emph{N. densiflorus} and \emph{Arbutus menziesii}, with an irregular, often open, higher layer of tall needle-leaved evergreen trees, typically \emph{P. menziesii}. A small number of pole and sapling trees occur throughout stands. On wetter sites, shrub layers are well developed, often with 100\% cover. Cover of the herbaceous layer under the shrubs can be up to 10 percent. At higher elevations, the shrubs disappear and the herb layer is often 100\%. Diversity of tree size typically increases with stand age, along with tree spacing. Young stands have closely spaced and uniformly distributed trees, whereas older stands have a more patchy stem distribution. Snags and downed logs, an important structural component of this habitat, increase in density or volume with stand age (Raphael 1988). Potential additional conifer associates include \emph{Abies concolor}, \emph{Pinus lambertiana}, \emph{Calocedrus decurrens}, and \emph{Pinus ponderosa} (Tappeiner 1990). A large variety of shrubs, forbs, grasses, sedges, and ferns, along with \emph{N. densiflorus} sprouts, can become aggressive on burned or cutover areas. This is especially true in areas where high severity fires have locally eliminated conifer seed sources (Tappeiner 1990).

\medskip
\noindent \textbf{Xeric Modifier (MEG\_X)}
A pronounced hardwood tree layer is typical, with an infrequent and poorly developed shrub stratum, and a sparse herbaceous layer (McDonald 1988). Characteristic oaks include \emph{Q. chrysolepis}, \emph{Q. wislizeni}, \emph{Q. kelloggi}, and \emph{Quercus garryana}. \emph{ Q. chrysolepis} and \emph{Q. wislizeni} are the most common oaks in the project area. They may individually form almost pure stands on steep canyon slopes and rocky ridgetops throughout the Sierra Nevada, or co-occur. They have tremendously variable growth forms, ranging from shrubs with multiple trunks on rocky, steep slopes, to magnificently spreading tall trees on deeper soils in moister areas. Both are evergreen with dense canopies (Allen-Diaz et al. 2007). Tree spacing is close (3-4 m) on better sites, and wider (8-10 m) on poor sites. In general, snags and downed woody material are sparse. Lower elevation associates are \emph{Pinus sabiniana}, \emph{Pinus attenuata}, \emph{N. densiflorus}, \emph{A. menziesii}, \emph{Quercus wislizeni}, \emph{C. chrysophylla}, and scrubby \emph{U. californica} (McDonald 1988).

\medskip
\noindent \textbf{Ultramafic Modifier (MEG\_U)}
\emph{Notholithocarpus densiflorus} var. \emph{echinoides}, or dwarf tanoak, grows on ultramafic and other less productive sites (Estes 2013). It is unclear if the 2 varieties differ genetically or if the small stature of dwarf tanoak is due to unproductive site conditions. Ecology literature does not usually distinguish between the 2 infrataxa (Fryer 2008). However, its identification is pertinent to management decisions. While \emph{N. lithocarpus} is generally protected as an oak species, the dwarf variety may be classified as a shrub and therefore subject to treatment or removal. Typically, \emph{P. menziesii} attains less dominance and may replaced by open stands of various conifers, such as \emph{Pinus ponderosa}, \emph{Pinus sabiniana}, or \emph{Pinus jeffreyi}. Trees occur within a generally open grassland or shrubland. The shrub layer is likely to include \emph{Quercus vaccinifolia}, \emph{N. densiflorus}, \emph{U. californica}, \emph{Quercus breweri}, and \emph{Rhamnus}. Common grasses include \emph{Stipa}, \emph{Festuca}, and \emph{Danthonia} (LandFire 2007b, McDonald 1988, O'Geen et al. 2007, Raphael 1988). 

\end{adjustwidth}

\subsubsection{Distribution}
\paragraph{Mixed Evergreen Forest}		This highly variable cover type occurs in the Sierra Nevada on all aspects at elevations of 350 m (1150 ft) to over 1700 m (5575 ft) (LandFire 2007a). Soil depth classes range from shallow to deep. The large number of species in the type, both conifer and hardwood, allow it to occupy and persist in a wide range of environments. Good soils and poor, steep slopes and slight, frequently disturbed and pristine all are at least adequate habitats for one or more species (McDonald 1988).

A xeric-mesic gradient was developed based on four variables: 1) aspect, 2) potential evapotranspiration, 3) topographic wetness index, and 4) soil water storage. The variables were standardized by z-score such that higher values correspond to more mesic environments. Thus, potential evapotranspiration was inverted to maintain this balance. The four variables were combined with equal weights. This final variables was split into xeric vs. mesic, with xeric occupying the negative end of the range up to $\frac{1}{4}$ standard deviation below the mean (zero) and mesic occupying the remaining portion of the spectrum.

\begin{adjustwidth}{2cm}{}
\textbf{Mesic Modifier }
Soils are deep, well-drained, and loamy, sandy, or gravelly. Found in valleys, coves, ravines, along streams, and on north as well as east slopes. It typically occurs in areas that are cool and moist sites in areas where precipitation is highest most likely in the form of rain and snow.

\medskip
\noindent \textbf{Xeric Modifier}
Q. chrysolepis and associates are found on a wide range of slopes, especially those that are moderate to steep. Soils are for the most part rocky, alluvial, coarse textured, poorly developed, and well drained. 

\medskip
\noindent \textbf{Ultramafic Modifier} Ultramafics have been mapped at various spatial densities throughout the elevational range of the landcover type. Low to moderate elevations in ultramafic and serpentinized areas often produce soils low in essential minerals like calcium potassium, and nitrogen, and have excessive accumulations of heavy metals such as nickel and chromium. These sites vary widely in the degree of serpentinization and effects on their overlying plant communities (``CalVeg Zone 1'' 2011). Note, the terms ``ultramafic rock'' and ``serpentine'' are broad terms used to describe a number of different but related rock types, including serpentinite, peridotite, dunite, pyroxenite, talc and soapstone, among others (O'Geen et al. 2007). 

\end{adjustwidth}

%%%

\subsection*{Disturbances}
\subsubsection{Wildfire}

\paragraph{Mixed Evergreen Forest}		Fire is the dominant disturbance event. Wildfires are common and frequent; mortality depends on vegetation vulnerability and wildfire intensity. Low mortality fires kill small trees and may consume above-ground portions of small oaks, shrubs and herbs, but do not kill large trees or below-ground organs of most oaks, shrubs and herbs which promptly resprout. High mortality fires kill trees of all sizes and may kill many of the shrubs and herbs as well. However, high mortality fires typically kill only the above ground portions of the oaks, shrubs and herbs; consequently, most oaks, shrubs and herbs promptly resprout from surviving below ground organs.

The vast majority of fires occur in late summer or early fall and are associated with lightning storms. Native American burns locally increased the frequency and may have been extensive prior to 1850. However, research also suggests that fire frequencies actually increased after European settlement (Merriam, pers. comm. 2013). Fires in the past were often large in area due to the high number of ignition points associated with fire events, and created patches of varying age and species composition (LandFire 2007a). 

Hardwoods typically provide the greatest cover after fire due to root-crown sprouting. Depending upon fire severity many hardwoods may have epicormic sprouting well into the crown. Species composition, density and interspecific competition within stands contributes to multiple pathways following disturbance. If fire has been absent from an area for an extended period of time, some conifers may be able to establish and persist even with the return of frequent low severity fire. But, if low severity fire is frequent after a stand-replacing fire, conifers will be more or less excluded and hardwoods will dominate (LandFire 2007a).

Estimates of fire rotations for these variants are available from the LandFire project and a few review papers. The LandFire project’s published fire return intervals are based on a series of associated models created using the Vegetation Dynamics Development Tool (VDDT). In VDDT, fires are specified concurrently with the transition that follows them. For example, a replacement fire causes a transition to the early development stage. In the RMLands model, such fires are classified as high mortality. However, in VDDT mixed severity fires may cause a transition to early development, a transition to a more open seral stage, or no transition at all. In this case, we categorize the first example as a high mortality fire, and the second and third examples as a low mortality fire. Based on this approach, we calculated fire rotations and the probability of high mortality fire for each of the MEG seral stages across the three variants (Tables~\ref{tab:megmdesc_fire}--\ref{tab:megudesc_fire}). We computed overall target fire rotations based on expert input from Safford and Estes, values from Mallek et al. (2013), and Van de Water and Safford (2011). 

\begin{adjustwidth}{2cm}{}
\textbf{Mesic Modifier }
\emph{N. densiflorus} is adapted to ignite easily. In the lower montane zone of the Sierra Nevada where \emph{N. densiflorus} occurs, the historic fire regime was characterized by dormant season fires of mostly low to moderate severity (Tappeiner 1990). In stands with high \emph{N. densiflorus} cover, \emph{N. densiflorus} may dominate the stand for many years before conifers re-establish. Patchy, stand-replacement fires were most common on north-facing slopes and during extended droughts. \emph{N. densiflorus} seedlings and saplings are typically top-killed by even low severity surface fire. Large trees usually survive moderate-severity fire, bearing fire scars afterward. Even \emph{N. densiflorus} with thick bark (3-10 cm) typically sustain bole damage from fire. Relative to associated conifers, mature \emph{P. menziesii} is fairly resistant to surface fires. Crown fires cause extensive mortality (Tappeiner 1990).

\medskip
\noindent \textbf{Xeric Modifier} \emph{Q. chrysolepis} has loose, dead, flaky bark that catches fire readily and burns intensely. Occasional fire often changes a stand of \emph{Q. chrysolepis} to \emph{Q. wislizeni}-chaparral, but without fire for sufficient time, trees again develop. Where fire is frequent, this oak becomes scarce or even drops out of the montane hardwood community (McDonald 1988).

\medskip
\noindent \textbf{Ultramafic Modifier} Historically, these woodland types had frequent low-severity fire. However, now there is higher susceptibility to stand replacing fire because of fire exclusion.

\end{adjustwidth}

%%%


\begin{table}[]
\small
\centering
\caption{Fire rotation (years) and proportion of high (versus low) mortality fires for Mixed Evergreen Forest - Mesic. Values were derived from VDDT model 0610790 (LandFire 2007a) and Safford and Estes (personal communication). }
\label{tab:megmdesc_fire}
\begin{tabular}{@{}lcc@{}}
\toprule
\textbf{Condition}         & \multicolumn{1}{l}{\textbf{Fire Rotation}} & \multicolumn{1}{l}{\textbf{\begin{tabular}[c]{@{}l@{}}Proportion \\ High Mortality\end{tabular}}} \\ \midrule
Target                      & 50            & n/a                           \\
Early Development - All     & 68            & 1                             \\
Mid Development - Closed    & 46            & 0.11                          \\
Mid Development - Moderate  & 26            & 0.11                          \\
Mid Development - Open      & 18            & 0.11                          \\
Late Development - Closed   & 44            & 0.21                          \\
Late Development - Moderate & 25            & 0.11                          \\
Late Development - Open     & 17            & 0.11   						\\ \bottomrule
\end{tabular}
\end{table}

\begin{table}[]
\small
\centering
\caption{Fire rotation (years) and proportion of high (versus low) mortality fires for Mixed Evergreen Forest - Xeric. Values were derived from VDDT model 0610790 (LandFire 2007a), and Safford and Estes (personal communication). }
\label{tab:megxdesc_fire}
\begin{tabular}{@{}lcc@{}}
\toprule
\textbf{Condition}         & \multicolumn{1}{l}{\textbf{Fire Rotation}} & \multicolumn{1}{l}{\textbf{\begin{tabular}[c]{@{}l@{}}Proportion \\ High Mortality\end{tabular}}} \\ \midrule
Target                      & 40            & n/a     \\
Early Development - All     & 85            & 1       \\
Mid Development - Closed    & 39            & 0.10    \\
Mid Development - Moderate  & 22            & 0.10    \\
Mid Development - Open      & 15            & 0.10    \\
Late Development - Closed   & 37            & 0.10    \\
Late Development - Moderate & 21            & 0.10    \\
Late Development - Open     & 15            & 0.03 	  \\ \bottomrule
\end{tabular}
\end{table}

\begin{table}[]
\small
\centering
\caption{Fire rotation (years) and proportion of high (versus low) mortality fires for Mixed Evergreen Forest - Ultramafic. Values were derived from VDDT model 0711700 (LandFire 2007b), and Safford and Estes (personal communication). }
\label{tab:megudesc_fire}
\begin{tabular}{@{}lcc@{}}
\toprule
\textbf{Condition}         & \multicolumn{1}{l}{\textbf{Fire Rotation}} & \multicolumn{1}{l}{\textbf{\begin{tabular}[c]{@{}l@{}}Probability of \\ High Mortality\end{tabular}}} \\ \midrule
Target                      & 50            & n/a                           \\
Early Development - All     & 68            & 1                             \\
Mid Development - Closed    & 46            & 0.11                          \\
Mid Development - Moderate  & 26            & 0.11                          \\
Mid Development - Open      & 18            & 0.11                          \\
Late Development - Closed   & 44            & 0.21                          \\
Late Development - Moderate & 25            & 0.11                          \\
Late Development - Open     & 17            & 0.11   						\\ \bottomrule
\end{tabular}
\end{table}

%%%

\subsubsection{Other Disturbance}
Other disturbances are not currently modeled, but may, depending on the seral stage affected and mortality levels, reset patches to early development, maintain existing seral stages, or shift/accelerate succession to a more open seral stage. All of the tree species associated with this vegetation type are susceptible to a wide variety of pathogens and insects (such as sudden oak death for \emph{N. densiflorus}, which is caused by the pathogen \emph{Phytophthora ramorum}).

\subsection*{Vegetation Seral Stages}
We recognize seven separate seral stages for MEG: Early Development (ED), Mid Development - Open Canopy Cover (MDO), Mid Development - Moderate Canopy Cover, Mid Development - Closed Canopy Cover (MDC), Late Development - Open Canopy Cover (LDO), Late Development - Moderate Canopy Cover (LDM), and Late Development - Closed Canopy Cover (LDC) (Figure~\ref{meg_transmodel}). Our seral stages are an alternative to ``successional'' classes that imply a linear progression of states and tend not to incorporate disturbance. The seral stages identified here are derived from a combination of successional processes and anthropogenic and natural disturbance, and are intended to represent a composition and structural condition that can be arrived at from multiple other conditions described for that landcover type. Thus our seral stages incorporate age, size, canopy cover, and vegetation composition. In general, the delineation of stages has originated from the LandFire biophysical setting model descriptive of a given landcover type; however, seral stages are not necessarily identical to the classes identified in those models.

\begin{figure}[htbp]
\centering
\includegraphics[width=0.8\textwidth]{/Users/mmallek/Documents/Thesis/statetransmodel/StateTransitionModel/7class.png}
\caption{State and Transition Model for Mixed Evergreen Forest. Each dark grey box represents one of the seven seral stages for this landcover type. Each column of boxes represents a stage of development: early, middle, and late. Each row of boxes represents a different level of canopy cover: closed (70-100\%), moderate (40-70\%), and open (0-40\%). Transitions between states/seral stages may occur as a result of high mortality fire, low mortality fire, or succession. Specific pathways for each are denoted by the appropriate color line and arrow: red lines relate to high mortality fire, orange lines relate to low mortality fire, and green lines relate to natural succession.} 
\label{meg_transmodel}
\end{figure}

\paragraph{Early Development (ED)}

\paragraph{Description} This seral stage is characterized by the diversity of species establishing and reestablishing into an open area created by a stand-replacing disturbance. 

\begin{adjustwidth}{2cm}{}
\textbf{Mesic Modifier } On mesic sites, abundant grasses, forbs, low shrubs, found under sparse to moderate cover of trees (primarily \emph{P. menziesii} and \emph{N. densiflorus}) seedlings/saplings with an open canopy. Seedling establishment of \emph{P. menziesii} following fire is dependent on the spacing and number of surviving seed trees. Seedling establishment following large stand-replacing fires may be slow if seed trees are killed over extensive areas. Or, if there are numerous, well-spaced surviving seed trees within the burned area, a new cohort of seedlings can quickly establish (Uchytil 1991). Nearly all \emph{N. densiflorus} burls sprout after fire, and survivorship is high. \emph{Q. chrysolepis}, if present, also sprouts readily, and shrubs such as \emph{Mahonia}, \emph{Gaultheria}, and \emph{Rhododendron} may be significant. Shrub growth from seed banks, e.g. \emph{Ceanothus integerrimus}, can also be high (LandFire 2007a). Thus, \emph{N. densiflorus} and other shrubs usually dominante the initial seral stage if \emph{P. menziesii} isn’t able to seed in quickly (Raphael 1988).

\medskip
\noindent \textbf{Xeric Modifier}  On xeric sites, grasses, forbs, low shrubs, and sparse cover of tree seedlings and saplings are found under an open canopy. Forest openings contain a dense cover of hardwood sprouts. Sprouting shrubs such as \emph{M. aquifolium}, \emph{Gaultheria shallon}, and \emph{Rhododendron} may be significant. Shrub growth from seed banks, e.g. \emph{Ceanothus integerrimus}, can also be high (LandFire 2007a). 


\medskip
\noindent \textbf{Ultramafic Modifier}  On ultramafic sites, \emph{P. menziesii} may be stunted and slow-growing, and \emph{N. densiflorus} var. \emph{echinoides} may be present. Grasses like \emph{Festuca}, \emph{Danthonia}, and \emph{Acnatherum}, or else chaparral shrubs establish. Scattered \emph{Pinus ponderosa}, \emph{Pinus sabiniana}, or \emph{Pinus jeffreyi} may also be present (LandFire 2007b).

\end{adjustwidth}

\paragraph{Succession Transition}
\begin{adjustwidth}{2cm}{}
\textbf{Mesic and Xeric Modifier } In the absence of disturbance, patches in this seral stage will begin transitioning to MDM at 20 years. The rate of succession per time step is 0.8. At 40 years, all patches will succeed. On average, patches remain in ED for 26 years.


\medskip
\noindent \textbf{Ultramafic Modifier} Succession may be delayed. Thus, in the absence of disturbance, patches in this seral stage will begin transitioning to MDM after 30 years and may be delayed in the ED seral stage for as long as 80 years. A patch in this seral stage succeeds at a rate of 0.4 per time step. On average, patches remain in ED for 43 years.

\end{adjustwidth}

\paragraph{Wildfire Transition} High mortality wildfire (100\% of fires in this seral stage) recycles the patch through the ED seral stage. Low mortality wildfire is not modeled for this seral stage.

\noindent\hrulefill


\paragraph{Mid Development - Open Canopy Cover (MDO)}

\paragraph{Description}
\begin{adjustwidth}{2cm}{}
\textbf{Mesic Modifier } Sparse ground cover of grasses, forbs, and shrubs; open tree canopy cover (primarily \emph{P. menziesii} and \emph{N. densiflorus}). Other \emph{Quercus} and \emph{Arctostaphylos} species may also be present. In this stage, hardwoods are dominant, but \emph{P. menziesii} and possibly other conifers are established or establishing under the predominantly \emph{N. densiflorus} canopy (LandFire 2007a, McDonald 1988). 


\medskip
\noindent \textbf{Xeric Modifier}  Sparse ground cover of grasses, forbs, and shrubs; open tree canopy cover, primarily hardwoods such as \emph{Q. chrysolepis} and \emph{Q. kelloggii}. Conifers such as \emph{P. menziesii} are present at low densities in emergent status. The shrub understory is still a significant presence (LandFire 2007a). 


\medskip
\noindent \textbf{Ultramafic Modifier}  Ultramafic sites are characterized by open \emph{P. menziesii}, \emph{Pinus ponderosa}, \emph{Pinus sabiniana}, or \emph{Pinus jeffreyi} stands with an understory comprised of \emph{N. densiflorus} var. \emph{echinoides} or \emph{Q. chrysolepis} as well as grasses, forbs, and shrubs (LandFire 2007b).

\end{adjustwidth}
\paragraph{Succession Transition}
\begin{adjustwidth}{2cm}{}
\textbf{Mesic and Xeric Modifier } Patches in this seral stage may stay in this seral stage under low mortality disturbance, but after 15 years without fire they begin transitioning to MDM at a rate of 0.8 per time step. After 20 years in a mid development stage, patches in this seral stage will begin transitioning to LDO. The rate of succession per time step is 0.8. At 40 years, all patches succeed. On average, patches remain in the mid development stage for 26 years.


\medskip
\noindent \textbf{Ultramafic Modifier}  Succession may be delayed. Thus, in the absence of low mortality disturbance, patches in the MDO seral stage will begin transitioning to MDM after 20 years in MDO at a rate of 0.7 per timestep. Patches in this seral stage will begin transitioning to LDO after 30 years in a mid development stage, and may be delayed in this stage for as long as 80 years. A patch in this seral stage succeeds at a rate of 0.4 per time step. On average, patches remain in the mid development stage for 43 years.

\end{adjustwidth}
\paragraph{Wildfire Transition}
\begin{adjustwidth}{2cm}{}
\textbf{Mesic Modifier } High mortality wildfire (11\% of fires in this seral stage) recycles the patch through the ED seral stage. Low mortality wildfire (89\%) does not effect a change in the MDO seral stage.


\medskip
\noindent \textbf{Xeric Modifier}  High mortality wildfire (10\% of fires in this seral stage) recycles the patch through the ED seral stage. Low mortality wildfire (90\%) does not effect a change in the MDO seral stage.


\medskip
\noindent \textbf{Ultramafic Modifier} High mortality wildfire (11\% of fires) recycles the patch through the ED seral stage. Low mortality wildfire (89\%) does not effect a change in the MDO seral stage.

\end{adjustwidth}

\noindent\hrulefill

\paragraph{Mid Development - Moderate Canopy Cover (MDM)}

\paragraph{Description}
\begin{adjustwidth}{2cm}{}
\textbf{Mesic Modifier } Sparse ground cover of grasses, forbs, and shrubs; moderate tree canopy cover (primarily \emph{P. menziesii} and \emph{ N. densiflorus}). Other \emph{Quercus} and \emph{Arctostaphylos} species may also be present. In this stage, hardwoods are dominant, but \emph{P. menziesii} and possibly other conifers are established or establishing under the predominantly \emph{N. densiflorus} canopy (LandFire 2007a, McDonald 1988). 


\medskip
\noindent \textbf{Xeric Modifier} Sparse ground cover of grasses, forbs, and shrubs; moderate tree canopy cover, primarily hardwoods such as \emph{Q. chrysolepis} and \emph{Q. kelloggii}. Conifers such as \emph{P. menziesii} are present at low densities in emergent status. The shrub understory is still a significant presence (LandFire 2007a). 


\medskip
\noindent \textbf{Ultramafic Modifier}  Ultramafic sites are characterized by open \emph{P. menziesii}, \emph{Pinus ponderosa}, \emph{Pinus sabiniana}, or \emph{Pinus jeffreyi} stands with an understory comprised of \emph{N. densiflorus} var. \emph{echinoides} or \emph{Q. chrysolepis} as well as grasses, forbs, and shrubs (LandFire 2007b).

\end{adjustwidth}
\paragraph{Succession Transition}
\begin{adjustwidth}{2cm}{}
\textbf{Mesic and Xeric Modifier } Patches in this seral stage may stay in this seral stage under low mortality disturbance, but after 15 years without fire they begin transitioning to MDC at a rate of 0.8 per time step. After 20 years in a mid development stage, patches in this seral stage will begin transitioning to LDM. The rate of succession per time step is 0.8. At 40 years, all patches succeed. On average, patches remain in the mid development stage for 26 years.


\medskip
\noindent \textbf{Ultramafic Modifier} Succession may be delayed. Thus, in the absence of low mortality disturbance, patches in the MDM seral stage will begin transitioning to MDC after 20 years in MDM at a rate of 0.7 per timestep. Patches in this seral stage will begin transitioning to LDM after 30 years in a mid development stage, and may be delayed in this stage for as long as 80 years. A patch in this seral stage succeeds at a rate of 0.4 per time step. On average, patches remain in the mid development stage for 43 years.

\end{adjustwidth}
\paragraph{Wildfire Transition}
\begin{adjustwidth}{2cm}{}
\textbf{Mesic Modifier } High mortality wildfire (11\% of fires in this seral stage) recycles the patch through the ED seral stage. Low mortality wildfire (89\%) triggers a transition to MDM 14\% of the time; otherwise, it remains in MDC.

\medskip
\noindent \textbf{Xeric Modifier} High mortality wildfire (10\% of fires in this seral stage) recycles the patch through the ED seral stage. Low mortality wildfire (90\%) triggers a transition to MDM 14\% of the time; otherwise, it remains in MDC.

\medskip
\noindent \textbf{Ultramafic Modifier} High mortality wildfire (11\% of fires) recycles the patch through the ED seral stage. Low mortality wildfire (89\%) triggers a transition to MDM 13\% of the time; otherwise, it remains in MDC.

\end{adjustwidth}
\noindent\hrulefill

\paragraph{Mid Development - Closed Canopy Cover (MDC)}

\paragraph{Description}
\begin{adjustwidth}{2cm}{}
\textbf{Mesic Modifier } Sparse ground cover of grasses, forbs, and shrubs; closed tree canopy cover (primarily \emph{P. menziesii} and \emph{N. densiflorus}). Other \emph{Quercus} and \emph{Arctostaphylos} species may also be present. In this stage, hardwoods are dominant, but \emph{P. menziesii} and possibly other conifers are established or establishing under the predominantly \emph{N. densiflorus} canopy (LandFire 2007a, McDonald 1988). 

\medskip
\noindent \textbf{Xeric Modifier} Sparse ground cover of grasses, forbs, and shrubs; closed tree canopy cover, primarily hardwoods such as \emph{Q. chrysolepis} and \emph{Q. kelloggii}. Conifers such as \emph{P. menziesii} are present at low densities in emergent status. The shrub understory is still a significant presence (LandFire 2007a). 

\medskip
\noindent \textbf{Ultramafic Modifier} Ultramafic sites are characterized by open \emph{P. menziesii}, \emph{Pinus ponderosa}, \emph{Pinus sabiniana}, or \emph{Pinus jeffreyi} stands with an understory comprised of \emph{N. densiflorus} var. \emph{echinoides} or \emph{Q. chrysolepis} as well as grasses, forbs, and shrubs (LandFire 2007b).

\end{adjustwidth}
\paragraph{Succession Transition}
\begin{adjustwidth}{2cm}{}
\textbf{Mesic and Xeric Modifier } After 20 years in a mid development stage, patches in this seral stage will begin transitioning to LDC. The rate of succession per time step is 0.8. At 40 years, all patches succeed. On average, patches remain in the mid development stage for 26 years.

\medskip
\noindent \textbf{Ultramafic Modifier} Succession may be delayed. Patches in this seral stage will begin transitioning to LDC after 30 years in a mid development stage, and may be delayed in this stage for as long as 80 years. A patch in this seral stage succeeds at a rate of 0.4 per time step. On average, patches remain in the mid development stage for 43 years.

\end{adjustwidth}
\paragraph{Wildfire Transition}
\begin{adjustwidth}{2cm}{}
\textbf{Mesic Modifier } High mortality wildfire (11\% of fires in this seral stage) recycles the patch through the ED seral stage. Low mortality wildfire (89\%) triggers a transition to MDM 22\% of the time; otherwise, it remains in MDC.

\medskip
\noindent \textbf{Xeric Modifier} High mortality wildfire (10\% of fires in this seral stage) recycles the patch through the ED seral stage. Low mortality wildfire (90\%) triggers a transition to MDM 20\% of the time; otherwise, it remains in MDC.

\medskip
\noindent \textbf{Ultramafic Modifier} High mortality wildfire (11\% of fires) recycles the patch through the ED seral stage. Low mortality wildfire (89\%) triggers a transition to MDM 22\% of the time; otherwise, it remains in MDC.

\end{adjustwidth}
\noindent\hrulefill


\paragraph{Late Development - Open Canopy Cover (LDO)}

\paragraph{Description}
\begin{adjustwidth}{2cm}{}
\textbf{Mesic Modifier } Overstory of large and very large trees, primarily \emph{P. menziesii}. Canopy cover less than 40\%. \emph{P. lambertiana} also occurs. \emph{N. densiflorus} is tolerant of both full sun and shade, and usually dominates the subcanopy at this stage. Co-dominance of the upper canopy with \emph{P. menziesii} is uncommon but possible after extended periods without disturbance (Uchytil 1991, LandFire 2007a). There is also some evidence that the senescence of late development \emph{N. densiflorus} may cause openings in the canopy and allow for continued \emph{P. menziesii} dominance (Estes pers. comm. 2013). \emph{Quercus} and \emph{Arctostaphylos} species may also be present in the sub-canopy (LandFire 2007a).

\medskip
\noindent \textbf{Xeric Modifier}  Overstory of large and very large trees, often with canopy cover less than 40\%. \emph{P. menziesii}, \emph{Q. chrysolepis}, and \emph{Arctostaphylos mewukka} may occur. Conifers are taller and larger than in MD and clearly form the upper canopy layer here. Shrubs persist in openings but those in shade are likely to begin senescing (LandFire 2007a). On ultramafic sites, large \emph{Pinus ponderosa} may additionally be present. Grass savannah persists on sites experiencing low intensity fire (with \emph{Festuca}, \emph{Achnatherum}, and \emph{Danthonia}). Where fire is less frequent, chaparral shrubland develops (with \emph{Arctostaphylos} and \emph{Quercus breweri}) (LandFire 2007b).

\medskip
\noindent \textbf{Ultramafic Modifier} On ultramafic sites, large \emph{Pinus ponderosa}, \emph{Pinus sabiniana}, or \emph{Pinus jeffreyi} may be present along with \emph{P. menziesii} and \emph{N. densiflorus} var. \emph{echinoides}. Grass savannah persists on sites experiencing low intensity fire (with \emph{Festuca}, \emph{Achnatherum}, and \emph{Danthonia}). Where fire is less frequent, chaparral shrubland develops (with \emph{Arctostaphylos} and \emph{Quercus breweri}) (LandFire 2007b).

\end{adjustwidth}
\paragraph{Succession Transition}
\begin{adjustwidth}{2cm}{}
\textbf{Mesic and Xeric Modifier } Patches in this seral stage may stay in this seral stage under low mortality disturbance, but after 15 years without fire they begin transitioning to LDM at a rate of 0.8 per time step. 

\medskip
\noindent \textbf{Ultramafic Modifier} Succession may be delayed. Thus, in the absence of low mortality disturbance, patches in the LDO seral stage will begin transitioning to LDM after 20 years in LDO at a rate of 0.7 per timestep. 

\end{adjustwidth}
\paragraph{Wildfire Transition}
\begin{adjustwidth}{2cm}{}
\textbf{Mesic Modifier } High mortality wildfire (11\% of fires in this seral stage) recycles the patch through the ED seral stage. Low mortality wildfire (89\%) does not effect a change in the MDO seral stage. 

\medskip
\noindent \textbf{Xeric Modifier} High mortality wildfire (3\% of fires in this seral stage) recycles the patch through the ED seral stage. Low mortality wildfire (97\%) does not effect a change in the MDO seral stage.

\medskip
\noindent \textbf{Ultramafic Modifier} High mortality wildfire (11\% of fires) recycles the patch through the ED seral stage. Low mortality wildfire (89\%) does not effect a change in the MDO seral stage.

\end{adjustwidth}
\noindent\hrulefill

\paragraph{Late Development - Moderate Canopy Cover (LDM)}

\paragraph{Description}
\begin{adjustwidth}{2cm}{}
\textbf{Mesic Modifier } Overstory of large and very large trees, primarily \emph{P. menziesii}. Canopy cover between 40\% and 60\%. \emph{P. lambertiana} also occurs. \emph{N. densiflorus} is tolerant of both full sun and shade, and usually dominates the subcanopy at this stage. Co-dominance of the upper canopy with \emph{P. menziesii} is uncommon but possible after extended periods without disturbance (Uchytil 1991, LandFire 2007a). There is also some evidence that the senescence of late development \emph{N. densiflorus} may cause openings in the canopy and allow for continued \emph{P. menziesii} dominance (Estes pers. comm. 2013). \emph{Quercus} and \emph{Arctostaphylos} species may also be present in the sub-canopy (LandFire 2007a).

\medskip
\noindent \textbf{Xeric Modifier} Overstory of large and very large trees, often with canopy cover between 40\% and 60\%. \emph{P. menziesii}, \emph{Q. chrysolepis}, and \emph{Arctostaphylos mewukka} may occur. Conifers are taller and larger than in MD and clearly form the upper canopy layer here. Shrubs persist in openings but those in shade are likely to begin senescing (LandFire 2007a). On ultramafic sites, large \emph{Pinus ponderosa} may additionally be present. Grass savannah persists on sites experiencing low intensity fire (with \emph{Festuca}, \emph{Achnatherum}, and \emph{Danthonia}). Where fire is less frequent, chaparral shrubland develops (with \emph{Arctostaphylos} and \emph{Quercus breweri}) (LandFire 2007b).

\medskip
\noindent \textbf{Ultramafic Modifier} On ultramafic sites, large \emph{Pinus ponderosa}, \emph{Pinus sabiniana}, or \emph{Pinus jeffreyi} may be present along with \emph{P. menziesii} and \emph{N. densiflorus} var. \emph{echinoides}. Grass savannah persists on sites experiencing low intensity fire (with \emph{Festuca}, \emph{Achnatherum}, and \emph{Danthonia}). Where fire is less frequent, chaparral shrubland develops (with \emph{Arctostaphylos} and \emph{Quercus breweri}) (LandFire 2007b).

\end{adjustwidth}
\paragraph{Succession Transition}
\begin{adjustwidth}{2cm}{}
\textbf{Mesic and Xeric Modifier } Patches in this seral stage may stay in this seral stage under low mortality disturbance, but after 15 years without fire they begin transitioning to LDC at a rate of 0.8 per time step. 

\medskip
\noindent \textbf{Ultramafic Modifier} Succession may be delayed. Thus, in the absence of low mortality disturbance, patches in the LDM seral stage will begin transitioning to LDC after 20 years in LDM at a rate of 0.7 per timestep. 

\end{adjustwidth}
\paragraph{Wildfire Transition}
\begin{adjustwidth}{2cm}{}
\textbf{Mesic Modifier } High mortality wildfire (11\% of fires in this seral stage) recycles the patch through the ED seral stage. Low mortality wildfire (89\%) triggers a transition to LDO 17\% of the time; otherwise, it remains in LDM.

\medskip
\noindent \textbf{Xeric Modifier} High mortality wildfire (10\% of fires in this seral stage) recycles the patch through the ED seral stage. Low mortality wildfire (90\%) triggers a transition to LDO 15\% of the time; otherwise, it remains in LDM.

\medskip
\noindent \textbf{Ultramafic Modifier}  High mortality wildfire (11\% of fires) recycles the patch through the ED seral stage. Low mortality wildfire (89\%) triggers a transition to LDO 17\% of the time; otherwise, it remains in LDM.

\end{adjustwidth}
\noindent\hrulefill

\paragraph{Late Development - Closed Canopy Cover (LDC)}

\paragraph{Description}
\begin{adjustwidth}{2cm}{}
\textbf{Mesic Modifier } Overstory of large and very large trees, primarily \emph{P. menziesii}. Canopy cover exceeds 70\%. \emph{P. lambertiana} also occurs. \emph{N. densiflorus} is tolerant of both full sun and shade, and usually dominates the subcanopy at this stage. Co-dominance of the upper canopy with \emph{P. menziesii} is uncommon but possible after extended periods without disturbance (Uchytil 1991, LandFire 2007a). There is also some evidence that the senescence of late development \emph{N. densiflorus} may cause openings in the canopy and allow for continued \emph{P. menziesii} dominance (Estes pers. comm. 2013). \emph{Quercus} and \emph{Arctostaphylos} species may also be present in the sub-canopy (LandFire 2007a).

\medskip
\noindent \textbf{Xeric Modifier} Overstory of large and very large trees, often with canopy cover over 70\%. \emph{P. menziesii}, \emph{Q. chrysolepis}, and \emph{Arctostaphylos mewukka} may occur. Conifers are taller and larger than in MD and clearly form the upper canopy layer here. Shrubs persist in openings but those in shade are likely to begin senescing (LandFire 2007a). On ultramafic sites, large \emph{Pinus ponderosa} may additionally be present. Grass savannah persists on sites experiencing low intensity fire (with \emph{Festuca}, \emph{Achnatherum}, and \emph{Danthonia}). Where fire is less frequent, chaparral shrubland develops (with \emph{Arctostaphylos} and \emph{Quercus breweri}) (LandFire 2007b).

\medskip
\noindent \textbf{Ultramafic Modifier} On ultramafic sites, large \emph{Pinus ponderosa}, \emph{Pinus sabiniana}, or \emph{Pinus jeffreyi} may be present along with \emph{P. menziesii} and \emph{N. densiflorus} var. \emph{echinoides}. Grass savannah persists on sites experiencing low intensity fire (with \emph{Festuca}, \emph{Achnatherum}, and \emph{Danthonia}). Where fire is less frequent, chaparral shrubland develops (with \emph{Arctostaphylos} and \emph{Quercus breweri}) (LandFire 2007b).

\end{adjustwidth}

\paragraph{Succession Transition}
\begin{adjustwidth}{2cm}{}
\textbf{Mesic, Xeric, and Ultramafic Modifier } In the absence of disturbance, patches in this seral stage will remain in this seral stage. 

\end{adjustwidth}

\paragraph{Wildfire Transition}
\begin{adjustwidth}{2cm}{}
\textbf{Mesic Modifier } High mortality wildfire (21\% of fires in this seral stage) recycles the patch through the ED seral stage. Low mortality wildfire (79\%) triggers a transition to LDM 26\% of the time; otherwise, it remains in LDC.

\medskip
\noindent \textbf{Xeric Modifier} High mortality wildfire (21\% of fires in this seral stage) recycles the patch through the ED seral stage. Low mortality wildfire (79\%) triggers a transition to LDM 24\% of the time; otherwise, it remains in LDC.

\medskip
\noindent \textbf{Ultramafic Modifier} High mortality wildfire (11\% of fires) recycles the patch through the ED seral stage. Low mortality wildfire (79\%) triggers a transition to LDM 26\% of the time; otherwise, it remains in LDC.

\end{adjustwidth}
\noindent\hrulefill

\newpage

\subsection*{Seral Stage Classification}
\begin{table}[hbp]
\small
\centering
\caption{Classification of seral stage for MEG. Diameter at Breast Height (DBH) and Cover From Above (CFA) values taken from EVeg polygons. DBH categories are: null, 0-0.9'', 1-4.9'', 5-9.9'', 10-19.9'', 20-29.9'', 30''+. CFA categories are null, 0-10\%, 10-20\%, \dots , 90-100\%. Each row in the table below should be read with a boolean AND across each column.}
\label{meg_classification}
\begin{tabular}{@{}lrrrrr@{}}
\toprule
\textbf{\begin{tabular}[l]{@{}l@{}}Cover \\ Condition\end{tabular}} & \textbf{\begin{tabular}[r]{@{}r@{}}Overstory Tree \\ Diameter 1 \\ (DBH)\end{tabular}} & \textbf{\begin{tabular}[r]{@{}r@{}}Overstory Tree \\ Diameter 2 \\ (DBH)\end{tabular}} & \textbf{\begin{tabular}[r]{@{}r@{}}Total Tree\\ CFA (\%)\end{tabular}} & \textbf{\begin{tabular}[r]{@{}r@{}}Conifer \\ CFA (\%)\end{tabular}} & \textbf{\begin{tabular}[r]{@{}r@{}}Hardwood \\ CFA (\%)\end{tabular}} \\ \midrule
Early All        & 0-4.9''         & any & any    & any & any \\
Mid Open         & 5-19.9''        & any & 0-40   & any & any \\
Mid Moderate     & 5-19.9''        & any & 40-70  & any & any \\
Mid Closed       & 5-19.9''        & any & 70-100 & any & any \\
Late Open        & 20-40''+        & any & 0-40   & any & any \\
Late Moderate    & 20-40''+        & any & 40-70  & any & any \\
Late Closed      & 20-40''+        & any & 70-100 & any & any \\ \bottomrule
\end{tabular}
\end{table}


\clearpage
\subsection*{References}
\begin{hangparas}{.25in}{1} 
Allen-Diaz, Barbara, Richard Standiford, and Randall D. Jackson. ``Oak Woodlands and Forests.'' In \emph{Terrestrial Vegetation of California, 3rd Edition}, edited by Michael Barbour, Todd Keeler-Wolf, and Allan A. Schoenherr, 313-338. Berkeley and Los Angeles: University of California Press, 2007. 

``CalVeg Zone 1.'' Vegetation Descriptions. \emph{Vegetation Classification and Mapping}.  11 December 2008. U.S. Forest Service. \burl{http://www.fs.usda.gov/Internet/FSE\_DOCUMENTS/fsbdev3\_046448.pdf}. Accessed 2 April 2013.

Estes, Becky, Province Ecologist, USDA Forest Service. Personal communication, 15 August 2013 and 3 September 2013.

LandFire. ``Biophysical Setting Models.'' Biophysical Setting 0610430: Mediterranean California Mixed Evergreen Forest. 2007a. LANDFIRE Project, U.S. Department of Agriculture, Forest Service; U.S. Department of the Interior. \burl{http://www.landfire.gov/national\_veg\_models\_op2.php}. Accessed 9 November 2012.

LandFire. ``Biophysical Setting Models.'' Biophysical Setting 0711700: Klamath-Siskiyou Xeromorphic Serpentine Savanna and Chaparral. 2007b. LANDFIRE Project, U.S. Department of Agriculture, Forest Service; U.S. Department of the Interior. \burl{http://www.landfire.gov/national\_veg\_models\_op2.php}. Accessed 30 November 2012.

Mallek, Chris, Hugh Safford, Joshua Viers, and Jay Miller. ``Modern departures in fire severity and area vary by forest type, Sierra Nevada and southern Cascades, California, USA.'' Ecosphere 4.12 (2013): art153. doi: http://www.esajournals.org/doi/pdf/10.1890/ES13-00217.1. 

Manos, P. S., C. H. Cannon, and S. H. Oh. ``Phylogenetic relationships and taxonomic status of the paleoendemic Fagaceae of western North America: recognition of a new genus, Notholithocarpus.'' Madroño 55.3 (2008): 181-190. doi: 10.3120/0024-9637-55.3.181

McDonald, Philip M. ``Montane Hardwood (MHW).'' \emph{A Guide to Wildlife Habitats of California}, edited by Kenneth E. Mayer and William F. Laudenslayer. California Deparment of Fish and Game, 1988. \burl{http://www.dfg.ca.gov/biogeodata/cwhr/pdfs/MHW.pdf}. Accessed 4 December 2012.

Merriam, Kyle. Province Ecologist, USDA Forest Service. Personal communication, 9 July 2013.

O'Geen, Anthony T., Randy A. Dahlgren, and Daniel Sanchez-Mata. ``California Soils and Examples of Ultramafic Vegetation.'' In \emph{Terrestrial Vegetation of California, 3rd Edition}, edited by Michael Barbour, Todd Keeler-Wolf, and Allan A. Schoenherr, 71-106. Berkeley and Los Angeles: University of California Press, 2007. 

Raphael, Martin G. ``Douglas-Fir (DFR).'' \emph{A Guide to Wildlife Habitats of California}, edited by Kenneth E. Mayer and William F. Laudenslayer. California Deparment of Fish and Game, 1988. \burl{http://www.dfg.ca.gov/biogeodata/cwhr/pdfs/DFR.pdf}. Accessed 4 December 2012.

Safford, Hugh. Regional Ecologist, USDA Forest Service. Personal communication, 15 August 2013.

Skinner, Carl N. and Chi-Ru Chang. ``Fire Regimes, Past and Present.'' \emph{Sierra Nevada Ecosystem Project: Final report to Congress, vol. II, Assessments and scientific basis for management options}. Davis: University of California, Centers for Water and Wildland Resources, 1996.

Tappeiner, John C., Philip M. McDonald, Douglass F. Roy. ``Tanoak.'' Silvics of North America: 2. Hardwoods. Agriculture Handbook 654. Burns, Russell M., and Barbara H. Honkala, tech. cords. U.S. Department of Agriculture, Forest Service, 1990. \burl{http://www.na.fs.fed.us/spfo/pubs/silvics\_manual/volume\_2/quercus/chrysolepis.htm}. Accessed 7 December 2012.

Uchytil, Ronald J. ``Pseudotsuga menziesii var. menziesii''.  \emph{Fire Effects Information System}, U.S. Department of Agriculture, Forest Service,  Rocky Mountain Research Station, Fire Sciences Laboratory, 1991. \burl{http://www.fs.fed.us/database/feis/plants/tree/quekel/all.html}. Accessed 21 December 2012.

Van de Water, Kip M. and Hugh D. Safford. ``A Summary of Fire Frequency Estimates for California Vegetation Before Euro-American Settlement.'' \emph{Fire Ecology} 7.3 (2011): 26-57. doi: 10.4996/fireecology.0703026.
\end{hangparas}
