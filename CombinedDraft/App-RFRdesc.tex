% !TEX root = master.tex
\newpage
\section{Red Fir (RFR)}
\label{rfr-description}

\subsection*{General Information}

\subsubsection{Cover Type Overview}

\textbf{Red Fir (RFR)}
\newline
Crosswalks
\begin{itemize}
	\item EVeg: Regional Dominance Type 1
	\begin{itemize}
		\item Red Fir
	\end{itemize}

	\item Presettlement Fire Regime Type
	\begin{itemize}
		\item Red Fir
	\end{itemize}
\end{itemize}


\noindent Modifiers \\
\medskip
\noindent \textbf{Mesic Modifier } This type is created by intersecting a binary xeric/mesic layer with the existing vegetation layer. RFR cells that intersect with mesic cells are assigned to the mesic modifier.
\begin{itemize}
	\item LandFire BpS Model
	\begin{itemize}
		\item 0610322 Mediterranean California Red Fir Forest – Southern Sierra
	\end{itemize}
\end{itemize}

\noindent \textbf{Xeric Modifier} This type is created by intersecting a binary xeric/mesic layer with the existing vegetation layer. RFR cells that intersect with xeric cells are assigned to the xeric modifier.
\begin{itemize}
	\item LandFire BpS Model
	\begin{itemize}
		\item 0610322 Mediterranean California Red Fir Forest – Southern Sierra
	\end{itemize}
\end{itemize}

\noindent \textbf{Ultramafic Modifier} This type is created by intersecting an ultramafic soils/geology layer with the existing vegetation layer. Where ultramafic cells intersect with RFR they are assigned to the ultramafic modifier.
\begin{itemize}
	\item LandFire BpS Model
	\begin{itemize}
		\item 0710220 Klamath-Siskiyou Upper Montane Serpentine Mixed Conifer Woodland
	\end{itemize}
\end{itemize}

\noindent \textbf{Red Fir with Aspen (RFR\_ASP)} This type is created by overlaying the NRIS TERRA Inventory of Aspen on top of the EVeg layer. Where it intersects with RFR it is assigned to RFR-ASP.

\noindent Reviewed by Marc Meyer, Southern Sierra Province Ecologist, USDA Forest Service

\subsubsection{Vegetation Description}
\textbf{Red Fir} The Red Fir landcover type is characterized by the presence of \emph{Abies magnifica}. Other conifer species such as \emph{Pinus monticola}, \emph{Pinus contorta} ssp. \emph{murrayana}, \emph{Tsuga mertensiana}, \emph{Abies concolor}, and \emph{Pinus jeffreyi} occur at varying densities (LandFire 2007a, LandFire 2007b). Mature \emph{A. magnifica} stands are frequently monotypic, with very few other plant species in any layer. Heavy shade and a thick layer of duff tends to inhibit understory vegetation, especially in dense stands (Barrett 1988). However, there are many open or patchy stands on less productive soils that are not monotypic, but rather codominant with other tree species. These sites may have substantial shrub cover (Meyer pers. comm.).

Stand-replacing disturbances such as lightning-caused fires, windthrows, insect outbreaks, and disease kill groups of trees (Barrett 1988). Stand structure is complex. Most current (fire-suppressed) \emph{A. magnifica} stands that were logged in the 19th century have an even-aged structure. In contrast, current unlogged and fire-suppressed stands have an uneven-aged or irregular age structure. Lastly, presettlement stands with an active fire regime had a relatively flat age-class structure that did not fit a classic even- or uneven-aged distribution (Meyer pers. comm. 2013). That is, frequent small-scale disturbance led to small patches of even-aged trees within the average ``stand,'' and most age classes in a given stand are represented by some of these small patches (Taylor and Halpern 1991). After fire, \emph{A. magnifica} seedlings may establish in canopy gaps, especially if they are small to moderate in size. \emph{P. contorta} ssp. \emph{murrayana}, as well as \emph{P. jeffreyi} and \emph{P. monticola}, may also function as post-fire pioneer species (Meyer pers. comm., Chappell and Agee 1996). On sites where these pioneering types occur under an \emph{A. magnifica} canopy, the \emph{A. magnifica} will dominate over the long-term (Cope 1993).

In openings resulting from tree mortality or logging, and under open stands on poor sites, many species may occur. Large shrubfields can dominate areas after severe fire, although conifers eventually will reclaim these sites. In some cases, particularly on xeric sites with significant shrub cover, reforestation can be effectively delayed for decades. \emph{Ribes}, \emph{Arctostaphylos}, and \emph{Ceanothus} are the most commonly found shrubs (Laacke 1990). Other associated shrubs include \emph{Symphoricarpos rotundifolius, Lonicera conjugialis}, and \emph{Quercus vaccinifolia} (Meyer pers. comm.). Associated herbaceous genera include \emph{Carex, Lupinus, Xerophyllum, Eucephalus, Pedicularis, Gayophytum, Pyrola} and \emph{Monardella} (Cope 1993).



\begin{adjustwidth}{2cm}{}
\noindent \textbf{Mesic Modifier } In addition to \emph{A. magnifica}, mesic regions within the RFR landcover type are associated with the presence of \emph{P. monticola} and \emph{P. contorta} ssp. \emph{murrayana}. \emph{T. mertensiana} may occur on northern aspects. \emph{A. concolor} is uncommon, except at lower elevations (LandFire 2007b).

\medskip
\noindent \textbf{Xeric Modifier}  These sites often include and are occasionally codominated by \emph{A. concolor}, \emph{P. jeffreyi}, and \emph{P. contorta} ssp. \emph{marayanna}, although other conifer species (e.g. \emph{P. lambertiana}) can also be present in lesser amounts at lower elevations. \emph{A. concolor} is more prevalent at lower elevations. \emph{P. jeffreyi} is more common on shallow soils or when disturbance is frequent. Shrubs and herbs generally contribute less than 30\% cover each. If shrub cover is higher, the shrubs are short or prostrate (LandFire 2007a).

\medskip
\noindent \textbf{Ultramafic Modifier} Ultramafic soils, support a number of endemic plant species. Slowly growing and often stunted \emph{P. contorta} ssp. \emph{murrayana} and \emph{P. jeffreyi} occur in combinations or in nearly pure open stands. \emph{A. magnifica} may be less dominant. Hardwoods are usually sparse, but shrubs such as \emph{Arctostaphylos}, \emph{Quercus}, \emph{Rhamnus, Lithocarpus, Rhododendron}, and \emph{Ceanothus} may occur on these sites. (``CalVeg Zone 1'' 2011)

\end{adjustwidth}

\noindent \textbf{Red Fir with Aspen} When \emph{Populus tremuloides} co-occurs with RFR on the west side of the Sierran crest, it is typically found in smaller patches, often less than 2 ha (5 acres) in size. This variant is not subject to the modifiers described above because it is only found on mesic sites with deeper soils. Mature stands in which \emph{P. tremuloides} are still dominant are usually relatively open. Average canopy closures range from 35-95\%. The open nature of the stands results in substantial light penetration to the ground (Meyer pers. comm., Verner 1988).



\subsubsection{Distribution}

\textbf{Red Fir} This cover type occupies the elevational band from about 1900 to 2750 m (6000 to 9000 ft). It is bounded and intergrades with Sierran Mixed Conifer at lower elevations. Geology is quite variable (Barrett 1988).

A xeric-mesic gradient was developed based on four variables: 1) aspect, 2) potential evapotranspiration, 3) topographic wetness index, and 4) soil water storage. The variables were standardized by z-score such that higher values correspond to more mesic environments. Thus, potential evapotranspiration was inverted to maintain this balance. The four variables were combined with equal weights. This final variables was split into xeric vs. mesic, with xeric occupying the negative end of the range up to $\frac{1}{4}$ standard deviation below the mean (zero) and mesic occupying the remaining portion of the spectrum.


\begin{adjustwidth}{2cm}{}
\medskip
\noindent \textbf{Mesic Modifier } These sites generally receive more moisture, either from precipitation, by virtue of being positioned on middle or lower slopes or drainage bottoms, or both. They may be adjacent to meadows or riparian areas. They are found at the highest elevations and north-facing aspects.

\medskip
\noindent \textbf{Xeric Modifier} These sites are typically drier and tend to occupy the lower portion of the RFR zone. They are also more likely to exist on south-facing aspects and steeper slopes.

\medskip
\noindent \textbf{Ultramafic Modifier} Ultramafic soils have been mapped at various spatial densities throughout the elevational range of the Red Fir landcover type. Low to moderate elevations in ultramafic and serpentinized areas often produce soils low in essential minerals such as calcium and magnesium or have excessive accumulations of heavy metals such as nickel and chromium. These sites vary widely in the degree of serpentization and effects on their overlying plant communities (``CalVeg Zone 1''). Note, the terms ``ultramafic rock'' and ``serpentine'' are broad terms used to describe a number of different but related rock types, including serpentinite, peridotite, dunite, pyroxenite, talc and soapstone, among others (O'Geen et al. 2007).

\end{adjustwidth}

\noindent \textbf{Red Fir with Aspen} Sites supporting \emph{P. tremuloides} are associated with added soil moisture, i.e., azonal wet sites. These sites are found throughout the RFR zone, often close to streams and lakes. Other sites include meadow edges, rock reservoirs, springs and seeps. Terrain can be simple to complex. At lower elevations, topographic conditions for this type tends toward positions resulting in relatively colder, wetter conditions within the prevailing climate, e.g., ravines, north slopes, wet depressions, etc. (LandFire 2007c). In general, these sites lie on lower slope positions, and are associated with slopes under 25\% (Potter 1998).

\subsection*{Disturbances}

\subsubsection{Wildfire}

\textbf{Red Fir} Fires in high-elevation \emph{A. magnifica} forests are generally not as intense as those in the Rocky Mountains and are typically less intense than those at lower elevations. Lesser annual fuel accumulation, less severe fire weather conditions, and compact and patchy fuels are all factors (Meyer pers. comm.). Still, fire has an important role in maintaining species diversity within these forests. Fire creates canopy openings by killing mature pioneer species such as \emph{P. contorta} ssp. \emph{murrayana} or \emph{P. jeffreyi} and some mature \emph{A. magnifica} (Cope 1993). 

Estimates of fire rotations for these variants are available from the LandFire project and a few review papers. The LandFire project’s published fire return intervals are based on a series of associated models created using the Vegetation Dynamics Development Tool (VDDT). In VDDT, fires are specified concurrently with the transition that follows them. For example, a replacement fire causes a transition to the early development stage. In the RMLands model, such fires are classified as high mortality. However, in VDDT mixed severity fires may cause a transition to early development, a transition to a more open seral stage, or no transition at all. In this case, we categorize the first example as a high mortality fire, and the second and third examples as a low mortality fire. Based on this approach, we calculated fire rotations and the probability of high mortality fire for each of the RFR seral stages across the three variants, as well as for the RFR\_ASP variant (Tables~\ref{tab:rfrmdesc_fire}--\ref{tab:rfr-aspdesc_fire}). We computed overall target fire rotations based on expert input from Safford and Estes, values from Mallek et al. (2013), and Van de Water and Safford (2011). 





\begin{adjustwidth}{2cm}{}
\medskip
\noindent \textbf{Mesic Modifier } Most fires occur during the late season during tree dormancy, fire complexity is moderate to high, and fire size averages 400 acres. It is very difficult to determine the replacement fire return interval. Replacement fire likely varies with slope position, and landscapes with greater topographic variation are likely to experience more stand replacement fires.

\medskip
\noindent \textbf{Xeric Modifier} Because of slow fuel accumulation rates, it is possible to have long gaps between surface fires in some seral stages. The discontinuous nature of the fuels limit extent of fires, and while fires may burn less often, they may burn at high severities. High intensity crown fires are uncommon.

\medskip
\noindent \textbf{Ultramafic Modifier} This type has a very limited distribution and consequently limited information for fire occurrence history. Low mortality fire is more common than high mortality fire. Most medium and high severity fire may actually occur on middle and upper slope positions.

\end{adjustwidth}

\noindent \textbf{Red Fir with Aspen} Sites supporting \emph{P. tremuloides} are maintained by stand-replacing disturbances that allow regeneration from below-ground suckers. Upland clones are impaired or suppressed by conifer ingrowth and overtopping and intensive grazing that inhibits growth. In a reference condition scenario, a few stands will advance toward conifer dominance. In the current landscape scenario, where fire has been reduced from reference conditions, there are many more conifer-dominated mixed aspen stands (LandFire 2007c, Verner 1988).


\begin{table}[]
\small
\centering
\caption{Fire rotation (years) and proportion of high (versus low) mortality fires for Red Fir – Mesic. Values were derived from VDDT model 0610322 (LandFire 2007b), Mallek et al. (2013), and Safford and Estes (personal communication).}
\label{tab:rfrmdesc_fire}
\begin{tabular}{@{}lcc@{}}
\toprule
\textbf{Condition}         & \multicolumn{1}{l}{\textbf{Fire Rotation}} & \multicolumn{1}{l}{\textbf{\begin{tabular}[c]{@{}l@{}}Proportion \\ High Mortality\end{tabular}}} \\ \midrule
Target                      & 60            & n/a                           \\
Early Development – All     & 58            & 1                             \\
Mid Development – Closed    & 55            & 0.35                          \\
Mid Development – Moderate  & 34            & 0.17                          \\
Mid Development – Open      & 25            & 0.09                          \\
Late Development – Closed   & 52            & 0.41                          \\
Late Development – Moderate & 32            & 0.16                          \\
Late Development – Open     & 23            & 0.05                  \\ \bottomrule
\end{tabular}
\end{table}

\begin{table}[]
\small
\centering
\caption{Fire rotation (years) and proportion of high (versus low) mortality fires for Red Fir – Xeric. Values were derived from VDDT model 0610321 (LandFire 2007a), and Safford and Estes (personal communication). }
\label{tab:rfrxdesc_fire}
\begin{tabular}{@{}lcc@{}}
\toprule
\textbf{Condition}         & \multicolumn{1}{l}{\textbf{Fire Rotation}} & \multicolumn{1}{l}{\textbf{\begin{tabular}[c]{@{}l@{}}Proportion \\ High Mortality\end{tabular}}} \\ \midrule
Target                      & 40            & n/a                           \\
Early Development – All     & 50            & 1                             \\
Mid Development – Closed    & 94            & 0.50                          \\
Mid Development – Moderate  & 65            & 0.25                          \\
Mid Development – Open      & 50            & 0.13                          \\
Late Development – Closed   & 74            & 0.38                          \\
Late Development – Moderate & 55            & 0.19                          \\
Late Development – Open     & 43            & 0.09                  \\ \bottomrule
\end{tabular}
\end{table}

\begin{table}[]
\small
\centering
\caption{Fire rotation (years) and proportion of high (versus low) mortality fires for Red Fir – Ultramafic. Values were derived from VDDT model 0610322 (LandFire 2007b), and Safford and Estes (personal communication). }
\label{tab:rfrudesc_fire}
\begin{tabular}{@{}lcc@{}}
\toprule
\textbf{Condition}         & \multicolumn{1}{l}{\textbf{Fire Rotation}} & \multicolumn{1}{l}{\textbf{\begin{tabular}[c]{@{}l@{}}Proportion \\ High Mortality\end{tabular}}} \\ \midrule
Target                      & 120           & n/a                           \\
Early Development – All     & 117           & 1                             \\
Mid Development – Closed    & 110           & 0.35                          \\
Mid Development – Moderate  & 69            & 0.17                          \\
Mid Development – Open      & 50            & 0.09                          \\
Late Development – Closed   & 104           & 0.41                          \\
Late Development – Moderate & 63            & 0.16                          \\
Late Development – Open     & 46            & 0.05                  \\ \bottomrule
\end{tabular}
\end{table}

\begin{table}[]
\small
\centering
\caption{Fire rotation (years) and proportion of high (versus low) mortality fires for Red Fir – Aspen type. Values were derived from VDDT model 0610610 (LandFire 2007) and Van de Water and Safford (pers. comm. 2013).}
\label{tab:rfr-aspdesc_fire}
\begin{tabular}{@{}lcc@{}}
\toprule
\textbf{Condition}         & \multicolumn{1}{l}{\textbf{Fire Rotation}} & \multicolumn{1}{l}{\textbf{\begin{tabular}[c]{@{}l@{}}Proportion \\ High Mortality\end{tabular}}} \\ \midrule
Target                           & 60            & n/a                           \\
Early Development – Aspen        & 58            & 0.03                          \\
Mid Development – Aspen          & 55            & 0.41                          \\
Mid Development – Aspen-Conifer  & 34            & 0.15                          \\
Late Development – Conifer-Aspen & 32            & 0.13                          \\
Late Development – Closed        & 52            & 0.26                  \\ \bottomrule
\end{tabular}
\end{table}

\subsubsection{Other Disturbance}
Other disturbances are not currently modeled, but may, depending on the seral stage affected and mortality levels, reset patches to early development, maintain existing seral stages, or shift/accelerate succession to a more open seral stage. 

\subsection*{Vegetation Seral Stages}
We recognize seven separate seral stages for RFR: Early Development (ED), Mid Development – Open Canopy Cover (MDO), Mid Development – Moderate Canopy Cover, Mid Development – Closed Canopy Cover (MDC), Late Development – Open Canopy Cover (LDO), Late Development – Moderate Canopy Cover (LDM), and Late Development – Closed Canopy Cover (LDC) (Figure~\ref{transmodel_rfr}). The RFR-ASP variant is also assigned to five seral stages: Early Development – Aspen (ED-A), Mid Development – Aspen (MD-A), Mid Development – Aspen with Conifer (MD-AC), Late Development Closed (LDC), and Late Development – Conifer with Aspen (LD-CA) (Figure~\ref{transmodel_rfr-asp}). 

Our seral stages are an alternative to ``successional'' classes that imply a linear progression of states and tend not to incorporate disturbance. The seral stages identified here are derived from a combination of successional processes and anthropogenic and natural disturbance, and are intended to represent a composition and structural condition that can be arrived at from multiple other conditions described for that landcover type. Thus our seral stages incorporate age, size, canopy cover, and vegetation composition. In general, the delineation of stages has originated from the LandFire biophysical setting model descriptive of a given landcover type; however, seral stages are not necessarily identical to the classes identified in those models.

\begin{figure}[htbp]
\centering
\includegraphics[width=0.8\textwidth]{/Users/mmallek/Documents/Thesis/statetransmodel/StateTransitionModel/7class.png}
\caption{State and Transition Model for Red Fir Forest (not inclusive of the aspen variant). Each dark grey box represents one of the seven seral stages for this landcover type. Each column of boxes represents a stage of development: early, middle, and late. Each row of boxes represents a different level of canopy cover: closed (70-100\%), moderate (40-70\%), and open (0-40\%). Transitions between states/seral stages may occur as a result of high mortality fire, low mortality fire, or succession. Specific pathways for each are denoted by the appropriate color line and arrow: red lines relate to high mortality fire, orange lines relate to low mortality fire, and green lines relate to natural succession.} 
\label{transmodel_rfr}
\end{figure}

\begin{figure}[htbp]
\centering
\includegraphics[width=0.8\textwidth]{/Users/mmallek/Documents/Thesis/statetransmodel/StateTransitionModel/5class-asp.png}
\caption{State and Transition Model for Red Fir Forest - Aspen variant. Each dark grey box represents one of the seven seral stages for this landcover type. Each column of boxes represents a stage of development: early, middle, and late. Transitions between states/seral stages may occur as a result of high mortality fire, low mortality fire, or succession. Specific pathways for each are denoted by the appropriate color line and arrow: red lines relate to high mortality fire, orange lines relate to low mortality fire, and green lines relate to natural succession.} 
\label{transmodel_rfr-asp}
\end{figure}

\subsection*{Red Fir}

\paragraph{Description}
\paragraph{Early Development (ED)} This seral stage is characterized by the recruitment of a new cohort of early successional, shade-intolerant tree species into an open area created by a stand-replacing disturbance. Conifer associates regenerate from seed. Occasionally, large brush fields may develop after hot wildfires and are dominated by \emph{Ceanothus, Arctostaphylos, Chrysolepsis}, or other shrub species for many years (Barrett 1988). On mesic sites, \emph{P. monticola} and \emph{P. contorta} ssp. \emph{murrayana} regenerate from seed. \emph{A. magnifica} comes in over time. Shrub cover is an important component; herb cover varies (LandFire 2007b). On xeric sites, there is regeneration of \emph{A. magnifica} and \emph{A. concolor}, perhaps \emph{P. jeffreyi} or \emph{P. lambertiana} from seed. Shrub and herb cover varies. (LandFire 2007a). Ultramafic sites will have similar species composition, especially at edges, but \emph{P. jeffreyi}, are relatively more common. Shrubs and herbs are sparse (O'Geen et al. 2007).

\paragraph{Succession Transition}

\begin{adjustwidth}{2cm}{}

\noindent \textbf{Mesic Modifier } In the absence of disturbance, patches in this seral stage will begin transitioning to MDC at age 30 at a rate of 0.6 per timestep. At 70 years, all stands will succeed to MDC. On average, patches remain in ED for 38 years.

\medskip
\noindent \textbf{Xeric Modifier}  Transition to mid development seral stages may be somewhat delayed. In the absence of disturbance, patches in this seral stage will begin transitioning to MDO at 50 years and may be delayed in the ED seral stage for as long as 150 years. A patch in this seral stage succeeds at a rate of 0.3 per timestep. On average, patches remain in ED for 67 years.

\medskip
\noindent \textbf{Ultramafic Modifier}  Transition to mid development seral stages may be substantially delayed. Thus, in the absence of disturbance, patches in this seral stage will begin transitioning to MDO after 80 years and may be delayed in the ED seral stage for as long as 150 years. A patch in this seral stage succeeds at a rate of 0.2 per timestep. On average, patches remain in ED for 105 years.

\end{adjustwidth}



\paragraph{Wildfire Transition} High mortality wildfire (100\% of fires in this seral stage) recycles the patch through the Early Development seral stage, regardless of soil type. Low mortality wildfire is not modeled for this seral stage.

\noindent\hrulefill


\paragraph{Mid Development – Open Canopy Cover (MDO)} 

\paragraph{Description} The pole/medium tree seral stage produces dense stands of young \emph{A. magnifica} that grow slowly with little mortality for many years (Barrett 1988). Cover of grasses, forms, and shrubs is on the decline as conifer canopy cover ranges from 10-40\%. \emph{A. magnifica} either is or is transitioning to become the dominant tree species. Canopy cover is less than 40\% (LandFire 2007a, LandFire 2007b).

On mesic sites, \emph{P. monticola} and \emph{P. contorta} ssp. \emph{murrayana} are present in varying amounts. Grasses, forbs, and shrubs are declining, although chaparral type shrubs, such as \emph{Arctostaphylos} or \emph{Chrysolepsis} can contribute to a dense understory. On xeric sites, \emph{A. concolor} and \emph{P. jeffreyi} are present in varying amounts, and shrub cover varies (LandFire 2007a, LandFire 2007b). Ultramafic sites will have similar species composition, especially at edges, but \emph{P. jeffreyi} is relatively more common (O'Geen et al. 2007).


\paragraph{Succession Transition}
\begin{adjustwidth}{2cm}{}

\noindent \textbf{Mesic Modifier } In the absence of low mortality disturbance, patches in the MDO seral stage will begin transitioning to MDM after 10 years at a rate of 0.22 per timestep. Succession to LDO takes place at 80 years since entering a middle development seral stage. 

\medskip
\noindent \textbf{Xeric Modifier} In the absence of low mortality disturbance, patches in the MDO seral stage will begin transitioning to MDM at 25 years at a rate of 0.2 per timestep. Succession to LDO takes place variably beginning at 80 years since transition to middle development at a rate of 0.6 per timestep. All patches succeed to a late seral stage by 100 years. On average, patches remain in MDM for 88 years.

\medskip
\noindent \textbf{Ultramafic Modifier} In the absence of low mortality disturbance, patches in the MDO seral stage will begin transitioning to MDM after 40 years at a rate of 0.1 per timestep. Succession to LDO takes place variably beginning at 120 years since transition to middle development at a rate of 0.3 per timestep, and all patches succeed by 180 years. On average, patches remain in ED for 117 years.

\end{adjustwidth}

\paragraph{Wildfire Transition}
\begin{adjustwidth}{2cm}{}
\noindent \textbf{Mesic Modifier } High mortality wildfire (9\% of fires in this seral stage) returns the patch to Early Development. Low mortality fire (91\%) maintains the MDO seral stage and allows for succession to LDO.

\medskip
\noindent \textbf{Xeric Modifier}  High mortality wildfire (13\% of fires in this seral stage) returns the patch to Early Development. Low mortality fire (87\%) maintains the MDO seral stage and allows for succession to LDO. 

\medskip
\noindent \textbf{Ultramafic Modifier}  High mortality wildfire (9\% of fires in this seral stage) returns the patch to Early Development. Low mortality fire (91\%) maintains the MDO seral stage and allows for succession to LDO.

\end{adjustwidth}

\noindent\hrulefill

\paragraph{Mid Development – Moderate Canopy Cover (MDM)}

\paragraph{Description} The pole/medium tree seral stage produces stands of young \emph{A. magnifica} with moderate canopy cover that grow slowly with little mortality for many years (Barrett 1988). Cover of grasses, forms, and shrubs is on the decline as conifer canopy cover exceeds 40\%. \emph{A. magnifica} either is or is transitioning to become the dominant tree species. On mesic sites, \emph{P. monticola} and \emph{P. contorta} ssp. \emph{murrayana} are present in varying amounts, while on xeric sites \emph{P. jeffreyi} and \emph{A. concolor} are associates (LandFire 2007a, LandFire 2007b). \emph{P. jeffreyi} is the most likely associate on ultramafic sites (O'Geen et al. 2007).

\paragraph{Succession Transition}
\begin{adjustwidth}{2cm}{}
\noindent \textbf{Mesic Modifier } In the absence of low mortality disturbance, patches in the MDM seral stage will begin transitioning to MDC after 10 years at a rate of 0.22 per timestep. Succession to LDM takes place at 80 years since entering a middle development seral stage. 

\medskip
\noindent \textbf{Xeric Modifier}  In the absence of low mortality disturbance, patches in the MDM seral stage will begin transitioning to MDC after 25 years at a rate of 0.2 per timestep. Succession to LDM begins at 80 years since entering a middle development stage at a rate of 0.65 per timestep. At 100 years after entering a middle development stage, all stands transition to LDM. 

\medskip
\noindent \textbf{Ultramafic Modifier} Transition to late seral seral stages may be substantially delayed. Thus, in the absence of low mortality disturbance, patches in the MDM seral stage will begin transitioning to MDC after 40 years at a rate of 0.1 per timestep. Succession to LDM begins at 100 years since entering a middle development stage at a rate of 0.3 per timestep. At 165 years after entering a middle development stage, all stands transition to LDM. 

\end{adjustwidth}

\paragraph{Wildfire Transition}
\begin{adjustwidth}{2cm}{}
\noindent \textbf{Mesic Modifier } High mortality wildfire (17\% of fires in this seral stage) returns the patch to ED. Low mortality wildfire (83\%) opens the stand up to MDO 13\% of the time; otherwise, the patch remains in MDC. 

\medskip
\noindent \textbf{Xeric Modifier}  High mortality wildfire (25\% of fires in this seral stage) returns the patch to ED. Low mortality wildfire (75\%) opens the stand up to MDO 13\% of the time; otherwise, the patch remains in MDC.

\medskip
\noindent \textbf{Ultramafic Modifier} High mortality wildfire (17\% of fires in this seral stage) returns the patch to ED. Low mortality wildfire (83\%) opens the stand up to MDO 13\% of the time; otherwise, the patch remains in MDC.

\end{adjustwidth}

\noindent\hrulefill

\paragraph{Mid Development – Closed Canopy Cover (MDC)}

\paragraph{Description} The pole/medium tree seral stage produces dense stands of young \emph{A. magnifica} that grow slowly with little mortality for many years (Barrett 1988). Cover of grasses, forms, and shrubs is on the decline as conifer canopy cover exceeds 40\%. \emph{A. magnifica} either is or is transitioning to become the dominant tree species. On mesic sites, \emph{P. monticola} and \emph{P. contorta} ssp. \emph{murrayana} are present in varying amounts, while on xeric sites \emph{P. jeffreyi} and \emph{A. concolor} are associates (LandFire 2007a, LandFire 2007b). \emph{P. jeffreyi} is the most likely associate on ultramafic sites (O'Geen et al. 2007).

\paragraph{Succession Transition}
\begin{adjustwidth}{2cm}{}
\noindent \textbf{Mesic Modifier } After 80 years in the mid development stage and in the absence of stand-replacing fire, all patches transition to LDC.

\medskip
\noindent \textbf{Xeric Modifier}  Transition to late seral seral stages may be delayed. Thus, in the absence of disturbance, patches in this seral stage will begin transitioning to LDC at 80 years in mid development at a rate of 0.7 per timestep and may be delayed in the MDC seral stage for up to 100 years.

\medskip
\noindent \textbf{Ultramafic Modifier} Transition to late seral seral stages may be substantially delayed. Thus, in the absence of disturbance, patches in this seral stage will begin transitioning to LDC at 80 years in the mid development stage at a rate of 0.3 per time step and may be delayed in the MDC seral stage for up to 150 years.

\end{adjustwidth}

\paragraph{Wildfire Transition}
\begin{adjustwidth}{2cm}{}
\noindent \textbf{Mesic Modifier } High mortality wildfire (35\% of fires in this seral stage) returns the patch to ED. Low mortality wildfire (65\%) opens the stand up to MDM 17\% of the time; otherwise, the patch remains in MDC. 

\medskip
\noindent \textbf{Xeric Modifier} High mortality wildfire (50\% of fires in this seral stage) returns the patch to ED. Low mortality wildfire (50\%) opens the stand up to MDM 17\% of the time; otherwise, the patch remains in MDC.

\medskip
\noindent \textbf{Ultramafic Modifier} High mortality wildfire (35\% of fires in this seral stage) returns the patch to ED. Low mortality wildfire (65\%) opens the stand up to MDM 17\% of the time; otherwise, the patch remains in MDC.

\end{adjustwidth}

\noindent\hrulefill


\paragraph{Late Development – Open Canopy Cover (LDO)}

\paragraph{Description} In the large tree seral stage, subdominant trees die and add to a growing layer of duff and downed woody material, and dominant trees continue to grow for several hundred years. \emph{A. magnifica} is the most common tree species. The understory of mature stands may be limited to less than 5\% cover (e.g. \emph{Chimaphila menziesii, Pyrola picta}). This seral stage develops when low mortality disturbance is fairly frequent; it persists as long as low mortality fires continue to occur periodically. \emph{Ceanothus} and \emph{Arctostaphylos} populate disturbance-generated gaps. Canopy cover is less than 40\% (LandFire 2007a, LandFire 2007b).

On mesic sites, \emph{P. monticola} and \emph{P. contorta} ssp. \emph{murrayana} may comprise up to 20\% of tree cover each. \emph{P. contorta} ssp. \emph{murrayana} acts as the pioneering conifer. On xeric sites, \emph{A. concolor} and \emph{P. jeffreyi} are the common associates and pioneer conifer species after disturbance (Barrett 1988, LandFire 2007a, LandFire 2007b). Ultramafic sites will have similar species composition, especially at edges, but \emph{P. jeffreyi} is relatively more common (O'Geen et al. 2007).


\paragraph{Succession Transition}
\begin{adjustwidth}{2cm}{}
\noindent \textbf{Mesic Modifier } In the presence of low mortality disturbance, patches in this seral stage can self-perpetuate, but after 10 years with no fire, patches in this seral stage will begin transitioning to LDM at a rate of 0.2 per timestep.

\medskip
\noindent \textbf{Xeric Modifier}  In the presence of low mortality disturbance, patches in this seral stage can self-perpetuate, but after 25 years with no fire, patches in this seral stage will begin transitioning to LDM at a rate of 0.2 per timestep.

\medskip
\noindent \textbf{Ultramafic Modifier} Patches occurring on ultramafic soils may succeed to LDM after 35 years with no fire, but the rate is just 0.2 per timestep.

\end{adjustwidth}

\paragraph{Wildfire Transition}
\begin{adjustwidth}{2cm}{}
\noindent \textbf{Mesic Modifier } High mortality wildfire (11\% of fires in this seral stage) returns the patch to early development. Low mortality wildfire (89\%) maintains LDO. 

\medskip
\noindent \textbf{Xeric Modifier} High mortality wildfire (3\% of fires in this seral stage) returns the patch to early development. Low mortality wildfire (97\%) maintains LDO. 

\medskip
\noindent \textbf{Ultramafic Modifier} High mortality wildfire (11\% of fires in this seral stage) returns the patch to early development. Low mortality wildfire (89\%) maintains LDO.

\end{adjustwidth}

\noindent\hrulefill

\paragraph{Late Development – Moderate Canopy Cover (LDM)}

\paragraph{Description} In the large tree seral stage, subdominant trees die and add to a growing layer of duff and downed woody material, and dominant trees continue to grow for several hundred years to heights of 40 m (130 ft). Overall conifer cover ranges from 40\% to 70\%. \emph{A. magnifica} is the most common tree species. The understory of mature stands is limited to less than 5 percent cover of shade tolerant forbs (e.g., \emph{Chimaphila menziesii, Pyrola picta}). 

On mesic sites, \emph{P. monticola} is the primary associate, with some \emph{P. contorta} ssp. \emph{murrayana} occurring in the understory. On xeric sites, \emph{A. magnifica} occurs in pure to mixed stands, and \emph{A. concolor} and \emph{P. jeffreyi} are the primary associates (Barrett 1988, LandFire 2007a, LandFire 2007b). Ultramafic sites will have similar species composition, especially at edges, but \emph{P. jeffreyi} is relatively more common. (O'Geen et al. 2007).


\paragraph{Succession Transition} In the absence of disturbance, patches in this seral stage will maintain, regardless of soil characteristics.

\paragraph{Wildfire Transition}
\begin{adjustwidth}{2cm}{}
\noindent \textbf{Mesic Modifier } High mortality wildfire (16\% of fires in this seral stage) will return the patch to Early Development. Low mortality wildfire (84\%) opens the stand up to LDO 15\% of the time; otherwise, the patch remains in LDM. 

\medskip
\noindent \textbf{Xeric Modifier} High mortality wildfire (19\% of fires in this seral stage) will return the patch to Early Development. Low mortality wildfire (81\%) opens the stand up to LDO 15\% of the time; otherwise, the patch remains in LDM. 

\medskip
\noindent \textbf{Ultramafic Modifier} High mortality wildfire (16\% of fires in this seral stage) will return the patch to Early Development. Low mortality wildfire (84\%) opens the stand up to LDO 15\% of the time; otherwise, the patch remains in LDM.

\end{adjustwidth}

\noindent\hrulefill

\paragraph{Late Development – Closed Canopy Cover (LDC)}

\paragraph{Description} In the large tree seral stage, subdominant trees die and add to a growing layer of duff and downed woody material, and dominant trees continue to grow for several hundred years to heights of 40 m (130 ft). Overall conifer cover exceeds 70\%. \emph{A. magnifica} is the most common tree species. The understory of mature stands is limited to less than 5 percent cover of shade tolerant forbs (e.g., \emph{Chimaphila menziesii, Pyrola picta}). 

On mesic sites, \emph{P. monticola} is the primary associate, with some \emph{P. contorta} ssp. \emph{murrayana} occuring in the understory. On xeric sites, \emph{A. magnifica} occurs in pure to mixed stands, and \emph{A. concolor} and \emph{P. jeffreyi} are the primary associates (Barrett 1988, LandFire 2007a, LandFire 2007b). Ultramafic sites will have similar species composition, especially at edges, but \emph{P. jeffreyi} is relatively more common (O'Geen et al. 2007).


\paragraph{Succession Transition} In the absence of disturbance, patches in this seral stage will maintain, regardless of soil characteristics.



\paragraph{Wildfire Transition}

\begin{adjustwidth}{2cm}{}
\noindent \textbf{Mesic Modifier } High mortality wildfire (41\% of fires in this seral stage) will return the patch to Early Development. Low mortality wildfire (59\%) opens the stand up to LDM 10\% of the time; otherwise, the patch remains in LDC. 

\medskip
\noindent \textbf{Xeric Modifier} High mortality wildfire (38\% of fires in this seral stage) will return the patch to Early Development. Low mortality wildfire (62\%) opens the stand up to LDM 10\% of the time; otherwise, the patch remains in LDC. 

\medskip
\noindent \textbf{Ultramafic Modifier} High mortality wildfire (41\% of fires in this seral stage) will return the patch to Early Development. Low mortality wildfire (59\%) opens the stand up to LDM 10\% of the time; otherwise, the patch remains in LDC.

\end{adjustwidth}

\noindent\hrulefill
\noindent\hrulefill

\subsubsection{Aspen Variant}

\paragraph{Early Development – Aspen (ED–A)}

\paragraph{Description} Grasses, forbs, low shrubs, and sparse to moderate cover of tree seedlings/saplings (primarily \emph{P. tremuloides}) with an open canopy. This seral stage is characterized by the recruitment of a new cohort of early successional, shade-intolerant tree species into an open area created by a stand-replacing disturbance. 

Following disturbance, succession proceeds rapidly from an herbaceous layer to shrubs and trees, which invade together (Barrett 1988). \emph{P. tremuloides} suckers over 6ft tall develop within about 10 years (LandFire 2007c). 



\paragraph{Succession Transition} Unless it burns, a patch in ED–A persists for 10 years, at which point it transitions to MD-A.

\paragraph{Wildfire Transition} High mortality wildfire (100\% of fires in this seral stage) recycles the patch through the ED–A seral stage. Low mortality wildfire is not modeled for this seral stage.

\noindent\hrulefill


\paragraph{Mid Development – Aspen (MD–A)}

\paragraph{Description} \emph{P. tremuloides} trees 5-16'' DBH. Canopy cover is highly variable, and can range from 40-100\%. These patches range in age from 10 to 110 years. Some understory conifers, including \emph{P. contorta} ssp. \emph{murrayana}, \emph{A. concolor}, and \emph{A. magnifica} are encroaching, but \emph{P. tremuloides} is still the dominant component of the stand (LandFire 2007c).

\paragraph{Succession Transition} MD-A persists for at least 50 years in the absence of fire, after which patches in this seral stage begin transitioning to MD-AC at a rate of 0.6 per timestep. At 100 years since entering MD-A, any remaining patches transition to MD-AC.

\paragraph{Wildfire Transition} High mortality wildfire (35\% of fires in this seral stage) recycles the patch through the ED-A seral stage. No transition occurs as a result of low mortality fire.

\noindent\hrulefill

\paragraph{Mid Development – Aspen with Conifer (MD–AC)}

\paragraph{Description} These stands have been protected from fire since the last stand-replacing disturbance. \emph{P. tremuloides} trees are predominantly 16'' DBH and greater. Conifers are present and overtopping the \emph{P. tremuloides}. \emph{A. concolor} is a typical conifer that is successional to \emph{P. tremuloides}, and is depicted here, but other conifers including \emph{P. ponderosa} and \emph{P. lambertiana} are also possible. Conifers are pole to medium-sized, and conifer cover is at least 40\% (LandFire 2007c).

\paragraph{Succession Transition} MD-AC persists for 100 years in the absence of fire, after which patches in this seral stage transition to LDC. 

\paragraph{Wildfire Transition} High mortality wildfire (17\% of fires in this seral stage) returns the patch to ED-A. Low mortality wildfire (83\%) maintains the patch in MD–AC.

\noindent\hrulefill

\paragraph{Late Development – Closed (LDC)}

\paragraph{Description} Some \emph{P. tremuloides} continue to be present in the understory, but large conifers are now the dominant tree species, having overtopped the \emph{P. tremuloides}. Smaller conifers are present in the midstory as well. Conifer species likely present include \emph{A. concolor, A. magnifica}, and \emph{P. contorta} ssp. \emph{murrayana}. (LandFire 2007a). This seral stage is analogous to the LDC seral stage for the RFR variant.

\paragraph{Succession Transition} In the absence of disturbance, patches in this seral stage will maintain, regardless of soil characteristics.

\paragraph{Wildfire Transition} High mortality wildfire (41\% of fires in this seral stage) will return the patch to ED–A. Low mortality wildfire (59\%) usually has little effect, although 10\% of the time it opens the stand up to LD-CA.

\noindent\hrulefill


\paragraph{Late Development – Conifer with Aspen (LD–CA)}

\paragraph{Description} If stands are sufficiently protected from fire such that conifer species overtop \emph{P. tremuloides} and become large, they may be able to withstand some fire that more sensitive \emph{P. tremuloides} cannot. When this occurs, it creates a patch characterized by late development conifers, such as \emph{A. concolor} or \emph{A. magnifica}, and early seral \emph{P. tremuloides}. 

\paragraph{Succession Transition} LD-CA persists for 70 years in the absence of any fire, after which patches transition to LDC. 

\paragraph{Wildfire Transition} High mortality wildfire (16\% of fires in this seral stage) returns the patch to ED-A. Low mortality wildfire (84\%) maintains the stand in LD-CA.

\noindent\hrulefill


\newpage

\subsection*{Seral Stage Classification}
\begin{table}[hbp]
\small
\centering
\caption{Classification of cover seral stage for RFR. Diameter at Breast Height (DBH) and Cover From Above (CFA) values taken from EVeg polygons. DBH categories are: null, 0-0.9'', 1-4.9'', 5-9.9'', 10-19.9'', 20-29.9'', 30''+. CFA categories are null, 0-10\%, 10-20\%, \dots , 90-100\%. Each row in the table below should be read with a boolean AND across each column of a row.}
\label{rfr_classification}
\begin{tabular}{@{}lrrrrr@{}}
\toprule
\textbf{\begin{tabular}[l]{@{}l@{}}Cover \\ Condition\end{tabular}} & \textbf{\begin{tabular}[r]{@{}r@{}}Overstory Tree \\ Diameter 1 \\ (DBH)\end{tabular}} & \textbf{\begin{tabular}[r]{@{}r@{}}Overstory Tree \\ Diameter 2 \\ (DBH)\end{tabular}} & \textbf{\begin{tabular}[r]{@{}r@{}}Total Tree\\ CFA (\%)\end{tabular}} & \textbf{\begin{tabular}[r]{@{}r@{}}Conifer \\ CFA (\%)\end{tabular}} & \textbf{\begin{tabular}[r]{@{}r@{}}Hardwood \\ CFA (\%)\end{tabular}} \\ \midrule
Early All        & null           & any & any    & any    & any  \\
Early All        & 0-4.9''         & any & any    & any    & any  \\
Mid Open         & 5-19.9''        & any & null   & null   & null \\
Mid Open         & 5-19.9''        & any & 0-40   & any    & any  \\
Mid Open         & 5-19.9''        & any & null   & 0-40   & null \\
Mid Moderate     & 5-19.9''        & any & 40-70  & any    & any  \\
Mid Moderate     & 5-19.9''        & any & null   & 40-70  & null \\
Mid Closed       & 5-19.9''        & any & 70-100 & any    & any  \\
Mid Closed       & 5-19.9''        & any & null   & 70-100 & any  \\
Late Open        & 20''+           & any & null   & null   & null \\
Late Open        & 20''+           & any & 0-40   & any    & any  \\
Late Open        & 20''+           & any & null   & 0-40   & null \\
Late Moderate    & 20''+           & any & 40-70  & any    & any  \\
Late Moderate    & 20''+           & any & null   & 40-70  & null \\
Late Closed      & 20''+           & any & 70-100 & any    & any  \\
Late Closed      & 20''+           & any & null   & 70-100 & any  \\ \bottomrule
\end{tabular}
\end{table}

RFR-ASP seral stages were assigned manually using NAIP 2010 Color IR imagery to assess seral stage.



\clearpage

\subsection*{References}

\begin{hangparas}{.25in}{1} 
\interlinepenalty=10000
Barrett, Reginald H. ``Red Fir (RFR).'' \emph{A Guide to Wildlife Habitats of California}, edited by Kenneth E. Mayer and William F. Laudenslayer. California Department of Fish and Game, 1988. \burl{http://www.dfg.ca.gov/biogeodata/cwhr/pdfs/RFR.pdf}. Accessed 4 December 2012.

``CalVeg Zone 1.'' Vegetation Descriptions. Vegetation Classification and Mapping.  11 December 2008. U.S. Forest Service. \burl{http://www.fs.usda.gov/Internet/FSE_DOCUMENTS/fsbdev3_046448.pdf}. Accessed 2 April 2013.

Chappell, Christopher B. and James K. Agee. ``Fire Severity and Tree Seedling Establishment in Abies Magnifica Forests, Southern Cascades, Oregon.'' \emph{Ecological Applications} 6.2 (1996): 628-640.

Cope, Amy B. ``Abies magnifica.'' \emph{Fire Effects Information System}, U.S. Department of Agriculture, Forest Service,  Rocky Mountain Research Station, Fire Sciences Laboratory, 1993. \burl{http://www.fs.fed.us/database/feis/} [Accessed 4 December 2012].

Estes, Becky. Central Sierra Province Ecologist, USDA Forest Service. 2013.

Laacke, Robert J. ``California Red Fir.'' Russell M. Burns and Barbara H. Honkala, tech. coords. \emph{Silvics of North America, vol 1. Conifers}; Glossary. Agriculture handbook no. 654. Washington, D.C.: U.S. Dept. of Agriculture, Forest Service, 1990. 

LandFire. ``Biophysical Setting Models.'' Biophysical Setting 0610321: Mediterranean California Red Fir Forest - Cascades. 2007a. LANDFIRE Project, U.S. Department of Agriculture, Forest Service; U.S. Department of the Interior. \burl{http://www.landfire.gov/national_veg_models_op2.php}. Accessed 9 November 2012.

LandFire. ``Biophysical Setting Models.'' Biophysical Setting 0610322: Mediterranean California Red Fir Forest – Southern Sierra. 2007b. LANDFIRE Project, U.S. Department of Agriculture, Forest Service; U.S. Department of the Interior. \burl{http://www.landfire.gov/national_veg_models_op2.php}. Accessed 9 November 2012.

LandFire. ``Biophysical Setting Models.'' Biophysical Setting 0610610: Inter-Mountain Basins Aspen-Mixed Conifer Forest and Woodland. 2007c. LANDFIRE Project, U.S. Department of Agriculture, Forest Service; U.S. Department of the Interior. \burl{http://www.landfire.gov/national_veg_models_op2.php}. Accessed 7 January 2013.

LandFire. ``Biophysical Setting Models.'' Biophysical Setting 0710320: Mediterranean California Red Fir Forest. 2007d. LANDFIRE Project, U.S. Department of Agriculture, Forest Service; U.S. Department of the Interior. \burl{http://www.landfire.gov/national_veg_models_op2.php}. Accessed 30 November 2012.

LandFire. ``Biophysical Setting Models.'' Biophysical Setting 0710220: Klamath-Siskiyou Upper Montane Serpentine Mixed Conifer Woodland. 2007e. LANDFIRE Project, U.S. Department of Agriculture, Forest Service; U.S. Department of the Interior. \burl{http://www.landfire.gov/national_veg_models_op2.php}. Accessed 30 November 2012.

Meyer, Marc D. Personal communication, 19 June 2013.

Meyer, Marc D. ``Natural Range of Variation of Red Fir Forests in the Bioregional Assessment Area'' (unpublished paper, Ecology Group, Pacific Southwest Research Station, 2013).

O'Geen, Anthony T., Randy A. Dahlgren, and Daniel Sanchez-Mata. ``California Soils and Examples of Ultramafic Vegetation'' In \emph{Terrestrial Vegetation of California, 3rd Edition}, edited by Michael Barbour, Todd Keeler-Wolf, and Allan A. Schoenherr, 71-106. Berkeley and Los Angeles: University of California Press, 2007. 

Potter, Donald A. ``Forested Communities of the Upper Montane in the Central and Southern Sierra Nevada.'' Gen. Tech. Rep. PSW-GTR-169. Albany, CA: Pacific Southwest Research Station, Forest Service, U.S. Department of Agriculture, 1998.

Safford, Hugh S. Personal communication, 5 May 2013.

Skinner, Carl N. and Chi-Ru Chang. ``Fire Regimes, Past and Present.'' \emph{Sierra Nevada Ecosystem Project: Final report to Congress, vol. II, Assessments and scientific basis for management options}. Davis: University of California, Centers for Water and Wildland Resources, 1996.

Taylor, Alan H. and Charles B. Halpern. ``The structure and dynamics of Abies magnifica forests in the southern Cascade Range, USA.'' \emph{Journal of Vegetation Science} 2 (1991): 189-200.

Van de Water, Kip M. and Hugh D. Safford. ``A Summary of Fire Frequency Estimates for California Vegetation Before Euro-American Settlement.'' \emph{Fire Ecology} 7.3 (2011): 26-57. doi: 10.4996/fireecology.0703026.

Verner, Jared. ``Aspen (ASP).'' \emph{A Guide to Wildlife Habitats of California}, edited by Kenneth E. Mayer and William F. Laudenslayer. California Deparment of Fish and Game, 1988. \burl{http://www.dfg.ca.gov/biogeodata/cwhr/pdfs/ASP.pdf}. Accessed 4 December 2012.

\end{hangparas}

