% !TEX root = master.tex
\section{Results}
\label{sec:hrvresults}
\subsection{Disturbance Regime}

\subsubsection{Burned Area and Wildfire Frequency}

% 174830 eligible hectares
% 181553 hectares in core
% check math using Wildfire_darea_trajectory.csv
% redone 9/15

In this section the same results are presented in multiple ways; although this is obviously redundant, it is done with the intent of being as inclusive as possible with respect to the potential audience. The HRV analysis presented here forms the backbone of a formal report to be submitted to the U.S. Forest Service for use in understanding and planning future management actions. In my experience, variations in the training and background of individuals affects how they understand and interpret different values. Some prefer probabilistic results while others find results tied to real terms most useful. In order to not privilege either perspective, I have included a range of information translated in different ways. My goal is to facilitate understanding of the results for both academic and professional audiences.

Approximately 96\% of the landscape was eligible for wildfire disturbance (all cover types except Barren and Water).\footnote{In this section I report values based on percent of eligible landscape. There are 181,550 hectares in the core study area, and 174,830 remain after excluding Barren and Water.} As expected, the frequency and extent of simulated wildfires varied across timesteps. This variability is illustrated in Figure~\ref{fig:darea-a}. The same data is also represented as a histogram (Figure~\ref{fig:darea-b}), which highlights the rarity of extremely expansive wildfire damage. It was more common during the simulation for fire to burn across 5--20\% of the landscape during a timestep.

As a result, the specified rotation interval and percent mortality expected over time on this landscape, large proportions of the study area burned each (5-year) timestep. As detailed in Table \ref{tab:darea_atleast}, not only did fire occur in every timestep of the simulation, it also extended across over 10\% of the landscape during most timesteps. In the median timestep (Table \ref{tab:darea}), about 14\% of the landscape burned, which translates into 24,500 hectares of burned land, 6,600 hectares of which burned at high mortality (though not necessarily contiguously). High mortality fires do include the burning of Early Development vegetation, including chaparral, when it resets the successional process. Further details on the percent of the full landscape that burned, and the breakdown of low versus high mortality outcomes, are included in Tables \ref{tab:darea_atleast} and \ref{tab:darea}.

% created 9/17
\begin{table}[!htbp]
\footnotesize
\centering
\caption{Summary of disturbed area in terms of proportion of the landscape burned during the simulation (after the equilibration period). For each benchmark proportion ($>1$\%, $>10$\%, $>25$\%, or $>50$\%) of the landscape that burns, I provide a few representations of the data intended to support different ways of considering the results, that I hope will serve both modelers' and managers' perspectives. First, I counted the number of timesteps during the simulation when that extent burned at either high or low mortality ($n$). Then I calculated the proportion of timesteps that represents ($p = n/500$). The inverse of this is the interval in timesteps, i.e., approximately every 4 timesteps, at least 25\% of the landscape burned. ($t = 1/p$). Finally, I converted the interval in timesteps to the interval in years ($y = 5t$).}
\label{tab:darea_atleast}
\begin{tabular}{@{}lllll@{}}
\toprule
 \textbf{Proportion of Landscape Burned} & \textbf{$\mathbf{>1}$\%}     & \textbf{$\mathbf{>10}$\%}    & \textbf{$\mathbf{>25}$\%}    & \textbf{$\mathbf{>50}$\%} \\ \midrule
Number of timesteps ($n$)        & 459              & 313              & 115              & 13            \\
Proportion of timesteps ($p = n/500$)    & 1.00             & 0.68             & 0.25             & 0.03          \\
Interval (timesteps) ($t = 1/p$)      & 1.00             & 1.47             & 4.01             & 35.46         \\
Interval (years)    ($y = 5t$)       & 5.02             & 7.36             & 20.04            & 177.31        \\ \bottomrule
\end{tabular}
\end{table}

% redone 9/16
\begin{table}[!htbp]
\footnotesize
\centering
\caption{Summary statistics for wildfire frequency by area disturbed during the simulation. Values are expressed as percentage and areal extent (in hectares) of the landscape eligible for disturbance that was actually burned.}
\label{tab:darea}
\begin{tabular}{@{}llll@{}}
\toprule
\textbf{\begin{tabular}[c]{@{}l@{}}Summary Statistic \\ (burned area/timestep)\end{tabular}}    & \textbf{Low Mortality}   & \textbf{High Mortality}    & \textbf{Any Mortality}   \\
\midrule                      %Low                      %high                 %any
$5^{\text{th}}$ percentile         &   2.72 (4,763)        & 0.71 (1,244)     &    3.54 (6,184)         \\
$50^{\text{th}}$ percentile        &   10.47 (18,300)      & 3.75 (6,563)     &    14.04 (24,544)         \\
$95^{\text{th}}$ percentile        &   31.29 (54,703)      & 21.43 (37,461)   &    45.88 (80,209)          \\
   Mean                            &   13.20 (23,079)      & 4.87 (8,512)     &    18.07 (31,592)         \\
\bottomrule
\end{tabular}
\end{table}

%redone 9/13


\begin{figure}[!htbp]
  \centering
  \subfloat[][]{
    \centering
    \includegraphics[width=0.5\textwidth]{/Users/mmallek/Documents/Thesis/Plots/darea/hrv_all.png} 
    \label{fig:darea-a}
    }%
  \subfloat[][]{
    \includegraphics[width=0.5\textwidth]{/Users/mmallek/Documents/Thesis/Plots/darea/hrv_newhist_all.png}
    \label{fig:darea-b}
    }
  \caption{(a) Disturbance trajectory for wildfire during the simulation, excluding the equilibration period. Red bars represent high mortality fire, while green bars represent low mortality fire and are stacked on top of high mortality. (b) Histogram of the percent of the landscape burned per timestep.} 
  \label{fig:darea}
\end{figure}

While the tables and figures above represent the data aspatially, it can also be helpful to look at individual timesteps during the simulation. Because \textsc{RMLands} is spatially explicit in its wildfire generation processes, the burned areas generated within the model look like fire perimeter maps created following actual wildfire events, and occur in the areas on the landscape anticipated to be the most susceptible to wildfire. The extent of burned area observed during the simulation is far above what has been observed in recent decades \citep{calfire2012,usgs-fire-data2012}. Beyond saying it is much greater, however, it can be difficult to imagine and conceptualize what fire covering, for example, 50\% of the study area looks like. To facilitate understanding of the statistical outcomes of this work, I created maps displaying the mortality outcomes from wildfires at four key timesteps: the minimum, maximum, median, and mean area burned\footnote{Because the exact mean area burned value did not correspond to any timestep, I display the one with the closest value for the ``any mortality'' category.} in Figures~\ref{fig:darea_min_map}--\ref{fig:darea_mean_map}. During timesteps when a large amount of fire is recorded, multiple individual fires can ``run together'' on the landscape. Each timestep in the model represents five years, and the model does not differentiate between individual fire seasons below this level. Thus the included individual fire map (right-hand figures) offers a visualization of fires over the short term.

\newpage

% background color 24, 15, 41, 0
\begin{figure}[!htbp]
  \centering
  \subfloat[][]{
    \centering
    \includegraphics[width=0.5\textwidth]{/Users/mmallek/Documents/Thesis/maps/hrv-wfmort-5th.pdf}
    \label{fig:darea_min}
  }%
  \subfloat[][]{
    \includegraphics[width=0.5\textwidth]{/Users/mmallek/Documents/Thesis/maps/hrv-distid-5th.pdf}
    \label{fig:distid_min}
  }
  \caption{Maps of area burned during the timestep in the \textbf{$\mathbf{5}^{\text{th}}$ percentile for area burned (3.54\%)} during the simulation. (a) Map by mortality level. Red indicates high mortality fire, while orange indicates low mortality fire. (b) Map showing each individual fire in a different color.}
  \label{fig:darea_min_map}
\end{figure}

\begin{figure}[!htbp]
  \centering
  \subfloat[][]{
    \centering
    \includegraphics[width=0.5\textwidth]{/Users/mmallek/Documents/Thesis/maps/hrv-wfmort-median.pdf}
    \label{fig:darea_median}
  }%
  \subfloat[][]{
    \includegraphics[width=0.5\textwidth]{/Users/mmallek/Documents/Thesis/maps/hrv-distid-median.pdf}
    \label{fig:distid_median}
  }
  \caption{Maps of area burned during the timestep with the \textbf{$\mathbf{50}^{\text{th}}$ percentile for area burned (14.04\%)} during the simulation. (a) Map by mortality level. Red indicates high mortality fire, while orange indicates low mortality fire. (b) Map showing each individual fire in a different color.}
  \label{fig:darea_median_map}
\end{figure}

\begin{figure}[!htbp]
  \centering
  \subfloat[][]{
    \centering
    \includegraphics[width=0.5\textwidth]{/Users/mmallek/Documents/Thesis/maps/hrv-wfmort-95th.pdf}
    \label{fig:darea_max}
  }%
  \subfloat[][]{
    \includegraphics[width=0.5\textwidth]{/Users/mmallek/Documents/Thesis/maps/hrv-distid-95th.pdf}
    \label{fig:distid_max}
  }
  \caption{Maps of area burned during the timestep with the \textbf{$\mathbf{95}^{\text{th}}$ percentile for area burned (45.88\%)} during the simulation. (a) Map by mortality level. Red indicates high mortality fire, while orange indicates low mortality fire. (b) Map showing each individual fire in a different color.}
  \label{fig:darea_max_map}
\end{figure}

\begin{figure}[!htbp]
  \centering
  \subfloat[][]{
    \centering
    \includegraphics[width=0.5\textwidth]{/Users/mmallek/Documents/Thesis/maps/hrv-wfmort-mean.pdf}
    \label{fig:darea_mean}
  }%
  \subfloat[][]{
    \includegraphics[width=0.5\textwidth]{/Users/mmallek/Documents/Thesis/maps/hrv-distid-mean.pdf}
    \label{fig:distid_mean}
  }
  \caption{Maps of area burned during the timestep with the \textbf{mean total area burned (18.07\%)} during the simulation. (a) Map by mortality level. Red indicates high mortality fire, while orange indicates low mortality fire. (b) Map showing each individual fire in a different color.}
  \label{fig:darea_mean_map}
\end{figure}

\clearpage

%%%%%%%%%%%%%%%%%%%%%%%%%%%%%%%%%%%%%%%%%%%%%%%%%%%%%%%%%%%%%%%%%%%%%%%%%%%%%%%%%%%%%%%%%%%%%%%%%%%%%%%%%%%%%%%%%%%%%%%%%%%%%%%%%%%%

\paragraph*{Sierran Mixed Conifer - Mesic}
Sierran Mixed Conifer - Mesic (\textsc{smc\_m}) is the dominant cover type within the core study area, encompassing 57,853 ha and comprising roughly 32\% of the study area. Wildfire was prevalent in this cover type. I again present figures and tables that incorporate some redundancy in order to facilitate understanding by a broad audience, as described in the beginning of this section (Figure \ref{fig:darea_smcm}). I summarize the disturbance regime in Tables~\ref{tab:darea_smcm} and \ref{tab:darea_atleast_smcm}. The frequency and extent of burned area is similar to that for the landscape as a whole.

% plots redone
\begin{figure}[!htbp]
  \centering
  \subfloat[][]{
    \centering
    \includegraphics[width=0.5\textwidth]{/Users/mmallek/Documents/Thesis/Plots/darea/hrv_smcm.png}
    }%
  \subfloat[][]{
    \includegraphics[width=0.5\textwidth]{/Users/mmallek/Documents/Thesis/Plots/darea/hrv_newhist_smcm.png}
    }
  \caption{\small (a) Disturbance trajectory for Sierran Mixed Conifer - Mesic. High mortality fire in red; low mortality fire in green. (b) Histogram of the percent of the landscape burned per timestep.} 
  \label{fig:darea_smcm}
\end{figure}

% updated 2015-09-21
\begin{table}[!htbp]
\footnotesize
\centering
\caption{Disturbed area summary statistics for Sierran Mixed Conifer - Mesic (\textsc{smc\_m}). Proportions shown are relative to the total area of \textsc{smc\_m}.}
\label{tab:darea_smcm}
\begin{tabular}{@{}lrrr@{}}
\toprule
\textbf{\begin{tabular}[c]{@{}l@{}}Summary Statistic \\ (burned \textsc{SMC\_M}/timestep)\end{tabular}} & \textbf{Low Mortality} & \textbf{High Mortality} & \textbf{Any Mortality} \\ \midrule
$5^{\text{th}}$ percentile        & 2.60  & 0.47  & 3.17  \\
$50^{\text{th}}$ percentile       & 11.45 & 3.35  & 14.89 \\
$95^{\text{th}}$ percentile       & 34.17 & 11.57 & 45.27 \\
Mean                              & 14.42 & 4.42  & 18.83 \\
\bottomrule
\end{tabular}
\end{table}


\begin{table}[!htbp]
\footnotesize
\centering
\caption{Summary of disturbed area in terms of proportion of the amount of Sierran Mixed Conifer - Mesic (\textsc{smc\_m}). See Table~\ref{tab:darea_atleast} caption for details.}
\label{tab:darea_atleast_smcm}
\begin{tabular}{@{}lllll@{}}
\toprule
 \textbf{Proportion of SMC\_M Burned} & \textbf{$\mathbf{>1}$\%}     & \textbf{$\mathbf{>10}$\%}    & \textbf{$\mathbf{>25}$\%}    & \textbf{$\mathbf{>50}$\%} \\ \midrule
Number of timesteps ($n$)    & 458          & 311           & 126           & 15            \\
Proportion of timesteps ($p = n/500$) & 0.99         & 0.67          & 0.27          & 0.03          \\
Interval (timesteps) ($t = 1/p$)   & 1.01         & 1.48          & 3.66          & 30.73         \\
Interval (years) ($y = 5t$)       & 5.03         & 7.41          & 18.29         & 153.67       \\ \bottomrule
\end{tabular}
\end{table}



%\clearpage
%%%%%%%%%%%%%%%%%%%%%%%%%%%%%%%%%%%%%%%%%%%%%%%%%%%%%%%%%%%%%%%%%%%%%%%%%%%%%%%%%%%%%%%%%%%%%%%%%%%%%%%%%%%%%%%%%%%%%%%%%%%%%%%%%%%%


\paragraph*{Sierran Mixed Conifer - Xeric}
Sierran Mixed Conifer - Xeric (\textsc{smc\_x}) is the second most dominant cover type within the core study area, encompassing 52,198 ha and comprising roughly 29\% of the study area. Wildfire was prevalent in this cover type. I again present figures and tables that incorporate some redundancy in order to facilitate understanding by a broad audience (Figure \ref{fig:darea_smcx}). I summarize the disturbance regime in Tables \ref{tab:darea_smcx} and \ref{tab:darea_atleast_smcx}. Low and high mortality fires were more frequent and extensive on the xeric mixed conifer forests than in mesic mixed conifer forests or the study area as a whole. High mortality wildfire on xeric mixed conifer forests extended over a larger mean and median proportion compared to the overall landscape, although the $95^{\text{th}}$ percentile value for high mortality fire extent in xeric mixed conifer forests was 3.2\% percentiles less than that for mesic mixed conifer forests.

% updated 2015-09
\begin{figure}[!hbp]
  \centering
  \subfloat[][]{
    \centering
    \includegraphics[width=0.5\textwidth]{/Users/mmallek/Documents/Thesis/Plots/darea/hrv_smcx.png}
    }%
  \subfloat[][]{
    \includegraphics[width=0.5\textwidth]{/Users/mmallek/Documents/Thesis/Plots/darea/hrv_newhist_smcx.png}
    }
  \caption{\small (a) Disturbance trajectory for Sierran Mixed Conifer - Xeric. High mortality fire in red; low mortality fire in green. (b) Histogram of the percent of the landscape burned per timestep.} 
  \label{fig:darea_smcx}
\end{figure}

% updated 2015-09-21
\begin{table}[!htbp]
\footnotesize
\centering
\caption{Disturbed area summary statistics for Sierran Mixed Conifer - Xeric (\textsc{smc\_x}). Proportions shown are relative to the total area of \textsc{smc\_x}.}
\label{tab:darea_smcx}
\begin{tabular}{@{}llll@{}}
\toprule
\textbf{\begin{tabular}[c]{@{}l@{}}Summary Statistic \\ (disturbed SMC\_X/timestep)\end{tabular}} & \textbf{Low Mortality} & \textbf{High Mortality} & \textbf{Any Mortality} \\ \midrule
$5^{\text{th}}$ percentile    & 3.30  & 1.00  & 4.50  \\
$50^{\text{th}}$ percentile   & 11.92 & 5.17  & 17.55 \\
$95^{\text{th}}$ percentile   & 36.02 & 18.20 & 54.28 \\
Mean                          & 14.88 & 6.95  & 21.83 \\  \bottomrule
\end{tabular}
\end{table}


\begin{table}[!htbp]
\footnotesize
\centering
\caption{Summary of disturbed area in terms of proportion of the amount of Sierran Mixed Conifer - Xeric (\textsc{smc\_x}) burned during the simulation. See Table~\ref{tab:darea_atleast} caption for details.}
\label{tab:darea_atleast_smcx}
\begin{tabular}{@{}lllll@{}}
 \toprule
\textbf{Proportion of SMC\_X Burned} & \textbf{$\mathbf{>1}$\%}     & \textbf{$\mathbf{>10}$\%}    & \textbf{$\mathbf{>25}$\%}    & \textbf{$\mathbf{>50}$\%} \\ \midrule
Number of timesteps ($n$)             & 461          & 347           & 148           & 27            \\
Proportion of timesteps ($p = n/500$) & 1.00         & 0.75          & 0.32          & 0.06          \\
Interval (timesteps) ($t = 1/p$)      & 1.00         & 1.33          & 3.11          & 17.07         \\
Interval (years)    ($y = 5t$)        & 5.00         & 6.64          & 15.57         & 85.37         \\ \bottomrule
\end{tabular}
\end{table}

%%%%%%%%%%%%%%%%%%%%%%%%%%%%%%%%%%%%%%%%%%%%%%%%%%%%%%%%%%%%%%%%%%%%%%%%%%%%%%%%%%%%%%%%%%%%%%%%%%%%%%%%%%%%%%%%%%%%%%%%%%%%%%%%%%%%
%%%%%%%%%%%%%%%%%%%%%%%%%%%%%%%%%%%%%%%%%%%%%%%%%%%%%%%%%%%%%%%%%%%%%%%%%%%%%%%%%%%%%%%%%%%%%%%%%%%%%%%%%%%%%%%%%%%%%%%%%%%%%%%%%%%%

\subsubsection{Climate Effect} 
% fire size sentence is discussion
Climate has a positive relationship with disturbed area. The climate parameter is regressed against disturbed area in Figure \ref{fig:climate_darea}, but note the heteroskedastic variance about the mean. The relationship is weakly positive, in that as climate shifts from wet to drought, disturbed area increases. The climate parameter is defined such that 1 is the average value over the historical period. During wetter-than-average years, less area was disturbed. For example, no more than 20\% of the landscape burned in any of the timesteps during which the climate parameter was below 0.63. However, over 50\% of the landscape burned in several timesteps when the climate parameter was less than 1 (wet periods). Figure \ref{fig:compare_clim_darea} illustrates the climate parameter values and disturbed area proportion of the landscape for a subset of timesteps during the simulation to illustrate that in some years, a high climate parameter occurs with a higher disturbed area percentage, but in other years a low climate parameter occurs with a higher disturbed area percentage that in mesic mixed conifer forests. Therefore, while a correlation certainly exists between climate and disturbed area, it is not associated with a firm ceiling or floor.

\begin{figure}[!htbp]
  \centering
  \subfloat[][]{
    \centering
  \includegraphics[width=0.5\textwidth]{/Users/mmallek/Documents/Thesis/Plots/darea/hrv_climdarea.png}
    \label{fig:climate_darea}
  } 
  \subfloat[][]{
    \includegraphics[width=0.5\textwidth]{/Users/mmallek/Documents/Thesis/Plots/darea/climate_darea_vert.png}
    \label{fig:compare_clim_darea}
  } 
  \caption{(a) Scatterplot of the climate parameter and disturbed area value for each timestep of the simulation (excluding the equilibration period). A linear model has been fit to the data and is shown as a blue line; the grey shaded area represents the 95\% confidence interval around the mean. (b) Climate parameter and proportion of eligible landscape disturbed by wildfire for timesteps 250 to 310 of the simulation, illustrating the wide variability in both climate parameter values and disturbed area per timestep. The purple shaded area helps map the climate parameter value and proportion of landscape burned during the same individual timestep.}
  \label{fig:climate_darea_combofig}
\end{figure}

\clearpage

\newpage
\subsubsection{Rotation Period} 
As described in Chapter \ref{sec:hrvmethods}, I calibrated the model by adjusting seral stage-specific susceptibility values until the nine cover types with more than 1000 ha extent across the study area were within 10\% of their target fire rotation. I present here the results for the two focal cover types and the full study area. Full results for all cover types in the study area are presented in Table~\ref{tab:all-rotations}. 

% updated 2015-09
\begin{table}[!htbp]
\footnotesize
\centering
\caption{Fire rotation for Sierran Mixed Conifer - Mesic, Sierran Mixed Conifer - Xeric, and the full landscape.}
\label{tab:ch2-rotations}
\begin{tabular}{@{}lrrr@{}}
\toprule
\begin{tabular}[c]{@{}l@{}}Land Cover \\ Type\end{tabular}     & \begin{tabular}[c]{@{}l@{}}Low Mortality \\ Fire Rotation\end{tabular} & \begin{tabular}[c]{@{}l@{}}High Mortality \\ Fire Rotation\end{tabular} & \begin{tabular}[c]{@{}l@{}}All Fires \\ Rotation\end{tabular} \\ \midrule
\textsc{smc\_m   }      & 35  & 113   & 27    \\
\textsc{smc\_x   }      & 34  & 72    & 23    \\
Full Landscape          & 38  & 103   & 28    \\ \bottomrule
\end{tabular}
\end{table}


\subsubsection{Point-specific Return Interval}
Visualizing the point-specific fire rotation (return interval) for an individual grid cell calls attention to the variability in wildfire recurrence across the study area. Barplots show the spread and underlying values in the distribution of point-specific fire rotations, and maps demonstrate the spatial variability in this metric across the study area. Overall, the point-specific fire rotation for an individual cell ranged from 17 years to \textgreater 2500 years (cells that never burned during the simulation) for both classes of wildfire mortality (Figure \ref{fig:preturn}). The grand mean return interval across all cover types was 42 years for low mortality fire, 111 year for high mortality fire, and 29 years for any fire. The point-specific fire rotation plots and maps specific to Sierran Mixed Conifer Mesic and Xeric follow (Figures~\ref{fig:preturn_smcm} and \ref{fig:preturn_smcx}). Under this wildfire regime, the point-specific fire rotation for an individual point between fires (of any mortality level) for both of these mixed conifer forest types varied widely from about 17 years to over 500 years, with grand means of 28 years (for mesic) and 23 years (for xeric) (Figures~\ref{fig:preturn_smcm} and \ref{fig:preturn_smcx}). Results for the other seven focal cover types are included in Appendix~\ref{sec:indiv_cov_results}. 



% first plot redone 9/13
% second plot not redone yet
\begin{figure}[!htbp]
  \centering
  \subfloat[][]{
    \centering
    \includegraphics[height=.4\textheight]{/Users/mmallek/Documents/Thesis/Plots/preturn/not-called-preturn/hrv-total.png}
    \label{fig:preturn_plot}
  }%
  \qquad
  \subfloat[][]{
    \includegraphics[height=.4\textheight]{/Users/mmallek/Documents/Thesis/Plots/preturn-maps/fri_all.png}
    \label{fig:preturn_map}
  }
  \caption{(a) Distribution of point-specific fire rotations for the full landscape under study. The ``full landscape'' includes each cell in the raster with a cover type eligible to burn. The point-specific fire rotation is the average interval between fires over the length of the simulation, excluding the equilibration period, at each individual grid cell. (b) Spatially-explicit depiction of these point-specific fire rotations across the landscape, for all cover types.}
  \label{fig:preturn}
\end{figure}

% first plot updated 9/13
\begin{figure}[!htbp]
  \centering
  \subfloat[][]{
    \centering
    \includegraphics[width=0.5\textwidth]{/Users/mmallek/Documents/Thesis/Plots/preturn/not-called-preturn/hrv-smcm.png}
    }%
  \subfloat[][]{
    \includegraphics[width=0.5\textwidth]{/Users/mmallek/Documents/Thesis/Plots/preturn-maps/fri_smcm.png}
    }
  \caption{(a) Distribution of point-specific fire rotations for Sierran Mixed Conifer - Mesic. The point-specific fire rotation is the average interval between fires over the length of the simulation, excluding the equilibration period, at each individual grid cell. (b) Spatially-explicit depiction of these point-specific fire rotations across the landscape. Cover types other than Sierran Mixed Conifer - Mesic are partially obscured in grey.}
\label{fig:preturn_smcm}
\end{figure}

%first plot redone 9/13
\begin{figure}[!htbp]
  \centering
  \subfloat[][]{
    \centering
    \includegraphics[width=0.5\textwidth]{/Users/mmallek/Documents/Thesis/Plots/preturn/not-called-preturn/hrv-smcx.png}
    }%
  \subfloat[][]{
    \includegraphics[width=0.5\textwidth]{/Users/mmallek/Documents/Thesis/Plots/preturn-maps/fri_smcx.png}
    }
  \caption{(a) Distribution of point-specific fire rotations for Sierran Mixed Conifer - Xeric. The point-specific fire rotation is the average interval between fires over the length of the simulation, excluding the equilibration period, at each individual grid cell. (b) Spatially-explicit depiction of these point-specific fire rotations across the landscape. Cover types other than Sierran Mixed Conifer - Xeric are partially obscured in grey.}
\label{fig:preturn_smcx}
\end{figure}

\clearpage


%%%%%%%%%%%%%%%%%%%%%%%%%%%%%%%%
%%%%%%%%%%%%%%%%%%%%%%%%%%%%%%%%
%%%%%%%%%%%%%%%%%%%%%%%%%%%%%%%%

%\pagebreak[4]
\subsection{Vegetation Response}
\label{subsec:HRVvegresponse}


\subsubsection{Landscape Composition} 

% fixed plots - equilibration line is hard coded in. ocfwu calibration changed; now seems okay by ts 40. potentially could even have cut off equilibration at ts 20 but it's arbitrary. good to keep in mind for future stuff though.
The seral stage distribution for each cover type varied over time, but did appear to be in dynamic equilibrium \citep{Diamond1969}. Evidence of both high mortality fire, which triggers a transition to the Early Development seral stage for all cover types, and low mortality fire, which can thin a stand and cause a transition to a more open canopy seral stage (within the same development level), are visible in examining the output grids. Figure \ref{fig:covcondmaps} illustrates these changes for a sequence of four timesteps during the simulation. The seral stage dynamics and current seral stage distribution plots specific to Sierran Mixed Conifer - Mesic and Sierran Mixed Conifer - Xeric follow (Figures~\ref{fig:hrv-covcond_smcm} and \ref{fig:hrv-covcond_smcx}). I compare the current landscape's seral stage distribution to the simulated distribution and assess the current landscape's departure from the HRV in Tables~\ref{tab:covcond_smcm} and \ref{tab:covcond_smcx}. Plots and tabular results for the other seven focal types are included in Appendix~\ref{sec:indiv_cov_results}.

% new plots 2015-09-18
\begin{figure}[!htbp]
  \centering
  \subfloat[][]{
    \includegraphics[width=0.5\textwidth]{/Users/mmallek/Documents/Thesis/maps/hrv-covcondseq-5.pdf}
  }%
  \subfloat[][]{
    \includegraphics[width=0.5\textwidth]{/Users/mmallek/Documents/Thesis/maps/hrv-covcondseq-6.pdf}
  }\\%
  \subfloat[][]{
    \includegraphics[width=0.5\textwidth]{/Users/mmallek/Documents/Thesis/maps/hrv-covcondseq-7.pdf}
    }
  \subfloat[][]{
    \centering
    \includegraphics[width=0.5\textwidth]{/Users/mmallek/Documents/Thesis/maps/hrv-covcondseq-8.pdf}
  }%
  \caption{A sequence of four timesteps during the middle of the simulation, showing changes in seral stages over time. Here I highlight the dominant cover type, Sierran Mixed Conifer - Mesic, and its classes, in order to illustrate the dynamics that play out over many years. (a) Timestep 1 (b) Timestep 2 (c) Timestep 3 (d) Timestep 4. Patches in shades of brown and tan belong to other cover types.}
  \label{fig:covcondmaps}
\end{figure}


\clearpage

\paragraph*{Sierran Mixed Conifer - Mesic}

% hrv plot updated 2015-09
\begin{figure}[!htbp]
  \centering
  \subfloat[][]{
    \centering
    \includegraphics[width=0.6\textwidth]{/Users/mmallek/Documents/Thesis/Plots/covcond-dynamics/notcalledcovcond/SMCM.pdf}
  }%
  \subfloat[][]{
    \includegraphics[height=2.65in]{/Users/mmallek/Tahoe/R/Rplots/November2014/covcond_current_smcm.png}
  } \\
  \subfloat[][]{
    \includegraphics[width=\textwidth]{/Users/mmallek/Documents/Thesis/Plots/covcond-bycover/SMCM-HRV-boxplots-.png}
  }
  \caption{(a) Cover type-Seral stage dynamics for Sierran Mixed Conifer - Mesic. The black vertical line at 40 timesteps marks the end of the equilibration period used in this study. (b) Current seral stage distribution for Sierran Mixed Conifer - Mesic. (c) Boxplots showing the range of variability for each seral stage over the course of the simulation, excluding the equilibration period. Boxplots were modified so that whiskers extend from the $5^{\text{th}} - 95^{\text{th}}$ percentiles of the observed results. Thick black bars in line with the boxplots denote the current proportion of mesic mixed conifer forests in a given seral stage.} 
  \label{fig:hrv-covcond_smcm}
\end{figure}

The distribution of area among stand seral stages within mesic mixed conifer forests fluctuated over time, but appeared to be in dynamic equilibrium (Figure~\ref{fig:hrv-covcond_smcm}). The percentage of mesic mixed conifer forests in the Early Development seral stage varied from approximately 8\%--25\%. This seral stage is currently within the simulated HRV (48$^{\text{th}}$ percentile). The Late--Moderate seral stage is currently moderately departed within the HRV, but at the edge (95$^{\text{th}}$ percentile). Current proportions of the other five seral stages are completely departed from the simulated HRV.  The current seral-stage distribution was never observed under the simulated HRV (Table~\ref{tab:covcond_smcm}). The current landscape contains more Mid--Moderate, Mid--Open, and Late--Closed, and less Mid--CLosed and Late--Open, than the simulated HRV.


%\begin{landscape}
% table updated 2015-09
\begin{table}[!htbp]
\footnotesize
\centering
\caption{Range of variability in landscape structure, illustrating the cover type-seral stage class dynamics for Sierran Mixed Conifer - Mesic. Included are the $5^{\text{th}}$ percentile, $25^{\text{th}}$ percentile, $50^{\text{th}}$ percentile, $75^{\text{th}}$ percentile, and $95^{\text{th}}$ percentiles of the distribution, as well as the current landscape proportion, the current percentile range of variability (\%RV) for that proportion, and the departure classification. For seral stage abbreviations, see Table \ref{condtable}. Departure is classified as follows: if the current landscape metric value falls within the $25^{\text{th}}-75^{\text{th}}$ percentile range (the box in the boxplots), it is considered not departed (Departure is ``none'' in the table). If it falls within the $5^{\text{th}}-25^{\text{th}}$ percentile range or the $75^{\text{th}}-95^{\text{th}}$ percentile range (the whiskers in the boxplots), it is moderately departed (Departure is ``moderate'' in the table). If it falls outside that range, it is completely departed (Departure is ``complete'' in the table).}
\label{tab:covcond_smcm}
\begin{tabular}{@{}lrrrrr|rrr@{}}
\toprule
\textbf{\begin{tabular}[c]{@{}l@{}}Seral \\ Stage\end{tabular}}  &  \textbf{$\mathbf{5}^{\text{th}}$} &   \textbf{$\mathbf{25}^{\text{th}}$} &   \textbf{$\mathbf{50}^{\text{th}}$} &   \textbf{$\mathbf{75}^{\text{th}}$} &   \textbf{$\mathbf{95}^{\text{th}}$}  &  \textbf{\begin{tabular}[c]{@{}l@{}}Current\\ \%cover\end{tabular}} & \textbf{\begin{tabular}[c]{@{}l@{}}Current\\ \%RV\end{tabular}} & \textbf{\begin{tabular}[c]{@{}l@{}}Departure\end{tabular}} \\ \midrule
\textsc{early\_all}        &   7.75        &  12.34   &  15.11     &  18.68   &  24.74     &  14.98    &  48    &  none      \\
\textsc{mid\_cl   }        &   21.52       &  26.15   &  29.69     &  32.58   &  37.01     &  9.74     &  0     &  complete     \\
\textsc{mid\_mod  }        &   6.8         &  7.98    &  9.03      &  10.3    &  12.63     &  17.97    &  100   &  complete     \\
\textsc{mid\_op   }        &   6.68        &  9.2     &  11.21     &  13.08   &  16.15     &  16.29    &  96    &  complete     \\
\textsc{late\_cl  }        &   5.31        &  9.54    &  12.87     &  17.2    &  22.91     &  23.23    &  97    &  complete      \\
\textsc{late\_mod }        &   8.56        &  10.32   &  11.24     &  12.56   &  14.41     &  14.18    &  95    &  moderate      \\
\textsc{late\_op  }        &   4.96        &  7.39    &  9.26      &  12.12   &  14.95     &  3.6      &  1     &  complete      \\
\bottomrule
\end{tabular}
\end{table}





%\end{landscape}
%\clearpage
%%%%%%%%%%%%%%%%%%%%%%%%%%%%%%%%%%%%%%%%%%%%%%%%%%%%%%%%%%%%%%%%%%%%%%%%%%%%%%%%%%%%%%%%%%%%%%%%
\paragraph*{Sierran Mixed Conifer - Xeric}

% plot updated 2015-09
\begin{figure}[!htbp]
  \centering
  \subfloat[][]{
    \centering
    \includegraphics[width=0.6\textwidth]{/Users/mmallek/Documents/Thesis/Plots/covcond-dynamics/notcalledcovcond/SMCX.pdf}
  }%
  \subfloat[][]{
    \includegraphics[height=2.65in]{/Users/mmallek/Tahoe/R/Rplots/November2014/covcond_current_smcx.png}
  } \\
  \subfloat[][]{
    \includegraphics[width=\textwidth]{/Users/mmallek/Documents/Thesis/Plots/covcond-bycover/SMCX-HRV-boxplots-.png}
  }
  \caption{(a) Cover type-Seral stage dynamics for Sierran Mixed Conifer - Xeric. The black vertical line at 40 timesteps marks the end of the equilibration period used in this study. (b) Current seral stage distribution for Sierran Mixed Conifer - Xeric. (c) Boxplots showing the range of variability for each seral stage over the course of the simulation, excluding the equilibration period. Boxplots were modified so that whiskers extend from the $5^{\text{th}} - 95^{\text{th}}$ percentiles of the observed results. Thick black bars in line with the boxplots denote the current proportion of xeric mixed conifer forests in a given seral stage.}  
  \label{fig:hrv-covcond_smcx}
\end{figure}

The distribution of area among seral stages within xeric mixed conifer forests fluctuated over time, but appeared to be in dynamic equilibrium (Figure~\ref{fig:covcond_smcx}). The percentage of xeric mixed conifer forests in the Early Development varied from approximately 25\% to 43\% (Table~\ref{tab:covcond_smcx}). During the simulation, Early Development (which includes post-fire chaparral fields) and Mid--Open seral stages dominated, in contrast to the current distribution, which is somewhat even across classes (although Late--Open is currently quite rare).
%
The current seral stage distribution was never observed under the simulated HRV, and all of the seral stages were fully departed from the HRV (Table~\ref{tab:covcond_smcx}). The current landscape contains more closed and moderate canopy forest, and less Early Development and open canopy forest, than the simulated HRV.


% table updated 2015-09
\begin{table}[!htbp]
\footnotesize
\centering
\caption{Range of variability in landscape structure, illustrating the cover type-seral stage dynamics for Sierran Mixed Conifer - Xeric. Included are the $5^{\text{th}}$ percentile, $25^{\text{th}}$ percentile, $50^{\text{th}}$ percentile, $75^{\text{th}}$ percentile, and $95^{\text{th}}$ percentiles of the distribution, as well as the current landscape proportion, the current percentile range of variability (\%RV) for that proportion, and the departure classification. For seral stage abbreviations, see Table \ref{condtable}. Departure is classified as follows: if the current landscape metric value falls within the $25^{\text{th}}-75^{\text{th}}$ percentile range (the box in the boxplots), it is considered not departed (Departure is ``none'' in the table). If it falls within the $5^{\text{th}}-25^{\text{th}}$ percentile range or the $75^{\text{th}}-95^{\text{th}}$ percentile range (the whiskers in the boxplots), it is moderately departed (Departure is ``moderate'' in the table). If it falls outside that range, it is completely departed (Departure is ``complete'' in the table).}
\label{tab:covcond_smcx}
\begin{tabular}{@{}lrrrrr|rrr@{}}
\toprule
\textbf{\begin{tabular}[c]{@{}l@{}}Seral \\ Stage\end{tabular}}  &  \textbf{$\mathbf{5}^{\text{th}}$} &   \textbf{$\mathbf{25}^{\text{th}}$} &   \textbf{$\mathbf{50}^{\text{th}}$} &   \textbf{$\mathbf{75}^{\text{th}}$} &   \textbf{$\mathbf{95}^{\text{th}}$}  &  \textbf{\begin{tabular}[c]{@{}l@{}}Current\\ \%cover\end{tabular}} &   \textbf{\begin{tabular}[c]{@{}l@{}}Current\\ \%RV \end{tabular}} &   \textbf{\begin{tabular}[c]{@{}l@{}}Departure\end{tabular}} \\ \midrule
 \textsc{early\_all}      &  25.2          &  29.63    &  34.53    &  38.95    &  42.82     &  19.48       &   0      &  complete    \\
 \textsc{mid\_cl   }      &  0.02          &  0.06     &  0.13     &  0.36     &  1.07      &  11.96       &   100    &  complete      \\
 \textsc{mid\_mod  }      &  0.9           &  1.62     &  2.88     &  4.35     &  7.6       &  14.92       &   100    &  complete    \\
 \textsc{mid\_op   }      &  26.55         &  30.59    &  33.79    &  36.58    &  39.36     &  11.48       &   0      &  complete    \\
 \textsc{late\_cl  }      &  1.19          &  2.51     &  3.81     &  5.99     &  8.69      &  24.72       &   100    &  complete      \\
 \textsc{late\_mod }      &  5.83          &  7.49     &  9.16     &  10.71    &  13.03     &  13.31       &   97     &  complete     \\
 \textsc{late\_op  }      &  9.39          &  12.4     &  15       &  17.42    &  22.45     &  4.13        &   0      &   complete  \\ \bottomrule 
\end{tabular}
\end{table}

\clearpage


%%%%%%%%%%%%%%%%%%%%%%%%%%%%%%%%%%%%%%%%%%%%%%%%%%%%%%%%%%%%%%%%%%%%%%%%%%%%%%%%%%%%%%%%%%%%%%%%
%%%%%%%%%%%%%%%%%%%%%%%%%%%%%%%%%%%%%%%%%%%%%%%%%%%%%%%%%%%%%%%%%%%%%%%%%%%%%%%%%%%%%%%%%%%%%%%%

\subsubsection{Landscape Configuration}
I summarized the structure and patterns in the landscape using a suite of statistical measures calculated using \textsc{Fragstats}. Table \ref{tab:fragland} shows the range of variability for the simulation period as well as the current value, the current percentile range of variability (\%RV) for that proportion, and the departure classification. I show here a subset of metrics most useful for understanding patch characteristics in the study area; complete results are included in Appendix~\ref{app:full-land-results}.  Appendix~\ref{app:metricdescriptions} contains a detailed description of each \textsc{Fragstats} metric calculated for this project. At the landscape-level, most computed metrics have values outside the HRV. 

In Figures~\ref{fig:fragland1} and \ref{fig:fragland2} I graphically display the results from Table \ref{tab:fragland}. For these six metrics, the current landscape is fully departed from the historical range of variability. The average patch size is larger, and the average patch shape more complex, than the current landscape. Patches during the HRV had more edge, and on average, contained more core area than the current landscape. The landscape during the HRV is much more contagious than the current landscape. Values for Simpson's Evenness are near 1 during the HRV and in the present landscape, but the HRV values are well below the current conditions.

% plots updated 2015-09

\begin{figure}[!htbp]
  \centering
  \subfloat[][]{
    %\centering
    \includegraphics[width=0.5\textwidth]{/Users/mmallek/Documents/Thesis/Plots/fragland-hrv/ED1.png}
    \label{fig:fragland_ed}
  }%
  \subfloat[][]{
    %\centering
    \includegraphics[width=0.5\textwidth]{/Users/mmallek/Documents/Thesis/Plots/fragland-hrv/AREA_AM1.png}
    \label{fig:fragland_area}
  } \\
  \subfloat[][]{
    \includegraphics[width=0.5\textwidth]{/Users/mmallek/Documents/Thesis/Plots/fragland-hrv/SHAPE_AM1.png}
    \label{fig:fragland_shape}
  } 
  \subfloat[][]{
    \includegraphics[width=0.5\textwidth]{/Users/mmallek/Documents/Thesis/Plots/fragland-hrv/CORE_AM1.png}
    \label{fig:fragland_core}
    }
\caption{Landscape \textsc{Fragstats} Metrics. (a) Edge Density, a measure of patch perimeter complexity, (b) Area-weighted Mean Patch Area, a measure of patch size (c) Area-weighted Mean Shape, a measure of patch shape complexity (d) Area-weighted Mean Core Area, a measure of interior habitat available at the patch level. The red line indicates the metric value on the current landscape, the dotted lines indicate the 5th and 95th percentiles of the simulated data, the dashed line indicates the 50th percentile of the simulated data, and the blue line indicates the value for that metric at each timestep of the simulation.}
\label{fig:fragland1}
\end{figure}

\begin{figure}[!htbp]
  \centering
  \subfloat[][]{
    \includegraphics[width=0.5\textwidth]{/Users/mmallek/Documents/Thesis/Plots/fragland-hrv/CONTAG1.png}
    \label{fig:fragland_contag}
  } 
  \subfloat[][]{
    \includegraphics[width=0.5\textwidth]{/Users/mmallek/Documents/Thesis/Plots/fragland-hrv/SIEI1.png}
    \label{fig:fragland_siei}
  } 
\caption{Landscape \textsc{Fragstats} Metrics. (a) Contagion, a measure of patch dispersion and interspersion (b) Simpson's Evenness Index, a measure of diversity, or evenness, across all landscape patches. The red line indicates the metric value on the current landscape, the dotted lines indicate the 5th and 95th percentiles of the simulated data, the dashed line indicates the 50th percentile of the simulated data, and the blue line indicates the value for that metric at each timestep of the simulation.}
\label{fig:fragland2}
\end{figure}

%\clearpage

% repaired table 9/13
%\begin{landscape}
\begin{table}[!htbp]
\footnotesize
\centering
\caption{Range of variability during the simulation for selected landscape configuration metrics. See Appendix~\ref{app:metricdescriptions} for descriptions. Abbreviations are: 
\textsc{ed} = edge density;
\textsc{area\_am} = area-weighted mean patch size; 
\textsc{shape\_am} = area-weighted mean patch shape index; 
\textsc{core\_am} = area-weighted mean patch core area; 
\textsc{contag} = contagion; 
\textsc{siei} = Simpson's evenness index.
Included are the $5^{\text{th}}$ percentile, $25^{\text{th}}$ percentile, $50^{\text{th}}$ percentile, $75^{\text{th}}$ percentile, and $95^{\text{th}}$ percentiles of the distribution, as well as the current landscape proportion, the current percentile range of variability (\%RV) for that proportion, and the departure classification.
} 
\label{tab:fragland}
\begin{tabular}{@{}lrrrrr|rrr@{}}
\toprule
\textbf{\begin{tabular}[c]{@{}l@{}}Landscape\\ Metric\end{tabular}}  &   
\textbf{$5^{\text{th}}$ } &   
\textbf{$25^{\text{th}}$ } &   
\textbf{$50^{\text{th}}$ } &   
\textbf{$75^{\text{th}}$ } &   
\textbf{$95^{\text{th}}$ }  &  
\textbf{\begin{tabular}[c]{@{}l@{}}Current\\ Value\end{tabular}} &   
\textbf{\begin{tabular}[c]{@{}l@{}}Current\\ \%RV\end{tabular}} &   
\textbf{\begin{tabular}[c]{@{}l@{}}Departure\end{tabular}} \\ 
\midrule
\textsc{ed}         & 120.581         & 121.880           & 122.903          & 123.691          & 124.813          & 128.875     & 100     & complete  \\
\textsc{area\_am}   & 156.549         & 166.016          & 174.884          & 184.448          & 205.209          & 119.985     & 0       & complete \\
\textsc{shape\_am}  & 3.560            & 3.621            & 3.667            & 3.727            & 3.847            & 3.243       & 0       & complete \\
\textsc{core\_am}   & 135.146         & 141.964          & 149.582          & 157.587          & 169.545          & 106.710      & 0       & complete \\
\textsc{contag}     & 53.943          & 54.455           & 54.744           & 55.064           & 55.523           & 51.172      & 0       & complete \\
\textsc{siei}       & 0.946           & 0.949            & 0.951            & 0.953            & 0.956            & 0.971       & 100     & complete  \\
\bottomrule
\end{tabular}
\end{table}
%\end{landscape}




\clearpage

%%%%%%%%%%%%%%%%%%%%%%%%%%%%%%%%%%%%%%%%%%%%%%%%%%%%%%%%%%%%%%%%%%%%%%%%%%%%%%%%%%%%%%%%%%%%%%%%

\paragraph*{Class-level Results}

In addition to the landscape-level results, I also summarized structure and patterns at the cover type level. Figures~\ref{fig:fragclass-smcm} and \ref{fig:fragclass-smcx} show a subset of metrics most useful to understanding patch characteristics at the cover type-seral stage level for the two most prevalent cover types, Sierran Mixed Conifer - Mesic and Sierran Mixed Conifer - Xeric. Boxplots depict the range of variability for the simulation period as well as the current value. See Appendix~\ref{app:full-class-results} for full tabular results for the nine focal cover types.


\subparagraph*{Sierran Mixed Conifer - Mesic} %updated analysis 2015-09-20
The spatial configuration of stand conditions fluctuated markedly over time, although there was considerable variation in the magnitude of variability among configuration metrics (see Appendix~\ref{app:full-class-results}, Table~\ref{tab:fragclass_smcm}). Early and Mid Development patches in this cover type tended to have wide ranges of variability in metric outcomes, and were larger, less fragmented, more geometrically complex, and had more core area during the HRV than during the current conditions (Figure~\ref{fig:fragclass_smcm}). Metric values for these seral stages tended to be completely or nearly outside the simulated HRV. 
% B liked my writing here
In contrast, the other seral stages all fall within the simulated HRV in terms of patch size and core area. Results for geometric complexity and fragmentation were less consistent across the other seral stages. While Late--Open stands were more geometrically complex during the HRV than on the current landscape, Mid--Moderate, Mid--Late, and Late--Moderate patches were all less geometrically complex. Late--Closed patches currently fall within the simulated HRV. Meanwhile, the open canopy seral stages are currently within the HRV in terms of fragmentation, while the Mid--Moderate, Late--Closed, and Late--Moderate stages are all currently completely departed the range of variability and more fragmented today than during the simulated HRV.  

\begin{figure}[!htbp]
  \centering
  \subfloat[][]{
    %\centering
    \includegraphics[width=0.7\textwidth]{/Users/mmallek/Documents/Thesis/Plots/fragclass-bymetrics/HRV/SMC_M-AREA_AM-boxplots.png}
    \label{fig:smcm_areaam}
  } \\
  \subfloat[][]{
    %\centering
    \includegraphics[width=0.7\textwidth]{/Users/mmallek/Documents/Thesis/Plots/fragclass-bymetrics/HRV/SMC_M-CORE_AM-boxplots.png}
    \label{fig:smcm_coream}
  } \\
  \subfloat[][]{
    \includegraphics[width=0.7\textwidth]{/Users/mmallek/Documents/Thesis/Plots/fragclass-bymetrics/HRV/SMC_M-SHAPE_AM-boxplots.png}
    \label{fig:smcm_shapeam}
  } \\
  \subfloat[][]{
    %\centering
    \includegraphics[width=0.7\textwidth]{/Users/mmallek/Documents/Thesis/Plots/fragclass-bymetrics/HRV/SMC_M-CLUMPY-boxplots.png}
    \label{fig:smcm_clumpy}
  }%
\caption{Fragstats class-level results for Sierran Mixed Conifer - Mesic. (a) area-weighted mean patch area (AREA\_AM) (b) area-weighted mean core area (CORE\_AM) (c) area-weighted mean shape index (SHAPE\_AM) (d) clumpiness (CLUMPY). Boxplot whiskers extend from the $5^{\text{th}}-95^{\text{th}}$ percentile of the observed distribution. The thick grey bar denotes the metric value on the current landscape.}
\label{fig:fragclass_smcm}
\end{figure}


%\clearpage
%%%%%%%%%%%%%%%%%%%%%%%%%%%%%%%%%%%%%%%%%%%%%%%%%%%%%%%%%%%%%%%%%%%%%%%%%%%%%%%%%%%%%%%%%%%%%%%%


\subparagraph*{Sierran Mixed Conifer - Xeric}
The spatial configuration of stand conditions fluctuated markedly over time as well, although there was considerable variation in the magnitude of variability among configuration metrics (see Appendix~\ref{app:full-class-results}, Table~\ref{tab:fragclass_smcx}). Early Development and Mid--Open had wide ranges of variability in patch and core area size, while Mid--Closed had a wide range of variability in geometric complexity and fragmentation. In contrast to the mesic mixed conifer forests, results in this cover type were consistent across different metrics. Mid--Closed, Mid--Moderate, and Late--Moderate stages currently fall within the simulated HRV in terms of area-weighted mean patch size and core area, as well as for the shape and clumpiness indices. However, the other stages were generally currently completely departed from the simulated HRV or moderately departed within the simulated HRV. Early successional and open canopy stands are currently smaller, more fragmented, less geometrically complex, and have less core area than during the simulated HRV, while the opposite is true for Late--Closed patches (Figure~\ref{fig:fragclass_smcx}).

% figures updated 2015-09-20
\begin{figure}[!htbp]
  \centering
  \subfloat[][]{
    %\centering
    \includegraphics[width=0.7\textwidth]{/Users/mmallek/Documents/Thesis/Plots/fragclass-bymetrics/HRV/SMC_X-AREA_AM-boxplots.png}
    \label{fig:smcx_areaam}
  } \\
  \subfloat[][]{
    %\centering
    \includegraphics[width=0.7\textwidth]{/Users/mmallek/Documents/Thesis/Plots/fragclass-bymetrics/HRV/SMC_X-CORE_AM-boxplots.png}
    \label{fig:smcx_coream}
  } \\
  \subfloat[][]{
    \includegraphics[width=0.7\textwidth]{/Users/mmallek/Documents/Thesis/Plots/fragclass-bymetrics/HRV/SMC_X-SHAPE_AM-boxplots.png}
    \label{fig:smcx_shapeam}
  } \\
  \subfloat[][]{
    %\centering
    \includegraphics[width=0.7\textwidth]{/Users/mmallek/Documents/Thesis/Plots/fragclass-bymetrics/HRV/SMC_X-CLUMPY-boxplots.png}
    \label{fig:smcx_clumpy}
  }%
\caption{Fragstats class-level results for Sierran Mixed Conifer - Xeric. (a) area-weighted mean patch area (AREA\_AM) (b) area-weighted mean core area (CORE\_AM) (c) area-weighted mean shape index (SHAPE\_AM) (d) clumpiness (CLUMPY). Boxplot whiskers extend from the $5^{\text{th}}-95^{\text{th}}$ percentile of the observed distribution. The thick grey bar denotes the metric value on the current landscape.}
\label{fig:fragclass_smcx}
\end{figure}

\clearpage
%