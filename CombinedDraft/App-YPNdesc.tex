% !TEX root = master.tex
\newpage
\section{Yellow Pine (YPN)}
\label{ypn-description}

\subsection*{General Information}

\subsubsection{Cover Type Overview}

\textbf{Yellow Pine (YPN)}
\newline
Crosswalks
\begin{itemize}
	\item EVeg: Regional Dominance Type 1
	\begin{itemize}
		\item Eastside Pine
		\item Jeffrey Pine
		\item Ponderosa Pine
	\end{itemize}

	\item LandFire BpS Model
	\begin{itemize}
		\item Yellow Pine
	\end{itemize}

	\item Presettlement Fire Regime Type
	\begin{itemize}
		\item 0610310 California Montane Jeffrey Pine (-Ponderosa Pine) Woodland
	\end{itemize}

	\item Only occurs on the east side of the Sierra crest.
\end{itemize}

\noindent \textbf{Yellow Pine with Aspen (YPN-ASP)}
\newline
This type is created by overlaying the NRIS TERRA Inventory of Aspen on top of the EVeg layer. Where it intersects with YPN it is assigned to YPN-ASP.

\noindent Reviewed by Hugh Safford, Regional Ecologist, USDA Forest Service

\subsubsection{Vegetation Description}
\textbf{Yellow Pine (YPN)}	This landcover type is characterized by yellow pine species such as \emph{Pinus ponderosa} or \emph{Pinus jeffreyi} that occur on the east side of the Sierra crest (LandFire 2007a). Relatively pure stands of yellow pine may occur, or they may mix with other tree species including \emph{Abies concolor, Juniperus occidentalis, Pinus contorta} ssp. \emph{murrayana}, and \emph{Quercus kelloggi} (Fites-Kaufman et al. 2007, Fitzhugh 1988). Their understory may include both montane forest and Great Basin shrubs, including but not limited to \emph{Ceanothus, Arctostaphylos, Symphoricarpos, Artemisia tridentata, Purshia tridentata, Ericameria nauseosa, Cercocarpus}, and \emph{Holodiscus}. Herbaceous plants and grasses may include \emph{Wyethia, Balsamorhiza sagittata, Festuca, Calamagrostis}, and \emph{Elymus} (LandFire 2007a, Fitzhugh 1988).

Without disturbance, except for naturally occurring fire, a mosaic of uneven-aged patches develops, with open spaces and dense sapling stands (Safford 2013). \emph{Q. kelloggi} or \emph{Juniperus occidentalis} may form an understory, but pure stands of pine also are found. An open stand of low shrubs, and a grassy herb layer are typical. Crowns of pines are open, allowing light, wind and rain to penetrate, whereas other associated trees provide more dense foliage (Fitzhugh 1988).

\textbf{Yellow Pine with Aspen (YPN-ASP)} These are upland forests and woodlands dominated by \emph{Populus tremuloides} without a significant conifer component, often termed ``stable aspen.'' The understory structure may be complex with multiple shrub and herbaceous layers, or simple with just an herbaceous layer. The herbaceous layer may be dense or sparse, dominated by graminoids or forbs. Common shrubs include \emph{Acer, Amelanchier, Artemisia, Juniperus, Prunus, Rosa, Shepherdia, Symphoricarpos}, and the dwarf-shrubs \emph{Mahonia} and \emph{Vaccinium}. Common graminoids may include \emph{Bromus, Calamagrostis, Carex, Elymus, Festuca}, and \emph{Hesperostipa}. Associated forbs may include \emph{Achillea, Eucephalus, Delphinium, Geranium, Heracleum, Ligusticum, Lupinus, Osmorhiza, Pteridium, Rudbeckia, Thalictrum, Valeriana, Wyethia}, and many others (LandFire 2007b).


\subsubsection{Distribution}
\textbf{Yellow Pine}	This landcover type occurs on all aspects from about 1200 m to 1980 m (4000-6500 ft) in elevation, east of the Sierra Nevada crest (Fitzhugh 1988). It is usually found on volcanic and granitic substrates, in shallow soils with a frigid soil temperature regime (LandFire 2007a).

\textbf{Yellow Pine with Aspen}	Sites supporting \emph{P. tremuloides} are associated with added soil moisture, i.e., azonal wet sites. These sites are often close to streams, lakes, and meadows. Other sites include rock reservoirs, springs and seeps. Terrain can be simple to complex. At lower elevations, topographic conditions for this type tends toward positions resulting in relatively colder, wetter conditions within the prevailing climate, e.g., ravines, north slopes, wet depressions, etc. (LandFire 2007b). \emph{P. tremuloides} stands may also be associated with lateral or terminal moraine boulder material, talus-colluvium, rock falls, or lava flows. In addition, pure stands may be found in topographic positions where snow accumulates, mostly at higher north facing elevations, where snow presence means the growing season is too short to support conifers (Shepperd et al. 2006). 


\subsection*{Disturbances}

\subsubsection{Wildfire}
\textbf{Yellow Pine} Wildfires are common and frequent; mortality depends on vegetation vulnerability and wildfire intensity. Low mortality fires kill small trees and consume above-ground portions of shrubs and herbs, but do not kill large trees or below-ground organs of most shrubs and herbs which promptly re-sprout. High mortality fires kill large as well as small trees, and may kill many of the shrubs and herbs as well. Fire kills the above-ground portions of the shrubs and herbs, but most shrubs and herbs resprout from surviving below-ground organs. Wildfires may trigger transitions between developmental seral stages.

The relatively long needles of yellow pines and relatively open structure of theses stands make for dry surface and ground fuels that burn readily. Thus, fires in these stands burn more frequently than those in adjacent forests (Fites-Kaufman et al. 2007). In fact, fire is an integral part of the ecology of yellow pines. Fire has allowed yellow pines to dominate sites where it is the potential climax as well as sites where it would otherwise be seral to more shade-tolerant tree species. \emph{P. ponderosa} and \emph{P. jeffreyi} have evolved with a thick bark and open crown structure that allows them to survive most fires. Mature trees will self-prune, leaving a smooth bole which reduces aerial fire spread. Also, fire creates favorable seedbeds for seedling establishment (Habeck 1992). 

\medskip
\noindent \textbf{Yellow Pine with Aspen}	Sites supporting \emph{P. tremuloides} are maintained by stand-replacing disturbances that allow regeneration from below-ground suckers. Replacement fire and ground fire are thought to have been common in stable \emph{P. tremuloides} stands historically. Because \emph{P. tremuloides} is associated with mesic conditions, it rarely burns during the normal lightning season. However, during years with little precipitation stands may be more susceptible to burning. Evidence from fire scars and historical studies show that past fires occurred mostly during the spring and fall. These are typically self-perpetuating stands (LandFire 2007b)

Van de Water and Safford (2011) found a mean fire return interval of 19 years, median of 20 years, mean min interval of 10 years and mean max of 90 years for Aspen. The LandFire model for northern Sierra Nevada ``stable aspen'' predicts a mean FRI of 31 years. Replacement FRI has a mean of 68 years with a range of 50-300 years, while mixed severity FRI has a mean of 57 years with a range of 20-60 years, and low severity fire is not modeled (LandFire 2007b). We recalculated these numbers using seral stage-specific information and using only high and low mortality fire categories, which resulted in an interval of 38 years for high mortality fire, 111 years for low mortality fire, and 29 years for any fire.

Estimates of fire rotations for these variants are available from the LandFire project and a few review papers. The LandFire project’s published fire return intervals are based on a series of associated models created using the Vegetation Dynamics Development Tool (VDDT). In VDDT, fires are specified concurrently with the transition that follows them. For example, a replacement fire causes a transition to the early development stage. In the RMLands model, such fires are classified as high mortality. However, in VDDT mixed severity fires may cause a transition to early development, a transition to a more open seral stage, or no transition at all. In this case, we categorize the first example as a high mortality fire, and the second and third examples as a low mortality fire. Based on this approach, we calculated fire rotations and the probability of high mortality fire for each of the YPN and YPN-ASP seral stages (Tables~\ref{tab:ypndesc_fire} and \ref{tab:ypn-aspdesc_fire}). We computed overall target fire rotations based on values from Mallek et al. (2013) and Van de Water and Safford (2011). 




\begin{table}[]
\small
\centering
\caption{Fire rotation (years) and proportion of high (versus low) mortality fires for Yellow Pine. Values were derived from VDDT model 0610581 (LandFire 2007), Mallek et al. (2013), and Safford (personal communication). }
\label{tab:ypndesc_fire}
\begin{tabular}{@{}lcc@{}}
\toprule
\textbf{Condition}         & \multicolumn{1}{l}{\textbf{Fire Rotation}} & \multicolumn{1}{l}{\textbf{\begin{tabular}[c]{@{}l@{}}Proportion \\ High Mortality\end{tabular}}} \\ \midrule
Target                      & 21            & n/a                           \\
Early Development - All     & 30            & 1                             \\
Mid Development - Closed    & 15            & 0.26                          \\
Mid Development - Moderate  & 10            & 0.14                          \\
Mid Development - Open      & 8             & 0.05                          \\
Late Development - Closed   & 11            & 0.20                          \\
Late Development - Moderate & 9             & 0.08                          \\
Late Development - Open     & 8             & 0.01      					\\ \bottomrule
\end{tabular}
\end{table}

\begin{table}[]
\small
\centering
\caption{Fire rotation (years) and proportion of high (versus low) mortality fires for Yellow Pine - Aspen type. Values were derived from VDDT model 0610110 (LandFire 2007) and Safford (personal communication).}
\label{tab:ypn-aspdesc_fire}
\begin{tabular}{@{}lcc@{}}
\toprule
\textbf{Condition}         & \multicolumn{1}{l}{\textbf{Fire Rotation}} & \multicolumn{1}{l}{\textbf{\begin{tabular}[c]{@{}l@{}}Proportion \\ High Mortality\end{tabular}}} \\ \midrule
Target                           & 21            & n/a                           \\
Early Development - Aspen        & 30            & 1                             \\
Mid Development - Aspen          & 11            & 0.26                          \\
Late Development - Conifer-Aspen & 10            & 0.08  				      	 \\ \bottomrule
\end{tabular}
\end{table}

\subsubsection{Other Disturbance}
Other disturbances are not currently modeled, but may, depending on the seral stage affected and mortality levels, reset patches to early development, maintain existing seral stages, or shift/accelerate succession to a more open seral stage. 

\subsection*{Vegetation Seral Stages}
We recognize seven separate seral stages for YPN: Early Development (ED), Mid Development - Open Canopy Cover (MDO), Mid Development - Moderate Canopy Cover, Mid Development - Closed Canopy Cover (MDC), Late Development - Open Canopy Cover (LDO), Late Development - Moderate Canopy Cover (LDM), and Late Development - Closed Canopy Cover (LDC) (Figure~\ref{transmodel_ypn}).  The YPN -ASP variant is assigned to three seral stages: Early Development - Aspen (ED-A), Mid Development - Aspen (MD-A), and Late Development - Conifer with Aspen (LD-CA) (Figure~\ref{transmodel_ypn-asp}).

Our seral stages are an alternative to ``successional'' classes that imply a linear progression of states and tend not to incorporate disturbance. The seral stages identified here are derived from a combination of successional processes and anthropogenic and natural disturbance, and are intended to represent a composition and structural condition that can be arrived at from multiple other conditions described for that landcover type. Thus our seral stages incorporate age, size, canopy cover, and vegetation composition. In general, the delineation of stages has originated from the LandFire biophysical setting model descriptive of a given landcover type; however, seral stages are not necessarily identical to the classes identified in those models.

\begin{figure}[htbp]
\centering
\includegraphics[width=0.8\textwidth]{/Users/mmallek/Documents/Thesis/statetransmodel/StateTransitionModel/7class.png}
\caption{State and Transition Model for Yellow Pine Forest and Woodland (not inclusive of the aspen variant). Each dark grey box represents one of the seven seral stages for this landcover type. Each column of boxes represents a stage of development: early, middle, and late. Each row of boxes represents a different level of canopy cover: closed (70-100\%), moderate (40-70\%), and open (0-40\%). Transitions between states/seral stages may occur as a result of high mortality fire, low mortality fire, or succession. Specific pathways for each are denoted by the appropriate color line and arrow: red lines relate to high mortality fire, orange lines relate to low mortality fire, and green lines relate to natural succession.} 
\label{transmodel_ypn}
\end{figure}

\begin{figure}[htbp]
\centering
\includegraphics[width=0.8\textwidth]{/Users/mmallek/Documents/Thesis/statetransmodel/StateTransitionModel/3class-asp.png}
\caption{State and Transition Model for Yellow Pine Forest and Woodland, Aspen variant. Each dark grey box represents one of the three seral stages for this landcover type. Three seral stages of development are represented: early, middle, and late. Transitions between states/seral stages may occur as a result of high mortality fire, low mortality fire, or succession. Specific pathways for each are denoted by the appropriate color line and arrow: red lines relate to high mortality fire, orange lines relate to low mortality fire, and green lines relate to natural succession.} 
\label{transmodel_ypn-asp}
\end{figure}

\paragraph{Early Development (ED)}

\paragraph{Description} Grasses, forbs, low shrubs, and sparse to moderate cover of trees (primarily \emph{P. ponderosa} or \emph{P. jeffreyi}) seedlings/saplings with an open canopy. This seral stage is characterized by the recruitment of a new cohort of early successional, shade-intolerant tree species into an open area created by a stand-replacing disturbance. Following such disturbance, some sites are dominated by dense shrub stands composed of \emph{P. tridentata}, \emph{Arctostaphylos}, and/or \emph{Ceanothus}, depending on location. Other postfire sites are more open and dominated by dense pine seedlings, bunchgrasses and forbs.

\paragraph{Succession Transition} In the absence of disturbance, patches in this seral stage will begin transitioning to MDC or MDO after 40 years at a rate of 0.7 per timestep. The transition to MDO is twice as likely as transition to MDC.  At 80 years, all remaining patches will succeed to either MDC or MDO. 

\paragraph{Wildfire Transition} High mortality wildfire (100\% of fires in this seral stage) recycles the patch through the Early Development seral stage. Low mortality wildfire is not modeled for this seral stage. 

\noindent\hrulefill


\paragraph{Mid Development - Open Canopy Cover (MDO)}

\paragraph{Description} Open mid-development forest with diverse herbaceous understory and scattered woody shrubs. Conifers, primarily \emph{P. ponderosa} or \emph{P. jeffreyi}, are medium sized. Herbs and other species gradually decline as growing trees begin to shade understory. Maintained by frequent burning. Canopy cover is less than 40\% (LandFire 2007a). 

\paragraph{Succession Transition} Patches in this seral stage will maintain under low mortality disturbance, but after 20 years without fire they begin transitioning to MDM at a rate of 0.8 per time step. Succession to LDO occurs once the patch has been in mid development for 170 years. The rate of succession per time step is 0.4. After 230 years, all patches will have succeeded.

\paragraph{Wildfire Transition} High mortality wildfire (5\% of fires in this seral stage) recycles the patch through the Early Development seral stage. Low mortality wildfire (95\%) maintains the patch in MDO.

\noindent\hrulefill

\paragraph{Mid Development - Moderate Canopy Cover (MDM)}

\paragraph{Description} Mid-development forest with moderate canopy cover. Somewhat ``overstocked'' pole to large pole size trees, primarily \emph{P. ponderosa} or \emph{P. jeffreyi}, susceptible to stagnation. Marginal understory associated with limited site resources. Develops where fire frequency is too low to thin small trees. Canopy cover is 40-70\% (LandFire 2007a).

\paragraph{Succession Transition} Patches in this seral stage may maintain under low mortality disturbance, but after 20 years without fire they begin transitioning to MDC at a rate of 0.8 per time step. At 130 years since succession to a mid development seral stage, these patches will begin transitioning to LDC. The rate of succession per time step is 0.3. After 230 years, all patches will have succeeded.

\paragraph{Wildfire Transition} High mortality wildfire (14\% of fires in this seral stage) recycles the patch through the Early Development seral stage. Low mortality wildfire (86\%) opens the stand up to MDO 32\% of the time; otherwise, the patch remains in MDC.

\noindent\hrulefill

\paragraph{Mid Development - Closed Canopy Cover (MDC)}

\paragraph{Description} Dense mid-development forest. ``Overstocked'' pole to large pole size trees, primarily \emph{P. ponderosa} or \emph{P. jeffreyi}, susceptible to stagnation. Marginal understory associated with limited site resources. Develops where fire frequency is too low to thin small trees. Canopy cover is over 70\% (LandFire 2007a).

\paragraph{Succession Transition} At 100 years since succession to a mid development seral stage, these patches will begin transitioning to LDC. The rate of succession per time step is 0.2. After 200 years, all patches will have succeeded.

\paragraph{Wildfire Transition} High mortality wildfire (26\% of fires in this seral stage) recycles the patch through the Early Development seral stage. Low mortality wildfire (74\%) opens the stand up to MDM 60\% of the time; otherwise, the patch remains in MDC.

\noindent\hrulefill


\paragraph{Late Development - Open Canopy Cover (LDO)}

\paragraph{Description} Open late-development forest with large and very large trees, primarily \emph{P. ponderosa} or \emph{P. jeffreyi}. Trees grow in often widely spaced clumps and the understory is open and often diverse. Surface fuels are limited due to frequent burning. Canopy cover is less than 40\% (LandFire 2007a, Safford 2013).

\paragraph{Succession Transition} Patches in this seral stage will maintain under low mortality disturbance, but after 25 years without fire, these patches succeed to LDM at a rate of 0.7 per timestep.

\paragraph{Wildfire Transition} High mortality wildfire (1\% of fires in this seral stage) recycles the patch through the Early Development seral stage. Low mortality wildfire (99\%) maintains the patch in LDO.

\noindent\hrulefill

\paragraph{Late Development - Moderate Canopy Cover (LDM)}

\paragraph{Description} Open late-development forest with large and very large trees, primarily \emph{P. ponderosa} or \emph{P. jeffreyi}. Trees grow in often widely spaced clumps, although they are becoming more dense, and the understory is fairly open and often diverse. Surface fuels are accumulating. Canopy cover is 40-70\% (LandFire 2007a, Safford 2013).

\paragraph{Succession Transition} Patches in this seral stage may maintain under low mortality disturbance, but after 25 years without fire, these patches succeed to LDC at a rate of 0.7 per timestep.

\paragraph{Wildfire Transition} High mortality wildfire (8\% of fires in this seral stage) recycles the patch through the Early Development seral stage. Low mortality wildfire (92\%) opens the stand up to LDO 18\% of the time; otherwise, the patch remains in LDM.

\noindent\hrulefill

\paragraph{Late Development - Closed Canopy Cover (LDC)}

\paragraph{Description} Dense late-development forest, primarily \emph{P. ponderosa} or \emph{P. jeffreyi} with large and very large trees, sometimes with significant within-stand mortality. Substantial surface fuel accumulation and ladder fuels. Canopy cover exceeds 70\% (LandFire 2007a).

\paragraph{Succession Transition} Patches in this seral stage will maintain in the absence of disturbance.

\paragraph{Wildfire Transition} High mortality wildfire (20\% of fires in this seral stage) recycles the patch through the Early Development seral stage. Low mortality wildfire (80\%) opens the stand up to LDM 58\% of the time; otherwise, the patch remains in LDC.

\noindent\hrulefill
\noindent\hrulefill

\subsubsection{Aspen Variant}

\paragraph{Early Development - Aspen (ED-A)}

\paragraph{Description} Grasses, forbs, low shrubs, and sparse to moderate cover of tree seedlings/saplings (primarily \emph{P. tremuloides}) with an open canopy. This seral stage is characterized by the recruitment of a new cohort of early successional, shade-intolerant tree species into an open area created by a stand-replacing disturbance. 

Following disturbance, succession proceeds rapidly from an herbaceous layer to shrubs and trees, which invade together (Verner 1988). \emph{P. tremuloides} suckers over 6ft tall develop within about 10 years (LandFire 2007b). 

\paragraph{Succession Transition} Unless it burns, a patch in the early seral stage persists for 10 years, at which point it transitions to MD-A.

\paragraph{Wildfire Transition} High mortality wildfire (100\% of fires in this seral stage) recycles the patch through the ED-A seral stage. Low mortality wildfire is not modeled for this seral stage.

\noindent\hrulefill


\paragraph{Mid Development - Aspen (MD-A)}

\paragraph{Description} \emph{P. tremuloides} trees 5-16'' DBH. Canopy cover is highly variable, and can range from 40-100\%. These patches range in age from 10 to 110 years. (LandFire 2007b).

\paragraph{Succession Transition} Patches in the MD-A seral stage persist for at least 80 years in the absence of any fire, after which they begin transitioning to LD-CA at a rate of 0.6 per timestep. After 130 years without fire all remaining MD-A patches transition to LD-CA. 

\paragraph{Wildfire Transition} High mortality wildfire (26\% of fires in this seral stage) recycles the patch through the ED-A seral stage. No transition occurs as a result of low mortality fire (74\%).

\noindent\hrulefill


\paragraph{Late Development - Aspen with Conifer (LD-AC)}

\paragraph{Description} If stands are sufficiently protected from fire such that conifer species overtop \emph{P. tremuloides} and become large, they may be able to withstand some fire that more sensitive \emph{P. tremuloides} cannot. When this occurs, it creates a patch characterized by late development conifers, such as \emph{P. contorta} ssp. \emph{murrayana}, and early seral \emph{P. tremuloides}. 

\paragraph{Succession Transition} LD-CA persists for 70 years in the absence of any fire, at which point patches transition to LDC. 

\paragraph{Wildfire Transition} High mortality wildfire (13\% of fires in this seral stage) returns the patch to ED-A. Low mortality wildfire (87\%) maintains the stand in LD-CA. 

\noindent\hrulefill




\subsection*{Seral Stage Classification}
\begin{table}[hbp]
\small
\centering
\caption{Classification of seral stage for YPN. Diameter at Breast Height (DBH) and Cover From Above (CFA) values taken from EVeg polygons. DBH categories are: null, 0-0.9'', 1-4.9'', 5-9.9'', 10-19.9'', 20-29.9'', 30''+. CFA categories are null, 0-10\%, 10-20\%, \dots , 90-100\%. Each row in the table below should be read with a boolean AND across each column.}
\label{ypn_classification}
\begin{tabular}{@{}lrrrrr@{}}
\toprule
\textbf{\begin{tabular}[l]{@{}l@{}}Cover \\ Condition\end{tabular}} & \textbf{\begin{tabular}[r]{@{}r@{}}Overstory Tree \\ Diameter 1 \\ (DBH)\end{tabular}} & \textbf{\begin{tabular}[r]{@{}r@{}}Overstory Tree \\ Diameter 2 \\ (DBH)\end{tabular}} & \textbf{\begin{tabular}[r]{@{}r@{}}Total Tree\\ CFA (\%)\end{tabular}} & \textbf{\begin{tabular}[r]{@{}r@{}}Conifer \\ CFA (\%)\end{tabular}} & \textbf{\begin{tabular}[r]{@{}r@{}}Hardwood \\ CFA (\%)\end{tabular}} \\ \midrule
Early All        & 0-4.9''         & any & any    & any & any \\
Mid Open         & 5-19.9''        & any & 0-40   & any & any \\
Mid Moderate     & 5-19.9''        & any & 40-70  & any & any \\
Mid Closed       & 5-19.9''        & any & 70-100 & any & any \\
Late Open        & 20-40''+        & any & 0-40   & any & any \\
Late Moderate    & 20-40''+        & any & 40-70  & any & any \\
Late Closed      & 20-40''+        & any & 70-100 & any & any \\ \bottomrule
\end{tabular}
\end{table}

YPN-ASP seral stages were assigned manually using NAIP 2010 Color IR imagery to assess seral stage.



\clearpage

\subsection*{References}

\begin{hangparas}{.25in}{1} 
\interlinepenalty=10000

Fitzhugh, E. Lee. ``Eastside Pine (EPN).'' \emph{A Guide to Wildlife Habitats of California}, edited by Kenneth E. Mayer and William F. Laudenslayer. California Deparment of Fish and Game, 1988. \burl{http://www.dfg.ca.gov/biogeodata/cwhr/pdfs/EPN.pdf}. Accessed 4 December 2012.

Habeck, R. J. ``Pinus ponderosa var. ponderosa.'' \emph{Fire Effects Information System}, U.S. Department of Agriculture, Forest Service,  Rocky Mountain Research Station, Fire Sciences Laboratory, 1992. \burl{http://www.fs.fed.us/database/feis/plants/tree/quekel/all.html}. Accessed 21 December 2012.

LandFire. ``Biophysical Setting Models.'' Biophysical Setting 0610310: California Montane Jeffrey Pine (-Ponderosa Pine) Woodland. 2007a. LANDFIRE Project, U.S. Department of Agriculture, Forest Service; U.S. Department of the Interior. \burl{http://www.landfire.gov/national_veg_models_op2.php}. Accessed 9 November 2012.

LandFire. ``Biophysical Setting Models.'' Biophysical Setting 0610110: Rocky Mountain Aspen Forest and Woodland. 2007b. LANDFIRE Project, U.S. Department of Agriculture, Forest Service; U.S. Department of the Interior. \burl{http://www.landfire.gov/national_veg_models_op2.php}. Accessed 7 January 2013.

Safford, Hugh S. Personal communications, 5 May 2013, 26 July 2013, 15 August 2013.

Shepperd, Wayn De, Paul C. Rogers, David Burton, and Dale L. Bartos. ``Ecology, Biodiversity, Management, and Restoration of Aspen in the Sierra Nevada.'' General Technical Report RMRS-GTR-178. Rocky Mountain Research Station, Forest Service, U.S. Department of Agriculture, 2006.

Skinner, Carl N. and Chi-Ru Chang. ``Fire Regimes, Past and Present.'' \emph{Sierra Nevada Ecosystem Project: Final report to Congress, vol. II, Assessments and scientific basis for management options}. Davis: University of California, Centers for Water and Wildland Resources, 1996.

Van de Water, Kip M. and Hugh D. Safford. ``A Summary of Fire Frequency Estimates for California Vegetation Before Euro-American Settlement.'' \emph{Fire Ecology} 7.3 (2011): 26-57. doi: 10.4996/fireecology.0703026.

Verner, Jared. ``Aspen (ASP).'' \emph{A Guide to Wildlife Habitats of California}, edited by Kenneth E. Mayer and William F. Laudenslayer. California Deparment of Fish and Game, 1988. \burl{http://www.dfg.ca.gov/biogeodata/cwhr/pdfs/ASP.pdf}. Accessed 4 December 2012.

\end{hangparas}

