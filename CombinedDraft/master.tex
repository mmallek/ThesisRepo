\documentclass[12pt]{book}
\usepackage[left=1in,right=1in,top=1in,bottom=1in]{geometry}
\usepackage{latexsym}
\usepackage{listings}
\lstset{breaklines=true, language=Python, showspaces=false, showstringspaces=false}
\usepackage{wrapfig,subfig,graphicx} %check
\usepackage{textcomp}
\usepackage{sidecap}
\usepackage{booktabs, colortbl} %check
\usepackage[table,xcdraw,dvipsnames]{xcolor} %check
\usepackage{multirow}
\usepackage{changepage}
\usepackage{amsmath}
%\usepackage[textwidth=0.8in]{todonotes}
\usepackage[]{todonotes}
\usepackage{siunitx}
\usepackage[figuresright]{rotating}
\usepackage[format=hang,labelfont=bf,size=small]{caption}
%\setlength{\textfloatsep}{40pt plus 2.0pt minus 4.0pt}
\usepackage[modulo]{lineno}
\usepackage{hanging}
\usepackage{changepage}
\usepackage{lipsum}

\usepackage{hyperref}
%\usepackage{cleverref}
\newcommand\myshade{85}
\colorlet{mylinkcolor}{violet}
\colorlet{mycitecolor}{Aquamarine}
\colorlet{myurlcolor}{YellowOrange}
\usepackage{breakurl}

\hypersetup{colorlinks=true, citecolor=mycitecolor!\myshade!black, linkcolor=mylinkcolor!\myshade!black, urlcolor = myurlcolor!\myshade!black}

\usepackage{pdflscape}
\usepackage{longtable}
\usepackage{natbib}


\setlength{\marginparwidth}{2cm}
\linenumbers

\begin{document}

\author{Maritza Mallek}
\title{Modeling Historical Range of Variability and Future Climate Change in the Yuba River Watershed, Tahoe National Forest}
\date{March 2016}

\maketitle

\section*{Abstract}
In California's northern Sierra Nevada mountains, the historical processes of fire-dependent forest ecosystems have been interrupted and altered by changing human land use and fire suppression policies. In directing National Forest units to plan for a future that includes multiple use and restoration of resilient ecosystems, Forest Service policy directs land managers to explore the range of variability of ecological processes at multiple scales, and to use the results as a tool in restoration and planning. Quantifying the degree to which human actions may shift a system out of its natural range of variability is an important part of such evaluations. Climate also underlies and influences ecosystem processes related to wildfires and forest succession. A changing climate means that cycles of drought and wet periods are shifting, and some climate variables will shift from oscillation around a steady mean to changing along a positive trajectory.

In this thesis, I adapted the Rocky Mountain Landscape Simulator (\textsc{RMLands}), a spatially explicit, stochastic, landscape disturbance and succession model, for use in the Sierra Nevada. I merged theoretical ecosystem concepts with empirical data to generate inputs to \textsc{RMLands} and simulate landscape change on a portion of the Tahoe National Forest in California. I simulated wildfire and forest succession under the historical climate settings and under a set of alternative climate trajectories based on the Representative Concentration Pathway RCP8.5 prjoections. I then quantified the historical and the future range of variability in the disturbance regime, seral stage distrbution, and patch configuration, and assessed departure of the current landscape from the ranges of variability using R and \textsc{Fragstats} statistical software.

I found that fire burned more frequently and across larger extents during the simulated historical period than in the last 100 years, and that burned area and the proportion of high severity burned area increased with increasingly warm and dry climate scenarios. Results from both the historical range of variability and future range of variability analysis indicate that overall, the current landscape departs from either historical or future conditions across a range of statistical measures, due to the effects of fire on the vegetation. Based on these findings, I suggest managers implement more aggressive restoration efforts and utilize mitigation measures where the consequences of changing fire regimes are socially unacceptable. Managers and the public will be faced with decisions about how to balance the needs of different ecosystems as well as the communities that inhabit them. My study results can be used to inform goals and specific strategies in restoration planning and help project-level planners think about impacts at the landscape scale. Once restoration activities have been implemented, the study landscape can be re-analyzed to assess success in shifting the landscape to conditions more typical of those in the range(s) of variability. [456 words, max is 350]

\frontmatter 


\tableofcontents 
\listoffigures
\listoftables
%\include{preface}

\mainmatter 







% !TEX root = master.tex

\chapter{Introduction} %Becky says this is the lit review. OK that some things might be methods more typically.
\label{CH1}
\setcitestyle{notesep={;},aysep={}}

\newcommand{\source}[1]{%
  \nobreak\parbox[t]{\linewidth}{\raggedleft #1}% Placing a quote source
}%

\begin{quote}
\emph{“In California, vegetation is the meeting place of fire and ecosystems. The plants are the fuel and fire is the driver of vegetation change. Fire and vegetation are often so interactive that they can scarcely be considered separately from each other.”} 
\medskip
  \source{%
    ---M.G. Barbour, B. Pavlik, F. Drysdale, and S. Lindstrom, 1993 (quoted in \citet{Sugihara2006}) \\
    %
  }
\end{quote}



\section{Thesis organization and project objectives}\todo{keep section heading?}

The overall purpose of this project was to quantify the historical range of variability (HRV) in landscape structure in the Yuba River watershed on the Tahoe National Forest and evaluate the relative effects of alternative future climate scenarios on landscape structure. My first objective was to synthesize the empirical and expert knowledge on disturbance and succession processes characteristic of the pre-settlement period in the ecoregion containing the Yuba River watershed. The introduction to this thesis includes a broad introduction to the study area and the motivation for this study. I include a description of the physical and biological geography of the study area and how range of variability analyses are relevant to forest and restoration planning efforts. I introduce the concept of ``range of variability'' and outline a methodology for describing the historical and potential future ranges of variability in the disturbance regime and landscape pattern.

My second objective was to quantify the historical range of variability in the disturbance regime, landscape composition, and landscape configuration in the watershed. In Chapter 2, I provide a detailed presentation of the methodology used in developing the historical range of variability. This includes the development of the input spatial data layers, selection of values for model parameterization, model calibration and execution, and the suite of tools used to conduct the analysis. Results are presented by focusing first on the disturbance regime and second on the vegetation response. A more focused analysis of the two most extensive cover types, Sierran Mixed Conifer - Mesic and Sierran Mixed Conifer - Xeric, which are key components of Sierra Nevada forests and thus a major focus for resource managers \citep{North2010}, are included in the main body of this chapter. However, analysis and interpretation for seven additional cover types is included in Appendix~\ref{app:full-results}. 

My third objective was to quantify a future range of variability by varying the climate parameter value used in the model based on values representing a suite of alternative climate trajectories, while maintaining all other parameters and model methodology used in the HRV analysis. This study comprises Chapter 3, and has been written in the format of a journal article to facilitate eventual submission. Results and analysis focus on the disturbance regime for the landscape as a whole, the seral stage distribution of two key cover types (Sierran Mixed Conifer - Mesic and Sierran Mixed Conifer - Xeric), and the early seral patch configuration for those two cover types. The emphasis is on comparing the future scenario results to the current condition. 

My fourth and final objective was to interpret the results, develop management recommendations, and consider management implications. The dicussion sections of both Chapter 2 and Chapter 3 include my overall assessment of the landscape under the historical and future ranges of variability, a comparison to the current conditions, and a set of management recommendations. 


%%%%%%%%%%%%%%%%%%%%%%%%%%%%%%%%%%%%%%%%%%%%%%%%%%%%%%%%%%%%%%%%%%%%%%%%%%%%%%%%%%%
\section{Study Area}

\subsection{Physical Geography}

\begin{figure}[!htbp]
\includegraphics[width=\textwidth]{/Users/mmallek/Tahoe/Report3/images/studyarea.png}
\caption{The Sierra Nevada Ecoregion is outlined in red. The study area (outlined in black) is located in the northern extent of the Sierra Nevada on the Tahoe National Forest, comprising the Yuba River watershed.}
\label{projectarea}
\end{figure}

The Sierra Nevada is a major North American mountain range and ecological region, located east of California's Central Valley and extending from Fredonyer Pass in the north to southern Kern County in the south. Much of the Sierra Nevada is reserved as federally-held public land, managed by the U.S. Forest Service, the Bureau of Land Management, and the National Park Service. The Plumas and Tahoe National Forests are located in the northern portion of the Sierra Nevada. The study area (see Figure~\ref{projectarea}) is located on the northern part of the Tahoe National Forest, on the Yuba River and Sierraville Ranger Districts, and comprises about 181,550 ha. It is defined by a set of three HUC-5 watersheds\footnote{HUCs are Hydrologic Unit Codes that refer to a nested system of watersheds in the United States, defined by the United States Geological Survey. The HUC-2 watershed scale includes the largest defined watersheds, which are then subdivided down to a smallest level of HUC-16. HUCs are commonly used by agencies like the Forest Service to organize land management.}, the Upper North Yuba River, the Middle Yuba River, and the Lower North Yuba River, all of which are collectively referred to in this document as the Upper Yuba River watershed. 

The topography of the study area consists of rugged mountains incised by two major and a few minor river drainages. Elevation ranges from about 350--2500 m. The area receives 30--260 cm of precipitation annually, most of which falls as snow in the middle to upper elevations \citep{Storer1963}. Like the rest of the Sierra Nevada, the study area has a Mediterranean climate, in which summer drought typically persists from May to September. This annual drought translates into increased importance for the development of a significant snowpack during the winter months, since snowmelt runoff is a key source of soil moisture during the late spring and summer months \citep{Minnich2007,Skinner1996}. In the Sierra Nevada, the heaviest precipitation occurs to the east and north of the San Francisco Bay area; my study area is within this region \citep{VanWag2006}. Datasets of the 30-year normal precipitation at 800 m resolution for the northern Sierra (obtained from the Oregon State PRISM for a side project), illustrate that particularly high amounts of precipitation falls across the middle elevations of the study area compared to the larger region \citep{PRISMClimateGroup2004}. This increased moisture contributes to the occurrence of exceptionally productive patches of forest \citep[][ Alan Doerr, personal communication]{Littell2012}.




%%%%%%%%%%%%%%%%%%%%%%%%%%%%%%%%%%%%%%%%%%%%%%%%%%%%%%%%%%%%%%%%%%%%%%%%%%%%%%%%%%%

\paragraph{Current Management Context}

The arrival of Europeans in the 1850s sparked a transformation of the landscape as the new settlers harvested timber, extracted gold using hydraulic mining techniques, and suppressed wildfires at ever-increasing scales \citep{Storer1963} (Figure~\ref{figs:historicalphotos}). 
\begin{figure}[!htbp]
  \centering
  \subfloat[][]{
    \centering
		\includegraphics[height=2in]{/Users/mmallek/Documents/Thesis/Seminar/oldsugarpinerailroad.jpg}
    %\label{fig:covermap}
  } 
  \subfloat[][]{
    \includegraphics[height=2in]{/Users/mmallek/Documents/Thesis/Seminar/oldsheepgrazing.jpg}
   	% \label{fig:conditionmap}
   } \\
  \subfloat[][]{
    \includegraphics[height=2in]{/Users/mmallek/Documents/Thesis/Seminar/historicalfiresuppression.jpg}
    %\label{fig:conditionmap}
   }
     \subfloat[][]{
    \includegraphics[height=2in]{/Users/mmallek/Documents/Thesis/Seminar/retardantUSFS.jpg}
    %\label{fig:conditionmap}
    }
   \caption{(a) Harvested  sugar pine trees awaiting transport by rail to markets. Timber harvest historically focused on the most valuable trees, which in this area tended to be fire resistant species such as sugar pine and ponderosa pine. (b) Hundreds of thousands of sheep were grazed on mountain meadows. Grazing removes fine fuels, altering the fire regime by reducing the frequency and extent of wildfire. (c) Fire suppression to protect natural resources, especially timber, was implemented soon after the arrival of Europeans. (d) After World War II, fire suppression technology became more advanced, and most fires were quickly put out.} 
\label{figs:historicalphotos}
\end{figure}

Today, forestry, mining, grazing, and dozens of recreational activities, including hunting, mountain biking, and hiking, all take place under the multiple-use mandate of the Tahoe National Forest. Grazing allotments also exist within the study area. In addition, 
%231,368 hectares inside of the project area + bufferhave non-Forest Service ownership
%125,637 hectares inside project area only are FS
%project area is 181,550 so 
%percent of project area not in USFS is 0.3079757643
about 30\% of the land inside the study area is not held by the U.S. Forest Service. Many of these lands were privately owned, often by timber companies, before the original forest reserve was created. Land within the reserve was occasionally given to other public or private entities, leading to a ``checkerboard'' pattern of public versus private ownership persists today (Figure~\ref{ownership}). Mining of gold and other minerals also continues. These economic activities affect and interact with ongoing vegetation succession and disturbance processes in the area \citep{USDAForestService2014}.

\begin{figure}[!htbp]
\centering
\includegraphics[width=0.4\textheight]{/Users/mmallek/Tahoe/Report3/images/ownership_resized.png}
\caption{Map of National Forest lands and lands held by other entities (including private, industry, and other public land). Forest lands are in green, with other ownership in tan. The boundary of the study area is in black; this image shows the 10 km buffer.} 
\label{ownership}
\end{figure}

In the western Sierra Nevada, foothill communities and lower elevation oak-conifer woodlands have experience a loss of species diversity, fragmentation, and outright habitat conversion due to the overlap with private lands and population growth. Middle elevation forests were and are more affected by mining and forestry; most easily accessible trees were probably cut before national forests were established \citep{SNEP1996}. The wildfire regime has been significantly altered in hardwoods, yellow pine, and mixed conifer forests \citep{Merriam2013,Safford2013}, and much less so in red fir and subalpine forests \citep{Meyer2013,Meyer2013a}. However, other human activities since the late 1800s have altered the structure of western Sierra Nevada forests, most notably to simplify it in several ways, including a decrease in species, multi-story canopies, and snags \citep{SNEP1996}. These activities are related mainly to timber harvest and to the extensive network of roads constructed to support timber harvest, fire control, and recreation. This simplification of landscape structure may have a negative impact on wildlife and potentially lead to a loss of biodiversity in forests \citep{Thompson2003,Manley2004,Hunter2011}. The inherent heterogeneity of both abiotic and biotic characteristics of the Sierra Nevada complicates efforts to measure the effects of human-caused fragmentation, as Sierran forests tend to be somewhat patchy even in the absence of human alterations \citep{Franklin1996}. In the Pacific Northwest, ``old-growth'' connotes very large blocks of uniformly very old trees. However, in the Sierra Nevada ``old-growth'' indicates not only the presence of very large and old trees, but also a complex, patchy, ``messy'' forest of varying age classes, species, fuel quantities, and vegetation structure \citep{SNEP1996}.

Although many uses of the forest led to changes in vegetation structure and composition, logging and wildfire suppression in combination have altered the historical fire regime and vegetation patterns most significantly \citep{Storer1963,Stephens2015,Knapp2013,Hessburg2005}. These activities persist today. Clearcutting, shelterwood, salvage cutting, and plantation management have been major components of timber management on the Forest \citep{USDAForestService2014}. Between 1988 and 2002, timber sales in the Sierra Nevada as a whole dropped drastically, but on the Tahoe National Forests timber sale levels have fluctuated both up and down (although annual sawtimber sold has decreased similarly to other Sierran Forests) \citep{USDAForestService2004}. Although very large fires have burned in the Sierra Nevada recently (e.g., the 2013 Rim Fire), few large fires have impacted the Tahoe National Forest in the last 100 years \citep{USDAForestService1990}. In 1960, approximately 40,500 ha burned on the Forest. The low total burned acreage is despite fairly high fire starts (both human- and lightning-caused), indicating that suppression efforts have been very successful \citep{USDAForestService1990,calfire2012,usgs-fire-data2012}. The 1999 Pendola Fire burned a total of 4,735 ha. The final fire perimeter included a total of 1,565 ha on the Tahoe and Plumas National Forests \citep{Shaw2009,USDOJ2009}. Post-fire analysis of the burn in the Tahoe National Forest around the Bullards Bar Reservoir and west of Camptonville, CA quantified the total area burned at high severity at 70\%, prompting the need for restoration actions \citep{USDAForestService1999}. 




%\clearpage

%%%%%%%%%%%%%%%%%%%%%%%%%%%%%%%%%%%%%%%%%%%%%%%%%%%%%%%%%%%%%%%%%%%%%%%%%%%%%%%%%%%
\subsection{Disturbance Regime}
\paragraph{Fire in the Sierra Nevada}
In the Sierra Nevada, cycles of fire and vegetation recovery occur variably over large extents, as well as over long periods of time. Ongoing disturbance results in increased heterogeneity, which can be quantified by various metrics used to describe vegetation composition and configuration \citep{Monica2008}. Prior to European settlement, wildfire was the major source of disturbance in Sierran forests, shaping the composition and configuration of vegetation communities. Fires were primarily lightning-caused, although indigenous peoples are thought to have set fires for vegetation management, especially in the lower elevations \citep{Safford2013}. 

\begin{figure}[!htbp]
  \centering
  \subfloat[][]{
    \centering
		\includegraphics[height=2in]{/Users/mmallek/Documents/Thesis/Seminar/lowseverityfire-AlanTaylor.jpg}
    \label{fig:lowseverity}
  } 
  \subfloat[][]{
    \includegraphics[height=2in]{/Users/mmallek/Documents/Thesis/Seminar/smokywoodsUSFS.jpg}
    \label{fig:highseverity}
   } 
   \caption{(a) Low severity fires have short flame heights, staying near the ground. Fire primarily consumes fuels and understory vegetation. Photo by Alan Taylor. (b) 2013 Rim Fire burning near a highway. Fire consumes fuels and both understory and overstory vegetation. Photo from USFS Region 5.} 
\label{figs:fireseverity}
\end{figure}

In general, regardless of vegetation type, fires during the pre-settlement period were thought to burn primarily at low intensities (Figure~\ref{fig:lowseverity}). High mortality (over 70\% overstory canopy mortality) was uncommon \citep{Skinner1996, SNEP1996a,Mallek2013,Stephens2015}. Under this disturbance regime, stand-replacing fire initiated early development conditions on the landscape, but since they were uncommon, most fires only affected the understory by removing fuels \citep{Skinner1996, SNEP1996a,Mallek2013}. In some cases, individual trees or groups of tree could ``torch'' and burn down during a fire that otherwise consumed ground fuels. This is sometimes referred to as a ``moderate severity'' fire. In these instances, fire thinned the forest. Trees left alive after fire were more widely spaced. Thus the overall age of the stand, defined as the age of the overstory trees, would be unaffected, but succession would be reset to an extent among the understory vegetation. Fires of moderate severity, that result in more open overstory canopy cover, were more prevalent in more xeric environments, including south-facing slopes and ridges \citep{Mallek2013,Safford2014,SNEP1996a,SNEP1996}. Where fires did not recur frequently or only occurred at very low severity levels, succession processes such as infill or overstory growth led to gradual closing of the overstory tree canopy. It should be noted that closed canopy forests could be even-aged or multi-aged; the term ``closed'' alone does not distinguish fine-scale heterogeneity at a finer-scale than the 30 m pixel used in my model (as explained in Chapter~\ref{sec:hrvmethods}). For most of the cover types in the study area, high severity fire rates were low, allowing stands to succeed into late development and old growth conditions with a variety of canopy structures \citep{Mallek2013,Safford2014,SNEP1996a,SNEP1996}. 


Fire rotations, defined as the time it takes to burn an area equivalent to the total area under study \citep{Agee1993}, have been calculated for the major cover types in the study area \citep{Mallek2013}. Wildfires were frequent, with a mean rotation as short as 20 years in \emph{Pinus ponderosa} (Ponderosa pine)-dominated forests. Wetter mixed conifer areas are predicted to have had a mean fire rotation of 30 years. Fire rotation is thought to increase gradually with elevation. For example, mesic \emph{Abies magnifica} (Red Fir) forests, which exist around 2,000 feet higher in elevation than \emph{P. ponderosa} forests, had a mean fire rotation of 60 years \citep{Mallek2013}. Thus, under historical conditions, fire rotations increased with increased moisture and elevation. Mean fire rotations do not tell the full story, however, as the variance around a mean fire rotation could be remarkable, with some parts of the forest experiencing fire frequently, while other escaped fire for long periods. 

\begin{wrapfigure}{r}{0.5\textwidth} % use a capital R to allow figure to float
\includegraphics[width=0.5\textwidth]{/Users/mmallek/Documents/Thesis/Seminar/aerialpostfireUSFS.jpg}
\caption{Because fire suppression creates homogeneous forest stands that extend across large areas, when fires escape control they can grow large very quickly, and burn with high intensity over large areas. As a result, the consequence of altering fire regimes has been that instead of fires occurring mostly as low severity with patches of high severity, the opposite take place. During the historical period, the amount of high versus low severity fire in this image probably would have been inverted.} 
\label{fig:bigfirephoto}
\end{wrapfigure} 

Since then, fire suppression, logging, grazing, and mining have all interacted to alter the historical fire regime and vegetation patterns \citep{Stephens2015,Knapp2013}. For the xeric forest types within this landscape, frequent fires (usually having low mortality) were the norm. After large-scale fire suppression became the norm in the second half of the 19th century, less fire-tolerant species (such as \emph{Pseudotsuga menziesii} (Douglas fir) and \emph{Abies concolor} (white fir)) have come to dominate areas where they were once only one part of the vegetation community. These species have increased in both density and volume. Grazing and development made fires less common by altering or removing the fine fuels that carried fire. Timber harvest, especially of fire-tolerant species such as \emph{P. ponderosa} and \emph{Pinus lambertiana} (sugar pine), accelerated the increased cover of species such as \emph{A. concolor}. Finally, fire suppression allowed the buildup of medium size fuels and ladder fuels, which promotes larger and hotter fires when they do occur (Figure~\ref{fig:highseverity}). Moreover, the lack of natural fires has meant that variation in fuel loading has decreased, which allows large fires to spread over very large areas \citep{Hessburg2005,Beaty2007,Meyer2008} (Figures~\ref{fig:bigfirephoto}). Consequently, recent research has focused on what options are available to try and prevent the occurrence of large, highly severe fires, and what management actions are appropriate after such events (\citep{Stephens2013,North2010}. Managing for and within the range of variability for a landscape is one potential solution.


%\clearpage

%%%%%%%%%%%%%%%%%%%%%%%%%%%%%%%%%%%%%%%%%%%%%%%%%%%%%%%%%%%%%%%%%%%%%%%%%%%%%%%%%%%
\subsection{Ecology of vegetation systems}
Vegetation in the study area is tremendously diverse and changes slowly along an elevational gradient and in response to local changes in drainage, aspect, and soil structure. Grasslands, chaparral, oak woodlands, mixed conifer forests, and subalpine forests are all found within the study area (Figure~\ref{fig:majorvegtypes}). Many species exhibit fire-adapted traits, such as resprouting from roots after a fire, fire-induced germination, or thick bark\todo{cite sugihara book?}. In collaboration with USDA Forest Service staff, a system of land cover and seral stage classification based on LandFire (\mbox{\burl{www.landfire.org}}) and Presettlement Fire Regimes frm \citet{VandeWater2011} was developed. I then crosswalked Forest Service corporate spatial data based on the Northern Sierra \textsc{CalVeg} classification to each cover type. I also considered information from \emph{A Guide to Wildlife Habitats of California}, popularly known as the ``Wildlife Habitat Relationship (WHR)'' cover types. 

There are 13 major vegetation types, several of which include one or more of the following variants: mesic, xeric, ultramafic, and aspen. In total there are 31 defined cover types. In order to realistically simulate fires within the study area, I buffered it by 10 km in all directions. This allowed space for fires to move in and out of the focal landscape. A few of the 31 cover types only occur within this buffer, but they are still fully defined within the model. I completed analysis on the nine (of 31) cover types that extend across at least 1000 ha of the study area because I had the most confidence in results for cover types at least that well represented. Within this thesis, I focus most of my reported results and discussion on the two most prevalent forest cover types, Sierra Mixed Conifer - Mesic and Sierran Mixed Conifer - Xeric, because these results are statistically reliable and a deeper understanding of the historical and potential future ranges of variability of the mixed conifer belt will be the most useful to managers. Results and analysis for the other cover types were included in reports to the Tahoe National Forest and are included in Appendix \ref{app:full-results}. In this introduction I review the cover types following the ecological zone groupings from \citet{VanWag2006}. Figure~\ref{fig:majorvegtypes} depicts the major vegetation types that undergo succession, plus the six static land cover types, within the study area. Figure~\ref{fig:ecologicalzonebands} illustrates the elevational distribution of ecological zones in the Sierra Nevada more broadly.


\begin{figure}[!htbp]
\centering
\includegraphics[width=\textwidth]{/Users/mmallek/Documents/Thesis/maps/majorvegtypes_shade.pdf}
\caption{The 13 major vegetation types that undergo succession, plus the 6 static land cover types. This map does not distinguish between xeric, mesic, ultramafic, and aspen variants. See Figure~\ref{fig:covermap} for more detail. The black inner boundary is defined by subwatersheds of the Yuba River. The outer boundary is the result of a buffer used in the simulation. In the study area, elevation increases from west to east, and the landscape is incised by three main river canyons, two of which lead to the Pendola reservoir in the southwest part of the map.
}
\label{fig:majorvegtypes}
\end{figure}

\begin{figure}[!htbp]
\centering
\includegraphics[height=0.3\textheight]{/Users/mmallek/Documents/Thesis/Plots/ImagefromFireTextbook.pdf}
\caption{Area of ecological zones by 500 m elevation bands. The elevational distribution of ecological zones is evident as area cover by each zone increases and then decreases as elevation increases. Aspect, soil quality, moisture availability, and elevation all contribute to the exact distribution of individual tree species within these broad vegetation types. Figure from ``Sierra Nevada Ecoregion,'' by Van Wagtendonk and Fites-Kaufman, Chapter 12 in \emph{Fire in California Ecosystems}, p. 269.}
\label{fig:ecologicalzonebands}
\end{figure}

\paragraph{Foothill shrubland and woodland} This ecological zone lies directly adjacent and to the west of the study area. A small part of the buffer used around the study area includes this zone, which is represented by the Oak Woodland cover type. This type is characterized by savannas, woodlands, or forests of either monospecific or mixed stands of various oak species. \emph{Quercus douglasii}, \emph{Quercus lobata}, \emph{Quercus wislizenii}, and \emph{Quercus garryana} are the major dominants. (Appendix~\ref{oak-description}). 

\paragraph{Lower Montane forest} This ecological zone includes Oak-Conifer Forests and Woodlands and Mixed Evergreen forests at lower elevations, developing into Sierran Mixed Conifer forests with increasing elevation. All three of these cover types are typified by a combination of both coniferous and broadleaved trees. 

Mixed Evergreen is characterized by dense stands of \emph{Notholithocarpus densiflorus} (Tanoak) and \emph{Arbutus menziesii} complemented by \emph{Pseudotsuga menziesii} on more mesic soils. In xeric sites, conifers are less common, and a hardwood tree layer composed of evergreen oaks such as \emph{Quercus chrysolepis}, \emph{Quercus wislizeni}, \emph{Quercus kelloggi}, and \emph{Quercus garryana} instead dominates. Fires are fairly common in this cover type, but the vegetation quickly recovers, and many of the hardwoods resprout after fire (Appendix~\ref{meg-description}). 

Oak-Conifer Forests and Woodlands are characterized by the conifers \emph{Pinus ponderosa} or \emph{Pinus jeffreyi}, with one or more oaks, such as \emph{Quercus kelloggii}, \emph{Quercus garryana}, \emph{Quercus wislizeni}, or \emph{Quercus chrysolepsis}. Historically, low severity fires were extremely common. Fire is integral to the ecology of the yellow pines and oaks that characterize this system, and this cover type is one of the most altered by fire suppression (Appendix~\ref{ocfw-description}).

Sierran Mixed Conifer forests (Figure~\ref{fig:smctrees}) are characterized by five conifers and one hardwood: \emph{Abies concolor, Pseudotsuga menziesii, Pinus ponderosa, Pinus lambertiana, Calocedrus decurrens}, and \emph{Quercus kelloggii}. At least three conifers are typically present in any given stand.All of these species can be found in either cover type, but some are more closely associated with either the mesic or xeric variant. The characteristic species of the mesic type, \emph{A.~concolor} and \emph{P.~menziesii}, are less adapted to fire. Species characteristic of the xeric type, \emph{P.~ponderosa}, \emph{P.~lambertiana}, plus \emph{Q.~kelloggii}, are more fire-adapted. \emph{C.~decurrens} is found in both subtypes, but is very rarely dominant. The distrbution of these species is normally an outcome of the variation in the frequency and intensity of wildfire under natural conditions, although alteration of these conditions can affect ther distrbution. \emph{A.~concolor} tends to be the most ubiquitous species, especially on north-facing slopes. \emph{P.~ponderosa} was historically the dominant species, under the previous frequent low severity fire regime. It is still the most prevalent on south slopes and is present continuously from the Oak-Conifer Forest and Woodland belt below it in elevation (Appendix~\ref{smc-description}).


\begin{figure}[!htbp]
\centering
\includegraphics[width=\textwidth]{/Users/mmallek/Documents/Thesis/Seminar/mixedconiferslide.png}
\caption{The Sierran Mixed Conifer mesic and xeric cover types dominate the study landscape, extending across 61\% of it. They are characterized by five conifer and one deciduous oak species. The characteristic species of the mesic type, \emph{Abies concolor} (white fir) and \emph{Pseudotsuga menziesii} (Douglas fir), are less fire-tolerant. Species characteristic of the xeric type, \emph{Pinus ponderosa} (ponderosa pine), \emph{Pinus lambertiana} (sugar pine), plus \emph{Quercus kelloggii} (black oak), are more fire-adapted. \emph{Calocedrus decurrens} (incense cedar) is found in both subtypes, but is very rarely dominant.}
\label{fig:smctrees}
\end{figure}


\paragraph{Upper Montane forest} The Upper Montane zone is defined by the presence of Red Fir forests \citep{Potter1998}. The Red Fir cover type is dominated by \emph{Abies magnifica}, but other species do co-occur. On mesic sites, \emph{Pinus monticola} and \emph{Pinus contorta} ssp. \emph{murrayana} are also found, while on xeric sites \emph{Abies concolor} and \emph{Pinus jeffreyi} are more common. Red Fir forests are extremely resilient to disturbance. Wildfires were less common in these forests than those of the Lower Montane zone, and stands were characterized by complex patches of even-aged trees within a single stand arising from localized disturbance events. The boundaries between Sierran Mixed Conifer and Red Fir are fuzzy, such that the types overlap with one another at their boundaries. Similarly, Red Fir forests of the Upper Montane zone blend into the lodgepole and subalpine conifers of the Subalpine forest zone (Appendix~\ref{rfr-description}).


\paragraph{Subalpine forest} I defined two cover types within the subalpine forest. Lodgepole Pine occurs along the lower elevation portion of the zone, usually in wetter soils, such as along meadow edge, although it is also found on rocky soils. Unlike the subspecies of Lodgepole pine found in the Rocky Mountains, \emph{Pinus contorta} ssp.\@ \emph{murrayana} does not have serotinous cones. Wildfires in Lodgepole Pine tend to be high severity and recur at long intervals (Appendix~\ref{lpn-description}).% Fites-Kaufman et al 2007, Landfire). 

Above it, Subalpine Conifer is found, consisting of a mosaic of forest, woodland, meadow, and scrub vegetation. Trees often grow as krummholz forms at the highest elevations. \emph{Tsuga mertensiana} is often the most common tree species and mixes with \emph{Pinus contorta} ssp.\@ \emph{murrayana, Abies magnifica, Pinus monticola}, and \emph{Pinus albicaulis}. Wildfire is very rare in this cover type (Appendix~\ref{scn-description}). 

Western White Pine sometimes occurs in sufficiently continuous patches to be classified as its own type, separate from the Subalpine Conifer group. Typified by \emph{Pinus monticola}, species from the Subalpine Conifer and Red Fir cover types sometimes co-occur as well. This cover type tends to occur on drier soils. Most fires are low severity fires that promote the development of late successional forests (Appendix~\ref{wwp-description}).


\paragraph{Alpine meadow and shrubland} All lands within the study area are below climatic treeline, and I did not identify alpine meadows and shrublands that are not appropriately considered an early successional stage of Subalpine Conifer.


\paragraph{Eastside forest and woodland} Although the study area does not include any lands east of the Sierra crest, the buffer around the study area does, and I classified and described the Yellow Pine cover type to capture most of this vegetation community. It is characterized by \emph{Pinus ponderosa} or \emph{Pinus jeffreyi}, but other conifers and oaks, as well as \emph{Juniperus occidentalis} may occur. Under historical conditions, wildfires were extremely common and were almost always low severity; fire is integral to the ecology of yellow pines (Appendix~\ref{ypn-description}). 
%
Oak-Conifer Forests and Woodlands, as described above, are also present (Appendix~\ref{ocfw-description}). 
%
Also east of the crest are two shrub community types: Big Sagebrush (Appendix~\ref{sage-description}), typified by \emph{Artemisia tridentata} and Black and Low Sagebrush (Appendix~\ref{lsg-description}), typified by \emph{Artemisia arbuscula} or \emph{Artemisia nova}. 


\paragraph{Other cover types and variants} Some cover types not listed above can be found in any zone. I classified riparian vegetation into a Montane Riparian cover type (Appendix~\ref{mrip-description}). 
%
Curl-leaf Mountain Mahogany is most common in the Upper Montane zone and above, but may occur in the Lower Montane zone as well. Typified by \emph{Cercocarpus ledifolius}, it is also a shrub cover type (Appendix~\ref{cmm-description}). 
%
Several of the cover types listed here include ultramafic (a rock type that includes serpentine) variants, which are characterized by discontinuous fuel cover of grasses and low shrubs. The most common tree species to grow on ultramafic sites is \emph{Pinus jeffreyi}. Many endemic species grow on ultramafic soils, which are high in metal concentrations and fairly unproductive (\citealt{OGeen2007}, Appendices~\ref{smc-description}, \ref{ocfw-description}, \ref{rfr-description}, \ref{meg-description}).
%
Finally, I included aspen variants to several cover types. Lodgepole Pine, Red Fir, and Sierran Mixed Conifer all included variants that were seral to conifer forests in the absence of fire to maintain the aspen stand (Appendices~\ref{smc-description}, \ref{rfr-description}, \ref{lpn-description}). Subalpine Conifer and Yellow Pine included variants considered ``stable'' aspen (Appendices~\ref{scn-description}, \ref{ypn-description}). In all cases, the total area classified to aspen is very small. 

%%%%%%%%%%%%%%%%%%%%%%%%%%%%%%%%%%%%%%%%%%%%%%%%%%%%%%%%%%%%%%%%%%%%%%%%%%%%%%%%%%%

\section{Forest Planning}

With the emergence of ecosystem management as an organizing framework for natural resources management in the early 1990s, the need to recognize ecosystems as dynamic and constantly-changing became well accepted, and calls to manage forests sustainably became common \citep{Christensen1996}. Just as timber harvest and fire suppression were included in Forest Service policy, so now is guidance on how to restore forests to more resilient conditions. Within the context of forest and land management planning, the restoration of ecosystems to their pre-European settlement states was incorporated as a goal or desired future condition into various plans, including the Sierra Nevada Ecosystem Project \citep{SNEP1996a}. 

Each National Forest in the country has a Forest Plan, which guides the management of all resources on a National Forest. By 2000, the U.S. Forest Service's formal Planning Rule, which provides direction on the development of these plans, explicitly called for the agency to estimate and describe the range of variability under natural disturbance regimes, and manage for those characteristics (36 CFR \textsection 219 2000). Decisions were to be grounded in the context of ``maintain[ing] or restor[ing] ecological conditions that are similar to the biological and physical range of expected variability'' (36 CFR \textsection 219.4). The need to consider the natural range of variability was maintained through various amendments to the rule, and is still present in the new 2012 rule, finalized in early 2015: ``Plan decisions affecting ecosystem diversity must provide for maintenance or restoration of the characteristics of ecosystem composition and structure within the range of variability that would be expected to occur under natural disturbance regimes of the current climatic period'' (36 CFR \textsection 219 2012). Thus, the Planning Rule instructs land managers to restore and maintain ecosystems characteristic of the conditions that would be expected to occur in the absence of modern humans, and recognizes that ecosystems are dynamic and variable over time.





%%%%%%%%%%%%%%%%%%%%%%%%%%%%%%%%%%%%%%%%%%%%%%%%%%%%%%%%%%%%%%%%%%%%%%%%%%%%%%%%%%%
\section{Range of Variability Analysis}

Historic range of variability (HRV) analysis is a useful paradigm in landscape planning. HRV analysis is intended to help conceptualize the mechanisms behind large-scale ecosystem functions and provide a basis from which to make predictions about how a given ecosystem will react to disturbances in the future \citep{Nonaka2005,Landres1999}. Methods for quantifying the natural range of variability for a diversity of landscapes in the United States augmented the development of research focused on this task \citep{Landres1999}. Of these, simulation of the historical dynamics became fairly popular. By 2004, some 45 landscape fire and succession models alone had been developed \citep{Keane2004}. Many of these, such as \textsc{landis} \citep{He1999}, \textsc{zelig-l} \citep{Miller1999}, \textsc{safe-forests} \cite{Sessions1997} and \textsc{landsum} \citep{Keane2012} are still in use today. Landscape fire and succession models are used to create spatially-explicit simulations of both of these key forest processes, typically outputting a set of GIS layers for each timestep of the model that can then be analyzed to quantify trajectories and patterns in the disturbance regime, seral stage composition, and landscape configuration over time \citep{Keane2004}. A component of many landscape fire and succession models are state and transition models, which form a framework for defining the fundamental vegetation communities and the probabilities over time for transitions from one state to another \citep{Stringham2003,Blankenship2015}.

Although empirical data may sometimes be available on some variables affecting HRV, the time scales and broad spatial extents under study make simulations a logical choice, allowing researchers to incorporate all parameters of interest and ultimately derive a meaningful quantification \citep{Swetnam1999,Mladenoff1999}. Range of variability analyses have been conducted using literature searches exclusively, including within the Sierra Nevada \citep[e.g.,]{Safford2013}. Results of such analyses depend on the assumption that an aggregation of many small studies is sufficient to address long-term, large-scale questions, and require researchers to accept many unknowns about research methodologies. Moreover, in landscapes severely impacted by European settlement, such as those of the northern Sierra Nevada, it is impossible to observe trajectories in which fire suppression has not influenced landscape structure \citep{Keane2012}. In the absence of consistent and complete data, simulations can be used to incorporate the data that do exist and generate new datasets of otherwise unobservable landscape trajectories. From these new datasets, statistical analyses can be used to describe the landscape quantitatively, and subsequently make inferences about the HRV of an area, as well as compare current conditions to the HRV (Figure~\ref{fig:hrvplot}). The Rocky Mountains and Oregon Coast Range have been the focus of several simulated HRV studies, while the only one conducted in the Sierra Nevada took place in Sequoia National Park, in the southern Sierra \citep{Miller1999}. My study area is in the northern Sierra, and is different in its land use history, vegetation, and disturbance regime. Consequently, my study is the first major HRV analysis in the northern Sierra Nevada. 

\begin{figure}[!htbp]
\centering
\includegraphics[width=\textwidth]{/Users/mmallek/Documents/Thesis/Seminar/HRVgraphic.png}
\caption{The range of variability is the dynamic change over time of a given attribute of an ecosystem under study. It represents a resilient prior state because these ecosystems developed and were maintained by a particular set of disturbance regimes and climatic conditions. It is therefore a reasonable target for restoration and maintenance efforts. This figure shows a generic range of variability plot. Any statistical measure derived from the outcome of my simulations may be evaluated under the range of variability framework. The range of variability can be defined for a particular case in various ways: by the green box, by the red horizontal lines, by minimum and maximum all historical data (arrow), etc.
}
\label{fig:hrvplot}
\end{figure}

This thesis includes an analysis of the simulated historical range of variability for the Upper Yuba River Watershed and an analysis of disturbance and succession trends associated with future climate scenarios. I define the HRV as the variation in disturbance processes and landscape composition and configuration over the 300 years prior to European settlement. A primary motivation for this study was to aid in planning on the Tahoe National Forest. The quantitative assessment of the HRV of this landscape provides managers with a statistical, ecosystem-level analysis of the disturbance and succession processes that characterize this portion of the northern Sierra Nevada. Quantification of HRV provides managers with a neutral assessment of the current departure from HRV, which they can use to prioritize certain vegetation types, disturbance processes, or their intersection for restoration or maintenance. Because the simulation captures landscape changes over hundreds of years, far longer than the planning cycle, the results allow managers to ground near-term plans and expectations within a larger context. 

In planning restoration efforts, it is reasonable to use as a reference the last known period during which a dynamic but resilient landscape existed \citep{Swetnam1999}. The arrival of European settlers to the Sierra Nevada led to sweeping ecological changes that now have greatly altered many Sierran landscapes through fire suppression, grazing, road building, timber cutting, recreation, and other activities \citep{Storer1963,Stephens2015,Knapp2013,Hessburg2005}. The period prior to European settlement, then, is a suitable reference condition against which I can compare current landscape structure and dynamics. Moreover, it is frequently used in the western United States as the historical reference period for restoration planning \citep{Safford2013}. It is thought to represent a resilient prior state during which ecosystems developed and were maintained by a particular combination of disturbance regimes and climatic conditions \citep{VandeWater2011,Meyer2013}. The period is also several times the length of rotation periods identified for well-understood cover types within the study area. Finally, it is a time frame for which I have sufficient information to have confidence in model results. 

I am mindful of the fact that this reference period overlaps the ``Little Ice Age,'' which may temper the utility of the results as specific management targets, but does not diminish their usefulness in other ways \citep{Minnich2007,Safford2013}. The chosen reference period was a not time of stasis, climatically, ecologically, or culturally. The oscillation of the Palmer Drought Severity Index, a measure of climate variability in terms of precipitation and temperature over time, illustrates this (see Chapter~\ref{subsubsec:distparams} for a detailed discussion). In addition, multi-year droughts and El Ni\~no/La Ni\~na events also occurred over this time frame \citep{Minnich2007}. Ecologically, my historical period occurred during a very long-term (on the scale of millennia) shift to a warmer and drier climate, with an associated shift toward species more tolerant of such conditions, such as yellow pine species, and away from species like white fir, which prefer more mesic conditions. A slow shift toward more frequent fire occurred in conjunction with the warming and drying climate \citep{Safford2013}. At the same time, several Native American tribes were living throughout the study area during the reference period. Debate is ongoing among scientists and researchers as to the extent to which those peoples managed vegetation through setting fires \citep{Anderson1996}. As a result, fire history data includes evidence from both lightning-caused and human-caused fires, and the historical record does not always distinguish them from one another. And, regardless of how fires begin, they burn and impact forest vegetation. I lack an empirical basis for excluding some fires from the record, and so decided to include all the available data.

Comparing the current landscape to the chosen reference period should not implly that a proper management goal would be to recreate all of the ecological conditions and dynamics of this period. Such a goal may not be possible, nor potentially desirable in light of ongoing climate change, ecological shifts, and social realities. However, using the chosen reference period provides an opportunity to compare current conditions to a baseline set of data on ecosystem conditions (composition, configuration, and disturbance processes) and `` develop an idea of trend over time and idea of the level of departure of altered ecosystems from their `natural' state'' \citep{Safford2013}. The results presented here will complement the Natural Range of Variability assessments compiled by the Forest Service's Pacific Southwest Region Ecology group \citep[e.g.,][]{Safford2013,Merriam2013,Meyer2013a,Meyer2013,Estes2013,Estes2013a,Gross2013}. An understanding of natural landscape structures and variability during this reference period also provides a basis for forest management policies and associated actions that seek to mimic natural disturbance patterns \citep{Romme2000,Buse2002}. 






%%%%%%%%%%%%%%%%%%%%%%%%%%%%%%%%%%%%%%%%%%%%%%%%%%%%%%%%%%%%%%%%%%%%%%%%%%%%%%%%%%%\subsection{Modeling Framework}
\section{Modeling Framework}

To simulate the historic range of variability during the reference period in the study area, I used a modified version of the Rocky Mountain Landscape Simulator (\textsc{RMLands}), a spatially-explicit, stochastic, landscape-level disturbance and succession model capable of simulating fine-grained processes over large spatial and long temporal extents \citep{McGarigal2001}. It is grid-based and simulates fire on landscapes in a spatially explicit and realistic manner. State transitions are simulated at the 30 m pixel scale. As a result, I did not assign fires as a whole to a \emph{low}, \emph{mixed}, or \emph{high severity} status. Instead, I focused on defining conditions under which transitions among potential states within a given cover type occur or not. Transitions may also take place in the absence of fire due to natural succession \citep{McGarigal2012}. Outputs from the model are readable by the landscape pattern analysis software \textsc{Fragstats} \citep{Fragstats2012}, which facilitates the landscape configuration analysis.

\textsc{RMLands} was originally developed to simulate the historical range of variability of forests in southwestern Colorado. Reports on the historical range of variability were completed in 2005 for the San Juan National Forest and the Uncompaghre Plateau \citep{McGarigal2005,McGarigal2005a}. \textsc{RMLands} has also been used to simulate wildfire and vegetation succession on the Lolo National Forest in Montana \citep{Cushman2011}. I worked closely with Kevin McGarigal and Eduard Ene to adapt the software for use in the Sierra Nevada. I then used the modified software to prepare an HRV analysis for part of the Tahoe National Forest in California. This work has been a collaborative effort between the University of Massachusetts, Amherst, staff on the Tahoe National Forest in California, and ecologists from the USFS Region 5 Ecology program. In addition to developing cover types and seral stage definitions for Sierran vegetation that were compatible with the \textsc{RMLands} modeling framework, I also modified the level at which it handles susceptibility and mortality, and introduced a new parameter, the Topographic Position Index. It is worth noting that (, like many researchers using simulations to conduct HRV analyses, parameterized RMLands to simulate passive management of fire and vegetation \citep{Wimberly2002,Nonaka2005,Mcgarigal2012}. That is, I did not simulate vegetation treatments, nor did I attempt to emulate additional fire suppression efforts.






%%%%%%%%%%%%%%%%%%%%%%%%%%%%%%%%%%%%%%%%%%%%%%%%%%%%%%%%%%%%%%%%%%%%%%%%%%%%%%%%%%%




\section{Climate Change and Future Ranges of Variability}

In addition to the HRV analysis, the need to explore and understand the ramifications of climate change on the disturbance regime and its relationship to the forest has become well recognized. As important as it is to understand the dynamics characteristic of the historical period, the future climate will differ from the historic climate. 
%
\begin{figure}[!htbp]
\centering
\includegraphics[width=\textwidth]{/Users/mmallek/Documents/Thesis/Seminar/snowpack_comparison.png}
\caption{The Sierra Nevada March snowpack levels, as seen from a NASA satellite. The top image is from March 2010, the last year with average winter snowfall in the region. The second image is from March 2015. The red circle surrounds my study area. Lower snowpack levels mean less water over less time in mountain rivers, causing moisture stress in forest plants and increasing their susceptibility to wildfire.
}
\label{fig:satellitesnowpack}
\end{figure}

Recent warming and drying trends, and the current drought (Figure~\ref{fig:satellitesnowpack}), have already influenced a more frequent and proportionally more severe fire regime in western forests in general and the Sierra Nevada in particular \citep{McKenzie2004,Westerling2011,Miller2012}. These trends are anticipated to continue under warmer and drier climate change scenarios \citep{Westerling2008}.  Changes have also been reported in the elevation of fires in the Sierra Nevada, increasing the potential for upward shifts of the elevational range occupied by species and vegetation assemblages \citep{Schwartz2015}. In addition, concern about the impact of changes to precipitation and temperature anticipated under climate change in the northern Sierra on local disturbance regimes, and subsequently, seral stage distribution and patch configuration has motivated analysis that consider not only at the current and historical conditions, but also future conditions \citep{Fule2008,North2012}. 

Range of variability analyses that offer a complementary analysis of future scenarios under climate change are rare (but see \cite{Keane2008} and \cite{Duveneck2014}). Where the focus of management efforts has in the past been restoration, current policy emphasizes using adaptive strategies to ensure resilient ecosystems \citep{Stephens2010}. By simulating a range of potential future climate scenarios, I generate data to use in evaluating trends in landscape pattern related to trends projected under climate change, and place the current landscape in that context. Moreover, I use this additional information to consider which restoration strategies are likely to promote resilient forests and make sense ecologically for the area under study \citep{Duncan2010}. 

The range of potential future climate scenarios used to parameterize the model in this study come from models initialized using the set of parameters for Representative Concentration Pathway (RCP) 8.5. The RCP scenarios are those currently used in climate change research. They replaced the previous set of scenarios, known as SRES after the Special Report on Emissions Scenarios that detailed them \citep{VanVuuren2011a}. They were developed in an effort that overhauled the IPCC and climate change research communities' approach to developing and using climate scenarios \citep{Moss2008}. A key difference between the approach used to build the SRES scenarios and that used to build the RCP scenarios was the shift from a sequential to a parallel approach (Figure~\ref{fig:scenarioapproach}). In SRES, emission and socioeconomic scenarios were designed first, and the outputs used to model radiative forcing, then climate projections, and finally impacts. Conversely, in the current parallel approach the first step is to design RCPs that correspond to levels of radiative forcing, and then allow climate modeling and emissions and socioeconomic scenario building to happen concurrently based on the same starting set of assumptions. The outputs from both components of the parallel process are then used to analyze impacts \citep{Moss2010}. 

The new process has clear advantages, including the fact that by not prescribing the mechanisms that lead to particular RCPs, flexibility is provided to examine a huge range of different factors and how they combine that can lead to a particular outcome \citep{VanVuuren2011}. However, it can be slightly more confusing to think about RCPs on their own, since they are not really intended to be used on their own, but rather as the shared basis for analyzing possible future outcomes \citep{VanVuuren2011}. The RCP outcomes are standardized in that they are based on (named after) the predicted conditions in 2100 \citep{VanVuuren2011a}. They are also spatially explicit, and all RCP scenarios are based on the same geographic locations \citep{VanVuuren2011a}. Finally, they are actually trajectories defined by their end point, such that they explicitly include a sequence of data points from the starting conditions to the final conditions \citep{VanVuuren2011}. I use data built from the trajectory to 2100 to parameterize \textsc{RMLands} for simulating disturbance and succession into the future.

\begin{figure}[!htbp]
\includegraphics[width=\textwidth]{/Users/mmallek/Documents/Thesis/Plots/rcp-sequence.png}
\caption{Differences between the SRES scenario development and use (sequential approach) and the current RCP scenario development and use (parallel appraoch). Figure from \citet{Moss2008}.}
\label{fig:scenarioapproach}
\end{figure}

RCP8.5 includes no specific climate mitigation target, unlike the other three RCP scenarios in use \citep{Riahi2011}. As a result, it is considered a reference, or baseline scenario, in which greenhouse gas emission and concentrations increase over time without leveling out \citep{Riahi2011}. A literature review during the RCP development process designated radiative forcing in 2100 of 8.5 W/m$^2$ as the high end of plausible futures that had been modeled \citep{VanVuuren2011}. The corresponding concentration of $> \sim 1370 \text{ CO}_2$ -eq in 2100, compared to 375 $\text{CO}_2$ -eq in 2005. The 66\% range for the variable of temperature increase above pre-industrial levels under the RCP8.5 scenario is 4.0\textdegree -- 6.1\textdegree C \citep{Rogelj2012}. Since the development of RCP8.5 as a scenario, narratives illustrating one potential set of socioeconomic and political conditions have been developed, one of which I include here:
%
\begin{quote} 
The scenario’s storyline describes a heterogeneous world with continuously increasing global population, resulting in a global population of 12 billion by 2100. Per capita income growth is slow and both internationally as well as regionally there is only little convergence between high and low income countries...The slow economic development also implies little progress in terms of efficiency. Combined with the high population growth, this leads to high energy demands...the future energy system moves toward coal-intensive technology choices with high GHG emissions...agricultural productivity increases to feed a steadily increasing population...Compared to the scenario literature RCP8.5 depicts thus a relatively conservative business as usual case with low income, high population and high energy demand due to only modest improvements in energy intensity \citep{Riahi2011}.
\end{quote}
%
Thus the RCP8.5 scenario, developed in 2008 is intended to serve as the upper boundary for changes to the climate and associated consequences by 2100 \citep{Moss2008}. The extent to which it can still be considered an upper bound in 2015 is outside the scope of this thesis. Regardless, the results presented in Chapter~\ref{ch:FRV} are explicitly tied to the RCP8.5 scenario. I present and discuss them within this context.


As part of my investigation into future landscape trends, I observed a dramatic shift in the proportion of both xeric and mesic mixed conifer forests in the Early Development seral stage. I then focused my analysis of patch configuration on the Early Development stage of these two cover types. Early successional habitats are not a major focus of forest ecology research, in part because they are seen as an intermediate phase that for some forest users is kept short \citep{Swanson2011}. However, they are a critical component of all systems, functioning as a major contributor to biodiversity and supporting a range of species' habitat needs \citep{Chang1995,Hutto2008,Swanson2011}. The Sierra Nevada Framework, last updated in 2007, identifies management indicator species that use openings and early seral habitat \citep{USDAForestService2004,USDAForestService2007}. Recent trends in wildfire extent and severity mean that managers are faced with decisions about when and how to manage post-fire early successional habitat \citep{Stephens2013,Dellasala2014}. My model results will provide insight into the spatial configuration of early successional forests under a natural fire regime for the intensively used mixed conifer zone. These results may also be helpful in planning restoration efforts using both prescribed fire and mechanical harvest techniques.

%Had a note in here about adding more about early seral, but not sure there's a lot more to say. Don't want to talk a lot about range shifts because I can't model it anyway.




 
% !TEX root = master.tex

\chapter{Historical Range of Variability}
\label{ch:hrv}
\setcitestyle{notesep={:},aysep={}}

In this chapter I present the methods, results, and discussion for my analysis of the historical range of variability (HRV)in disturbance and forest structure in the Yuba River Watershed in the northern Sierra Nevada. The historical period defined for this project was 1550 to 1850. Unfortunately, I did not have consistent and complete empirical data on wildfire and vegetation growth and pattern during this period. Instead, I used simulations to incorporate the data that do exist, generate new datasets of otherwise unobservable landscape trajectories, and ultimately quantify the HRV for my study area. Many landscape disturbance and succession models exist, with different input requirements, purposes, intended geographic applicability, etc. Typically, each timestep of the model produces a set of GIS layers, which can then be analyzed to quantify trajectories and patterns; in other words, to describe the range of variability. In this study I used the Rocky Mountain Landscape Simulator (\textsc{RMLands}). Originally developed for use in the Rocky Mountains, an early phase of this project was to adapt \textsc{RMLands} to the Sierra Nevada. \textsc{RMLands} is disturbance and succession model software that is stochastic, spatially explicit, and raster-based. The model was parameterized to simulate passive management of fire and vegetation. That is, I did not simulate vegetation treatments, nor did I attempt to emulate additional fire suppression efforts.

% !TEX root = master.tex

\section{Methods}
\label{sec:hrvmethods}

% removed study area, it's now in the Introduction Chapter

\subsection{Modeling Framework}
\label{sec:modelframe}

A partial introduction to \textsc{RMLands} is included in Chapter~\ref{CH1}, but here I provide a more detailed description of the model and how I it was used.
% intro to RMLands in Introduction Chapter (modeling framework, methodological limitations)

\subsubsection{Input Layers}
\label{subsec:hrvinputlayers}

All input layers to \textsc{RMLands} must be custom-built to work with the software. For technical details on the data structure requirements of \textsc{RMLands}, see Appendix \ref{app:inputs}. A brief overview of each input layer is included below.

\paragraph{Cover} Cover type is based on the potential or current natural vegetation of a site and includes both natural and anthropogenic cover types. For example, cover types include not only Lodgepole Pine, Sierran Mixed Conifer, and Red Fir, but also Barren and Agriculture. Succession pathways are defined uniquely for each cover type, susceptibility to natural disturbances varies among cover types, and suitability or eligibility for various vegetation treatments varies among cover types. Cover is a static (constant) grid and therefore provides a fixed template upon which disturbance and succession processes play out over time. 

The source for the cover layer is the Region 5 Existing Vegetation Layer (``EVeg''), first mapped to the \textsc{calveg} classification developed by the Region's Ecology Program in 1978. When deciding on land cover types, including determining xeric and mesic subtypes, our focus was to best represent the project area and the surrounding landscape. We used the \textsc{calveg} Mapping Zone boundary for the ``North Sierra'' (Figure~\ref{calveg}) as our focus for defining vegetation and disturbance, including susceptibility, response to fire, and fire size and distribution. Within the project area, the EVeg layer was developed based on three separate efforts: a satellite-based imagery analyses in 2000, and two orthoimagery analysis completed by contracting firms in 2005. Generally, specific cover type names were derived from the California Fire Return Interval Departure (FRID) report by \citet{VandeWater2011}. We also considered information from \emph{A Guide to Wildlife Habitats of California}, popularly known as the ``Wildlife Habitat Relationship (WHR)'' cover types \citep{WHR1988}. 

\begin{wrapfigure}{R}{0.5\textwidth} % use a capital R to allow figure to float
\includegraphics[width=0.48\textwidth]{/Users/mmallek/Tahoe/Report3/images/CALVEGmappingzones.png}
\caption{\small CALVEG Mapping Zones. These zones meet U.S. Forest Service standard at national and regional levels. These ecological provinces are associated with dozens of vegetation alliances, which are used to classify vegetation in spatial data products. We used vegetation alliance definitions for the North Sierra zone to classify the land cover spatial data shared by the U.S. Forest Service.} 
\label{calveg}
\end{wrapfigure}

\subparagraph{Alternative Cover Layers}
The original intent of our team was to utilize two separate cover layers: one for the historical reference period, and one for the current period to be used in projections of future scenarios. Two layers were identified as potentially suitable for the historic analysis: a map created from forest survey and inventory efforts under Albert Wieslander conducted between 1928 and 1940 (``Wieslander'') \citep{Thorne2006}, and a map of Potential Natural Vegetation created by a Forest Service Enterprise Team for the Tahoe National Forest in the 2000s (Forest Service internal GIS data). Our intent was to use the PNV, Wieslander, or a combination thereof to derive the land cover layer for the HRV phase of the project. 

In order to validate the historical maps, we needed to develop a crosswalk between the vegetation type methodologies for the EVeg, PNV, and Wieslander maps. We also examined the spatial consistency in cover types across the maps. With significant assistance from the Tahoe National Forest, we attempted to create a crosswalk from these maps to the set of land cover types to be used in the project. However, we were unable to develop a consistent and comprehensive set of rules for this purpose. A major reason for this is that both the PNV and Wieslander maps used species lists, rather than assemblages (as in \textsc{calveg} and LandFire). For example, Sierran mixed conifer forests do not appear as a dominant ``cover type'' in the PNV map. The Wieslander maps do contain an internal crosswalk to a mixed conifer alliance, but only rarely. 

In addition, the PNV map contained a more significant error: we learned that, for the purposes of the modeling used to create the PNV map, ``potential natural vegetation'' meant the so-called ``climax'' community that would develop in the complete absence of disturbance, regardless of whether that disturbance was human-caused or natural. Since we are seeking to mimic the natural historic range of variability, we decided to discard this layer. The Wieslander map had its own issues. Most problematic was the non-systematic spatial error of up to 300 meters, which meant it would not be suitable for comparing specific locations. In addition, crosswalking precisely was impossible because coded vegetation was not necessarily in order of most prevalent vegetation, but instead prioritized tree species over shrubs, and commercially important trees over others. As an example from the handbook states, a plot consisting of 75\% \emph{Quercus kelloggi} (black oak), 15\% \emph{Pinus ponderosa} (ponderosa pine), and 10\% \emph{Pinus lambertiana} (grey pine) would be coded as ponderosa pine, grey pine, black oak. Finally, the Wieslander maps were developed from surveys done in the 1930s, decades after the huge influx of settlers in the 1850s; by the 1930s, vegetation patterns may have already been significantly altered \citep{Thorne2006}. Consequently, the Wieslander map is also not a reliable predictor of land cover type without extensive review of the original data and maps, which would be beyond the scope of this project. 

To confirm these problems, we examined the overlap in land cover types between different maps in ArcGIS. In general, the overlap between EVeg and either the PNV or the Wieslander layers was no better than random, and in many cases it was worse. We decided, in conjunction with Tahoe National Forest staff, to proceed using only the EVeg map, and omit the calibration period of the model from our analysis of the characteristics of the HRV. %This ensured that our analysis of future management scenarios and comparison of spatial metrics between those results and the HRV results was credible.
% in retrospect I wonder if we should have analyzed the configuration more. in the end the biggest problem was probably the lack of crosswalk, since a precise spatial equivalence wasn't assumed.

\subparagraph{Selection of Specific Cover Types}
In the early stages of this project, the team created a suite of land cover types based roughly on the Wildlife Habitat Relationships (WHR) types used in California and by Forest Service managers and planners. These consisted of the WHR types with a few additional types where additional specificity or refinement was desired. For example, Red Fir was split up into two subtypes. The original concept was to begin with the WHR types and modify them as needed based on other attributes in the EVeg layer. However, creating a crosswalk from WHR to the project-specific types also proved problematic. First, we realized that the WHR values were actually derived from the \textsc{calveg} species alliances included in the EVeg layer, but the methodology used was unavailable or missing. The crosswalks we did find were not mutually exclusive and all-inclusive, and do not always make ecological sense \citep{Keeler-Wolf2007,DeBecker1988,Game2005}. This is probably due in part to the fact that WHR is not a mapping classification. It is always derived secondarily. So, we were unable to create consistent rules for mapping from WHR to other types. Others have encountered similar issues:
%
\begin{quote}
WHR has been less successful in differentiating between vegetation types. Because the habitat types are inconsistently defined, a broad familiarity with its detailed descriptions is needed to differentiate among types of similar structure. Although mappers have constructed rules for discriminating among types, difficulties still remain because species dominance varies substantially within some types and broad overlaps in dominant plants occur among types. Other problems arise due to the small number of classes and the inconsistencies in scale among them \citep[p.~23]{Keeler-Wolf2007}
\end{quote}
%
In collaboration with National Forest staff we decided to instead base our land cover types on, at the first order, Presettlement Fire Regime (PFR) types as defined in the Fire Return Interval Departure (FRID) report by \citet{VandeWater2011}. The PFR types, as part of the FRID, were developed through a combination of literature and expert workshops, soliciting peer review prior to submitting the framework to additional review via the academic publication process. The PFR was also useful for this project specifically because it grouped vegetation types based on their relationship to wildfire, which is the disturbance type simulated in this study \citep{VandeWater2011}. We used the methodology from the FRID rather than using the second-order WHR classification and trying to reverse-engineer it to fit into our custom land cover types. 

Thus I created a new structure of cover types in a nested regime. At the coarsest level are the PFR types, created by aggregating \textsc{calveg} as described above. These are futher split using the Biophysical Settings from LandFire. Finally, a few types are further refined, ultimately generating a set of land cover types specific to the Yuba River Watershed, but applicable to the northern Sierra Nevada in general. A mutually exclusive and all-inclusive crosswalk for each land cover type used in this analysis to a single LandFire Biophysical Setting and Presettlement Fire Regime type thus exists; see Appendix~\ref{app:covertypedesc}.

I used Python scripts and ArcGIS to conduct the geoprocessing necessary to prepare the EVeg layer for use in \textsc{RMLands}. All processing was done after converting shapefiles (vector data) to the raster format. Land cover types were differentiated based on spatial location, presence of aspen stands, presence of ultramaifc soils, and position along a xeric to mesic gradient. 

First, Aspen variants of forested land cover types were created by overlaying an aspen layer onto the vegetation layer and using ArcGIS tools and Python scripts to create combined types (``[type] - Aspen'') where appropriate. Second, areas mapped as a vegetation type characteristic of early seral (e.g. chaparral) were remapped using ArcGIS tools and Python scripts to an appropriate forest cover type based on the land cover types in the area immediately adjacent to the patch considered to be in an early successional stage.

Next, vegetation data and elevation data were analyzed together to distinguish east- and west-side areas from one another. This information was used to appropriately identify land cover types that, in my application, are mapped only on the east-side. Yellow Pine and stable Aspen variants of forested land cover types were confined to the east-side in my application. 

Ultramafic\footnote{Ultramafic soils are those created from the weathering of igneous rocks, brought to the earth's surface as magma, where they then cooled. Ultramafic soils are typically shallow, rocky, and nutrient deficient, with high levels of metals uncommon in other soils. Only a few species of plants have evolved to live on them, many of which are endemic to such soils. Plants that do grow mature more slowly and cover the land less continuously than the same plant would on better soil. In the study area, the most common ultramafic rock is serpentine \citep{Safford2004}.} Ultramafic land cover types were mapped by overlaying a geology layer obtained from the Tahoe National Forest (1:100,000 scale) onto the vegetation layer and using ArGIS tools and Python scripts to create ``[type] - Ultramafic''. 

Finally, the Sierran Mixed Conifer, Red Fir, and Mixed Evergreen cover types, which cover broad swaths of land across elevation and aspect, were subclassified into either a mesic or xeric variant. Note that when present, aspen or ultramafic soils supersede xeric-mesic classification. Although the WHR classification system does not divide, for example, Sierran Mixed Conifer, into xeric or mesic types, other classification systems often do. In some cases this division is recognized at the PFR level (Sierran Mixed Conifer), while in others the refinement occurs at the Biophysical Setting level (Red Fir). The PFR method does crosswalk directly from \textsc{calveg} assignments, so for certain cases it would be possible to simply use this classification strategy. However, this method is based on existing vegetation only and does not consider abiotic factors. In addition, when I showed a map based on the PFR classification to local experts, they felt the distribution of xeric versus mesic types did not accurately represent the project landscape. Consequently, I explored some biophysical indicators related to moisture that could be used to designate and separate the mesic subtype from the xeric subtype.

Ultimately, in conjunction with the team I chose four metrics to comprise the mesic-xeric index. All metrics consist of modeled values. Climatic water deficit (CWD) is the annual evaporative demand that exceeds available water, measured annually in the summer. It is derived by subtracting actual evapotranspiration from potential evapotranspiration. The second metric, the topographic wetness index (TWI), measures topographic moisture. It is a function of slope and the catchment area of a particular point. Soil water storage (STOR) is the average amount of water stored in the soil annually. It is derived from precipitation, snowmelt rates, actual evapotranspiration, groundwater recharge rates, and surface water runoff rates. The final metric is the result of precipitation minus potential evapotranspiration (PPET), a measure of climatic moisture.

These variables were standardized by z-score such that higher values correspond to more mesic environments. Thus, potential evapotranspiration was inverted for this purpose. The mean for each metric is zero and the units are in terms of standard deviation. To combine the metrics, I combined the z-score value raster grids with equal weights. In conjunction with local experts, a break point in the resulting xeric-mesic gradient was selected and then applied using ArcGIS tools, creating ``[type] - Mesic'' and ``[type] - Xeric''. For the Sierran Mixed Conifer and Red Fir cover types, index values from the negative end of the range up to $-1/4$ standard deviations below the mean (zero) were used to create xeric variants, while the remaining portion of the spectrum was used to designate the mesic variants. For the Mixed Evergreen cover type, the break point along the gradient was $-1/2$ standard deviations below the mean. 


Ultimately, 31 cover types were generated for the buffered project area, as listed in Table~\ref{covertable} and shown in Figure~\ref{fig:covermap}.\footnote{Larger images of all of the input layers are included in Appendix \ref{app:inputs}.}. %A thorough description of geoprocessing steps necessary to recreate this data layer will be available soon. 
As Table~\ref{covertable} demonstrates, most cover types occupy a small extent of the project area. The cover types with an extent of less than 1000 ha within the core project area may have statistically unreliable results; this problem increases as the extent of given cover type decreases. We caution against attempting to make inferences for these less common cover types. However, because the nine cover types that do occur over at least 1000 ha represent approximately 93\% of the core project area, we have high confidence in the landscape-level results. These nine cover types were considered our focal cover types, and were all fully analyzed as part of the historical range of variability assessment. For space and continuity, in the main body of this thesis we discuss in detail only the two most common cover types, Sierran Mixed Conifer - Mesic and Serrian Mixed Conifer - Xeric, which comprise the bulk of the land in the project area actively managed by the Tahoe National Forest. Results for the other seven cover types are included in the appropriate appendices. 

%%%%%%%%%%%%%%%%%%%%%%
%%% COVER TABLE %%%%%%
%%%%%%%%%%%%%%%%%%%%%%

\begin{table}[!htbp]
\small
\centering
\caption{List of land cover types developed for this project. Included are the cover type abbreviation, full cover type name, and total area in the buffered project landscape in both acres and hectares. Cover types are listed in descending order based on area within the core study area only. Cover types that undergo succession appear in the first group, while static cover types appear in the second group. We use the cover type abbreviation in tables and figures for space.}
\label{covertable}
\begin{tabular}{@{}llrr@{}}
\toprule
 \textbf{\begin{tabular}[c]{@{}l@{}}Land Cover \\ Abbreviation\end{tabular}} & \textbf{Land Cover Name}    & \textbf{\begin{tabular}[c]{@{}l@{}}Area \\ Core Only\\ (Hectares)\end{tabular}} & \textbf{\begin{tabular}[c]{@{}l@{}}Area\\ Core+Buffer\\ (Hectares)\end{tabular}} \\ \midrule
                        \textsc{smc\_m  }     & Sierran Mixed Conifer - Mesic                & 57,853         & 133,920        \\
\rowcolor[HTML]{CAD6BA} \textsc{smc\_x  }     & Sierran Mixed Conifer - Xeric                & 52,198         & 91,443         \\
                        \textsc{ocfw    }     & Oak-Conifer Forest and Woodland              & 23,729       & 56,941    \\
\rowcolor[HTML]{CAD6BA} \textsc{rfr\_m  }     & Red Fir - Mesic                              & 8,563          & 19,626         \\
                        \textsc{rfr\_x  }     & Red Fir - Xeric                              & 7,493          & 9,989          \\
\rowcolor[HTML]{CAD6BA} \textsc{meg\_m  }     & Mixed Evergreen - Mesic                      & 7,273        & 13,547    \\
                        \textsc{meg\_x  }     & Mixed Evergreen - Xeric                      & 6,768        & 13,771    \\
\rowcolor[HTML]{CAD6BA} \textsc{smc\_u  }     & Sierran Mixed Conifer - Ultramafic           & 4,124          & 9,774          \\
                        \textsc{ocfw\_u }     & \begin{tabular}[c]{@{}l@{}}Oak-Conifer Forest and \\ Woodland -  Ultramafic\end{tabular} & 1,060   & 2,185   \\
\rowcolor[HTML]{CAD6BA} \textsc{lpn     }     & Lodgepole Pine                               & 837         & 2,816     \\
                        \textsc{mrip    }     & Montane Riparian                             & 732         & 2,216     \\
\rowcolor[HTML]{CAD6BA} \textsc{scn     }     & Subalpine Conifer                            & 638                  & 12,543         \\
                        \textsc{meg\_u  }     & Mixed Evergreen - Ultramafic                 & 604         & 1,655     \\
\rowcolor[HTML]{CAD6BA} \textsc{rfr\_u  }     & Red Fir - Ultramafic                         & 294                  & 321            \\
                        \textsc{wwp     }     & Western White Pine                           & 273                  & 510            \\
\rowcolor[HTML]{CAD6BA} \textsc{smc\_asp}     & Sierran Mixed Conifer with Aspen             & 58                   & 121            \\
                        \textsc{oak     }     & Oak Woodland                                 & 19          & 4,186     \\
\rowcolor[HTML]{CAD6BA} \textsc{cmm     }     & Curl-leaf Mountain Mahogany                  & 18          & 41        \\
                        \textsc{lpn\_asp}     & Lodgepole Pine with Aspen                    & 8             & 31   \\
\rowcolor[HTML]{CAD6BA} \textsc{rfr\_asp}     & Red Fir with Aspen                           & 0                    & 34             \\
                        \textsc{sage    }     & Big Safebrush                                & 0                    & 1,600          \\
\rowcolor[HTML]{CAD6BA} \textsc{lsg     }     & Black and Low Sagebrush                      & 0           & 5         \\
                        \textsc{scn\_asp}     & Subalpine Conifer with Aspen                 & 0                    & 6              \\
\rowcolor[HTML]{CAD6BA} \textsc{ypn     }     & Yellow Pine                                  & 0                    & 10,499         \\
                        \textsc{ypn\_asp}     & Yellow Pine with Aspen                       & 0                    & 3              \\ \midrule

                        \textsc{wat     }     & Water                                        & 4,058          & 8,212          \\
\rowcolor[HTML]{CAD6BA} \textsc{bar     }     & Barren                                       & 2,665        & 8,751     \\
                        \textsc{grass   }     & Grassland                                    & 1,379        & 4,617     \\
\rowcolor[HTML]{CAD6BA} \textsc{med     }     & Meadow                                       & 1,201        & 3,435     \\
                        \textsc{urb     }     & Urban                                        & 114                  & 782            \\
\rowcolor[HTML]{CAD6BA} \textsc{agr     }     & Agriculture                                  & 16          & 5,416     \\ \bottomrule

\end{tabular}
\end{table}
\normalsize

\paragraph{Seral Stage}
Seral stage classes combine developmental stage and canopy cover, and are defined for all cover types that undergo succession. Seral stages in this application are based on LandFire structural classes, and were further modified in collaboration with local experts on the Tahoe National Forest. In \textsc{RMLands}, susceptibility to and mortality from natural disturbances varies among seral stages. Unlike the cover grid, the seral stage grid changes dynamically over time in response to simulated succession and disturbance events. The combination of cover type and seral stage forms the basis for characterizing vegetation patterns and dynamics.

The source for the seral stage layer is the Region 5 Existing Vegetation Layer, mapped to the \textsc{calveg} classification. The \textsc{calveg} classification was developed by the Region's Ecology Program in 1978. Within the project area, the Existing Vegetation Layer was developed based on three separate efforts: a satellite-based imagery analysis in 2000, and two orthoimagery analysis completed by contracting firms in 2005. All members of the team discussed potential attributes to be used for this classification, and identified attributes for tree diameter at breast height and cover from above to classify pixels into early, middle, or late development, and open, moderate, and closed canopy. In this application, aspen and shrub types have seral stages that differ from that of the remaining forest types. The other forested types use a consistent set of seral stages.

Extensive geoprocessing was required to prepare this layer for \textsc{RMLands}. Beyond converting the vector data to a raster format, further analysis was required to update the layer to a year 2010 condition. Spatial data on wildfire and timber management history was used to provide a more accurate assessment of seral stage based on estimated stand age. In addition, areas currently mapped as chaparral in the Existing Vegetation Layer were assigned to the early development stage. The full set of seral stages is provided in Table~\ref{condtable} and depicted in Figure~\ref{fig:conditionmap}.

%%%%%%%%%%%%%%%%%%%%%%
%%% CONDITION TABLE %%
%%%%%%%%%%%%%%%%%%%%%%

\begin{table}[!htbp]
\small
\centering
\caption{List of seral stages developed for this project. Seral stages describe developmental stage (e.g. ``early'') and canopy closure (e.g. ``open''). The non-seral ``stage'' applies to land cover types for which I do not simulate succession (Barren, Grassland, Urban, Agriculture, Water, and Meadow). Included are the seral stage abbreviations, and full names.}
\label{condtable}
\begin{tabular}{@{}ll@{}}
\toprule
\textbf{\begin{tabular}[c]{@{}l@{}}Seral Stage \\ Abbreviation\end{tabular}} & \textbf{\begin{tabular}[c]{@{}l@{}}Seral Stage  \\ Name\end{tabular}} \\ \midrule
\rowcolor[HTML]{CAD6BA}   \textsc{ns}             & Non-Seral                     			    \\
                          \textsc{early\_all }    & Early 					                     \\
\rowcolor[HTML]{CAD6BA}   \textsc{mid\_cl    }    & Mid--Closed                    \\
                          \textsc{mid\_mod   }    & Mid--Moderate                  \\
\rowcolor[HTML]{CAD6BA}   \textsc{mid\_op    }    & Mid--Open                      \\
                          \textsc{late\_cl   }    & Late--Closed                   \\
\rowcolor[HTML]{CAD6BA}   \textsc{late\_mod   }   & Late--Moderate                 \\
                          \textsc{late\_cl     }  & Late--Open                     \\
\rowcolor[HTML]{CAD6BA}   \textsc{early\_asp  }   & Early--Aspen                   \\
                          \textsc{mid\_asp   }    & Mid--Aspen                     \\
\rowcolor[HTML]{CAD6BA}   \textsc{mid\_ac    }    & Mid--Aspen Conifer             \\
                          \textsc{late\_ca    }   & Late--Conifer Aspen            \\ \bottomrule
\end{tabular}
\end{table}



\paragraph{Age}
Age represents the number of years since the last stand-replacing disturbance (high mortality wildfire). Because the characteristic species of a given cover type may not immediately establish after a stand-replacing fire, it is likely that the age value is larger than the actual age of the oldest individuals in a stand. Several of the cover types in this area may go through a chaparral-dominated early development stage; in those cases the oldest trees in the stand could be decades older than the formal stand age. In \textsc{RMLands}, age is used to trigger potential successional transitions and to calculate susceptibility to disturbance. In this application, we rounded all modeled and derived ages to the nearest five years (the length of one timestep).

In the HRV analysis, the initial age value assigned to a given cell is not necessarily important to the outcome of the simulation, due to the exclusion of the first (in our case) 40 timesteps from the analyzed results. In the future scenario analysis, the initial age value carries more weight, because the total simulation length is only 18 timesteps.

In this application, we used data from stand exams dating to the 1960s and from recent Forest Service Region 5 Ecology group survey plots to estimate stand age across the buffered project area. We then interpolated that information across the landscape. Due to insufficient data, we were unable to disaggregate the data below the landscape scale to cover type or another more finely resolved classification. We also acknowledge that the stand exam and modern veg plots do not constitute a true sample and were conducted almost exclusively in mid-mature and mature stands of commercially viable trees, thus skewing the results to some unquantifiable degree.

We updated the interpolated data with wildfire and timber management history, and assigned ages to types coded as chaparral in the Existing Vegetation layer to the midpoint of the age spread of early development for the forest cover type to which it was converted. Remaining ages out of compliance with allowed ages for the corresponding seral stage of a given cell were modified to be in compliance, based on the assumption that the seral stage assignment was more accurate that the interpolated age information. The input Age layer, showing the map at timestep 0, is shown in Figure~\ref{fig:agemap}.


\paragraph{Seral stage-Age}
Seral stage-Age represents the age since transitioning to the current seral stage. In \textsc{RMLands} it affects most transitions between seral stages: typically there is a threshold seral stage-age below which transitions do not occur. After creating both the seral stage and age layers, we used a Python function to derive seral stage-age based on the youngest possible age for a cell of that cover and seral stage. For example, if we determine that a particular cell on the landscape has a cover type of Lodgepole Pine, seral stage of Mid Development Closed, and age of 50 years, we take the minimum age for that cover-seral stage combination (10 years old), and subtract it from the age to arrive at a seral stage-age of 40. The same caveats and assumptions that apply to the seral stage and age layers also apply to the seral stage-age layer. The final, original map for the initial seral stage-age is shown in Figure~\ref{fig:condagemap}.


\paragraph{Topographic Position Index}
Our topographic position index (TPI) combines heat load, which is based on aspect and slope, with slope position (Figure~\ref{fig:tpimap}), and ranges from -300 to 300. High values for TPI are correlated with locations on steep, south and west-facing, upper slopes. Low values are correlated with locations on gentle, north and east-facing, valley bottoms. Values in between occur along a gradient of these characteristics. The TPI is scaled to the project area and the region immediately surrounding it, and is therefore a local index only. We implemented this in order to better mimic fire behavior on the landscape. Past research has indicated that fires generally burn more frequently on southerly, steep slopes than on gentle, northerly slopes. Because they are drier and more exposed, and because of the way fuels allow preheating and ready ignition of vegetation as fire travels uphill \citep{Rothermel1983}, the likelihood of overstory tree mortality increases with increasing TPI (these statements may also be interpreted as the inverse, wetter valley bottoms are less likely to produce a crown fire) \citep{North2012,Taylor2003a,}. We use TPI to adjust vegetation susceptibility and mortality, as described in the model parameterization section. As part of model calibration, we plot the average canopy cover (over the simulated period) against TPI and calculate the proportional effect of TPI on cover.



\paragraph{Elevation} 
Elevation represents the height above sea level in meters. In \textsc{RMLands}, the elevation layer affects disturbance spread. The elevation grid used in this analysis was a digital elevation model (DEM) provided by the Tahoe National Forest GIS staff and rescaled from 10 m$^2$ to 30 m$^2$ pixels. It is shown as a map in Figure~\ref{fig:elevationmap}.



\paragraph{Slope} 
Slope represents the steepness of a cell as measured in percent and is derived from the elevation layer. Slope is used in GIS preprocessing to define cover types. Within \textsc{RMLands}, slope affects disturbance spread. The slope for the study area was derived from the elevation layer described above, and is shown in Figure~\ref{fig:slopemap}.


\paragraph{Aspect} Aspect represents the direction a cell is facing in terms of eight cardinal directions. Flat aspects are also recognized. Within \textsc{RMLands}, aspect affects disturbance spread. The aspect for the study area was derived from the elevation layer described above, and is shown in Figure~\ref{fig:aspectmap}. 


\paragraph{Streams} 
Streams represents linear hydrological features, classified as small, medium or large based on stream order. In this application, the streams layer was created from a line coverage containing hydrography data, including an attribute for stream size or order, by converting to a grid based on the stream size attribute. Streams may inhibit the spread of wildfire in \textsc{RMLands}, depending on both stream size and potential wildfire size. In order to function as a barrier, cells in the stream raster coded as streams much share a side, rather than only a vertex. The stream layer was used to overwrite any cells not coded as ``Water'' in the cover type layer, such that all ``large'' (first order) streams are represented as water in the cover type layer. The final streams input layer is shown in Figure~\ref{fig:streamsmap}.


\paragraph{Buffer/Core} 
This layer identifies and distinguishes the ``core'' project area from the 10 km ``buffer'' added to allow wildfires to initiate outside of and burn beyond the formal project area. Without a buffer, edge effects would alter results for all aspects of the disturbance regime, as well as resulting landscape composition and configuration. A 10 km buffer was selected arbitrarily, but has in the past been sufficient to offset edge effects (McGarigal, personal communication). Because the simulation plays out on the full extent of the core plus the buffer, all input grids are developed to that larger extent as well. To create this raster, the original project polygon was buffered in ArcMap by 10 km, then converted to raster using the same procedure as for other layers. The buffer and core are easily distinguished in Figure~\ref{fig:inputlayermaps}: the core is the interior area delinated by a thick black line, while the buffer is area outside of this line, displayed at a decreased brightness level.


\begin{figure}[!htbp]
  \centering
  \subfloat[][]{
    \centering
	\includegraphics[width=0.4\textwidth]{/Users/mmallek/Tahoe/Report3/images/cover_resized.png}
    \label{fig:covermap}
  } \qquad
  \subfloat[][]{
    \includegraphics[width=0.4\textwidth]{/Users/mmallek/Tahoe/Report3/images/condition_resized.png}
    \label{fig:conditionmap}
  } %\\
  %
	\subfloat[][]{
    \includegraphics[width=0.4\textwidth]{/Users/mmallek/Tahoe/Report3/images/age_resized.png}
    \label{fig:agemap}
  } \qquad
	\subfloat[][]{
    \includegraphics[width=0.4\textwidth]{/Users/mmallek/Tahoe/Report3/images/condage_resized.png}
    \label{fig:condagemap}
  } 
  %
  \subfloat[][]{
    \centering
	\includegraphics[width=0.4\textwidth]{/Users/mmallek/Tahoe/Report3/images/tpi_resized.png}
    \label{fig:tpimap}
  } \qquad
  \subfloat[][]{
    \includegraphics[width=0.4\textwidth]{/Users/mmallek/Tahoe/Report3/images/elevation_resized.png}
    \label{fig:elevationmap}
  } %\\
  %
	\subfloat[][]{
    \includegraphics[width=0.4\textwidth]{/Users/mmallek/Tahoe/Report3/images/slope_resized.png}
    \label{fig:slopemap}
  } \qquad
	\subfloat[][]{
    \includegraphics[width=0.4\textwidth]{/Users/mmallek/Tahoe/Report3/images/aspect_11oct.png}
    \label{fig:aspectmap}
  } 
  %\\
  	\subfloat[][]{
    \includegraphics[width=0.4\textwidth]{/Users/mmallek/Tahoe/Report3/images/streams_resized.png}
    \label{fig:streamsmap}
  } 

  \caption{RMLands input layers. (a) Cover type map (b) Seral stage map (c) Age map at Timestep 0 (d) Seral stage-Age map at Timestep 0 (e) Topographic Position Index (f) Elevation (g) Slope (h) Aspect (i) Streams}
  \label{fig:inputlayermaps}
\end{figure}


\subsection{Model Parameterization}
\label{subsec:hrvmodelparam}

\paragraph{State and Transition Models}
We have created a detailed cover type description document for each cover type in the simulated landscape that experiences transitions between cover class. These documents describe crosswalks to other data layers, detailed accounts of the multiple species characteristic of the cover type, cover type distribution, relationship and response to wildfire, predicted fire rotations, plus descriptions of each seral stage present within the cover type and their succession and wildfire transition conditions and rates (Appendix \ref{app:covertypedesc}). Each detailed document can be summarized as a state and transition model for a particular cover type, which is implemented in the model by specifying susceptibility to wildfire, rules for vegetational succession, and rules for transitions after a fire event. Figure~\ref{transmodel} shows a generic example state and transition model for the forested cover types.

An important characteristic of \textsc{RMLands} is that it treats fire somewhat different from other landscape succession and disturbance models, which affected how we created and specified the state and transition models. In \textsc{RMLands}, fires spread probabilistically based on the susceptibility of an individual cell. It does not contain a fire model and fuels are not directly incorporated into fire spread. In addition, we do not classify individual fires as a whole to a ``low,'' ``mixed,'' or ``high'' severity status. Some fire ecologists combine fire attributes such as flame length and fire size into their interpretation of the relative ``severity'' of a particular fire \citep{Agee1993}.   Ecologists working at other scales and not working with models often describe ``mixed severity'' regimes \citep[e.g.,][]{Kane2013}, which \citet{Collins2010} define as ``stand-replacing patches within a matrix of low to moderate fire-induced effects.'' Because at the 30 m cell size of our model, nearly all fires would be classified as ``mixed severity'' by the prior definition, it becomes moot. Instead, we focus on defining conditions under which transitions among potential states within a given cover type occur or not. All burned cells are evaluated probabilitistically and assigned to a high severity--``high mortality'' outcome or not. All non-high mortality outcomes are considered low mortality. If a cell burns at high severity, then it transitions to the Early seral stage. Recently, some researchers have differed on whether 75\% or 95\% overstory tree mortality is a more appropriate cutoff point for defining a ``stand-replacing'' event \citep{Fule2014,Mallek2013}. In this paper, we use 75\% as our cutoff, which is widely accepted in the literature \citep{Miller2009,Baker2014,Agee2007,Agee1993}. 

To derive probabilities for post-fire transitions, both for transitions to the Early seral stage or to a more open seral stage, we used the VDDT models associated with the Biophysical Settings Models from the LandFire project. From the VDDT models, we used the probabilities of a transition to the early seral stage, a more open canopy seral stage, or of no transition. We ignored the classified type of fire (as replacement, mixed, or low severity), focusing instead on the outcome from fire in terms of the seral stage, if any, to which a cell transitioned after wildfire. High mortality fires are those that result in conversion to early seral (regardless of whether they are called ``replacement'' or ``mixed''). All other fires are considered low mortality. The probability of a high mortality outcome from fire was calculated by dividing the summed probabilities of high mortality fires as defined above by the summed probabilities of all fires.  We also used LandFire data to derive probabilities of succession. We then evaluated and refined the probabilities with input from local experts to capture subtle changes in succession and transition at the project scale.


\begin{figure}[htbp]
\centering
\includegraphics[width=0.8\textwidth]{/Users/mmallek/Tahoe/Report3/images/state_trans_model.pdf}
\caption{Generic state and transition model for all non-shrub seral cover types. Each dark grey box represents one of the seven seral stages for this landcover type. Each column of boxes represents a stage of development: early, middle, and late. Each row of boxes represents a different level of canopy cover: closed (70-100\%), moderate (40-70\%), and open (0-40\%). Transitions between states/seral stages may occur as a result of high mortality fire, low mortality fire, or succession. Specific pathways for each are denoted by the appropriate color line and arrow: red lines relate to high mortality fire, orange lines relate to low mortality fire, and green lines relate to natural succession.} 
\label{transmodel}
\end{figure}

To illustrate the parameterization, in the following tables we present values for the Sierran Mixed Conifer - Mesic cover type model. The target fire rotation for this cover type is 29 years. A fire return interval index is used as the parameter controlling the relative susceptibility to fire of the seral stages within an individual cover type. In addition, the probability of high severity fire leading to at least 75\% overstory mortality is specified for each seral stage (Table~\ref{smcm_fri_phm}). We also specified transition probabilities for natural succession between the early, middle, and late stages of development, as well as between closed, moderate, and open canopy cover. This type of succession also depends on the time in the current seral stage both in terms of the early-middle-late sequence (\emph{Development-Age}) and the specific stage-canopy cover combination (\emph{Seral Stage-Age}) (Table~\ref{smcm_vegtrans}). Finally, probabilities are specified for vegetation transitions after wildfire (Table~\ref{smcm_firetrans}).



% edited 2015-9
\begin{table}[htbp]
\small
\centering
\caption{Relative susceptibility to fire and proportion of high severity fire (at least 75\% overstory tree mortality) probabilities for Sierran Mixed Conifer - Mesic.}
\label{smcm_fri_phm}
\begin{tabular}{lcc}
\hline
\textbf{Seral Stage}    & \textbf{\begin{tabular}[c]{@{}c@{}}Relative Susceptibility \\ to Fire\end{tabular}} & \textbf{\begin{tabular}[c]{@{}c@{}}Probability of \\ High Severity Fire\end{tabular}} \\ \hline
Early (All)     			& 5.5        & 1                 \\
Mid--Closed    				& 2.4        & 0.23              \\
Mid--Moderate  				& 1.6        & 0.17              \\
Mid--Open      				& 1.3        & 0.14              \\
Late--Closed   				& 4.3        & 0.37              \\
Late--Moderate 				& 1.6        & 0.14              \\
Late--Open     				& 1          & 0.09              \\ 
\emph{Target Fire Rotation}    			& \emph{29 years}  &   \\ \hline
\end{tabular}

\end{table}

\begin{table}[!htbp]
\small
\centering
\caption{Timeframes for transitions between seral stages in \textsc{RMLands} for Sierran Mixed Conifer - Mesic. ``Early to Mid'' and ``Mid to Late'' times are based on the time in a developmental stage, regardless of disturbance history. ``Open to Moderate'' and ``Moderate to Closed'' times are based on the time in a seral stage since a disturbance.}
\label{smcm_vegtrans}
\begin{tabular}{cccc}
\hline
\textbf{\begin{tabular}[c]{@{}c@{}}Seral Stage \\ Transition\end{tabular}} & \textbf{Minimum (years)} & \textbf{Average (years)} & \textbf{Maximum (years)} \\ \hline
Early to Mid 	& 20      & 26      & 40      \\
Mid to Late 	& 100     & 113     & 150     \\
\begin{tabular}[c]{@{}c@{}}Open to Moderate or\\ Moderate to Closed\end{tabular}  & 15      & 21      &    ---     \\ \hline
\end{tabular}

\end{table}


\begin{table}[!htbp]
\small
\centering
\caption{Transition probabilities for Sierran Mixed Conifer - Mesic following low mortality fire.}
\label{smcm_firetrans}
\begin{tabular}{lcc}
\hline
\textbf{Seral Stage Transition} & \textbf{Probability}\\
\hline
Mid--Closed to Mid--Moderate     	& 0.53   	\\
Mid--Moderate to Mid--Open    	& 0.36		\\
Late--Closed to Late--Moderate	& 0.54    \\
Late--Moderate to Late--Open     	& 0.24    \\
\hline
\end{tabular}
\end{table}

Transitions between Early and Middle Development, and between Middle and Late Development are governed by the time in the Early or Middle stage (canopy cover usually does not affect these probabilities). These transitions may begin at the minimum time in a specified \emph{Development-Age}, and proceed at rates that vary across cover types. Table~\ref{smcm_vegtrans} displays the average \emph{Seral Stage-Age} of transition. If a cell reaches the maximum stage-age listed, its probability of transitioning goes to 1. 

Transitions between the canopy cover types occur within one developmental stage: i.e., between Mid--Open and Mid--Moderate, but not between Mid--Open and Late--Moderate. These transitions are governed by the time in the full seral stage specified since the last disturbance. This ``years since'' value may be affected by a low mortality fire, a transition between developmental stages, or a transition between canopy cover levels. Similarly to the developmental transitions, the shift from, for example, Mid--Open to Mid--Closed may begin when the minimum time is reached, and also proceeds at rates that vary across cover types. No maximum age is specified for this type of transition.

\paragraph{Disturbance Parameters} 
\label{subsubsec:distparams}

%\begin{adjustwidth}{5ex}{0pt}
\begin{itemize}
\item \emph{Climate:} The climate parameters are based on a rescaling of the Palmer Drought Severity Index (PDSI). PDSI is a long-term measure of drought, on the scale of months to years. It is based on precipitation and temperature and incorporates soil moisture. Resconstructed PDSI values for summer months during the historic period of this project (1550-1850) are available from the National Oceanic and Atmospheric Administration (\burl{http://www.ncdc.noaa.gov/paleo/pdsi.html}). We used datasets from \citet{Zhangetal.2004} and \citet{Cook2004}. These data are summarized at large scales; for example, the \citet{Cook2004} data are calculated for a grid with points spaced at 2.5\textdegree. We selected the five closest points to the center of the project area from these two datasets and calculated the inverse distance-weighted mean of the values. We then converted the yearly data into five-year averages to align with the five-year timesteps in our model. By recentering the mean value around 1 and then taking the inverse, we create a dataset in which a value of 1 is neither wetter nor dryer than average, values between 0 and 1 represent wetter-than-normal timesteps, and values greater than 1 represent dryer-than-normal timesteps (Figure~\ref{pdsi}). Climate interacts with other disturbance parameters in \textsc{RMLands}, including initiation, susceptibility, and spread.

\begin{figure}[htbp]
\centering
\includegraphics[height=0.3\textheight]{/Users/mmallek/Documents/Thesis/Plots/pdsi/hrv-bigtext.png}
\caption{Palmer Drought Severity Index, rescaled, inverted, and presented as a 5-year average for the ``historical'' period in this study (1550-1850).} 
\label{pdsi}
\end{figure}

%\medskip

%\noindent 
\item \emph{Susceptibility:} Cover and seral stage are both inputs to susceptibility. Cover modifies susceptibility via the ability to specify the influence of topographic position on susceptibility (Table~\ref{covtpi}). The magnitude of this effect is estimated as a potential reduction in susceptibility of 30\% between the minimum and maximum TPI values used in the model. We implement this by using a logistic function to convert the TPI grid values into an appropriate multiplier within the susceptibility equation:

$$\text{TPI Susceptibility Factor} = L + \frac{R-L}{1+e^{k(x_0-x)}}$$

in which $L= 0.7$, $R=1$, slope $k=1$, inflection point $x_0=0$, and $x=\text{TPI}$.%at a given cell
It therefore simplifies to 

$$\text{TPI Susceptibility Factor} = 0.7 + \frac{0.3}{1+e^{-x}}$$

In this way, when the TPI Susceptibility Factor = 1, TPI has no effect. This happens only when the TPI value at an individual cell is zero. The effect of this is that areas with low TPI (generally north-facing and flatter slopes) burn less frequently than areas with high TPI (generally south-facing and steeper slopes).


\begin{table}[htbp]
\small
\centering
\caption{Cover types whose susceptibility is modified by Topographic Position Index. All cover types are modified in the same way.}
\label{covtpi}
\begin{tabular}{ll}
\hline
\multicolumn{2}{c}{\textbf{Cover Types with TPI Adjustment}} \\
\hline
Grassland     									& Red Fir - Mesic   					\\
Lodgepole Pine    								& Red Fir - Ultramafic					\\
Mixed Evergreen - Mesic							& Red Fir - Xeric    					\\
Mixed Evergreen - Ultramafic     				& Sierran Mixed Conifer - Mesic    		\\
Mixed Evergreen - Xeric 						& Sierran Mixed Conifer - Ultramafic 	\\
Montane Riparian								& Sierran Mixed Conifer - Xeric 		\\
Oak Woodland 									& Western White Pine					\\
Oak-Conifer Forest and Woodland 				& Yellow Pine 							\\
Oak-Conifer Forest and Woodland - Ultramafic 	&										\\
\hline
\end{tabular}

\end{table}

Seral stage further modifies susceptibility. We use the Weibull cumulative distribution function and specify a scale parameter $\lambda$ (mean return interval), shape parameter $k$, and the reset point for the function (\emph{age since high mortality disturbance} or \emph{age since any disturbance}). The fire return index for the seral stage is used as a calibration parameter and was initially set as equal to the mean return interval values provided in analogous LandFire Biophysical Setting types \citep{Landfire2007}. Some modifications were made based on consultation with Forest Service staff. All fire index values within a cover type were modified as a group and kept relative to one another even as the magnitude of the index was adjusted. The relative difference for Sierra Mixed Conifer - Mesic is shown in Table~\ref{smcm_fri_phm}; these values for each cover type are included in the cover type description documents (Appendix \ref{app:covertypedesc}). We set $k=3$ for all cover types and seral stages. We selected between (\emph{age since high mortality disturbance} and \emph{age since any disturbance}) based on whether wildfires in that cover type are climate-driven (in which case we select the former) or fuels-driven (in which case we select the latter) (Figure~\ref{howdriven}).

\begin{table}[htbp]
\small
\centering
\caption{Cover types sorted by whether wildfire disturbance in them is characterized by fuels present or overarching climatic conditions. If the likelihood of wildfire depends on the accumulation of fuels, the value of $x$ (``time since'') reverts to 0 after any disturbance. If the likelihood of wildfire depends primarily on climate and weather conditions, the value of $x$ reverts to 0 only after a high mortality disturbance.}
\label{howdriven}
\begin{tabular}{ll}
\hline
\textbf{Fuel-Driven Cover Types} 				& \textbf{Climate-Driven Cover Types}	\\
\hline
Curl-leaf Mountain Mahogany 					& Agriculture   						\\
Grassland     									& Big Sagebrush 						\\
Lodgepole Pine    								& Black and Low Sagebrush				\\
Meadow											& Lodgepole Pine with Aspen 			\\
Mixed Evergreen - Mesic							& Montane Riparian						\\
Mixed Evergreen - Ultramafic     				& Red Fir with Aspen   					\\
Mixed Evergreen - Xeric 						& Red Fir - Mesic    					\\
Oak Woodland 									& Red Fir - Ultramafic 					\\
Oak-Conifer Forest and Woodland 				& Red Fir - Xeric 						\\ 	
Oak-Conifer Forest and Woodland - Ultramafic 	& Subalpine Conifer 					\\
Sierran Mixed Conifer - Ultramafic 				& Subalpine Conifer with Aspen 			\\
Sierran Mixed Conifer - Xeric 					& Sierran Mixed Conifer with Aspen 		\\
Urban 											& Sierran Mixed Conifer - Mesic 		\\
Yellow Pine 									& Western White Pine 					\\
												& Yellow Pine with Aspen 				\\
\hline
\end{tabular}
\end{table}


%\medskip

%\noindent 
\item \emph{Initiation:} In \textsc{RMLands}, parameters for initiation are used as calibration parameters. The probability of wildfire initiation is a function of its susceptibility to wildfire and the climate modifier value for that timestep, and is applied at the cell level. The ignition calibration coefficient refers to the number of attempted ignitions per 100,000 ha per year. For the HRV simulation, we set this coefficient at 42. We applied the coefficient evenly across the landscape based on local expert knowledge of lighting strike locations in the area. Fires may be initiated anywhere within the project area or the 10 km buffer around it. The total area cover within that boundary is 409,411 ha, so up to 860 fire starts were possible during each 5-year timestep in our simulation (not all potential ignitions result in fire). Climate also influences initiation.

%\medskip

%\noindent 
\item \emph{Spread:} The probability of fire spread in \textsc{RMLands} is a function of climate, susceptibility to wildfire, potential wildfire size, wind, spotting, relative elevation, and presence of streams. The first two are described above. The disturbance size distribution that regulates potential fire size was created by analyzing the size distribution of all mapped fires in the Northern Sierra \textsc{calveg} mapping zone and west of the Sierran crest, available from the Forest Service and the California Department of Forestry and Fire Protection, which goes back to approximately 1900. 

Wind is incorporated in two parts. First, a prevailing wind direction for the fire is selected probabilistically from the eight cardinal directions. To compute the wind distribution values, we first consulted local experts to determine the dates of fire season (May 15 to October 15) and burning period times (1000 hours to 1800 hours). We then downloaded all available historic wind direction data from 6 local weather stations (Rice Canyon, Saddleback, Downieville, White Cloud, Emigrant Gap, and Blue Canyon, Figure~\ref{weather}). Data from all weather stations was weighted equally. After the wind direction is selected, fires are able to grow in all directions, but are relatively more likely to spread with wind than against it. We parameterized the influence of \emph{relative wind} as a reduction in spread likelihood. Thus, spread in the same direction as wind has a neutral effect, spread at $\ang{45}$ angles is reduced by 30\%, spread at $\ang{90}$  angles is reduced by 70\%, spread at $\ang{135}$ angles is reduced by 90\%, and spread opposite the prevailing wind direction is reduced by 95\%. 

\begin{figure}[htbp]
\centering
\includegraphics[width=0.3\textheight]{/Users/mmallek/Tahoe/Report3/images/weather.png}
\caption{Weather stations used to inform wind direction parameters. Weather stations are denoted by red circles. A black boundary line identifies the study area.}
\label{weather}
\end{figure}

Relative elevation also modifies spreading potential. We parameterized the model such that spread downhill is extremely unlikely. Spotting and the extent to which streams act as barriers to spread are affected by the fire size. As fires become larger, their probability of spotting and spotting distance increases. Similarly, streams function as a barrier to smaller fires, but large fires are able to spread past streams regardless of size. This decision is based on the idea that large fires are more influenced by wind and climatic conditions. Stream size does impact smaller fires; the largest streams and rivers are usually an effective barrier to smaller fires, although even fairly small fires often spread past intermittent and small perennial streams. 


\item \emph{Mortality:} Cover and seral stage are both inputs to mortality. Cover modifies susceptibility via the ability to specify the influence of topographic position on mortality (Table~\ref{covtpi_mort}). The magnitude of this effect is estimated as a potential reduction in mortality of 30\% between the minimum and maximum TPI values used in the model. We implement this by using a logistic function to convert the TPI grid values into an appropriate multiplier within the mortality equation:


$$\text{TPI Mortality Factor} = L + \frac{R-L}{1+e^{k(x_0-x)}}$$

in which $L= 0.7$, $R=1$, slope $k=1$, inflection point $x_0=0$, and $x=\text{TPI}$.%at a given cell
It therefore simplifies to 

$$\text{TPI Mortality Factor} = 0.7 + \frac{0.3}{1+e^{-x}}$$

In this way, when the TPI Mortality Factor = 1, TPI has no effect. This happens only when the TPI value at an individual cell is zero. The effect of this is that areas with low TPI (generally north-facing and flatter slopes) are less likely to be high severity than areas with high TPI (generally south-facing and steeper slopes).

\begin{table}[htbp]
\small
\centering
\caption{Cover types whose mortality is modified by Topographic Position Index.}
\label{covtpi_mort}
\begin{tabular}{ll}
\hline
\multicolumn{2}{c}{\textbf{Cover Types with TPI Adjustment}} \\
\hline
Grassland     									& Red Fir - Mesic   					\\
Lodgepole Pine    								& Red Fir - Ultramafic					\\
Mixed Evergreen - Mesic							& Red Fir - Xeric    					\\
Mixed Evergreen - Ultramafic     				& Sierran Mixed Conifer - Mesic    		\\
Mixed Evergreen - Xeric 						& Sierran Mixed Conifer - Ultramafic 	\\
Montane Riparian								& Sierran Mixed Conifer - Xeric 		\\
Oak Woodland 									& Western White Pine					\\
Oak-Conifer Forest and Woodland 				& Yellow Pine 							\\
Oak-Conifer Forest and Woodland - Ultramafic 	&										\\
\hline
\end{tabular}
\end{table}

Seral stage further modifies mortality. We extracted the likelihood of mortality from the VDDT models built during the LandFire project, as described at the beginning of section~\ref{subsec:hrvmodelparam}. As an example, these probabilities for Sierran Mixed Conifer - Mesic are provided in Table~\ref{smcm_fri_phm}.



\end{itemize}
%\end{adjustwidth}

\subsection{Model Calibration}
Although \textsc{RMLands} is a process-based model with parameters sourced from the literature, my team had greater confidence in some parameters than others, especially as to how they function within the \textsc{RMLands} framework. Consequently, I calibrated, or verified, the model by iteratively adjusting certain parameters in which there was less confidence about the appropriate values until the outputs were tuned to a set of parameters in which the team had high confidence. Specifically, I manipulated the ignition calibration coefficient and the fire return index and measured calibration success based on conformity to pre-specified rotation values at the cover type level. Fire return index values were changed by a constant multiplier across all seral stages of a given cover type; cover types were modified as groups but the index ratios within them were maintained. I set our calibration target as rotation values for the nine focal cover types within 10\% of the original target rotations. Target values were based on empirical published values and local expert opinion. I chose rotation as the calibration target because targets were available from the literature and because fire rotation is a fundamental measurement that \textsc{RMLands} was designed to capture. In addition, using rotation ties calibration to a parameter that is relateable to Forest Service staff and that can be used at the landscape scale as a target by managers in various programs.

For example, while calibrating, the target rotation for Sierran Mixed Conifer - Mesic was 29 years. I adjusted the input seral stage fire return index by multiplying it by different constants, eventually arriving at an increase by a factor of 24 from the original calculated ratio values. That is, each initial scale parameter value was multiplied by 24 in order to modify its susceptibility to fire without changing the relative susceptibility among its seral stages. 



%%%
%in a nutshell, we have to calibrate the model because we it's not completely mechanistic and we're making guesses on a lot of things
%so we pick a few model parameters that we have high confidence in, and decide  not to change those
%and then we pick a few model outputs that we have confidence in
%because our goal is to simulate a regime we believe we already understand
%so we want it to look "right"
%so we figure if we can make the model outputs agree with the numbers we're pretty sure are right, then we trust the other outputs where we weren't totally sure what to expect
%and we do this by adjusting the parameters that we have lower confidence we got right hte first time, or that don't relate to the real world direclty and mechanistically



\subsubsection{Model Execution}
During the calibration phase of the model, a typical simulation was three runs of 200 timesteps each. The equilibration period of 40 timesteps was chosen based on visual analysis of the disturbed area and rotation plots from the combined runs. Once calibration was complete, I conducted one run of 500 timesteps in order to capture multiple disturbance and succession cycles across the most common cover types. Each timestep represents five years. The five-year timestep was chosen based on the short fire return intervals (how frequently fire occurred in the same location) recorded from dendrochronology analysis in the literature and our desire to capture these very short return intervals in the simulation.

\subsection{Data Analysis}
\label{subsec:dataanalysis}

Sierra Nevada vegetation is extremely diverse and complex, both ecologically and spatially. In this thesis I limit our results to an evaluation of the full landscape and of xeric and mesic mixed conifer forests, which together comprise 63\% of the study area. Results for the next seven most extensive types (xeric and mesic mixed evergreen forests, xeric and mesic red fir forests, oak-conifer forests and woodlands, ultramafic oak-conifer forests and woodlands, and ultramafic mixed conifer forests) are included in the appendices. In general, our confidence in the results decline as the extent of a cover type declines, because the results are statistical and large samples are needed. I do not provide results for cover types that extend across less than 1000 ha of the study area.

\paragraph{Disturbance Regime} I quantified the following overall temporal and spatial characteristics of the wildfire disturbance regime:
\begin{itemize}
	\item \emph{Disturbed Area:} I calculated disturbed area for each timestep, divided into low mortality and high mortality disturbance, and summed to produce an ``any mortality'' statistic. I summarize the results for minimum, maximum, mean, and median area disturbed as a proportion of the total area eligible for disturbance for the full simulation excluding the equilibration period (460 timesteps, or 2300 years). Because it can be difficult to visualize what our quantitative results look like, I include several maps that illustrate the results, demonstrating that model results are spatially-explicit and realistic. To do this, I include maps of the landscape illustrating the $5^{\text{th}}$, $50^{\text{th}}$, $95^{\text{th}}$, and mean area burned during the simulation, plus a example 4-timestep sequence illustrating changes to the seral stage pattern for mesic mixed conifer forests due to successional and disturbance procceses. Finally, I use a histogram to display the distribution of wildfire extents during the simulation, excluding the equilibration period.
	\item \emph{Disturbance Frequency:} I calculated the number of years between disturbances exceeding a particular threshold in total disturbed area. I report the frequency of timesteps during which thresholds of at least 10\%, 25\%, or 50\% of the landscape experienced wildfire.
	\item \emph{Climate Effect:} Climate interacts with several components of the model. I present plots illustrating the value of the climate parameter by timesteps concurrently with the area disturbed per timestep. It is not practical to further illustrate its effect everywhere, and in some cases its influence is not easily separated from the other inputs to the model. 
	\item \emph{Rotation Period:} I calculated the rotation period---the number of years required to burn an area equivalent to the total eligible area---for each cover type within the project area. I report the rotation values for low mortality fire, high mortality fire, and any fire for each of the nine focal cover types individually and the study area as a whole.
	\item \emph{Return Interval:} I summarized the cell-specific population mean fire return interval---the average number of years between disturbances at a single cell---and present it as the distribution of the percentage of eligible cells that experienced each possible mean return interval. I use histograms to visualize the distribution of this return interval for low mortality fire, high mortality fire, and any fire, along with their median values. This method is based on the full landscape results and is equivalent to the cell-specific grand mean return interval for a given cover type across the landscape. I also display this result spatially as a map showing the fire return interval for each raster cell across the landscape.
\end{itemize}

\paragraph{Vegetation Response} 

\par \emph{Landscape Composition:} We quantified the distribution and dynamics of landscape composition by cover type. For our single 2500 year simulation (with 200 year equilibration period), we summarize the results in a table and graphically. For the tabular results, we present $5^{\text{th}}$, $25^{\text{th}}$, $50^{\text{th}}$, $75^{\text{th}}$, and $95^{\text{th}}$ percentiles of the distribution. We compared the current landscape seral stage distribution to this simulated historic range of variability to determine whether the current landscape deviates, and to what degree, from the HRV. 

Using a stacked bar plot, we visualize the proportion of the total area of a given cover type occurring at each seral stage, for each timestep in the model. In addition, we show a bar plot of the current seral stage distribution, allowing a visual comparison between current conditions and the historical range of variability in the distribution of the seral stages. While the bar plots are useful for visualizing the cover-seral stage dynamics, box plots more simply $5^{\text{th}}--95^{\text{th}}$ percentile distribution of interest and compare that to the current landscape values.






\emph{Landscape Structure and Patterns:} We used \textsc{Fragstats} \citep{Fragstats2012} to compute several landscape-level and class-level metrics that summarize landscape structure over the course of the simulation. We present the results in a series of tables and figures. 

The descriptions in Appendix~\ref{app:metricdescriptions} are intended as a general introduction to the \textsc{Fragstats} metrics; for a much more detailed and mathematical description of all \textsc{Fragstats} metrics, see the \href{http://www.umass.edu/landeco/research/fragstats/documents/fragstats.help.4.2.pdf}{documentation}. Each metric is computed on the study area for a single timestep and the results are displayed in tabular format by quantiles and in graphical format with line graphs and boxplots. Table~\ref{tab:fragland-desc} summarizes the \textsc{Fragstats} metrics selected as focal metrics to provide a simple and understandable explanation of the characteristics of landscape structure during the simulated HRV. We selected metrics to represent commonly identified groups of landscape metrics: patch area and edge, patch shape complexity, core area, aggregation, and diversity \citep{McGarigal2015}. It is fairly intuitive to understand how these metrics may be affected by natural disturbance and human management efforts, thus allowing us to describe the HRV and develop suggestions tying management actions to results for these metrics.

One of the principal purposes of gaining a better quantitative understanding of the historic reference period is to know whether recent human activities have caused landscapes to move outside their historical range of variability. 


\begin{table}[!htbp]
\footnotesize
\centering
\caption{A subset of \textsc{Fragstats} metrics we selected to emphasize in order to provide a parsimonious explanation of the variability in landscape structure during the simulated HRV. An `X' in the landscape or class column denotes whether that metric is calculated at that level. Abbreviations are included because they are used in tables and figures later in the document and appendices for space.} 
\label{tab:fragland-desc}
%
\begin{tabular}{@{}llccc@{}}
\toprule
{\bf Metric}                    & {\bf Abbreviation} & {\bf \begin{tabular}[c]{@{}c@{}}Landscape-\\ level\end{tabular}} & {\bf \begin{tabular}[c]{@{}c@{}}Class-\\ level\end{tabular}} & {\bf Category} \\ 
\midrule
Edge Density                    & \textsc{ed} 			& X        & X     & area and edge metric		\\ 
Area-Weighted Mean Area         & \textsc{area\_am}  	& X        & X     & area and edge metric		\\
Area-Weighted Mean Shape        & \textsc{shape\_am} 	& X        & X     & shape metric 				\\
Area-Weighted Mean Core Area    & \textsc{core\_am}  	& X        & X     & core area metric		\\
Contagion                       & \textsc{contag} 		& X        & --    & aggregation metric		\\
Clumpiness Index                & \textsc{clumpy} 		& --       & X     & aggregation metric		\\
Simpson’s Evenness Index        & \textsc{siei}      	& X        & --    & diversity metric		\\
\bottomrule
\end{tabular}
\end{table}



For both the composition and pattern metrics, we quantified the current landscape's departure from the HRV conditions by calculating a departure index for each cover-seral stage type and each \textsc{Fragstats} metric. We summarized the distribution of each metric calculated over the length of the simulation, minus the equilibration period. We computed the $5^{\text{th}}$, $25^{\text{th}}$, $50^{\text{th}}$, $75^{\text{th}}$, and $95^{\text{th}}$ percentiles of the distribution of observed values. We calculated a current percentile of the range of variability value (\%RV) by computing where along the $0^{\text{th}}-100^{\text{th}}$ percentile range of variability for the simulated historical period the current landscape metric value falls. If the current landscape is outside of the HRV, its current \%RV value is noted as either 0 or 100. The departure index indicates the distance from the 50$^{\text{th}}$ percentile value for a given metric. A value of 0 means that the current value is identical to the median from the simulated HRV, and a value of either less than -95 or greater than 95 means that the current value is below or above and outside the simulated HRV. This departure index is computed by subtracting 50 from the \%RV, then dividing by 50 and multiplying by 100. Thus, for the landscape metric \emph{Patch Density}, 19.507 is equivalent to the 32$^{\text{nd}}$ percentile of observations during the HRV simulation, and the departure index is $(39-50)/50*100 = -22$). This value is within the HRV for the landscape. However, the landscape metric \emph{Edge Density} is 100, because $128.875 > 125.316$, the largest value observed during the HRV simulation. Therefore, edge density at the landscape level is outside the HRV.

\todo{I think I'd like to use the same standards I developed for FRV, actually. Based on interquartile range (within), secondary range (moderate), or outside (full).}

\clearpage

%%%%%%%%%%%%%%%%%%%%%%%%%%%%%%%%%%%%%%%%%%%%%%%%%%%%%%%%%%%%%%%%%%%%%%%%%%%%%%%%%%%%%%%%%%%%%%%%%%%%%%%%%%%%
\subsection{Model Assessment}

\paragraph{Sensitivity Analysis} Ideally, a sensitivity analysis would be performed to assess the sensitivity of the input parameters to the model, and subsequently indicate areas for future research. In this case, I did not complete a rigorous sensitivity analysis due primarily to practical constraints. The chief constraints were related to time, for rerunning the model many times under varying parameter sets, and disk space, due to the large amount of data generated by each run of the model. Despite this, as part of the model calibration process, I gained insight into the relative sensitivity of some parameters. Thus I can offer a qualitative sensitivity analysis. First, the ignition parameter is quite sensitive; changing it by a few interval values changes model outcomes for most analysis measures. Presumably this happens because increasing the number of potential fire starts increases the odds of a fire initiating on a susceptible cell (A formal evaluation of this effect is outside the scope of this project.). In comparison, the fire return index is relatively insensitive; I often modified it by more than an order of magnitude in order to effect a small change in the rotation outcome. Third, the probability of high mortality fire at the seral stage level is fairly sensitive. This is logical because the conversion of forest to early seral conditions directly impacts most of the metrics by which we evaluate landscape structure and composition.

Of the parameters observed to be more sensitive, the seral stage-level likelihood of high mortality fire is the one whose effects on model outcomes have the most important implications for my results, and are most important to invest further research effort on. Currently, probabilities of high mortality fire at the seral stage level are extremely difficult to find in the literature because no record can be taken from a tree completely consumed in a fire. Only a few researchers have attempted to infer high severity fire based on factors such as later reports of dense young conifer or shrub cover \citep{Baker2014}. Most studies of stand-replacing fire have relied on satellite imagery to confirm ``stand-replacement'' effects \citep[e.g.,][]{Collins2010,Mallek2013}. Even if legacy trees\footnote{trees that are much older than the overall stand and that are presumed to have been left standing after a prior stand-replacing disturbance} existed and could be sampled, it would be difficult to determine when and over what extent past stand-replacing fires burned because of low sample sizes associated with legacy trees, as well as other factors such as post-fire drought \citep{Minnich2000,Baker2014}. Furthermore, the studies that derive percent high severity based on imagery produce overall cover type estimates, rather than estimates based on seral stage. Because it is the seral stage estimates that are needed to improve the model, research into this area would fill a gap in our ecological understanding.

\paragraph{Uncertainty Analysis} The model calibration process involved systematically varying certain input parameters and testing resultant model outcomes, which allowed me to complete a rudimentary and qualitative sensitivity analysis. An analogous process did not occur that might approach an uncertainty analysis. Conducting an uncertainty analysis on the parameter set could theoretically be accomplished, but the same constraints exist as for the sensitivity analysis. In addition to these, established uncertainties were not available for all input parameters (such as probability of high mortality fire at the seral stage level) and were often used to parameterize the stochastic part of the model when available (such as succession rates by land cover type and seral stage). Although not conducted as part of this study, an uncertainty analysis would add value to the study results.

\paragraph{Model Validation} I did not conduct a formal model validation, defined as testing the model outputs against independent data. One way to do this would have been to test the parameter set on a neighboring geography with a similar ecological composition. The comparison would then be done either on the results of a hindcasting exercise or on the results of recreating the historical conditions for the next few hundred years. Obviously, the latter exercise is impossible, so model validation could not be accomplished that way. Unfortunately, a hindcasting exercise is also not implementable. First, the model only simulates wildfires; many other small-scale disturbances also occur in the study area and affect landscape pattern. Second, attempts to recreate the current conditions on the landscape would be confounded with the history of vegetation management from the last 150 years. At the time of this study, \textsc{RMLands} functionality for simulating vegetation treatments in the Sierra Nevada was still under development. Even if it was available, however, the existing descriptions of past vegetation treatments are not sufficiently detailed to use in a model validation exercise. A final potential method for validating the model would be to use an old land cover type and seral stage map, and compare its composition and configuration to the HRV results. However, no such map exists, and if it had, I would have used it as my starting condition and eliminated the need for model equilibration.

In addition, it is important to understand that this model can never be fully validated because, while useful, it is like all models an abstract and simplified representation of reality. \textsc{RMLands} was set up to simulate wildfires, but there are many other disturbance processes that exist at varying scales that are not simulated here, including insects and disease, wind-throw, wild ungulate and beaver herbivory, avalanches, and other forms of soil movement. The complex interactions among them that characterize real landscapes are also, as a result, omitted from consideration.

To parameterize the mdoel, I used local empirical data wherever possible. However, I also drew on relevant scientific studies, often from other geographic locations, and relied heavily on expert opinion when scientific studies and local empirical data were not available. As a consequence, the pool of potentially available independent data is limited. 

\paragraph{Model Evaluation} Because true model validation was not possible for this study, a secondary method for validation is to test whether the model outputs make sense ecologically and based on available empirical data. These strategies bleed into model evaluation, or the degree to which the model outputs line up with empirical observations. As outlined in the previous paragraph, the most straightforward method for model evaluation would be to employ a hindcasting strategy, but this is not practicable. 

To some extent, the fact that model calibration was highly successful, in that output fire rotation were within 10\% or less of target values (Table~\ref{rotation-diff}), provides a positive form of model evaluation. I used rotation values as the calibration target because targets were available from the literature and because fire rotation is a fundamental measurement that \textsc{RMLands} was designed to capture. In addition, using rotation ties calibration to a parameter that is relatable to Forest Service staff and that can be used as a target by managers in various programs. 


\begin{table}[!htbp]
\centering
\caption{Comparison of the target versus actual fire rotations recorded during the simulated historical range of variability. Includes calculated final percent difference.}
\label{rotation-diff}
\begin{tabular}{lrrr}
\textbf{Land Cover Type}           & \textbf{Target Rotation} & \textbf{Actual Rotation} & \textbf{Percent Difference} \\
Mixed Evergreen - Mesic            & 50    & 52    & 4\%   \\
Mixed Evergreen - Xeric            & 40    & 41    & 3\%   \\
Oak-Conifer Forest and Woodland    & 21    & 22    & 5\%   \\
Red Fir - Mesic                    & 60    & 63    & 5\%   \\
Red Fir - Xeric                    & 40    & 38    & 5\%   \\
Sierran Mixed Conifer - Mesic      & 29    & 27    & 7\%   \\
Sierran Mixed Conifer - Ultramafic & 60    & 66    & 10\%  \\
Sierran Mixed Conifer - Xeric      & 22    & 23    & 5\% 
\end{tabular}
\end{table}

A second method of model evaluation was a visual inspection of the output grids demonstrating wildfire extents to verify that they were similar to actual wildfire perimeters. In addition, I plotted the actual disturbance size distribution against the expected distribution (Figure~\ref{fig:dsize}). 


% updated 9/13
\begin{figure}[!htbp]
  \centering
    \centering
    \includegraphics[height=0.3\textheight]{/Users/mmallek/Documents/Thesis/Plots/dsize/hrv-ggplot.png}
  \caption{Side by side barplot of the observed and target wildfire size distribution for our 500-timestep long run of the model.}
  \label{fig:dsize}
\end{figure}

As a further effort toward model evaluation, I examined the results of implementating the topographic position index (TPI). The TPI value for a given cell acts as an input into the susceptibility and mortality values otherwise defined for that cover type and seral stage combination. Early development and open canopy seral stages tend to result from fire, and I predicted that an increase in fires and in the likelihood of high mortality fire would lead to a decrease in the average canopy cover values for cells with large TPI values. Table~\ref{tab:tpi_cc} in Appendix \ref{app:full-results} displays the results for this simulation for the nine most common cover types. All show decreased average canopy cover as TPI increases. Figure \ref{fig:tpi_cc_smc} shows the plotted data and fitted linear regression line for mesic and xeric sierran mixed conifer forests. Figure~\ref{fig:averagecc} is a map displaying average canopy cover across the landscape for the full simulated HRV timeframe, excluding the equilibration period. In general, return intervals and canopy cover varied spatially across the forest and decreased with increasing TPI, reflecting empirical observations that higher, more southerly aspects are drier and more susceptible to fires. In mesic mixed conifer forests, canopy cover decreased by about 13\% when comparing minimum to maximum TPI, from an average of 49\% to an average of 43\%. In xeric mixed conifer forests, canopy cover decreased by about 25\% when comparing minimum to maximum TPI, from an average of 36\% to an average of 27\% (Table~\ref{tab:tpi_cc_smcs}).

% figure redone
\todo{this figure should have grey for water/barren; looks like high canopy cover now}
\begin{figure}[!htbp]
\centering
\includegraphics[width=0.8\textwidth]{/Users/mmallek/Documents/Thesis/maps/averagecanopycover.pdf}
\caption{Smoothed visualization of the average canopy cover across the project area over the course of the simulation. Higher percent cover is shown in dark blue, transitioning to red where average percent cover was low. Land cover types Water and Barren have no canopy cover value and appear as grey.}
\label{fig:averagecc}
\end{figure}

% figure redone
\begin{figure}[!htbp]
\centering
\includegraphics[width=.8\textwidth]{/Users/mmallek/Documents/Thesis/Plots/tpi/hrv-facet-smc.png}
\caption{Average canopy cover for Sierran Mixed Conifer Mesic and Xeric during the simulated HRV. Each blue point represents one pixel of an individual cover type on the landscape grid. The black line is the result of a linear regression fit to the data. Table \ref{tab:tpi_cc} provides the numerical representation of the shift from minimum to maximum TPI values for each cover type. (a) Sierran Mixed Conifer - Mesic; (b) Sierran Mixed Conifer - Xeric.} 
\label{fig:tpi_cc_smc}
\end{figure}

%redone 9/15
\begin{table}[!htbp]
\centering
\caption{The percent change in canopy cover from the minimum TPI value for that cover type to the maximum TPI value. Results for Sierran Mixed Conifer Mesic and Xeric shown here; results for other focal cover types available in Appendix~\ref{app:full-results}}.
\label{tab:tpi_cc_smcs}
\begin{tabular}{@{}lrrrrr@{}}
\toprule
\small \textbf{\begin{tabular}[c]{@{}l@{}}Cover \\ Name\end{tabular}} & \small \textbf{\begin{tabular}[c]{@{}l@{}}Minimum \\ TPI\end{tabular}} & \small \textbf{\begin{tabular}[c]{@{}l@{}}Maximum \\ TPI\end{tabular}} & \small \textbf{\begin{tabular}[c]{@{}l@{}}Average Canopy \\Cover at \\ Minimum TPI\end{tabular}} & \small \textbf{\begin{tabular}[c]{@{}l@{}}Average Canopy \\ Cover at \\ Maximum TPI\end{tabular}}  & \small \textbf{\begin{tabular}[c]{@{}l@{}}Percent \\ Change in \\ Canopy \\ Cover\end{tabular}} \\ \midrule
\textsc{smc\_m   }    & -300                 & 300                  & 55.5       & 50.4              & -9.3      \\
\textsc{smc\_x   }    & -300                 & 300                  & 27.6       & 21.9              & -20.5     \\ \bottomrule
\end{tabular}
\end{table}
% !TEX root = master.tex
\section{Results}
\label{sec:hrvresults}
\subsection{Disturbance Regime}

\subsubsection*{Burned Area and Wildfire Frequency}\todo{say burned/fire or disturbed??}

% 174830 eligible hectares
% 181553 hectares in core
% check math using Wildfire_darea_trajectory.csv
% redone 9/15

In this section the same results are presented in multiple ways; although this is obviously redundant, it is done with the intent of being as inclusive as possible with respect to the potential audience. The HRV analysis presented here forms the backbone of a formal report submitted to the U.S. Forest Service for use in understanding and planning future management actions. In my experience, variations in the training and background of individuals affects how they understand and interpret different values. For some, probabilistic results are the only ones that make sense or appear valid. For others, results expressed in real terms rather than model outputs are the most useful. In order to not privilege either perspective, I have included a range of information translated in different ways. My goal is to facilitate understanding of the results for both academic and professional audiences.

Approximately 96\% of the landscape was eligible for wildfire disturbance (all cover types except Barren and Water)\footnote{In this section I report values based on percent of eligible landscape. There are 181,550 hectares in the core study area, and 174,830 remain after excluding Barren and Water.}. As expected, the frequency and extent of simulated wildfires varied across timesteps. This variability is illustrated in Figure~\ref{fig:darea-a}. The same data is also represented as a histogram (Figure~\ref{fig:darea-b}) because it can be difficult to see the timesteps with small values. The histogram of the time series data to highlight the rarity of extremely expansive wildfire damage. It was more common during the simulation for fire to burn across 5--20\% of the landscape.

Also, given the specified rotation interval and percent mortality expected over time on this landscape, large proportions of the study area burned each (5-year) timestep. As detailed in Table \ref{tab:darea_atleast}, not only did fire occur in every timestep of the simulation, it also extended across over 10\% of the landscape during most timesteps. Although wildfire was common and covered a lot of area on a regular basis, very large events were relatively rare under the historical scenario. However, it is important to keep in mind that the simulation covers a very large spatial extent, and thus results in many more acres of fire burning per timestep than typically burn today \citep{calfire2012,usgs-fire-data2012}. In the median timestep (Table \ref{tab:darea}), about 14\% of the landscape burned, which may seem like a low number, but actually translates into a 24,500 hectares of burned land, 6,600 hectares of which burned at high mortality (though not necessarily contiguously). High mortality fires do include the burning of Early Development vegetation, including chaparral, when it resets the successional process. Further details on the percent of the full landscape that burned, and the breakdown of low versus high mortality outcomes, are included in Tables \ref{tab:darea_atleast} and \ref{tab:darea}.

% created 9/17
\begin{table}[!htbp]
\footnotesize
\centering
\caption{Summary of disturbed area in terms of proportion of the landscape burned during the simulation (after the equilibration period). For each benchmark proportion (1,10, 25, or 50+) of the landscape that burns, I provide a few representations of the data intended to support different ways of considering the results, that I hope will serve both modelers' and managers' perspectives. First, I counted the number of timesteps during the simulation when that extent burned at either high or low mortality ($n$). Then I calculated the proportion of timesteps that represents ($p = n/500$). The inverse of this is the interval in timesteps, i.e., approximately every 4 timesteps, at least 25\% of the landscape burned. ($t = 1/p$). Finally, I converted the interval in timesteps to the interval in years ($y = t * 5$).}
\label{tab:darea_atleast}
\begin{tabular}{@{}lllll@{}}
\toprule
\textbf{Proportion of Landscape Burned} & \textbf{1\%+}     & \textbf{10\%+}    & \textbf{25\%+}    & \textbf{50\%+} \\ \midrule
Number of timesteps ($n$)        & 459              & 313              & 115              & 13            \\
Proportion of timesteps ($p = n/500$)    & 1.00             & 0.68             & 0.25             & 0.03          \\
Interval (timesteps) ($t = 1/p$)      & 1.00             & 1.47             & 4.01             & 35.46         \\
Interval (years)    ($y = t * 5$)       & 5.02             & 7.36             & 20.04            & 177.31        \\ \bottomrule
\end{tabular}
\end{table}

% redone 9/16
\begin{table}[!htbp]
\footnotesize
\centering
\caption{Summary statistics for wildfire frequency by area disturbed during the simulation. Values are expressed as percentage and areal extent (in hectares) of the landscape eligible for disturbance that was actually burned.}
\label{tab:darea}
\begin{tabular}{@{}llll@{}}
\toprule
\textbf{\begin{tabular}[c]{@{}l@{}}Summary Statistic \\ (burned area/timestep)\end{tabular}}    & \textbf{Low Mortality}   & \textbf{High Mortality}    & \textbf{Any Mortality}   \\
\midrule                      %Low                      %high                 %any
$5^{\text{th}}$ percentile         &   2.72 (4,763)        & 0.71 (1,244)     &    3.54 (6,184)         \\
$50^{\text{th}}$ percentile        &   10.47 (18,300)      & 3.75 (6,563)     &    14.04 (24,544)         \\
$95^{\text{th}}$ percentile        &   31.29 (54,703)      & 21.43 (37,461)   &    45.88 (80,209)          \\
   Mean                            &   13.20 (23,079)      & 4.87 (8,512)     &    18.07 (31,592)         \\
\textbf{Fire Rotation}            & \textbf{38}       & \textbf{103}       & \textbf{28}   \\ 
\bottomrule
\end{tabular}
\end{table}

%redone 9/13


\begin{figure}[!htbp]
  \centering
  \subfloat[][]{
    \centering
    \includegraphics[width=0.5\textwidth]{/Users/mmallek/Documents/Thesis/Plots/darea/hrv_all.png} 
    \label{fig:darea-a}
    }%
  \subfloat[][]{
    \includegraphics[width=0.5\textwidth]{/Users/mmallek/Documents/Thesis/Plots/darea/hrv_newhist_all.png}
    \label{fig:darea-b}
    }
  \caption{(a) Disturbance trajectory for wildfire during the simulation. The first timestep is 40 because I excluded earlier timesteps as equilibration. Red bars represent high mortality fire, while green bars represent low mortality fire and are stacked on top of high mortality. (b) Histogram of the percent of the landscape burned per timestep.} 
  \label{fig:darea}
\end{figure}
\clearpage

While the tables and figures above represent the data aspatially, it can also be helpful to look at individual timesteps during the simulation. Because \textsc{RMLands} is spatially explicit in its wildfire generation processes, the burned areas generated within the model look like fire perimeter maps created following actual wildfire events, and occur in the areas on the landscape anticipated to be the most susceptible to wildfire. The extent of burned area observed during the simulation is far above what has been observed in recent decades \citep{calfire2012,usgs-fire-data2012}. Beyond saying it is much greater, however, it can be difficult to imagine and conceptualize what fire covering, for example, 50\% of the study area looks like. To facilitate understanding of the statistical outcomes of this work, I created maps displaying the mortality outcomes from wildfires at four key timesteps: the minimum, maximum, median, and mean area burned (because the exact mean area burned value did not correspond to any timestep, I display the one with the closest value for the ``any mortality'' category) in Figures~\ref{fig:darea_min_map}--\ref{fig:darea_mean_map}. During timesteps when a large amount of fire is recorded, multiple individual fires can ``run together'' on the landscape. Each timestep in the model represents five years, and the model does not differentiate between individual fire seasons below this level. Thus the included individual fire map (right-hand figures) offers a visualization of fires over the short term.

\newpage

% background color 24, 15, 41, 0
\begin{figure}[!htbp]
  \centering
  \subfloat[][]{
    \centering
    \includegraphics[width=0.5\textwidth]{/Users/mmallek/Documents/Thesis/maps/hrv-wfmort-5th.pdf}
    \label{fig:darea_min}
  }%
  \subfloat[][]{
    \includegraphics[width=0.5\textwidth]{/Users/mmallek/Documents/Thesis/maps/hrv-distid-5th.pdf}
    \label{fig:distid_min}
  }
  \caption{Maps of area burned during the timestep in the \textbf{$5^{\text{th}}$ percentile for area burned (3.54\%)} during the simulation. (a) Map by mortality level. Red indicates high mortality fire, while orange indicates low mortality fire. (b) Map showing each individual fire in a different color.}
  \label{fig:darea_min_map}
\end{figure}

\begin{figure}[!htbp]
  \centering
  \subfloat[][]{
    \centering
    \includegraphics[width=0.5\textwidth]{/Users/mmallek/Documents/Thesis/maps/hrv-wfmort-95th.pdf}
    \label{fig:darea_max}
  }%
  \subfloat[][]{
    \includegraphics[width=0.5\textwidth]{/Users/mmallek/Documents/Thesis/maps/hrv-distid-95th.pdf}
    \label{fig:distid_max}
  }
  \caption{Maps of area burned during the timestep with the \textbf{$95^{\text{th}}$ percentile for area burned (45.88\%)} during the simulation. (a) Map by mortality level. Red indicates high mortality fire, while orange indicates low mortality fire. (b) Map showing each individual fire in a different color.}
  \label{fig:darea_max_map}
\end{figure}

\begin{figure}[!htbp]
  \centering
  \subfloat[][]{
    \centering
    \includegraphics[width=0.5\textwidth]{/Users/mmallek/Documents/Thesis/maps/hrv-wfmort-median.pdf}
    \label{fig:darea_median}
  }%
  \subfloat[][]{
    \includegraphics[width=0.5\textwidth]{/Users/mmallek/Documents/Thesis/maps/hrv-distid-median.pdf}
    \label{fig:distid_median}
  }
  \caption{Maps of area burned during the timestep with the \textbf{median total area burned (14.04\%)} during the simulation. (a) Map by mortality level. Red indicates high mortality fire, while orange indicates low mortality fire. (b) Map showing each individual fire in a different color.}
  \label{fig:darea_median_map}
\end{figure}

\begin{figure}[!htbp]
  \centering
  \subfloat[][]{
    \centering
    \includegraphics[width=0.5\textwidth]{/Users/mmallek/Documents/Thesis/maps/hrv-wfmort-mean.pdf}
    \label{fig:darea_mean}
  }%
  \subfloat[][]{
    \includegraphics[width=0.5\textwidth]{/Users/mmallek/Documents/Thesis/maps/hrv-distid-mean.pdf}
    \label{fig:distid_mean}
  }
  \caption{Maps of area burned during the timestep with the \textbf{mean total area burned (18.07\%)} during the simulation. (a) Map by mortality level. Red indicates high mortality fire, while orange indicates low mortality fire. (b) Map showing each individual fire in a different color.}
  \label{fig:darea_mean_map}
\end{figure}

\clearpage

%%%%%%%%%%%%%%%%%%%%%%%%%%%%%%%%%%%%%%%%%%%%%%%%%%%%%%%%%%%%%%%%%%%%%%%%%%%%%%%%%%%%%%%%%%%%%%%%%%%%%%%%%%%%%%%%%%%%%%%%%%%%%%%%%%%%

\paragraph*{Sierran Mixed Conifer - Mesic}
Sierran Mixed Conifer - Mesic (\textsc{smc\_m}) is the dominant cover type within the core study area, encompassing 57,853 ha and comprising roughly 32\% of the study area. Wildfire was prevalent in this cover type. I again present figures and tables that incorporate some redundancy in order to facilitate understanding by a broad audience, as described in the beginning of this section (Figure \ref{fig:darea_smcm}). I summarize the disturbance regime in Tables~\ref{tab:darea_smcm} and \ref{tab:darea_atleast_smcm}. The frequency and extent of burned area is similar to that for the landscape as a whole.

% plots redone
\begin{figure}[!htbp]
  \centering
  \subfloat[][]{
    \centering
    \includegraphics[width=0.5\textwidth]{/Users/mmallek/Documents/Thesis/Plots/darea/hrv_smcm.png}
    }%
  \subfloat[][]{
    \includegraphics[width=0.5\textwidth]{/Users/mmallek/Documents/Thesis/Plots/darea/hrv_newhist_smcm.png}
    }
  \caption{\small (a) Disturbance trajectory for Sierran Mixed Conifer - Mesic. High mortality fire in red; low mortality fire in green. (b) Histogram of the percent of the landscape burned per timestep.} 
  \label{fig:darea_smcm}
\end{figure}

% updated 2015-09-21
\begin{table}[!htbp]
\footnotesize
\centering
\caption{Disturbed area summary statistics for Sierran Mixed Conifer - Mesic (\textsc{smc\_m}). Proportions shown are relative to the total area of \textsc{smc\_m}.}
\label{tab:darea_smcm}
\begin{tabular}{@{}lrrr@{}}
\toprule
\textbf{\begin{tabular}[c]{@{}l@{}}Summary Statistic \\ (burned \textsc{scm\_m}/timestep)\end{tabular}} & \textbf{Low Mortality} & \textbf{High Mortality} & \textbf{Any Mortality} \\ \midrule
$5^{\text{th}}$ percentile        & 2.60  & 0.47  & 3.17  \\
$50^{\text{th}}$ percentile       & 11.45 & 3.35  & 14.89 \\
$95^{\text{th}}$ percentile       & 34.17 & 11.57 & 45.27 \\
Mean                              & 14.42 & 4.42  & 18.83 \\
\textbf{Fire Rotation}            & \textbf{35}       & \textbf{113}       & \textbf{27}   \\ \bottomrule
\end{tabular}
\end{table}


\begin{table}[!htbp]
\footnotesize
\centering
\caption{Summary of disturbed area in terms of proportion of the amount of Sierran Mixed Conifer - Mesic (\textsc{smc\_m}). See Table~\ref{tab:darea_atleast} caption for details.}
\label{tab:darea_atleast_smcm}
\begin{tabular}{@{}lllll@{}}
\toprule
\textbf{Proportion of SMC\_M Burned} & \textbf{1\%+}     & \textbf{10\%+}    & \textbf{25\%+}    & \textbf{50\%+} \\ \midrule
Number of timesteps     & 458          & 311           & 126           & 15            \\
Proportion of timesteps & 0.99         & 0.67          & 0.27          & 0.03          \\
Interval (timesteps)    & 1.01         & 1.48          & 3.66          & 30.73         \\
Interval (years)        & 5.03         & 7.41          & 18.29         & 153.67       \\ \bottomrule
\end{tabular}
\end{table}

%\clearpage
%%%%%%%%%%%%%%%%%%%%%%%%%%%%%%%%%%%%%%%%%%%%%%%%%%%%%%%%%%%%%%%%%%%%%%%%%%%%%%%%%%%%%%%%%%%%%%%%%%%%%%%%%%%%%%%%%%%%%%%%%%%%%%%%%%%%


\paragraph*{Sierran Mixed Conifer - Xeric}
Sierran Mixed Conifer - Xeric (\textsc{smc\_x}) is the second most dominant cover type within the core study area, encompassing 52,198 ha and comprising roughly 29\% of the study area. Wildfire was prevalent in this cover type. I again present figures and tables that incorporate some redundancy in order to facilitate understanding by a broad audience, as described in the beginning of this section (Figure \ref{fig:darea_smcx}). I summarize the disturbance regime in Tables \ref{tab:darea_smcx} and \ref{tab:darea_atleast_smcx}. Both low and high mortality fires were more common on the xeric mixed conifer forests than in mesic mixed conifer forests or the study area as a whole. High mortality wildfire on xeric mixed conifer forests extended over a larger mean and median proportion compared to the overall landscape, although the $95^{\text{th}}$ percentile for high mortality fire in xeric mixed conifer forests was 3.2\% less.

% updated 2015-09
\begin{figure}[!hbp]
  \centering
  \subfloat[][]{
    \centering
    \includegraphics[width=0.5\textwidth]{/Users/mmallek/Documents/Thesis/Plots/darea/hrv_smcx.png}
    }%
  \subfloat[][]{
    \includegraphics[width=0.5\textwidth]{/Users/mmallek/Documents/Thesis/Plots/darea/hrv_newhist_smcx.png}
    }
  \caption{\small (a) Disturbance trajectory for Sierran Mixed Conifer - Xeric. High mortality fire in red; low mortality fire in green. (b) Histogram of the percent of the landscape burned per timestep.} 
  \label{fig:darea_smcx}
\end{figure}

% updated 2015-09-21
\begin{table}[!htbp]
\footnotesize
\centering
\caption{Disturbed area summary statistics for Sierran Mixed Conifer - Xeric (\textsc{smc\_x}). Proportions shown are relative to the total area of SMC\_X.}
\label{tab:darea_smcx}
\begin{tabular}{@{}llll@{}}
\toprule
\textbf{\begin{tabular}[c]{@{}l@{}}Summary Statistic \\ (disturbed SMC\_X/timestep)\end{tabular}} & \textbf{Low Mortality} & \textbf{High Mortality} & \textbf{Any Mortality} \\ \midrule
$5^{\text{th}}$ percentile    & 3.30  & 1.00  & 4.50  \\
$50^{\text{th}}$ percentile   & 11.92 & 5.17  & 17.55 \\
$95^{\text{th}}$ percentile   & 36.02 & 18.20 & 54.28 \\
Mean                          & 14.88 & 6.95  & 21.83 \\
\textbf{Fire Rotation}        & 34       & 72        & 23 \\  \bottomrule
\end{tabular}
\end{table}


\begin{table}[!htbp]
\footnotesize
\centering
\caption{Summary of disturbed area in terms of proportion of the amount of Sierran Mixed Conifer - Xeric (\textsc{smc\_x})burned during the simulation. See Table~\ref{tab:darea_atleast} caption for details.}
\label{tab:darea_atleast_smcx}
\begin{tabular}{@{}lllll@{}}
\toprule
\textbf{Proportion of SMC\_X Burned} & \textbf{1\%+}     & \textbf{10\%+}    & \textbf{25\%+}    & \textbf{50\%+} \\ \midrule
Number of timesteps     & 461          & 347           & 148           & 27            \\
Proportion of timesteps & 1.00         & 0.75          & 0.32          & 0.06          \\
Interval (timesteps)    & 1.00         & 1.33          & 3.11          & 17.07         \\
Interval (years)        & 5.00         & 6.64          & 15.57         & 85.37         \\ \bottomrule
\end{tabular}
\end{table}


%%%%%%%%%%%%%%%%%%%%%%%%%%%%%%%%%%%%%%%%%%%%%%%%%%%%%%%%%%%%%%%%%%%%%%%%%%%%%%%%%%%%%%%%%%%%%%%%%%%%%%%%%%%%%%%%%%%%%%%%%%%%%%%%%%%%
%%%%%%%%%%%%%%%%%%%%%%%%%%%%%%%%%%%%%%%%%%%%%%%%%%%%%%%%%%%%%%%%%%%%%%%%%%%%%%%%%%%%%%%%%%%%%%%%%%%%%%%%%%%%%%%%%%%%%%%%%%%%%%%%%%%%

\subsubsection*{Climate Effect} 
% fire size sentence is discussion
Climate has a positive relationship with disturbed area (Figure \ref{fig:climate_darea}). A regression line is plotted, but note the heteroskedastic variance about the mean. The relationship is weakly positive, in that as climate shifts from wet to drought, the disturbed area increases. The climate parameter is defined such that 1 is the average value over the historical period. During wetter-than-average years, less area was disturbed. For example, no more than 20\% of the landscape burned in any of the timesteps during which the climate parameter was below 0.63. However, over 50\% of the landscape burned in several timesteps when the climate parameter was less than 1 (wet periods). Figure \ref{fig:compare_clim_darea} illustrates the climate parameter values and disturbed area proportion of the landscape for a subset of timesteps during the simulation to illustrate that in some years, a high climate parameter occurs with a higher disturbed area percentage, but in other years a low climate parameter occurs with a higher disturbed area percentage that in mesic mixed conifer forests. I concluded that while a correlation certainly exists between climate and disturbed area, it is not associated with a firm ceiling or floor.

\begin{figure}[!htbp]
  \centering
    \includegraphics[width=0.5\textwidth]{/Users/mmallek/Documents/Thesis/Plots/darea/hrv_climdarea.png}
  \caption{Plot of the climate parameter and disturbed area value for each timestep of the simulation (excluding the equilibration period). A linear model has been fit to the data and is shown as a blue line; the grey shaded area represents the 95\% confidence interval around the mean.}
  \label{fig:climate_darea}
\end{figure}


% updated 9/13
\begin{figure}[!htbp]
\centering
\includegraphics[width=0.5\textwidth]{/Users/mmallek/Documents/Thesis/Plots/darea/climate_darea_vert.png}
\caption{Climate parameter and proportion of eligible landscape disturbed by wildfire for timesteps 250 to 310 of the simulation, illustrating the wide variability in both climate parameter values and disturbed area per timestep. Purple lines are intended to aid in visualization of the climate paramter value and proportion of landscape burned during a particular timestep.}
\label{fig:compare_clim_darea}
\end{figure}

\clearpage




\newpage
\subsubsection*{Rotation Period} 
As described in Chapter \ref{sec:hrvmethods}, I calibrated the model by adjusting seral stage-specific susceptibility values until the nine cover types with more than 1000 ha extent across the study area were within about 10\% of their target fire rotation. I present here the results for the nine focal cover types. Full results for all cover types in the study area are presented in Table~\ref{tab:all-rotations}. 

% updated 2015-09
\begin{table}[!htbp]
\footnotesize
\centering
\caption{Fire rotation for Sierran Mixed Conifer Mesic and Xeric.}
\begin{tabular}{@{}lrrr@{}}
\toprule
\begin{tabular}[c]{@{}l@{}}Land Cover \\ Type\end{tabular}     & \begin{tabular}[c]{@{}l@{}}Low Mortality \\ Fire Rotation\end{tabular} & \begin{tabular}[c]{@{}l@{}}High Mortality \\ Fire Rotation\end{tabular} & \begin{tabular}[c]{@{}l@{}}All Fires \\ Rotation\end{tabular} \\ \midrule
\textsc{smc\_m   }             & 35                          & 113                          & 27                 \\
\textsc{smc\_x   }             & 34                          & 72                           & 23                 \\
\emph{Full Landscape    }      &\emph{ 38}                   & \emph{103}                   & \emph{28  }        \\ \bottomrule
\end{tabular}
\end{table}


\subsubsection*{Point-specific Return Interval}
Visualizing the point (cell)-specific fire rotation (return interval) calls attention to the variability in wildfire recurrence across the study area. I use barplots to show the spread and underlying values in the distribution of cell-specific fire rotations, and maps to demonstrate the spatial variability in this metric across the study area. Overall, the cell-specific fire rotation for an individual cell ranged from 17 years to \textgreater 2500 years (cells that never burned during the simulation) for both classes of wildfire mortality (Figure \ref{fig:preturn}). The grand mean return interval across all cover types was 42 years for low mortality fire, 111 year for high mortality fire, and 29 years for any fire. The cell-specific fire rotation plots and maps specific to Sierran Mixed Conifer Mesic and Xeric follow (Figures~\ref{fig:preturn_smcm} and \ref{fig:preturn_smcx}). Under this wildfire regime, the cell-specific fire rotation for an individual cell between fires (of any mortality level) for both of these mixed conifer forest types varied widely from about 17 years to over 500 years, with grand means of 28 years (for mesic) and 23 years (for xeric) (Figures~\ref{fig:preturn_smcm} and \ref{fig:preturn_smcx}). Results for the other seven focal cover types are included in Appendix~\ref{sec:indiv_cov_results}. 



% first plot redone 9/13
% second plot not redone yet
\begin{figure}[!htbp]
  \centering
  \subfloat[][]{
    \centering
    \includegraphics[height=.4\textheight]{/Users/mmallek/Documents/Thesis/Plots/preturn/not-called-preturn/hrv-total.png}
    \label{fig:preturn_plot}
  }%
  \qquad
  \subfloat[][]{
    \includegraphics[height=.4\textheight]{/Users/mmallek/Documents/Thesis/Plots/preturn-maps/fri_all.png}
    \label{fig:preturn_map}
  }
  \caption{(a) Distribution of point-specific fire rotations for the full landscape under study. The ``full landscape'' includes each cell in the raster with a cover type eligible to burn. The point-specific fire rotation is the average interval between fires over the length of the simulation, excluding the equilibration period. (b) Spatially-explicit depiction of these point-specific fire rotations across the landscape, for all cover types.}
  \label{fig:preturn}
\end{figure}

% first plot updated 9/13
\begin{figure}[!htbp]
  \centering
  \subfloat[][]{
    \centering
    \includegraphics[width=0.5\textwidth]{/Users/mmallek/Documents/Thesis/Plots/preturn/not-called-preturn/hrv-smcm.png}
    }%
  \subfloat[][]{
    \includegraphics[width=0.5\textwidth]{/Users/mmallek/Documents/Thesis/Plots/preturn-maps/fri_smcm.png}
    }
  \caption{(a) Distribution of point-specific fire rotations for Sierran Mixed Conifer - Mesic. The point-specific fire rotation is the average interval between fires over the length of the simulation, excluding the equilibration period. (b) Spatially-explicit depiction of these point-specific fire rotations across the landscape. Cover types other than Sierran Mixed Conifer - Mesic are partially obscured in grey.}
\label{fig:preturn_smcm}
\end{figure}

%first plot redone 9/13
\begin{figure}[!htbp]
  \centering
  \subfloat[][]{
    \centering
    \includegraphics[width=0.5\textwidth]{/Users/mmallek/Documents/Thesis/Plots/preturn/not-called-preturn/hrv-smcx.png}
    }%
  \subfloat[][]{
    \includegraphics[width=0.5\textwidth]{/Users/mmallek/Documents/Thesis/Plots/preturn-maps/fri_smcx.png}
    }
  \caption{(a) Distribution of point-specific fire rotations for Sierran Mixed Conifer - Xeric. The point-specific fire rotation is the average interval between fires over the length of the simulation, excluding the equilibration period. (b) Spatially-explicit depiction of these point-specific fire rotations across the landscape. Cover types other than Sierran Mixed Conifer - Xeric are partially obscured in grey.}
\label{fig:preturn_smcx}
\end{figure}

\clearpage


%%%%%%%%%%%%%%%%%%%%%%%%%%%%%%%%
%%%%%%%%%%%%%%%%%%%%%%%%%%%%%%%%
%%%%%%%%%%%%%%%%%%%%%%%%%%%%%%%%

%\pagebreak[4]
\subsection{Vegetation Response}
\label{subsec:HRVvegresponse}


\subsubsection*{Landscape Composition} 

% fixed plots - equilibration line is hard coded in. ocfwu calibration changed; now seems okay by ts 40. potentially could even have cut off equilibration at ts 20 but it's arbitrary. good to keep in mind for future stuff though.
The seral stage distribution for each cover type varied over time, but did appear to be in dynamic equilibrium. Evidence of both high mortality fire, which triggers a transition to the Early Development seral stage for all cover types, and low mortality fire, which can thin a stand and cause a transition to a more open canopy seral stage (within the same development level), are visible in examining the output grids. Figure \ref{fig:covcondmaps} illustrates these changes for a sequence of four timesteps during the simulation. The seral stage dynamics and current seral stage distribution plots specific to Sierran Mixed Conifer - Mesic and Sierran Mixed Conifer - Xeric follow (Figures~\ref{fig:hrv-covcond_smcm} and \ref{fig:hrv-covcond_smcx}). I compare the current landscape's seral stage distribution to the simulated distribution and asess the current landscape's departure frmo the HRV in Tables~\ref{tab:covcond_smcm} and \ref{tab:covcond_smcx}. Plots and tabular results for the other seven focal types are included in Appendix~\ref{sec:indiv_cov_results}.

% new plots 2015-09-18
\begin{figure}[!htbp]
  \centering
  \subfloat[][]{
    \includegraphics[width=0.5\textwidth]{/Users/mmallek/Documents/Thesis/maps/hrv-covcondseq-5.pdf}
  }%
  \subfloat[][]{
    \includegraphics[width=0.5\textwidth]{/Users/mmallek/Documents/Thesis/maps/hrv-covcondseq-6.pdf}
  }\\%
  \subfloat[][]{
    \includegraphics[width=0.5\textwidth]{/Users/mmallek/Documents/Thesis/maps/hrv-covcondseq-7.pdf}
    }
  \subfloat[][]{
    \centering
    \includegraphics[width=0.5\textwidth]{/Users/mmallek/Documents/Thesis/maps/hrv-covcondseq-8.pdf}
  }%
  \caption{A sequence of four timesteps during the middle of the simulation, showing changes in seral stages over time. Here I highlight the dominant cover type, Sierran Mixed Conifer - Mesic, and its classes, in order to illustrate the dynamics that play out over many years. (a) Timestep 1 (b) Timestep 2 (c) Timestep 3 (d) Timestep 4. Patches in shades of brown and tan belong to other cover types.}
  \label{fig:covcondmaps}
\end{figure}


\clearpage

\paragraph*{Sierran Mixed Conifer - Mesic}

% hrv plot updated 2015-09
\begin{figure}[!htbp]
  \centering
  \subfloat[][]{
    \centering
    \includegraphics[width=0.6\textwidth]{/Users/mmallek/Documents/Thesis/Plots/covcond-dynamics/notcalledcovcond/SMCM.pdf}
  }%
  \subfloat[][]{
    \includegraphics[height=2.65in]{/Users/mmallek/Tahoe/R/Rplots/November2014/covcond_current_smcm.png}
  } \\
  \subfloat[][]{
    \includegraphics[width=\textwidth]{/Users/mmallek/Documents/Thesis/Plots/covcond-bycover/SMCM-HRV-boxplots-.png}
  }
  \caption{(a) Cover type-Seral stage dynamics for Sierran Mixed Conifer - Mesic. The black vertical line at 40 timesteps marks the end of the equilibration period used in this study. (b) Current seral stage distribution for Sierran Mixed Conifer - Mesic. (c) Boxplots showing the range of variability for each seral stage over the course of the simulation, excluding the equilibration period. Boxplots were modified so that whiskers extend from the $5^{\text{th}} - 95^{\text{th}}$ percentiles of the observed results. Thick black bars in line with the boxplots denote the current proportion of mesic mixed conifer forests in a given seral stage.} 
  \label{fig:hrv-covcond_smcm}
\end{figure}

The distribution of area among stand seral stages within mesic mixed conifer forests fluctuated over time, but appeared to be in dynamic equilibrium (Figure~\ref{fig:hrv-covcond_smcm}). For example, the percentage of mesic mixed conifer forests in the Early Development seral stage varied from approximately 8\%--25\%, reflecting the dynamic nature of this cover type. This seral stage is currently within the simulated HRV (48$^{\text{th}}$ percentile). Mid--Closed was typically the most extensive seral stage (22\%--37\%), but most of the seral stages were common throughout the simulation. % for later note dominance of closed canopies
The current seral-stage distribution was never observed under the simulated HRV (Table~\ref{tab:covcond_smcm}). The most notable departures were an increase in Mid--Closed and Late--Open extent, and a decrease in Mid--Moderate extent during the simulated HRV. These seral stages are currently all outside of the simulated HRV. In fact, Late--Open is rare on the current landscape (3.6\%), but present in similar proportions to  other classes during the HRV. 


%\begin{landscape}
% table updated 2015-09
\begin{table}[!htbp]
\footnotesize
\centering
\caption{Range of variability in landscape structure, illustrating the cover type-seral stage class dynamics for Sierran Mixed Conifer - Mesic. Included are the $5^{\text{th}}$ percentile, $25^{\text{th}}$ percentile, $50^{\text{th}}$ percentile, $75^{\text{th}}$ percentile, and $95^{\text{th}}$ percentiles of the distribution, as well as the current landscape proportion, the curent percentile range of variability (\%RV) for that proportion, and the departure classification. For seral stage abbreviations, see Table \ref{condtable}.}
\label{tab:covcond_smcm}
\begin{tabular}{@{}lrrrrr|rrr@{}}
\toprule
\textbf{\begin{tabular}[c]{@{}l@{}}Seral \\ Stage\end{tabular}}  &  \textbf{$5^{\text{th}}$} &   \textbf{$25^{\text{th}}$} &   \textbf{$50^{\text{th}}$} &   \textbf{$75^{\text{th}}$} &   \textbf{$95^{\text{th}}$}  &  \textbf{\begin{tabular}[c]{@{}l@{}}Current\\ \%cover\end{tabular}} & \textbf{\begin{tabular}[c]{@{}l@{}}Current\\ \%RV\end{tabular}} & \textbf{\begin{tabular}[c]{@{}l@{}}Departure\end{tabular}} \\ \midrule
\textsc{early\_all}        &   7.75        &  12.34   &  15.11     &  18.68   &  24.74     &  14.98    &  48    &  none      \\
\textsc{mid\_cl   }        &   21.52       &  26.15   &  29.69     &  32.58   &  37.01     &  9.74     &  0     &  complete     \\
\textsc{mid\_mod  }        &   6.8         &  7.98    &  9.03      &  10.3    &  12.63     &  17.97    &  100   &  complete     \\
\textsc{mid\_op   }        &   6.68        &  9.2     &  11.21     &  13.08   &  16.15     &  16.29    &  96    &  complete     \\
\textsc{late\_cl  }        &   5.31        &  9.54    &  12.87     &  17.2    &  22.91     &  23.23    &  97    &  complete      \\
\textsc{late\_mod }        &   8.56        &  10.32   &  11.24     &  12.56   &  14.41     &  14.18    &  95    &  complete      \\
\textsc{late\_op  }        &   4.96        &  7.39    &  9.26      &  12.12   &  14.95     &  3.6      &  1     &  complete      \\
\bottomrule
\end{tabular}
\end{table}





%\end{landscape}
%\clearpage
%%%%%%%%%%%%%%%%%%%%%%%%%%%%%%%%%%%%%%%%%%%%%%%%%%%%%%%%%%%%%%%%%%%%%%%%%%%%%%%%%%%%%%%%%%%%%%%%
\paragraph*{Sierran Mixed Conifer - Xeric}

% plot updated 2015-09
\begin{figure}[!htbp]
  \centering
  \subfloat[][]{
    \centering
    \includegraphics[width=0.6\textwidth]{/Users/mmallek/Documents/Thesis/Plots/covcond-dynamics/notcalledcovcond/SMCX.pdf}
  }%
  \subfloat[][]{
    \includegraphics[height=2.65in]{/Users/mmallek/Tahoe/R/Rplots/November2014/covcond_current_smcx.png}
  } \\
  \subfloat[][]{
    \includegraphics[width=\textwidth]{/Users/mmallek/Documents/Thesis/Plots/covcond-bycover/SMCX-HRV-boxplots-.png}
  }
  \caption{(a) Cover type-Seral stage dynamics for Sierran Mixed Conifer - Xeric. The black vertical line at 40 timesteps marks the end of the equilibration period used in this study. (b) Current seral stage distribution for Sierran Mixed Conifer - Xeric. (c) Boxplots showing the range of variability for each seral stage over the course of the simulation, excluding the equilibration period. Boxplots were modified so that whiskers extend from the $5^{\text{th}} - 95^{\text{th}}$ percentiles of the observed results. Thick black bars in line with the boxplots denote the current proportion of mesic mixed conifer forests in a given seral stage.}  
  \label{fig:hrv-covcond_smcx}
\end{figure}

The distribution of area among seral stages within xeric mixed conifer forests fluctuated over time, but appeared to be in dynamic equilibrium (Figure~\ref{fig:covcond_smcx}). For example, the percentage of xeric mixed conifer forests in the Early Development varied from approximately25\% to 43\%, reflecting the dynamic nature of this cover type (Table~\ref{tab:covcond_smcx}). During the simulation, Early Development (which includes post-fire chaparral fields) and Mid--Open seral stages dominated, in contrast to the current distribution, which is somewhat even across classes (although Late--Open is currently quite rare).
%
The current seral stage distribution was never observed under the simulated HRV, and all of the seral stages were fully departed from the HRV (Table~\ref{tab:covcond_smcx}). The most dramatic departure was the increase in Early Development and Mid--Open during the simulated HRV compared to the current landscape (currently at 19\% and 11\%, respectively). I also observed a much lower proportion of xeric mixed conifer forest in Late--Closed during the simulation than in the current landscape. 

% table updated 2015-09
\begin{table}[!htbp]
\footnotesize
\centering
\caption{Range of variability in landscape structure, illustrating the cover type-seral stage dynamics for Sierran Mixed Conifer - Xeric. Included are the $5^{\text{th}}$ percentile, $25^{\text{th}}$ percentile, $50^{\text{th}}$ percentile, $75^{\text{th}}$ percentile, and $95^{\text{th}}$ percentiles of the distribution, as well as the current landscape proportion, the current percentile range of variability (\%RV) for that proportion, and the departure classification. For seral stage abbreviations, see Table \ref{condtable}.}
\label{tab:covcond_smcx}
\begin{tabular}{@{}lrrrrr|rrr@{}}
\toprule
\textbf{\begin{tabular}[c]{@{}l@{}}Seral \\ Stage\end{tabular}}  &  \textbf{$5^{\text{th}}$} &   \textbf{$25^{\text{th}}$} &   \textbf{$50^{\text{th}}$} &   \textbf{$75^{\text{th}}$} &   \textbf{$95^{\text{th}}$}  &  \textbf{\begin{tabular}[c]{@{}l@{}}Current\\ \%cover\end{tabular}} &   \textbf{\begin{tabular}[c]{@{}l@{}}Current\\ \%RV \end{tabular}} &   \textbf{\begin{tabular}[c]{@{}l@{}}Departure\end{tabular}} \\ \midrule
 \textsc{early\_all}      &  25.2          &  29.63    &  34.53    &  38.95    &  42.82     &  19.48       &   0      &  complete    \\
 \textsc{mid\_cl   }      &  0.02          &  0.06     &  0.13     &  0.36     &  1.07      &  11.96       &   100    &  complete      \\
 \textsc{mid\_mod  }      &  0.9           &  1.62     &  2.88     &  4.35     &  7.6       &  14.92       &   100    &  complete    \\
 \textsc{mid\_op   }      &  26.55         &  30.59    &  33.79    &  36.58    &  39.36     &  11.48       &   0      &  complete    \\
 \textsc{late\_cl  }      &  1.19          &  2.51     &  3.81     &  5.99     &  8.69      &  24.72       &   100    &  complete      \\
 \textsc{late\_mod }      &  5.83          &  7.49     &  9.16     &  10.71    &  13.03     &  13.31       &   97     &  complete     \\
 \textsc{late\_op  }      &  9.39          &  12.4     &  15       &  17.42    &  22.45     &  4.13        &   0      &   complete  \\ \bottomrule 
\end{tabular}
\end{table}

\clearpage


%%%%%%%%%%%%%%%%%%%%%%%%%%%%%%%%%%%%%%%%%%%%%%%%%%%%%%%%%%%%%%%%%%%%%%%%%%%%%%%%%%%%%%%%%%%%%%%%
%%%%%%%%%%%%%%%%%%%%%%%%%%%%%%%%%%%%%%%%%%%%%%%%%%%%%%%%%%%%%%%%%%%%%%%%%%%%%%%%%%%%%%%%%%%%%%%%

\subsubsection*{Landscape Configuration}
I summarized the structure and patterns in the landscape using a suite of statistical measures calculated using \textsc{Fragstats}. Table \ref{tab:fragland} shows the range of variability for the simulation period as well as the current value, the current percentile range of variability (\%RV) for that proportion, and the departure classification. I show here a subset of metrics most useful for understanding patch characteristics in the study area; complete results are included in Appendix~\ref{app:full-land-results}.  Appendix~\ref{app:metricdescriptions} contains a detailed description of each \textsc{Fragstats} metric calculated for this project. At the landscape-level, most computed metrics have values outside the HRV. 

In Figures~\ref{fig:fragland1} and \ref{fig:fragland2} I graphically display the results from Table \ref{tab:fragland}. For these six metrics, the current landscape is fully departed from the historical range of variability. The average patch size is larger, and the average patch shape more complex, than the current landscape. Patches during the HRV had more edge. During the HRV, the average patch contains more core area than in the current landscape. The landscape during the HRV is much more contagious than the current landscape. Values for Simpson's Evenness are near 1 during the HRV and in the present landscape, but the HRV values are well below the current conditions.

% plots updated 2015-09

\begin{figure}[!htbp]
  \centering
  \subfloat[][]{
    %\centering
    \includegraphics[width=0.5\textwidth]{/Users/mmallek/Documents/Thesis/Plots/fragland-hrv/ED1.png}
    \label{fig:fragland_ed}
  }%
  \subfloat[][]{
    %\centering
    \includegraphics[width=0.5\textwidth]{/Users/mmallek/Documents/Thesis/Plots/fragland-hrv/AREA_AM1.png}
    \label{fig:fragland_area}
  } \\
  \subfloat[][]{
    \includegraphics[width=0.5\textwidth]{/Users/mmallek/Documents/Thesis/Plots/fragland-hrv/SHAPE_AM1.png}
    \label{fig:fragland_shape}
  } 
  \subfloat[][]{
    \includegraphics[width=0.5\textwidth]{/Users/mmallek/Documents/Thesis/Plots/fragland-hrv/CORE_AM1.png}
    \label{fig:fragland_core}
    }
\caption{Landscape \textsc{Fragstats} Metrics. (a) Edge Density, a measure of patch perimeter complexity, (b) Area-weighted Mean Patch Area, a measure of patch size (c) Area-weighted Mean Shape, a measure of patch shape complexity (d) Area-weighted Mean Core Area, a measure of interior habitat available at the patch level. The red line indicates the current condition, the dotted lines indicate the 5th and 95th percentiles of the simulated data, the dashed line indicates the 50th percentile of the simulated data, and the blue line indicates the value for that metric at each timestep of the simulation.}
\label{fig:fragland1}
\end{figure}

\begin{figure}[!htbp]
  \centering
  \subfloat[][]{
    \includegraphics[width=0.5\textwidth]{/Users/mmallek/Documents/Thesis/Plots/fragland-hrv/CONTAG1.png}
    \label{fig:fragland_contag}
  } 
  \subfloat[][]{
    \includegraphics[width=0.5\textwidth]{/Users/mmallek/Documents/Thesis/Plots/fragland-hrv/SIEI1.png}
    \label{fig:fragland_siei}
  } 
\caption{Landscape \textsc{Fragstats} Metrics. (a) Contagion, a measure of patch dispersion and interspersion (b) Simpson's Evenness Index, a measure of diversity, or evenness, across all landscape patches. The red line indicates the current condition, the dotted lines indicate the 5th and 95th percentiles of the simulated data, the dashed line indicates the 50th percentile of the simulated data, and the blue line indicates the value for that metric at each timestep of the simulation.}
\label{fig:fragland2}
\end{figure}

%\clearpage

% repaired table 9/13
%\begin{landscape}
\begin{table}[!htbp]
\footnotesize
\centering
\caption{Range of variability during the simulation for a selected suite of landscape configuration metrics calculated using \textsc{Fragstats}. See Appendix~\ref{app:metricdescriptions} for descriptions. Abbreviations are: 
\textsc{ed} = edge density;
\textsc{area\_am} = area-weighted mean patch size; 
\textsc{shape\_am} = area-weighted mean patch shape index; 
\textsc{core\_am} = area-weighted mean patch core area; 
\textsc{contag} = contagion; 
\textsc{siei} = Simpson's evenness index.
Included are the $5^{\text{th}}$ percentile, $25^{\text{th}}$ percentile, $50^{\text{th}}$ percentile, $75^{\text{th}}$ percentile, and $95^{\text{th}}$ percentiles of the distribution, as well as the current landscape proportion, the current percentile range of variability (\%RV) for that proportion, and the departure classification.
} 
\label{tab:fragland}
\begin{tabular}{@{}lrrrrr|rrr@{}}
\toprule
\textbf{\begin{tabular}[c]{@{}l@{}}Landscape\\ Metric\end{tabular}}  &   
\textbf{$5^{\text{th}}$ } &   
\textbf{$25^{\text{th}}$ } &   
\textbf{$50^{\text{th}}$ } &   
\textbf{$75^{\text{th}}$ } &   
\textbf{$95^{\text{th}}$ }  &  
\textbf{\begin{tabular}[c]{@{}l@{}}Current\\ Value\end{tabular}} &   
\textbf{\begin{tabular}[c]{@{}l@{}}Current\\ \%RV\end{tabular}} &   
\textbf{\begin{tabular}[c]{@{}l@{}}Departure\end{tabular}} \\ 
\midrule
\textsc{ed}         & 120.581         & 121.880           & 122.903          & 123.691          & 124.813          & 128.875     & 100     & complete  \\
\textsc{area\_am}   & 156.549         & 166.016          & 174.884          & 184.448          & 205.209          & 119.985     & 0       & complete \\
\textsc{shape\_am}  & 3.560            & 3.621            & 3.667            & 3.727            & 3.847            & 3.243       & 0       & complete \\
\textsc{core\_am}   & 135.146         & 141.964          & 149.582          & 157.587          & 169.545          & 106.710      & 0       & complete \\
\textsc{contag}     & 53.943          & 54.455           & 54.744           & 55.064           & 55.523           & 51.172      & 0       & complete \\
\textsc{siei}       & 0.946           & 0.949            & 0.951            & 0.953            & 0.956            & 0.971       & 100     & complete  \\
\bottomrule
\end{tabular}
\end{table}
%\end{landscape}




\clearpage

%%%%%%%%%%%%%%%%%%%%%%%%%%%%%%%%%%%%%%%%%%%%%%%%%%%%%%%%%%%%%%%%%%%%%%%%%%%%%%%%%%%%%%%%%%%%%%%%

\paragraph*{Class-level Results}

In addition to the landscape-level results, I also summarized structure and patterns at the cover type level using \textsc{Fragstats}. I show here a subset of metrics most useful to understanding patch characteristics at the cover type - seral stage level for the two most prevalent cover types, Sierran Mixed Conifer - Mesic and Sierran Mixed Conifer - Xeric. Again, I display boxplots depicting the range of variability for the simulation period as well as the current value. See Appendix~\ref{app:full-class-results} for full tabular results for the nine focal cover types.


\subparagraph*{Sierran Mixed Conifer - Mesic} %updated analysis 2015-09-20
The spatial configuration of stand conditions fluctuated markedly over time, although there was considerable variation in the magnitude of variability among configuration metrics (see Appendix~\ref{app:full-class-results}, Table~\ref{tab:fragclass_smcm}). Early and Mid development patches in this cover type tended to have wide ranges of variability in metric outcomes, and were larger, less fragmented, more geometrically complex, and had more core area during the HRV than during the current conditions (Figure~\ref{fig:fragclass_smcm}). Metric values for these seral stages tended to be completely or nearly outside the simulated HRV. 
% B liked my writing here
In contrast, the other seral stages all fall within the simulated HRV in terms of patch size and core area. Results for geometric complexity and fragmentation were less consistent across the other seral stages. While Late--Open stands were more geometrically complex during the HRV than on the current landscape, Mid--Moderate, Mid--Late, and Late--Moderate patches were all less geometrically complex. Late--Closed patches currently fall within the simulated HRV. Meanwhile, the open canopy seral stages are currently within the HRV in terms of fragmentation, while the Mid--Moderate, Late--Closed, and Late--Moderate stages are all currently outside the range of variability and more fragmented today than during the simulated HRV.  

\begin{figure}[!htbp]
  \centering
  \subfloat[][]{
    %\centering
    \includegraphics[width=0.5\textwidth]{/Users/mmallek/Documents/Thesis/Plots/fragclass-bymetrics/HRV/SMC_M-AREA_AM-boxplots.png}
    \label{fig:smcm_areaam}
  } 
  \subfloat[][]{
    %\centering
    \includegraphics[width=0.5\textwidth]{/Users/mmallek/Documents/Thesis/Plots/fragclass-bymetrics/HRV/SMC_M-CORE_AM-boxplots.png}
    \label{fig:smcm_coream}
  } \\
  \subfloat[][]{
    \includegraphics[width=0.5\textwidth]{/Users/mmallek/Documents/Thesis/Plots/fragclass-bymetrics/HRV/SMC_M-SHAPE_AM-boxplots.png}
    \label{fig:smcm_shapeam}
  } 
  \subfloat[][]{
    %\centering
    \includegraphics[width=0.5\textwidth]{/Users/mmallek/Documents/Thesis/Plots/fragclass-bymetrics/HRV/SMC_M-CLUMPY-boxplots.png}
    \label{fig:smcm_clumpy}
  }%
\caption{Fragstats class-level results for Sierran Mixed Conifer - Mesic. (a) area-weighted mean patch area. (b) area-weighted mean core area (c) area-weighted mean shape index (d) clumpiness. Boxplot whiskers extend to the $5^{\text{th}}$ and $95^{\text{th}}$ percentile of the observed distribution. The thick grey bar denotes the metric value on the current landscape.}
\label{fig:fragclass_smcm}
\end{figure}


%\clearpage
%%%%%%%%%%%%%%%%%%%%%%%%%%%%%%%%%%%%%%%%%%%%%%%%%%%%%%%%%%%%%%%%%%%%%%%%%%%%%%%%%%%%%%%%%%%%%%%%


\subparagraph*{Sierran Mixed Conifer - Xeric}
The spatial configuration of stand conditions fluctuated markedly over time as well, although there was considerable variation in the magnitude of variability among configuration metrics (see Appendix~\ref{app:full-class-results}, Table~\ref{tab:fragclass_smcx}). Early Development and Mid--Open had wide ranges of variability in patch and core area size, while Mid--Closed had a wide range of variability in geometric complexity and fragmentation. In contrast to the mesic mixed conifer forests, results in this cover type were consistent across different metrics. Mid--Closed, Mid--Moderate, and Late--Moderate stages currently fall within the simulated HRV in terms of area-weighted mean patch size and core area, as well as for the shape and clumpiness indices. However, the other stages were generally currently outside or nearly outside the simulated HRV. Early seral and open canopy stands are currently smaller, more fragmented, less geometrically complex, and have less core area than during the simulated HRV, while the opposite is true for Late--Closed patches (Figure~\ref{fig:fragclass_smcx}).

% figures updated 2015-09-20
\begin{figure}[!htbp]
  \centering
  \subfloat[][]{
    %\centering
    \includegraphics[width=0.5\textwidth]{/Users/mmallek/Documents/Thesis/Plots/fragclass-bymetrics/HRV/SMC_X-AREA_AM-boxplots.png}
    \label{fig:smcx_areaam}
  } 
  \subfloat[][]{
    %\centering
    \includegraphics[width=0.5\textwidth]{/Users/mmallek/Documents/Thesis/Plots/fragclass-bymetrics/HRV/SMC_X-CORE_AM-boxplots.png}
    \label{fig:smcx_coream}
  } \\
  \subfloat[][]{
    \includegraphics[width=0.5\textwidth]{/Users/mmallek/Documents/Thesis/Plots/fragclass-bymetrics/HRV/SMC_X-SHAPE_AM-boxplots.png}
    \label{fig:smcx_shapeam}
  } 
  \subfloat[][]{
    %\centering
    \includegraphics[width=0.5\textwidth]{/Users/mmallek/Documents/Thesis/Plots/fragclass-bymetrics/HRV/SMC_X-CLUMPY-boxplots.png}
    \label{fig:smcx_clumpy}
  }%
\caption{Fragstats class-level results for Sierran Mixed Conifer - Xeric. (a) area-weighted mean patch area. (b) area-weighted mean core area (c) area-weighted mean shape index (d) clumpiness. Boxplot whiskers extend to the $5^{\text{th}}$ and $95^{\text{th}}$ percentile of the observed distribution. The thick grey bar denotes the metric value on the current landscape.}
\label{fig:fragclass_smcx}
\end{figure}

\clearpage
%
% !TEX root = master.tex

\section{Discussion}
\label{sec:hrvdiscussion}

% becky said to move this from the results section, not sure how it'll fit in. Maybe as a limitation of the model or explanation of caveats?
In the sections below, we describe the simulated disturbance regime in terms of its spatial extent and distribution, frequency, and temporal variability, for the landscape as a whole. Variations among vegetation types are described below and in Appendix~\ref{app:covtype_analysis}. We also acknowledge that some results were used to evaluate whether the model was correctly calibrated; specifically, fire rotation values were used in model calibration, as described in Chapter~\ref{sec:hrvmethods}. These rotation values are also an outcome of the model, and are therefore reported here. While this may seem a bit circular, it was a necessary part of the process of simulating the historical range of variability.\todo{add management context} In addition to the disturbance regime, this chapter includes results for the seral stage dynamics and analysis of the landscape configuration metrics from \textsc{Fragstats}.

\section{Model evaluation and Sensitivity Analysis}
It would have been idea to complete model evaluation using a hindcasting strategy and a sensitivity analysis on the final parameter set. Unfortunately, at the time of this study, \textsc{RMLands} was not capable of simulating vegetation treatments. Moreover, we would need more detailed descriptions of the vegetation treatments carried out in the past than exist in order to truly test our model. A rigorous sensitivity analysis was not completed because of the very long run times and large amounts of data associated with using the model. However, we have a few observations about the relative sensitivity of some parameters based on our experience calibrating. First, the ignitiation parameter is quite sensitive; changing it by a few interval values changes all the outcomes. In comparison, the fire return index is relatively insensitive; we often modified it by over an order of magnitude in order to effect a small change in the rotation outcome. During model development we learned that the probability of high mortality fire at the seral stage level is relatively sensitive, which is logical because the conversion of forest to early seral conditions directly impacts most of the metrics by which we evaluate landscape structure and composition. Probabilities of high mortality fire at the seral stage level are extremely difficult to find in the literature; field research is needed to move the field from its current focus on overall cover type estimates to more refined estimates based on seral stage. Better information here would improve the model and our results.


%%%%%%%%%%%%%%%%%%%%%%%%%%%%%%%%%%%%%%%%%%%%%%%%%%%%%%%%%%%%%%%%%%%%%%%%%%%%%
%%%%%%%%%%%%%%%%%%%%%%%%%%%%%%%%%%%%%%%%%%%%%%%%%%%%%%%%%%%%%%%%%%%%%%%%%%%%%

\section{Overall Landscape Assessment}
% rewrote 2015-09-20
With respect to the condition class (seral stage) distribution, a few patterns emerged across the nine focal cover types. A numerical representation of these dynamics is available in Tables \ref{tab:covcond1} to \ref{tab:covcond3}. Overall, the study area departs from the HRV not only at the landscape scale, but also at the cover type and seral stage level. Being completely outside the simulated HRV is the norm for seral stages with the focal nine areas. In just 12 of 63 potential combinations, the current proportion of a given seral stage is within the HRV, but this generally means either between the 5th and 25th percentiles, or between the 75th and 95th percentiles. Only four cases (\textsc{ocfw\_u} early, \textsc{rfr\_x} late closed, \textsc{smc\_m} early, and \textsc{smc\_u} mid moderate) are characterized by a current proportion within the 25th-75th interquartile range. Mesic and xeric mixed evergreen forests, as well as xeric mixed conifer forests, had no seral stages within the HRV.

%rewrote this 2015-09-20
Interestingly, early seral conditions were less common during the simulation than on the current landscape for the mixed evergreen and ultramafic mixed forests, but more common for the xeric mixed conifer forests; for the other types, however, the proportion of early seral is within the HRV. Both mixed evergreen types and both oak-conifer types have a smaller proportion of forest in late development now than during the simulation. Late development, open canopy conditions were far more common during the HRV. For all cover types except for mesic red fir forests, this seral stage is always outside the HRV. Mid open was more common for xeric and ultramafic mixed conifer forests during the simulation than today, while mid closed was dominant during the HRV for mesic mixed conifer forests, but is not now. Finally, closed canopy conditions were much more dominant in mesic red fir forests than on the present landscape; these stages are also completely outside the HRV.

%-- Closed canopy ocfw dominant now, but not during HRV.  (didn't use)
%-- Closed conditions were much more dominant in mesic red fir forests.
%-- Mid open more common for smcx and smcm
%-- Mid closed dominant during hrv for Smcm, but not now
%-- Mixed evergreen forests and oak-conifer forests and woodlands generally have less area in late development now than during the simulated HRV.
%-- Early seral was much more prevalent in SMCX and somewhat more prevalent in RFRX, but consistent with the other cover types.
%-- Early more common for smcx
%-- Late open more common for all cover types except rfrm, always outside of hrv
%-- Mixed evergreen and xeric mixed conifer had no seral stages within the HRV.
%-- 12 of 63 cover type - seral stage combinations with the HRV
%-- Only OCFWU-Early, RFRX-LDC, SMCM-Early, SMCU-MDM, were within interquartile range.
%-- Being completely outside the simulated HRV is the norm for seral stages with the focal nine areas. In many cases, the current proportion of a given seral stage is within the HRV, but this generally means either between the 5th and 25th percentiles, or between the 75th and 95th percentiles. In just four cases (\textsc{ocfw\_u} early, \textsc{rfr\_x} late closed, \textsc{smc\_m} early, and \textsc{smc\_u} mid moderate) was the current proportion within the 25th-75th interquartile range.

An analysis of \textsc{fragstats} metrics at the landscape scale also yields insight into the historical period. First, we note that during the HRV, the landscape was composed of larger and more extensive patches, as illustrated by Figure~\ref{fig:fragland_areashape}). This trend was heavily influenced by the presence of wildfires on the landscape, as high mortality fires in particular created large areas of early development (Figure~\ref{fig:patchmaps1-early}). However, we also observed large patches in the other condition classes, which were more likely to form long and/or convoluted patches that were nonetheless extensive. Some large patches have fairly simple shapes, as in the highlighted patch in the upper right of Figure~\ref{fig:patchmaps1-mid}, while the one in the center is more complex, and the third to the left is somewhere in between.

% figure redone 2015-09-20
\begin{figure}[!htbp]
  \centering
  \subfloat[][]{
    \centering
    \includegraphics[width=0.5\textwidth]{/Users/mmallek/Documents/Thesis/maps/hrv-largepatch2410.pdf}
    \label{fig:patchmaps1-early}
    }%
  \subfloat[][]{
    \includegraphics[width=0.5\textwidth]{/Users/mmallek/Documents/Thesis/maps/hrv-largepatch2430.pdf}
    \label{fig:patchmaps1-mid}
    }
  \caption{(a) A large patch, highlighted with dark grey, of Sierran Mixed Conifer - Mesic in Early Development. The patch is 5,780 hectares, one of the second-largest patch during this timestep. (b) Two patches of Sierran Mixed Conifer - Mesic in Late Development - Closed. These are two of the largest patches during that timestep, at 10,930 hectares and 4,750 hectares.} 
  \label{fig:patchmaps1}
\end{figure}


% big 2622 12,600

In addition, we observe increased dominance by certain cover-condition types, as illustrated by smaller values for the \emph{Simpson's Evenness Index} during the HRV (Figure~\ref{fig:fragland_contagsiei}). For example, within the Sierran Mixed Conifer - Xeric cover type, Early Development and Mid Development - Open were much more widespread during the simulated HRV than in the current landscape (Figure~\ref{fig:patchmaps2}). Because Sierran Mixed Conifer - Xeric is so widespread, this shift would directly influence \emph{Simpson's Evenness}, lowering its value.

% figure redone 2015-09-20
\begin{figure}[!htbp]
  \centering
  \subfloat[][]{
    \centering
    \includegraphics[width=0.45\textwidth]{/Users/mmallek/Documents/Thesis/maps/hrv-dominance-ts0.pdf}
    \label{fig:patchmaps2-ts0}
    }%
  \subfloat[][]{
    \includegraphics[width=0.45\textwidth]{/Users/mmallek/Documents/Thesis/maps/hrv-dominance-26.pdf}
    \label{fig:patchmaps2-ts570}
    }
  \caption{Cover-Condition map focused on patches from Sierran Mixed Conifer - Xeric, showing increased dominance by certain cover-condition types during the HRV. (a) The current landscape. (b) The same region of the map during a randomly selected timestep after the equilibration period. Note the contrast between the two maps with respect to the condition classes and size of individual patches.} 
  \label{fig:patchmaps2}
\end{figure}


Third, we find that patches on the landscape were more aggregated at the cell-level during HRV, which is illustrated by the \emph{Contagion} metric. In general, patches have low levels of both dispersion and interspersion. Of course, there are many ``edgy'' areas on the landscape, but this metric indicates that across the full landscape aggregation is more typical, particularly in comparison to the current landscape. Again, the homogeneity of post-fire early development stands likely aids in increasing the contagion value (Figure~\ref{fig:patchmaps3}). 

% figure redone 2015-09-20
\begin{figure}[!htbp]
  \centering
  \subfloat[][]{
    \centering
    \includegraphics[width=0.45\textwidth]{/Users/mmallek/Documents/Thesis/maps/hrv-agg-ts0.pdf}
    \label{fig:patchmaps3-ts0}
    }%
  \subfloat[][]{
    \includegraphics[width=0.45\textwidth]{/Users/mmallek/Documents/Thesis/maps/hrv-agg-24.pdf}
    \label{fig:patchmaps3-ts615}
    }
  \caption{Cover-Condition map focused on patches from Sierran Mixed Conifer - Mesic. (a) The current landscape. (b) The same region of the map during a randomly selected timestep after the equilibration period. Note the contrast between the two maps with respect to the contagion (at the cell level).} 
  \label{fig:patchmaps3} %year 1075 is timestep 215; year 675 is timestep 135
\end{figure}

However, despite having a higher edge-to-area ratio, and being more geometrically complex, patches during the simulated HRV still show an increase in core area over the present landscape (Figure~\ref{fig:fragland_core}). This indicates that the large patches that contain core area are sufficiently large to surpass the relatively high amount of core area on the present landscape. We expected that the current landscape might have large amounts of core area because current patches, bordered by roads and human-designed treatment unit boundaries, are likely to create simple shapes. Since this is an area-weighted measure, we conclude that it is the presence of many large patches that are large enough to contain significant core area that resulted in a simulated historical range of variability that does not overlap the current landscape.\todo{More? keep rest of paragraph?} The results for the landscape metric \emph{Shape} confirm this analysis (Figure~\ref{fig:fragland_areashape}). Especially among the largest patches on the landscape, convoluted shapes are common. Again, this is not to say that large and simple shapes do not occur---they do---but in comparison to the current landscape, complex shapes were characteristic of the simulated HRV (Figure~\ref{fig:patchmaps4}).

% figure redone 2015-09-20
\begin{figure}[!htbp]
  \centering
  \subfloat[][]{
    \centering
    \includegraphics[width=0.45\textwidth]{/Users/mmallek/Documents/Thesis/maps/hrv-complexpatch-2622.pdf}
    \label{fig:patchmaps4-ts555}
    }%
  \subfloat[][]{
    \includegraphics[width=0.45\textwidth]{/Users/mmallek/Documents/Thesis/maps/hrv-simplepatch2422.pdf}
    \label{fig:patchmaps4-ts690}
    }
  \caption{Cover-Condition map focused on patches from Sierran Mixed Conifer. (a) A large patch of Sierran Mixed Conifer - Xeric in the mid development open seral stage illustrates how very large patches (this one is 12,600 hectares) may have relatively small amounts of core for their shape, yet accumulate a lot of core because of their overall size. (b) Simpler shapes do exist, such as this patch of Sierran Mixed Conifer - Mesic in the mid development open seral stage, which has a lot of core area. However, they are often much smaller: this one is 2,320 hectares.}
  \label{fig:patchmaps4}
\end{figure}



Examining some of the class-level results for similar \textsc{Fragstats} metrics as for the landscape as a whole, we see consistencies and some interesting diversions (Figures~\ref{fig:smcm_areaam}--\ref{fig:smcx_clumpy}. In general, for the area-weighted metrics, if higher proportions of the landscape are occupied by a given seral stage, that cover-seral stage combination is likely to have high values for a given metric. Again, results for area-weighted mean patch area and mean core area are consistent. Early development patches are typically characterized by more complex shapes that are less aggregated during the HRV, as compared to the current landscape. This is due to the fact early development patches during the simulation are created by fires that are allowed to burn naturally, rather than by vegetation treatments. In addition, seral stages that became rare during the HRV, tend to be smaller, aggregated and to have simple shapes. Larger shapes have a greater potential to have complicated shapes and to not be aggregated. This result may reflect the nature of the metrics at the class level as much as the seral stage structure. Overall, the class-level metrics reflect the interplay between wildfire and succession, and can best be used when considering management alternatives at the class level. Otherwise, the landscape-level interpretation is more appropriate when speaking generally. 


% Tried to come up with other ideas of what the class-level metrics mean, but not really sure.

\clearpage
\section{Management Implications}
%"Drawing comparisons between past and current fire frequencies can assist resource managers in prioritizing areas for ecological restoration, fuels reduction, certain fire or habitat management practices, and other activities. " (Hugh)

The primary goal of this study was to use our knowledge and understanding of vegetation dynamics and disturbance processes to simulate changes to landscape composition and configuration over time under a historical reference framework. We then compared those dynamics to observations of current conditions, and assessed the departure from the historic range of variability. Our landscape-level assessment is that both composition and configuration deviate substantially from the HRV. In general, the current landscape is dominated by Mid and Late Development forest and lacks the fire-dependent stand conditions (e.g. open canopy forest) and spatial heterogeneity in vegetation that were maintained by natural disturbances during the reference period. This landscape condition appear to be largely a legacy of the last 150 years of land management practices, especially fire suppression and timber management \citep{Safford2014,Stephens2007}. Substantial changes to the ecology and landscape function in western forest landscapes have been documented not only in the Sierra Nevada, but also in the Cascades and Rocky Mountains \citep{Hessburg2005,Baker2012,Baker2014,Mallek2013,Agee1993} . However, little research has been done that incorporates spatially explicit disturbance and succession modeling in combination with analyses of landscape structure.\todo{Kevin do you think the last sentence is accurate?}

In the western Sierra Nevada, foothill communities and lower elevation oak-conifer woodlands have experience a loss of species diversity, fragmentation, and outright habitat conversion due to the overlap with private lands and population growth. Middle elevation forests were and are more affected by mining and forestry; most easily accessible trees were probably cut before national forests were established \citep{SNEP1996}. The wildfire regime has been significantly altered in hardwoods, yellow pine, and mixed conifer forests \citep{Merriam2013,Safford2013}, and much less so in red fir and subalpine forests \citep{Meyer2013,Meyer2013a}. However, other human activities since the late 1800s have altered the structure of western Sierra Nevada forests, most notably to simplify it in several ways, including a decrease in species, multi-story canopies, and snags \citep{SNEP1996}. These activities are related mainly to timber harvest and to the extensive network of roads constructed to support timber harvest, fire control, and recreation. It has been suggested that this simplification of landscape structure may have a negative impact on wildlife and potentially lead to a loss of biodiversity in forests \citep{Thompson2003,Manley2004,Hunter2011}. Isolating the effect of fragmentation in our study landscape is made more difficult due to the inherent heterogeneity of Sierran landscapes---a consequence of steep natural gradients in elevation, topography, and substrate---and forests in this region tend to be somewhat patchy even in the absence of human alterations \citep{Franklin1996}. In the Pacific Northwest, ``old-growth'' connotes very large blocks of uniformly very old trees. However, in the Sierra Nevada ``old-growth'' indicates not only the presence of very large and old trees, but also a complex, patchy, ``messy'' forest of varying age classes, species, fuel quantities, and vegetation structure \citep{SNEP1996}.

% don't think climate was benign in the SN during recent past - after settlement there was a lot of mining, a lot of harvest, and a lot of fire. fire slowed down when suppression began. 
% Forest fragmentation in the sierra nevada
% changes to plants and animals from fire suppression, logging, road-building
% included Spotted Owl and Northern Goshawk!!
% logging removes habitat that would be used by birds, true for salvage also (Cahall and Hayes 2008)
% roads are bad - show map with roads (Gucinski et al. 2001)
% would not say we are in early stages of understanding fragmentation
% late successional very important - Franklin et al. Alternative Approaches, call for 5-30k acre patches (run fragstats on the rec4 without special tables?)


% omitting this paragraph - not sure it's relevant
%\paragraph{p2}Our findings are particularly interesting in light of increasing concern over anthropogenic habitat loss and fragmentation (Rochelle et al. 1999; Knight et al. 2000). Forest fragmentation has received considerable research attention in many regions of North America (e.g., Whitcomb et al. 1981; Robbins et al. 1989; Lehmkuhl and Ruggiero 1991; McGarigal and McComb 1995; Schmiegelow et al. 1997; Trzcinski et al. 1999; Villard et al. 1999). %Girvetz eta al. 2008 (California). However, we are in the earliest stages of understanding the patterns, processes, and ecological significance of forest fragmentation in the southern Rocky Mountain region (Knight et al. 2000). It is not clear, for example, how the native biota responds to anthropogenic changes in landscape patterns caused by logging and road-building and disruption of natural disturbance regimes (e.g., fire suppression). 



Based on our results, it might be tempting for managers to reach the simple conclusion that the landscape is less fragmented today than during the reference period. For example, today's landscape is simpler (lower \textsc{shape} values), contains less core area (based on \textsc{core\_am}), and has less contrast between patch types (\textsc{econ\_am}). However, this conclusion is not as straightforward as it might seem. Fragmentation is a landscape-level process in which a specific habitat is progressively sub-divided into smaller, geometrically altered, and more isolated fragments as a result of both natural and human activities. This process involves changes in landscape composition, structure, and function at many scales and occurs on a backdrop of a natural patch mosaic created by vegetation transitions, both those mediated by and independent from natural disturbances \citep{McGarigal1995}. The scale at which fragmentation occurs is at the level of a specific habitat type; it is the habitat, rather than the landscape, that becomes fragmented. In this study we evaluated the spatial pattern---and by implication, the fragmentation---of many different patch types (defined by unique combinations of cover type and condition class). Certaintly, some of these patch types are less fragmented in the current landscape than they were under the simulated HRV. 

% Taking this out because adding an analysis of clumped "late" or "mid" is a lot of extra work. Probably beyond scope right now; could be part of phase II if this happens (language would be to revise HRV analysis and future scenarios analysis to include groupings - could take this in several directions, such as lumping the mesic and xeric types together for analysis also.)
%\todo{Need help with script to evaluate stages by themselves}\emph{This is true in general for most of the late-seral forest patch types. However, not all patch types are less fragmented in the current landscape. For example, many of the early-seral forest patch types are in fact much more fragmented in the current landscape than they were under the simulated HRV.} Thus, conclusions about habitat fragmentation in the current landscape must be qualified with specific reference to one or more well-defined habitats.

In addition, we evaluated vegetation patterns in the current landscape after excluding roads (i.e., we removed roads from the land cover map by filling in those areas with the nearest alternate cover type). Figure \ref{fig:roadcovermap} shows the cover type layer with roads overlaid. %
%
\begin{figure}[!htbp]
  \centering
  \includegraphics[height=.4\textheight]{/Users/mmallek/Tahoe/Report2/images/roads_cover.png}
  \caption{The core project area superimposed with the cover type map and all roads in black. There are a two designated roadless areas, but in general roads are common throughout the watershed. The closeup area is just northeast of the Pendola reservoir.} 
  \label{fig:roadcovermap}
\end{figure}
%
We did this in order to be consistent with our simulation of landscape structure changes during the reference period. However, the significant and extensive impact of roads on Sierran ecosystems is well documented in the literature \citep{Karr2004,Trombulak2000,Gucinski2001,Theobald2011}.  In particular, roads are linear landscape features that can create high-contrast edges and bisect patches. They may cause greater fragmentation of habitats than the direct loss of habitat from associated land use activities \citep{Gucinski2001,Tinker1998,Mcgarigal2001}. Given the large amount of roads within the project area and their disproportionate influence on landscape structure and function, any conclusions regarding departure in relation to habitat fragmentation that does not consider road impacts should be viewed with extreme caution. The impacts of roads on landscape structure will be addressed in our evaluation of alternative land management scenarios in the next phase of this project\todo{Keep this last sentence?}.

%"Drawing comparisons between past and current fire frequencies can assist resource managers in prioritizing areas for ecological restoration, fuels reduction, certain fire or habitat management practices, and other activities. " (Hugh)

Our results imply that any attempt to restore the landscape structure to a composition and configuration that aligns with the historic range of variability described in this thesis would likely be a difficult and long-term undertaking. Our model equilibration period is the length of the interval between the starting condition and structure (in this case, of the current landscape) and beginning of a stable range of variation. It is a function of not only how far outside the stable range of variation the current landscape is, but also the speed at which disturbance and succession processes interact to affect a change in the landscape trajectory. Thus, we can infer that if management activities were designed to emulate natural disturbance processes, then it would take a length of time equal to the equilibration period to return the landscape to its HRV. The equilibration period for our simulation was 40 timesteps, or 200 years. Given that forest planning horizons are on timescales of 5--30 years, and the fact that climate change is predicted to have measurable impacts on both wildfires and vegetation community succession and structure over the next several decades, an effort to ``restore'' the current landscape to the exact conditions aligned with our simulated HRV may not be practical. Moreover, the extent and intensity of disturbance required to emulate the natural disturbance regime is significant, and a simple restoration of the historical fire regimes would not be possible with public input and environmental consultation. Certainly, social, economic, and political challenges exist. However, as several ecologists have pointed out, our results can still be used to guide the prioritization of areas for restoration, for designing fire and fuels management projects, or describing desired future conditions \citep{Safford2013,Keeley2000}.


\section{Individual Cover Type Recomendations}\todo{worth including these? Maybe just take the ideas but not the cover types specificity since most ideas are pertinent to class-level results regardless of cover.}
Our study methodology makes it possible to single out separate cover types and condition classes for analysis, and in this section we provide some interpretation of those results that is targeted toward providing recommendations for specific strategies that could be used to push landscape structure toward the HRV, and other recommendations of management strategies to avoid. Because in reality cover types are interspersed with one another, managers will need to consider the recommendations for individual cover types within the context of the vegetation actually present within a given management unit.

\subsection{Mixed Evergreen - Mesic}
Mesic mixed evergreen forests in our simulation burned with a fire rotation of 55 years under historical conditions. Low mortality fire was much more common than high mortality fire, although both are necessary to create stand structure and composition similar to that observed during the simulation. Late development stands (predominantly closed canopy) of mixed evergreen were dominant during the simulation, especially compared to the current landscape. The current landscape also contains more area in the Early Development condition (Figure \ref{fig:covcondbar_megm}. 

Based on these observations, we recommend management strategies that imitate low-mortality disturbance; that is, removing less than 70\% of the existing top-level canopy cover. Such a strategy would provide a mechanism for the forest to age into the late development conditions. It is not necessary for all stands of mesic mixed evergreen forests to receive the same treatment on the same schedule. In certain parts of the forest, fires would have recurred more frequently than the 55 year rotation (see Figure \ref{fig:preturn_megm}). The point-specific fire return interval for low mortality fires within this cover type ranged from around 20 years to 130 years. Managers who are charged with focusing fuels reduction on certain areas within the forest could set goals of carrying out vegetation treatments somewhat more frequently in certain parts of the forest and less frequently in other parts, thereby also contributing to the overall more complex landscape pattern observed during the simulated HRV. 



With respect to the landscape structure metrics (computed using \textsc{Fragstats}), we highlight the values for the Late Development - Closed and Moderate conditions, since they dominated during the HRV simulation. We observed that present-day patches of mesic mixed evergreen forests in these conditions were smaller, more clumped, less geometrically complex, and contained less core area than during the simulated HRV. Having less core area and being more clumped may seem contradictory, but this outcome is likely due to the smaller size of the current patches. Restoration of these forests to patches that reflect a more natural succession process may be challenging for managers, given practical needs like using roads and riparian buffers as the edges of treatment units. It may not be practical to perform mechanical treatments over large areas within this cover type. However, when conducting treatments using prescription fires, creative solutions should be sought to generate more complex edges and to complete burns over sufficiently large areas that larger core areas can be generated.

Because the most commonly occuring condition classes for mesic mixed forests were actually late development closed and moderate canopy cover, it may also be more helpful for managers to think about this cover type not as receiving direct treatment, but as being influenced by other treatments targeting cover types such as oak-conifer forests and woodlands or sierran mixed conifer forests. When mesic mixed evergreen forests border or serve as the edge of treatment units targeting other vegetation types, managers can evaluate the patches of mesic mixed evergreen forests present at the time of treatment and consider whether to expand the management unit to treat them, to try and achieve an irregular edge to increase patch complexity, or to exclude them from treatment in order to promote development into the late development conditions.                                                                                                                                                                                                                                                                                                                                                                                                                                                    
\subsection{Sierran Mixed Conifer - Mesic}
Mesic mixed conifer forests are thought to have burned with a fire rotation of 28 years under historical conditions. Low mortality fire was much more common than high mortality fire, but both are necessary to create stand structure and composition similar to that observed during the simulation. 

All seven condition classes were well represented during the simulated historical period. The most common condition (highest median value) during the simulation was the mid development, closed canopy cover condition. The proportion of mesic mixed conifer on the current landscape is close to the median values for early development, mid development open, and late development moderate. There is much more mid development moderate and late development closed, and much less mid development closed and late development open, now compared to the simulated HRV. Based on these observations, in the short term we recommend maintaining existing open stands of mesic mixed conifer though prescribed fire and other understory vegetation treatments. Because patch sizes are much smaller now on average, when identifying stands to push towards, for example, mid closed or late open, we recommend looking for treatment units near or adjacent to existing patches of the target condition, in order to create large patches in that type. Overall, closed canopy forest across development classes is near the median values for closed canopy conditions during the simulated HRV. We do not believe more closed canopy forest is needed to bring the landscape toward HRV conditions. That said, it probably makes sense to work to create conditions for younger closed canopy forests, so that all the closed canopy stands on the forest do not belong to the oldest age class.

In the medium term, we recommend the restoration of fire wherever practicable. Fire was quite common on the historical landscape and was the main driver of the complexity of patches we see in the simulated landscape. The point-specific fire return interval for low mortality fire ranged greatly, from 18 years to around 100 years. Managers can transfer this variability into flexibility when planning and executing vegetation treatments and wildfire response. In fact, the spatial variability of fire is critical for creating spatial variability of forests and plants observed as outputs of the simulation. Managers who are charged with focusing fuels reduction on certain areas within the forest could set goals of carrying out vegetation treatments somewhat more frequently in certain parts of the forest and less frequently in other parts, thereby also contributing to the overall more complex landscape pattern observed during the simulated HRV. 

%\begin{figure}[!htbp]
%  %\centering
%  \subfloat[][]{
%%    \centering
%    \includegraphics[width=0.8\textwidth]{/Users/mmallek/Tahoe/Report2/images/CovcondHRVBarplots/MixedEvergreen-Mesic_Early-AllStructures_srvplot.pdf}
%  }\\%
%  \subfloat[][]{
%%    \centering
%    \includegraphics[width=0.8\textwidth]{/Users/mmallek/Tahoe/Report2/images/CovcondHRVBarplots_nolegend/MixedEvergreen-Mesic_Mid-Closed_srvplot.pdf}
%  }\\%
%    \subfloat[][]{
%%    \centering
%    \includegraphics[width=0.8\textwidth]{/Users/mmallek/Tahoe/Report2/images/CovcondHRVBarplots_nolegend/MixedEvergreen-Mesic_Mid-Moderate_srvplot.pdf}
%  }\\%
%    \subfloat[][]{
%%    \centering
%    \includegraphics[width=0.8\textwidth]{/Users/mmallek/Tahoe/Report2/images/CovcondHRVBarplots_nolegend/MixedEvergreen-Mesic_Mid-Open_srvplot.pdf}
%  }\\%
%    \subfloat[][]{
%%    \centering
%    \includegraphics[width=0.8\textwidth]{/Users/mmallek/Tahoe/Report2/images/CovcondHRVBarplots_nolegend/MixedEvergreen-Mesic_Late-Closed_srvplot.pdf}
%  }\\%
%    \subfloat[][]{
%%    \centering
%    \includegraphics[width=0.8\textwidth]{/Users/mmallek/Tahoe/Report2/images/CovcondHRVBarplots_nolegend/MixedEvergreen-Mesic_Late-Moderate_srvplot.pdf}
%  }\\%
%    \subfloat[][]{
%%    \centering
%    \includegraphics[width=0.8\textwidth]{/Users/mmallek/Tahoe/Report2/images/CovcondHRVBarplots_nolegend/MixedEvergreen-Mesic_Late-Open_srvplot.pdf}
%  }\\%
%  \caption{Cover-condition barplots for Mixed Evergreen - Mesic dynamics. For each condition class, the color the bar represents the distance from the median value during the simulated HRV. Green represents the 25th-75th percentiles; yellow represents the 5th-25th and 75th to 95th percentiles; red represents the 0th-5th and 95th-100th percentiles. The blue vertical line marks the 50th percentile and the black vertical line indicates the current cover extent. To read the ``Early–All Structures'' barplot, for a given percentage of the cover type extent, the x-axis value indicates an observed proportion, and the color corresponding to that point indicates the percentile range that value falls within. In this example, the current percent of cover extent for this cover type and condition class falls within the 95th-100th percentile range during the simulated HRV.}
%  \label{fig:covcondbar_megm}
%\end{figure}

With respect to the landscape structure metrics (computed using \textsc{Fragstats}), we observed that present-day patches of mesic mixed conifer forests in these conditions were smaller, less clumped, less geometrically complex, and contained less core area than during the simulated HRV. Restoration of these forests to patches that reflect a more natural succession process may be challenging for managers, given practical needs like using roads and riparian buffers as the edges of treatment units. It may not be practical to perform mechanical treatments over large areas within this cover type. However, when conducting treatments using prescription fires, creative solutions should be sought to generate more complex edges and to complete burns over sufficiently large areas that large core areas are a byproduct of the treatment.
       











 
% !TEX root = master.tex

\chapter{Future Range of Variability}
\label{ch:FRV}

\section{Abstract}
In the Sierra Nevada, cycles of fire and vegetation recovery occur variably over large extents, as well as over long periods of time. The U.S. Forest Service's 2012 Planning Rule explicitly calls for the agency to estimate and describe the range of variability under natural disturbance regimes, and manage for those characteristics. Recent warming and drying trends have already influenced a more frequent and proportionally more severe fire regime in the Sierra Nevada. These trends are anticipated to continue under warmer and drier climate change scenarios. I used \textsc{RMLands}, a spatially-explicit, stochastic, landscape-level disturbance and succession model capable of simulating fine-grained processes over large spatial and long temporal extents, to evaluate trends in landscape composition and configuration under a range of potential future climate scenarios. My study area is on the Yuba River watershed on the Tahoe National Forest, in California. My results show increasing burned area and increasing high severity fire with increasing climate values. I also find that today's landscape is outside the future range of variability, and that this departure increases with increasingly warm and dry climate conditions. Based on these findings, I recommend more aggressive restoration efforts and the implementation of mitigation measures where the consequences of changing fire regimes are socially unacceptable.



%%%%%%%%%%%%%%%%%%%%%%%%%%%%%%%%%%%%%%%%%%%%%%%%%%%%%%%%%%%%%%%%%%%%%%%%%%%%%%%%%%%%%%%%%%%%%%
%%%%%%%%%%%%%%%%%%%%%%%%%%%%%%%%%%%%%%%%%%%%%%%%%%%%%%%%%%%%%%%%%%%%%%%%%%%%%%%%%%%%%%%%%%%%%%
%%%%%%%%%%%%%%%%%%%%%%%%%%%%%%%%%%%%%%%%%%%%%%%%%%%%%%%%%%%%%%%%%%%%%%%%%%%%%%%%%%%%%%%%%%%%%%
\section{Introduction}
\todo{intro is a bit too fluid. need to clearly define sections and make sure sections only contain what's in the bullets. may be best to outline the intro and then do some revising.}
\subsection{Disturbance in the Sierra Nevada}
In the Sierra Nevada, cycles of fire and vegetation recovery occur variably over large extents, as well as over long periods of time. Ongoing disturbance results in heterogeneity in vegetation composition and configuration, which can be captured by various statistical metrics \citep{Monica2008}. Prior to European settlement, wildfire was the major source of disturbance in Sierran forests, shaping the composition and configuration of vegetation communities \citep{SNEP1996a}. Fires were primarily lightning-caused, although indigenous peoples are thought to have set fires for vegetation management, especially in the lower elevations \citep{Anderson1996}. 

In general, regardless of vegetation type, fires during the pre-settlement period were thought to burn primarily at low intensities. High mortality (over 75\% overstory canopy mortality) was uncommon \citep{Skinner1996, SNEP1996a,Mallek2013,Stephens2015}. Under this disturbance regime, stand-replacing fire initiated early successional conditions on the landscape, but since they were uncommon, most fires only affected the understory by removing fuels \citep{Skinner1996, SNEP1996a,Mallek2013}. Fires recurred frequently in most forest types, and mean rotations could be as short as 20 years in \emph{Pinus ponderosa} (Ponderosa pine)-dominated forests \citep{Mallek2013}.  Fire rotations generally increased with increased moisture and elevation, and could vary widely around the observed mean rotation period \citep{Mallek2013}. Fires of moderate severity, that result in more open overstory canopy cover, were more prevalent in more xeric environments, including south-facing slopes and ridges \citep{Mallek2013,Safford2014,SNEP1996a,SNEP1996}. Where fires did not recur frequently or only occurred at very low severity levels, succession processes such as infill or overstory growth led to gradual closing of the overstory tree canopy. For most of the forest cover types in the study area, high severity fire rates were low, allowing stands to succeed into late development and old growth conditions with a variety of canopy structures \citep{Mallek2013,Safford2014,SNEP1996a,SNEP1996}.


Since then, fire suppression, logging, grazing, and mining have all interacted to alter the historical fire regime and vegetation patterns \citep{Stephens2015,Knapp2013}. After large-scale fire suppression became the norm in the second half of the 19th century, less fire-tolerant species (such as \emph{Pseudotsuga menziesii} (Douglas fir) and \emph{Abies concolor} (white fir)) have come to dominate areas where they were once a minor part of the vegetation community. Grazing and development made fires less common by altering or removing the fine fuels that carried fire. Timber harvest, especially of fire-tolerant species such as \emph{P.~ponderosa} and \emph{Pinus lambertiana} (sugar pine), accelerated the increased cover of species such as white fir. Moreover, fire suppression allowed the buildup of medium size fuels and ladder fuels, which promotes larger more severe fires when they do occur. Finally, the lack of natural fires has meant that variation in fuel loading has decreased, which allows large fires to spread over very large areas \citep{Hessburg2005,Beaty2007,Meyer2008}.



\subsection{Forest Planning}
With the emergence of ecosystem management in the early 1990s, the need to recognize ecosystems as dynamic and constantly changing became well accepted, and calls to manage forests sustainably became common \citep{Christensen1996}. Within the context of forest and land management planning, the restoration of ecosystems to their pre-European settlement states was incorporated as a goal or desired future condition into various plans, including the Sierra Nevada Ecosystem Project \citep{SNEP1996a}. The U.S. Forest Service's 2000 Planning Rule explicitly called for the agency to estimate and describe the range of variability under natural disturbance regimes, and manage for those characteristics (36 CFR \textsection 219 2000). The need to consider the natural range of variability was maintained through various amendments to the rule, and is still present in the 2012 Planning Rule, finalized in early 2015 (36 CFR \textsection 219 2012).



\subsection{Range of Variability Analysis}
Historic range of variability (HRV) analysis is a useful tool in landscape planning. HRV analysis is intended to help conceptualize the mechanisms behind large-scale ecosystem functions and provide a basis from which to make predictions about how a given ecosystem will react to disturbances in the future \citep{Nonaka2005,Landres1999}. Methods for quantifying the natural range of variability for a diversity of landscapes in the United States augmented the development of research focused on this task \citep{Landres1999}. Of these, simulation of the historical dynamics became fairly popular. By 2004, some 45 landscape fire and succession models alone had been developed \citep{Keane2004}. Many of these, such as \textsc{landis} \citep{He1999}, \textsc{zelig-l} \citep{Miller1999}, \textsc{safe-forests} \cite{Sessions1997} and \textsc{landsum} \citep{Keane2012} are still in use today. Landscape fire and succession models are used to create spatially-explicit simulations of both of these key forest processes, typically outputting a set of GIS layers for each timestep of the model that can then be analyzed to quantify trajectories and patterns in the disturbance regime, seral stage composition, and landscape configuration over time \citep{Keane2004}. A component of many landscape fire and succession models are state and transition models, which are as much as anything a framework for defining the fundamental vegetation communities and the probability over time of completing a transition from one defined set of vegetative community characteristics to another \citep{Stringham2003,Blankenship2015}.

%Many range of variability analyses in the United States focus on the historical range of variability (HRV) of an area. The Rocky Mountains and Oregon Coast Range in particular have been the focus of several HRV studies, while only one has been conducted in the Sierra Nevada \citep{Miller1999}, which took place in Sequoia National Park in the southern Sierra.

\subsection{Climate and range of variability}

While HRV studies can play an important role in informing natural range of variability, the need to explore and understand the ramifications of climate change on the disturbance regime and forest structure is also critical. Changes to precipitation and temperature regimes as a result of climate change are anticipated to occur in the northern Sierra. Concern about the influence these changes will have on local disturbance regimes, and subsequently, on seral stage distributions and patch configurations, motivates this study \citep{Fule2008,North2012}.
%
Recent warming and drying trends, and the current drought, have already influenced a more frequent and proportionally more severe fire regime in western forests in general and the Sierra Nevada in particular  \citep{McKenzie2004,Westerling2011,Miller2012}. These trends are anticipated to continue under warmer and drier climate change scenarios \citep{Westerling2008,Dale2001}. CChanges have also been reported in the elevation of fires in the Sierra Nevada, increasing the potential for upward shifts of the elevational range occupied by species and vegetation assemblages \citep{Schwartz2015}. Where the focus of management efforts has in the past been restoration, current policy emphasizes using adaptive strategies to ensure resilient ecosystems \citep{Stephens2010}.

Range of variability analyses that offer a complementary analysis of future scenarios under climate change are rare (but see \cite{Keane2008} and \cite{Duveneck2014}). By simulating a range of potential future climate scenarios, I generate data to use in evaluating trends in landscape pattern related to trends projected under climate change, and place the current landscape in that context. Moreover, I use this additional information to consider which restoration strategies are likely to promote resilient forests and make sense ecologically for the area under study \citep{Duncan2010}.

Early successional habitats are not a major focus of forest ecology research, in part because they are seen as an intermediate phase that is ideally short \citep{Swanson2011}. However, they are a critical component of all systems, functioning as a major contributor to biodiversity and supporting a range of species' habitat needs \citep{Chang1995,Hutto2008,Swanson2011}. The Sierra Nevada Framework, last updated in 2007, identifies management indicator species that use openings and early seral habitat \citep{USDAForestService2004,USDAForestService2007}. Recent trends of increasing wildfire extent and severity mean that managers face more decisions about when and how to manage post-fire early successional habitat \citep{Stephens2013,Dellasala2014}. My model results will provide insight into the spatial configuration of early successional forests under a natural fire regime for the intensively used mixed conifer zone. These results may be used when designing restoration efforts using both prescribed fire and mechanical harvest techniques.
%%%

\textsc{RMLands} has been used previously to assess the HRV on the San Juan National Forest and the Uncompahgre Plateau in Colorado \citep{McGarigal2005,McGarigal2005a,Romme2009}, as well as the Lolo National Forest in Montana \citep{Cushman2011}. Following the Montana study, which adapted \textsc{RMLands} to use data from the LandFire project (\burl{http://www.landfire.gov}), I further adapted the software for use in the Sierra Nevada in order to prepare an HRV analysis for part of the Tahoe National Forest in California. In this paper I quantify and describe a ``future range of variability'' (FRV) that can inform restoration and planning under a changing climate \citep{Duncan2010}.


\subsection{Objectives}
In this study, my objectives were to simulate wildfires and vegetation succession under a set of potential future climate scenarios, compare the results to the current landscape across metrics describing landscape composition and configuration, and assess management implications. To do this, I simulated forest fires and succession using \textsc{RMLands}, holding all model parameters except the climate parameter constant. The climate parameter incorporated Palmer Drought Severity Index (PDSI) values from a suite of seven climate trajectories developed by the National Center for Atmospheric Research (USA) and the Canadian Centre for Climate Modelling and Analysis to the year 2100 \citep{Cook2014}. I used \textsc{Fragstats} software and R to analyze outputs and report the 90\% range of variability for simulated future metrics. Ultimately, I evaluated the results from a series of simulations of these future scenarios and compared them to the current conditions, using the conclusions to assess implications for restoration and forest planning.









%%%%%%%%%%%%%%%%%%%%%%%%%%%%%%%%%%%%%%%%%%%%%%%%%%%%%%%%%%%%%%%%%%%%%%%%%%%%%%%%%%%%%%%%%%%%%%%%%%%%%%%%%%%%%%%%%%%%%%%%%%%%%%%%%%%%%%%%%%%%%%%%%%%%%%%%%%%%%%%%%%%%%%%%%%%%%%%%%%%%%%%%%%%%%%%%%%%%%%%%%%%%%%%%%%%%%%%%%%%%%%%%%%%%%%%%%%%%%%%%%%%%%%%%%%%%%%%%%%%%%%%%%%%%%%%%%%%%%%%%%%%%%%%%%%%%%%%%%%%%%%%%%%%%%%%%%%%%%%%%%%%%%%%%%%%%%%%%%%%%%%%%%%%%%%%%%%%%%%%%%%%%%%%%%%%%%%%%%%%%%%%%%%%%%%%%%%%%%%%%%%%%%%%%%%%%%%%%%%%%%%%%%%%%%%%%%%%%%%%%%%%%%%%%%%%%%%%%%%%%%%%%%%%%%%%%%%%%%%%%%%%%%%%%%%%%%%%%%%%%%%%%%%%%%%%%%%%%%%%%%%%%%%%%%%%%%%%%%%%%%%%%%%%%%%%%%%%%%%%%%%%%%%%%%%%%%%%%%%%%%%%%%%%%%%%%%%%%%%%%%%%%%%%%%%%%%%%%%%%%%%%%%%%%%%%%%%%%%%%%%%%%%%%%%%%%%%%%%%

\section{Methods}

\subsection*{Study area}
The study area (see Figure~\ref{projectarea-ch3}) is located on the northern part of the Tahoe National Forest, on the Yuba River and Sierraville Ranger Districts, and comprises about 181,550 hectares. The topography of the study area consists of rugged mountains incised by two major and a few minor river drainages. Elevation ranges from about 350 m to 2500 m. The area receives 30 cm to 260 cm of precipitation annually, most of which falls as snow in the middle to upper elevations \citep{Storer1963}. Like the rest of the Sierra, the study area has a Mediterranean climate, in which summer drought typically persists from May to September. This increases the importance of developing a significant snowpack during the winter months, since snowmelt runoff is a key source of soil moisture during the late spring and summer months \citep{Minnich2007,Skinner1996}. Datasets of the 30-year normal precipitation at 800 m resolution for the northern Sierra (obtained from the Oregon State PRISM as part of a separate effort), illustrate that particularly high amounts of precipitation falls across the middle elevations of the study area compared to the larger region \citep{PRISMClimateGroup2004}. This increased moisture contributes to the occurrence of exceptionally productive patches of forest \citep{Littell2012}. Vegetation is tremendously diverse and changes slowly along an elevational gradient and in response to local changes in drainage, aspect, and soil structure. Grasslands, chaparral, oak woodlands, mixed conifer forests, and subalpine forests are all found within the study area.

% brad said to make study area more obvious for non-US readers; will probably have to redo plot for publication but this is ok for now I think
\begin{figure}
\centering
\includegraphics[width=.8\textwidth]{/Users/mmallek/Tahoe/Report3/images/studyarea.png}
\caption{The Sierra Nevada Ecoregion is outlined in brown. The study area (outlined in green) is located in the northern extent of the Sierra Nevada on the Tahoe National Forest, comprising the Yuba River watershed.}
\label{projectarea-ch3}
\end{figure}

Sierran Mixed Conifer - Xeric and Sierran Mixed Conifer - Mesic forests are the two most prevalent cover types within the study area, together comprising 63\% of the landscape. They are characterized by five conifers and one hardwood: \emph{Abies concolor, Pseudotsuga menziesii, Pinus ponderosa, Pinus lambertiana, Calocedrus decurrens}, and \emph{Quercus kelloggii} (Figure~\ref{fig:smctrees}). At least three conifers are typically present in any given stand. All of these species can be found in either cover type, but some are more closely associated with either the mesic or xeric variant. The characteristic species of the mesic type, \emph{A.~concolor} and \emph{P.~menziesii}, are less adapted to fire. Species characteristic of the xeric type, \emph{P.~ponderosa}, \emph{P.~lambertiana}, plus \emph{Q.~kelloggii}, are more fire-adapted. \emph{C.~decurrens} is found in both subtypes, but is very rarely dominant. The distrbution of these species is normally an outcome of the variation in the frequency and intensity of wildfire under natural conditions, although alteration of these conditions can affect ther distrbution. \emph{A.~concolor} tends to be the most ubiquitous species, especially on north-facing slopes. \emph{P.~ponderosa} was historically the dominant species, and is strongly associated with a frequent, low severity fire regime \cite{WHR1988,Landfire2007}.


In my model, early successional conditions within both mesic and xeric mixed conifer forests are always created by high mortality fire. High mortality fire, defined as fire in which over 75\% of the overstory canopy is killed, resets the successional pathway. Either chaparral or trees may establish after a cell experiences this level of fire during the simulations. Chaparral, a community that includes \emph{Arctostaphylos, Ceanothus}, and \emph{Chrysolepis} species, is often treated as its own cover type. This is in part due to the fact that chaparral establishment tends to inhibit the establishment and growth of conifer species, thus delaying succession process to (mid-development) forest. When studies occur along short temporal scales, it is often more meaningful to categorize these two vegetation communities as separate. However, my study is focused on long temporal scales, and within the study area chaparral will eventually succeed to trees. Consequently, both communities are considered early development.\todo{added more about early} 

\subsection{RMLands}

\textsc{RMLands} is a spatially-explicit, stochastic, landscape-level disturbance and succession model capable of simulating fine-grained processes over large spatial and long temporal extents \citep{McGarigal2005}. It is grid-based and simulates fire on landscapes in a spatially explicit and realistic manner, in that fire perimeters resemble those that occur naturally. State transitions are simulated at the 30 m pixel scale. Transitions may take place in response to fire or in the absence of it (natural succession) \citep{McGarigal2012}. Outputs from the model are readable by the landscape pattern analysis software \textsc{Fragstats} \citep{Fragstats2012}, which facilitates the landscape configuration analysis.

In \textsc{RMLands}, fires spread probabilistically based on the susceptibility of an individual cell. It does not contain a mechanistic fire model and fuels are not directly incorporated into fire spread. In addition, I do not classify individual fires as a whole to a \emph{low, mixed, or high severity} status. Some fire ecologists combine fire attributes such as flame length and fire size into their interpretation of the relative \emph{severity} of a particular fire \citep{Agee1993}. Ecologists working at other scales and not working with models often describe \emph{mixed severity} regimes \citep[e.g.,][]{Kane2013}, which \citet{Collins2010} define as ``stand-replacing patches within a matrix of low to moderate fire-induced effects.'' If I were to adopt that definition, nearly all fires would be classified as \emph{mixed severity} due to the 30 m cell size and resolution at which fire mortality is defined, rendering this perspective moot for my study. Instead, I focus on defining conditions under which transitions among potential states within a given cover type occur or not. I evaluate and classify fire by its effects on individual cells. First, I evaluate whether a cell burned. Next, all burned cells are evaluated probabilistically and assigned either a high severity (``high mortality'') outcome or low mortality outcome. If a cell burns at high severity, then it is deemed to have had a high mortality outcome and transitions to the Early Development seral stage. Recently, some researchers have differed on whether 75\% or 95\% overstory tree mortality is a more appropriate cutoff point for defining a ``stand-replacing'' event \citep{Fule2014,Mallek2013}. In this paper, I use 75\% as the cutoff, which is widely accepted in the literature \citep{Agee1993,Agee2007,Miller2009,Baker2014}.

%%%%
%%%%


In collaboration with USDA Forest Service staff, I developed a system of land cover and seral stage classification based on LandFire (\burl{http://www.landfire.org}) and \citet{VandeWater2011}'s Presettlement Fire Regimes, and crosswalked Forest Service corporate spatial data based on the Northern Sierra \textsc{calveg} classification \citep{USDAForestService2008} to the 31 cover types. I also used Forest Service corporate spatial data to develop layers for seral stage and age, and for the physical environment (e.g., elevation). State and transition models for 25 of the 31 land cover types were developed based on the Vegetation Dynamics Development Tool (VDDT) models associated with the LandFire project \citep{Landfire2007}, and refined with input from local experts to capture subtle changes in succession and transition at the project scale. In general model parameters were developed using meta-analyses published in the literature. For example, transition probabilities were calculated using LandFire data, several fire rotation calibration parameters were taken from \citet{Mallek2013}, and wind direction information was obtained from several area weather stations. I used landscape conditions as of 2010 as the starting point for all simulations and as the ``current'' conditions for comparison with the HRV results.

Although \textsc{RMLands} is a process-based model with parameters sourced from the literature, the collaboration had greater confidence in some parameters than others, especially as to how they function within the \textsc{RMLands} framework. Consequently, I calibrated the model parameters by iteratively adjusting certain low-confidence parameters to optimize the output values for parameters known with high confidence. Specifically, I manipulated an ignition calibration coefficient (number of attempted fire starts) and a fire rotation index (mean return interval value for the Weibull distribution), and measured calibration success based on conformity to empirically derived rotation values at the cover type level. Fire rotation index values were multiplied by a single factor across all seral stages of a given cover type; that is, cover types were modified as groups but the index ratios within them were held constant.

I set the calibration target as rotation values for the nine most prevalent cover types within 10\% of their original target rotations (Sierran Mixed Conifer - Mesic, Xeric, and Ultramafic variants; Red Fir - Mesic and Xeric variants; Oak-Conifer Forest and Woodland - standard and Ultramafic variants; Mixed Evergreen - Mesic and Xeric variants). I focused on these nine types because they all extend across more than 1,000 ha, and are thus statistically stable from simulation to simulation. Target values were based on published empirical values and refined with input from local experts. I chose rotation as the calibration target because targets were available from the literature and because fire rotation is a fundamental measurement that \textsc{RMLands} was designed to capture. In addition, using rotation ties calibration to a parameter that is relatable to Forest Service staff and that can be used as a target by managers in various programs.

In a historical range of variability analysis, the model is typically run for a very long time to capture full disturbance cycles and the consequent effects on vegetation. This is possible when the climate parameter oscillates around a mean, but not when a clear trend exists, such as in the case of climate change. Therefore, instead of running the model for a long time, I ran it out to the year 2100 100 times for each included climate parameter sequence. No true model equilibration was done for the future simulations, but I elected to include only the final five timesteps (25 years) of the results in order to achieve some distance from the starting (current) conditions, maintain focus on the ending trajectory of climate and the range of variation possible in the landscape at the end of the 21st century, yet still capture the variability inherent to the PDSI-based climate parameter (Figure \ref{fig:pdsi-final5}).



%%%% Brad's comment: Is it possible to validate this model in some way? In particular, how do we know that varying the climate multiplier affects fires in a realistic way? Could you hindcast to see how well the PDSI for past years matches the historical fire pattern?

%%%% Notes in response: not easily. some language in hrv that addresses this. we don't know, beyond my assumption that the model treats it realistically in the first place, which is based on Kevin's word. You could hindcast, maybe. But I'm not expecting a super strong relationship. Don't know exactly how to deal with this comment but it feels important.
%%%% More notes: HRV article has a brief comparison of climate vs. disturbed area. Weak relationship. We expect cascading results affecting landscape structure and composition as well as changes to fires. Some non-standard fire behavior may occur but no real way to test this because we don't know what fires will "look" like in the future.

\subsection{Climate Parameter and PDSI}

The range of potential future climate scenarios used to parameterize the model in this study come from models initialized using the set of parameters for Representative Concentration Pathway (RCP) 8.5. RCP8.5 includes no specific climate mitigation target, unlike the other three RCP scenarios in use \citep{Riahi2011}. As a result, it is considered a reference, or baseline scenario, in which greenhouse gas emission and concentrations increase over time without leveling out \citep{Riahi2011}. A literature review during the RCP development process designated radiative forcing in 2100 of 8.5 W/m$^2$ as the high end of plausible futures that had been modeled \citep{VanVuuren2011}. The corresponding concentration of $>\sim 1370 \text{CO}_2$ -eq in 2100, compared to 375 $\text{CO}_2$ -eq in 2005. The 66\% range of temperature increase above pre-industrial levels under the RCP8.5 scenario is 4.0\textdegree -- 6.1\textdegree \citep{Rogelj2012}. Since the development of RCP8.5 as a scenario, storylines have been developed that describe how such a scenario could come about. The main storyline released for RCP8.5 described a world in which human populations continue to increase, rising to 12 billion by 2100. Little progress in energy efficiency and the food demands of the increasing population lead to high energy demands, which are met by coal-intensive technology choices \citep{Riahi2011}. Data from the RCP8.5 scenario was used to model climate variables like temperature and precipitation, which \citet{Cook2014} used to predict trajectories of drought severity to 2100.

A climate variable unique to each timestep in the model is the key parameter that varies across the scenarios in this study. The Palmer Drought Severity Index (PDSI), a commonly used tool to assess drought in the western United States, forms the basis of this parameter \citep{Cook2004}. The PDSI is appropriate for use at local scales like this section of the Tahoe National Forest, and incorporates precipitation and temperature within a water balance model \citep{HeimJr2002}. I used PDSI data from 2010 to 2099 calculated by \citet{Cook2014} to generate the climate parameter for all future-based simulations. The \citet{Cook2014} modeled PDSI values were fit to observed PDSI values from 1900 to the present, and used the same methodology as the North American Climate Atlas \citep{Cook2004}. Project partners analyzed the suite of climate models for which \citet{Cook2014} had calculated PDSI, and selected the \textsc{ccsm4} from the National Center for Atmospheric Research and the \textsc{gfdl-esm2m} from the NOAA Geophysical Fluid Dynamics Laboratory. The \textsc{ccsm4} model projects warmer temperatures and similar precipitation levels to the past several hundred years, while the \textsc{gfdl-esm2m} model projects hotter and drier weather.

Six PDSI sequences based on the \textsc{ccsm4} model were available, so I treated each run as a separate scenario. To generate climate parameters from the PDSI sequences, I calculated the inverse Euclidean distance-weighted mean of PDSI values at 21 points surrounding the centroid of the study area. I then rescaled the results around the mean and standard deviation of the set of PDSI values representative of the 300 years prior to European settlement, during which the climate was more stable, and characterized by dynamic equilibrium rather than a trend. I used 1 as the neutral value so that the parameter could be used as a multiplier within the model. Thus climate parameter values less than 1 reduce susceptibility, fire starts, and spread, while climate parameter values greater than 1 increase these properties. Each of the seven total runs followed a unique pattern and trend (Figure \ref{fig:pdsi_future}). I present results in order of increasing median value for the climate parameter during the simulations to facilitate interpretation (Figure \ref{pdsi-boxplots}). The \textsc{ccsm-1} model has a distribution of climate parameter values similar to that of the historical period, with a median near 1.


\begin{figure}[!htbp]
\centering
  \subfloat[][]{
    \centering
	\includegraphics[height=0.25\textheight]{/Users/mmallek/Documents/Thesis/Plots/pdsi/futureclimate_wlm.png}
    \label{fig:pdsi-lm}
  }%
  \subfloat[][]{
  	\centering
	\includegraphics[height=0.25\textheight]{/Users/mmallek/Documents/Thesis/Plots/pdsi/future_last5timesteps.png}
	\label{fig:pdsi-final5}
	}
    \caption{(a) Climate parameter trajectory for 18 timesteps used in the simulations for the 6 scenarios from the \textsc{ccsm4} model and the single scenario from the \textsc{gfdl-esm2m} model. Solid lines connect the climate parameter values for each timestep, and the dashed line represents a fitted linear regression to the data. (b) Zoom on the final five timesteps (without regression lines) for better visualization of variability in each scenario. The climate parameter in \textsc{RMLands} is based on the Palmer Drought Severity Index. Models in the legend appear in descending order from least to greatest mean value for the full time series of the simulation. In a historical range of variability analysis, the PDSI values would be centered around a mean value of 1.0.}
\label{fig:pdsi_future}

\end{figure}

\begin{figure}[!htbp]
\centering
\includegraphics[width=0.7\textwidth]{/Users/mmallek/Documents/Thesis/Seminar/frv-climparam-slide.png}
\caption{Boxplots of climate parameter value for the six runs of the \textsc{ccsm4} model and the single run from the \textsc{gfdl-esm2m} model. The climate parameter in \textsc{RMLands} is based on the Palmer Drought Severity Index. Models are arranged left-to-right from least to greatest mean value for the full time series of the simulation, after the HRV. Boxplot whiskers extend from the $5^{\text{th}}-95^{\text{th}}$ range of variability for each model.}
\label{pdsi-boxplots}
\end{figure}


\subsection*{Evaluating Future Range of Variability}
I evaluated the fire regime across the future scenarios by evaluating the median disturbed area for each across severity levels. %\todo{compared to what we'd expect based on proportional difference?}
I also computed the fire rotation for the two most prevalent cover types in the study area, Sierran Mixed Conifer - Mesic and Sierran Mixed Conifer - Xeric, and compared the simulated future scenario results to the historical fire rotation (Figure~\ref{fig:frotation}). Fire rotation is defined as the time it takes to burn an area equivalent to the area under study (either for a particular cover type or the landscape as a whole) \citep{Agee1993}.%
%


%\subsection*{Landscape composition}
My evaluation of landscape composition compares the seral stage distribution for mesic and xeric mixed conifer forests across future scenarios and against the current distribution. I report the 90\% range of variability and assess departure for these two focal cover types across their seral stages. Both cover types include seven seral stages: Early, Mid--Closed, Mid--Moderate, Mid--Open, Late--Closed, Late--Moderate, and Late--Open. The first term refers to the developmental stage and the second to the level of canopy cover (breakpoints for canopy cover are at 40\% and 70\%).


%
%\subsection*{Landscape configuration}
I used \textsc{Fragstats} computer software \citep{Fragstats2012} to conduct the spatial pattern analysis and assess differences in landscape configuration between the current landscape and the future range(s) of variability. However, several of the metrics are redundant with one another, and so I focused on a subset of four metrics to simplify interpretation: Area-weighted mean patch area, area-weighted mean core area, clumpiness index, area-weighted mean shape index. I report the $5^{\text{th}}-95^{\text{th}}$ percentile range of variability using boxplots. %and for SMC?

To assess landscape composition and configuration, I compared the current landscape to the FRV, and report departure based on the following standards. If the current landscape metric value falls within the $25^{\text{th}}-75^{\text{th}}$ percentile range (the box in my boxplots), it is considered not departed. If it falls within the $5^{\text{th}}-25^{\text{th}}$ percentile range or the $75^{\text{th}}-95^{\text{th}}$ percentile range (the whiskers in my boxplots), it is moderately departed. If it falls outside that range, it is completely departed.

%  (\textsc{area\_am}) (\textsc{core\_am}) (\textsc{contag}) (\textsc{gyrate\_am}) (\textsc{shape\_am}) (\textsc{siei})


\subsection*{Methodological Limitations} % brad wanted more in this section, hopefully he just meant explain things.
I acknowledge some limitations that should be understood before applying the results in a management context. \textsc{RMLands} simulates wildfires, but there are many other disturbance processes that exist at varying scales that are not simulated here, including insects and disease, wind-throw, wild ungulate and beaver herbivory, avalanches, and other forms of soil movement. The complex interactions among them that characterize real landscapes are also, as a result, omitted from consideration. %
My input cover type and seral stage data was the best available, but is not perfect because of both human and computer errors in classification, and because it combines three separate classification efforts. Improvements to the existing vegetation layer, such as the LiDAR maps currently in development, would improve the model by offering a standardized classification for the full study area and a more accurate picture of the existing vegetation. If new data became available that would alter the model parameterization, then I would expect the estimate of the future range of variability to change. 

In addition, I did not simulate changes in the spatial configuration of land cover types. That is, I did not model upward movement of forest types or type conversions after fire or drought events. Such change is likely to take place in the next 100 years and has already occurred in many places \citep{Bachelet2001}. Because of this, my results do not specifically represent a likely future outcome, and should be used in conjunction with other studies, especially on range shifts, to predict future vegetation patterns and manage accordingly. 

Finally, the climate parameter used in \textsc{RMLands} is a proxy for climate, not a direct measure of it. It incorporates two key pieces of climate data, temperature and precipitation, that closely relate to wildfire as a disturbance process. It is based on the Palmer Drought Severity Index (PDSI) \citep{HeimJr2002}, but I removed some of the variability and extremes present in the raw PDSI in order to make it compatible with the model. Specifically, I collapsed the projected PDSI data into 5-year summer averages and rescaled it \citep{Cushman2011}. In addition, I utilize a single RCP, 8.5, which is the scenario projecting the most significant change in climate, after \citet{Cook2014}. It represents a ``business as usual'' storyline, and thus may not accurately predict the change to the landscape range of variability in the future if significant action is taken to reduce greenhouse gas outputs and slow the rate of climate change in the next few decades \citep{Riahi2011}.







%%%%%%%%%%%%%%%%%%%%%%%%%%%%%%%%%%%%%%%%%%%%%%%%%%%%%%%%%%%%%%%%%%%%%%%%%%%%%%%%%%%%%%%%%%%%%%%%%%%%%%%%%%%%%%%%%%%%%%%%%%%%%%%%%%%%%%%%%%%%%%%%%%%%%%%%%%%%%%%%%%%%%%%%%%%%%%%%%%%%%%%%%%%%%%%%%%%%%%%%%%%%%%%%%%%%%%%%%%%%%%%%%%%%%%%%%%%%%%%%%%%%%%%%%%%%%%%%%%%%%%%%%%%%%%%%%%%%%%%%%%%%%%%%%%%%%%%%%%%%%%%%%%%%%%%%%%%%%%%%%%%%%%%%%%%%%%%%%%%%%%%%%%%%%%%%%%%%%%%%%%%%%%%%%%%%%%%%%%%%%%%%%%%%%%%%%%%%%%%%%%%%%%%%%%%%%%%%%%%%%%%%%%%%%%%%%%%%%%%%%%%%%%%%%%%%%%%%%%%%%%%%%%%%%%%%%%%%%%%%%%%%%%%%%%%%%%%%%%%%%%%%%%%%%%%%%%%%%%%%%%%%%%%%%%%%%%%%%%%%%%%%%%%%%%%%%%%%%%%%%%%%%%%%%%%%%%%%%%%%%%%%%%%%%%%%%%%%%%%%%%%%%%%%%%%%%%%%%%%%%%%%%%%%%%%%%%%%%%%%%%%%%%%%%%%%%%%%%%
\section{Results}
%* Looks like the absolute range of variability is a lot wider in the future than during the HRV!  \\
%* In some cases the landscape is out of HRV but within most of the FRVs

% brad says figures > table. precise results not necessary
% becky found this confusing
\subsection*{Natural fire regime}

I analyzed the wildfire disturbance regime in terms of its effect on the full study area, on mesic mixed conifer forests, and on xeric mixed conifer forests.

The median area of land burned by wildfire during simulations of seven alternative future climate trajectories generally increased as the climate parameter increased (Figure~\ref{fig:dareacomp}). In general, this trend was strongest for area burned at high mortality, which drove the increase in total area burned, as differences in area burned at low mortality did not appear to be ecologically significant. These observations hold for the full study area as well as the mixed conifer forests alone. More striking is the fact that the area burned at high mortality increased relative to low mortality. Compared to the full landscape, this was slightly less conspicuous for mesic mixed conifer forests. However, in xeric mixed conifer forests, the \textsc{gfdl-esm2m} scenario resulted in similar extents of high versus low mortality.

% (\textsc{smc\_}) (\textsc{smc\_x})

%make these plots to the same scale
\begin{figure}[!htbp]
  \centering
    \subfloat[][]{
	\centering
	\includegraphics[width=0.4\textwidth]{/Users/mmallek/Documents/Thesis/Plots/darea/darea-allfmodels.png}
	\label{fig:darea_modelcomp}
	} \\
  \subfloat[][]{
    \centering
    \includegraphics[width=0.4\textwidth]{/Users/mmallek/Documents/Thesis/Plots/darea/darea-allfmodels-smcm.png}
    \label{fig:dareacomp_smcm}
  } 
  \subfloat[][]{
  \centering
    \includegraphics[width=0.4\textwidth]{/Users/mmallek/Documents/Thesis/Plots/darea/darea-allfmodels-smcx.png}
    \label{fig:dareacomp_smcx}
  }
    \caption{Barplots showing (a) proportion of the full study landscape, (b) proportion of Sierran Mixed Conifer - Mesic, and (c) proportion of Sierran Mixed Conifer - Xeric burned for three mortality levels and across the historical simulation and the seven future climate scenarios. Future scenarios presented in order of increasing median value for climate parameter. From left to right, scenarios are presented in order of increasing median climate parameter value.}
  \label{fig:dareacomp}
\end{figure}

% Fire rotation
I also calculated the fire rotation for each scenario, and ploted these results across all scenarios (Figure~\ref{fig:frotation}). The historical rotation values for both mixed conifer forest types are always within the range of rotations from the seven future climate scenarios. Although there is considerable variability in the results for all scenarios, as the climate parameter values increase, rotation values decrease for high mortality events. As with the disturbed area, changes to the low mortality rotations are slight, but seem to increase when high mortality rotation decreases more, such that the ``any mortality'' values decline is modest.



\begin{figure}
\centering
\includegraphics[width=0.6\textwidth]{/Users/mmallek/Documents/Thesis/Plots/rotation/rotation_all.png}
\caption{Fire rotation values across scenarios for the full extent of the study area (red), Sierran Mixed Conifer - Mesic (green), and Sierran Mixed Conifer - Xeric (blue), during the last five timesteps of the simulations. The ``Historical'' values are the target values representing the pre-European settlement period fire rotations, which were used in initial model calibration. Point shapes correspond to different mortality levels from fire: low mortality (circles), high mortality (squares), and overall mortality (both high and low combined, triangles). Connecting lines have been included to aid in finding cover and mortality values across scenarios.}
\label{fig:frotation}
\end{figure}



\subsection*{Landscape Pattern}

\paragraph*{Seral Stage Distribution}
My landscape pattern analysis focuses first on changes to the seral stage distribution of mesic and xeric mixed conifer forests. Evidence of both high mortality fire, which triggers a transition to the early seral stage for all cover types, and low mortality fire, which can thin a stand and cause a transition to a more open canopy condition (within the middle or late developmental stages), is visible in examining the output grids.

In order to determine whether the results were an artifact of the initial condition, I also examined the trajectory of all seral stages for both mesic and xeric mixed conifer forests (Figures~\ref{fig:median_trajectory_smcm} and \ref{fig:median_trajectory_smcx}). Because the initial condition for both forest types included a large amount of land in the Late Development and closed canopy types, which are generally parameterized to be more susceptible to fire than other seral stages, it seemed plausible that these initial conditions could have a disproportionate impact on the trajectory and the range of variability, as displayed in Figures~\ref{fig:covcond_smcm} and \ref{fig:covcond_smcx}. This could occur if the higher susceptibility resulted in high rates of wildfire in addition to high rates of high mortality from wildfire. In particular, this could generate a large amount of land in the Early Development stage that would then dominate the cover type composition for the remainder of the simulation. However, Figures~\ref{fig:median_trajectory_smcm} and \ref{fig:median_trajectory_smcx}, which depict the median values across all runs of each climate model scenario, and for each seral stage of mesic and xeric mixed conifer forests, demonstrate that this is not in fact the case. 

As can be seen in the trajectory plots of the early successional stage, the anticipated increase does occur in both forest types. However, in mesic mixed conifer the proportion dips back down, and appears to be responding more to the climate parameter than the initial conditions, particularly by timestep 14, which is the point at which I begin using the output landscape grids for analysis. In xeric mixed conifer, the initial increase continues throughout the length of the simulation, again indicating that the driver behind this increase is related primarily to the climate setting and related feedback, rather than the initial conditions. In the same way, if the preceding hypothesis were true, I would expect to see a fairly stable distribution once the first few timesteps had passed. Instead, for most seral stages, including the Late--Closed stage, I observe a clear trend. This trend suggests the climate parameter trend, and appears stronger than the influence of the initial conditions. Furthermore, my analysis and conclusions are focused on the end of the simulation. I did this in part to avoid incorporating artifacts related to initial conditions. These plots show that the shift in seral stage proportions over time follow trends, rather than oscillate in equilibrium, and thus the short length of the simulation (compared to a multi-century HRV) is not a main factor in the decline of middle and late successional forest observed in the simulations.

\begin{figure}[htbp]
 \captionsetup[subfigure]{labelformat=empty}
  \centering
  \subfloat[][]{
    \centering
    \includegraphics[height=0.33\textwidth]{/Users/mmallek/Documents/Thesis/Plots/seralstage-trajectory-medians/2410-trajectory-median-legend.png}
  }\\%
  %\qquad
  \subfloat[][]{
    \includegraphics[width=0.33\textwidth]{/Users/mmallek/Documents/Thesis/Plots/seralstage-trajectory-medians/2420-trajectory-median-title.png}
  } 
    \subfloat[][]{
    \centering
    \includegraphics[width=0.33\textwidth]{/Users/mmallek/Documents/Thesis/Plots/seralstage-trajectory-medians/2421-trajectory-median-title.png}
  }%
  %\qquad
  \subfloat[][]{
    \includegraphics[width=0.33\textwidth]{/Users/mmallek/Documents/Thesis/Plots/seralstage-trajectory-medians/2422-trajectory-median-title.png}
  } \\
    \subfloat[][]{
    \centering
    \includegraphics[width=0.33\textwidth]{/Users/mmallek/Documents/Thesis/Plots/seralstage-trajectory-medians/2430-trajectory-median-title.png}
  }%
      \subfloat[][]{
    \centering
    \includegraphics[width=0.33\textwidth]{/Users/mmallek/Documents/Thesis/Plots/seralstage-trajectory-medians/2431-trajectory-median-title.png}
  } 
  \subfloat[][]{
    \includegraphics[width=0.33\textwidth]{/Users/mmallek/Documents/Thesis/Plots/seralstage-trajectory-medians/2432-trajectory-median-title.png}
  }
    \caption{Median trajectory across all climate scenarios and seral stages, for mesic mixed conifer forests. Each climate scenario is shown in a different color. After a noisy beginning, a trend emerges for most seral stages by the final timesteps of the simulation.}
  \label{fig:median_trajectory_smcm}
\end{figure} 

\begin{figure}[htbp]
 \captionsetup[subfigure]{labelformat=empty}
  \centering
  \subfloat[][]{
    \centering
    \includegraphics[height=0.33\textwidth]{/Users/mmallek/Documents/Thesis/Plots/seralstage-trajectory-medians/2610-trajectory-median-legend.png}
  }\\%
  %\qquad
  \subfloat[][]{
    \includegraphics[width=0.33\textwidth]{/Users/mmallek/Documents/Thesis/Plots/seralstage-trajectory-medians/2620-trajectory-median-title.png}
  } 
    \subfloat[][]{
    \centering
    \includegraphics[width=0.33\textwidth]{/Users/mmallek/Documents/Thesis/Plots/seralstage-trajectory-medians/2621-trajectory-median-title.png}
  }%
  %\qquad
  \subfloat[][]{
    \includegraphics[width=0.33\textwidth]{/Users/mmallek/Documents/Thesis/Plots/seralstage-trajectory-medians/2622-trajectory-median-title.png}
  } \\
    \subfloat[][]{
    \centering
    \includegraphics[width=0.33\textwidth]{/Users/mmallek/Documents/Thesis/Plots/seralstage-trajectory-medians/2630-trajectory-median-title.png}
  }%
      \subfloat[][]{
    \centering
    \includegraphics[width=0.33\textwidth]{/Users/mmallek/Documents/Thesis/Plots/seralstage-trajectory-medians/2631-trajectory-median-title.png}
  } 
  \subfloat[][]{
    \includegraphics[width=0.33\textwidth]{/Users/mmallek/Documents/Thesis/Plots/seralstage-trajectory-medians/2632-trajectory-median-title.png}
  }
    \caption{Median trajectory across all climate scenarios and seral stages, for xeric mixed conifer forests. Each climate scenario is shown in a different color. After a noisy beginning, a trend emerges for most seral stages by the final timesteps of the simulation.}
  \label{fig:median_trajectory_smcx}
\end{figure} %smcx

I observe clear trends in three seral stages across both cover types (Figures~\ref{fig:covcond_smcm}-\ref{fig:covcond_smcx}). The proportion of Early Development increased in both with increasing climate parameter values, while Late--Closed and Late--Moderate both decreased. Surprisingly, as the climate parameter increased, the proportion of Mid--Open in the mesic mixed conifer forest increased, while in the xeric mixed conifer the proportion of Mid--Open decreased. In general, the proportion of the current landscape in each seral stage differed substantially from the future ranges of variability. Across all the seral stages, the proportion of each cover type in the early seral stage increased most dramatically. I focus the configuration metrics analysis, then, on the Early Development stage of Sierran Mixed Conifer - Mesic and Sierran Mixed Conifer - Xeric.

%or could write as focused on one cover type at a time. 
In the mesic mixed conifer cover type, as the climate parameter increased, the current seral stage distribution shifted from falling within the projected future range of variability to falling outside it. The larger increase in the proportion of the landscape in the Early Development and Mid--Open stages comes at the expense of all the Late Development stages, which decline with increasing climate parameter ranges.

In the xeric mixed conifer type, the results were even more dramatic. As Figure~\ref{fig:covcond_smcx} shows, I observed a substantial increase in the proportion of Early Development, while all other stages showed a declining trend. After Early Development, the Mid-Open and Late--Open stages were the next most common, and were fairly prevalent on the landscape. Thus I observed complete departure of the current landscape from each seral stage, across all scenarios.

\begin{figure}[htbp]
 \captionsetup[subfigure]{labelformat=empty}
  \centering
  \subfloat[][]{
    \centering
    \includegraphics[width=0.5\textwidth]{/Users/mmallek/Documents/Thesis/Plots/covcond-byscenario/2410-boxplots.png}
  }%
  %\qquad
  \subfloat[][]{
    \includegraphics[width=0.5\textwidth]{/Users/mmallek/Documents/Thesis/Plots/covcond-byscenario/2420-boxplots.png}
  } \\
    \subfloat[][]{
    \centering
    \includegraphics[width=0.5\textwidth]{/Users/mmallek/Documents/Thesis/Plots/covcond-byscenario/2421-boxplots.png}
  }%
  %\qquad
  \subfloat[][]{
    \includegraphics[width=0.5\textwidth]{/Users/mmallek/Documents/Thesis/Plots/covcond-byscenario/2422-boxplots.png}
  } \\
    \subfloat[][]{
    \centering
    \includegraphics[width=0.5\textwidth]{/Users/mmallek/Documents/Thesis/Plots/covcond-byscenario/2430-boxplots.png}
  }%
  %\qquad
    \subfloat[][]{
    \centering
    \includegraphics[width=0.5\textwidth]{/Users/mmallek/Documents/Thesis/Plots/covcond-byscenario/2431-boxplots.png}
  } \\
  \subfloat[][]{
    \includegraphics[width=0.5\textwidth]{/Users/mmallek/Documents/Thesis/Plots/covcond-byscenario/2432-boxplots.png}
  }
    %\qquad
  %\subfloat[][]{
  %  \includegraphics[width=0.5\textwidth]{/Users/mmallek/Documents/Thesis/Plots/covcond-frvhrv/SMCM-frvhrv-boxplots.png}
  %}
  \caption{Boxplots illustrating the range of variability across future climate trajectories for Sierran Mixed Conifer - Mesic. The dashed black horizontal line represents the current condition. Boxplot whiskers extend from the $5^{\text{th}} - 95^{\text{th}}$ range of variability for each model. Climate models appear left-to-right in order of increasing median climate parameter value.}
  \label{fig:covcond_smcm}
\end{figure} %smcm

\begin{figure}[htbp]
 \captionsetup[subfigure]{labelformat=empty}
  \centering
  \subfloat[][]{
    \centering
    \includegraphics[width=0.5\textwidth]{/Users/mmallek/Documents/Thesis/Plots/covcond-byscenario/2610-boxplots.png}
  }%
  %\qquad
  \subfloat[][]{
    \includegraphics[width=0.5\textwidth]{/Users/mmallek/Documents/Thesis/Plots/covcond-byscenario/2620-boxplots.png}
  } \\
    \subfloat[][]{
    \centering
    \includegraphics[width=0.5\textwidth]{/Users/mmallek/Documents/Thesis/Plots/covcond-byscenario/2621-boxplots.png}
  }%
  %\qquad
  \subfloat[][]{
    \includegraphics[width=0.5\textwidth]{/Users/mmallek/Documents/Thesis/Plots/covcond-byscenario/2622-boxplots.png}
  } \\
    \subfloat[][]{
    \centering
    \includegraphics[width=0.5\textwidth]{/Users/mmallek/Documents/Thesis/Plots/covcond-byscenario/2630-boxplots.png}
  }%
      \subfloat[][]{
    \centering
    \includegraphics[width=0.5\textwidth]{/Users/mmallek/Documents/Thesis/Plots/covcond-byscenario/2631-boxplots.png}
  } \\
  \subfloat[][]{
    \includegraphics[width=0.5\textwidth]{/Users/mmallek/Documents/Thesis/Plots/covcond-byscenario/2632-boxplots.png}
  }
    %\qquad
  %\subfloat[][]{
  %  \includegraphics[width=0.5\textwidth]{/Users/mmallek/Documents/Thesis/Plots/covcond-frvhrv/SMCX-frvhrv-boxplots.png}
  %}
    \caption{Boxplots illustrating the range of variability across future climate trajectories for Sierran Mixed Conifer - Xeric. The dashed black horizontal line represents the current condition. Boxplot whiskers extend from the $5^{\text{th}} - 95^{\text{th}}$ range of variability for each model. Climate models appear left-to-right in order of increasing median climate parameter value.}
  \label{fig:covcond_smcx}
\end{figure} %smcx



% debparture categories
%not departed - boxes completely overlap/contain each other
%slightly departed - medians don't overlap the boxes, but boxes overlap
%moderately departed - box overlaps whiskers
%highly departed - only whiskers overlap
%completely departed - no overlap of full rv

% include any others?
% what about total edge?

\paragraph*{Early Seral Patch Configuration}
Results for the configuration metrics associated with the Early Development seral stage of both Sierran Mixed Conifer - Mesic and Sierran Mixed Conifer - Xeric indicate that the current landscape is completely departed from the future ranges of variability across climate scenarios (Figures~\ref{fig:fragclass-smcm} and \ref{fig:fragclass-smcx}. The exception to this observation is that the mean patch size and mean core area size results in the xeric mixed conifer forests overlapped the whiskers of the boxplot results in all scenarios except the \textsc{ccsm3} and \textsc{gfdl-esm2m}, indicating moderate departure. In no case did I find that the current landscape was within the range of variability of the simulated future scenarios. In both mesic and xeric mixed conifer forests, mean patch size, mean core area size, and mean shape index (area-weighted, in all cases), increased with increasing climate parameter values. The trend is stronger in xeric forests, with the biggest difference apparent in the results for the \textsc{gfdl-esm2m} scenario. Again in the case of both mesic and xeric variants, the level of fragmentation is completely departed from the current landscape. However, no trend appears with respect to the climate parameter. Thus, patches of early seral mixed conifer forests were large, contained more core area, featured more complex shapes, and were less fragmented than patches on the present-day landscape.


\begin{figure}[!htbp]
 \captionsetup[subfigure]{labelformat=empty}
  \centering
  \subfloat[][]{
    \centering
    \includegraphics[width=0.5\textwidth]{/Users/mmallek/Documents/Thesis/Plots/fragclass-smcmetrics/SMC_M_EARLY_ALL_AREA_AM_boxplots.png}
    \label{fig:boxplot-class-smcm-areaam}
  }%
  %\qquad
  \subfloat[][]{
    \includegraphics[width=0.5\textwidth]{/Users/mmallek/Documents/Thesis/Plots/fragclass-smcmetrics/SMC_M_EARLY_ALL_CLUMPY_boxplots.png}
    \label{fig:boxplot-class-smcm-contag}
  } \\
    \subfloat[][]{
    \includegraphics[width=0.5\textwidth]{/Users/mmallek/Documents/Thesis/Plots/fragclass-smcmetrics/SMC_M_EARLY_ALL_CORE_AM_boxplots.png}
    \label{fig:boxplot-class-smcm-coream}
  }
    %\qquad
    \subfloat[][]{
    \includegraphics[width=0.5\textwidth]{/Users/mmallek/Documents/Thesis/Plots/fragclass-smcmetrics/SMC_M_EARLY_ALL_SHAPE_AM_boxplots.png}
    \label{fig:boxplot-class-smcm-shapeam}
} %\\
  %  \subfloat[][]{
  %  \includegraphics[width=0.5\textwidth]{/Users/mmallek/Documents/Thesis/Plots/fragclass-smcmetrics/SMC_M_EARLY_ALL_ECON_AM_boxplots.png}
  %  \label{fig:boxplot-class-smcm-econam}
  %}
    %\qquad
  %  \subfloat[][]{
  %  \includegraphics[width=0.5\textwidth]{/Users/mmallek/Documents/Thesis/Plots/fragclass-smcmetrics/SMC_M_EARLY_ALL_AI_boxplots.png}
  %  \label{fig:boxplot-class-smcm-ai}
  %}
 \caption{Boxplots illustrating the range of variability in Sierran Mixed Conifer - Mesic, Early Development across future climate trajectories. The dashed black bar represents the current condition. Boxplot whiskers extend from the $5^{\text{th}} - 95^{\text{th}}$ percentile range of variability for each model.}
  \label{fig:fragclass-smcm}
\end{figure} %fragland

\begin{figure}[!htbp]
 \captionsetup[subfigure]{labelformat=empty}
  \centering
  \subfloat[][]{
    \centering
    \includegraphics[width=0.5\textwidth]{/Users/mmallek/Documents/Thesis/Plots/fragclass-smcmetrics/SMC_X_EARLY_ALL_AREA_AM_boxplots.png}
    \label{fig:boxplot-class-smcx-areaam}
  }%
  %\qquad
  \subfloat[][]{
    \includegraphics[width=0.5\textwidth]{/Users/mmallek/Documents/Thesis/Plots/fragclass-smcmetrics/SMC_X_EARLY_ALL_CLUMPY_boxplots.png}
    \label{fig:boxplot-class-smcx-contag}
  } \\
    \subfloat[][]{
    \includegraphics[width=0.5\textwidth]{/Users/mmallek/Documents/Thesis/Plots/fragclass-smcmetrics/SMC_X_EARLY_ALL_CORE_AM_boxplots.png}
    \label{fig:boxplot-class-smcx-coream}
  }
    %\qquad
    \subfloat[][]{
    \includegraphics[width=0.5\textwidth]{/Users/mmallek/Documents/Thesis/Plots/fragclass-smcmetrics/SMC_X_EARLY_ALL_SHAPE_AM_boxplots.png}
    \label{fig:boxplot-class-smcx-shapeam}
} %\\
  %  \subfloat[][]{
  %  \includegraphics[width=0.5\textwidth]{/Users/mmallek/Documents/Thesis/Plots/fragclasssmcmetrics/SMC_X_EARLY_ALL_ECON_AM_boxplots.png}
  %  \label{fig:boxplot-class-smcx-econam}
  %}
    %\qquad
  %  \subfloat[][]{
  %  \includegraphics[width=0.5\textwidth]{/Users/mmallek/Documents/Thesis/Plots/fragclasssmcmetrics/SMC_X_EARLY_ALL_AI_boxplots.png}
  %  \label{fig:boxplot-class-smcx-ai}
  %}
    \caption{Boxplots illustrating the range of variability in Sierran Mixed Conifer - Xeric, EarlyDevelopment across future climate trajectories. The dashed black bar represents the current condition.Boxplot whiskers extend from the $5^{\text{th}} - 95^{\text{th}}$ percentile range of variability foreach model.}
  \label{fig:fragclass-smcx}
\end{figure} %fragland




\clearpage







%%%%%%%%%%%%%%%%%%%%%%%%%%%%%%%%%%%%%%%%%%%%%%%%%%%%%%%%%%%%%%%%%%%%%%%%%%%%%%%%%%%%%%%%%%%%%%
%%%%%%%%%%%%%%%%%%%%%%%%%%%%%%%%%%%%%%%%%%%%%%%%%%%%%%%%%%%%%%%%%%%%%%%%%%%%%%%%%%%%%%%%%%%%%%
%%%%%%%%%%%%%%%%%%%%%%%%%%%%%%%%%%%%%%%%%%%%%%%%%%%%%%%%%%%%%%%%%%%%%%%%%%%%%%%%%%%%%%%%%%%%%%
%%%%%%%%%%%%%%%%%%%%%%%%%%%%%%%%%%%%%%%%%%%%%%%%%%%%%%%%%%%%%%%%%%%%%%%%%%%%%%%%%%%%%%%%%%%%%%
%%%%%%%%%%%%%%%%%%%%%%%%%%%%%%%%%%%%%%%%%%%%%%%%%%%%%%%%%%%%%%%%%%%%%%%%%%%%%%%%%%%%%%%%%%%%%%
%%%%%%%%%%%%%%%%%%%%%%%%%%%%%%%%%%%%%%%%%%%%%%%%%%%%%%%%%%%%%%%%%%%%%%%%%%%%%%%%%%%%%%%%%%%%%%
%%%%%%%%%%%%%%%%%%%%%%%%%%%%%%%%%%%%%%%%%%%%%%%%%%%%%%%%%%%%%%%%%%%%%%%%%%%%%%%%%%%%%%%%%%%%%%


\section{Discussion}


\subsection{Future landscape dynamics of forests in the Yuba River watershed and comparison to current conditions}

I observed a slight to moderate increase in total area burned per timestep with increasing climate parameter sets. Although on its own this might indicate that northern Sierra Nevada forests are resilient to climate change and have stable outcomes, the total burned area results subsume the important finding that high severity fire dramatically increased relative to low severity fire, especially as the climate parameter increased (Figure~\ref{fig:dareacomp}). This finding was more pronounced for the xeric mixed conifer forests than for the mesic mixed conifer forests or the landscape as a whole, which was surprising because I expected the xeric forests to be resilient to an increase in high severity fire because of the ubiquity of low severity fire. Specifically, I found that results for xeric mixed conifer forests from the \textsc{ccsm1} model, which has a distribution of climate parameter values similar to presettlement conditions (although still exhibiting a trend of increasing temperatures), were a ratio of low to high mortality fire of about 2.6; results from the \textsc{gfdl-esm2m} model were a ratio of about 1.0. My analysis of fire rotation confirms and illustrates this (Figure~\ref{fig:frotation}), making more clear the slight decrease in low mortality fire evident in results for the more extreme climate parameter sets.

From this result I expected to observe changes in the seral stage distribution, especially for xeric forests. I did, and they were dramatic. The trend of increasing fire and the trends in seral stage distribution have sharp jumps in the first few timesteps in most of the cover type-seral stage plots (Figures~\ref{fig:median_trajectory_smcm} and \ref{fig:median_trajectory_smcx}). This can be considered an artifact of the initial conditions, as it is a result that can be traced directly to them. However, I also point out that the starting conditions are those from 2010. Large fires have been recorded elsewhere in the Sierra Nevada in the past five years, and there is a trend in the last 30 years of increasing fire \citep{Miller2012}, suggesting that this model result is not far-fetched simply because it projects a lot of fire in the short-term.

The proportion of Early Development patches was strongly correlated with an increase in the climate parameter. In addition, open canopies became more prevalent and closed became less prevalent, which in my model can be attributed to an increase in fire. In the mesic mixed conifer forests, this decline was a clear trend following increasing sets of climate parameter values, and losses of Late--Closed forest were correlated with gains in Early Development conditions. This implies that the decline in prevalence of late successional conditions in mesic mixed conifer forests is likely due to the increased amount of fires that result in high mortality, which precludes many cells from succeeding to a Late Development stage without experiencing a high severity fire. The complete departure of Late--Open across all scenarios compared to the current conditions may be due to the fact that fire exclusion has already reduced the amount of Late--Open on the landscape. In the xeric mixed conifer forests, seral stages other than Early Development, Mid--Open, and Late--Open were virtually absent. This makes sense because more fire, and especially more fire with high mortality effects, would be likely to produce such a seral stage distribution. Again, cells would be so frequently affected by fire that succession to closed canopy conditions becomes statistically unlikely.

%In my model, early seral conditions are always created by high mortality fire, and either chaparral or trees may after a cell experiences fire. Chaparral is often treated as its own cover type. This is in part due to the fact that chaparral establishment tends to delay the succession process to forest. When studies occur along short temporal scales, it is often more meaningful to categorize these two vegetation communities as separate. However, my study is focused on long temporal scales, and within the study area chaparral will eventually succeed to trees. Consequently, both communities are considered early development. 
With that said, because more early seral is projected in the future, even under fire suppression, examining patch behavior in a range of variability framework can provide insights into how to manage it. I observed an effect of the increased amount of fire, in both extent and severity, on configuration metrics at the seral stage level. I focused on the Early Development stage in mesic and xeric mixed conifer forests because it experienced such a dramatic increase as the climate parameter increased and because the Forest Service is actively developing management practices for early seral habitats. 

I focused my configuration analysis on asking how big, how complex, and how fragmented early seral patches were, as well as how much core area they contained, during the simulated period. In general the results diverge from the current conditions; in most cases the current condition is fully departed from the 90\% range of variability in the future scenarios. Compared to the current landscape, early seral patches under the simulated future scenarios were larger, had larger core areas, were less fragmented, were more irregularly shaped, and had less edge contrast. I observe that the early xeric forests exhibit a stronger trend than the early mesic. Increasing the climate parameter has a more subtle effect on the configuration metrics compared to the seral stage distribution.

These results imply that current trends of increased amounts of high mortality fire relative to low mortality fire are related to climate, and may be difficult to reverse. I observed a loss of structural diversity (compared to current conditions) within the xeric forests, which shifted to a distribution composed almost entirely of Early Development, Mid--Open, and Late--Open forest. Mesic forests contained more structural complexity, but a large increase in Early Development comes at the loss of the Late--Closed and Late--Moderate stages, which is problematic because this cover type is a major source of late successional, closed canopy forest for the study area.

%Because this study relied on the use of computer models, the most appropriate use of the results is to help identify the most influential factors driving landscape change, implications of our simulated disturbance and succession regime, and areas where further research is needed to delineate key parameters. 

%***From powerpoint
%Keeping in mind that they are the outcome of modeling passive management of wildfire and forests. Generally, I observed more fire and more high mortality outcomes from fire, which increases the proportion of early seral and open canopy forest. Other seral stages decline. The resulting early seral patches are larger and have more complex shapes.

%In particular, in the disturbance regime, I observed that: Wildfires were frequent in all simulations, but I still recorded large fires with high proportions of high severity. This makes sense given the known, and specified relationship between drought and wildfire mortality. Thus, frequent fire alone does not necessarily provide increased resistance to high severity fire under more frequent or intense drought in the future. 

%(Seral Stage Distribution) Wide range of future variability in each scenario. Indicates that stability and predictability not likely to be characteristics of future forest and wildfire patterns and that our ability to predict when and where large fires will occur will be constrained by large uncertainty. Need to manage public, agency expectations

%(Early seral patch Configuration) My results were consistent among scenarios, showing the current landscape departs from the RV. Indicates that allowing fires to burn naturally is bigger factor than climate



\subsection{Historic range of variability}
Prior to this study I completed a historical range of variability (HRV) analysis for the same study area (Chapter~\ref{ch:hrv}). The HRV analysis used the same parameter set as the FRV, but used climate parameters from the pre-European settlement period of 1550--1850, as a reference period \citep{Safford2013}. The combination of results depicting the relationship between the current landscape, HRV, and FRV provides a mechanism for land managers to identify and prioritize management strategies for promoting resilient forests. In a preliminary comparison of the seral stage distribution results from the HRV and FRV simulations, I observed that current conditions are generally departed from both the simulated ranges of variability (see Appendix~\ref{app:futurecovcond}). Often the HRV and FRV are comparable. In these cases, restoration toward the conditions represented by the range of variability analysis results should be evaluated for practicality, with specific implementation being done at a site-specific basis using additional local data to inform specific management actions.

\subsection{Insect Disturbance}
Forest Service scientists are actively researching insect outbreaks that affect western forests, including those in the Sierra Nevada \citep{Liebhold2011}. Unlike wildfire, which is a physical process that is fundamentally the same everywhere, even though its effects are incredibly diverse, insects are a biological agent. Thus the life history characteristics of insects, the method individual species use to invade trees and reproduce, and a tree's response to this invasion result in a much more complex disturbance ecology than that of wildfire. To examine even the effects of bark beetles, a specific type of insects, is a complex undertaking \citep{Fettig2007}. The effects of climate change, especially increased temperatures and decreased precipitation, may enhance the invasion potential for some insect species in some locations, but may also inhibit it \citep{Logan2003,Bentz2010}. If the influence of insect disturbance increases in the study area during this century, there are sure to be interactive effects with wildfire that would influence reality and potentially reduce the predictive power of my results \citep{Ferrell1996}.


\subsection{Using results for management at various scales}
It is important to understand that one limitation of this study is that it was not designed to address questions below the landscape level. While it may be tempting for managers to view subregions of the study area through the lens of the results, it would not be automatically appropriate to set management targets to achieve, for example, proportions of cover type seral stages at a local scale such as a project analysis area that mirror those that characterize the FRV. Any management targets set using the results of this study should be measured at the full landscape level of this study. In addition, my results are organized by non-flexible boundaries such as the watershed and the area assigned to each cover type. Consequently it would not make sense to target a certain proportion of the landscape to be a particular cover type, since that was not modeled in the simulation. Furthermore, the results rely on the generation and analysis of a large quantity of data. When the scale of analysis is reduced, so is the quantity of data produced by the model, and with it, my confidence in the statistical validity of the results and their implications. 

To be more specific, I outline a simple example. The Tahoe National Forest could draw a boundary over an area identified as being appropriate for fire restoration. It would not be appropriate to assume that because across the full study area, late development, open canopy xeric mixed conifer forests ranged from about 25\% to 35\% of the total xeric mixed conifer forest, that within the newly identified project area, similar proportions should prevail. Many other variables should be taken into account, including environmental (topographic position) and social (wildland-urban interface) considerations. My study does not speak to an appropriate management target for any area smaller than the full study area. That said, managers could draw up a target increase or decrease in extent of a particular seral stage for xeric mixed conifer forests at the project level. This project-level information could then be used to consider how implementing the project would move the landscape as a whole toward or away from the FRV. 

Put another way, these results should no more be unquestioningly applied at the project level than the model should be considered a fire behavior model. It has been optimized to perform well statistically when used to examine landscape change over large extents and long time periods. While spatially explicit in its dynamics, it is a simulation, producing many potential outcomes, and is not intended to predict specific outcomes at specific places and points in time. Rather, the data produced are aggregated in order to develop an understanding of the area as a whole. Because this model should not be used to predict an actual fire rotation value for a particular point on the landscape, it should also not be used to predict the effect of a particular vegetation management strategy \emph{at a particular point on the landscape.} Other forest and fire simulation models are designed for this purpose. \textsc{RMLands} is designed to produce outputs that facilitate measuring the effect of vegetation management implemented across space and time, which in turn enables assessing and understanding the potential impacts of such actions at scales larger than is practical or possible to measure experimentally.  A more appropriate use of the results would be to run a similar analysis on another similar sized and ecologically similar area and compare the two. Another potential future study could assess the influence of different management strategies. This study only explored the no management alternative, but exploring the impact of different levels of fire suppression or the addition of vegetation treatments could also yield insights not otherwise obtainable through local-scale modeling.


\subsection{Implications for Restoration}

Restoration toward a resilient, somewhat stable ecology is often an important goal of resource managers. The comparisons made here consider the difference between the current and future disturbance regimes and landscape patterns, given a scenario in which natural fire regimes were allowed to occur. Since letting all fires burn naturally is not practicable, I note that the results do not provide a simple roadmap for restoration. However, they should provide insights into what landscape patterns may be resilient with climate change. In addition, the results indicate when restoration toward a historical regime, composition, or configuration appears likely to succeed under climate change, or whether the future is likely to significantly diverge from the past. 

One clear result of this study is that more fire occurs under natural conditions and future climate scenarios than under current and presumed historic conditions. The implications are therefore that a shift towards more fire and especially more high severity fire is likely. Under a suppression-based strategy, there is no reason to believe that future fires would be less severe than those modeled here. I did not observe a plateau in landscape change by 2100, indicating that a stable ecology in terms of fire is not likely in the near-term. If restoration processes take many decades, stability should not be expected to return until restoration is completed, at the earliest. Ongoing trends in climate change will inhibit efforts to oppose analogous trends in forest disturbances. The wide observed range of variability further strengthens my conclusion that managers will have difficulty in maintaining a stable and predictable relationship between fire and the landscape. Large uncertainties in projections will be the norm for some time under a changing climate. 

In addition, given that both the frequency and incidence of high mortality increased under increasingly severe climate conditions, I conclude that restoring frequent fire alone should not be assumed to eliminate the risk of high severity fire (although it ought to reduce it) under more frequent or intense drought in the future. Moreover, my results were consistent among scenarios, indicating that allowing fires to burn naturally may be at least as large, if not larger, a factor than climate change on its own. These results point to a need for restoration of more fires that extend across much larger areas, many of which reset part of the burned area to an early seral condition. Such restoration is unlikely to be politically or socially popular. As a result, public expectations and the capabilities of agencies will need to be carefully managed and articulated. 

Given my findings that seral stage distributions and patch configuration metrics are changing in response to changing climate, it is important to recognize that species will be affected differently based on their particular habitat requirements. Thus, using the future range of variability results as a restoration target has implications for wildlife species that may not be socially or legally acceptable. Species that rely on early successional conditions or open canopies will gain habitat in the future, thanks to the increase in the prevalence of these conditions. However, the associated reduction in late successional conditions, especially the more closed canopies currently characteristic of mesic mixed conifer forests, will likely have negative ramifications for species dependent on that structural context, such as spotted owl and fisher. Consequently, as managers and decision-makers designate which areas to restore and to what conditions, they will encounter real, unavoidable tradeoffs.

Opportunities to restore forests to patch configurations observed in the simulation may arise when designing timber harvest or fuels treatment projects. However, I caution that patch configurations under natural regimes may be challenging to re-create manually, especially given constraints surrounding the use of fire. For example, it is common to use linear features such as roads and streams as boundaries for treatment. However, such strategies may inhibit managers' ability to can achieve patch configurations similar to those that would occur naturally\todo{all i wrote was "so what can they do"}.

As just mentioned, my results can be used to identify and prioritize management strategies. Since the Forest Service is directed to manage within the natural range of variability, it is important to define what the natural range of variability is, as well as the desired condition, and recognize what is possible. Some of the trends observed in this analysis, such as increased proportions of Early Development, will likely occur without active management. Others will not occur without changes to fire suppression strategies and related fuels management efforts. Once desired conditions are defined, a comparison to the FRV from this study is warranted. However, this is not to assign a value judgement to achieving a seral stage distribution within the FRV. Rather, the restoration of the natural process (occurrence and effects) of wildfire is a high-level goal, and future seral stage distrbutions are one of several metrics that can be used to evaluate whether efforts to restore wildfire are successful. Whether or not the desired condition aligns with the FRV will affect what kinds of management actions are taken. Based soley on my results, a proactive approach to restoration would steer xeric forests toward more open conditions now. This should benefit both the ecosystem and the public.

One limitation of the model is that it did not predict a longer burning season. It may be possible to counteract some of the high severity fire occurrence by injecting more low severity fire through prescribed burning during the shoulder seasons and winter. This could dampen the effects of warmer and drier conditions. Variability density treatments designed to create fine-scale heterogeneity could also ameliorate some of the effects of more frequent fire \citep{Stephens2010,Knapp2012,North2012a}. The vulnerability of closed canopy forest should be an impetus to focus initial restoration and prescribed burning efforts in areas where these values can be protected. Of course, increased amounts of fire will pose challenges in a landscape with complex ownership patterns \citep{Stephens2013}. Excellent coordination at levels beyond the Tahoe National Forest or the Forest Service will be necessary to address this successfully. Federal land management agencies, the California Department of Forestry and Fire Protection (``CalFire''), and local governments are working together more and more to coordinate fire suppression of large wildfires that cross political boundaries.

Assessments of the effects of post-fire treatments of various kinds need to be continued and expanded in a systematic and rigorous way. Stand initiation patterns of the past, and the meteorological condition under which they took place, should be evaluated to understand whether similar conditions are likely in the future. Some researchers have called for special management of early seral habitat. For example, \citet{Dellasala2014} call for not managing post-disturbance early seral vegetation. \citet{Swanson2011} suggest mapping and managing early seral communities as a unique cover type. However, given the high probability of more early successional habitat being created by fire and the realities of federal budgets and personnel needs, it seems unlikely that special attention is warranted to ensure that some newly created patches of early seral habitat are allowed to succeed naturally. In addition, mapping and managing a transient cover type does not fit within most forest management frameworks, including ours; I assert that early seral forest is a condition within a cover type. Practicing no management at all under climate change is risky, given that current species assemblages may need help to re-establish, and the very real threat of invasive species \citep{Stephens2010}. That said, the large amount of early seral habitat projected to occur on the landscape also means that managers will have options when deciding where to implement restoration efforts that are designed to speed up succession or reduce susceptibility to subsequent burns; it is here that more research on the response to Burned Area Emergency Response and restoration treatments would be most useful.



\subsection{Implications for Planning}
Many National Forests will undergo Forest Plan revisions in the next decade. The 2012 Planning Rule instructs managers to manage for resilient conditions within a natural range of variability. While my results may be used as one potential range of variability, the fact is that my results are also the outcome of a model that simplified several aspects of the landscape, including ownership patterns and tolerable amounts of fire. For example, I did not model varying levels of fire suppression effort. Assuming the Forest Service and other stakeholders in the area do not view the amount of wildfire I simulated as tolerable, they will need to adopt various strategies to try and reduce the likelihood of frequent large fires, as well as the negative impacts of suppression. 

Overall, the results of my study strongly indicate that more time, personnel, and funding will be needed in the future to fight fire, and subsequently repair and restore the damage from firefighting efforts and the flames themselves. I assert that there is a critical need to define desired future conditions for forests, and compare them to what would naturally develop under the simulated conditions, which are intended to portray the future range of variability for the landscape under stuy. I believe it is likely that there will be a gap between the projected conditions and the desired future conditions. Clearly, a ``no-management'' strategy of fire and forests is not compatible with human settlement in the region: before European settlement, fires were used by native peoples to manage vegetation and fire behavior. However, an analysis comparing my results to desired future conditions would provide information on the feasibility of restoration efforts, and provide one anchor for conversations about the future of forest and fire management in the area.

%permanent firebreaks
Managers and planners should also consider the values at risk from fire or fire exclusion and prepare to mitigate them. This need extends to ecological values, such as old growth, closed canopy forest, and social values, such as infrastructure and development. Mitigation of this risk probably requires some form of active management. I outlined some methods for restoring disturbance above. In addition, stakeholders could choose to develop fire breaks in advance of wildfires, for example by clearing vegetation along roads \citep{Conard2003}. However, imposing such ``unnatural'' features has other ecological tradeoffs, especially from increased fragmentation and reduced patch shape complexity \citep{Trombulak2000}. Such a strategy could be evaluating using my model, either by increasing the ability of roads to function as barriers to fire spread, or by creating a new landcover class for firebreaks and assigning it lowered susceptibility values. In order to completely evaluate such a strategy it would also be necessary to rerun the simulation with the currently maintained roads parameterized as barriers to smaller fires.

%mimic old-growth
On a different note, if more frequent fire does reduce the quantity of old growth forests in the area, mechanical techniques in combination with prescribed fire could be used to mimic the structural complexity of old growth in much younger stands \citep{Franklin2002}. Because it is so productive, this portion of the Tahoe National Forest may be a good place to experiment with this strategy; trees grow large more quickly here than in other locations in the mixed conifer belt \citep{PRISMClimateGroup2004,Littell2012}.

%type conversions
The risks of type conversions are also pertinent to managers, and potentially one of the biggest risks of an increase in high severity fire as a result of climate change. While not explicitly explored in this study, it is predicted that cover type shifts and conversions are more likely to follow stand-replacing disturbances \citep{Stephens2013}. This risk will increase with climate change, and the additional increase in stand-replacing events suggested by the model indicates an interaction between climate change and high mortality fire that should be taken into account by managers planning restoration after fires, especially when selecting what species to plant or encourage \citep{Fule2008,Schwartz2015}. 

% range shifts
Active management of forest resources may also be needed to address climate-induced range shifts of cover types. Such shifts are already happening, and high severity fires, since they reset the vegetation to early seral conditions, provide additional opportunities for these shifts to occur by facilitating the progression of vegetation through alternate stages of succession \citep{Bachelet2001}. Changes to vegetation under climate change may be as simple as the upward or northerly movement of species ranges’, but they could also include the establishment of nonnative, invasive species. Managers can choose to try to influence vegetation communities through replanting prior vegetation, engaging in assisted migration, or controlling undesirable species. Unfortunately, there are no obvious or easy answers here. More research on specific large fires, such as the 2013 Rim Fire, should yield insights into how shifts in the fire regime may change spread and susceptibility patterns, which could in turn be used to update and improve the \textsc{RMLands} parameterization \citep{Lydersen2014}.




 


%% !TEX root = master.tex
\appendix


\chapter{Input Layers to \textsc{RMLands}}
\label{app:inputs}

\paragraph{Technical details on \textsc{RMLands} Data Structure}
\label{app:rmlspecs}
\textsc{RMLands} uses raster GeoTiffs (.tif files) as its data structure. Rasters are based on uniform square units called cells (or pixels). Each cell represents an actual portion of geographic space. In this application, we use the Universal Transverse Mercator (UTM) projection, Zone 10 North. The extent of the raster is rectangular although the area of study is not. Cells outside of the buffered project area are assigned a null value.\footnote{Latitude and longitude are commonly pictured when describing coordinates. In such cases the X value refers to longitude and Y refers to latitude. However, because we use UTMs in this project, the correct convention is actually that the X value is the Easting and the Y value is the Northing. For simplicity we discuss X and Y only in this document.} In the Yuba River watershed landscape, each grid cell is 30 meters on a side (i.e., 900 m$^2$ or 0.09 ha), and the input grid measures 2910 by 2245 pixels. \textsc{RMLands} requires that all input grids are perfectly aligned. We accomplished this by setting the Extent and Snap Raster to the same parameters whenever we manipulated the layers in ArcMap. This ``base'' spatial layer was created by taking the primary elevation layer used on the Tahoe National Forest, resampling it to a 30 meter grid, and clipping its extent to match that of the buffered project area. Each cell is assigned a single class value, where valid class values are positive non-zero integers. Integer values are mapped to more descriptive class names using csv files with names identical to the grid name. All grids are created in ArcMap and saved as GeoTiff files before being loaded into to the model. 

\section{Input Layer Maps}
\label{app:sec:inputmaps}

% cover layer
\begin{figure}[!htbp]
\centering
\includegraphics[height=0.4\textheight]{/Users/mmallek/Tahoe/Report2/images/cover.png}
\caption{Cover Type Map for the project area. Also shows the 10 km buffer from the project area boundary. See Table~\ref{covertable} for full land cover type names.}
\label{covermap}
\end{figure}

% condition layer
\begin{figure}[!htbp]
\centering
\includegraphics[height=0.4\textheight]{/Users/mmallek/Tahoe/Report2/images/condition.png}
\caption{Condition Class Map for the project area. Also shows the 10 km buffer from the project area boundary.} 
\label{conditionmap}
\end{figure}

% age layer
\begin{figure}[htbp]
\centering
\includegraphics[height=0.4\textheight]{/Users/mmallek/Tahoe/Report2/images/age.png}
\caption{Age map at Timestep 0 for the project area. Also shows the 10 km buffer from the project area boundary.} 
\label{agemap}
\end{figure}

% condition-age layer
\begin{figure}[htbp]
\centering
\includegraphics[height=0.4\textheight]{/Users/mmallek/Tahoe/Report2/images/condage.png}
\caption{Condition-Age map at Timestep 0 for the project area. Also shows the 10 km buffer from the project area boundary.} 
\label{condagemap}
\end{figure}

% tpi 
\begin{figure}[htbp]
\centering
\includegraphics[height=0.4\textheight]{/Users/mmallek/Tahoe/Report2/images/tpi.png}
\caption{Topographic Position Index for the project area. Also shows the 10 km buffer from the project area boundary.} 
\label{tpimap}
\end{figure}

% elevation layer
\begin{figure}[htbp]
\centering
\includegraphics[height=0.4\textheight]{/Users/mmallek/Tahoe/Report2/images/elevation.png}
\caption{Elevation for the project area. Also shows the 10 km buffer from the project area boundary.} 
\label{elevationmap}
\end{figure}

% slope layer
\begin{figure}[htbp]
\centering
\includegraphics[height=0.4\textheight]{/Users/mmallek/Tahoe/Report2/images/slope.png}
\caption{Slope for the project area, which ranges from flat to 126\%. Also shows the 10 km buffer from the project area boundary.} 
\label{slopemap}
\end{figure}

% aspect layer
\begin{figure}[htbp]
\centering
\includegraphics[height=0.4\textheight]{/Users/mmallek/Tahoe/Report2/images/aspect.png}
\caption{Aspect for the project area. Also shows the 10 km buffer from the project area boundary.} 
\label{aspectmap}
\end{figure}

% streams layer
\begin{figure}[htbp]
\centering
\includegraphics[height=0.4\textheight]{/Users/mmallek/Tahoe/Report2/images/streams.png}
\caption{Streams in the project area. Also shows the 10 km buffer from the project area boundary.} 
\label{streamsmap}
\end{figure}

%%%%%%%%%%%%%%%%%%%%%%%%%%%%%%%%%%%%%%%%%%%%%%%%%%%%%%%%%%%%%%%%%%%%%%%%%%%%%%%%%%%%%%%%%%%%%%%%%%%%%%%%%%%%%%%%%%%%%%%%%%%%%%%%%%%%%%%%%%%%%%%%%%%%%%%%%%%%%%%%%%%%%%%%%%%%%%%%%%%%%%%%%%%%%%%%%%%%%%%%%%%%%%%%%%%%%%%%%%%%%%%%%%%%%%%%%%%%%%%%%%%%%%%%%%%%%%%%%%%%%%%%%%%%%%%%%%%%%%%%%%%%%%%%%%%%%%%%%%%%%%%%%%%%%%%%%%%%%%%%%%%%%%%%%%%%%%%%%%%%%%%%%%%%%%%%%%%%%%%%%%%%%%%%%%%%%%%%%%%%%%%%%%%%%%%%
\chapter{Cover Type Descriptions}
\label{app:covertypedesc}

%% !TEX root = master.tex

\section{Big Sagebrush (SAGE)}
\label{sage-description}
\subsection*{General Information}

\subsubsection{Cover Type Overview}

\textbf{Big Sagebrush (SAGE)}
\newline
Crosswalks
\begin{itemize}
	\item EVeg: Regional Dominance Type 1
	\begin{itemize}
		\item Bitterbrush 
		\item Basin Sagebrush
		\item Great Basin Mixed Scrub
		\item Bitterbrush - Sagebrush
	\end{itemize}

	\item LandFire BpS Model
	\begin{itemize}
		\item 0610800 Inter-Mountain Basins Big Sagebrush Shrubland
	\end{itemize}

	\item Presettlement Fire Regime Type
	\begin{itemize}
		\item Big Sagebrush
	\end{itemize}
\end{itemize}

\noindent Reviewed by Michele Slaton, GIS Specialist, Inyo National Forest, USDA Forest Service

\subsubsection{Vegetation Description}
The Big Sagebrush landcover type is typified by large, open, discontinuous stands of \emph{Artemisia tridentata} of fairly uniform height. \emph{A. tridentata} tends to have a single short, thick, stem that branches into a nearly globular crown (Neal 1988). \emph{Ericameria nauseosa} is a frequent associate or co-dominant (LandFire 2007).

Shrub canopy cover generally ranges from very open, widely spaced, small plants to large, closely spaced plants with canopies touching. Cover may be greater at higher elevations and in areas receiving more precipitation. In addition to a deep root system, \emph{A. tridentata} has a well-developed system of lateral roots near the soil surface (LandFire 2007, Neal 1988). Consequently, well-established sagebrush plants exclude most other shrubs in an area up to three times their crown area. Forbs and graminoids are often more abundant beneath these crowns (Slaton pers. comm. 2013). This produces stands of shrubs of very uniform size and spacing (Neal 1988).

Often the habitat is composed of pure stands of \emph{A. tridentata}, but many stands include other species of \emph{Artemisia, Ericameria, Tetradymia, Ribes, Prunus, Cercocarpus,} and \emph{Purshia}. In communities not fully occupied by \emph{Artemisia}, various amounts of herbaceous understory are found. Perennial forb cover is usually less than 10\% with perennial grass cover reaching 20-25\% on the more productive sites. \emph{Pseudoroegneria spicata} may be a dominant species following replacement fires and a co-dominant after 20 years. \emph{Elymus elymoides} and \emph{Oryzopsis hymenoides} are common on more xeric sites. \emph{Festuca, Stipa, Poa}, and \emph{Leymus} are among the more common grasses. Percent cover and species richness of understory are determined by site limitations. \emph{Pinus monophylla} and \emph{Juniperus osteosperma} may be present, especially in areas protected from fire (Neal 1988, LandFire 2007).


\subsubsection{Distribution}
This widespread system is common to the Basin and Range province. It ranges in elevation from 900 m to 2450+ m (3000 ft - 8000+ ft) and occurs on well-drained soils on foothills, terraces, slopes, and plateaus. It is found on deeper soils (LandFire 2007).

\subsection*{Disturbances}


\subsubsection{Wildfire}
Wildfires tend to be high mortality, stand-replacing fires that initiate a process of post-fire forest succession. High mortality fires kill large as well as small shrubs, and may kill many of the forbs and grasses as well, although below-ground organs of at least some individual shrubs and herbs survive and re-sprout. 

Replacement fires generally occur where shrub canopy exceeds 25\% cover, or where grass cover is greater than 15\% and shrub cover is greater than 20\%. Surface fires occur in areas dominated by grasses but are otherwise uncommon (LandFire 2007). \emph{A tridentata} does not sprout after burning but most of the other shrubs common to the type do (Neal 1988). For the last several decades, post-settlement conversion to \emph{Bromus tectorum} has become common and results in changes to fire frequency and vegetation dynamics. Extended periods of fire suppression or absence can lead to \emph{P. monophylla-J. osteosperma} encroachment and subsequent decline of other shrubs and herbaceous plants (LandFire 2007). 

Estimates of fire rotations are available from the LandFire project and a review paper (LandFire 2007, Van de Water and Safford 2011). The LandFire project’s published fire return intervals are based on a series of associated models created using the Vegetation Dynamics Development Tool (VDDT). In VDDT, fires are specified concurrently with the transition that follows them. For example, a replacement fire causes a transition to the early development stage. In the RMLands model, such fires are classified as high mortality. However, in VDDT mixed severity fires may cause a transition to early development, a transition to a more open seral stage, or no transition at all. In this case, we categorize the first example as a high mortality fire, and the second and third examples as a low mortality fire. Based on this approach, we calculated fire rotations and the probability of high mortality fire for each of the three SAGE seral stages (Table~\ref{tab:sagedesc_fire}). 

\subsubsection{Other Disturbance}
Other disturbances are not currently modeled, but may, depending on the seral stage affected and mortality levels, reset patches to early development, maintain existing seral stages, or shift/accelerate succession to a more open seral stage. 

\begin{table}[]
\small
\centering
\caption{Fire rotations (years) and probability of high versus low mortality fires. Values were derived from BpS model 0610800 (LandFire 2007), Van de Water and Safford (2011), and Safford (pers. comm. 2013).}
\label{tab:sagedesc_fire}
\begin{tabular}{@{}lcc@{}}
\toprule
\textbf{Condition}         & \multicolumn{1}{l}{\textbf{Fire Rotation}} & \multicolumn{1}{l}{\textbf{\begin{tabular}[c]{@{}l@{}}Proportion \\ High Mortality\end{tabular}}} \\ \midrule
Target                     & 115      & n/a       \\
Early Development - All    & 200      & 0         \\
Mid Development - Moderate & 125      & 1         \\
Late Development - Closed  & 100      & 0.9       \\ \bottomrule
\end{tabular}
\end{table}



\subsection*{Vegetation Seral Stages}
We recognize three separate seral stages for SAGE: Early Development (ED), Mid Development - Moderate Canopy Cover (MDM), and Late Development - Closed Canopy Cover (LDC) (Figure~\ref{sage_transmodel}). Our seral stages are an alternative to ``successional'' classes that imply a linear progression of states and tend not to incorporate disturbance. The seral stages identified here are derived from a combination of successional processes and anthropogenic and natural disturbance, and are intended to represent a composition and structural condition that can be arrived at from multiple other conditions described for that landcover type. Thus our seral stages incorporate age, size, canopy cover, and vegetation composition. In general, the delineation of stages has originated from the LandFire biophysical setting model descriptive of a given landcover type; however, seral stages are not necessarily identical to the classes identified in those models.

\begin{figure}[htbp]
\centering
\includegraphics[width=0.8\textwidth]{/Users/mmallek/Documents/Thesis/statetransmodel/StateTransitionModel/shrub.png}
\caption{State and Transition Model for Big Sagebrush. Each dark grey box represents one of the three seral stages for this landcover type. Three stages of development are represented: early, middle, and late. We describe the middle development stage as characterized by moderate canopy cover and the late development stage as characterized by closed canopy cover, but these are not hard and fast rules. Transitions between states/seral stages may occur as a result of high mortality fire, low mortality fire, or succession. Specific pathways for each are denoted by the appropriate color line and arrow: red lines relate to high mortality fire, orange lines relate to low mortality fire, and green lines relate to natural succession.} 
\label{sage_transmodel}
\end{figure}

\subsubsection{Early Development (ED)}

\paragraph{Description} \emph{A. tridentata} does not sprout after burning but most of the other shrubs common to the type do. Consequently, for as long as 20 years after fire the vegetative community may be dominated by \emph{Chrysothamnus}, \emph{Tetradymia}, and grasses. A very hot fire in a degraded site may result in a seral community dominated by annual grasses and forbs. Perennial bunchgrasses frequently survive fires and become dominant (Neal 1988). Canopy cover is less than 40\%, but shrub cover may be as little as 10\%. Fuel loading is discontinuous (LandFire 2007).

\paragraph{Succession Transition} In the absence of disturbance, patches in this seral stage will transition to MDM at 20 years. 

\paragraph{Wildfire Transition} High mortality wildfire is not modeled for this seral stage. Low mortality wildfire (100\% of fires in this seral stage) maintains the patch in the ED seral stage. 

\noindent\hrulefill


\subsubsection{Mid Development - Moderate Canopy Cover (MDM)}

\paragraph{Description} \emph{A. tridentata} usually reaches fairly stable dominance 10 to 20 years after disturbance, with or without an understory of perennial bunchgrass. \emph{A. tridentata} usually remains dominant indefinitely or until the next disturbance (Neal 1988). Shrub density is sufficient in old stands to carry the fire without fine fuels. Shrubs and herbaceous vegetation can be codominant. Generally, shrub cover averages 30\% (LandFire 2007).

\paragraph{Succession Transition} At 40 years without disturbance, patches in this seral stage will transition to LDC. 

\paragraph{Wildfire Transition} High mortality wildfire (90\% of fires in this seral stage) recycles the patch through the ED seral stage. Low mortality wildfire (10\%) maintains the patch in the MDM seral stage.

\noindent\hrulefill


\subsubsection{Late Development - Closed Canopy Cover (LDC)}

\paragraph{Description} Shrublands with some encroachment from \emph{P. monophylla} and \emph{J. osteosperma} possible. Wildfire has not occurred for at least 60 years. Tree species cover is highly variable. In the continued absence of disturbance, shrub cover will decline (LandFire 2007).

\paragraph{Succession Transition} In the absence of disturbance, patches in this seral stage will maintain. 

\paragraph{Wildfire Transition} High mortality wildfire (100\% of fires in this seral stage) recycles the patch through the ED seral stage. Low mortality wildfire is not modeled for this seral stage.

\noindent\hrulefill

\subsection*{Condition Classification}
Because seral stage classification was done through orthophoto analysis, no polygons are assigned to the LDC seral stage, which is actually not an \emph{Artemisia}-dominated seral stage. Polygons with a cover value (not Null) are assigned to the MDM stage. Polygons with a Null value for shrub cover are assigned to ED.



\clearpage
\subsection*{References}
\begin{hangparas}{.25in}{1} 
LandFire. ``Biophysical Setting Models.'' Biophysical Setting 0610800: Inter-Mountain Basins Big Sagebrush Shrubland. 2007. LANDFIRE Project, U.S. Department of Agriculture, Forest Service; U.S. Department of the Interior. \burl{http://www.landfire.gov/national_veg_models_op2.php}. Accessed 9 November 2012.

Neal, Donald L. ``Sagebrush (SGB).'' \emph{A Guide to Wildlife Habitats of California}, edited by Kenneth E. Mayer and William F. Laudenslayer. California Deparment of Fish and Game, 1988. \burl{http://www.dfg.ca.gov/biogeodata/cwhr/pdfs/SGB.pdf}. Accessed 4 December 2012.

Safford, Hugh. Regional Ecologist, USDA Forest Service. Personal communication, 15 August 2013.

Van de Water, Kip M. and Hugh D. Safford. ``A Summary of Fire Frequency Estimates for California Vegetation Before Euro-American Settlement.'' \emph{Fire Ecology} 7.3 (2011): 26-57. doi: 10.4996/fireecology.0703026.
\end{hangparas}


%% !TEX root = master.tex
\newpage
\section{Black and Low Sagebrush (LSG)}
\label{lsg-description}

\subsection*{General Information}

\subsubsection{Cover Type Overview}

\textbf{Black and Low Sagebrush (LSG)}
\newline
\textbf{Crosswalks}
\begin{itemize}
	\item EVeg: Regional Dominance Type 1
	\begin{itemize}
		\item Low Sagebrush
		\item Black Sagebrush
	\end{itemize}

	\item LandFire BpS Model
	\begin{itemize}
		\item 0610790: Great Basin Xeric Mixed Sagebrush Shrubland
	\end{itemize}

	\item Presettlement Fire Regime Type
	\begin{itemize}
		\item Black and Low Sagebrush
	\end{itemize}
\end{itemize}

\noindent Reviewed by Michele Slaton, GIS Specialist, Inyo National Forest, USDA Forest Service

\subsubsection{Vegetation Description}
\paragraph{Black and Low Sagebrush (LSG)}	This landcover type is generally dominated by broad-leaved, evergreen shrubs of short stature, typically averaging about 15\% cover. Depending on site conditions, crowns may touch. Deciduous shrubs and small trees are sometimes sparsely scattered within this type. The ground cover of grasses and forbs is typically a sparse 5-15\% cover (Verner 1988). LSG may be dominated by either \emph{Artemisia arbuscula} or \emph{Artemisia nova}, often in association with \emph{Chrysothamnus viscidiflorus}, \emph{Purshia tridentata}, or \emph{Artemisia tridentata}; \emph{A. nova} is also commonly associated with \emph{Krascheninnikovia} and \emph{Ephedra}. \emph{Juniperus occidentalis} may be sparsely scattered in stands dominated by \emph{Artemisia arbuscula}, and \emph{Juniperus osteosperma} and \emph{Pinus monophylla} are sometimes scattered in stands dominated by \emph{A. nova}. A rich variety of forbs is usually present, including \emph{Eriogonum}, \emph{Erigeron}, \emph{Phlox}, \emph{Castilleja}, \emph{Sphaeralcea}, and \emph{Lupinus}. Common grasses include \emph{Poa}, \emph{Pseudoroegneria}, \emph{Elymus}, \emph{Stipa} and \emph{Festuca}. The abundance and distribution of associated plants is highly influenced by soils and precipitation (Verner 1988, LandFire 2007).

\subsubsection{Distribution}
Stands of \emph{A. arbuscula} are usually found on shallow soils with impaired drainage in the transition zone between the wetter bottom and open timber on the mountainsides. The type also occurs on terraces with hardpan or heavy clay soils. In mosaics formed with \emph{P. tridentata}, \emph{A. arbuscula} occurs on harsher sites with shallow, well-drained soils, while \emph{P. tridentata} occupies areas with deeper soils. Soils typically associated with stands of \emph{A. nova} are shallow, contain a high percentage of gravel, and are rich in mineral carbonates. It is prevalent on limestone soils (Verner 1988).

\emph{A. arbuscula} communities are generally restricted to elevated arid plains along the eastern flanks of the Sierra Nevada. \emph{A. nova} can occur in subalpine areas, at elevations above 2420 m (8000 ft). Stands dominated by \emph{A. arbuscula} range in elevation from 1210 to 2740 m (4000-9000 ft) (Verner 1988).


\subsection*{Disturbances}

\subsubsection{Wildfire}
Wildfires tend to be high mortality, stand-replacing fires that initiate a process of post-fire forest succession. High mortality fires kill large as well as small trees, and may kill many of the shrubs and herbs as well, although below-ground organs of at least some individual shrubs and herbs survive and re-sprout. 

\emph{A. nova} generally supports more fire than other dwarf sagebrushes. Stand-replacing fire is rare due to relatively low fuel loads and herbaceous cover. Bare ground acts as a micro-barrier to fire between low-statured shrubs. Stand-replacing fires can occur in this type when successive years of above average precipitation are followed by an average or dry year. Stand-replacing fires predominate in the late successional class where the herbaceous component has diminished or where trees dominate (LandFire 2007).

Although it is not included in this iteration of the model, scientists have noted that \emph{Bromus tectorum} has invaded most of these communities, altering successional pathways and disturbance regimes. It burns readily and is an early-season post-fire colonizer (Verner 1988).

Estimates of fire rotations are available from the LandFire project and a review paper (LandFire 2007, Van de Water and Safford 2011). The LandFire project’s published fire return intervals are based on a series of associated models created using the Vegetation Dynamics Development Tool (VDDT). In VDDT, fires are specified concurrently with the transition that follows them. For example, a replacement fire causes a transition to the early development stage. In the RMLands model, such fires are classified as high mortality. However, in VDDT mixed severity fires may cause a transition to early development, a transition to a more open seral stage, or no transition at all. In this case, we categorize the first example as a high mortality fire, and the second and third examples as a low mortality fire. Based on this approach, we calculated fire rotations and the probability of high mortality fire for each of the three LSG seral stages (Table~\ref{tab:lsgdesc_fire}). We computed the overall target fire rotation of 82 years based on values from Van de Water and Safford (2011). 




\begin{table}[!htbp]
\footnotesize
\centering
\caption{Fire rotation index values and probability of high severity fire (at least 75\% overstory tree mortality) probabilities. The seral stage that is most susceptible to fire (i.e., has the lowest predicted fire rotation) has a fire rotation index value of 1. Higher values correspond with lower likelihoods of experiencing wildfire. The values are relative only within an individual seral stage and should not be compared against other land cover types. Values were derived from VDDT model 0610790 (LandFire 2007) and Van de Water and Safford (2011). }
\label{tab:lsgdesc_fire}
\begin{tabular}{@{}lcc@{}}
\toprule
 \textbf{Seral Stage}    & \textbf{\begin{tabular}[c]{@{}c@{}}Fire Rotation \\ Index\end{tabular}} & \textbf{\begin{tabular}[c]{@{}c@{}}Probability of \\ High Severity Fire\end{tabular}} \\ \hline
Early Development - All    & 4.0     & 1      \\
Mid Development - Moderate & 1.0     & 1      \\
Late Development - Closed  & 2.4     & 0.31    \\ 
\emph{Target Fire Rotation}    			& \emph{82 years}  &   \\ 
\bottomrule
\end{tabular}
\end{table}

\subsubsection{Other Disturbance}
Other disturbances are not currently modeled, but may, depending on the seral stage affected and mortality levels, reset patches to early development, maintain existing seral stages, or shift/accelerate succession to a more open seral stage. 

\subsection*{Vegetation Seral Stages}
We recognize three separate seral stages for LSG: Early Development (ED), Mid Development - Moderate Canopy Cover (MDM), and Late Development - Closed Canopy Cover (LDC) (Figure~\ref{lsg_transmodel}). Our seral stages are an alternative to ``successional'' classes that imply a linear progression of states and tend not to incorporate disturbance. The seral stages identified here are derived from a combination of successional processes and anthropogenic and natural disturbance, and are intended to represent a composition and structural condition that can be arrived at from multiple other conditions described for that landcover type. Thus our seral stages incorporate age, size, canopy cover, and vegetation composition. In general, the delineation of stages has originated from the LandFire biophysical setting model descriptive of a given landcover type; however, seral stages are not necessarily identical to the classes identified in those models.


\begin{figure}[htbp]
\centering
\includegraphics[width=0.8\textwidth]{/Users/mmallek/Documents/Thesis/statetransmodel/StateTransitionModel/shrub.png}
\caption{State and Transition Model for Black and Low Sagebrush. Each dark grey box represents one of the three seral stages for this landcover type. Three stages of development are represented: early, middle, and late. We describe the middle development stage as characterized by moderate canopy cover and the late development stage as characterized by closed canopy cover, but these are not hard and fast rules. Transitions between states/seral stages may occur as a result of high mortality fire, low mortality fire, or succession. Specific pathways for each are denoted by the appropriate color line and arrow: red lines relate to high mortality fire, orange lines relate to low mortality fire, and green lines relate to natural succession.} 
\label{lsg_transmodel}
\end{figure}

\subsubsection{Early Development (ED)} 

\paragraph{Description} Early seral community dominated by herbaceous vegetation, including \emph{Poa}, \emph{Pseudoroegneria}, and \emph{Achnatherum}. Shrub canopy is less than 20\%. Fire-tolerant shrubs, such as \emph{Chrysothamnus} species are initial sprouters post-fire (LandFire 2007).

\paragraph{Succession Transition} In the absence of disturbance, patches in this seral stage will transition to MDM at 20 years. 

\paragraph{Wildfire Transition} High mortality wildfire (100\% of fires in this seral stage) recycles the patch through the ED seral stage. Low mortality wildfire is not modeled for this seral stage.

\noindent\hrulefill


\subsubsection{Mid Development - Moderate Canopy Cover (MDM)}

\paragraph{Description} Mid-seral community with a mixture of herbaceous and shrub vegetation. Vegetation present likely includes \emph{A. nova}, \emph{A. arbuscula}, \emph{Poa}, \emph{Achnatherum}, and \emph{Pseudoroegneria}.  Shrub cover often less than 25\% (LandFire 2007).

\paragraph{Succession Transition} After 120 years without high mortality disturbance, patches in this seral stage will transition to LDC. 

\paragraph{Wildfire Transition} High mortality wildfire (100\% of fires in this seral stage) recycles the patch through the ED seral stage. Low mortality wildfire is not modeled for this seral stage.

\noindent\hrulefill


\subsubsection{Late Development - Closed Canopy Cover (LDC)} 

\paragraph{Description} Late seral community with an increased presence of conifer trees (up to 40\% cover). The degree of tree canopy closure differs depending on whether it is an \emph{A. arbuscula} (closure likely under 15\%) or an \emph{A. nova} (closure up to 40\%) community. In \emph{A. arbuscula} communities a mixture of herbaceous and shrub vegetation with over 10\% shrub cover would still be present. In \emph{A. nova} communities the herbaceous and shrub component would be greatly reduced (less than 1\% cover). Vegetation present may also include \emph{Juniperus}, \emph{P. monophylla} and \emph{Achnatherum} (LandFire 2007).

\paragraph{Succession Transition} In the absence of disturbance, this class will maintain. 

\paragraph{Wildfire Transition} High mortality wildfire (31\% of fires in this seral stage) recycles the patch through the ED seral stage. Low mortality wildfire (69\%) maintains the LDO seral stage.

\noindent\hrulefill

\subsection*{Condition Classification}
Because seral stageification was done through orthophoto analysis, no polygons will be assigned to the LDC seral stage, which is actually not an \emph{Artemisia}-dominated seral stage. Only 3 polygons were assigned to LSG. Typical fields used to assign early-mid-late seral stage (overstory tree diameter) are null for shrubs. Cover is available. Polygons with cover less than 50\% were assigned to MDM and polygons with cover greater than 50\% were assigned to LDC.


\subsection*{References}

\begin{hangparas}{.25in}{1} 
\interlinepenalty=10000
LandFire. ``Biophysical Setting Models.'' Biophysical Setting 0610790: Great Basin Xeric Mixed Sagebrush Shrubland. 2007. LANDFIRE Project, U.S. Department of Agriculture, Forest Service; U.S. Department of the Interior. \burl{http://www.landfire.gov/national\_veg\_models\_op2.php}. Accessed 9 November 2012.

Van de Water, Kip M. and Hugh D. Safford. ``A Summary of Fire Frequency Estimates for California Vegetation Before Euro-American Settlement.'' \emph{Fire Ecology} 7.3 (2011): 26-57. doi: 10.4996/fireecology.0703026.

Verner, Jared. ``Low Sage (LSG).'' \emph{A Guide to Wildlife Habitats of California}, edited by Kenneth E. Mayer and William F. Laudenslayer. California Deparment of Fish and Game, 1988. \burl{http://www.dfg.ca.gov/biogeodata/cwhr/pdfs/SGB.pdf}. Accessed 4 December 2012.
\end{hangparas}



%% !TEX root = master.tex
\newpage
\section{Curl-leaf Mountain Mahogany (CMM)}
\label{cmm-description}

\subsection*{General Information}

\subsubsection{Cover Type Overview}

\textbf{Curl-leaf Mountain Mahogany (CMM)}
\newline
Crosswalks
\begin{itemize}
	\item EVeg: Regional Dominance Type 1
	\begin{itemize}
		\item Curl-leaf Mountain Mahogany
	\end{itemize}

	\item LandFire BpS Model
	\begin{itemize}
		\item 0610620: Inter-Mountain Basin Curl-leaf Mountan Mahogany Woodland and Shrubland
	\end{itemize}

	\item Presettlement Fire Regime Type
	\begin{itemize}
		\item Curl-leaf Mountain Mahogany
	\end{itemize}
\end{itemize}

\noindent Reviewed by Becky Estes, Central Sierra Province Ecologist, USDA Forest Service

\subsubsection{Vegetation Description}
This landcover type is characterized by the dominance or co-dominance of \emph{Cercocarpus ledifolius}. Other shrubs such as \emph{Artemisia}, \emph{Arctostaphylos}, \emph{Ceanothus}, and \emph{Ephedra} may be present. \emph{C. ledifolius} is both a primary early successional colonizer rapidly invading bare mineral soils after disturbance and the dominant long-lived species. Depending on the effects of a given fire on the seed bank, in some cases it could take 10 years to recolonize. Where \emph{C. ledifolius} has reestablished quickly after fire, \emph{Chrysothamnus nauseosus} may codominate. Litter and shading by woody plants inhibits the establishment of \emph{C. ledifolius}, particularly in late seral stages where canopy cover is high. Reproduction often appears more dependent upon geographic variables (slope, aspect, and elevation) than biotic factors. \emph{Artemisia arbuscula} and \emph{Artemisia nova} are infrequently associated. \emph{Symphoricarpos}, \emph{Amelanchier}, and \emph{Ribes} are present on cooler, moister sites. \emph{Pinus monophylla}, \emph{Juniperus}, \emph{Pseudotsuga menziesii}, \emph{Abies magnifica}, \emph{Abies concolor}, and \emph{Pinus jeffreyi} may have sporadic presence at very low densities. In older stands the understory may consist largely of \emph{Leptodactylon pungens} (LandFire 2007, Gucker 2006).

\subsubsection{Distribution}
\emph{C. ledifolius} communities are usually found on upper slopes and ridges between 2130 and 3200 m (7000-10,500 ft), although northern stands may occur as low as 600 m (200 ft). It is more common on northwestern and northeastern aspects. Most stands occur on rocky, shallow soils and outcrops, with mature stand cover from 10-55\%. In the absence of fire, old stands may occur on somewhat deeper soils, with more than 55\% cover (LandFire 2007).

\subsection*{Disturbances}

\subsubsection{Wildfire}
Wildfires tend to be high mortality, stand-replacing fires that initiate a process of post-fire forest succession. High mortality fires kill large as well as small trees, and may kill many of the shrubs and herbs as well, although below-ground organs of at least some individual shrubs and herbs survive and re-sprout. 

\emph{C. ledifolius} is easily killed by fire and does not resprout. However, it is a primary early successional colonizer, rapidly invading bare mineral soils after disturbance. Fires are not common in early seral stages, when there is little fuel, except in chaparral-dominated stands. Stand-replacing fires are more common in mid-seral stands, where herbs and smaller shrubs provide ladder fuels. When surface fire is relatively common, stands will adopt a savanna-like woodland structure with an understory characterized by \emph{Ribes}, \emph{L. pungens}, and various grasses. Trees can become very old and will rarely show fire scars. In late, closed stands, the absence of herbs and small forbs makes fire uncommon, requiring extreme winds and drought conditions. However, stands that do burn often experience high mortality fire (LandFire 2007).

Estimates of fire rotations are available from the LandFire project and a review paper (LandFire 2007, Van de Water and Safford 2011). The LandFire project's published fire return intervals are based on a series of associated models created using the Vegetation Dynamics Development Tool (VDDT). In VDDT, fires are specified concurrently with the transition that follows them. For example, a replacement fire causes a transition to the early development stage. In the RMLands model, such fires are classified as high mortality. However, in VDDT mixed severity fires may cause a transition to early development, a transition to a more open seral stage, or no transition at all. In this case, we categorize the first example as a high mortality fire, and the second and third examples as a low mortality fire. Based on this approach, we calculated fire rotations and the probability of high mortality fire for each of the three CMM seral stages (Table~\ref{tab:cmmdesc_fire}). We computed the overall target fire rotation of 76 years based on values from Van de Water and Safford (2011). 




\begin{table}[!htbp]
\footnotesize
\centering
\caption{Fire rotation index values and probability of high severity fire (at least 75\% overstory tree mortality) probabilities. The seral stage that is most susceptible to fire (i.e., has the lowest predicted fire rotation) has a fire rotation index value of 1. Higher values correspond with lower likelihoods of experiencing wildfire. The values are relative only within an individual seral stage and should not be compared against other land cover types. Values were derived from VDDT model 0610790 (LandFire 2007) and Van de Water and Safford (2011). }
\label{tab:cmmdesc_fire}
\begin{tabular}{@{}lcc@{}}
\toprule
 \textbf{Seral Stage}    & \textbf{\begin{tabular}[c]{@{}c@{}}Fire Rotation \\ Index\end{tabular}} & \textbf{\begin{tabular}[c]{@{}c@{}}Probability of \\ High Severity Fire\end{tabular}} \\ \hline
Early (All)     		   & 4.8  & 0.17        \\
Mid--Moderate  			   & 1.0  & 0.67        \\
Late--Closed               & 28.8  & 1      \\ 
\emph{Target Fire Rotation}    			& \emph{76 years}  &   \\ 
\bottomrule
\end{tabular}
\end{table}
   			
\subsubsection{Other Disturbance}
Other disturbances are not currently modeled, but may, depending on the seral stage affected and mortality levels, reset patches to early development, maintain existing seral stages, or shift/accelerate succession to a more open seral stage. 

\subsection*{Vegetation Seral Stages}
We recognize three separate seral stages for CMM: Early Development (ED), Mid Development - Moderate Canopy Cover (MDM), and Late Development - Closed Canopy Cover (LDC) (Figure~\ref{cmm_transmodel}). Our seral stages are an alternative to ``successional'' classes that imply a linear progression of states and tend not to incorporate disturbance. The seral stages identified here are derived from a combination of successional processes and anthropogenic and natural disturbance, and are intended to represent a composition and structural condition that can be arrived at from multiple other conditions described for that landcover type. Thus our seral stages incorporate age, size, canopy cover, and vegetation composition. In general, the delineation of stages has originated from the LandFire biophysical setting model descriptive of a given landcover type; however, seral stages are not necessarily identical to the classes identified in those models.

\begin{figure}[hbt]
\centering
\includegraphics[width=0.8\textwidth]{/Users/mmallek/Documents/Thesis/statetransmodel/StateTransitionModel/shrub.png}
\caption{State and Transition Model for Curl-leaf Mountain Mahogany. Each dark grey box represents one of the three seral stages for this landcover type. Three stages of development are represented: early, middle, and late. We describe the middle development stage as characterized by moderate canopy cover and the late development stage as characterized by closed canopy cover, but these are not hard and fast rules. Transitions between states/seral stages may occur as a result of high mortality fire, low mortality fire, or succession. Specific pathways for each are denoted by the appropriate color line and arrow: red lines relate to high mortality fire, orange lines relate to low mortality fire, and green lines relate to natural succession.} 
\label{cmm_transmodel}
\end{figure}

\subsubsection{Early Development (ED)}

\paragraph{Description} \emph{C. ledifolius} seedlings rapidly invade bare mineral soils after fire. Litter and shading by woody plants inhibits establishment. Bunchgrasses and disturbance-tolerant forbs and resprouting shrubs, such as \emph{Symphoricarpos}, may be present. \emph{Ericameria} and \emph{Artemisia} seedlings are likely present. Vegetation composition will affect fire behavior, especially if chaparral species like \emph{Arctostaphylos} or \emph{Ceanothus} are present (LandFire 2007).

\paragraph{Succession Transition} In the absence of disturbance, patches in this seral stage will transition to MDM upon reaching 20 years of age. 

\paragraph{Wildfire Transition} High mortality wildfire (17\% of fires in this seral stage) recycles the patch through the ED seral stage. No transition occurs as a result of low mortality fire.

\noindent\hrulefill


\subsubsection{Mid Development - Moderate Canopy Cover (MDM)}

\paragraph{Description} \emph{C. ledifolius} may co-dominate with mature \emph{Artemisia}, \emph{Purshia}, \emph{Symphoricarpos}, or \emph{Ericameria}. Few \emph{C. ledifolius} seedlings are present. Canopy cover is variable (LandFire 2007).

\paragraph{Succession Transition} After 120 years in this stage, patches in this seral stage will transition to LDC.

\paragraph{Wildfire Transition} High mortality wildfire (67\% of fires in this seral stage) recycles the patch through the ED seral stage. No transition occurs as a result of low mortality fire.

\noindent\hrulefill


\subsubsection{Late Development - Closed Canopy Cover (LDC)}

\paragraph{Description} Moderate to high cover of large shrub- or tree-like \emph{C. ledifolius}. When low mortality fire is relatively frequent, late-successional \emph{C. ledifolius} may exhibit evidence of infrequent fire scars on older trees. Patches may consist of open savanna-like woodlands with an herbaceous-dominated understory. Other shrub species may be abundant, but decadent. When low mortality fire is absent, very few other shrubs are present, and herbaceous cover is low. Duff may be very deep, and scattered trees may occur. \emph{C. ledifolius} trees reach very old age in the absence of stand-replacing fire, potentially living over 1000 years (LandFire 2007).

\paragraph{Succession Transition} In the absence of disturbance, patches in this seral stage will remain in this seral stage. 

\paragraph{Wildfire Transition} High mortality wildfire (100\% of fires in this seral stage) recycles the patch through the ED seral stage.

\noindent\hrulefill

\subsection*{Condition Classification}
To create the initial cover and seral stage layer (2010), polygons were randomly assigned to seral stages based on a 20:10:70 distribution for early/mid/late development (based on an analysis of past fire in the project area). Random numbers between 0 and 1 were generated using numpy for Python and used to assign each CMM polygon to a seral stage.




\subsection*{References}
\begin{hangparas}{.25in}{1} 
\interlinepenalty=10000
Gucker, Corey L. ``Cercocarpus ledifolius'' \emph{Fire Effects Information System}, U.S. Department of Agriculture, Forest Service, Rocky Mountain Research Station, Fire Sciences Laboratory, 2006.  \burl{http://www.fs.fed.us/database/feis/} [Accessed 29 July 2013.]. 

LandFire. ``Biophysical Setting Models.'' Biophysical Setting 0610790: Great Basin Xeric Mixed Sagebrush Shrubland. 2007. LANDFIRE Project, U.S. Department of Agriculture, Forest Service; U.S. Department of the Interior. \burl{http://www.landfire.gov/national\_veg\_models\_op2.php}. Accessed 9 November 2012.

Van de Water, Kip M. and Hugh D. Safford. ``A Summary of Fire Frequency Estimates for California Vegetation Before Euro-American Settlement.'' \emph{Fire Ecology} 7.3 (2011): 26-57. doi: 10.4996/fireecology.0703026.

\end{hangparas}


%% !TEX root = master.tex
\newpage
\section{Lodgepole Pine (LPN)}

\subsection*{General Information}

\subsubsection{Cover Type Overview}

\textbf{Lodgepole Pine (LPN)}
\newline
Crosswalks
\begin{itemize}
	\item EVeg: Regional Dominance Type 1
	\begin{itemize}
		\item Lodgepole Pine
	\end{itemize}

	\item LandFire BpS Model
	\begin{itemize}
		\item 0610581 Sierra Nevada Subalpine Lodgepole Pine Forest and Woodland - Wet
		\item 0610582 Sierra Nevada Subalpine Lodgepole Pine Forest and Woodland - Dry

	\end{itemize}

	\item Presettlement Fire Regime Type
	\begin{itemize}
		\item Lodgepole Pine
	\end{itemize}
\end{itemize}

\paragraph{Lodgepole Pine with Aspen (LPN-ASP)}
This type is created by overlaying the NRIS TERRA Inventory of Aspen on top of the EVeg layer. Where it intersects with LPN it is assigned to LPN-ASP.
\newline

\noindent Reviewed by Shana Gross, Ecologist, USDA Forest Service

\subsubsection{Vegetation Description}
\paragraph{Lodgepole Pine (LPN)} \emph{P. contorta} ssp. \emph{murrayana} is the overwhelming dominant within its forest community, mixing occasionally with \emph{Abies magnifica}, and with scattered \emph{Pinus jeffreyi}  and \emph{Pinus monticola}, and \emph{Tsuga mertensiana} at higher elevations (Fites-Kaufman et al. 2007). Mature Sierran stands often contain significant seedlings and saplings. Understory characteristics are influenced by proximity to meadow and stream margins. \emph{Arctostaphylos} and \emph{Ribes} are common shrubs. Stands associated with meadow edges and streams may have a rich herbaceous layer consisting of grasses, forbs, and sedges. Species associations are likely very location specific. Plants present may include but are not limited to \emph{Cassiope}, \emph{Vaccinium}, \emph{Phyllodoce}, \emph{Kalmia}, \emph{Ceanothus}, \emph{Chrysolepis}, and \emph{Carex}. Elsewhere, the understory may be virtually absent, consisting of scattered shrubs such as \emph{Quercus vaccinifolia}, and herbs like \emph{Antennaria}, \emph{Arabis}, \emph{Eriogonum}, and \emph{Gayophytum}. Fast-moving streams within the cover type are generally characterized by relatively dense populations of \emph{Salix} (Bartolome 1988, Fites-Kaufman et al. 2007, LandFire 2007a, LandFire 2007b).  

\paragraph{Lodgepole Pine with Aspen (LPN-ASP)}	When \emph{Populus tremuloides} co-occurs with LPN on the west side of the Sierran crest, it is typically found in smaller patches, often less than 2 ha (5 acres) in size. Mature stands in which \emph{P. tremuloides} are still dominant are usually relatively open. Average canopy closures range from 60 to 100 percent in young and intermediate-aged stands and from 25 to 60 percent in mature stands. The open nature of the stands results in substantial light penetration to the ground (Verner 1988).

\subsubsection{Distribution}
\paragraph{Lodgepole Pine (LPN)}	Open stands of \emph{P. contorta} ssp. \emph{murrayana}, which make up a widespread upper montane forest/woodland, tolerating both rocky soils and semisaturated meadow edges, in an elevational belt within and above the \emph{A. magnifica zone}. These forests, strongly dominated by \emph{P. contorta} ssp. \emph{murrayana}, generally occur at elevations of about 1,830 m to 2,400 m (6000 ft to 7875 ft) in the northern Sierra Nevada. Stands of \emph{P. contorta} ssp. \emph{murrayana} may reach much lower, however, with cold air drainage down glacial canyons (Fites-Kaufman et al. 2007, Anderson 1996). On infertile soils, \emph{P. contorta} ssp. \emph{murrayana} is often the only tree species that will grow (Lotan and Critchfield 1990).
More than any other Sierran conifer, \emph{P. contorta} ssp. \emph{murrayana} is relatively tolerant of poor soil aeration, and thus grows well around the margins of wet meadows and other moist areas. Many upper montane and subalpine meadows in the Sierra Nevada exhibit invasion of young \emph{P. contorta} ssp. \emph{murrayana} moving inward from their drier margins. It is not clear how much this process has been influenced by changes in fire frequency or grazing over the last 150 years (Fites-Kaufman et al. 2007).

\paragraph{Lodgepole Pine with Aspen (LPN-ASP)}		Sites supporting \emph{P. tremuloides} are associated with added soil moisture, i.e., azonal wet sites. These sites are found throughout the LPN zone, often close to streams, lakes, and meadows. Other sites include rock reservoirs, springs and seeps. Terrain can be simple to complex (LandFire 2007c). 


\subsection*{Disturbances}

\subsubsection{Wildfire}

\paragraph{Lodgepole Pine (LPN)} 	Wildfires tend to be high mortality, stand-replacing fires that initiate a process of post-fire forest succession. High mortality fires kill large as well as small trees, and may kill many of the shrubs and herbs as well, although below-ground organs of at least some individual shrubs and herbs survive and resprout. Low mortality fires tend to only kill small seedlings and depend on the herbaceous layer to carry fire.

Unlike the Rocky Mountain subspecies of \emph{P. contorta} (ssp. \emph{latifolia}), \emph{P. contorta} ssp. \emph{murrayana} does not have serotinous cones (Fites-Kaufman et al. 2007). Following high mortality fire, it initially establishes in even-aged stands, but small-scale disturbances and the ability of the subspecies to regenerate in the absence of fire promote uneven-aged structure (Cope 1993, Gross 2013).

High mortality fire occurs at long intervals. Mixed severity fire is related to fire behavior across the often moist areas where \emph{P. contorta} ssp. \emph{murrayana} is found. Surface fires are more common on drier sites, although in general sparse fuels limit fire ignition and spread. Most fires are small (less than 1 ha) but very large fires covering hundreds of hectares do occur (LandFire 2007a, LandFire 2007b). This is due in part to the high susceptibility to fire mortality by \emph{P. contorta} ssp. \emph{murrayana} because of its thin bark and shallower roots. Postfire conditions provide an ideal seedbed, and \emph{P. contorta} ssp. \emph{murrayana} is an early post-fire colonizer (Cope 1993).

\paragraph{Lodgepole Pine with Aspen (LPN-ASP)}	Sites supporting \emph{P. tremuloides} are maintained by stand-replacing disturbances that allow regeneration from below-ground suckers. Upland clones are impaired or suppressed by conifer ingrowth and overtopping and intensive grazing that inhibits growth. In a reference condition scenario, a few stands will advance toward conifer dominance, but in the current landscape scenario where fire has been reduced from reference conditions there are many more conifer-dominated mixed aspen stands (LandFire 2007c, Verner 1988). 

Estimates of fire rotations for these variants are available from the LandFire project and a few review papers. The LandFire project’s published fire return intervals are based on a series of associated models created using the Vegetation Dynamics Development Tool (VDDT). In VDDT, fires are specified concurrently with the transition that follows them. For example, a replacement fire causes a transition to the early development stage. In the RMLands model, such fires are classified as high mortality. However, in VDDT mixed severity fires may cause a transition to early development, a transition to a more open seral stage, or no transition at all. In this case, we categorize the first example as a high mortality fire, and the second and third examples as a low mortality fire. Based on this approach, we calculated fire rotations and the probability of high mortality fire for each of the LPN and LPN-ASP seral stages (Tables~\ref{tab:lpndesc_fire} and \ref{tab:lpnaspdesc_fire}). We computed overall target fire rotations based on values from Mallek et al. (2013) and Van de Water and Safford (2011). 



\begin{table}[]
\centering
\caption{Fire rotation (years) and proportion of high (versus low) mortality fires for Lodgepole Pine type. Values were derived from VDDT model 0610790 (LandFire 2007), Mallek et al. (2013), and Estes (personal communication). }
\label{tab:lpndesc_fire}
\begin{tabular}{@{}lcc@{}}
\toprule
\textbf{Condition}          & \textbf{Fire Rotation} & \multicolumn{1}{l}{\textbf{\begin{tabular}[c]{@{}l@{}}Proportion \\ High Mortality\end{tabular}}} \\ \midrule
Target                      & 52    & n/a        \\
Early Development - All     & 29    & 0.03       \\
Mid Development - Closed    & 59    & 0.41       \\
Mid Development - Moderate  & 27    & 0.15       \\
Mid Development - Open      & 18    & 0.07       \\
Late Development - Closed   & 37    & 0.26       \\
Late Development - Moderate & 24    & 0.13       \\
Late Development - Open     & 18    & 0.07       \\ \bottomrule
\end{tabular}
\end{table}

\begin{table}[]
\centering
\caption{Fire rotation (years) and proportion of high (versus low) mortality fires for Lodgepole Pine - Aspen type. Values were derived from VDDT model 0610790 (LandFire 2007) and Van de Water and Safford (pers. comm. 2013).}
\label{tab:lpnaspdesc_fire}
\begin{tabular}{@{}lcc@{}}
\toprule
\textbf{Condition}               & \textbf{Fire Rotation} & \multicolumn{1}{l}{\textbf{\begin{tabular}[c]{@{}l@{}}Proportion \\ High Mortality\end{tabular}}} \\ \midrule
Target                           & 52     & n/a        \\
Early Development - Aspen        & 29     & 0.03       \\
Mid Development - Aspen          & 59     & 0.41       \\
Mid Development - Aspen-Conifer  & 27     & 0.15       \\
Late Development - Conifer-Aspen & 24     & 0.13       \\
Late Development - Closed        & 37     & 0.26       \\ \bottomrule
\end{tabular}
\end{table}

\subsubsection{Other Disturbance}
Other disturbances are not currently modeled, but may, depending on the seral stage affected and mortality levels, reset patches to early development, maintain existing seral stages, or shift/accelerate succession to a more open seral stage. 

\subsection*{Vegetation Seral Stages}
We recognize seven separate seral stages for LPN: Early Development (ED), Mid Development - Open Canopy Cover (MDO), Mid Development - Moderate Canopy Cover, Mid Development - Closed Canopy Cover (MDC), Late Development - Open Canopy Cover (LDO), Late Development - Moderate Canopy Cover (LDM), and Late Development - Closed Canopy Cover (LDC) (Figure~\ref{transmodel_lpn}). The LPN-ASP variant is assigned to five seral stages: Early Development - Aspen (EDA), Mid Development - Aspen (MDA), Mid Development - Aspen with Conifer (MDAC), Late Development - Conifer with Aspen (LDCA), and Late Development - Closed Canopy Cover (LDC) (Figure~\ref{transmodel_lpn-asp}).

Our seral stages are an alternative to ``successional'' classes that imply a linear progression of states and tend not to incorporate disturbance. The seral stages identified here are derived from a combination of successional processes and anthropogenic and natural disturbance, and are intended to represent a composition and structural condition that can be arrived at from multiple other conditions described for that landcover type. Thus our seral stages incorporate age, size, canopy cover, and vegetation composition. In general, the delineation of stages has originated from the LandFire biophysical setting model descriptive of a given landcover type; however, seral stages are not necessarily identical to the classes identified in those models.


\begin{figure}[htbp]
\centering
\includegraphics[width=0.8\textwidth]{/Users/mmallek/Documents/Thesis/statetransmodel/StateTransitionModel/7class.png}
\caption{State and Transition Model for Lodgepole Pine Forest (not inclusive of the aspen variant). Each dark grey box represents one of the seven seral stages for this landcover type. Each column of boxes represents a stage of development: early, middle, and late. Each row of boxes represents a different level of canopy cover: closed (70-100\%), moderate (40-70\%), and open (0-40\%). Transitions between states/seral stages may occur as a result of high mortality fire, low mortality fire, or succession. Specific pathways for each are denoted by the appropriate color line and arrow: red lines relate to high mortality fire, orange lines relate to low mortality fire, and green lines relate to natural succession.} 
\label{transmodel_lpn}
\end{figure}

\begin{figure}[htbp]
\centering
\includegraphics[width=0.8\textwidth]{/Users/mmallek/Documents/Thesis/statetransmodel/StateTransitionModel/5class-asp.png}
\caption{State and Transition Model for Lodgepole Pine Forest - Aspen variant. Each dark grey box represents one of the seven seral stages for this landcover type. Each column of boxes represents a stage of development: early, middle, and late. Transitions between states/seral stages may occur as a result of high mortality fire, low mortality fire, or succession. Specific pathways for each are denoted by the appropriate color line and arrow: red lines relate to high mortality fire, orange lines relate to low mortality fire, and green lines relate to natural succession.} 
\label{transmodel_lpn-asp}
\end{figure}


\subsubsection{Lodgepole Pine}

\paragraph{Early Development (ED)}

\paragraph{Description} Grasses, forbs, low shrubs, and sparse to moderate cover of trees (primarily \emph{P. contorta} ssp. \emph{murrayana}) seedlings/saplings with an open canopy. This seral stage is characterized by the recruitment of a new cohort of early successional, shade-intolerant tree species into an open area created by a stand-replacing disturbance. 


A short period of herbaceous productivity precedes closure of the tree canopy on productive sites. The prolific seed output, establishment, and seedling growth of \emph{P. contorta} ssp. \emph{murrayana} makes the period of herbaceous production short (Bartolome 1988). \emph{P. contorta} ssp. \emph{murrayana} regeneration density ranges from moderate to dog hair thickets (LandFire 2007a).


\paragraph{Succession Transition} In the absence of disturbance, patches in this seral stage will begin transitioning to MDC at 10 years at a rate of 0.6 per time step. At 40 years, all patches will succeed. On average, patches remain in early development for 18 years.

\paragraph{Wildfire Transition} High mortality wildfire (100\% of fires in this seral stage) recycles the patch through the Early Development seral stage. No transition occurs as a result of low mortality fire.

\noindent\hrulefill


\paragraph{Mid Development - Open Canopy Cover (MDO)}

\paragraph{Description} Sparse ground cover of grasses, forbs, and shrubs. Mid-maturity \emph{P. contorta} ssp. \emph{murrayana} where surface fire or other disturbance has opened the stand. Canopy cover ranges from 10-50\% (LandFire 2007a).
Continued recruitment into stands produces overstocking and slow growth of the overcrowded trees. This overcrowding may make them susceptible to insects, although others have argued that the more vigorously growing trees are more likely to be attacked. Beetle infestation creates large quantities of fuel that increase the probability of wildfire (Bartolome 1988).


\paragraph{Succession Transition} Patches in this seral stage may stay in this seral stage under low mortality disturbance, but after 10 years without fire they begin transitioning to MDM at a rate of 0.8 per time step. Succession to LDO occurs once the patch has been in mid development for 50 years. The rate of succession per time step is 0.5. At 100 years, all stands will succeed to LDO. On average, patches remain in mid development for 54 years.

\paragraph{Wildfire Transition} High mortality wildfire (7\% of fires in this seral stage) recycles the patch through the Early Development seral stage. Low mortality wildfire (93\%) maintains the patch in MDO.

\noindent\hrulefill

\paragraph{Mid Development - Moderate Canopy Cover (MDM)}

\paragraph{Description} Sparse ground cover of grasses, forbs, and shrubs. Mid-maturity \emph{P. contorta} ssp. \emph{murrayana} where surface fire or other disturbance has opened the stand. Canopy cover ranges from 10-50\% (LandFire 2007a).

Continued recruitment into stands produces overstocking and slow growth of the overcrowded trees. This overcrowding may make them susceptible to insects, although others have argued that the more vigorously growing trees are more likely to be attacked. Beetle infestation creates large quantities of fuel that increase the probability of wildfire (Bartolome 1988).


\paragraph{Succession Transition} Patches in this seral stage may stay in this seral stage under low mortality disturbance, but after 10 years without fire they begin transitioning to MDC at a rate of 0.8 per time step. Succession to LDM occurs once the patch has been in mid development for 45 years. The rate of succession per time step is 0.55. At 90 years, all stands will succeed to LDM.

\paragraph{Wildfire Transition} High mortality wildfire (15\% of fires in this seral stage) recycles the patch through the Early Development seral stage. Low mortality wildfire (85\%) triggers a transition to MDO 68\% of the time; otherwise, it remains in MDM.

\noindent\hrulefill

\paragraph{Mid Development - Closed Canopy Cover (MDC)}

\paragraph{Description} Sparse ground cover of grasses, forbs, and shrubs; mid-maturity \emph{P. contorta} ssp. \emph{murrayana} undergoing intrinsic stand thinning. Considerable surface fuel from tree mortality from previous fire. Canopy cover is greater than 50\% (LandFire 2007a).

Continued recruitment into stands produces overstocking and slow growth of the overcrowded trees. This overcrowding may make them susceptible to insects, although others have argued that the more vigorously growing trees are more likely to be attacked. Beetle infestation creates large quantities of fuel that increase the probability of wildfire. (Bartolome 1988).


\paragraph{Succession Transition} After 40 years in a MD seral stage without a wildfire-triggered transition, patches in this seral stage will begin transitioning to LDC. The rate of succession per time step is 0.6. At 80 years, all patches succeed to LDC.

\paragraph{Wildfire Transition} High mortality wildfire (41\% of fires in this seral stage) recycles the patch through the Early Development seral stage. Low mortality wildfire (59\%) triggers a transition to MDM.

\noindent\hrulefill


\paragraph{Late Development - Open Canopy Cover (LDO)}

\paragraph{Description} Areas that have experienced one or more low severity understory fires that had reduced stand density or old stands that have not experienced fire but have been thinned by other processes (tree falls, etc.). Stands are uneven aged. Canopy cover ranges from 10-50\% (LandFire 2007a).

\paragraph{Succession Transition} Patches in this seral stage may maintain under low mortality disturbance, but after 25 years without fire, these patches succeed to LDM at a rate of 0.7 per timestep.

\paragraph{Wildfire Transition} High mortality wildfire (7\% of fires in this seral stage) recycles the patch through the Early Development seral stage. Low mortality wildfire (93\%) maintains the patch in LDO.

\noindent\hrulefill

\paragraph{Late Development - Moderate Canopy Cover (LDM)}

\paragraph{Description} Sparse ground cover of grasses, forbs, and shrubs. Mid-maturity \emph{P. contorta} ssp. \emph{murrayana} where surface fire or other disturbance has opened the stand. Canopy cover ranges from 10-50\% (LandFire 2007a).

Continued recruitment into stands produces overstocking and slow growth of the overcrowded trees. This overcrowding may make them susceptible to insects, although others have argued that the more vigorously growing trees are more likely to be attacked. Beetle infestation creates large quantities of fuel that increase the probability of wildfire (Bartolome 1988).


\paragraph{Succession Transition} Patches in this seral stage may stay in this seral stage under low mortality disturbance, but after 25 years without fire they begin transitioning to LDC at a rate of 0.7 per time step. 

\paragraph{Wildfire Transition} High mortality wildfire (13\% of fires in this seral stage) recycles the patch through the Early Development seral stage. Low mortality wildfire (87\%) triggers a transition to LDO 73\% of the time; otherwise, it remains in LDM.

\noindent\hrulefill

\paragraph{Late Development - Closed Canopy Cover (LDC)}

\paragraph{Description} Old \emph{P. contorta} ssp. \emph{murrayana} stands where fire has had minimal influence. Canopy cover exceeds 50\%.

\paragraph{Succession Transition} This class will maintain in the absence of disturbance.

\paragraph{Wildfire Transition} High mortality wildfire (26\% of fires in this seral stage) recycles the patch through the Early Development seral stage. Low mortality wildfire (73\%) triggers a transition to LDM.

\noindent\hrulefill
\noindent\hrulefill

\subsubsection{Aspen Variant}

\paragraph{Early Development - Aspen (ED-A)}

\paragraph{Description} Grasses, forbs, low shrubs, and sparse to moderate cover of tree seedlings/saplings (primarily \emph{P. tremuloides}) with an open canopy. This seral stage is characterized by the recruitment of a new cohort of early successional, shade-intolerant tree species into an open area created by a stand-replacing disturbance. 

Following disturbance, succession proceeds rapidly from an herbaceous layer to shrubs and trees, which invade together (Verner 1988). \emph{P. tremuloides} suckers over 6ft tall develop within about 10 years (LandFire 2007c). 


\paragraph{Succession Transition} Unless it burns, a patch in the early seral stage persists for 10 years, at which point it transitions to MD-A.

\paragraph{Wildfire Transition} High mortality wildfire (100\% of fires in this seral stage) recycles the patch through the Early Development seral stage. No transition occurs as a result of low mortality fire.

\noindent\hrulefill


\paragraph{Mid Development - Aspen (MD-A)}

\paragraph{Description} \emph{P. tremuloides} trees 5-16'' DBH. Canopy cover is highly variable, and can range from 40-100\%. These patches range in age from 10 to 110 years. Some understory conifers, predominantly \emph{P. contorta} ssp. \emph{murrayana}, are encroaching, but \emph{P. tremuloides} is still the dominant component of the stand (LandFire 2007c).

\paragraph{Succession Transition} MD-A persists for at least 50 years in the absence of fire, after which patches begin transitioning to MD-AC at a rate of 0.6 per timestep. After 100 years all remaining MD-A patches transition to MD-AC. 

\paragraph{Wildfire Transition} High mortality wildfire (41\% of fires in this seral stage) recycles the patch through the Early Development seral stage. No transition occurs as a result of low mortality fire.

\noindent\hrulefill

\paragraph{Mid Development - Aspen with Conifer (MD-AC)}

\paragraph{Description} These stands have been protected from fire since the last stand-replacing disturbance. \emph{P. tremuloides} trees are predominantly 16'' DBH and greater. Conifers (predominantly \emph{P. contorta} ssp. \emph{murrayana}) are present and becoming increasingly dominant over the \emph{P. tremuloides}. Conifers are pole to medium-sized, and conifer cover is at least 40\% (LandFire 2007c).

\paragraph{Succession Transition} MD-AC persists for 100 years in the absence of high mortality fire, after which patches transition to LDC. 

\paragraph{Wildfire Transition} High mortality wildfire (15\% of fires in this seral stage) returns the patch to ED-A. Low mortality wildfire (85\%) maintains the patch in MD-AC.

\noindent\hrulefill

\paragraph{Late Development - Closed (LDC)}

\paragraph{Description} Some \emph{P. tremuloides} continue to be present in the understory, but large\emph{ P. contorta} ssp. \emph{murrayana} are now the dominant tree species, having overtopped the \emph{P. tremuloides}. Smaller conifers are present in the midstory as well (LandFire 2007a). This seral stage is analogous to the LDC seral stage for the LPN variant.

\paragraph{Succession Transition} Patches in this seral stage will maintain in the absence of disturbance.

\paragraph{Wildfire Transition} High mortality wildfire (26\% of fires in this seral stage) will return the patch to ED-A. Low mortality wildfire (74\%) opens the stand up to LD-CA.

\noindent\hrulefill


\paragraph{Late Development - Conifer with Aspen (LD-CA)}

\paragraph{Description} If stands are sufficiently protected from fire such that conifer species overtop \emph{P. tremuloides} and become large, they may be able to withstand some fire that more sensitive \emph{P. tremuloides} cannot. When this occurs, it creates a patch characterized by late development conifers, such as \emph{P. contorta} ssp. \emph{murrayana}, and early seral \emph{P. tremuloides}. 

\paragraph{Succession Transition} LD-CA persists for 70 years in the absence of any fire, at which point patches transition to LDC. 

\paragraph{Wildfire Transition} High mortality wildfire (13\% of fires in this seral stage) returns the patch to ED-A. Low mortality wildfire (87\%) maintains the stand in LD-CA. 

\noindent\hrulefill

\newpage
\subsection*{Condition Classification}

\begin{table}[hbp]
\small
\centering
\caption{Classification of seral stage for LPN. Diameter at Breast Height (DBH) and Cover From Above (CFA) values taken from EVeg polygons. DBH categories are: null, 0-0.9'', 1-4.9'', 5-9.9'', 10-19.9'', 20-29.9'', 30''+. CFA categories are null, 0-10\%, 10-20\%, \dots , 90-100\%. Each row in the table below should be read with a boolean AND across each column.}
\label{lpn_classification}
\begin{tabular}{@{}lrrrrr@{}}
\toprule
\textbf{\begin{tabular}[l]{@{}l@{}}Cover \\ Condition\end{tabular}} & \textbf{\begin{tabular}[r]{@{}r@{}}Overstory Tree \\ Diameter 1 \\ (DBH)\end{tabular}} & \textbf{\begin{tabular}[r]{@{}r@{}}Overstory Tree \\ Diameter 2 \\ (DBH)\end{tabular}} & \textbf{\begin{tabular}[r]{@{}r@{}}Total Tree\\ CFA (\%)\end{tabular}} & \textbf{\begin{tabular}[r]{@{}r@{}}Conifer \\ CFA (\%)\end{tabular}} & \textbf{\begin{tabular}[r]{@{}r@{}}Hardwood \\ CFA (\%)\end{tabular}} \\ \midrule
Early All & 0-4.9'' & any & any & any & any \\
Mid Open & 5-9.9'' & any & 0-40 & any & any \\
Mid Moderate & 5-9.9'' & any & 40-70 & any & any \\
Mid Closed & 5-9.9'' & any & 70-100 & any & any \\
Late Open & 10''+ & any & 0-40 & any & any \\
Late Moderate & 10''+ & any & 40-70 & any & any \\
Late Closed & 10''+ & any & 70-100 & any & any \\ \bottomrule
\end{tabular}
\end{table}

LPN-ASP seral stages were assigned manually using NAIP 2010 Color IR imagery to assess seral stage.



\clearpage
\subsection*{References}
\begin{hangparas}{.25in}{1} 
Bartolome, James W. ``Lodgepole Pine (LPN).'' \emph{A Guide to Wildlife Habitats of California}, edited by Mayer, Kenneth E. and William F. Laudenslayer. California Deparment of Fish and Game. 1988. \burl{http://www.dfg.ca.gov/biogeodata/cwhr/pdfs/LPN.pdf}. Accessed 4 December 2012.

``CalVeg Zone 1.'' Vegetation Descriptions. \emph{Vegetation Classification and Mapping}.  11 December 2008. U.S. Forest Service. \burl{http://www.fs.usda.gov/Internet/FSE\_DOCUMENTS/fsbdev3\_046448.pdf}. Accessed 2 April 2013.

Cope, Amy B. 1993. ``Pinus contorta var. murrayana.'' In: Fire Effects Information System, [Online].  U.S. Department of Agriculture, Forest Service,  Rocky Mountain Research Station, Fire Sciences Laboratory (Producer).  \burl{http://www.fs.fed.us/database/feis/} [Accessed 4 December 2012].

Fites-Kaufman, Jo Ann, Phil Rundel, Nathan Stephenson, and Dave A. Wixelman. ``Montane and Subalpine Vegetation of the Sierra Nevada and Cascade Ranges.'' In \emph{Terrestrial Vegetation of California, 3rd Edition}, edited by Michael Barbour, Todd Keeler-Wolf, and Allan A. Schoenherr, 456-501. Berkeley and Los Angeles: University of California Press, 2007. 

Gross, Shana. Ecologist, USDA Forest Service. Personal communication, 3 July 2013.

Lotan, James E. and William B. Critchfield. ``Lodgepole Pine.'' Russell M. Burns and Barbara H. Honkala, tech. coords. Silvics of North America, vol 1. Conifers; Glossary. Agriculture handbook no.654. Washington, D.C.: U.S. Dept. of Agriculture, Forest Service, 1990. 

LandFire. ``Biophysical Setting Models.'' Biophysical Setting 0610581: Sierra Nevada Subalpine Lodgepole Pine Forest and Woodland. 2007a. LANDFIRE Project, U.S. Department of Agriculture, Forest Service; U.S. Department of the Interior. \burl{http://www.landfire.gov/national\_veg\_models\_op2.php}. Accessed 9 November 2012.

LandFire. ``Biophysical Setting Models.'' Biophysical Setting 0610582: Sierra Nevada Subalpine Lodgepole Pine Forest and Woodland. 2007b. LANDFIRE Project, U.S. Department of Agriculture, Forest Service; U.S. Department of the Interior. \burl{http://www.landfire.gov/national\_veg\_models\_op2.php}. Accessed 9 November 2012.

LandFire. ``Biophysical Setting Models.'' Biophysical Setting 0610610: Inter-Mountain Basins Aspen-Mixed Conifer Forest and Woodland. 2007c. LANDFIRE Project, U.S. Department of Agriculture, Forest Service; U.S. Department of the Interior. \burl{http://www.landfire.gov/national\_veg\_models\_op2.php}. Accessed 7 January 2013.

Safford, Hugh S. Regional Ecologist, USDA Forest Service. Personal communication, 5 May 2013.

Skinner, Carl N. and Chi-Ru Chang. ``Fire Regimes, Past and Present.'' \emph{Sierra Nevada Ecosystem Project: Final report to Congress, vol. II, Assessments and scientific basis for management options}. Davis: University of California, Centers for Water and Wildland Resources, 1996.

Van de Water, Kip M. and Hugh D. Safford. ``A Summary of Fire Frequency Estimates for California Vegetation Before Euro-American Settlement.'' \emph{Fire Ecology} 7.3 (2011): 26-57. doi: 10.4996/fireecology.0703026.

Verner, Jared. ``Aspen (ASP).'' \emph{A Guide to Wildlife Habitats of California}, edited by Kenneth E. Mayer and William F. Laudenslayer. California Deparment of Fish and Game, 1988. \burl{http://www.dfg.ca.gov/biogeodata/cwhr/pdfs/ASP.pdf}. Accessed 4 December 2012.

\end{hangparas}


%% !TEX root = master.tex
\newpage
\section{Mixed Evergreen Forest (MEG)}
\label{meg-description}

\subsection*{General Information}

\subsubsection{Cover Type Overview}

\textbf{Mixed Evergreen Forest (MEG)}
\newline
Crosswalks
\begin{itemize}
	\item EVeg: Regional Dominance Type 1
	\begin{itemize}
		\item Interior Mixed Hardwood
		\item California Bay
		\item Canyon Live Oak
		\item Madrone
		\item Bigleaf Maple
		\item Interior Live Oak
		\item Montane Mixed Hardwood 
		\item Pacific Douglas Fir
		\item Tanoak
	\end{itemize}

	\item EVeg: Regional Dominance Type 2
	\begin{itemize}
		\item Tanoak (regardless of RD Type 1 value, and therefore inclusive of all potential Type 1 vegetation types)
	\end{itemize}

	\item LandFire BpS Model
	\begin{itemize}
		\item 0610430 Mediterranean California Mixed Evergreen Forest
	\end{itemize}

	\item Presettlement Fire Regime Type
	\begin{itemize}
		\item Mixed Evergreen Forest
	\end{itemize}
\end{itemize}

\paragraph{Mesic Modifier (MEG\_M)}
This type is created by intersecting a binary xeric/mesic layer with the existing vegetation layer. MEG cells that intersect with mesic cells are assigned to the mesic modifier.
\paragraph{Xeric Modifier (MEG\_X)}
This type is created by intersecting a binary xeric/mesic layer with the existing vegetation layer. MEG cells that intersect with xeric cells are assigned to the xeric modifier.
\paragraph{Ultramafic Modifier (MEG\_U)}
This type is created by intersecting an ultramafic soils/geology layer with the existing vegetation layer. Where ultramafic cells intersect with MEG they are assigned to the ultramafic modifier.



\noindent Reviewed by Kyle Merriam, Sierra-Cascade Province Ecologist, USDA Forest Service; Becky Estes, Central Sierra Province Ecologist, USDA Forest Service


\subsubsection{Vegetation Description}
\paragraph{Mixed Evergreen Forest (MEG)} 	This landcover type forms a complex mosaic of forest due to the geologic, topographic, and successional variation typical within its range. This type is characterized by a combination of coniferous and broadleaved trees. Characteristic trees include \emph{Pseudotsuga menziesii}, \emph{Quercus chrysolepis}, \emph{Notholithocarpus densiflorus},\footnote{Tan oak was known as \emph{Lithocarpus densiflorus} for over 90 years before botanists renamed it \emph{Notholithocarpus densiflorus} in 2008 (Manos et al. 2008). Some sources and database continue to use the old name and plant symbol.}  \emph{Arbutus menziesii}, \emph{Umbellularia californica}, and \emph{Chrysolepis chrysophylla}. Species composition is primarily determined by the environmental gradients of temperature and moisture availability. \emph{Quercus kelloggii} is found on drier sites on inland portion of the range. \emph{Pinus lambertiana} and \emph{Pinus ponderosa} can be present in this type. These stands tend to have dense or diverse shrub understories with \emph{Ceanothus}, \emph{Corylus}, Gaultheria, Morella, Rhododendron, Ribes, Rubus, Toxicodendron diversilobum, and Vaccinium. Grass species include Bromus, Festuca, and Hierochloe. Polystichum \emph{munitum} and \emph{Pteridium aquilinum} var. \emph{pubescens} sometimes grow abundantly. \emph{Carex} spp. are present in some places (LandFire 2007, McDonald 1988, Tappeiner 1990).

\begin{adjustwidth}{2cm}{}
\textbf{Mesic Modifier (MEG\_M)}
Deep mesic soils support aggregations that include a lower or midstory layer of dense, sclerophyllous, broad-leaved evergreen trees like \emph{N. densiflorus} and \emph{Arbutus menziesii}, with an irregular, often open, higher layer of tall needle-leaved evergreen trees, typically \emph{P. menziesii}. A small number of pole and sapling trees occur throughout stands. On wetter sites, shrub layers are well developed, often with 100\% cover. Cover of the herbaceous layer under the shrubs can be up to 10 percent. At higher elevations, the shrubs disappear and the herb layer is often 100\%. Diversity of tree size typically increases with stand age, along with tree spacing. Young stands have closely spaced and uniformly distributed trees, whereas older stands have a more patchy stem distribution. Snags and downed logs, an important structural component of this habitat, increase in density or volume with stand age (Raphael 1988). Potential additional conifer associates include \emph{Abies concolor}, \emph{Pinus lambertiana}, \emph{Calocedrus decurrens}, and \emph{Pinus ponderosa} (Tappeiner 1990). A large variety of shrubs, forbs, grasses, sedges, and ferns, along with \emph{N. densiflorus} sprouts, can become aggressive on burned or cutover areas. This is especially true in areas where high severity fires have locally eliminated conifer seed sources (Tappeiner 1990).

\medskip
\noindent \textbf{Xeric Modifier (MEG\_X)}
A pronounced hardwood tree layer is typical, with an infrequent and poorly developed shrub stratum, and a sparse herbaceous layer (McDonald 1988). Characteristic oaks include \emph{Q. chrysolepis}, \emph{Q. wislizeni}, \emph{Q. kelloggi}, and \emph{Quercus garryana}. \emph{ Q. chrysolepis} and \emph{Q. wislizeni} are the most common oaks in the project area. They may individually form almost pure stands on steep canyon slopes and rocky ridgetops throughout the Sierra Nevada, or co-occur. They have tremendously variable growth forms, ranging from shrubs with multiple trunks on rocky, steep slopes, to magnificently spreading tall trees on deeper soils in moister areas. Both are evergreen with dense canopies (Allen-Diaz et al. 2007). Tree spacing is close (3-4 m) on better sites, and wider (8-10 m) on poor sites. In general, snags and downed woody material are sparse. Lower elevation associates are \emph{Pinus sabiniana}, \emph{Pinus attenuata}, \emph{N. densiflorus}, \emph{A. menziesii}, \emph{Quercus wislizeni}, \emph{C. chrysophylla}, and scrubby \emph{U. californica} (McDonald 1988).

\medskip
\noindent \textbf{Ultramafic Modifier (MEG\_U)}
\emph{Notholithocarpus densiflorus} var. \emph{echinoides}, or dwarf tanoak, grows on ultramafic and other less productive sites (Estes 2013). It is unclear if the 2 varieties differ genetically or if the small stature of dwarf tanoak is due to unproductive site conditions. Ecology literature does not usually distinguish between the 2 infrataxa (Fryer 2008). However, its identification is pertinent to management decisions. While \emph{N. lithocarpus} is generally protected as an oak species, the dwarf variety may be classified as a shrub and therefore subject to treatment or removal. Typically, \emph{P. menziesii} attains less dominance and may replaced by open stands of various conifers, such as \emph{Pinus ponderosa}, \emph{Pinus sabiniana}, or \emph{Pinus jeffreyi}. Trees occur within a generally open grassland or shrubland. The shrub layer is likely to include \emph{Quercus vaccinifolia}, \emph{N. densiflorus}, \emph{U. californica}, \emph{Quercus breweri}, and \emph{Rhamnus}. Common grasses include \emph{Stipa}, \emph{Festuca}, and \emph{Danthonia} (LandFire 2007b, McDonald 1988, O'Geen et al. 2007, Raphael 1988). 

\end{adjustwidth}

\subsubsection{Distribution}
\paragraph{Mixed Evergreen Forest}		This highly variable cover type occurs in the Sierra Nevada on all aspects at elevations of 350 m (1150 ft) to over 1700 m (5575 ft) (LandFire 2007a). Soil depth classes range from shallow to deep. The large number of species in the type, both conifer and hardwood, allow it to occupy and persist in a wide range of environments. Good soils and poor, steep slopes and slight, frequently disturbed and pristine all are at least adequate habitats for one or more species (McDonald 1988).

A xeric-mesic gradient was developed based on four variables: 1) aspect, 2) potential evapotranspiration, 3) topographic wetness index, and 4) soil water storage. The variables were standardized by z-score such that higher values correspond to more mesic environments. Thus, potential evapotranspiration was inverted to maintain this balance. The four variables were combined with equal weights. This final variables was split into xeric vs. mesic, with xeric occupying the negative end of the range up to $\frac{1}{4}$ standard deviation below the mean (zero) and mesic occupying the remaining portion of the spectrum.

\begin{adjustwidth}{2cm}{}
\textbf{Mesic Modifier }
Soils are deep, well-drained, and loamy, sandy, or gravelly. Found in valleys, coves, ravines, along streams, and on north as well as east slopes. It typically occurs in areas that are cool and moist sites in areas where precipitation is highest most likely in the form of rain and snow.

\medskip
\noindent \textbf{Xeric Modifier}
Q. chrysolepis and associates are found on a wide range of slopes, especially those that are moderate to steep. Soils are for the most part rocky, alluvial, coarse textured, poorly developed, and well drained. 

\medskip
\noindent \textbf{Ultramafic Modifier} Ultramafics have been mapped at various spatial densities throughout the elevational range of the landcover type. Low to moderate elevations in ultramafic and serpentinized areas often produce soils low in essential minerals like calcium potassium, and nitrogen, and have excessive accumulations of heavy metals such as nickel and chromium. These sites vary widely in the degree of serpentinization and effects on their overlying plant communities (``CalVeg Zone 1'' 2011). Note, the terms ``ultramafic rock'' and ``serpentine'' are broad terms used to describe a number of different but related rock types, including serpentinite, peridotite, dunite, pyroxenite, talc and soapstone, among others (O'Geen et al. 2007). 

\end{adjustwidth}

%%%

\subsection*{Disturbances}
\subsubsection{Wildfire}

\paragraph{Mixed Evergreen Forest}		Fire is the dominant disturbance event. Wildfires are common and frequent; mortality depends on vegetation vulnerability and wildfire intensity. Low mortality fires kill small trees and may consume above-ground portions of small oaks, shrubs and herbs, but do not kill large trees or below-ground organs of most oaks, shrubs and herbs which promptly resprout. High mortality fires kill trees of all sizes and may kill many of the shrubs and herbs as well. However, high mortality fires typically kill only the above ground portions of the oaks, shrubs and herbs; consequently, most oaks, shrubs and herbs promptly resprout from surviving below ground organs.

The vast majority of fires occur in late summer or early fall and are associated with lightning storms. Native American burns locally increased the frequency and may have been extensive prior to 1850. However, research also suggests that fire frequencies actually increased after European settlement (Merriam, pers. comm. 2013). Fires in the past were often large in area due to the high number of ignition points associated with fire events, and created patches of varying age and species composition (LandFire 2007a). 

Hardwoods typically provide the greatest cover after fire due to root-crown sprouting. Depending upon fire severity many hardwoods may have epicormic sprouting well into the crown. Species composition, density and interspecific competition within stands contributes to multiple pathways following disturbance. If fire has been absent from an area for an extended period of time, some conifers may be able to establish and persist even with the return of frequent low severity fire. But, if low severity fire is frequent after a stand-replacing fire, conifers will be more or less excluded and hardwoods will dominate (LandFire 2007a).

Estimates of fire rotations for these variants are available from the LandFire project and a few review papers. The LandFire project’s published fire return intervals are based on a series of associated models created using the Vegetation Dynamics Development Tool (VDDT). In VDDT, fires are specified concurrently with the transition that follows them. For example, a replacement fire causes a transition to the early development stage. In the RMLands model, such fires are classified as high mortality. However, in VDDT mixed severity fires may cause a transition to early development, a transition to a more open seral stage, or no transition at all. In this case, we categorize the first example as a high mortality fire, and the second and third examples as a low mortality fire. Based on this approach, we calculated fire rotations and the probability of high mortality fire for each of the MEG seral stages across the three variants (Tables~\ref{tab:megmdesc_fire}--\ref{tab:megudesc_fire}). We computed overall target fire rotations based on expert input from Safford and Estes, values from Mallek et al. (2013), and Van de Water and Safford (2011). 

\begin{adjustwidth}{2cm}{}
\textbf{Mesic Modifier }
\emph{N. densiflorus} is adapted to ignite easily. In the lower montane zone of the Sierra Nevada where \emph{N. densiflorus} occurs, the historic fire regime was characterized by dormant season fires of mostly low to moderate severity (Tappeiner 1990). In stands with high \emph{N. densiflorus} cover, \emph{N. densiflorus} may dominate the stand for many years before conifers re-establish. Patchy, stand-replacement fires were most common on north-facing slopes and during extended droughts. \emph{N. densiflorus} seedlings and saplings are typically top-killed by even low severity surface fire. Large trees usually survive moderate-severity fire, bearing fire scars afterward. Even \emph{N. densiflorus} with thick bark (3-10 cm) typically sustain bole damage from fire. Relative to associated conifers, mature \emph{P. menziesii} is fairly resistant to surface fires. Crown fires cause extensive mortality (Tappeiner 1990).

\medskip
\noindent \textbf{Xeric Modifier} \emph{Q. chrysolepis} has loose, dead, flaky bark that catches fire readily and burns intensely. Occasional fire often changes a stand of \emph{Q. chrysolepis} to \emph{Q. wislizeni}-chaparral, but without fire for sufficient time, trees again develop. Where fire is frequent, this oak becomes scarce or even drops out of the montane hardwood community (McDonald 1988).

\medskip
\noindent \textbf{Ultramafic Modifier} Historically, these woodland types had frequent low-severity fire. However, now there is higher susceptibility to stand replacing fire because of fire exclusion.

\end{adjustwidth}

%%%


\begin{table}[!htbp]
\footnotesize
\centering
\caption{Fire rotation index values and probability of high severity fire (at least 75\% overstory tree mortality) probabilities for Mixed Evergreen Forest - Mesic. The seral stage that is most susceptible to fire (i.e., has the lowest predicted fire rotation) has a fire rotation index value of 1. Higher values correspond with lower likelihoods of experiencing wildfire. The values are relative only within an individual seral stage and should not be compared against other land cover types. Values were derived from VDDT model 0610790 (LandFire 2007a) and Safford and Estes (personal communication). }
\label{tab:megmdesc_fire}
\begin{tabular}{@{}lcc@{}}
\toprule
 \textbf{Seral Stage}    & \textbf{\begin{tabular}[c]{@{}c@{}}Fire Rotation \\ Index\end{tabular}} & \textbf{\begin{tabular}[c]{@{}c@{}}Probability of \\ High Severity Fire\end{tabular}} \\ \hline
Early (All)     		 & 3.9     & 1                             \\
Mid--Closed    			 & 2.7     & 0.11                          \\
Mid--Moderate  			 & 1.5    & 0.11                          \\
Mid--Open      			 & 1.0   & 0.11                          \\
Late--Closed   			 & 2.5   & 0.21                          \\
Late--Moderate 			 & 1.4   & 0.11                          \\
Late--Open     			 & 1.0   & 0.11   						\\  
\emph{Target Fire Rotation}    			& \emph{50 years}  &   \\ 
\bottomrule
\end{tabular}
\end{table}

\begin{table}[!htbp]
\footnotesize
\centering
\caption{Fire rotation index values and probability of high severity fire (at least 75\% overstory tree mortality) probabilities for Mixed Evergreen Forest - Xeric. The seral stage that is most susceptible to fire (i.e., has the lowest predicted fire rotation) has a fire rotation index value of 1. Higher values correspond with lower likelihoods of experiencing wildfire. The values are relative only within an individual seral stage and should not be compared against other land cover types. Values were derived from VDDT model 0610790 (LandFire 2007a), and Safford and Estes (personal communication). }
\label{tab:megxdesc_fire}
\begin{tabular}{@{}lcc@{}}
\toprule
 \textbf{Seral Stage}    & \textbf{\begin{tabular}[c]{@{}c@{}}Relative Susceptibility \\ to Fire\end{tabular}} & \textbf{\begin{tabular}[c]{@{}c@{}}Probability of \\ High Severity Fire\end{tabular}} \\ \hline
Early (All)     		 & 5.8            & 1       \\
Mid--Closed    			 & 2.7        		 & 0.10    \\
Mid--Moderate  			 & 1.5      		& 0.10    \\
Mid--Open      			 & 1.0   		 & 0.10    \\
Late--Closed   			 &  2.5           & 0.10    \\
Late--Moderate 			 &  1.4        & 0.10    \\
Late--Open     			 &  1.0     		& 0.03 	  \\  
\emph{Target Fire Rotation}    			& \emph{40 years}  &   \\ 
\bottomrule
\end{tabular}
\end{table}

\begin{table}[!htbp]
\footnotesize
\centering
\caption{Fire rotation index values and probability of high severity fire (at least 75\% overstory tree mortality) probabilities for Mixed Evergreen Forest - Ultramafic. The seral stage that is most susceptible to fire (i.e., has the lowest predicted fire rotation) has a fire rotation index value of 1. Higher values correspond with lower likelihoods of experiencing wildfire. The values are relative only within an individual seral stage and should not be compared against other land cover types. Values were derived from VDDT model 0711700 (LandFire 2007b), and Safford and Estes (personal communication). }
\label{tab:megudesc_fire}
\begin{tabular}{@{}lcc@{}}
\toprule
 \textbf{Seral Stage}    & \textbf{\begin{tabular}[c]{@{}c@{}}Relative Susceptibility \\ to Fire\end{tabular}} & \textbf{\begin{tabular}[c]{@{}c@{}}Probability of \\ High Severity Fire\end{tabular}} \\ \hline
Early (All)     		 & 3.9            & 1           \\
Mid--Closed    			 & 2.7           & 0.11        \\
Mid--Moderate  			 & 1.5           & 0.11        \\
Mid--Open      			 & 1.0           & 0.11        \\
Late--Closed   			 & 2.5            & 0.21        \\
Late--Moderate 			 & 1.4           & 0.11        \\
Late--Open     			 & 1.0           & 0.11   		\\ 
\emph{Target Fire Rotation}    			& \emph{50 years}  &   \\ 
\bottomrule
\end{tabular}
\end{table}

%%%

\subsubsection{Other Disturbance}
Other disturbances are not currently modeled, but may, depending on the seral stage affected and mortality levels, reset patches to early development, maintain existing seral stages, or shift/accelerate succession to a more open seral stage. All of the tree species associated with this vegetation type are susceptible to a wide variety of pathogens and insects (such as sudden oak death for \emph{N. densiflorus}, which is caused by the pathogen \emph{Phytophthora ramorum}).

\subsection*{Vegetation Seral Stages}
We recognize seven separate seral stages for MEG: Early Development (ED), Mid Development - Open Canopy Cover (MDO), Mid Development - Moderate Canopy Cover, Mid Development - Closed Canopy Cover (MDC), Late Development - Open Canopy Cover (LDO), Late Development - Moderate Canopy Cover (LDM), and Late Development - Closed Canopy Cover (LDC) (Figure~\ref{meg_transmodel}). Our seral stages are an alternative to ``successional'' classes that imply a linear progression of states and tend not to incorporate disturbance. The seral stages identified here are derived from a combination of successional processes and anthropogenic and natural disturbance, and are intended to represent a composition and structural condition that can be arrived at from multiple other conditions described for that landcover type. Thus our seral stages incorporate age, size, canopy cover, and vegetation composition. In general, the delineation of stages has originated from the LandFire biophysical setting model descriptive of a given landcover type; however, seral stages are not necessarily identical to the classes identified in those models.

\begin{figure}[htbp]
\centering
\includegraphics[width=0.8\textwidth]{/Users/mmallek/Documents/Thesis/statetransmodel/StateTransitionModel/7class.png}
\caption{State and Transition Model for Mixed Evergreen Forest. Each dark grey box represents one of the seven seral stages for this landcover type. Each column of boxes represents a stage of development: early, middle, and late. Each row of boxes represents a different level of canopy cover: closed (70-100\%), moderate (40-70\%), and open (0-40\%). Transitions between states/seral stages may occur as a result of high mortality fire, low mortality fire, or succession. Specific pathways for each are denoted by the appropriate color line and arrow: red lines relate to high mortality fire, orange lines relate to low mortality fire, and green lines relate to natural succession.} 
\label{meg_transmodel}
\end{figure}

\paragraph{Early Development (ED)}

\paragraph{Description} This seral stage is characterized by the diversity of species establishing and reestablishing into an open area created by a stand-replacing disturbance. 

\begin{adjustwidth}{2cm}{}
\textbf{Mesic Modifier } On mesic sites, abundant grasses, forbs, low shrubs, found under sparse to moderate cover of trees (primarily \emph{P. menziesii} and \emph{N. densiflorus}) seedlings/saplings with an open canopy. Seedling establishment of \emph{P. menziesii} following fire is dependent on the spacing and number of surviving seed trees. Seedling establishment following large stand-replacing fires may be slow if seed trees are killed over extensive areas. Or, if there are numerous, well-spaced surviving seed trees within the burned area, a new cohort of seedlings can quickly establish (Uchytil 1991). Nearly all \emph{N. densiflorus} burls sprout after fire, and survivorship is high. \emph{Q. chrysolepis}, if present, also sprouts readily, and shrubs such as \emph{Mahonia}, \emph{Gaultheria}, and \emph{Rhododendron} may be significant. Shrub growth from seed banks, e.g. \emph{Ceanothus integerrimus}, can also be high (LandFire 2007a). Thus, \emph{N. densiflorus} and other shrubs usually dominante the initial seral stage if \emph{P. menziesii} isn’t able to seed in quickly (Raphael 1988).

\medskip
\noindent \textbf{Xeric Modifier}  On xeric sites, grasses, forbs, low shrubs, and sparse cover of tree seedlings and saplings are found under an open canopy. Forest openings contain a dense cover of hardwood sprouts. Sprouting shrubs such as \emph{M. aquifolium}, \emph{Gaultheria shallon}, and \emph{Rhododendron} may be significant. Shrub growth from seed banks, e.g. \emph{Ceanothus integerrimus}, can also be high (LandFire 2007a). 


\medskip
\noindent \textbf{Ultramafic Modifier}  On ultramafic sites, \emph{P. menziesii} may be stunted and slow-growing, and \emph{N. densiflorus} var. \emph{echinoides} may be present. Grasses like \emph{Festuca}, \emph{Danthonia}, and \emph{Acnatherum}, or else chaparral shrubs establish. Scattered \emph{Pinus ponderosa}, \emph{Pinus sabiniana}, or \emph{Pinus jeffreyi} may also be present (LandFire 2007b).

\end{adjustwidth}

\paragraph{Succession Transition}
\begin{adjustwidth}{2cm}{}
\textbf{Mesic and Xeric Modifier } In the absence of disturbance, patches in this seral stage will begin transitioning to MDM at 20 years. The rate of succession per time step is 0.8. At 40 years, all patches will succeed. On average, patches remain in ED for 26 years.


\medskip
\noindent \textbf{Ultramafic Modifier} Succession may be delayed. Thus, in the absence of disturbance, patches in this seral stage will begin transitioning to MDM after 30 years and may be delayed in the ED seral stage for as long as 80 years. A patch in this seral stage succeeds at a rate of 0.4 per time step. On average, patches remain in ED for 43 years.

\end{adjustwidth}

\paragraph{Wildfire Transition} High mortality wildfire (100\% of fires in this seral stage) recycles the patch through the ED seral stage. Low mortality wildfire is not modeled for this seral stage.

\noindent\hrulefill


\paragraph{Mid Development - Open Canopy Cover (MDO)}

\paragraph{Description}
\begin{adjustwidth}{2cm}{}
\textbf{Mesic Modifier } Sparse ground cover of grasses, forbs, and shrubs; open tree canopy cover (primarily \emph{P. menziesii} and \emph{N. densiflorus}). Other \emph{Quercus} and \emph{Arctostaphylos} species may also be present. In this stage, hardwoods are dominant, but \emph{P. menziesii} and possibly other conifers are established or establishing under the predominantly \emph{N. densiflorus} canopy (LandFire 2007a, McDonald 1988). 


\medskip
\noindent \textbf{Xeric Modifier}  Sparse ground cover of grasses, forbs, and shrubs; open tree canopy cover, primarily hardwoods such as \emph{Q. chrysolepis} and \emph{Q. kelloggii}. Conifers such as \emph{P. menziesii} are present at low densities in emergent status. The shrub understory is still a significant presence (LandFire 2007a). 


\medskip
\noindent \textbf{Ultramafic Modifier}  Ultramafic sites are characterized by open \emph{P. menziesii}, \emph{Pinus ponderosa}, \emph{Pinus sabiniana}, or \emph{Pinus jeffreyi} stands with an understory comprised of \emph{N. densiflorus} var. \emph{echinoides} or \emph{Q. chrysolepis} as well as grasses, forbs, and shrubs (LandFire 2007b).

\end{adjustwidth}
\paragraph{Succession Transition}
\begin{adjustwidth}{2cm}{}
\textbf{Mesic and Xeric Modifier } Patches in this seral stage may stay in this seral stage under low mortality disturbance, but after 15 years without fire they begin transitioning to MDM at a rate of 0.8 per time step. After 20 years in a mid development stage, patches in this seral stage will begin transitioning to LDO. The rate of succession per time step is 0.8. At 40 years, all patches succeed. On average, patches remain in the mid development stage for 26 years.


\medskip
\noindent \textbf{Ultramafic Modifier}  Succession may be delayed. Thus, in the absence of low mortality disturbance, patches in the MDO seral stage will begin transitioning to MDM after 20 years in MDO at a rate of 0.7 per timestep. Patches in this seral stage will begin transitioning to LDO after 30 years in a mid development stage, and may be delayed in this stage for as long as 80 years. A patch in this seral stage succeeds at a rate of 0.4 per time step. On average, patches remain in the mid development stage for 43 years.

\end{adjustwidth}
\paragraph{Wildfire Transition}
\begin{adjustwidth}{2cm}{}
\textbf{Mesic Modifier } High mortality wildfire (11\% of fires in this seral stage) recycles the patch through the ED seral stage. Low mortality wildfire (89\%) does not effect a change in the MDO seral stage.


\medskip
\noindent \textbf{Xeric Modifier}  High mortality wildfire (10\% of fires in this seral stage) recycles the patch through the ED seral stage. Low mortality wildfire (90\%) does not effect a change in the MDO seral stage.


\medskip
\noindent \textbf{Ultramafic Modifier} High mortality wildfire (11\% of fires) recycles the patch through the ED seral stage. Low mortality wildfire (89\%) does not effect a change in the MDO seral stage.

\end{adjustwidth}

\noindent\hrulefill

\paragraph{Mid Development - Moderate Canopy Cover (MDM)}

\paragraph{Description}
\begin{adjustwidth}{2cm}{}
\textbf{Mesic Modifier } Sparse ground cover of grasses, forbs, and shrubs; moderate tree canopy cover (primarily \emph{P. menziesii} and \emph{ N. densiflorus}). Other \emph{Quercus} and \emph{Arctostaphylos} species may also be present. In this stage, hardwoods are dominant, but \emph{P. menziesii} and possibly other conifers are established or establishing under the predominantly \emph{N. densiflorus} canopy (LandFire 2007a, McDonald 1988). 


\medskip
\noindent \textbf{Xeric Modifier} Sparse ground cover of grasses, forbs, and shrubs; moderate tree canopy cover, primarily hardwoods such as \emph{Q. chrysolepis} and \emph{Q. kelloggii}. Conifers such as \emph{P. menziesii} are present at low densities in emergent status. The shrub understory is still a significant presence (LandFire 2007a). 


\medskip
\noindent \textbf{Ultramafic Modifier}  Ultramafic sites are characterized by open \emph{P. menziesii}, \emph{Pinus ponderosa}, \emph{Pinus sabiniana}, or \emph{Pinus jeffreyi} stands with an understory comprised of \emph{N. densiflorus} var. \emph{echinoides} or \emph{Q. chrysolepis} as well as grasses, forbs, and shrubs (LandFire 2007b).

\end{adjustwidth}
\paragraph{Succession Transition}
\begin{adjustwidth}{2cm}{}
\textbf{Mesic and Xeric Modifier } Patches in this seral stage may stay in this seral stage under low mortality disturbance, but after 15 years without fire they begin transitioning to MDC at a rate of 0.8 per time step. After 20 years in a mid development stage, patches in this seral stage will begin transitioning to LDM. The rate of succession per time step is 0.8. At 40 years, all patches succeed. On average, patches remain in the mid development stage for 26 years.


\medskip
\noindent \textbf{Ultramafic Modifier} Succession may be delayed. Thus, in the absence of low mortality disturbance, patches in the MDM seral stage will begin transitioning to MDC after 20 years in MDM at a rate of 0.7 per timestep. Patches in this seral stage will begin transitioning to LDM after 30 years in a mid development stage, and may be delayed in this stage for as long as 80 years. A patch in this seral stage succeeds at a rate of 0.4 per time step. On average, patches remain in the mid development stage for 43 years.

\end{adjustwidth}
\paragraph{Wildfire Transition}
\begin{adjustwidth}{2cm}{}
\textbf{Mesic Modifier } High mortality wildfire (11\% of fires in this seral stage) recycles the patch through the ED seral stage. Low mortality wildfire (89\%) triggers a transition to MDM 14\% of the time; otherwise, it remains in MDC.

\medskip
\noindent \textbf{Xeric Modifier} High mortality wildfire (10\% of fires in this seral stage) recycles the patch through the ED seral stage. Low mortality wildfire (90\%) triggers a transition to MDM 14\% of the time; otherwise, it remains in MDC.

\medskip
\noindent \textbf{Ultramafic Modifier} High mortality wildfire (11\% of fires) recycles the patch through the ED seral stage. Low mortality wildfire (89\%) triggers a transition to MDM 13\% of the time; otherwise, it remains in MDC.

\end{adjustwidth}
\noindent\hrulefill

\paragraph{Mid Development - Closed Canopy Cover (MDC)}

\paragraph{Description}
\begin{adjustwidth}{2cm}{}
\textbf{Mesic Modifier } Sparse ground cover of grasses, forbs, and shrubs; closed tree canopy cover (primarily \emph{P. menziesii} and \emph{N. densiflorus}). Other \emph{Quercus} and \emph{Arctostaphylos} species may also be present. In this stage, hardwoods are dominant, but \emph{P. menziesii} and possibly other conifers are established or establishing under the predominantly \emph{N. densiflorus} canopy (LandFire 2007a, McDonald 1988). 

\medskip
\noindent \textbf{Xeric Modifier} Sparse ground cover of grasses, forbs, and shrubs; closed tree canopy cover, primarily hardwoods such as \emph{Q. chrysolepis} and \emph{Q. kelloggii}. Conifers such as \emph{P. menziesii} are present at low densities in emergent status. The shrub understory is still a significant presence (LandFire 2007a). 

\medskip
\noindent \textbf{Ultramafic Modifier} Ultramafic sites are characterized by open \emph{P. menziesii}, \emph{Pinus ponderosa}, \emph{Pinus sabiniana}, or \emph{Pinus jeffreyi} stands with an understory comprised of \emph{N. densiflorus} var. \emph{echinoides} or \emph{Q. chrysolepis} as well as grasses, forbs, and shrubs (LandFire 2007b).

\end{adjustwidth}
\paragraph{Succession Transition}
\begin{adjustwidth}{2cm}{}
\textbf{Mesic and Xeric Modifier } After 20 years in a mid development stage, patches in this seral stage will begin transitioning to LDC. The rate of succession per time step is 0.8. At 40 years, all patches succeed. On average, patches remain in the mid development stage for 26 years.

\medskip
\noindent \textbf{Ultramafic Modifier} Succession may be delayed. Patches in this seral stage will begin transitioning to LDC after 30 years in a mid development stage, and may be delayed in this stage for as long as 80 years. A patch in this seral stage succeeds at a rate of 0.4 per time step. On average, patches remain in the mid development stage for 43 years.

\end{adjustwidth}
\paragraph{Wildfire Transition}
\begin{adjustwidth}{2cm}{}
\textbf{Mesic Modifier } High mortality wildfire (11\% of fires in this seral stage) recycles the patch through the ED seral stage. Low mortality wildfire (89\%) triggers a transition to MDM 22\% of the time; otherwise, it remains in MDC.

\medskip
\noindent \textbf{Xeric Modifier} High mortality wildfire (10\% of fires in this seral stage) recycles the patch through the ED seral stage. Low mortality wildfire (90\%) triggers a transition to MDM 20\% of the time; otherwise, it remains in MDC.

\medskip
\noindent \textbf{Ultramafic Modifier} High mortality wildfire (11\% of fires) recycles the patch through the ED seral stage. Low mortality wildfire (89\%) triggers a transition to MDM 22\% of the time; otherwise, it remains in MDC.

\end{adjustwidth}
\noindent\hrulefill


\paragraph{Late Development - Open Canopy Cover (LDO)}

\paragraph{Description}
\begin{adjustwidth}{2cm}{}
\textbf{Mesic Modifier } Overstory of large and very large trees, primarily \emph{P. menziesii}. Canopy cover less than 40\%. \emph{P. lambertiana} also occurs. \emph{N. densiflorus} is tolerant of both full sun and shade, and usually dominates the subcanopy at this stage. Co-dominance of the upper canopy with \emph{P. menziesii} is uncommon but possible after extended periods without disturbance (Uchytil 1991, LandFire 2007a). There is also some evidence that the senescence of late development \emph{N. densiflorus} may cause openings in the canopy and allow for continued \emph{P. menziesii} dominance (Estes pers. comm. 2013). \emph{Quercus} and \emph{Arctostaphylos} species may also be present in the sub-canopy (LandFire 2007a).

\medskip
\noindent \textbf{Xeric Modifier}  Overstory of large and very large trees, often with canopy cover less than 40\%. \emph{P. menziesii}, \emph{Q. chrysolepis}, and \emph{Arctostaphylos mewukka} may occur. Conifers are taller and larger than in MD and clearly form the upper canopy layer here. Shrubs persist in openings but those in shade are likely to begin senescing (LandFire 2007a). On ultramafic sites, large \emph{Pinus ponderosa} may additionally be present. Grass savannah persists on sites experiencing low intensity fire (with \emph{Festuca}, \emph{Achnatherum}, and \emph{Danthonia}). Where fire is less frequent, chaparral shrubland develops (with \emph{Arctostaphylos} and \emph{Quercus breweri}) (LandFire 2007b).

\medskip
\noindent \textbf{Ultramafic Modifier} On ultramafic sites, large \emph{Pinus ponderosa}, \emph{Pinus sabiniana}, or \emph{Pinus jeffreyi} may be present along with \emph{P. menziesii} and \emph{N. densiflorus} var. \emph{echinoides}. Grass savannah persists on sites experiencing low intensity fire (with \emph{Festuca}, \emph{Achnatherum}, and \emph{Danthonia}). Where fire is less frequent, chaparral shrubland develops (with \emph{Arctostaphylos} and \emph{Quercus breweri}) (LandFire 2007b).

\end{adjustwidth}
\paragraph{Succession Transition}
\begin{adjustwidth}{2cm}{}
\textbf{Mesic and Xeric Modifier } Patches in this seral stage may stay in this seral stage under low mortality disturbance, but after 15 years without fire they begin transitioning to LDM at a rate of 0.8 per time step. 

\medskip
\noindent \textbf{Ultramafic Modifier} Succession may be delayed. Thus, in the absence of low mortality disturbance, patches in the LDO seral stage will begin transitioning to LDM after 20 years in LDO at a rate of 0.7 per timestep. 

\end{adjustwidth}
\paragraph{Wildfire Transition}
\begin{adjustwidth}{2cm}{}
\textbf{Mesic Modifier } High mortality wildfire (11\% of fires in this seral stage) recycles the patch through the ED seral stage. Low mortality wildfire (89\%) does not effect a change in the MDO seral stage. 

\medskip
\noindent \textbf{Xeric Modifier} High mortality wildfire (3\% of fires in this seral stage) recycles the patch through the ED seral stage. Low mortality wildfire (97\%) does not effect a change in the MDO seral stage.

\medskip
\noindent \textbf{Ultramafic Modifier} High mortality wildfire (11\% of fires) recycles the patch through the ED seral stage. Low mortality wildfire (89\%) does not effect a change in the MDO seral stage.

\end{adjustwidth}
\noindent\hrulefill

\paragraph{Late Development - Moderate Canopy Cover (LDM)}

\paragraph{Description}
\begin{adjustwidth}{2cm}{}
\textbf{Mesic Modifier } Overstory of large and very large trees, primarily \emph{P. menziesii}. Canopy cover between 40\% and 60\%. \emph{P. lambertiana} also occurs. \emph{N. densiflorus} is tolerant of both full sun and shade, and usually dominates the subcanopy at this stage. Co-dominance of the upper canopy with \emph{P. menziesii} is uncommon but possible after extended periods without disturbance (Uchytil 1991, LandFire 2007a). There is also some evidence that the senescence of late development \emph{N. densiflorus} may cause openings in the canopy and allow for continued \emph{P. menziesii} dominance (Estes pers. comm. 2013). \emph{Quercus} and \emph{Arctostaphylos} species may also be present in the sub-canopy (LandFire 2007a).

\medskip
\noindent \textbf{Xeric Modifier} Overstory of large and very large trees, often with canopy cover between 40\% and 60\%. \emph{P. menziesii}, \emph{Q. chrysolepis}, and \emph{Arctostaphylos mewukka} may occur. Conifers are taller and larger than in MD and clearly form the upper canopy layer here. Shrubs persist in openings but those in shade are likely to begin senescing (LandFire 2007a). On ultramafic sites, large \emph{Pinus ponderosa} may additionally be present. Grass savannah persists on sites experiencing low intensity fire (with \emph{Festuca}, \emph{Achnatherum}, and \emph{Danthonia}). Where fire is less frequent, chaparral shrubland develops (with \emph{Arctostaphylos} and \emph{Quercus breweri}) (LandFire 2007b).

\medskip
\noindent \textbf{Ultramafic Modifier} On ultramafic sites, large \emph{Pinus ponderosa}, \emph{Pinus sabiniana}, or \emph{Pinus jeffreyi} may be present along with \emph{P. menziesii} and \emph{N. densiflorus} var. \emph{echinoides}. Grass savannah persists on sites experiencing low intensity fire (with \emph{Festuca}, \emph{Achnatherum}, and \emph{Danthonia}). Where fire is less frequent, chaparral shrubland develops (with \emph{Arctostaphylos} and \emph{Quercus breweri}) (LandFire 2007b).

\end{adjustwidth}
\paragraph{Succession Transition}
\begin{adjustwidth}{2cm}{}
\textbf{Mesic and Xeric Modifier } Patches in this seral stage may stay in this seral stage under low mortality disturbance, but after 15 years without fire they begin transitioning to LDC at a rate of 0.8 per time step. 

\medskip
\noindent \textbf{Ultramafic Modifier} Succession may be delayed. Thus, in the absence of low mortality disturbance, patches in the LDM seral stage will begin transitioning to LDC after 20 years in LDM at a rate of 0.7 per timestep. 

\end{adjustwidth}
\paragraph{Wildfire Transition}
\begin{adjustwidth}{2cm}{}
\textbf{Mesic Modifier } High mortality wildfire (11\% of fires in this seral stage) recycles the patch through the ED seral stage. Low mortality wildfire (89\%) triggers a transition to LDO 17\% of the time; otherwise, it remains in LDM.

\medskip
\noindent \textbf{Xeric Modifier} High mortality wildfire (10\% of fires in this seral stage) recycles the patch through the ED seral stage. Low mortality wildfire (90\%) triggers a transition to LDO 15\% of the time; otherwise, it remains in LDM.

\medskip
\noindent \textbf{Ultramafic Modifier}  High mortality wildfire (11\% of fires) recycles the patch through the ED seral stage. Low mortality wildfire (89\%) triggers a transition to LDO 17\% of the time; otherwise, it remains in LDM.

\end{adjustwidth}
\noindent\hrulefill

\paragraph{Late Development - Closed Canopy Cover (LDC)}

\paragraph{Description}
\begin{adjustwidth}{2cm}{}
\textbf{Mesic Modifier } Overstory of large and very large trees, primarily \emph{P. menziesii}. Canopy cover exceeds 70\%. \emph{P. lambertiana} also occurs. \emph{N. densiflorus} is tolerant of both full sun and shade, and usually dominates the subcanopy at this stage. Co-dominance of the upper canopy with \emph{P. menziesii} is uncommon but possible after extended periods without disturbance (Uchytil 1991, LandFire 2007a). There is also some evidence that the senescence of late development \emph{N. densiflorus} may cause openings in the canopy and allow for continued \emph{P. menziesii} dominance (Estes pers. comm. 2013). \emph{Quercus} and \emph{Arctostaphylos} species may also be present in the sub-canopy (LandFire 2007a).

\medskip
\noindent \textbf{Xeric Modifier} Overstory of large and very large trees, often with canopy cover over 70\%. \emph{P. menziesii}, \emph{Q. chrysolepis}, and \emph{Arctostaphylos mewukka} may occur. Conifers are taller and larger than in MD and clearly form the upper canopy layer here. Shrubs persist in openings but those in shade are likely to begin senescing (LandFire 2007a). On ultramafic sites, large \emph{Pinus ponderosa} may additionally be present. Grass savannah persists on sites experiencing low intensity fire (with \emph{Festuca}, \emph{Achnatherum}, and \emph{Danthonia}). Where fire is less frequent, chaparral shrubland develops (with \emph{Arctostaphylos} and \emph{Quercus breweri}) (LandFire 2007b).

\medskip
\noindent \textbf{Ultramafic Modifier} On ultramafic sites, large \emph{Pinus ponderosa}, \emph{Pinus sabiniana}, or \emph{Pinus jeffreyi} may be present along with \emph{P. menziesii} and \emph{N. densiflorus} var. \emph{echinoides}. Grass savannah persists on sites experiencing low intensity fire (with \emph{Festuca}, \emph{Achnatherum}, and \emph{Danthonia}). Where fire is less frequent, chaparral shrubland develops (with \emph{Arctostaphylos} and \emph{Quercus breweri}) (LandFire 2007b).

\end{adjustwidth}

\paragraph{Succession Transition}
\begin{adjustwidth}{2cm}{}
\textbf{Mesic, Xeric, and Ultramafic Modifier } In the absence of disturbance, patches in this seral stage will remain in this seral stage. 

\end{adjustwidth}

\paragraph{Wildfire Transition}
\begin{adjustwidth}{2cm}{}
\textbf{Mesic Modifier } High mortality wildfire (21\% of fires in this seral stage) recycles the patch through the ED seral stage. Low mortality wildfire (79\%) triggers a transition to LDM 26\% of the time; otherwise, it remains in LDC.

\medskip
\noindent \textbf{Xeric Modifier} High mortality wildfire (21\% of fires in this seral stage) recycles the patch through the ED seral stage. Low mortality wildfire (79\%) triggers a transition to LDM 24\% of the time; otherwise, it remains in LDC.

\medskip
\noindent \textbf{Ultramafic Modifier} High mortality wildfire (11\% of fires) recycles the patch through the ED seral stage. Low mortality wildfire (79\%) triggers a transition to LDM 26\% of the time; otherwise, it remains in LDC.

\end{adjustwidth}
\noindent\hrulefill

%\newpage

\subsection*{Seral Stage Classification}
\begin{table}[hbp]
\footnotesize
\centering
\caption{Classification of seral stage for MEG. Diameter at Breast Height (DBH) and Cover From Above (CFA) values taken from EVeg polygons. DBH categories are: null, 0-0.9'', 1-4.9'', 5-9.9'', 10-19.9'', 20-29.9'', 30''+. CFA categories are null, 0-10\%, 10-20\%, \dots , 90-100\%. Each row in the table below should be read with a boolean AND across each column.}
\label{meg_classification}
\begin{tabular}{@{}lrrrrr@{}}
\toprule
\textbf{\begin{tabular}[l]{@{}l@{}}Cover \\ Condition\end{tabular}} & \textbf{\begin{tabular}[r]{@{}r@{}}Overstory Tree \\ Diameter 1 \\ (DBH)\end{tabular}} & \textbf{\begin{tabular}[r]{@{}r@{}}Overstory Tree \\ Diameter 2 \\ (DBH)\end{tabular}} & \textbf{\begin{tabular}[r]{@{}r@{}}Total Tree\\ CFA (\%)\end{tabular}} & \textbf{\begin{tabular}[r]{@{}r@{}}Conifer \\ CFA (\%)\end{tabular}} & \textbf{\begin{tabular}[r]{@{}r@{}}Hardwood \\ CFA (\%)\end{tabular}} \\ \midrule
Early All        & 0-4.9''         & any & any    & any & any \\
Mid Open         & 5-19.9''        & any & 0-40   & any & any \\
Mid Moderate     & 5-19.9''        & any & 40-70  & any & any \\
Mid Closed       & 5-19.9''        & any & 70-100 & any & any \\
Late Open        & 20-40''+        & any & 0-40   & any & any \\
Late Moderate    & 20-40''+        & any & 40-70  & any & any \\
Late Closed      & 20-40''+        & any & 70-100 & any & any \\ \bottomrule
\end{tabular}
\end{table}


\clearpage
\subsection*{References}

\begin{hangparas}{.25in}{1} 
\interlinepenalty=10000
Allen-Diaz, Barbara, Richard Standiford, and Randall D. Jackson. ``Oak Woodlands and Forests.'' In \emph{Terrestrial Vegetation of California, 3rd Edition}, edited by Michael Barbour, Todd Keeler-Wolf, and Allan A. Schoenherr, 313-338. Berkeley and Los Angeles: University of California Press, 2007. 

``CalVeg Zone 1.'' Vegetation Descriptions. \emph{Vegetation Classification and Mapping}.  11 December 2008. U.S. Forest Service. \burl{http://www.fs.usda.gov/Internet/FSE\_DOCUMENTS/fsbdev3\_046448.pdf}. Accessed 2 April 2013.

Estes, Becky, Province Ecologist, USDA Forest Service. Personal communication, 15 August 2013 and 3 September 2013.

LandFire. ``Biophysical Setting Models.'' Biophysical Setting 0610430: Mediterranean California Mixed Evergreen Forest. 2007a. LANDFIRE Project, U.S. Department of Agriculture, Forest Service; U.S. Department of the Interior. \burl{http://www.landfire.gov/national\_veg\_models\_op2.php}. Accessed 9 November 2012.

LandFire. ``Biophysical Setting Models.'' Biophysical Setting 0711700: Klamath-Siskiyou Xeromorphic Serpentine Savanna and Chaparral. 2007b. LANDFIRE Project, U.S. Department of Agriculture, Forest Service; U.S. Department of the Interior. \burl{http://www.landfire.gov/national\_veg\_models\_op2.php}. Accessed 30 November 2012.

Mallek, Chris, Hugh Safford, Joshua Viers, and Jay Miller. ``Modern departures in fire severity and area vary by forest type, Sierra Nevada and southern Cascades, California, USA.'' Ecosphere 4.12 (2013): art153. doi: http://www.esajournals.org/doi/pdf/10.1890/ES13-00217.1. 

Manos, P. S., C. H. Cannon, and S. H. Oh. ``Phylogenetic relationships and taxonomic status of the paleoendemic Fagaceae of western North America: recognition of a new genus, Notholithocarpus.'' Madroño 55.3 (2008): 181-190. doi: 10.3120/0024-9637-55.3.181

McDonald, Philip M. ``Montane Hardwood (MHW).'' \emph{A Guide to Wildlife Habitats of California}, edited by Kenneth E. Mayer and William F. Laudenslayer. California Deparment of Fish and Game, 1988. \burl{http://www.dfg.ca.gov/biogeodata/cwhr/pdfs/MHW.pdf}. Accessed 4 December 2012.

Merriam, Kyle. Province Ecologist, USDA Forest Service. Personal communication, 9 July 2013.

O'Geen, Anthony T., Randy A. Dahlgren, and Daniel Sanchez-Mata. ``California Soils and Examples of Ultramafic Vegetation.'' In \emph{Terrestrial Vegetation of California, 3rd Edition}, edited by Michael Barbour, Todd Keeler-Wolf, and Allan A. Schoenherr, 71-106. Berkeley and Los Angeles: University of California Press, 2007. 

Raphael, Martin G. ``Douglas-Fir (DFR).'' \emph{A Guide to Wildlife Habitats of California}, edited by Kenneth E. Mayer and William F. Laudenslayer. California Deparment of Fish and Game, 1988. \burl{http://www.dfg.ca.gov/biogeodata/cwhr/pdfs/DFR.pdf}. Accessed 4 December 2012.

Safford, Hugh. Regional Ecologist, USDA Forest Service. Personal communication, 15 August 2013.

Skinner, Carl N. and Chi-Ru Chang. ``Fire Regimes, Past and Present.'' \emph{Sierra Nevada Ecosystem Project: Final report to Congress, vol. II, Assessments and scientific basis for management options}. Davis: University of California, Centers for Water and Wildland Resources, 1996.

Tappeiner, John C., Philip M. McDonald, Douglass F. Roy. ``Tanoak.'' Silvics of North America: 2. Hardwoods. Agriculture Handbook 654. Burns, Russell M., and Barbara H. Honkala, tech. cords. U.S. Department of Agriculture, Forest Service, 1990. \burl{http://www.na.fs.fed.us/spfo/pubs/silvics\_manual/volume\_2/quercus/chrysolepis.htm}. Accessed 7 December 2012.

Uchytil, Ronald J. ``Pseudotsuga menziesii var. menziesii''.  \emph{Fire Effects Information System}, U.S. Department of Agriculture, Forest Service,  Rocky Mountain Research Station, Fire Sciences Laboratory, 1991. \burl{http://www.fs.fed.us/database/feis/plants/tree/quekel/all.html}. Accessed 21 December 2012.

Van de Water, Kip M. and Hugh D. Safford. ``A Summary of Fire Frequency Estimates for California Vegetation Before Euro-American Settlement.'' \emph{Fire Ecology} 7.3 (2011): 26-57. doi: 10.4996/fireecology.0703026.
\end{hangparas}

%% !TEX root = master.tex
\newpage
\section{Montane Riparian (MRIP)}
\label{mrip-description}

\subsection*{General Information}

\subsubsection{Cover Type Overview}

\textbf{Montane Riparian (MRIP)}
\newline
Crosswalks
\begin{itemize}
	\item EVeg: Regional Dominance Type 1
	\begin{itemize}
		\item Riparian Mixed Hardwood
		\item White Alder
		\item Willow
		\item Black Cottonwood
		\item Willow - Alder
		\item Mountain Alder
		\item Willow (Shrub)
	\end{itemize}

	\item LandFire BpS Model
	\begin{itemize}
		\item 0611520: California Montane Riparian Systems 
	\end{itemize}

	\item Presettlement Fire Regime Type
	\begin{itemize}
		\item N/A
	\end{itemize}
\end{itemize}

\noindent Reviewed by Sarah Sawyer, Assistant Pacific Southwest Regional Ecologist, USDA Forest Service

\subsubsection{Vegetation Description}
This system often occurs as a highly variable mosaic of multiple communities that are tree-dominated with a diverse shrub component. The variety of plant associations connected to this system reflect elevation, stream gradient, floodplain width, and flooding events. Usually, the montane riparian zone occurs as a narrow, often dense grove of broad-leaved, winter deciduous trees with a sparse understory. At high mountain elevations, there are usually more shrubs in the understory. At high elevations, the type may not be well developed or may occur in the shrub stage only (LandFire 2007, Grenfell 1988). Due to the methodology of assigning the landscape to particular landcover types, the montane riparian type is limited to those sites determined to be dominated by the species assemblages listed in the above crosswalk section. While we recognize that the riparian zone commonly includes areas near watercourses that are dominated by conifers and other trees, for the purposes of this model those sites have been sorted into the pertinent landcover type in accordance with the dominant vegetation observed. We do not have the capacity at this time to groundtruth or map riparian zones based on understory or midstory vegetation.

Characteristic species are many, including those from the following genera: \emph{Acer}, \emph{Alnus}, \emph{Cornus}, \emph{Populus}, \emph{Rhododendron}, and \emph{Salix}. These habitats can occur as \emph{Alnus} or \emph{Salix} stringers along streams of seeps. In other situations an overstory of \emph{Populus} and/or \emph{Alnus} may be present (Grenfell 1988). Other tree species may include \emph{Pseudotsuga menziesii}, \emph{Platanus racemosa}, and \emph{Quercus agrifolia}. At lower elevations, the riparian areas may contain \emph{Arbutus menziesii}, \emph{Lithocarpus densiflorus}, \emph{Umbellularia californica}, \emph{Cornus}, \emph{Acer} and \emph{Fraxinus}. \emph{Salix} species are common throughout, following a series of species as elevation increases (LandFire 2007).


\subsubsection{Distribution}
MRIP is associated with montane lakes, ponds, seeps, bogs and meadows as well as rivers, streams and springs. Water may be permanent or ephemeral. The transition between MRIP and adjacent non-riparian vegetation may be abrupt, especially where the topography is steep. Typically, this vegetation type occurs below 2440 m (8000 ft) (Grenfell 1988).

\subsection*{Disturbances}

\subsubsection{Wildfire}
Fire frequency is highly variable within the riparian zone. Factors that include but are not limited to topography, elevation, climate, dominant vegetation, and existing vegetation all affect fire frequency and intensity. Riparian zones are heavily influenced by the fire regime of adjacent landcover types and so are still susceptible to disturbance by wildfire, even frequent and high mortality fires. Streams also act as an inhibitor of fire spread, thus contributing to spatial and temporal diversity of landscapes beyond what their relative area would suggest (Grenfell 1988). 

In some forested riparian areas, pre-fire suppression fire return intervals were likely lower than adjacent uplands, while in others, fire frequency appears to have been comparable in riparian and upland areas. FRI values are shorter for riparian zones bordering narrow streams compared to zones around wider and deeper streams. In arid ecosystems, FRIs may be shorter than the surrounding areas in part because the increased productivity of these sites results in more fuels to carry fire. Lower elevation and adjacency to fire-tolerant vegetation also contribute to shorter FRIs for some riparian areas (Sawyer 2013).

Estimates of fire rotations are available from the LandFire project and a review paper (LandFire 2007, Van de Water and Safford (2011). The LandFire project’s published fire return intervals are based on a series of associated models created using the Vegetation Dynamics Development Tool (VDDT). In VDDT, fires are specified concurrently with the transition that follows them. For example, a replacement fire causes a transition to the early development stage. In the RMLands model, such fires are classified as high mortality. However, in VDDT mixed severity fires may cause a transition to early development, a transition to a more open seral stage, or no transition at all. In this case, we categorize the first example as a high mortality fire, and the second and third examples as a low mortality fire. Based on this approach, we calculated fire rotations and the probability of high mortality fire for each of the three MRIP seral stages (Table~\ref{tab:mripdesc_fire}). We computed the overall target fire rotation of 53 years based on values from Van de Water and Safford (2011). 




\begin{table}[!htbp]
\footnotesize
\centering
\caption{Fire rotation index values and probability of high severity fire (at least 75\% overstory tree mortality) probabilities. The seral stage that is most susceptible to fire (i.e., has the lowest predicted fire rotation) has a fire rotation index value of 1. Higher values correspond with lower likelihoods of experiencing wildfire. The values are relative only within an individual seral stage and should not be compared against other land cover types. Values were derived from VDDT model 0611520 (LandFire 2007) and Van de Water and Safford (2011). }
\label{tab:mripdesc_fire}
\begin{tabular}{@{}lcc@{}}
\toprule
 \textbf{Seral Stage}    & \textbf{\begin{tabular}[c]{@{}c@{}}Fire Rotation \\ Index\end{tabular}} & \textbf{\begin{tabular}[c]{@{}c@{}}Probability of \\ High Severity Fire\end{tabular}} \\ \hline
Early (All)       & 1.0  & 1        \\
Mid--Open   		& 1.0  & 0.5        \\
Late--Closed      & 1.0  & 0.5      \\ 
\emph{Target Fire Rotation}    			& \emph{53 years}  &   \\ 
\bottomrule
\end{tabular}
\end{table}

\subsubsection{Other Disturbance}
Other disturbances are not currently modeled, but may, depending on the seral stage affected and mortality levels, reset patches to early development, maintain existing stages, or shift/accelerate succession to a more open stage. 

\subsection*{Vegetation Seral Stages}
We recognize three separate seral stages for MRIP: Early Development (ED), Mid Development - Open Canopy Cover (MDO), and Late Development - Open Canopy Cover (LDO) (Figure~\ref{mrip_transmodel}). Our seral stages are an alternative to ``successional'' classes that imply a linear progression of states and tend not to incorporate disturbance. The seral stages identified here are derived from a combination of successional processes and anthropogenic and natural disturbance, and are intended to represent a composition and structural condition that can be arrived at from multiple other conditions described for that landcover type. Thus our seral stages incorporate age, size, canopy cover, and vegetation composition. In general, the delineation of stages has originated from the LandFire biophysical setting model descriptive of a given landcover type; however, seral stages are not necessarily identical to the classes identified in those models.

\begin{figure}[htbp]
\centering
\includegraphics[width=0.8\textwidth]{/Users/mmallek/Documents/Thesis/statetransmodel/StateTransitionModel/mrip.png}
\caption{State and Transition Model for Montane Riparian Forest. Each dark grey box represents one of the three seral stages for this landcover type. Three stages of development are represented: early, middle, and late. We describe the middle development stage as characterized by open canopy cover and the late development stage as characterized by closed canopy cover, but these are not hard and fast rules. Transitions between states/seral stages may occur as a result of high mortality fire, low mortality fire, or succession. Specific pathways for each are denoted by the appropriate color line and arrow: red lines relate to high mortality fire, orange lines relate to low mortality fire, and green lines relate to natural succession.} 
\label{mrip_transmodel}
\end{figure}



\paragraph{Early Development (ED)}

\paragraph{Description} Immediate post-disturbance responses are dependent on pre-burn vegetation composition. Typically tree dominated, but shrubs may co-dominate. \emph{Salix} and \emph{Alnus} are common, though overall composition is highly variable (LandFire 2007).

\paragraph{Succession Transition} In the absence of disturbance, patches in this seral stage will transition to MDO at 10 years.

\paragraph{Wildfire Transition} High mortality wildfire (100\% of fires in this seral stage) recycles the patch through the ED seral stage. Low mortality wildfire is not modeled for this seral stage.

\noindent\hrulefill


\paragraph{Mid Development - Open Canopy Cover (MDO)}

\paragraph{Description} Vegetation composition in this seral stage includes tall trees and shrubs. \emph{Salix}, \emph{Populus}, and \emph{Alnus} are common. Patches in MDO are more susceptible to fire than the early seral stage (LandFire 2007).

\paragraph{Succession Transition} After 20 years without a wildfire-triggered transition, patches in this seral stage will succeed to LDO.

\paragraph{Wildfire Transition} High mortality wildfire (50\% of fires in this seral stage) recycles the patch through the ED seral stage. Low mortality wildfire (50\%) does not effect a change in the MDO seral stage.

\noindent\hrulefill


\paragraph{Late Development - Open Canopy Cover (LDO)}

\paragraph{Description} This class represents the mature, large \emph{Populus}, \emph{Alnus}, etc. woodlands (LandFire 2007).

\paragraph{Succession Transition} In the absence of disturbance, patches in this seral stage will maintain, regardless of soil characteristics.

\paragraph{Wildfire Transition} High mortality wildfire (50\% of fires) recycles the patch through the ED seral stage. Low mortality wildfire (50\%) does not effect a change in the LDO seral stage.

\noindent\hrulefill





\subsection*{Seral Stage Classification}
\begin{table}[hbp]
\footnotesize
\centering
\caption{Classification of seral stage for MEG. Diameter at Breast Height (DBH) and Cover From Above (CFA) values taken from EVeg polygons. DBH categories are: null, 0-0.9'', 1-4.9'', 5-9.9'', 10-19.9'', 20-29.9'', 30''+. CFA categories are null, 0-10\%, 10-20\%, \dots , 90-100\%. Each row in the table below should be read with a boolean AND across each column.}
\label{mrip_classification}
\begin{tabular}{@{}lrrrrr@{}}
\toprule
\textbf{\begin{tabular}[l]{@{}l@{}}Cover \\ Condition\end{tabular}} & \textbf{\begin{tabular}[r]{@{}r@{}}Overstory Tree \\ Diameter 1 \\ (DBH)\end{tabular}} & \textbf{\begin{tabular}[r]{@{}r@{}}Overstory Tree \\ Diameter 2 \\ (DBH)\end{tabular}} & \textbf{\begin{tabular}[r]{@{}r@{}}Total Tree\\ CFA (\%)\end{tabular}} & \textbf{\begin{tabular}[r]{@{}r@{}}Conifer \\ CFA (\%)\end{tabular}} & \textbf{\begin{tabular}[r]{@{}r@{}}Hardwood \\ CFA (\%)\end{tabular}} \\ \midrule
Early            & Null           & any & any & any & any \\
Early            & 0-9.9''         & any & any & any & any \\
Mid Open         & 10-19.9''       & any & any & any & any \\
Late Open        & 20-30''+        & any & any & any & any \\ \bottomrule
\end{tabular}
\end{table}


\subsection*{References}
\begin{hangparas}{.25in}{1} 
\interlinepenalty=10000
Grenfell, Jr., William E. ``Montane Riparian (MRI).'' \emph{A Guide to Wildlife Habitats of California}, edited by Kenneth E. Mayer and William F. Laudenslayer. California Deparment of Fish and Game, 1988. \burl{http://www.dfg.ca.gov/biogeodata/cwhr/pdfs/MRI.pdf}. Accessed 4 December 2012.

LandFire. ``Biophysical Setting Models.'' Biophysical Setting 0611520: California Montane Riparian Systems. 2007. LANDFIRE Project, U.S. Department of Agriculture, Forest Service; U.S. Department of the Interior. \burl{http://www.landfire.gov/national_veg_models_op2.php}. Accessed 9 November 2012.

Sawyer, Sarah C. ``Natural Range of Variation of Non-Meadow Riparian Habitat in the Bioregional Assessment Area'' (unpublished paper, Ecology Group, Pacific Southwest Research Station, 2013).

Skinner, Carl N. and Chi-Ru Chang. ``Fire Regimes, Past and Present.'' \emph{Sierra Nevada Ecosystem Project: Final report to Congress, vol. II, Assessments and scientific basis for management options}. Davis: University of California, Centers for Water and Wildland Resources, 1996.

Van de Water, Kip and Malcom North. ``Fire history of coniferous riparian forests in the Sierra Nevada.'' \emph{Forest Ecology and Management} 260: 384-395. 2010.

\end{hangparas}





%% !TEX root = master.tex
\newpage
\section{Oak Woodland (OAK)}
\label{oak-description}

\subsection*{General Information}

\subsubsection{Cover Type Overview}

\textbf{Oak Woodland (OAK)} 
\newline
\textbf{Crosswalks}
\begin{itemize}
	\item EVeg: Regional Dominance Type 1
	\begin{itemize}
		\item Gray Pine
		\item Blue Oak
		\item Valley Oak
	\end{itemize}

	\item LandFire BpS Model
	\begin{itemize}
		\item 0611140: California Lower Montane Blue Oak-Foothill Pine Woodland and Savanna
	\end{itemize}

	\item Presettlement Fire Regime Type
	\begin{itemize}
		\item Oak Woodland
	\end{itemize}
\end{itemize}

\noindent Reviewed by Becky Estes, Central Sierra Province Ecologist, USDA Forest Service

\subsubsection{Vegetation Description}
The Oak Woodland landcover type is characterized by savannas, woodlands, or forests of either monospecific or mixed stands of various oak species. \emph{Quercus douglasii}, \emph{Quercus lobata}, \emph{Quercus wislizenii}, and \emph{Quercus garryana} are the major dominants. In oak forests where mixtures of tree oak and conifer species exist \emph{Quercus kelloggii} and \emph{Quercus chrysolepis} occur along with \emph{Pinus sabiniana} (Allen-Diaz et al. 2007). 

Both \emph{Q. douglasii} and \emph{Q. lobata} are endemic to California. \emph{Q. lobata} are among the oldest and largest oaks in North America. Tree age can exceed 500 years. \emph{Q. douglasii} are relatively slow-growing, long-lived trees. On \emph{Q. douglasii-P. sabiniana} woodlands, \emph{P. sabiniana} is taller and dominates the overstory, but is shorter-lived (at approximately 80 years) than \emph{Q. douglasii} (150-250 years). \emph{Q. douglasii} is usually the more abundant of the two trees, but \emph{P. sabiniana} contributes as much basal area as \emph{Q. douglasii} (Allen-Diaz et al. 2007).

Typical vegetation is dominated by open oak savannah with relatively uniform mature trees at low densities (less than 40\% cover), where understory vegetation structure is a function of frequent surface fire that mediates woody plant development. In some instances and in some sites tree density will increase to 70\% or more, forming a relatively stable hardwood forest type subject to surface fires in the hardwood litter and rare stand replacement fire (LandFire 2007).

In riparian forests, associates include \emph{Platanus racemosa}, \emph{Juglans hindsii}, \emph{Acer negundo}, \emph{Populus fremontii}, \emph{Salix}, and \emph{Fraxinus latifolia}. In drier areas and open woodlands, shrubs usually clump together in open areas with full sun. Species may include \emph{Aesculus californica}, \emph{Ceanothus}, \emph{Arctostaphylos}, \emph{Rhamnus}, \emph{Toxicodendron diversilobum}, and \emph{Cercis occidentalis} (Allen-Diaz et al. 2007). The shrub layer is best developed along natural drainages, becoming insignificant in the uplands. Ground cover consists of a well-developed carpet of grasses and forbs (Ritter 1988b). Common forbs include \emph{Daucus}, \emph{Geranium}, \emph{Madia}, and \emph{Trifolium}. Most understory cover is created by annual grasses, including \emph{Bromus}, \emph{Lolium}, and \emph{Hordeum} (Allen-Diaz et al. 2007).

Oak recruitment is poor in many areas today, due to both natural and human causes. Many stands exist as groups of medium-to-large trees with few or no young oaks. There is concern that these woodlands may be slowly changing into savannas and grasslands as trees die and are not replaced. Mortality of oak saplings seems to be related to competition for moisture with grasses and forbs, wild and domestic animals feeding on acorns and seedlings, fire suppression, and flood control. Most recent work suggests that recruitment is limited not by reproduction, but by the establishment and survival of saplings (Allen-Diaz et al. 2007).


\subsubsection{Distribution}
Oak Woodland has a patchy distribution embedded in a matrix of agriculture, urban development, grasslands, riparian forests, and other conifer and oak woodland types. It occurs in a band along the western Sierra Nevada foothills, generally below 800 m (2642 feet) in elevation, although individual species described here are capable of surviving at higher elevations. In general, tree density is highest along natural drainages with deeper soils, and lower in uplands and on steeper slopes. The transition from savanna to woodland to forest is largely driven by soil, precipitation, and elevation (Allen-Diaz et al. 2007).

Soils in this type vary significantly, with different types conducive to the establishment of differing dominant tree species. \emph{Q. lobata} is best developed on deep, well-drained alluvial soils, usually in valley bottoms (Ritter 1988b). \emph{Q. wislizeni} becomes more abundant on steeper slopes, shallower soils, and at higher elevations. \emph{Q. douglasii} woodlands occur on a wide range of soils; however, they are often shallow, rocky, infertile, and well drained. The overstory ranges from sparsely scattered trees on poor sites to nearly closed canopies on good quality sites (Allen-Diaz et al. 2007, Ritter 1988a). \emph{Q. douglasii-P. sabiniana} woodlands are found on variety of generally well-drained parent materials, ranging from gravelly loam through stony clay loam. They occupy steeper, drier slopes with shallower and rockier soils than pure oak woodlands (Verner 1988). 


\subsection*{Disturbances}

\subsubsection{Wildfire}
An overstory dominated by deciduous hardwood species results in an herbaceous surface fuel complex dominating fuel/fire influences (LandFire 2007). Because of the long period of human habitation of oak woodlands, it is extremely difficult to define the ``natural'' fire regime. Lightning-caused fires certainly occurred in the past, but decades may pass between these events. Native Americans used fire in their stewardship of oak woodlands; however, it is difficult to document the frequency, intensity, and extent of burning by Native Americans. Some estimate the fire return interval (FRI) of that period to be around 25 years. The first European settlers continued to use fire as a management practice; burning intervals ranged from 8-15 years. Ranchers continued the practice through the 1950s, but since then fire suppression has emerged as the standard management policy (Allen-Diaz et al. 2007).  

The fire regime which produced this landcover type is thought to be frequent; mortality depends on vegetation vulnerability and wildfire intensity. Younger oaks are fire-sensitive and frequently killed by even low severity fires. However, they typically sprout post-disturbance. Older, decadent oaks are not likely to sprout after being damaged or killed by fire. Therefore, younger stands are more likely to regrow after fires and fire exclusion can have a significant effect on stand structure. \emph{P. sabiniana}’s regeneration is dependent on regeneration from seed, although it, too, is fire-adapted. It also grows faster than \emph{Q. douglasii} and is an important colonizer (Allen-Diaz et al. 2007). 

Estimates of fire rotations are available from the LandFire project and Mallek et al. (2013). The LandFire project’s published fire return intervals are based on a series of associated models created using the Vegetation Dynamics Development Tool (VDDT). In VDDT, fires are specified concurrently with the transition that follows them. For example, a replacement fire causes a transition to the early development stage. In the RMLands model, such fires are classified as high mortality. However, in VDDT mixed severity fires may cause a transition to early development, a transition to a more open seral stage, or no transition at all. In this case, we categorize the first example as a high mortality fire, and the second and third examples as a low mortality fire. Based on this approach, we calculated fire rotations and the probability of high mortality fire for each of the OAK seral stages (Table~\ref{tab:oakdesc_fire}). We computed the overall target fire rotation of 26 years based on values from Mallek et al. (2013). 




\begin{table}[!htbp]
\footnotesize
\centering
\caption{Fire rotation index values and probability of high severity fire (at least 75\% overstory tree mortality) probabilities. The seral stage that is most susceptible to fire (i.e., has the lowest predicted fire rotation) has a fire rotation index value of 1. Higher values correspond with lower likelihoods of experiencing wildfire. The values are relative only within an individual seral stage and should not be compared against other land cover types. Values were derived from BpS model 0610800 (LandFire 2007), Van de Water and Safford (2011), and Safford (pers. comm. 2013).}
\label{tab:oakdesc_fire}
\begin{tabular}{@{}lcc@{}}
\toprule
 \textbf{Seral Stage}    & \textbf{\begin{tabular}[c]{@{}c@{}}Fire Rotation \\ Index\end{tabular}} & \textbf{\begin{tabular}[c]{@{}c@{}}Probability of \\ High Severity Fire\end{tabular}} \\ \hline
Early (All)     		 & 1.3           & 0.01                          \\
Mid--Closed    			 & 1.5           & 0.07                          \\
Mid--Moderate  			 & 1.4           & 0.06                          \\
Mid--Open      			 & 1.2           & 0.05                          \\
Late--Closed   			 & 3.3           & 0.5                           \\
Late--Moderate 			 & 1.5           & 0.18                          \\
Late--Open     			 & 1.0             & 0.08       \\ 
\emph{Target Fire Rotation}    			& \emph{26 years}  &   \\ 
\bottomrule
\end{tabular}
\end{table}

\subsubsection{Other Disturbance}
Other disturbances are not currently modeled, but may, depending on the seral stage affected and mortality levels, reset patches to early development, maintain existing seral stages, or shift/accelerate succession to a more open condition. 

\subsection*{Vegetation Seral Stages}
We recognize seven separate seral stages for OAK: Early Development (ED), Mid Development - Open Canopy Cover (MDO), Mid Development - Moderate Canopy Cover, Mid Development - Closed Canopy Cover (MDC), Late Development - Open Canopy Cover (LDO), Late Development - Moderate Canopy Cover (LDM), and Late Development - Closed Canopy Cover (LDC) (Figure~\ref{oak_transmodel}). Our seral stages are an alternative to ``successional'' classes that imply a linear progression of states and tend not to incorporate disturbance. The seral stages identified here are derived from a combination of successional processes and anthropogenic and natural disturbance, and are intended to represent a composition and structural condition that can be arrived at from multiple other conditions described for that landcover type. Thus our seral stages incorporate age, size, canopy cover, and vegetation composition. In general, the delineation of stages has originated from the LandFire biophysical setting model descriptive of a given landcover type; however, seral stages are not necessarily identical to the classes identified in those models.


\begin{figure}[htbp]
\centering
\includegraphics[width=0.8\textwidth]{/Users/mmallek/Documents/Thesis/statetransmodel/StateTransitionModel/7class.png}
\caption{State and Transition Model for Oak Woodland. Each dark grey box represents one of the seven seral stages for this landcover type. Each column of boxes represents a stage of development: early, middle, and late. Each row of boxes represents a different level of canopy cover: closed (70-100\%), moderate (40-70\%), and open (0-40\%). Transitions between states/seral stages may occur as a result of high mortality fire, low mortality fire, or succession. Specific pathways for each are denoted by the appropriate color line and arrow: red lines relate to high mortality fire, orange lines relate to low mortality fire, and green lines relate to natural succession.} 
\label{oak_transmodel}
\end{figure}

\paragraph{Early Development (ED)}

\paragraph{Description} Post-replacement sapling/regeneration phase. Largely a function of either early seral remaining in early seral due to replacement fire, or due to less common late seral replacement fire. Re-establishment can occur from basal resprouting or sexual reproduction, depending on composition, growth form, and seed dynamics. Patch size likely ranges from very small gap recruitment to areas approximately 100 acres. May include \emph{Q. douglasii}, \emph{Q. chrysolepis}, \emph{Q. garryana}, \emph{P. sabiniana}, and a variety of shrubs (LandFire 2007).


\paragraph{Succession Transition} In the absence of disturbance, patches in this seral stage will begin transitioning to MDM at 20 years at a rate of 0.6 per time step. At 60 years in ED, all remaining patches transition to MDM. On average, patches remain in early development for 28 years.

\paragraph{Wildfire Transition} High mortality wildfire (1\% of fires in this seral stage) recycles the patch through the ED seral stage. No transition occurs as a result of low mortality fire. 

\noindent\hrulefill


\paragraph{Mid Development - Open Canopy Cover (MDO)}

\paragraph{Description} Intermediate phase, older than 20 years. Sparse new recruitment of cohorts occurs in the later stages of this seral stage, leading to an open canopy. Periodic surface fire is relatively common, but replacement fire rare due to low intensity fire type and resilience of typical species to top kill. Patch size is typically in the hundreds of acres. May include \emph{Q. douglasii}, \emph{Q. chrysolepis}, \emph{Q. garryana}, \emph{P. sabiniana}, and a variety of shrubs (LandFire 2007).

\paragraph{Succession Transition} In the absence of stand-replacing disturbance, patches in this seral stage will begin transitioning to MDM at 15 years at a rate of 0.7 per time step. Succession to LDO begins after 40 years in a mid development stage. The rate of succession per time step is 0.7. At 70 years in MDO, all remaining patches transition to LDO. On average, patches remain in mid development for 47 years.

\paragraph{Wildfire Transition} High mortality wildfire (5\% of fires in this seral stage) recycles the patch through the ED seral stage. Low mortality fire (95\%) maintains the MDO seral stage and allows for succession to LDO.

\noindent\hrulefill

\paragraph{Mid Development - Moderate Canopy Cover (MDM)}

\paragraph{Description} Intermediate phase, older than 20 years. Some new recruitment of cohorts occurs in the later stages of this seral stage, resulting in moderate canopy cover. Periodic surface fire is relatively common, but replacement fire rare due to low intensity fire type and resilience of typical species to top kill. Patch size is typically in the hundreds of acres. May include \emph{Q. douglasii}, \emph{Q. chrysolepis}, \emph{Q. garryana}, \emph{P. sabiniana}, and a variety of shrubs (LandFire 2007).

\paragraph{Succession Transition} In the absence of stand-replacing disturbance, patches in this seral stage will begin transitioning to MDC at 15 years at a rate of 0.7 per time step. Succession to LDM begins after 40 years in a mid development stage. The rate of succession per time step is 0.7. At 70 years in MDM, all remaining patches transition to LDM. On average, patches remain in mid development for 47 years.

\paragraph{Wildfire Transition} High mortality wildfire (6\% of fires in this seral stage) recycles the patch through the ED seral stage. Low mortality fire (94\%) maintains the MDM seral stage and allows for succession to LDM.

\noindent\hrulefill

\paragraph{Mid Development - Closed Canopy Cover (MDC)}

\paragraph{Description} Intermediate phase, older than 20 years. Significant new recruitment of cohorts occurs in the later stages of this seral stage, resulting in a closed canopy. Periodic surface fire is relatively common, but replacement fire rare due to low intensity fire type and resilience of typical species to top kill. Patch size is typically in the hundreds of acres. May include \emph{Q. douglasii}, \emph{Q. chrysolepis}, \emph{Q. garryana}, \emph{P. sabiniana}, and a variety of shrubs (LandFire 2007).

\paragraph{Succession Transition} Succession to LDC begins after 40 years in a mid development stage. The rate of succession per time step is 0.7. At 70 years in a mid development stage, all remaining patches transition to LDC. On average, patches remain in mid development for 47 years.

\paragraph{Wildfire Transition} High mortality wildfire (7\% of fires in this seral stage) recycles the patch through the ED seral stage. Low mortality fire (93\%) maintains the MDC seral stage and allows for succession to LDC.

\noindent\hrulefill


\paragraph{Late Development - Open Canopy Cover (LDO)}

\paragraph{Description} Open woodland with mature oak and conifer trees. This seral stage is highly stable, as most fire is frequent, low severity fire acting as a maintenance agent. Tree density and canopy cover increase over time to relatively stable conditions. In some cases woody encroachment and increased tree density occurs under missed fire cycles. If P. sabiniana occurs, it quickly becomes very large. Some replacement fire occurs initiating secondary succession in the ED seral stage. Patch size in the hundreds, to possibly thousands, of acres. Canopy cover ranges from 11-40\%. May include \emph{Q. douglasii, Q. chrysolepis, Q. garryana, P. sabiniana}, and a variety of shrubs (LandFire 2007).

\paragraph{Succession Transition} In the absence of disturbance, patches in this seral stage will begin transitioning to LDM after 15 years at a rate of 0.7 per time step. 

\paragraph{Wildfire Transition} High mortality wildfire (8\% of fires in this seral stage) recycles the patch through the ED seral stage. Low mortality fire (92\%) maintains the LDO seral stage.

\noindent\hrulefill

\paragraph{Late Development - Moderate Canopy Cover (LDM)}

\paragraph{Description} Woodland with mature oak and conifer trees. This seral stage is fairly stable, as fire tends to be frequent, low severity fire acting as a maintenance agent. Tree density and canopy cover are increasing over time due to missed fire cycles or high productivity. Periodic surface fire is relatively common, but replacement fire is uncommon due to low intensity fire type and resilience of typical species to top kill. If \emph{P. sabiniana} occurs, it quickly becomes very large. Patch size is  in the hundreds of acres. Canopy cover ranges from 40-70\%. May include \emph{Q. douglasii}, \emph{Q. chrysolepis}, \emph{Q. garryana}, \emph{P. sabiniana}, and a variety of shrubs (LandFire 2007).

\paragraph{Succession Transition} In the absence of disturbance, patches in this seral stage will begin transitioning to LDC after 15 years at a rate of 0.7 per time step. 

\paragraph{Wildfire Transition} High mortality wildfire (18\% of fires in this seral stage) recycles the patch through the ED seral stage. Low mortality fire (82\%) opens the patch up to LDO 14\% of the time; otherwise, the patch remains in LDM.

\noindent\hrulefill

\paragraph{Late Development - Closed Canopy Cover (LDC)}

\paragraph{Description} Late seral stage arising from a rare period of no fire in the LDM seral stage for at least 15 years, allowing woody understory encroachment and higher tree density. If \emph{P. sabiniana} occurs, it quickly becomes very large. Fire that does not effect a change in seral stage is rare; low mortality fire is the normal pathway back to late development, open seral stages, while high mortality results in a return to early seral conditions. Patch size is likely in the tens of acres. May include \emph{Q. douglasii}, \emph{Q. chrysolepis}, \emph{Q. garryana}, \emph{P. sabiniana}, and a variety of shrubs. If the closed seral stage persists for decades and \emph{P. sabiniana} is present, it can begin to shade out the oak trees (LandFire 2007).

\paragraph{Succession Transition} In the absence of disturbance, patches in this seral stage will maintain.

\paragraph{Wildfire Transition} High mortality wildfire (50\% of fires in this seral stage) recycles the patch through the ED seral stage. Low mortality fire (50\%) opens the patch up to LDM.

\noindent\hrulefill

%\clearpage
\subsection*{Seral Stage Classification}
\begin{table}[hbp]
\footnotesize
\centering
\caption{Classification of seral stage for OAK. Diameter at Breast Height (DBH) and Cover From Above (CFA) values taken from EVeg polygons. DBH categories are: null, 0-0.9'', 1-4.9'', 5-9.9'', 10-19.9'', 20-29.9'', 30''+. CFA categories are null, 0-10\%, 10-20\%, \dots , 90-100\%. Each row in the table below should be read with a boolean AND across each column.}
\label{oak_classification}
\begin{tabular}{@{}lrrrrr@{}}
\toprule
\textbf{\begin{tabular}[l]{@{}l@{}}Cover \\ Condition\end{tabular}} & \textbf{\begin{tabular}[r]{@{}r@{}}Overstory Tree \\ Diameter 1 \\ (DBH)\end{tabular}} & \textbf{\begin{tabular}[r]{@{}r@{}}Overstory Tree \\ Diameter 2 \\ (DBH)\end{tabular}} & \textbf{\begin{tabular}[r]{@{}r@{}}Total Tree\\ CFA (\%)\end{tabular}} & \textbf{\begin{tabular}[r]{@{}r@{}}Conifer \\ CFA (\%)\end{tabular}} & \textbf{\begin{tabular}[r]{@{}r@{}}Hardwood \\ CFA (\%)\end{tabular}} \\ \midrule
Early            & 0-4.9''         & any & any    & any    & any    \\
Mid Open         & 5-9.9''         & any & 0-40   & any    & any    \\
Mid Moderate     & 5-9.9''         & any & 40-70  & any    & any    \\
Mid Closed       & 5-9.9''         & any & 70-100 & any    & any    \\
Late Open        & 10''+           & any & 0-40   & any    & any    \\
Late Open        & 10''+           & any & null   & 0-40   & 0-40   \\
Late Moderate    & 10''+           & any & 40-70  & any    & any    \\
Late Moderate    & 10''+           & any & null   & 40-70  & 0-70   \\
Late Moderate    & 10''+           & any & null   & 0-70   & 40-70  \\
Late Closed      & 10''+           & any & 70-100 & any    & any    \\
Late Closed      & 10''+           & any & null   & 70-100 & any    \\
Late Closed      & 10''+           & any & null   & any    & 70-100 \\ \bottomrule
\end{tabular}
\end{table}


\clearpage

\subsection*{References}

\begin{hangparas}{.25in}{1} 
\interlinepenalty=10000
Allen-Diaz, Barbara, Richard Standiford, and Randall D. Jackson. ``Oak Woodlands and Forests.'' In \emph{Terrestrial Vegetation of California, 3rd Edition}, edited by Michael Barbour, Todd Keeler-Wolf, and Allan A. Schoenherr, 313-338. Berkeley and Los Angeles: University of California Press, 2007. 

``CalVeg Zone 1.'' Vegetation Descriptions. \emph{Vegetation Classification and Mapping}.  11 December 2008. U.S. Forest Service. \burl{http://www.fs.usda.gov/Internet/FSE_DOCUMENTS/fsbdev3_046448.pdf}. Accessed 2 April 2013.

LandFire. ``Biophysical Setting Models.'' Biophysical Setting 0611140: California Lower Montane Blue Oak-Foothill Pine Woodland and Savanna. 2007. LANDFIRE Project, U.S. Department of Agriculture, Forest Service; U.S. Department of the Interior. \burl{http://www.landfire.gov/national_veg_models_op2.php}. Accessed 9 November 2012.

Ritter, Lyman V. ``Blue Oak Woodland (BOW).'' \emph{A Guide to Wildlife Habitats of California}, edited by Kenneth E. Mayer and William F. Laudenslayer. California Deparment of Fish and Game, 1988a. \burl{http://www.dfg.ca.gov/biogeodata/cwhr/pdfs/BOW.pdf}. Accessed 4 December 2012.

Ritter, Lyman V. ``Valley Oak Woodland (VOW).'' \emph{A Guide to Wildlife Habitats of California}, edited by Kenneth E. Mayer and William F. Laudenslayer. California Deparment of Fish and Game, 1988b. \burl{http://www.dfg.ca.gov/biogeodata/cwhr/pdfs/VOW.pdf}. Accessed 4 December 2012.

Skinner, Carl N. and Chi-Ru Chang. ``Fire Regimes, Past and Present.'' \emph{Sierra Nevada Ecosystem Project: Final report to Congress, vol. II, Assessments and scientific basis for management options}. Davis: University of California, Centers for Water and Wildland Resources, 1996.

Van de Water, Kip M. and Hugh D. Safford. ``A Summary of Fire Frequency Estimates for California Vegetation Before Euro-American Settlement.'' \emph{Fire Ecology} 7.3 (2011): 26-57. doi: 10.4996/fireecology.0703026.

Verner, Jared. ``Blue Oak-Foothill Pine (BOP).'' \emph{A Guide to Wildlife Habitats of California}, edited by Kenneth E. Mayer and William F. Laudenslayer. California Deparment of Fish and Game, 1988. \burl{http://www.dfg.ca.gov/biogeodata/cwhr/pdfs/BOP.pdf}. Accessed 4 December 2012. 

\end{hangparas}


%% !TEX root = master.tex
\newpage
\section{Oak-Conifer Forest and Woodland (OCFW)}
\label{ocfw-description}

\subsection*{General Information}

\subsubsection{Cover Type Overview}

\textbf{Oak-Conifer Forest and Woodland (OCFW)}
\newline
Crosswalks
\begin{itemize}
	\item East of the Sierra Crest
	\begin{itemize}
		\item Eveg: Regional Dominance Type 1
		\begin{itemize}
			\item Black Oak
			\item Eastside Pine
			\item Jeffrey Pine
			\item Ponderosa Pine
		\end{itemize}
		\emph{And}
		\item Eveg: Regional Dominance Type 2
		\begin{itemize}
			\item Black Oak
			\item Canyon Live Oak
			\item Madrone
			\item Montane Mixed Hardwood
			\item Scrub Oak
		\end{itemize}
	\end{itemize}

	\item West of the Sierra Crest
	\begin{itemize}
		\item Eveg: Regional Dominance Type 1
		\begin{itemize}
			\item Black Oak
			\item Eastside Pine
			\item Jeffrey Pine
			\item Ponderosa Pine
		\end{itemize}

		\item LandFire BpS Model
		\begin{itemize}
			\item 0610300 Mediterranean California Lower Montane Black Oak-Conifer Forest and Woodland
		\end{itemize}
		
		\item Presettlement Fire Regime Type
		\begin{itemize}
			\item Yellow Pine
		\end{itemize}
\end{itemize}
\end{itemize}

Modifiers
\begin{itemize}
	\item Ultramafic: This type is created by intersecting an ultramafic soils/geology layer with the existing vegetation layer. Where cells intersect with OCFW they are assigned to the ultramafic modifier.
\end{itemize}

\noindent Reviewed by Becky Estes, Central Sierra Province Ecologist, USDA Forest Service; Kyle Merriam, Sierra-Cascade Province Ecologist, USDA Forest Service


\subsubsection{Vegetation Description}
\textbf{Oak-Conifer Forest and Woodland (OCFW)} The Oak-Conifer Forest and Woodland landcover type is characterized by woodlands or forests of \emph{Pinus ponderosa} or \emph{Pinus jeffreyi} with one or more oaks, such as \emph{Quercus kelloggii}, \emph{Quercus garryana}, \emph{Quercus wislizeni}, or \emph{Quercus chrysolepis}. \emph{Pseudotsuga menziesii} and other conifer species are uncommon but may co-occur, especially after long-term fire suppression (LandFire 2007a). \emph{Pinus jeffreyi} tends to dominate on ultramafic sites (Fitzhugh 1988). In some areas, sites are dominated initially by oaks, which form a dense subcanopy. Eventually, and especially on locally mesic sites, conifers will form a persistent emergent canopy over the oak as a bi-layered canopy (LandFire 2007a). In other cases, characteristic species occur in a mosaic-like pattern with small pure stands of conifers interspersed with small stands of broad-leaved trees. Most of the broad-leaved trees are schlerophyllous evergreen, but winter-deciduous species also occur (Anderson 1988). The understory is composed of shrubs such as \emph{Arctostaphylos}, \emph{Ceanothus, Chamaebatia, Cornus, Eriodictyon, Garrya, Prunus, Rhamnus, Ribes,} and \emph{Toxicodendron diversilobum}. Grasses and forbs are diverse and include \emph{Bromus}, \emph{Melica}, \emph{Poa}, \emph{Elymus}, \emph{Carex}, \emph{Collinsia}, \emph{Saltugilia}, \emph{Iris}, \emph{Lupinus}, \emph{Streptanthus}, \emph{Viola}, and \emph{Pteridium aquilnum} (LandFire 2007a, Fitzhugh 1988).

\begin{adjustwidth}{2cm}{}

\textbf{Ultramafic Modifier (OCFW\_U)}  \emph{P. ponderosa} or \emph{P. jeffreyi} woodlands occur mainly on low-elevation ultramafics. They grow on strongly serpentinized soil, and are typically adjacent to the non-ultramafic form of the cover type. While \emph{P. ponderosa} or \emph{P. jeffreyi} dominates, it may be associated with \emph{Calocedrus decurrens, Pinus attentuata, Pinus lambertiana, P. sabiniana}, and \emph{Q. chysolepis} (O'Geen et al. 2007). \emph{Q. kelloggi} is rare on ultramafic soils (Fryer 2007). The shrub layer is dominated by \emph{Arctostaphylos, Ceanothus, Eriodictyon, Heteromeles}, and \emph{Pickeringia}. The herb layer is a mix of sparse perennials and many annual grasses and forbs (O'Geen et al. 2007). 

\end{adjustwidth}


\subsubsection{Distribution}
This type occurs in the valleys and lower slopes of mountainous terrain, on a variety of parent materials including granitics, metamorphic and Franciscan metasedimentary parent material and deep, well developed soils, although rocky soils are also possible. Slopes are generally steep and all aspects are included. In the northern Sierra Nevada the elevational range is 240 to 1800 m (800 to 5000 ft) (LandFire 2007a, Anderson 1988).

\begin{adjustwidth}{2cm}{}

\textbf{Ultramafic Modifier} Ultramafics have been mapped at various spatial densities throughout the elevational range of the OCFW landcover type. Low to moderate elevations in ultramafic and serpentinized areas often produce soils low in essential minerals like calcium potassium, and nitrogen, and have excessive accumulations of heavy metals such as nickel and chromium. These sites vary widely in the degree of serpentinization and effects on their overlying plant communities (``CalVeg Zone 1'' 2011). Note, the terms ``ultramafic rock'' and ``serpentine'' are broad terms used to describe a number of different but related rock types, including serpentinite, peridotite, dunite, pyroxenite, talc and soapstone, among others (O'Geen et al. 2007).


\end{adjustwidth}

\subsection*{Disturbances}

\subsubsection{Wildfire}
Wildfires are common and frequent; mortality depends on vegetation vulnerability and wildfire intensity. Low mortality fires kill small trees and consume above-ground portions of shrubs and herbs, but do not kill large trees or below-ground organs of most shrubs and herbs which promptly re-sprout. High mortality fires kill large as well as small trees, and may kill many of the shrubs and herbs as well. Fire kills the above-ground portions of the shrubs and herbs, but most shrubs and herbs promptly resprout from surviving below-ground organs. Wildfires may trigger transitions between seral stages.

OCFW sites are fire-adapted and had frequent, low severity surface fires prior to fire exclusion in the late nineteenth century. Historically, fire return intervals (FRIs) in \emph{P. ponderosa-Q. kelloggii} forests increased with increasing elevation in the Sierra Nevada, with a tendency towards shorter mean FRIs (5-15 years) on dry, west- and south-facing slopes and longer FRIs (15-25 years) on mesic, east- and north-facing slopes. Mid-elevation forests typically had mixed-severity fires that created patchy mosaics (Fryer 2007).

Estimates of fire rotations for these variants are available from the LandFire project and a few review papers. The LandFire project’s published fire return intervals are based on a series of associated models created using the Vegetation Dynamics Development Tool (VDDT). In VDDT, fires are specified concurrently with the transition that follows them. For example, a replacement fire causes a transition to the early development stage. In the RMLands model, such fires are classified as high mortality. However, in VDDT mixed severity fires may cause a transition to early development, a transition to a more open seral stage, or no transition at all. In this case, we categorize the first example as a high mortality fire, and the second and third examples as a low mortality fire. Based on this approach, we calculated fire rotations and the probability of high mortality fire for each of the OCFW seral stages (including the ultramafic modifier) (Tables~\ref{tab:ocfwdesc_fire} and \ref{tab:ocfwudesc_fire}). We computed overall target fire rotations based on expert input from Safford and Estes, and values from Mallek et al. (2013), and Van de Water and Safford (2011). 




\begin{table}[!htbp]
\footnotesize
\centering
\caption{Fire rotation index values and probability of high severity fire (at least 75\% overstory tree mortality) probabilities for Oak-Conifer Forest and Woodland. The seral stage that is most susceptible to fire (i.e., has the lowest predicted fire rotation) has a fire rotation index value of 1. Higher values correspond with lower likelihoods of experiencing wildfire. The values are relative only within an individual seral stage and should not be compared against other land cover types. Values were derived from VDDT model 0610300 (LandFire 2007a), Mallek et al. (2013), and Safford and Estes (personal communication). }
\label{tab:ocfwdesc_fire}
\begin{tabular}{@{}lcc@{}}
\toprule
 \textbf{Seral Stage}    & \textbf{\begin{tabular}[c]{@{}c@{}}Fire Rotation \\ Index\end{tabular}} & \textbf{\begin{tabular}[c]{@{}c@{}}Probability of \\ High Severity Fire\end{tabular}} \\ \hline
Early (All)     		 & 3.8            & 1                             \\
Mid--Closed    			 & 1.4            & 0.26                          \\
Mid--Moderate  			 & 1.2           & 0.14                          \\
Mid--Open      			 & 1.0           & 0.05                          \\
Late--Closed   			 & 1.9           & 0.20                          \\
Late--Moderate 			 & 1.3           & 0.08                          \\
Late--Open     			 & 1.0           & 0.01        \\ 
\emph{Target Fire Rotation}    			& \emph{21 years}  &   \\ 
\bottomrule
\end{tabular}
\end{table}

\begin{table}[!htbp]
\footnotesize
\centering
\caption{Fire rotation index values and probability of high severity fire (at least 75\% overstory tree mortality) probabilities for Oak-Conifer Forest and Woodland - Ultramafic. The seral stage that is most susceptible to fire (i.e., has the lowest predicted fire rotation) has a fire rotation index value of 1. Higher values correspond with lower likelihoods of experiencing wildfire. The values are relative only within an individual seral stage and should not be compared against other land cover types. Values were derived from VDDT model 0610210 (LandFire 2007b), Mallek et al. (2013), and Safford and Estes (personal communication).}
\label{tab:ocfwudesc_fire}
\begin{tabular}{@{}lcc@{}}
\toprule
 \textbf{Seral Stage}    & \textbf{\begin{tabular}[c]{@{}c@{}}Fire Rotation \\ Index\end{tabular}} & \textbf{\begin{tabular}[c]{@{}c@{}}Probability of \\ High Severity Fire\end{tabular}} \\ \hline
Early (All)     		 & 3.8            & 1                             \\
Mid--Closed    			 & 1.4           & 0.26                          \\
Mid--Moderate  			 & 1.2           & 0.14                          \\
Mid--Open      			 & 1.0           & 0.05                          \\
Late--Closed   			 & 1.9           & 0.20                          \\
Late--Moderate 			 & 1.3           & 0.08                          \\
Late--Open     			 & 1.0           & 0.01        \\ 
\emph{Target Fire Rotation}    			& \emph{21 years}  &   \\ 
\bottomrule
\end{tabular}
\end{table}

\subsubsection{Other Disturbance}

\subsection*{Vegetation Seral Stages}
We recognize seven separate seral stages for OCFW and OCFW\_U: Early Development (ED), Mid Development - Open Canopy Cover (MDO), Mid Development - Moderate Canopy Cover, Mid Development - Closed Canopy Cover (MDC), Late Development - Open Canopy Cover (LDO), Late Development - Moderate Canopy Cover (LDM), and Late Development - Closed Canopy Cover (LDC) (Figure~\ref{transmodel_ocfw}). Our seral stages are an alternative to ``successional'' classes that imply a linear progression of states and tend not to incorporate disturbance. The seral stages identified here are derived from a combination of successional processes and anthropogenic and natural disturbance, and are intended to represent a composition and structural condition that can be arrived at from multiple other conditions described for that landcover type. Thus our seral stages incorporate age, size, canopy cover, and vegetation composition. In general, the delineation of stages has originated from the LandFire biophysical setting model descriptive of a given landcover type; however, seral stages are not necessarily identical to the classes identified in those models.

\begin{figure}[htbp]
\centering
\includegraphics[width=0.8\textwidth]{/Users/mmallek/Documents/Thesis/statetransmodel/StateTransitionModel/7class.png}
\caption{State and Transition Model for Oak-Conifer Forest and Woodland. Each dark grey box represents one of the seven seral stages for this landcover type. Each column of boxes represents a stage of development: early, middle, and late. Each row of boxes represents a different level of canopy cover: closed (70-100\%), moderate (40-70\%), and open (0-40\%). Transitions between states/seral stages may occur as a result of high mortality fire, low mortality fire, or succession. Specific pathways for each are denoted by the appropriate color line and arrow: red lines relate to high mortality fire, orange lines relate to low mortality fire, and green lines relate to natural succession.} 
\label{transmodel_ocfw}
\end{figure}

\paragraph{Early Development (ED)} 


\paragraph{Description}
The early seral stage is the initial post-disturbance community dominated by coppicing oak sprouts (predominantly \emph{Q. kelloggi}, but potentially also \emph{Q. chrysolepis}). \emph{T. diversilobum} may be abundant. Bunchgrasses and associated forbs dominate understory. Localized native herbivory may maintain oak sprouts in ``shrub'' form for extended period. Vegetation may also include conifer seedling/saplings (LandFire 2007a).

On sites or areas that are dry or of low quality, significant pine regeneration may depend on concurrent disturbance of shrub species and a good pine seed crop with favorable weather. Thus, it may require 50-100 years for significant pine regeneration in the absence of intervention. Dense brush is typical in young stands and an herbaceous layer may develop on some sites. On drier sites, there is less tendency for succession toward shade-adapted species. As young, dense stands age and attain a closed canopy, they exclude most undergrowth. When other adapted conifers occur in moist pine stands of medium to high site quality, they may form a significant understory in about 20 years in the absence of fire (Fitzhugh 1988).

\paragraph{Succession Transition} In the absence of disturbance, patches in this seral stage will begin transitioning to a mid development seral stage at 20 years. The rate of succession per time step is 0.7. The transition may be to either MDC or MDO. The secondary rate of succession to MDO is 0.4 and to MDC is 0.6. At 50 years, all patches will have succeeded to either MDC or MDO. On average, patches remain in ED for 27 years.
\begin{adjustwidth}{2cm}{}
\medskip
\textbf{Ultramafic Modifier} Succession may be substantially delayed. Thus, in the absence of disturbance, patches in this seral stage will begin transitioning to MDO at 50 years and may be delayed in the ED seral stage for as long as 100 years. A patch in this seral stage succeeds at a rate of 0.2 per time step. 

\end{adjustwidth}
\paragraph{Wildfire Transition}
High mortality wildfire (100\% of fires in this seral stage) recycles the patch through the Early Development seral stage, regardless of soil type. Low mortality wildfire is not modeled for this seral stage.


\noindent\hrulefill


\paragraph{Mid Development - Open Canopy Cover (MDO)}

\paragraph{Description} The mid-seral, open seral stage has hardwoods dominating the canopy and may have sporadic conifer presence at low coverage levels. Oaks are pole-sized to very large. Bunchgrasses and shade-intolerant shrubs, most notably, will be prominent on the majority of sites. This seral stage is distinguished from MDM and MDC primarily by its reduced conifer presence (LandFire 2007a).

\paragraph{Succession Transition} Patches in this seral stage will maintain under low mortality disturbance, but after 15 years without fire they begin transitioning to MDM at a rate of 0.7 per timestep. At 150 years since transitioning to a mid development seral stage, succession to LDO occurs at a rate of 0.3 per timestep. All remaining patches transition at 230 years. 
\begin{adjustwidth}{2cm}{}
\medskip

\textbf{Ultramafic Modifier}  In the absence of low mortality disturbance, patches will begin transitioning to MDC at 30 years at a rate of 0.1 per timestep. At 200 years in the mid development seral stage, succession to LD occurs at a rate of 0.3 per timestep. All remaining patches transition at 280 years.

\end{adjustwidth}
\paragraph{Wildfire Transition}
High mortality wildfire (5\% of fires in this seral stage) recycles the patch through the ED seral stage. Low mortality wildfire (95\%) maintains the patch in MDO.

\noindent\hrulefill

\paragraph{Mid Development - Moderate Canopy Cover (MDM)}

\paragraph{Description} The mid-seral, moderate canopy cover seral stage may represent a drier, hardwood dominated site that has gone without fire for an extended period, or a mesic site supporting both oak and yellow pine species that has been opened up by fire. \emph{P. menziesii} may occur. Oaks are pole to medium sized with moderate crown closure. Conifers are generally medium to large, depending on stand age. Overall canopy cover ranges from 40-70\%. Sod-forming grasses and shade-tolerant shrubs will be prominent on the majority of sites. Species from more arid sites may be remnants of earlier, more open post-fire communities (LandFire 2007a).

\paragraph{Succession Transition} Patches in this seral stage may maintain under low mortality disturbance, but after 15 years without fire they begin transitioning to MDC at a rate of 0.7 per timestep. At 110 years since transitioning to a mid development seral stage, succession to LDO occurs at a rate of 0.3 per timestep. All remaining patches transition at 180 years.
\begin{adjustwidth}{2cm}{}

\medskip
\textbf{Ultramafic Modifier}  In the absence of low mortality disturbance, patches will begin transitioning to MDC at 30 years at a rate of 0.1 per timestep. At 130 years in the mid development seral stage, succession to LDM occurs at a rate of 0.2 per timestep. All remaining patches transition at 250 years.

\end{adjustwidth}
\paragraph{Wildfire Transition}
High mortality wildfire (14\% of fires in this seral stage) recycles the patch through the ED seral stage. Low mortality wildfire (86\%) triggers a transition to MDO 32\% of the time; otherwise the patch remains in MDC.

\noindent\hrulefill

\paragraph{Mid Development - Closed Canopy Cover (MDC)}

\paragraph{Description} The mid-seral, closed seral stage is representative of the more mesic end of the environmental gradient and supports a dense canopy of oak and \emph{P. ponderosa} and/or \emph{P. jeffreyi}. Occasional \emph{P. menziesii} may occur. Oaks are pole to medium sized with crown closure approaching 70\%. Conifers are generally medium to large, depending on stand age. Overall canopy cover is at least 50\%. Sod-forming grasses and shade-tolerant shrubs will be prominent on the majority of sites. Species from more arid sites may be remnants of earlier, more open post-fire communities (LandFire 2007a).

\paragraph{Succession Transition} In the absence of stand-replacing disturbance, patches in this seral stage will begin transitioning to LDC at 80 years in an mid development seral stage at a rate of 0.3 per time step. At 150 years, all remaining patches succeed to LDC.
\begin{adjustwidth}{2cm}{}

\medskip
\textbf{Ultramafic Modifier}  Transition to late seral seral stages may be substatially delayed. Thus, in the absence of stand-replacing disturbance, patches in this seral stage will begin transitioning to LDC after 80 years at a rate of 0.2 per time step and may be delayed in a mid development seral stage for up to 300 years.

\end{adjustwidth}
\paragraph{Wildfire Transition} High mortality wildfire (15\% of fires in this seral stage) recycles the patch through the ED seral stage. Low mortality wildfire (85\%) triggers a transition to MDM 60\% of the time; otherwise the patch remains in MDC.


\noindent\hrulefill


\paragraph{Late Development - Open Canopy Cover (LDO)}

\paragraph{Description} The late-seral seral stage occurs when stand-replacing fire has been excluded from a patch for an extended period of time. Oaks are being overtopped by conifers. Thus, in this seral stage, oaks comprise a smaller proportion of the stand. Oaks and conifers are mature and large (LandFire 2007a). In general, sites are relatively open (Estes 2013).

\paragraph{Succession Transition} Patches in this seral stage will maintain under low mortality disturbance, but after 15 years without fire they begin transitioning to LDM at a rate of 0.7 per timestep. 
\begin{adjustwidth}{2cm}{}

\medskip
\textbf{Ultramafic Modifier}  In the absence of disturbance, patches in LDO will maintain.

\end{adjustwidth}
\paragraph{Wildfire Transition}
High mortality wildfire (1\% of fires in this seral stage) recycles the patch through the ED seral stage. Low mortality wildfire (99\%) maintains the patch in LDO.

\noindent\hrulefill

\paragraph{Late Development - Moderate Canopy Cover (LDM)}

\paragraph{Description} The late-seral seral stage occurs when stand-replacing fire has been excluded from a patch for an extended period of time. Oaks are being overtopped by conifers, including shade-tolerant conifers such as \emph{P. menziesii}. Thus, in this seral stage, oaks and even pines comprise a smaller proportion of the stand. Oaks and conifers are mature and large (LandFire 2007a). 

\paragraph{Succession Transition} Patches in this seral stage will maintain under low mortality disturbance, but after 15 years without fire they begin transitioning to LDC at a rate of 0.7 per timestep.
\begin{adjustwidth}{2cm}{}

\medskip
\textbf{Ultramafic Modifier}  In the absence of disturbance, patches in LDM will maintain.

\end{adjustwidth}
\paragraph{Wildfire Transition} High mortality wildfire (8\% of fires in this seral stage) recycles the patch through the ED seral stage. Low mortality wildfire (92\%) triggers a transition to LDO 18\% of the time; otherwise the patch remains in LDC.

\noindent\hrulefill

\paragraph{Late Development - Closed Canopy Cover (LDC)}

\paragraph{Description} The late-seral seral stage occurs when stand-replacing fire has been excluded from a patch for an extended period of time. Oaks are being overtopped by conifers, especially shade-tolerant conifers such as \emph{P. menziesii}. Thus, in this seral stage, oaks and even pines comprise a smaller proportion of the stand (LandFire 2007a). 

\paragraph{Succession Transition} In the absence of transition-causing disturbance, patches in this seral stage will maintain, regardless of soil characteristics.

\paragraph{Wildfire Transition} High mortality wildfire (20\% of fires in this seral stage) recycles the patch through the ED seral stage. Low mortality wildfire (80\%) triggers a transition to LDM 58\% of the time; otherwise the patch remains in LDC.

\noindent\hrulefill


\newpage
\subsection*{Seral Stage Classification}
\begin{table}[!htbp]
\footnotesize
\centering
\caption{Classification of cover seral stage for OCFW, for early and mid development stages. Diameter at Breast Height (DBH) and Cover From Above (CFA) values taken from EVeg polygons. DBH categories are: null, 0-0.9'', 1-4.9'', 5-9.9'', 10-19.9'', 20-29.9'', 30''+. CFA categories are null, 0-10\%, 10-20\%, \dots , 90-100\%. Each row in the table below should be read with a boolean AND across each column of a row.}
\label{ocfw_classification}
\begin{tabular}{@{}lrrrrr@{}}
\toprule
\textbf{\begin{tabular}[l]{@{}l@{}}Cover \\ Condition\end{tabular}} & \textbf{\begin{tabular}[r]{@{}r@{}}Overstory Tree \\ Diameter 1 \\ (DBH)\end{tabular}} & \textbf{\begin{tabular}[r]{@{}r@{}}Overstory Tree \\ Diameter 2 \\ (DBH)\end{tabular}} & \textbf{\begin{tabular}[r]{@{}r@{}}Total Tree\\ CFA (\%)\end{tabular}} & \textbf{\begin{tabular}[r]{@{}r@{}}Conifer \\ CFA (\%)\end{tabular}} & \textbf{\begin{tabular}[r]{@{}r@{}}Hardwood \\ CFA (\%)\end{tabular}} \\ \midrule
Early All        & null           & null    & any    & any    & any    \\
Early All        & 0-4.9''         & 0-4.9''  & any    & any    & any    \\
Early All        & 0-4.9''         & null    & any    & any    & any    \\
Mid Open         & 0-4.9''         & 5-29.9'' & 0-40   & any    & any    \\
Mid Open         & 5-29.9''        & null    & 0-40   & any    & any    \\
Mid Open         & 5-29.9''        & null    & null   & 0-40   & null   \\
Mid Open         & 5-29.9''        & null    & null   & null   & 0-40   \\
Mid Open         & 5-29.9''        & null    & null   & 0-40   & 0-40   \\
Mid Open         & 5-29.9''        & 0-29.9'' & 0-40   & any    & any    \\
Mid Open         & 5-29.9''        & 0-29.9'' & null   & 0-40   & 0-40   \\
Mid Moderate     & 0-4.9''         & 5-29.9'' & 40-70  & any    & any    \\
Mid Moderate     & 5-29.9''        & null    & 40-70  & any    & any    \\
Mid Moderate     & 5-29.9''        & null    & null   & 40-70  & null   \\
Mid Moderate     & 5-29.9''        & null    & null   & null   & 40-70  \\
Mid Moderate     & 5-29.9''        & null    & null   & 40-70  & 0-70   \\
Mid Moderate     & 5-29.9''        & null    & null   & 0-70   & 40-70  \\
Mid Moderate     & 5-29.9''        & 0-29.9'' & 40-70  & any    & any    \\
Mid Moderate     & 5-29.9''        & 0-29.9'' & null   & 40-70  & 0-70   \\
Mid Moderate     & 5-29.9''        & 0-29.9'' & null   & 0-70   & 40-70  \\
Mid Closed       & 0-4.9''         & 5-29.9'' & 70-100 & any    & any    \\
Mid Closed       & 5-29.9''        & null    & 70-100 & any    & any    \\
Mid Closed       & 5-29.9''        & null    & null   & 70-100 & any    \\
Mid Closed       & 5-29.9''        & null    & null   & any    & 70-100 \\
Mid Closed       & 5-29.9''        & 0-29.9'' & 70-100 & any    & any    \\
Mid Closed       & 5-29.9''        & 0-29.9'' & null   & 70-100 & any    \\
Mid Closed       & 5-29.9''        & 0-29.9'' & null   & any    & 70-100 \\ \bottomrule
\end{tabular}
\end{table}

\begin{table}[!htbp]
\footnotesize
\centering
\caption{Classification of cover seral stage for OCFW, for late development stages. Diameter at Breast Height (DBH) and Cover From Above (CFA) values taken from EVeg polygons. DBH categories are: null, 0-0.9'', 1-4.9'', 5-9.9'', 10-19.9'', 20-29.9'', 30''+. CFA categories are null, 0-10\%, 10-20\%, \dots , 90-100\%. Each row in the table below should be read with a boolean AND across each column of a row.}
\label{ocfw_classification2}
\begin{tabular}{@{}lrrrrr@{}}
\toprule
\textbf{\begin{tabular}[l]{@{}l@{}}Cover \\ Condition\end{tabular}} & \textbf{\begin{tabular}[r]{@{}r@{}}Overstory Tree \\ Diameter 1 \\ (DBH)\end{tabular}} & \textbf{\begin{tabular}[r]{@{}r@{}}Overstory Tree \\ Diameter 2 \\ (DBH)\end{tabular}} & \textbf{\begin{tabular}[r]{@{}r@{}}Total Tree\\ CFA (\%)\end{tabular}} & \textbf{\begin{tabular}[r]{@{}r@{}}Conifer \\ CFA (\%)\end{tabular}} & \textbf{\begin{tabular}[r]{@{}r@{}}Hardwood \\ CFA (\%)\end{tabular}} \\ \midrule
Late Open        & 30''+           & any     & 0-40   & any    & any    \\
Late Open        & 30''+           & any     & null   & 0-40   & null   \\
Late Open        & 30''+           & any     & null   & null   & 0-40   \\
Late Open        & 30''+           & any     & null   & 0-40   & 0-40   \\
Late Open        & any            & 30''+    & 0-40   & any    & any    \\
Late Open        & any            & 30''+    & null   & 0-40   & null   \\
Late Open        & any            & 30''+    & null   & null   & 0-40   \\
Late Open        & any            & 30''+    & null   & 0-40   & 0-40   \\
Late Moderate    & 30''+           & any     & 40-70  & any    & any    \\
Late Moderate    & 30''+           & any     & null   & 40-70  & null   \\
Late Moderate    & 30''+           & any     & null   & null   & 40-70  \\
Late Moderate    & 30''+           & any     & null   & 40-70  & 0-70   \\
Late Moderate    & 30''+           & any     & null   & 0-70   & 40-70  \\
Late Moderate    & any            & 30''+    & 40-70  & any    & any    \\
Late Moderate    & any            & 30''+    & null   & 40-70  & null   \\
Late Moderate    & any            & 30''+    & null   & null   & 40-70  \\
Late Moderate    & any            & 30''+    & null   & 40-70  & 0-70   \\
Late Moderate    & any            & 30''+    & null   & 0-70   & 40-70  \\
Late Closed      & 30''+           & any     & 70-100 & any    & any    \\
Late Closed      & 30''+           & any     & null   & 70-100 & any    \\
Late Closed      & 30''+           & any     & null   & any    & 70-100 \\
Late Closed      & any            & 30''+    & 70-100 & any    & any    \\
Late Closed      & any            & 30''+    & null   & 70-100 & any    \\
Late Closed      & any            & 30''+    & null   & any    & 70-100 \\ \bottomrule
\end{tabular}
\end{table}


\clearpage

\subsection*{References}
\begin{hangparas}{.25in}{1} 
\interlinepenalty=10000
Anderson, Richard. ``Montane Hardwood-Conifer (MHC).'' \emph{A Guide to Wildlife Habitats of California}, edited by Kenneth E. Mayer and William F. Laudenslayer. California Deparment of Fish and Game, 1988. \burl{http://www.dfg.ca.gov/biogeodata/cwhr/pdfs/MHC.pdf}. Accessed 4 December 2012.

``CalVeg Zone 1.'' Vegetation Descriptions. \emph{Vegetation Classification and Mapping}.  11 December 2008. U.S. Forest Service. \burl{http://www.fs.usda.gov/Internet/FSE_DOCUMENTS/fsbdev3_046448.pdf}. Accessed 2 April 2013.
Estes, Becky L. Personal communication, 21 June 2013.

Fitzhugh, E. Lee. ``Ponderosa Pine (PPN).'' \emph{A Guide to Wildlife Habitats of California}, edited by Kenneth E. Mayer and William F. Laudenslayer. California Deparment of Fish and Game, 1988. \burl{http://www.dfg.ca.gov/biogeodata/cwhr/pdfs/PPN.pdf}. Accessed 4 December 2012.

Fryer, Janet L. ``Quercus kelloggii.'' \emph{Fire Effects Information System}, U.S. Department of Agriculture, Forest Service,  Rocky Mountain Research Station, Fire Sciences Laboratory, 2007. \burl{http://www.fs.fed.us/database/feis/plants/tree/quekel/all.html}. Accessed 21 December 2012.

LandFire. ``Biophysical Setting Models.'' Biophysical Setting 0610300: Mediterranean California Lower Montane Black Oak-Conifer Forest and Woodland. 2007a. LANDFIRE Project, U.S. Department of Agriculture, Forest Service; U.S. Department of the Interior. \burl{http://www.landfire.gov/national_veg_models_op2.php}. Accessed 9 November 2012.

LandFire. ``Biophysical Setting Models.'' Biophysical Setting 0610210: Klamath-Siskiyou Lower Montane Serpentine Mixed Conifer Woodland. 2007b. LANDFIRE Project, U.S. Department of Agriculture, Forest Service; U.S. Department of the Interior. \burl{http://www.landfire.gov/national_veg_models_op2.php}. Accessed 9 November 2012.

LandFire. ``Biophysical Setting Models.'' Biophysical Setting 0711700: Klamath-Siskiyou Xeromorphic Serpentine Savanna and Chaparral. 2007c. LANDFIRE Project, U.S. Department of Agriculture, Forest Service; U.S. Department of the Interior. \burl{http://www.landfire.gov/national_veg_models_op2.php}. Accessed 30 November 2012.

O'Geen, Anthony T., Randy A. Dahlgren, and Daniel Sanchez-Mata. ``California Soils and Examples of Ultramafic Vegetation.'' In \emph{Terrestrial Vegetation of California, 3rd Edition}, edited by Michael Barbour, Todd Keeler-Wolf, and Allan A. Schoenherr, 71-106. Berkeley and Los Angeles: University of California Press, 2007. 

Skinner, Carl N. and Chi-Ru Chang. ``Fire Regimes, Past and Present.'' \emph{Sierra Nevada Ecosystem Project: Final report to Congress, vol. II, Assessments and scientific basis for management options}. Davis: University of California, Centers for Water and Wildland Resources, 1996.

Van de Water, Kip M. and Hugh D. Safford. ``A Summary of Fire Frequency Estimates for California Vegetation Before Euro-American Settlement.'' \emph{Fire Ecology} 7.3 (2011): 26-57. doi: 10.4996/fireecology.0703026.

\end{hangparas}


%% !TEX root = master.tex
\newpage
\section{Red Fir (RFR)}
\label{rfr-description}

\subsection*{General Information}

\subsubsection{Cover Type Overview}

\textbf{Red Fir (RFR)}
\newline
Crosswalks
\begin{itemize}
	\item EVeg: Regional Dominance Type 1
	\begin{itemize}
		\item Red Fir
	\end{itemize}

	\item Presettlement Fire Regime Type
	\begin{itemize}
		\item Red Fir
	\end{itemize}
\end{itemize}


\noindent Modifiers \\
\medskip
\noindent \textbf{Mesic Modifier } This type is created by intersecting a binary xeric/mesic layer with the existing vegetation layer. RFR cells that intersect with mesic cells are assigned to the mesic modifier.
\begin{itemize}
	\item LandFire BpS Model
	\begin{itemize}
		\item 0610321 Mediterranean California Red Fir Forest – Cascades
	\end{itemize}
\end{itemize}

\noindent \textbf{Xeric Modifier} This type is created by intersecting a binary xeric/mesic layer with the existing vegetation layer. RFR cells that intersect with xeric cells are assigned to the xeric modifier.
\begin{itemize}
	\item LandFire BpS Model
	\begin{itemize}
		\item 0610322 Mediterranean California Red Fir Forest – Southern Sierra
	\end{itemize}
\end{itemize}

\noindent \textbf{Ultramafic Modifier} This type is created by intersecting an ultramafic soils/geology layer with the existing vegetation layer. Where ultramafic cells intersect with RFR they are assigned to the ultramafic modifier.
\begin{itemize}
	\item LandFire BpS Model
	\begin{itemize}
		\item 0710220 Klamath-Siskiyou Upper Montane Serpentine Mixed Conifer Woodland
	\end{itemize}
\end{itemize}

\noindent \textbf{Red Fir with Aspen (RFR\_ASP)} This type is created by overlaying the NRIS TERRA Inventory of Aspen on top of the EVeg layer. Where it intersects with RFR it is assigned to RFR-ASP.

\noindent Reviewed by Marc Meyer, Southern Sierra Province Ecologist, USDA Forest Service

\subsubsection{Vegetation Description}
\textbf{Red Fir} The Red Fir landcover type is characterized by the presence of \emph{Abies magnifica}. Other conifer species such as \emph{Pinus monticola}, \emph{Pinus contorta} ssp. \emph{murrayana}, \emph{Tsuga mertensiana}, \emph{Abies concolor}, and \emph{Pinus jeffreyi} occur at varying densities (LandFire 2007a, LandFire 2007b). Mature \emph{A. magnifica} stands are frequently monotypic, with very few other plant species in any layer. Heavy shade and a thick layer of duff tends to inhibit understory vegetation, especially in dense stands (Barrett 1988). However, there are many open or patchy stands on less productive soils that are not monotypic, but rather codominant with other tree species. These sites may have substantial shrub cover (Meyer pers. comm.).

Stand-replacing disturbances such as lightning-caused fires, windthrows, insect outbreaks, and disease kill groups of trees (Barrett 1988). Stand structure is complex. Most current (fire-suppressed) \emph{A. magnifica} stands that were logged in the 19th century have an even-aged structure. In contrast, current unlogged and fire-suppressed stands have an uneven-aged or irregular age structure. Lastly, presettlement stands with an active fire regime had a relatively flat age-class structure that did not fit a classic even- or uneven-aged distribution (Meyer pers. comm. 2013). That is, frequent small-scale disturbance led to small patches of even-aged trees within the average ``stand,'' and most age classes in a given stand are represented by some of these small patches (Taylor and Halpern 1991). After fire, \emph{A. magnifica} seedlings may establish in canopy gaps, especially if they are small to moderate in size. \emph{P. contorta} ssp. \emph{murrayana}, as well as \emph{P. jeffreyi} and \emph{P. monticola}, may also function as post-fire pioneer species (Meyer pers. comm., Chappell and Agee 1996). On sites where these pioneering types occur under an \emph{A. magnifica} canopy, the \emph{A. magnifica} will dominate over the long-term (Cope 1993).

In openings resulting from tree mortality or logging, and under open stands on poor sites, many species may occur. Large shrubfields can dominate areas after severe fire, although conifers eventually will reclaim these sites. In some cases, particularly on xeric sites with significant shrub cover, reforestation can be effectively delayed for decades. \emph{Ribes}, \emph{Arctostaphylos}, and \emph{Ceanothus} are the most commonly found shrubs (Laacke 1990). Other associated shrubs include \emph{Symphoricarpos rotundifolius, Lonicera conjugialis}, and \emph{Quercus vaccinifolia} (Meyer pers. comm.). Associated herbaceous genera include \emph{Carex, Lupinus, Xerophyllum, Eucephalus, Pedicularis, Gayophytum, Pyrola} and \emph{Monardella} (Cope 1993).



\begin{adjustwidth}{2cm}{}
\noindent \textbf{Mesic Modifier } In addition to \emph{A. magnifica}, mesic regions within the RFR landcover type are associated with the presence of \emph{P. monticola} and \emph{P. contorta} ssp. \emph{murrayana}. \emph{T. mertensiana} may occur on northern aspects. \emph{A. concolor} is uncommon, except at lower elevations (LandFire 2007b).

\medskip
\noindent \textbf{Xeric Modifier}  These sites often include and are occasionally codominated by \emph{A. concolor}, \emph{P. jeffreyi}, and \emph{P. contorta} ssp. \emph{marayanna}, although other conifer species (e.g. \emph{P. lambertiana}) can also be present in lesser amounts at lower elevations. \emph{A. concolor} is more prevalent at lower elevations. \emph{P. jeffreyi} is more common on shallow soils or when disturbance is frequent. Shrubs and herbs generally contribute less than 30\% cover each. If shrub cover is higher, the shrubs are short or prostrate (LandFire 2007a).

\medskip
\noindent \textbf{Ultramafic Modifier} Ultramafic soils, support a number of endemic plant species. Slowly growing and often stunted \emph{P. contorta} ssp. \emph{murrayana} and \emph{P. jeffreyi} occur in combinations or in nearly pure open stands. \emph{A. magnifica} may be less dominant. Hardwoods are usually sparse, but shrubs such as \emph{Arctostaphylos}, \emph{Quercus}, \emph{Rhamnus, Lithocarpus, Rhododendron}, and \emph{Ceanothus} may occur on these sites. (``CalVeg Zone 1'' 2011)

\end{adjustwidth}

\noindent \textbf{Red Fir with Aspen} When \emph{Populus tremuloides} co-occurs with RFR on the west side of the Sierran crest, it is typically found in smaller patches, often less than 2 ha (5 acres) in size. This variant is not subject to the modifiers described above because it is only found on mesic sites with deeper soils. Mature stands in which \emph{P. tremuloides} are still dominant are usually relatively open. Average canopy closures range from 35-95\%. The open nature of the stands results in substantial light penetration to the ground (Meyer pers. comm., Verner 1988).



\subsubsection{Distribution}

\textbf{Red Fir} This cover type occupies the elevational band from about 1900 to 2750 m (6000 to 9000 ft). It is bounded and intergrades with Sierran Mixed Conifer at lower elevations. Geology is quite variable (Barrett 1988).

A xeric-mesic gradient was developed based on four variables: 1) aspect, 2) potential evapotranspiration, 3) topographic wetness index, and 4) soil water storage. The variables were standardized by z-score such that higher values correspond to more mesic environments. Thus, potential evapotranspiration was inverted to maintain this balance. The four variables were combined with equal weights. This final variables was split into xeric vs. mesic, with xeric occupying the negative end of the range up to $\frac{1}{4}$ standard deviation below the mean (zero) and mesic occupying the remaining portion of the spectrum.


\begin{adjustwidth}{2cm}{}
\medskip
\noindent \textbf{Mesic Modifier } These sites generally receive more moisture, either from precipitation, by virtue of being positioned on middle or lower slopes or drainage bottoms, or both. They may be adjacent to meadows or riparian areas. They are found at the highest elevations and north-facing aspects.

\medskip
\noindent \textbf{Xeric Modifier} These sites are typically drier and tend to occupy the lower portion of the RFR zone. They are also more likely to exist on south-facing aspects and steeper slopes.

\medskip
\noindent \textbf{Ultramafic Modifier} Ultramafic soils have been mapped at various spatial densities throughout the elevational range of the Red Fir landcover type. Low to moderate elevations in ultramafic and serpentinized areas often produce soils low in essential minerals such as calcium and magnesium or have excessive accumulations of heavy metals such as nickel and chromium. These sites vary widely in the degree of serpentization and effects on their overlying plant communities (``CalVeg Zone 1''). Note, the terms ``ultramafic rock'' and ``serpentine'' are broad terms used to describe a number of different but related rock types, including serpentinite, peridotite, dunite, pyroxenite, talc and soapstone, among others (O'Geen et al. 2007).

\end{adjustwidth}

\noindent \textbf{Red Fir with Aspen} Sites supporting \emph{P. tremuloides} are associated with added soil moisture, i.e., azonal wet sites. These sites are found throughout the RFR zone, often close to streams and lakes. Other sites include meadow edges, rock reservoirs, springs and seeps. Terrain can be simple to complex. At lower elevations, topographic conditions for this type tends toward positions resulting in relatively colder, wetter conditions within the prevailing climate, e.g., ravines, north slopes, wet depressions, etc. (LandFire 2007c). In general, these sites lie on lower slope positions, and are associated with slopes under 25\% (Potter 1998).

\subsection*{Disturbances}

\subsubsection{Wildfire}

\textbf{Red Fir} Fires in high-elevation \emph{A. magnifica} forests are generally not as intense as those in the Rocky Mountains and are typically less intense than those at lower elevations. Lesser annual fuel accumulation, less severe fire weather conditions, and compact and patchy fuels are all factors (Meyer pers. comm.). Still, fire has an important role in maintaining species diversity within these forests. Fire creates canopy openings by killing mature pioneer species such as \emph{P. contorta} ssp. \emph{murrayana} or \emph{P. jeffreyi} and some mature \emph{A. magnifica} (Cope 1993). 

Estimates of fire rotations for these variants are available from the LandFire project and a few review papers. The LandFire project’s published fire return intervals are based on a series of associated models created using the Vegetation Dynamics Development Tool (VDDT). In VDDT, fires are specified concurrently with the transition that follows them. For example, a replacement fire causes a transition to the early development stage. In the RMLands model, such fires are classified as high mortality. However, in VDDT mixed severity fires may cause a transition to early development, a transition to a more open seral stage, or no transition at all. In this case, we categorize the first example as a high mortality fire, and the second and third examples as a low mortality fire. Based on this approach, we calculated fire rotations and the probability of high mortality fire for each of the RFR seral stages across the three variants, as well as for the RFR\_ASP variant (Tables~\ref{tab:rfrmdesc_fire}--\ref{tab:rfr-aspdesc_fire}). We computed overall target fire rotations based on expert input from Safford and Estes, values from Mallek et al. (2013), and Van de Water and Safford (2011). 





\begin{adjustwidth}{2cm}{}
\medskip
\noindent \textbf{Mesic Modifier } Most fires occur during the late season during tree dormancy, fire complexity is moderate to high, and fire size averages 400 acres. It is very difficult to determine the replacement fire return interval. Replacement fire likely varies with slope position, and landscapes with greater topographic variation are likely to experience more stand replacement fires.

\medskip
\noindent \textbf{Xeric Modifier} Because of slow fuel accumulation rates, it is possible to have long gaps between surface fires in some seral stages. The discontinuous nature of the fuels limit extent of fires, and while fires may burn less often, they may burn at high severities. High intensity crown fires are uncommon.

\medskip
\noindent \textbf{Ultramafic Modifier} This type has a very limited distribution and consequently limited information for fire occurrence history. Low mortality fire is more common than high mortality fire. Most medium and high severity fire may actually occur on middle and upper slope positions.

\end{adjustwidth}

\noindent \textbf{Red Fir with Aspen} Sites supporting \emph{P. tremuloides} are maintained by stand-replacing disturbances that allow regeneration from below-ground suckers. Upland clones are impaired or suppressed by conifer ingrowth and overtopping and intensive grazing that inhibits growth. In a reference condition scenario, a few stands will advance toward conifer dominance. In the current landscape scenario, where fire has been reduced from reference conditions, there are many more conifer-dominated mixed aspen stands (LandFire 2007c, Verner 1988).


\begin{table}[]
\small
\centering
\caption{Fire rotation (years) and proportion of high (versus low) mortality fires for Red Fir – Mesic. Values were derived from VDDT model 0610322 (LandFire 2007b), Mallek et al. (2013), and Safford and Estes (personal communication).}
\label{tab:rfrmdesc_fire}
\begin{tabular}{@{}lcc@{}}
\toprule
\textbf{Condition}         & \multicolumn{1}{l}{\textbf{Fire Rotation}} & \multicolumn{1}{l}{\textbf{\begin{tabular}[c]{@{}l@{}}Proportion \\ High Mortality\end{tabular}}} \\ \midrule
Target                      & 60            & n/a                           \\
Early Development – All     & 58            & 1                             \\
Mid Development – Closed    & 55            & 0.35                          \\
Mid Development – Moderate  & 34            & 0.17                          \\
Mid Development – Open      & 25            & 0.09                          \\
Late Development – Closed   & 52            & 0.41                          \\
Late Development – Moderate & 32            & 0.16                          \\
Late Development – Open     & 23            & 0.05                  \\ \bottomrule
\end{tabular}
\end{table}

\begin{table}[]
\small
\centering
\caption{Fire rotation (years) and proportion of high (versus low) mortality fires for Red Fir – Xeric. Values were derived from VDDT model 0610321 (LandFire 2007a), and Safford and Estes (personal communication). }
\label{tab:rfrxdesc_fire}
\begin{tabular}{@{}lcc@{}}
\toprule
\textbf{Condition}         & \multicolumn{1}{l}{\textbf{Fire Rotation}} & \multicolumn{1}{l}{\textbf{\begin{tabular}[c]{@{}l@{}}Proportion \\ High Mortality\end{tabular}}} \\ \midrule
Target                      & 40            & n/a                           \\
Early Development – All     & 50            & 1                             \\
Mid Development – Closed    & 94            & 0.50                          \\
Mid Development – Moderate  & 65            & 0.25                          \\
Mid Development – Open      & 50            & 0.13                          \\
Late Development – Closed   & 74            & 0.38                          \\
Late Development – Moderate & 55            & 0.19                          \\
Late Development – Open     & 43            & 0.09                  \\ \bottomrule
\end{tabular}
\end{table}

\begin{table}[]
\small
\centering
\caption{Fire rotation (years) and proportion of high (versus low) mortality fires for Red Fir – Ultramafic. Values were derived from VDDT model 0610322 (LandFire 2007b), and Safford and Estes (personal communication). }
\label{tab:rfrudesc_fire}
\begin{tabular}{@{}lcc@{}}
\toprule
\textbf{Condition}         & \multicolumn{1}{l}{\textbf{Fire Rotation}} & \multicolumn{1}{l}{\textbf{\begin{tabular}[c]{@{}l@{}}Proportion \\ High Mortality\end{tabular}}} \\ \midrule
Target                      & 120           & n/a                           \\
Early Development – All     & 117           & 1                             \\
Mid Development – Closed    & 110           & 0.35                          \\
Mid Development – Moderate  & 69            & 0.17                          \\
Mid Development – Open      & 50            & 0.09                          \\
Late Development – Closed   & 104           & 0.41                          \\
Late Development – Moderate & 63            & 0.16                          \\
Late Development – Open     & 46            & 0.05                  \\ \bottomrule
\end{tabular}
\end{table}

\begin{table}[]
\small
\centering
\caption{Fire rotation (years) and proportion of high (versus low) mortality fires for Red Fir – Aspen type. Values were derived from VDDT model 0610610 (LandFire 2007) and Van de Water and Safford (pers. comm. 2013).}
\label{tab:rfr-aspdesc_fire}
\begin{tabular}{@{}lcc@{}}
\toprule
\textbf{Condition}         & \multicolumn{1}{l}{\textbf{Fire Rotation}} & \multicolumn{1}{l}{\textbf{\begin{tabular}[c]{@{}l@{}}Proportion \\ High Mortality\end{tabular}}} \\ \midrule
Target                           & 60            & n/a                           \\
Early Development – Aspen        & 58            & 0.03                          \\
Mid Development – Aspen          & 55            & 0.41                          \\
Mid Development – Aspen-Conifer  & 34            & 0.15                          \\
Late Development – Conifer-Aspen & 32            & 0.13                          \\
Late Development – Closed        & 52            & 0.26                  \\ \bottomrule
\end{tabular}
\end{table}

\subsubsection{Other Disturbance}
Other disturbances are not currently modeled, but may, depending on the seral stage affected and mortality levels, reset patches to early development, maintain existing seral stages, or shift/accelerate succession to a more open seral stage. 

\subsection*{Vegetation Seral Stages}
We recognize seven separate seral stages for RFR: Early Development (ED), Mid Development – Open Canopy Cover (MDO), Mid Development – Moderate Canopy Cover, Mid Development – Closed Canopy Cover (MDC), Late Development – Open Canopy Cover (LDO), Late Development – Moderate Canopy Cover (LDM), and Late Development – Closed Canopy Cover (LDC) (Figure~\ref{transmodel_rfr}). The RFR-ASP variant is also assigned to five seral stages: Early Development – Aspen (ED-A), Mid Development – Aspen (MD-A), Mid Development – Aspen with Conifer (MD-AC), Late Development Closed (LDC), and Late Development – Conifer with Aspen (LD-CA) (Figure~\ref{transmodel_rfr-asp}). 

Our seral stages are an alternative to ``successional'' classes that imply a linear progression of states and tend not to incorporate disturbance. The seral stages identified here are derived from a combination of successional processes and anthropogenic and natural disturbance, and are intended to represent a composition and structural condition that can be arrived at from multiple other conditions described for that landcover type. Thus our seral stages incorporate age, size, canopy cover, and vegetation composition. In general, the delineation of stages has originated from the LandFire biophysical setting model descriptive of a given landcover type; however, seral stages are not necessarily identical to the classes identified in those models.

\begin{figure}[htbp]
\centering
\includegraphics[width=0.8\textwidth]{/Users/mmallek/Documents/Thesis/statetransmodel/StateTransitionModel/7class.png}
\caption{State and Transition Model for Red Fir Forest (not inclusive of the aspen variant). Each dark grey box represents one of the seven seral stages for this landcover type. Each column of boxes represents a stage of development: early, middle, and late. Each row of boxes represents a different level of canopy cover: closed (70-100\%), moderate (40-70\%), and open (0-40\%). Transitions between states/seral stages may occur as a result of high mortality fire, low mortality fire, or succession. Specific pathways for each are denoted by the appropriate color line and arrow: red lines relate to high mortality fire, orange lines relate to low mortality fire, and green lines relate to natural succession.} 
\label{transmodel_rfr}
\end{figure}

\begin{figure}[htbp]
\centering
\includegraphics[width=0.8\textwidth]{/Users/mmallek/Documents/Thesis/statetransmodel/StateTransitionModel/5class-asp.png}
\caption{State and Transition Model for Red Fir Forest - Aspen variant. Each dark grey box represents one of the seven seral stages for this landcover type. Each column of boxes represents a stage of development: early, middle, and late. Transitions between states/seral stages may occur as a result of high mortality fire, low mortality fire, or succession. Specific pathways for each are denoted by the appropriate color line and arrow: red lines relate to high mortality fire, orange lines relate to low mortality fire, and green lines relate to natural succession.} 
\label{transmodel_rfr-asp}
\end{figure}

\subsection*{Red Fir}

\paragraph{Description}
\paragraph{Early Development (ED)} This seral stage is characterized by the recruitment of a new cohort of early successional, shade-intolerant tree species into an open area created by a stand-replacing disturbance. Conifer associates regenerate from seed. Occasionally, large brush fields may develop after hot wildfires and are dominated by \emph{Ceanothus, Arctostaphylos, Chrysolepsis}, or other shrub species for many years (Barrett 1988). On mesic sites, \emph{P. monticola} and \emph{P. contorta} ssp. \emph{murrayana} regenerate from seed. \emph{A. magnifica} comes in over time. Shrub cover is an important component; herb cover varies (LandFire 2007b). On xeric sites, there is regeneration of \emph{A. magnifica} and \emph{A. concolor}, perhaps \emph{P. jeffreyi} or \emph{P. lambertiana} from seed. Shrub and herb cover varies. (LandFire 2007a). Ultramafic sites will have similar species composition, especially at edges, but \emph{P. jeffreyi}, are relatively more common. Shrubs and herbs are sparse (O'Geen et al. 2007).

\paragraph{Succession Transition}

\begin{adjustwidth}{2cm}{}

\noindent \textbf{Mesic Modifier } In the absence of disturbance, patches in this seral stage will begin transitioning to MDC at age 30 at a rate of 0.6 per timestep. At 70 years, all stands will succeed to MDC. On average, patches remain in ED for 38 years.

\medskip
\noindent \textbf{Xeric Modifier}  Transition to mid development seral stages may be somewhat delayed. In the absence of disturbance, patches in this seral stage will begin transitioning to MDO at 50 years and may be delayed in the ED seral stage for as long as 150 years. A patch in this seral stage succeeds at a rate of 0.3 per timestep. On average, patches remain in ED for 67 years.

\medskip
\noindent \textbf{Ultramafic Modifier}  Transition to mid development seral stages may be substantially delayed. Thus, in the absence of disturbance, patches in this seral stage will begin transitioning to MDO after 80 years and may be delayed in the ED seral stage for as long as 150 years. A patch in this seral stage succeeds at a rate of 0.2 per timestep. On average, patches remain in ED for 105 years.

\end{adjustwidth}



\paragraph{Wildfire Transition} High mortality wildfire (100\% of fires in this seral stage) recycles the patch through the Early Development seral stage, regardless of soil type. Low mortality wildfire is not modeled for this seral stage.

\noindent\hrulefill


\paragraph{Mid Development – Open Canopy Cover (MDO)} 

\paragraph{Description} The pole/medium tree seral stage produces dense stands of young \emph{A. magnifica} that grow slowly with little mortality for many years (Barrett 1988). Cover of grasses, forms, and shrubs is on the decline as conifer canopy cover ranges from 10-40\%. \emph{A. magnifica} either is or is transitioning to become the dominant tree species. Canopy cover is less than 40\% (LandFire 2007a, LandFire 2007b).

On mesic sites, \emph{P. monticola} and \emph{P. contorta} ssp. \emph{murrayana} are present in varying amounts. Grasses, forbs, and shrubs are declining, although chaparral type shrubs, such as \emph{Arctostaphylos} or \emph{Chrysolepsis} can contribute to a dense understory. On xeric sites, \emph{A. concolor} and \emph{P. jeffreyi} are present in varying amounts, and shrub cover varies (LandFire 2007a, LandFire 2007b). Ultramafic sites will have similar species composition, especially at edges, but \emph{P. jeffreyi} is relatively more common (O'Geen et al. 2007).


\paragraph{Succession Transition}
\begin{adjustwidth}{2cm}{}

\noindent \textbf{Mesic Modifier } In the absence of low mortality disturbance, patches in the MDO seral stage will begin transitioning to MDM after 10 years at a rate of 0.22 per timestep. Succession to LDO takes place at 80 years since entering a middle development seral stage. 

\medskip
\noindent \textbf{Xeric Modifier} In the absence of low mortality disturbance, patches in the MDO seral stage will begin transitioning to MDM at 25 years at a rate of 0.2 per timestep. Succession to LDO takes place variably beginning at 80 years since transition to middle development at a rate of 0.6 per timestep. All patches succeed to a late seral stage by 100 years. On average, patches remain in MDM for 88 years.

\medskip
\noindent \textbf{Ultramafic Modifier} In the absence of low mortality disturbance, patches in the MDO seral stage will begin transitioning to MDM after 40 years at a rate of 0.1 per timestep. Succession to LDO takes place variably beginning at 120 years since transition to middle development at a rate of 0.3 per timestep, and all patches succeed by 180 years. On average, patches remain in ED for 117 years.

\end{adjustwidth}

\paragraph{Wildfire Transition}
\begin{adjustwidth}{2cm}{}
\noindent \textbf{Mesic Modifier } High mortality wildfire (9\% of fires in this seral stage) returns the patch to Early Development. Low mortality fire (91\%) maintains the MDO seral stage and allows for succession to LDO.

\medskip
\noindent \textbf{Xeric Modifier}  High mortality wildfire (13\% of fires in this seral stage) returns the patch to Early Development. Low mortality fire (87\%) maintains the MDO seral stage and allows for succession to LDO. 

\medskip
\noindent \textbf{Ultramafic Modifier}  High mortality wildfire (9\% of fires in this seral stage) returns the patch to Early Development. Low mortality fire (91\%) maintains the MDO seral stage and allows for succession to LDO.

\end{adjustwidth}

\noindent\hrulefill

\paragraph{Mid Development – Moderate Canopy Cover (MDM)}

\paragraph{Description} The pole/medium tree seral stage produces stands of young \emph{A. magnifica} with moderate canopy cover that grow slowly with little mortality for many years (Barrett 1988). Cover of grasses, forms, and shrubs is on the decline as conifer canopy cover exceeds 40\%. \emph{A. magnifica} either is or is transitioning to become the dominant tree species. On mesic sites, \emph{P. monticola} and \emph{P. contorta} ssp. \emph{murrayana} are present in varying amounts, while on xeric sites \emph{P. jeffreyi} and \emph{A. concolor} are associates (LandFire 2007a, LandFire 2007b). \emph{P. jeffreyi} is the most likely associate on ultramafic sites (O'Geen et al. 2007).

\paragraph{Succession Transition}
\begin{adjustwidth}{2cm}{}
\noindent \textbf{Mesic Modifier } In the absence of low mortality disturbance, patches in the MDM seral stage will begin transitioning to MDC after 10 years at a rate of 0.22 per timestep. Succession to LDM takes place at 80 years since entering a middle development seral stage. 

\medskip
\noindent \textbf{Xeric Modifier}  In the absence of low mortality disturbance, patches in the MDM seral stage will begin transitioning to MDC after 25 years at a rate of 0.2 per timestep. Succession to LDM begins at 80 years since entering a middle development stage at a rate of 0.65 per timestep. At 100 years after entering a middle development stage, all stands transition to LDM. 

\medskip
\noindent \textbf{Ultramafic Modifier} Transition to late seral seral stages may be substantially delayed. Thus, in the absence of low mortality disturbance, patches in the MDM seral stage will begin transitioning to MDC after 40 years at a rate of 0.1 per timestep. Succession to LDM begins at 100 years since entering a middle development stage at a rate of 0.3 per timestep. At 165 years after entering a middle development stage, all stands transition to LDM. 

\end{adjustwidth}

\paragraph{Wildfire Transition}
\begin{adjustwidth}{2cm}{}
\noindent \textbf{Mesic Modifier } High mortality wildfire (17\% of fires in this seral stage) returns the patch to ED. Low mortality wildfire (83\%) opens the stand up to MDO 13\% of the time; otherwise, the patch remains in MDC. 

\medskip
\noindent \textbf{Xeric Modifier}  High mortality wildfire (25\% of fires in this seral stage) returns the patch to ED. Low mortality wildfire (75\%) opens the stand up to MDO 13\% of the time; otherwise, the patch remains in MDC.

\medskip
\noindent \textbf{Ultramafic Modifier} High mortality wildfire (17\% of fires in this seral stage) returns the patch to ED. Low mortality wildfire (83\%) opens the stand up to MDO 13\% of the time; otherwise, the patch remains in MDC.

\end{adjustwidth}

\noindent\hrulefill

\paragraph{Mid Development – Closed Canopy Cover (MDC)}

\paragraph{Description} The pole/medium tree seral stage produces dense stands of young \emph{A. magnifica} that grow slowly with little mortality for many years (Barrett 1988). Cover of grasses, forms, and shrubs is on the decline as conifer canopy cover exceeds 40\%. \emph{A. magnifica} either is or is transitioning to become the dominant tree species. On mesic sites, \emph{P. monticola} and \emph{P. contorta} ssp. \emph{murrayana} are present in varying amounts, while on xeric sites \emph{P. jeffreyi} and \emph{A. concolor} are associates (LandFire 2007a, LandFire 2007b). \emph{P. jeffreyi} is the most likely associate on ultramafic sites (O'Geen et al. 2007).

\paragraph{Succession Transition}
\begin{adjustwidth}{2cm}{}
\noindent \textbf{Mesic Modifier } After 80 years in the mid development stage and in the absence of stand-replacing fire, all patches transition to LDC.

\medskip
\noindent \textbf{Xeric Modifier}  Transition to late seral seral stages may be delayed. Thus, in the absence of disturbance, patches in this seral stage will begin transitioning to LDC at 80 years in mid development at a rate of 0.7 per timestep and may be delayed in the MDC seral stage for up to 100 years.

\medskip
\noindent \textbf{Ultramafic Modifier} Transition to late seral seral stages may be substantially delayed. Thus, in the absence of disturbance, patches in this seral stage will begin transitioning to LDC at 80 years in the mid development stage at a rate of 0.3 per time step and may be delayed in the MDC seral stage for up to 150 years.

\end{adjustwidth}

\paragraph{Wildfire Transition}
\begin{adjustwidth}{2cm}{}
\noindent \textbf{Mesic Modifier } High mortality wildfire (35\% of fires in this seral stage) returns the patch to ED. Low mortality wildfire (65\%) opens the stand up to MDM 17\% of the time; otherwise, the patch remains in MDC. 

\medskip
\noindent \textbf{Xeric Modifier} High mortality wildfire (50\% of fires in this seral stage) returns the patch to ED. Low mortality wildfire (50\%) opens the stand up to MDM 17\% of the time; otherwise, the patch remains in MDC.

\medskip
\noindent \textbf{Ultramafic Modifier} High mortality wildfire (35\% of fires in this seral stage) returns the patch to ED. Low mortality wildfire (65\%) opens the stand up to MDM 17\% of the time; otherwise, the patch remains in MDC.

\end{adjustwidth}

\noindent\hrulefill


\paragraph{Late Development – Open Canopy Cover (LDO)}

\paragraph{Description} In the large tree seral stage, subdominant trees die and add to a growing layer of duff and downed woody material, and dominant trees continue to grow for several hundred years. \emph{A. magnifica} is the most common tree species. The understory of mature stands may be limited to less than 5\% cover (e.g. \emph{Chimaphila menziesii, Pyrola picta}). This seral stage develops when low mortality disturbance is fairly frequent; it persists as long as low mortality fires continue to occur periodically. \emph{Ceanothus} and \emph{Arctostaphylos} populate disturbance-generated gaps. Canopy cover is less than 40\% (LandFire 2007a, LandFire 2007b).

On mesic sites, \emph{P. monticola} and \emph{P. contorta} ssp. \emph{murrayana} may comprise up to 20\% of tree cover each. \emph{P. contorta} ssp. \emph{murrayana} acts as the pioneering conifer. On xeric sites, \emph{A. concolor} and \emph{P. jeffreyi} are the common associates and pioneer conifer species after disturbance (Barrett 1988, LandFire 2007a, LandFire 2007b). Ultramafic sites will have similar species composition, especially at edges, but \emph{P. jeffreyi} is relatively more common (O'Geen et al. 2007).


\paragraph{Succession Transition}
\begin{adjustwidth}{2cm}{}
\noindent \textbf{Mesic Modifier } In the presence of low mortality disturbance, patches in this seral stage can self-perpetuate, but after 10 years with no fire, patches in this seral stage will begin transitioning to LDM at a rate of 0.2 per timestep.

\medskip
\noindent \textbf{Xeric Modifier}  In the presence of low mortality disturbance, patches in this seral stage can self-perpetuate, but after 25 years with no fire, patches in this seral stage will begin transitioning to LDM at a rate of 0.2 per timestep.

\medskip
\noindent \textbf{Ultramafic Modifier} Patches occurring on ultramafic soils may succeed to LDM after 35 years with no fire, but the rate is just 0.2 per timestep.

\end{adjustwidth}

\paragraph{Wildfire Transition}
\begin{adjustwidth}{2cm}{}
\noindent \textbf{Mesic Modifier } High mortality wildfire (11\% of fires in this seral stage) returns the patch to early development. Low mortality wildfire (89\%) maintains LDO. 

\medskip
\noindent \textbf{Xeric Modifier} High mortality wildfire (3\% of fires in this seral stage) returns the patch to early development. Low mortality wildfire (97\%) maintains LDO. 

\medskip
\noindent \textbf{Ultramafic Modifier} High mortality wildfire (11\% of fires in this seral stage) returns the patch to early development. Low mortality wildfire (89\%) maintains LDO.

\end{adjustwidth}

\noindent\hrulefill

\paragraph{Late Development – Moderate Canopy Cover (LDM)}

\paragraph{Description} In the large tree seral stage, subdominant trees die and add to a growing layer of duff and downed woody material, and dominant trees continue to grow for several hundred years to heights of 40 m (130 ft). Overall conifer cover ranges from 40\% to 70\%. \emph{A. magnifica} is the most common tree species. The understory of mature stands is limited to less than 5 percent cover of shade tolerant forbs (e.g., \emph{Chimaphila menziesii, Pyrola picta}). 

On mesic sites, \emph{P. monticola} is the primary associate, with some \emph{P. contorta} ssp. \emph{murrayana} occurring in the understory. On xeric sites, \emph{A. magnifica} occurs in pure to mixed stands, and \emph{A. concolor} and \emph{P. jeffreyi} are the primary associates (Barrett 1988, LandFire 2007a, LandFire 2007b). Ultramafic sites will have similar species composition, especially at edges, but \emph{P. jeffreyi} is relatively more common. (O'Geen et al. 2007).


\paragraph{Succession Transition} In the absence of disturbance, patches in this seral stage will maintain, regardless of soil characteristics.

\paragraph{Wildfire Transition}
\begin{adjustwidth}{2cm}{}
\noindent \textbf{Mesic Modifier } High mortality wildfire (16\% of fires in this seral stage) will return the patch to Early Development. Low mortality wildfire (84\%) opens the stand up to LDO 15\% of the time; otherwise, the patch remains in LDM. 

\medskip
\noindent \textbf{Xeric Modifier} High mortality wildfire (19\% of fires in this seral stage) will return the patch to Early Development. Low mortality wildfire (81\%) opens the stand up to LDO 15\% of the time; otherwise, the patch remains in LDM. 

\medskip
\noindent \textbf{Ultramafic Modifier} High mortality wildfire (16\% of fires in this seral stage) will return the patch to Early Development. Low mortality wildfire (84\%) opens the stand up to LDO 15\% of the time; otherwise, the patch remains in LDM.

\end{adjustwidth}

\noindent\hrulefill

\paragraph{Late Development – Closed Canopy Cover (LDC)}

\paragraph{Description} In the large tree seral stage, subdominant trees die and add to a growing layer of duff and downed woody material, and dominant trees continue to grow for several hundred years to heights of 40 m (130 ft). Overall conifer cover exceeds 70\%. \emph{A. magnifica} is the most common tree species. The understory of mature stands is limited to less than 5 percent cover of shade tolerant forbs (e.g., \emph{Chimaphila menziesii, Pyrola picta}). 

On mesic sites, \emph{P. monticola} is the primary associate, with some \emph{P. contorta} ssp. \emph{murrayana} occuring in the understory. On xeric sites, \emph{A. magnifica} occurs in pure to mixed stands, and \emph{A. concolor} and \emph{P. jeffreyi} are the primary associates (Barrett 1988, LandFire 2007a, LandFire 2007b). Ultramafic sites will have similar species composition, especially at edges, but \emph{P. jeffreyi} is relatively more common (O'Geen et al. 2007).


\paragraph{Succession Transition} In the absence of disturbance, patches in this seral stage will maintain, regardless of soil characteristics.



\paragraph{Wildfire Transition}

\begin{adjustwidth}{2cm}{}
\noindent \textbf{Mesic Modifier } High mortality wildfire (41\% of fires in this seral stage) will return the patch to Early Development. Low mortality wildfire (59\%) opens the stand up to LDM 10\% of the time; otherwise, the patch remains in LDC. 

\medskip
\noindent \textbf{Xeric Modifier} High mortality wildfire (38\% of fires in this seral stage) will return the patch to Early Development. Low mortality wildfire (62\%) opens the stand up to LDM 10\% of the time; otherwise, the patch remains in LDC. 

\medskip
\noindent \textbf{Ultramafic Modifier} High mortality wildfire (41\% of fires in this seral stage) will return the patch to Early Development. Low mortality wildfire (59\%) opens the stand up to LDM 10\% of the time; otherwise, the patch remains in LDC.

\end{adjustwidth}

\noindent\hrulefill
\noindent\hrulefill

\subsubsection{Aspen Variant}

\paragraph{Early Development – Aspen (ED–A)}

\paragraph{Description} Grasses, forbs, low shrubs, and sparse to moderate cover of tree seedlings/saplings (primarily \emph{P. tremuloides}) with an open canopy. This seral stage is characterized by the recruitment of a new cohort of early successional, shade-intolerant tree species into an open area created by a stand-replacing disturbance. 

Following disturbance, succession proceeds rapidly from an herbaceous layer to shrubs and trees, which invade together (Barrett 1988). \emph{P. tremuloides} suckers over 6ft tall develop within about 10 years (LandFire 2007c). 



\paragraph{Succession Transition} Unless it burns, a patch in ED–A persists for 10 years, at which point it transitions to MD-A.

\paragraph{Wildfire Transition} High mortality wildfire (100\% of fires in this seral stage) recycles the patch through the ED–A seral stage. Low mortality wildfire is not modeled for this seral stage.

\noindent\hrulefill


\paragraph{Mid Development – Aspen (MD–A)}

\paragraph{Description} \emph{P. tremuloides} trees 5-16'' DBH. Canopy cover is highly variable, and can range from 40-100\%. These patches range in age from 10 to 110 years. Some understory conifers, including \emph{P. contorta} ssp. \emph{murrayana}, \emph{A. concolor}, and \emph{A. magnifica} are encroaching, but \emph{P. tremuloides} is still the dominant component of the stand (LandFire 2007c).

\paragraph{Succession Transition} MD-A persists for at least 50 years in the absence of fire, after which patches in this seral stage begin transitioning to MD-AC at a rate of 0.6 per timestep. At 100 years since entering MD-A, any remaining patches transition to MD-AC.

\paragraph{Wildfire Transition} High mortality wildfire (35\% of fires in this seral stage) recycles the patch through the ED-A seral stage. No transition occurs as a result of low mortality fire.

\noindent\hrulefill

\paragraph{Mid Development – Aspen with Conifer (MD–AC)}

\paragraph{Description} These stands have been protected from fire since the last stand-replacing disturbance. \emph{P. tremuloides} trees are predominantly 16'' DBH and greater. Conifers are present and overtopping the \emph{P. tremuloides}. \emph{A. concolor} is a typical conifer that is successional to \emph{P. tremuloides}, and is depicted here, but other conifers including \emph{P. ponderosa} and \emph{P. lambertiana} are also possible. Conifers are pole to medium-sized, and conifer cover is at least 40\% (LandFire 2007c).

\paragraph{Succession Transition} MD-AC persists for 100 years in the absence of fire, after which patches in this seral stage transition to LDC. 

\paragraph{Wildfire Transition} High mortality wildfire (17\% of fires in this seral stage) returns the patch to ED-A. Low mortality wildfire (83\%) maintains the patch in MD–AC.

\noindent\hrulefill

\paragraph{Late Development – Closed (LDC)}

\paragraph{Description} Some \emph{P. tremuloides} continue to be present in the understory, but large conifers are now the dominant tree species, having overtopped the \emph{P. tremuloides}. Smaller conifers are present in the midstory as well. Conifer species likely present include \emph{A. concolor, A. magnifica}, and \emph{P. contorta} ssp. \emph{murrayana}. (LandFire 2007a). This seral stage is analogous to the LDC seral stage for the RFR variant.

\paragraph{Succession Transition} In the absence of disturbance, patches in this seral stage will maintain, regardless of soil characteristics.

\paragraph{Wildfire Transition} High mortality wildfire (41\% of fires in this seral stage) will return the patch to ED–A. Low mortality wildfire (59\%) usually has little effect, although 10\% of the time it opens the stand up to LD-CA.

\noindent\hrulefill


\paragraph{Late Development – Conifer with Aspen (LD–CA)}

\paragraph{Description} If stands are sufficiently protected from fire such that conifer species overtop \emph{P. tremuloides} and become large, they may be able to withstand some fire that more sensitive \emph{P. tremuloides} cannot. When this occurs, it creates a patch characterized by late development conifers, such as \emph{A. concolor} or \emph{A. magnifica}, and early seral \emph{P. tremuloides}. 

\paragraph{Succession Transition} LD-CA persists for 70 years in the absence of any fire, after which patches transition to LDC. 

\paragraph{Wildfire Transition} High mortality wildfire (16\% of fires in this seral stage) returns the patch to ED-A. Low mortality wildfire (84\%) maintains the stand in LD-CA.

\noindent\hrulefill


\newpage

\subsection*{Seral Stage Classification}
\begin{table}[hbp]
\small
\centering
\caption{Classification of cover seral stage for RFR. Diameter at Breast Height (DBH) and Cover From Above (CFA) values taken from EVeg polygons. DBH categories are: null, 0-0.9'', 1-4.9'', 5-9.9'', 10-19.9'', 20-29.9'', 30''+. CFA categories are null, 0-10\%, 10-20\%, \dots , 90-100\%. Each row in the table below should be read with a boolean AND across each column of a row.}
\label{rfr_classification}
\begin{tabular}{@{}lrrrrr@{}}
\toprule
\textbf{\begin{tabular}[l]{@{}l@{}}Cover \\ Condition\end{tabular}} & \textbf{\begin{tabular}[r]{@{}r@{}}Overstory Tree \\ Diameter 1 \\ (DBH)\end{tabular}} & \textbf{\begin{tabular}[r]{@{}r@{}}Overstory Tree \\ Diameter 2 \\ (DBH)\end{tabular}} & \textbf{\begin{tabular}[r]{@{}r@{}}Total Tree\\ CFA (\%)\end{tabular}} & \textbf{\begin{tabular}[r]{@{}r@{}}Conifer \\ CFA (\%)\end{tabular}} & \textbf{\begin{tabular}[r]{@{}r@{}}Hardwood \\ CFA (\%)\end{tabular}} \\ \midrule
Early All        & null           & any & any    & any    & any  \\
Early All        & 0-4.9''         & any & any    & any    & any  \\
Mid Open         & 5-19.9''        & any & null   & null   & null \\
Mid Open         & 5-19.9''        & any & 0-40   & any    & any  \\
Mid Open         & 5-19.9''        & any & null   & 0-40   & null \\
Mid Moderate     & 5-19.9''        & any & 40-70  & any    & any  \\
Mid Moderate     & 5-19.9''        & any & null   & 40-70  & null \\
Mid Closed       & 5-19.9''        & any & 70-100 & any    & any  \\
Mid Closed       & 5-19.9''        & any & null   & 70-100 & any  \\
Late Open        & 20''+           & any & null   & null   & null \\
Late Open        & 20''+           & any & 0-40   & any    & any  \\
Late Open        & 20''+           & any & null   & 0-40   & null \\
Late Moderate    & 20''+           & any & 40-70  & any    & any  \\
Late Moderate    & 20''+           & any & null   & 40-70  & null \\
Late Closed      & 20''+           & any & 70-100 & any    & any  \\
Late Closed      & 20''+           & any & null   & 70-100 & any  \\ \bottomrule
\end{tabular}
\end{table}

RFR-ASP seral stages were assigned manually using NAIP 2010 Color IR imagery to assess seral stage.



\clearpage

\subsection*{References}

\begin{hangparas}{.25in}{1} 
\interlinepenalty=10000
Barrett, Reginald H. ``Red Fir (RFR).'' \emph{A Guide to Wildlife Habitats of California}, edited by Kenneth E. Mayer and William F. Laudenslayer. California Department of Fish and Game, 1988. \burl{http://www.dfg.ca.gov/biogeodata/cwhr/pdfs/RFR.pdf}. Accessed 4 December 2012.

``CalVeg Zone 1.'' Vegetation Descriptions. Vegetation Classification and Mapping.  11 December 2008. U.S. Forest Service. \burl{http://www.fs.usda.gov/Internet/FSE_DOCUMENTS/fsbdev3_046448.pdf}. Accessed 2 April 2013.

Chappell, Christopher B. and James K. Agee. ``Fire Severity and Tree Seedling Establishment in Abies Magnifica Forests, Southern Cascades, Oregon.'' \emph{Ecological Applications} 6.2 (1996): 628-640.

Cope, Amy B. ``Abies magnifica.'' \emph{Fire Effects Information System}, U.S. Department of Agriculture, Forest Service,  Rocky Mountain Research Station, Fire Sciences Laboratory, 1993. \burl{http://www.fs.fed.us/database/feis/} [Accessed 4 December 2012].

Estes, Becky. Central Sierra Province Ecologist, USDA Forest Service. 2013.

Laacke, Robert J. ``California Red Fir.'' Russell M. Burns and Barbara H. Honkala, tech. coords. \emph{Silvics of North America, vol 1. Conifers}; Glossary. Agriculture handbook no. 654. Washington, D.C.: U.S. Dept. of Agriculture, Forest Service, 1990. 

LandFire. ``Biophysical Setting Models.'' Biophysical Setting 0610321: Mediterranean California Red Fir Forest - Cascades. 2007a. LANDFIRE Project, U.S. Department of Agriculture, Forest Service; U.S. Department of the Interior. \burl{http://www.landfire.gov/national_veg_models_op2.php}. Accessed 9 November 2012.

LandFire. ``Biophysical Setting Models.'' Biophysical Setting 0610322: Mediterranean California Red Fir Forest – Southern Sierra. 2007b. LANDFIRE Project, U.S. Department of Agriculture, Forest Service; U.S. Department of the Interior. \burl{http://www.landfire.gov/national_veg_models_op2.php}. Accessed 9 November 2012.

LandFire. ``Biophysical Setting Models.'' Biophysical Setting 0610610: Inter-Mountain Basins Aspen-Mixed Conifer Forest and Woodland. 2007c. LANDFIRE Project, U.S. Department of Agriculture, Forest Service; U.S. Department of the Interior. \burl{http://www.landfire.gov/national_veg_models_op2.php}. Accessed 7 January 2013.

LandFire. ``Biophysical Setting Models.'' Biophysical Setting 0710320: Mediterranean California Red Fir Forest. 2007d. LANDFIRE Project, U.S. Department of Agriculture, Forest Service; U.S. Department of the Interior. \burl{http://www.landfire.gov/national_veg_models_op2.php}. Accessed 30 November 2012.

LandFire. ``Biophysical Setting Models.'' Biophysical Setting 0710220: Klamath-Siskiyou Upper Montane Serpentine Mixed Conifer Woodland. 2007e. LANDFIRE Project, U.S. Department of Agriculture, Forest Service; U.S. Department of the Interior. \burl{http://www.landfire.gov/national_veg_models_op2.php}. Accessed 30 November 2012.

Meyer, Marc D. Personal communication, 19 June 2013.

Meyer, Marc D. ``Natural Range of Variation of Red Fir Forests in the Bioregional Assessment Area'' (unpublished paper, Ecology Group, Pacific Southwest Research Station, 2013).

O'Geen, Anthony T., Randy A. Dahlgren, and Daniel Sanchez-Mata. ``California Soils and Examples of Ultramafic Vegetation'' In \emph{Terrestrial Vegetation of California, 3rd Edition}, edited by Michael Barbour, Todd Keeler-Wolf, and Allan A. Schoenherr, 71-106. Berkeley and Los Angeles: University of California Press, 2007. 

Potter, Donald A. ``Forested Communities of the Upper Montane in the Central and Southern Sierra Nevada.'' Gen. Tech. Rep. PSW-GTR-169. Albany, CA: Pacific Southwest Research Station, Forest Service, U.S. Department of Agriculture, 1998.

Safford, Hugh S. Personal communication, 5 May 2013.

Skinner, Carl N. and Chi-Ru Chang. ``Fire Regimes, Past and Present.'' \emph{Sierra Nevada Ecosystem Project: Final report to Congress, vol. II, Assessments and scientific basis for management options}. Davis: University of California, Centers for Water and Wildland Resources, 1996.

Taylor, Alan H. and Charles B. Halpern. ``The structure and dynamics of Abies magnifica forests in the southern Cascade Range, USA.'' \emph{Journal of Vegetation Science} 2 (1991): 189-200.

Van de Water, Kip M. and Hugh D. Safford. ``A Summary of Fire Frequency Estimates for California Vegetation Before Euro-American Settlement.'' \emph{Fire Ecology} 7.3 (2011): 26-57. doi: 10.4996/fireecology.0703026.

Verner, Jared. ``Aspen (ASP).'' \emph{A Guide to Wildlife Habitats of California}, edited by Kenneth E. Mayer and William F. Laudenslayer. California Deparment of Fish and Game, 1988. \burl{http://www.dfg.ca.gov/biogeodata/cwhr/pdfs/ASP.pdf}. Accessed 4 December 2012.

\end{hangparas}


%% !TEX root = master.tex
\newpage
\section{Sierran Mixed Conifer (SMC)}

\subsection*{General Information}

\subsubsection{Cover Type Overview}

\textbf{Sierran Mixed Conifer (SMC)}
\newline
Crosswalks
\begin{itemize}
	\item EVeg: Regional Dominance Type 1
	\begin{itemize}
		\item Mixed Conifer - Fir
		\item Mixed Conifer - Pine
	\end{itemize}

	\item Presettlement Fire Regime Type
	\begin{itemize}
		\item Red Fir
	\end{itemize}
\end{itemize}


\noindent Modifiers \\
\medskip
\noindent \textbf{Mesic Modifier } This type is created by intersecting a binary xeric/mesic layer with the existing vegetation layer. SMC cells that intersect with mesic cells are assigned to the mesic modifier.
\begin{itemize}
	\item LandFire BpS Model
	\begin{itemize}
		\item 0610280 Mediterranean California Mesic Mixed Conifer Forest and Woodland
	\end{itemize}
		\item Presettlement Fire Regime Type: 
	\begin{itemize}
		\item Moist Mixed Conifer
	\end{itemize}
\end{itemize}

\noindent \textbf{Xeric Modifier} This type is created by intersecting a binary xeric/mesic layer with the existing vegetation layer. SMC cells that intersect with xeric cells are assigned to the xeric modifier.
\begin{itemize}
	\item LandFire BpS Model
	\begin{itemize}
		\item 0610270 Mediterranean California Dry-Mesic Mixed Conifer Forest and Woodland
	\end{itemize}
		\item Presettlement Fire Regime Type: 
	\begin{itemize}
		\item Dry Mixed Conifer
	\end{itemize}
\end{itemize}

\noindent \textbf{Ultramafic Modifier} This type is created by intersecting an ultramafic soils/geology layer with the existing vegetation layer. Where ultramafic cells intersect with SMC they are assigned to the ultramafic modifier.
\begin{itemize}
	\item LandFire BpS Model
	\begin{itemize}
		\item 0710220 Klamath-Siskiyou Upper Montane Serpentine Mixed Conifer Woodland
	\end{itemize}
	\item Presettlement Fire Regime Type: 
	\begin{itemize}
		\item N/A
	\end{itemize}
\end{itemize}

\noindent \textbf{Sierran Mixed Conifer with Aspen (SMC\_ASP)} This type is created by overlaying the NRIS TERRA Inventory of Aspen on top of the EVeg layer. Where it intersects with SMC it is assigned to SMC-ASP. \\



\noindent Reviewed by Hugh Safford, Regional Ecologist, USDA Forest Service; Becky Estes, Central Sierra Province Ecologist, USDA Forest Service


\subsubsection{Vegetation Description}

\textbf{Sierran Mixed Conifer (SMC)} The Sierran Mixed Conifer landcover type is typically composed of three or more conifers, sometimes mixed with hardwoods. In forests experiencing the natural fire regime, stand and landscape structure are both highly heterogeneous, and age structure is usually uneven. Past management (e.g. logging and fire suppression) and its effects on forest succession have resulted in greater structural homogeneity and a dramatic increase in the presence of shade tolerant/fire intolerant tree species. Old-growth stands where fire has been excluded are often multi-storied, with the overstory comprised of various species (often dominated by pines) and the understory dominated by \emph{Abies concolor} and \emph{Calocedrus decurrens}. In the absence of fire, forested stands can form closed, multilayered canopies with over 100\% overlapping cover. Such dense stands were probably relatively uncommon before settlement, and found in moist microsites, on north slopes, and at higher elevations. When openings occur, shrubs are common in the understory. Before Euroamerican settlement, this landcover type was dominated by open stand conditions and old forest, but today closed canopy conditions dominated by middle aged trees are more common. Even aged stands are also widespread (Allen 2005). 

Five conifers and one hardwood typify this landcover type: \emph{A. concolor, Pseudotsuga menziesii, Pinus ponderosa, Pinus lambertiana, C. decurrens}, and \emph{Quercus kelloggii}. \emph{A. concolor} tends to be the most ubiquitous species because it is the competitive dominant in this landcover type. It tolerates shade, reproduces prolifically in the absence of fire, and has the ability to survive long periods of overtopping in brush fields. \emph{P. menziesii} replaces white fir as the competitive dominant at lower elevations. \emph{P. ponderosa}, which was historically the dominant species in SMC forest, still dominates at lower elevations and on south slopes. Like \emph{P. lambertiana}, its densities have been much reduced by logging. \emph{Pinus jeffreyi} commonly replaces \emph{P. ponderosa} at high elevations, on cold sites, or on ultramafic soils. \emph{Abies magnifica} is a minor associate at the highest elevations, as are \emph{Pinus monticola} and \emph{Pinus contorta} ssp. \emph{murrayana}. \emph{P. lambertiana} is found throughout the landcover type, but its densities have been much reduced by selective logging and white pine blister rust. \emph{Q. kelloggii} is a common component in stands on warm, dry sites. It sprouts prolifically after fire, and although it does best on open sites, it is maintained under adverse conditions such as overtopping by conifers and thin soils (Allen 2005). In some locations, \emph{Populus tremuloides} is also a component of the stand and, when present, typically dominates during the early seral stages following disturbance.

\emph{Ceanothus, Arctostaphylos, Chrysolepis, Prunus, Ribes, Rosa}, and \emph{Chamaebatia} are common shrub genera in the understory (Allen 2005). Grasses and forbs are diverse but rarely contribute much cover, except where stand structure is open. 


\begin{adjustwidth}{2cm}{}
\medskip
\noindent \textbf{Mesic Modifier } The primary species associated with mesic sites are \emph{A. concolor, P. menziesii, C. decurrens}, and \emph{P. lambertiana}. \emph{P. contorta} ssp. \emph{murrayana} may also be associated with mesic forests at higher elevations. As elevations begin to increase, \emph{A. magnifica} becomes more prominent. \emph{Lithocarpus densiflora} is an indicator of lower elevation sites with high water availability, either from meteoric or surface water. Understory diversity is often low in these sites, as high canopy cover and tree density reduce solar incidence at the soil surface. Very often the ground is covered in thick litter and duff. Some shade tolerant shrub and herb species occur.

\medskip
\noindent \textbf{Xeric Modifier}  Xeric sites are characterized by the presence of shade intolerant/fire tolerant conifer species such as \emph{P. ponderosa}, \emph{P. jeffreyi}, and \emph{P. lambertiana}, as well as the occurrence of varying amounts of more shade tolerant species like \emph{A. concolor} and \emph{C. decurrens}  \emph{Q. kelloggii} is locally common. The pines normally are prominent on south and west facing slopes, \emph{A. concolor} and sometimes \emph{P. menziesii}  on north and east slopes, and \emph{C. decurrens} as a secondary component on all slopes. At lower elevations, \emph{Pinus sabiniana}, and \emph{Quercus chrysolepis} may become common associates. Understory shrubs include \emph{Ceanothus, Arctostaphylos, Chamaebatia}, and \emph{Artemisia} and \emph{Purshia} in dry, eastern sites.

\medskip
\noindent \textbf{Ultramafic Modifier} Ultramafic soils support a number of endemic plant species. Slowly growing and often stunted \emph{P. contorta} ssp. \emph{murrayana} and \emph{P. jeffreyi} occur in combinations or in nearly pure open stands. Other tree associates on ultramafics include \emph{P. menziesii}, \emph{C. decurrens}, and \emph{Pinus attenuata}. Hardwoods are usually sparse, but shrubs such as \emph{Arctostaphylos, Quercus, Rhamnus, Lithocarpus, Rhododendron}, and \emph{Ceanothus} may occur on these sites. Often, a dramatic landscape shift occurs across abrupt discontinuities between ultramafics and other rock types. For example, regional stands of dense conifer forests are replaced by stunted and open stands of other conifers, by chaparral or even by barrens on which woody vegetation is absent (``CalVeg Zone 1'' 2011).

\end{adjustwidth}

\medskip
\noindent \textbf{Sierran Mixed Conifer  with Aspen} When \emph{P. tremuloides} co-occurs with SMC on the west side of the crest, it is typically found in smaller patches, often less than 2 hectares (5 acres) in size. This variant is not subject to the modifiers described above because it is only found on mesic sites. Mature stands in which \emph{P. tremuloides} are still dominant are usually relatively open. Average canopy closures of stands in eastern California range from 60\% to 100\% in young and intermediate-aged stands and from 25\% to 60\% in mature stands. The open nature of the stands results in substantial light penetration to the ground (Verner 1988).



\subsubsection{Distribution}

\textbf{Sierran Mixed Conifer } SMC generally forms a vegetation band ranging from 500 to 2000 m (1500 to 6500 ft). It dominates the western middle elevation slopes of the Sierra Nevada. Soils supporting SMC are varied in depth and composition, and are derived primarily from Mesozoic granitic, Paleozoic metamorphic rocks, and Cenozoic volcanic rocks (Allen 2005). 

A xeric-mesic gradient was developed based on four variables: 1) aspect, 2) potential evapotranspiration, 3) topographic wetness index, and 4) soil water storage. The variables were standardized by z-score such that higher values correspond to more mesic environments. Thus, potential evapotranspiration was inverted to maintain this balance. The four variables were combined with equal weights. This final variables was split into xeric vs. mesic, with xeric occupying the negative end of the range up to $\frac{1}{4}$ standard deviation below the mean (zero) and mesic occupying the remaining portion of the spectrum.


\begin{adjustwidth}{2cm}{}
\textbf{Mesic Modifier } Generally found on favorable slopes, primarily north and east aspects throughout the geographic range, as well as along streams in drier areas. It is more common at higher elevations as compared to the xeric type (``CalVeg Zone 1'' 2011).

\medskip
\noindent \textbf{Xeric Modifier} Occurs on south and west-facing aspects (LandFire 2007b). At lower elevations patches may be found on north slopes. At higher elevations this landcover type most typically occurs on south, east and west aspects. 

\medskip
\noindent \textbf{Ultramafic Modifier} Ultramafics have been mapped at various spatial densities throughout the elevational range of the SMC landcover type. Low to moderate elevations in ultramafic and serpentinized areas often produce soils low in essential minerals like calcium potassium, and nitrogen, and have excessive accumulations of heavy metals such as nickel and chromium. These sites vary widely in the degree of serpentinization and effects on their overlying plant communities (``CalVeg Zone 1'' 2011). Note, the terms ``ultramafic rock'' and ``serpentine'' are broad terms used to describe a number of different but related rock types, including serpentinite, peridotite, dunite, pyroxenite, talc and soapstone, among others (O'Geen et al. 2007). 

\end{adjustwidth}

\medskip
\noindent \textbf{Sierran Mixed Conifer with Aspen} Sites supporting \emph{P. tremuloides} are usually associated with added soil moisture, i.e., azonal wet sites. These sites are found throughout the SMC zone, often close to streams and lakes. Other sites include meadow edges, rock reservoirs, springs and seeps. Terrain can be simple to complex. At lower elevations, topographic conditions for this type tends toward positions resulting in relatively colder, wetter conditions within the prevailing climate, e.g., ravines, north slopes, wet depressions, etc. (LandFire 2007c).

\subsection*{Disturbances}

\subsubsection{Wildfire}

\textbf{Sierran Mixed Conifer } Wildfires are common and frequent; mortality depends on vegetation vulnerability and wildfire intensity. Low mortality fires kill small trees and may consume above-ground portions of small oaks, shrubs and herbs, but do not kill large trees or below-ground organs of most oaks, shrubs and herbs which promptly resprout. High mortality fires kill trees of all sizes and may kill many of the shrubs and herbs as well. However, high mortality fires typically kill only the above ground portions of the oaks, shrubs and herbs; consequently, most oaks, shrubs and herbs promptly resprout from surviving below ground organs.

Data on fire return intervals (FRIs) are available from a few review papers. Mallek et al. (2013) calculated presettlement fire rotation for 7 major forest types in the Sierra Nevada. Skinner and Chang (1996) aggregated FRIs from the Sierra Nevada and separated pre-1850 data from overall data. Van de Water and Safford’s 2011 review paper aggregates hundreds of articles, conference proceedings, and LandFire data on fire return intervals, with an emphasis on Californian sources. We also include here data from the pertinent individual LandFire BpS models (2007a, 2007b, 2007c, 2007d).

Estimates of fire rotations for these variants are available from the LandFire project and a few review papers. The LandFire project’s published fire return intervals are based on a series of associated models created using the Vegetation Dynamics Development Tool (VDDT). In VDDT, fires are specified concurrently with the transition that follows them. For example, a replacement fire causes a transition to the early development stage. In the RMLands model, such fires are classified as high mortality. However, in VDDT mixed severity fires may cause a transition to early development, a transition to a more open seral stage, or no transition at all. In this case, we categorize the first example as a high mortality fire, and the second and third examples as a low mortality fire. Based on this approach, we calculated fire rotations and the probability of high mortality fire for each of the SMC seral stages across the three variants, as well as for the SMC\_ASP variant (Tables~\ref{tab:smcmdesc_fire}--\ref{tab:smc-aspdesc_fire}). We computed overall target fire rotations based on expert input from Safford and Estes, values from Mallek et al. (2013), and Van de Water and Safford (2011). 


\begin{adjustwidth}{2cm}{}
\textbf{Mesic Modifier } Low mortality fire is fairly frequent. Fire severity is typically positively correlated with slope position. 

\medskip
\noindent \textbf{Xeric Modifier} Fire of all severity levels is fairly common. This landcover type has one of the shortest fire rotations. 

\medskip
\noindent \textbf{Ultramafic Modifier} This type has a very limited distribution and consequently limited information for fire occurrence history. Low mortality fire is more common than high mortality fire. Most medium and high severity fire occurs on middle and upper slope positions.

\end{adjustwidth}

\medskip
\noindent \textbf{Sierran Mixed Conifer with Aspen} Sites supporting \emph{P. tremuloides} are maintained by stand-replacing disturbances that allow regeneration from below-ground suckers. Upland clones are impaired or suppressed by conifer ingrowth and overtopping and intensive grazing that inhibits growth. In a reference condition scenario, a few stands will advance toward conifer dominance, but in the current landscape scenario where fire has been reduced from reference conditions there are many more conifer-dominated mixed aspen stands (LandFire 2007c, Verner 1988).


\begin{table}[]
\small
\centering
\caption{Fire rotation (years) and proportion of high (versus low) mortality fires for Sierran Mixed Conifer - Mesic. Values were derived from Mallek et al. (2013) and VDDT model 0610280 (LandFire 2007a). }
\label{tab:smcmdesc_fire}
\begin{tabular}{@{}lcc@{}}
\toprule
\textbf{Condition}         & \multicolumn{1}{l}{\textbf{Fire Rotation}} & \multicolumn{1}{l}{\textbf{\begin{tabular}[c]{@{}l@{}}Proportion \\ High Mortality\end{tabular}}} \\ \midrule
Target                      & 29            & n/a                           \\
Early Development - All     & 44            & 1                             \\
Mid Development - Closed    & 19            & 0.23                          \\
Mid Development - Moderate  & 13            & 0.17                          \\
Mid Development - Open      & 10            & 0.14                          \\
Late Development - Closed   & 34            & 0.37                          \\
Late Development - Moderate & 13            & 0.14                          \\
Late Development - Open     & 8             & 0.08                     \\ \bottomrule
\end{tabular}
\end{table}

\begin{table}[]
\small
\centering
\caption{Fire rotation (years) and proportion of high (versus low) mortality fires for Sierran Mixed Conifer - Xeric. Values were derived from Mallek et al. (2013) and VDDT model 0610280 (LandFire 2007b).}
\label{tab:smcxdesc_fire}
\begin{tabular}{@{}lcc@{}}
\toprule
\textbf{Condition}         & \multicolumn{1}{l}{\textbf{Fire Rotation}} & \multicolumn{1}{l}{\textbf{\begin{tabular}[c]{@{}l@{}}Proportion \\ High Mortality\end{tabular}}} \\ \midrule
Target                      & 22            & n/a                           \\
Early Development - All     & 32            & 1                             \\
Mid Development - Closed    & 11            & 0.48                          \\
Mid Development - Moderate  & 10            & 0.26                          \\
Mid Development - Open      & 9             & 0.09                          \\
Late Development - Closed   & 16            & 0.25                          \\
Late Development - Moderate & 10            & 0.11                          \\
Late Development - Open     & 8             & 0.05                  \\ \bottomrule
\end{tabular}
\end{table}

\begin{table}[]
\small
\centering
\caption{Fire rotation (years) and proportion of high (versus low) mortality fires for Sierran Mixed Conifer - Ultramafic. Values were derived from Van de Water and Safford (2011), and Mallek et al. (2013) and VDDT model 071220 (LandFire 2007d). }
\label{tab:smcudesc_fire}
\begin{tabular}{@{}lcc@{}}
\toprule
\textbf{Condition}         & \multicolumn{1}{l}{\textbf{Fire Rotation}} & \multicolumn{1}{l}{\textbf{\begin{tabular}[c]{@{}l@{}}Proportion \\ High Mortality\end{tabular}}} \\ \midrule
Target                      & 60            & n/a                           \\
Early Development - All     & 89            & 1                             \\
Mid Development - Closed    & 39            & 0.23                          \\
Mid Development - Moderate  & 27            & 0.17                          \\
Mid Development - Open      & 21            & 0.14                          \\
Late Development - Closed   & 69            & 0.37                          \\
Late Development - Moderate & 27            & 0.14                          \\
Late Development - Open     & 16            & 0.08                  \\ \bottomrule
\end{tabular}
\end{table}

\begin{table}[]
\small
\centering
\caption{Fire rotation (years) and proportion of high (versus low) mortality fires for Sierran Mixed Conifer - Aspen type. Values were derived from VDDT models 0610280 and 0610610 (LandFire 2007a, LandFire 2007c) and Van de Water and Safford (2011). }
\label{tab:smc-aspdesc_fire}
\begin{tabular}{@{}lcc@{}}
\toprule
\textbf{Condition}         & \multicolumn{1}{l}{\textbf{Fire Rotation}} & \multicolumn{1}{l}{\textbf{\begin{tabular}[c]{@{}l@{}}Proportion \\ High Mortality\end{tabular}}} \\ \midrule
Target                           & 29            & n/a                           \\
Early Development - Aspen        & 44            & 1                             \\
Mid Development - Aspen          & 19            & 0.26                          \\
Mid Development - Aspen-Conifer  & 13            & 0.18                          \\
Late Development - Conifer-Aspen & 13            & 0.14                          \\
Late Development - Closed        & 34            & 0.37                  \\ \bottomrule
\end{tabular}
\end{table}

\subsubsection{Other Disturbance}
Other disturbances are not currently modeled, but may, depending on the seral stage affected and mortality levels, reset patches to early development, maintain existing seral stages, or shift/accelerate succession to a more open seral stage. All of the tree species associated with this vegetation type are susceptible to a wide variety of pathogens and insects. 

\subsection*{Vegetation Seral Stages}
We recognize seven separate seral stages for SMC: Early Development (ED), Mid Development - Open Canopy Cover (MDO), Mid Development - Moderate Canopy Cover, Mid Development - Closed Canopy Cover (MDC), Late Development - Open Canopy Cover (LDO), Late Development - Moderate Canopy Cover (LDM), and Late Development - Closed Canopy Cover (LDC) (Figure~\ref{transmodel_smc}). The SMC-ASP variant is also assigned to five seral stages: Early Development - Aspen (ED-A), Mid Development - Aspen (MD-A), Mid Development - Aspen with Conifer (MD-AC), Late Development Closed (LDC), and Late Development - Conifer with Aspen (LD-CA) (Figure~\ref{transmodel_smc-asp}).

Our seral stages are an alternative to ``successional'' classes that imply a linear progression of states and tend not to incorporate disturbance. The seral stages identified here are derived from a combination of successional processes and anthropogenic and natural disturbance, and are intended to represent a composition and structural condition that can be arrived at from multiple other conditions described for that landcover type. Thus our seral stages incorporate age, size, canopy cover, and vegetation composition. In general, the delineation of stages has originated from the LandFire biophysical setting model descriptive of a given landcover type; however, seral stages are not necessarily identical to the classes identified in those models.


\begin{figure}[hbp]
\centering
\includegraphics[width=0.8\textwidth]{/Users/mmallek/Documents/Thesis/statetransmodel/StateTransitionModel/7class.png}
\caption{State and Transition Model for Sierran Mixed Conifer Forest (not inclusive of the aspen variant). Each dark grey box represents one of the seven seral stages for this landcover type. Each column of boxes represents a stage of development: early, middle, and late. Each row of boxes represents a different level of canopy cover: closed (70-100\%), moderate (40-70\%), and open (0-40\%). Transitions between states/seral stages may occur as a result of high mortality fire, low mortality fire, or succession. Specific pathways for each are denoted by the appropriate color line and arrow: red lines relate to high mortality fire, orange lines relate to low mortality fire, and green lines relate to natural succession.} 
\label{transmodel_smc}
\end{figure}

\begin{figure}[htbp]
\centering
\includegraphics[width=0.8\textwidth]{/Users/mmallek/Documents/Thesis/statetransmodel/StateTransitionModel/5class-asp.png}
\caption{State and Transition Model for Sierran Mixed Conifer Forest - Aspen variant. Each dark grey box represents one of the seven seral stages for this landcover type. Each column of boxes represents a stage of development: early, middle, and late. Transitions between states/seral stages may occur as a result of high mortality fire, low mortality fire, or succession. Specific pathways for each are denoted by the appropriate color line and arrow: red lines relate to high mortality fire, orange lines relate to low mortality fire, and green lines relate to natural succession.} 
\label{transmodel_smc-asp}
\end{figure}

\subsection*{Sierran Mixed Conifer }

\paragraph{Description}
\paragraph{Early Development (ED)} This seral stage is characterized by the recruitment of a new cohort of early successional tree species into an open area created by a stand-replacing disturbance. After disturbance, succession proceeds from an ephemeral herb to perennial grass-herb community. This seral stage generally lasts only a few years before shifting to a shrub-seedling-sapling seral stage dominated by any of the following genera: \emph{Arctostaphylos, Ceanothus, Prunus, Ribes, and Chamaebatia}, as well as \emph{Q. vaccinifolia}. Tree seedlings/saplings typical of the cover type can be either high or low density depending on local environmental conditions and climate conditions following the disturbance. In some cases (e.g., favorable climate conditions develop following the stand-replacing disturbance and a good seed source), tree seedlings may develop a nearly continuous canopy and succeed relatively quickly to mid-development seral stages. In other cases, and more commonly on xeric or ultramafic sites, chaparral conditions may dominate and persist for long periods of time (LandFire 2007a, LandFire 2007b).

\paragraph{Succession Transition}

\begin{adjustwidth}{2cm}{}
\textbf{Mesic Modifier } In the absence of disturbance, patches in this seral stage will begin transitioning to MDC or MDO after 20 years at a rate of 0.8 per timestep. The transition to MDC is twice as likely as transition to MDO.  At 40 years, all remaining patches will succeed to either MDC or MDO. On average, patches remain in ED for 26 years.

\medskip
\noindent \textbf{Xeric Modifier}  Transition to the MD seral stages may be substantially delayed. Thus, in the absence of disturbance, patches in this seral stage will begin transitioning to MDO after 40 years and may be delayed in ED for as long as 80 years. During this period, succession occurs at a rate of 0.4 per timestep. On average, patches remain in ED for 53 years.

\medskip
\noindent \textbf{Ultramafic Modifier}  Transition to the MD seral stage may be substantially delayed. Thus, in the absence of disturbance, patches in this seral stage will begin transitioning to MDO after 80 years and may be delayed in ED for as long as 150 years. During this period, succession occurs at a rate of 0.2 per timestep. On average, patches remain in ED for 105 years.

\end{adjustwidth}



\paragraph{Wildfire Transition} High mortality wildfire (100\% of fires in this seral stage) recycles the patch through the Early Development seral stage, regardless of soil type. Low mortality wildfire is not modeled for this seral stage. 

\noindent\hrulefill


\paragraph{Mid Development - Open Canopy Cover (MDO)} 

\paragraph{Description} Heterogeneous ground cover of grasses, forbs, and shrubs. Trees present are pole to medium sized conifers with canopy cover less than 40\% (LandFire 2007a). Conifer species likely present include \emph{A. concolor, C. decurrens P. ponderosa, P. menziesii}, and \emph{P. lambertiana}. Pines predominate on xeric sites while firs predominate on mesic sites. \emph{Q. kelloggi} may occur as well, mostly on warmer slopes and where soils are less productive (LandFire 2007a). Ultramafic sites will have similar species composition, especially at edges, but \emph{P. jeffreyi} and \emph{C. decurrens} are relatively more common (O'Geen et al. 2007).

\paragraph{Succession Transition}
\begin{adjustwidth}{2cm}{}
\textbf{Mesic Modifier } In the absence of low mortality disturbance, patches in the MDO seral stage will begin transitioning to MDM after 15 years at a rate of 0.9 per timestep. Succession to LDO takes place variably after 100 years since entering a middle development seral stage, at a rate of 0.4 per timestep. All patches succeed by 150 years in MD.  On average (across all canopy cover seral stages), patches remain in mid development for 113 years. 

\medskip
\noindent \textbf{Xeric Modifier} In the absence of low mortality disturbance, patches in the MDO seral stage will begin transitioning to MDC after 84 years at a rate of 0.3 per timestep. Succession to LDO takes place variably beginning at 160 years since transition to middle development, at a rate of 0.4 per timestep. All patches succeed by 200 years. On average (across all canopy cover seral stages), patches remain in mid development for 173 years.

\medskip
\noindent \textbf{Ultramafic Modifier} In the absence of low mortality disturbance, patches in the MDO seral stage will begin transitioning to MDC after 40 years in MDO at a rate of 0.1 per timestep. Succession to LDO takes place variably beginning at 200 years since transition to middle development at a rate of 0.4 per timestep. All patches succeed by 260 years. On average (across all canopy cover seral stages), patches remain in mid development for 213 years.

\end{adjustwidth}

\paragraph{Wildfire Transition}
\begin{adjustwidth}{2cm}{}
\textbf{Mesic Modifier } High mortality wildfire (14\% of fires in this seral stage) returns the patch to Early Development. Low mortality fire (86\%) maintains the MDO seral stage and allows for succession to LDO. 

\medskip
\noindent \textbf{Xeric Modifier}  High mortality wildfire (9\% of fires in this seral stage) returns the patch to Early Development. Low mortality fire (91\%) maintains the MDO seral stage and allows for succession to LDO. 

\medskip
\noindent \textbf{Ultramafic Modifier}  High mortality wildfire (14\% of fires in this seral stage) returns the patch to Early Development. Low mortality fire (86\%) maintains the MDO seral stage and allows for succession to LDO.

\end{adjustwidth}

\noindent\hrulefill

\paragraph{Mid Development - Moderate Canopy Cover (MDM)}

\paragraph{Description} Sparse ground cover of grasses, forbs, and shrubs; moderate to dense cover of trees. Conifers are pole to medium-sized, with canopy cover from 40-70\%. Conifer species likely present include \emph{A. concolor, C. decurrens, P. ponderosa, P. menziesii}, and \emph{P. lambertiana}. \emph{Q. kelloggi} may occur as well, mostly on warmer slopes and where soils are less productive (LandFire 2007a, LandFire 2007b). Ultramafic sites will have similar species composition, especially at edges, but \emph{P. jeffreyi} and \emph{C. decurrens} are relatively more common (O'Geen et al. 2007).

\paragraph{Succession Transition}
\begin{adjustwidth}{2cm}{}
\textbf{Mesic Modifier } In the absence of low mortality disturbance, patches in the MDM seral stage will begin transitioning to MDC after 15 years at a rate of 0.9 per timestep. Patches in the MDM seral stage begin transitioning to LDM once the time since transition to a mid development seral stage is at least 100 years at a rate of 0.6 per timestep. All patches succeed by 150 years in mid development. On average (across all canopy cover seral stages), patches remain in mid development for 113 years.

\medskip
\noindent \textbf{Xeric Modifier}  Transition to late seral seral stages may be delayed. In the absence of low mortality disturbance, patches in the MDM seral stage will begin transitioning to MDC after 40 years at a rate of 0.3 per timestep. Patches in this seral stage will begin transitioning to LDC after 160 years in an MD seral stage at a rate of 0.4 per time step and may be delayed in the MDC seral stage for up to 200 years. On average (across all canopy cover seral stages), patches remain in mid development for 173 years. 

\medskip
\noindent \textbf{Ultramafic Modifier} Transition to late seral seral stages may be substantially delayed. Thus, in the absence of disturbance, patches in this seral stage will begin transitioning to MDC after 40 years at a rate of 0.1 per timestep. Patches in the MDM seral stage begin transitioning to LDM once the time since transition to a mid development seral stage is at least 200 years at a rate of 0.4 per timestep. All patches succeed by 260 years in mid development. On average (across all canopy cover seral stages), patches remain in mid development for 213 years.

\end{adjustwidth}

\paragraph{Wildfire Transition}
\begin{adjustwidth}{2cm}{}
\textbf{Mesic Modifier } High mortality wildfire (17\% of fires in this seral stage) returns the patch to ED. Low mortality wildfire (83\%) opens the stand up to MDO 36\% of the time; otherwise, the patch remains in MDM. 

\medskip
\noindent \textbf{Xeric Modifier}  High mortality wildfire (26\% of fires in this seral stage) returns the patch to ED. Low mortality wildfire (74\%) opens the stand up to MDO 32\% of the time; otherwise, the patch remains in MDM.

\medskip
\noindent \textbf{Ultramafic Modifier} High mortality wildfire (17\% of fires in this seral stage) returns the patch to ED. Low mortality wildfire (83\%) opens the stand up to MDO 36\% of the time; otherwise, the patch remains in MDM.

\end{adjustwidth}

\noindent\hrulefill

\paragraph{Mid Development - Closed Canopy Cover (MDC)}

\paragraph{Description} Sparse ground cover of grasses, forbs, and shrubs; moderate to dense cover of trees. Conifers are pole to medium-sized, with canopy cover from 70-100\%. Conifer species likely present include \emph{A. concolor, C. decurrens, P. ponderosa, P. menziesii}, and \emph{P. lambertiana}. \emph{Q. kelloggi} may occur as well, mostly on warmer slopes and where soils are less productive (LandFire 2007a, LandFire 2007b). Ultramafic sites will have similar species composition, especially at edges, but \emph{P. jeffreyi} and \emph{C. decurrens} are relatively more common (O'Geen et al. 2007).

\paragraph{Succession Transition}
\begin{adjustwidth}{2cm}{}
\textbf{Mesic Modifier }  Patches in the MDM seral stage begin transitioning to LDM once the time since transition to a mid development seral stage is at least 100 years in the absence of fire, at which point stands succeed to LDC at a rate of 0.4 per timestep. All patches succeed by 150 years in mid development. On average (across all canopy cover seral stages), patches remain in mid development for 113 years.

\medskip
\noindent \textbf{Xeric Modifier}  Transition to late seral seral stages may be delayed. Thus, in the absence of disturbance, patches in this seral stage will begin transitioning to LDC after 160 years in an mid development seral stage at a rate of 0.4 per time step and may be delayed in the mid development stage for up to 200 years. 

\medskip
\noindent \textbf{Ultramafic Modifier} Transition to late seral seral stages may be substantially delayed. Thus, in the absence of disturbance, patches in this seral stage will begin transitioning to LDC after 200 years in the mid development stage at a rate of 0.4 per time step and may be delayed in a mid development seral stage for up to 260 years.

\end{adjustwidth}

\paragraph{Wildfire Transition}
\begin{adjustwidth}{2cm}{}
\textbf{Mesic Modifier } High mortality wildfire (23\% of fires in this seral stage) returns the patch to ED. Low mortality wildfire (77\%) opens the stand up to MDM 53\% of the time; otherwise, the patch remains in MDC. 

\medskip
\noindent \textbf{Xeric Modifier} High mortality wildfire (48\% of fires in this seral stage) returns the patch to ED. Low mortality wildfire (52\%) opens the stand up to MDM 42\% of the time; otherwise, the patch remains in MDC.

\medskip
\noindent \textbf{Ultramafic Modifier} High mortality wildfire (23\% of fires in this seral stage) returns the patch to ED. Low mortality wildfire (77\%) opens the stand up to MDM 53\% of the time; otherwise, the patch remains in MDC.

\end{adjustwidth}

\noindent\hrulefill


\paragraph{Late Development - Open Canopy Cover (LDO)}

\paragraph{Description} Heterogenous ground cover of grasses, forbs, and low shrubs; low density (less than 40\% canopy cover) of large trees. Occurring in small to moderately-sized patches on southerly aspects and ridge tops. Upper canopy trees may be very large, but overall size classes vary with a patchy distribution and open canopy. This seral stage develops when low-mortality disturbance is fairly frequent; it persists as long as low-mortality fires continue to occur periodically. Conifer species likely present include \emph{A. concolor, C. decurrens, P. ponderosa, P. menziesii}, and \emph{P. lambertiana}. \emph{Q. kelloggi} may occur as well, mostly on warmer slopes and where soils are less productive (LandFire 2007a, LandFire 2007b). Ultramafic sites will have similar species composition, especially at edges, but \emph{P. jeffreyi} and \emph{C. decurrens} are relatively more common (O'Geen et al. 2007).


\paragraph{Succession Transition}
\begin{adjustwidth}{2cm}{}
\textbf{Mesic Modifier } In the presence of low mortality disturbance, patches in this seral stage can self-perpetuate, but after 15 years with no fire, these patches will begin transitioning to LDM at a rate of 0.9 per timestep.

\medskip
\noindent \textbf{Xeric Modifier}  Succession to LDM may occur after 20 years with no fire at a rate of 0.6 per timestep. 

\medskip
\noindent \textbf{Ultramafic Modifier} Patches occurring on ultramafic soils may succeed to LDC after 25 years with no fire at a rate of 0.2 per timestep.

\end{adjustwidth}

\paragraph{Wildfire Transition}
\begin{adjustwidth}{2cm}{}
\textbf{Mesic Modifier } High mortality wildfire (8\% of fires in this seral stage) returns the patch to early development. Low mortality wildfire (92\%) maintains LDO.

\medskip
\noindent \textbf{Xeric Modifier} High mortality wildfire (5\% of fires in this seral stage) returns the patch to early development. Low mortality wildfire (95\%) maintains LDO. 

\medskip
\noindent \textbf{Ultramafic Modifier} High mortality wildfire (8\% of fires in this seral stage) returns the patch to early development. Low mortality wildfire (92\%) maintains LDO.

\end{adjustwidth}

\noindent\hrulefill

\paragraph{Late Development - Moderate Canopy Cover (LDM)}

\paragraph{Description} Overstory of large and very large trees with canopy cover 40-70\%. Understory characterized by medium and smaller-sized shade-tolerant conifers (LandFire 2007a). Conifer species likely present include \emph{A. concolor, C. decurrens, P. ponderosa, P. menziesii}, and \emph{P. lambertiana}. \emph{Q. kelloggi} may occur as well, mostly on warmer slopes and where soils are less productive (LandFire 2007a, LandFire 2007b). Ultramafic sites will have similar species composition, especially at edges, but \emph{P. jeffreyi} and \emph{C. decurrens} are relatively more common (O'Geen et al. 2007).


\paragraph{Succession Transition} 
\begin{adjustwidth}{2cm}{}
\textbf{Mesic Modifier } In the presence of low mortality disturbance, patches in this seral stage can self-perpetuate, but after 15 years with no fire, these patches will begin transitioning to LDC at a rate of 0.9 per timestep.

\medskip
\noindent \textbf{Xeric Modifier} Succession to LDC may occur after 20 years with no fire at a rate of 0.6 per timestep. 

\medskip
\noindent \textbf{Ultramafic Modifier} Patches occurring on ultramafic soils may succeed to LDC after 25 years with no fire at a rate of 0.2 per timestep.

\end{adjustwidth}
\paragraph{Wildfire Transition}
\begin{adjustwidth}{2cm}{}
\textbf{Mesic Modifier } High mortality wildfire (14\% of fires in this seral stage) will return the patch to Early Development. Low mortality wildfire (86\%) usually has little effect, although 24\% of the time it opens the stand up to LDO. 

\medskip
\noindent \textbf{Xeric Modifier} High mortality wildfire (11\% of fires in this seral stage) will return the patch to Early Development. Low mortality wildfire (99\%) usually has little effect, although 30\% of the time it opens the stand up to LDO. 

\medskip
\noindent \textbf{Ultramafic Modifier} High mortality wildfire (14\% of fires in this seral stage) will return the patch to Early Development. Low mortality wildfire (86\%) usually has little effect, although 24\% of the time it opens the stand up to LDO. 

\end{adjustwidth}

\noindent\hrulefill

\paragraph{Late Development - Closed Canopy Cover (LDC)}

\paragraph{Description} Overstory of large and very large trees with canopy cover over 70\%. Understory characterized by medium and smaller-sized shade-tolerant conifers (LandFire 2007a). Conifer species likely present include \emph{A. concolor, C. decurrens, P. ponderosa, P. menziesii}, and \emph{P. lambertiana}. \emph{Q. kelloggi} may occur as well, mostly on warmer slopes and where soils are less productive (LandFire 2007a, LandFire 2007b). Ultramafic sites will have similar species composition, especially at edges, but \emph{P. jeffreyi} and \emph{C. decurrens} are relatively more common (O'Geen et al. 2007).

\paragraph{Succession Transition} In the absence of disturbance, patches in this seral stage will maintain, regardless of soil characteristics.

\paragraph{Wildfire Transition}

\begin{adjustwidth}{2cm}{}
\textbf{Mesic Modifier } High mortality wildfire (37\% of fires in this seral stage) will return the patch to Early Development. Low mortality wildfire (63\%) may have little effect, but 54\% of the time it opens the stand up to LDM. 

\medskip
\noindent \textbf{Xeric Modifier} High mortality wildfire (25\% of fires in this seral stage) will return the patch to Early Development. Low mortality wildfire (65.9\%) may have little effect, but 57\% of the time it opens the stand up to LDM. 

\medskip
\noindent \textbf{Ultramafic Modifier} High mortality wildfire (37\% of fires in this seral stage) will return the patch to Early Development. Low mortality wildfire (63\%) may have little effect, but 54\% of the time it opens the stand up to LDM.

\end{adjustwidth}

\noindent\hrulefill
\noindent\hrulefill

\subsubsection{Aspen Variant}

\paragraph{Early Development - Aspen (ED-A)}

\paragraph{Description} Grasses, forbs, low shrubs, and sparse to moderate cover of tree seedlings/saplings (primarily \emph{P. tremuloides}) with an open canopy. This seral stage is characterized by the recruitment of a new cohort of early successional, shade-intolerant tree species into an open area created by a stand-replacing disturbance.

Following disturbance, succession proceeds rapidly from an herbaceous layer to shrubs and trees, which invade together (Verner 1988). \emph{P. tremuloides} suckers over 6ft tall develop within about 10 years (LandFire 2007c). 



\paragraph{Succession Transition} Unless it burns, a patch in ED-A persists for 10 years, at which point it transitions to MD-A.

\paragraph{Wildfire Transition} High mortality wildfire (100\% of fires in this seral stage) recycles the patch through the ED-A seral stage. Low mortality wildfire is not modeled for this seral stage.

\noindent\hrulefill


\paragraph{Mid Development - Aspen (MD-A)}

\paragraph{Description} \emph{P. tremuloides} trees 5-16 in DBH. Canopy cover is highly variable, and can range from 40-100\%. These patches range in age from 10 to 110 years. Some understory conifers, including \emph{P. ponderosa}, \emph{P. lambertiana}, and \emph{A. concolor} are encroaching, but \emph{P. tremuloides} is still the dominant component of the stand (LandFire 2007c).

\paragraph{Succession Transition} Patches in the MD-A seral stage persist for at least 50 years in the absence of fire, after which stands begin transitioning to MD-AC at a rate of 0.6 per timestep. After 100 years since entering MD-A, any remaining patches transition to MD-AC. 

\paragraph{Wildfire Transition} High mortality wildfire (26\% of fires in this seral stage) recycles the patch through the ED-A seral stage. No transition occurs as a result of low mortality fire.

\noindent\hrulefill

\paragraph{Mid Development - Aspen with Conifer (MD-AC)}

\paragraph{Description} These stands have been protected from fire since the last stand-replacing disturbance. \emph{P. tremuloides} trees are predominantly 16in DBH and greater. Conifers are present and overtopping the \emph{P. tremuloides}. \emph{A. concolor} is a typical conifer that is successional to \emph{P. tremuloides}, and is depicted here, but other conifers including \emph{P. ponderosa} and \emph{P. lambertiana} are also possible. Conifers are pole to medium-sized, and conifer cover is at least 40\% (LandFire 2007c).

\paragraph{Succession Transition} Patches in the MD-AC seral stage persist for 100 years in the absence of high mortality fire, at which point which patches transition to LDC. 

\paragraph{Wildfire Transition} High mortality wildfire (18\% of fires in this seral stage) returns the patch to ED-A. Low mortality wildfire (82\%) maintains the patch in MD-AC.

\noindent\hrulefill

\paragraph{Late Development - Closed (LDC)}

\paragraph{Description} Some \emph{P. tremuloides} continue to be present in the understory, but large conifers are now the dominant tree species, having overtopped the \emph{P. tremuloides}. Smaller conifers are present in the midstory as well. Conifer species likely present include \emph{A. concolor, C. decurrens, P. ponderosa, P. menziesii}, and \emph{P. lambertiana}. (LandFire 2007a, LandFire 2007b, LandFire 2007c). This seral stage is analogous to the LDC seral stage for the SMC variant.

\paragraph{Succession Transition} In the absence of disturbance, patches in this seral stage will maintain, regardless of soil characteristics.

\paragraph{Wildfire Transition} High mortality wildfire (37\% of fires in this seral stage) will return the patch to ED-A. Low mortality wildfire (63\%) opens the stand up to LD-CA 54\% of the time.

\noindent\hrulefill


\paragraph{Late Development - Conifer with Aspen (LD-CA)}

\paragraph{Description} If stands are sufficiently protected from fire such that conifer species overtop \emph{P. tremuloides} and become large, they may be able to withstand some fire that more sensitive \emph{P. tremuloides} cannot. When this occurs, it creates a patch characterized by late development conifers, such as \emph{A. concolor, P. ponderosa}, or \emph{P. lambertiana}, and early seral \emph{P. tremuloides}. 

\paragraph{Succession Transition} Patches in the LD-CA seral stage persist for 70 years, at which time patches transition to LDC. 

\paragraph{Wildfire Transition} High mortality wildfire (14\% of fires in this seral stage) returns the patch to ED-A. Low mortality wildfire (86\%) maintains the stand in LD-CA. 

\noindent\hrulefill



\newpage
\subsection*{Seral Stage Classification}
\begin{table}[hbp]
\small
\centering
\caption{Classification of cover seral stage for SMC. Diameter at Breast Height (DBH) and Cover From Above (CFA) values taken from EVeg polygons. DBH categories are: null, 0-0.9'', 1-4.9'', 5-9.9'', 10-19.9'', 20-29.9'', 30''+. CFA categories are null, 0-10\%, 10-20\%, \dots , 90-100\%. Each row in the table below should be read with a boolean AND across each column of a row.}
\label{smc_classification}
\begin{tabular}{@{}lrrrrr@{}}
\toprule
\textbf{\begin{tabular}[l]{@{}l@{}}Cover \\ Condition\end{tabular}} & \textbf{\begin{tabular}[r]{@{}r@{}}Overstory Tree \\ Diameter 1 \\ (DBH)\end{tabular}} & \textbf{\begin{tabular}[r]{@{}r@{}}Overstory Tree \\ Diameter 2 \\ (DBH)\end{tabular}} & \textbf{\begin{tabular}[r]{@{}r@{}}Total Tree\\ CFA (\%)\end{tabular}} & \textbf{\begin{tabular}[r]{@{}r@{}}Conifer \\ CFA (\%)\end{tabular}} & \textbf{\begin{tabular}[r]{@{}r@{}}Hardwood \\ CFA (\%)\end{tabular}} \\ \midrule
Early All        & null           & any & any    & any    & any  \\
Early All        & 0-4.9''         & any & any    & any    & any  \\
Mid Open         & 5-19.9''        & any & null   & null   & null \\
Mid Open         & 5-19.9''        & any & 0-40   & any    & any  \\
Mid Open         & 5-19.9''        & any & null   & 0-40   & null \\
Mid Moderate     & 5-19.9''        & any & 40-70  & any    & any  \\
Mid Moderate     & 5-19.9''        & any & null   & 40-70  & null \\
Mid Closed       & 5-19.9''        & any & 70-100 & any    & any  \\
Mid Closed       & 5-19.9''        & any & null   & 70-100 & any  \\
Late Closed      & 20''+           & any & 70-100 & any    & any  \\
Late Closed      & 20''+           & any & null   & 70-100 & any  \\
Late Moderate    & 20''+           & any & 40-70  & any    & any  \\
Late Moderate    & 20''+           & any & null   & 40-70  & any  \\
Late Open        & 20''+           & any & null   & null   & null \\
Late Open        & 20''+           & any & 0-40   & any    & any  \\
Late Open        & 20''+           & any & null   & 0-40   & null  \\ \bottomrule
\end{tabular}
\end{table}

SMC-ASP seral stages were assigned manually using NAIP 2010 Color IR imagery to assess seral stage.



\clearpage

\subsection*{References}
\begin{hangparas}{.25in}{1} 
Allen, Barbara H. ``Sierran Mixed Conifer (SMC).'' \emph{A Guide to Wildlife Habitats of California}, edited by Kenneth E. Mayer and William F. Laudenslayer. California Deparment of Fish and Game, 1988, updated 2005. \burl{http://www.dfg.ca.gov/biogeodata/cwhr/pdfs/SMC.pdf}. Accessed 4 December 2012.

``CalVeg Zone 1.'' Vegetation Descriptions. Vegetation Classification and Mapping.  11 December 2008. U.S. Forest Service. \burl{http://www.fs.usda.gov/Internet/FSE_DOCUMENTS/fsbdev3_046448.pdf}. Accessed 2 April 2013.
Estes, Becky. Personal communication, 15 August 2013.

LandFire. ``Biophysical Setting Models.'' Biophysical Setting 0610280: Mediterranean California Mesic Mixed Conifer Forest and Woodland. 2007a. LANDFIRE Project, U.S. Department of Agriculture, Forest Service; U.S. Department of the Interior. \burl{http://www.landfire.gov/national_veg_models_op2.php}. Accessed 9 November 2012.

LandFire. ``Biophysical Setting Models.'' Biophysical Setting 0610270: Mediterranean California Dry-Mesic Mixed Conifer Forest and Woodland. 2007b. LANDFIRE Project, U.S. Department of Agriculture, Forest Service; U.S. Department of the Interior. \burl{http://www.landfire.gov/national_veg_models_op2.php}. Accessed 9 November 2012.

LandFire. ``Biophysical Setting Models.'' Biophysical Setting 0610610: Inter-Mountain Basins Aspen-Mixed Conifer Forest and Woodland. 2007c. LANDFIRE Project, U.S. Department of Agriculture, Forest Service; U.S. Department of the Interior. \burl{http://www.landfire.gov/national_veg_models_op2.php}. Accessed 7 January 2013.

LandFire. ``Biophysical Setting Models.'' Biophysical Setting 0710220: Klamath-Siskiyou Upper Montane Serpentine Mixed Conifer Woodland. 2007d. LANDFIRE Project, U.S. Department of Agriculture, Forest Service; U.S. Department of the Interior. \burl{http://www.landfire.gov/national_veg_models_op2.php}. Accessed 30 November 2012.

O'Geen, Anthony T., Randy A. Dahlgren, and Daniel Sanchez-Mata. ``California Soils and Examples of Ultramafic Vegetation.'' In \emph{Terrestrial Vegetation of California, 3rd Edition}, edited by Michael Barbour, Todd Keeler-Wolf, and Allan A. Schoenherr, 71-106. Berkeley and Los Angeles: University of California Press, 2007. 

Safford, Hugh S. Personal communication.

Skinner, Carl N. and Chi-Ru Chang. ``Fire Regimes, Past and Present.'' \emph{Sierra Nevada Ecosystem Project: Final report to Congress, vol. II, Assessments and scientific basis for management options}. Davis: University of California, Centers for Water and Wildland Resources, 1996.

Van de Water, Kip M. and Hugh D. Safford. ``A Summary of Fire Frequency Estimates for California Vegetation Before Euro-American Settlement.'' \emph{Fire Ecology} 7.3 (2011): 26-57. doi: 10.4996/fireecology.0703026.

Verner, Jared. ``Aspen (ASP).'' \emph{A Guide to Wildlife Habitats of California}, edited by Kenneth E. Mayer and William F. Laudenslayer. California Deparment of Fish and Game, 1988. \burl{http://www.dfg.ca.gov/biogeodata/cwhr/pdfs/ASP.pdf}. Accessed 4 December 2012.


\end{hangparas}


%% !TEX root = master.tex
\newpage
\section{Subalpine Conifer (SCN)}
\label{scn-description}

\subsection*{General Information}

\subsubsection{Cover Type Overview}

\textbf{Subalpine Conifer (SCN)}
\newline
Crosswalks
\begin{itemize}
	\item EVeg: Regional Dominance Type 1
	\begin{itemize}
		\item Alpine Mixed Scrub
		\item Mountain Hemlock
		\item Subalpine Conifers
		\item Whitebark Pine
	\end{itemize}

	\item LandFire BpS Model
	\begin{itemize}
		\item Subalpine Conifer
	\end{itemize}

	\item Presettlement Fire Regime Type
	\begin{itemize}
		\item 0610330 Mediterranean California Subalpine Woodland
		\item 0610440 Northern California Mesic Subalpine Woodland
		\item 0610710 Sierra Nevada Alpine Dwarf-Shrubland
	\end{itemize}
\end{itemize}

\noindent \textbf{Subalpine Conifer with Aspen (SCN-ASP)}
This type is created by overlaying the NRIS TERRA Inventory of Aspen on top of the EVeg layer. Where it intersects with SCN it is assigned to SCN-ASP.
\newline

\noindent Reviewed by Marc Meyer, Southern Sierra Province Ecologist, USDA Forest Service

\subsubsection{Vegetation Description}
\textbf{Subalpine Conifer (SCN)} The SCN landscape is comprised of a mosaic of subalpine forests/woodlands, meadows, rock outcrops, and scrub vegetation types. These forests are open stands of conifers occurring on generally sandy soils or rocky slopes at elevations above the upper montane forest stands of \emph{Abies magnifica}. Stand densities are low. Many, but not all, species form shrubby krummholz forms of growth near their upper elevational limits (Fites-Kaufman 2007). 

\emph{Tsuga mertensiana} is often the most common tree species and mixes with \emph{P. contorta} ssp. \emph{murrayana, A. magnifica, Pinus monticola}, and \emph{Pinus albicaulis}. In some areas, \emph{P. contorta} ssp. \emph{murrayana} dominates post-disturbances stands. \emph{T. mertensiana} seedlings are relatively shade tolerant compared to other subalpine conifers and do well under closed canopy conditions. \emph{P. albicaulis} presence increases in the southern portion of the project area (Fites-Kaufman 2007, LandFire 2007a).

Treeline growth of multistemmed trees and shrubby krummholz growth of conifers varies with latitude in the Sierra Nevada. Treeline in the northern Sierra Nevada is dominated by \emph{P. albicaulis}, which frequently occurs with a krummholz form of growth near its upper limit. Several other species may also form krummholz growth forms, including \emph{Juniperus occidentalis}, \emph{Tsuga mertensiana}, \emph{P. contorta} ssp. \emph{murrayana}, and rarely \emph{Pinus jeffreyi} (Fites-Kaufman 2007). 

Although typically of minor importance, a shrub understory may include \emph{Arctostaphylos, Ribes, Phyllodoce, Vaccinium}, and \emph{Kalmia} can occur on moist sites. Herbs present may include \emph{Lupinus, Hieracium, Arabis, Aster}, and \emph{Erigeron}. \emph{Carex} and various grasses are also common (Verner and Purcell 1988, LandFire 2007a).

\medskip
\noindent \textbf{Subalpine Conifer with Aspen (SCN-ASP)} These are upland forests and woodlands dominated by \emph{Populus tremuloides} without a significant conifer component. Conifers may be present in these systems; however, these patches of \emph{P. tremuloides} are not typically successional to conifers. The understory structure may be complex with multiple shrub and herbaceous layers, or simple with just an herbaceous layer. The herbaceous layer may be dense or sparse, dominated by graminoids or forbs. Common shrubs include \emph{Acer, Amelanchier, Artemisia, Juniperus, Prunus, Rosa, Shepherdia, Symphoricarpos}, and the dwarf-shrubs \emph{Mahonia} and \emph{Vaccinium}. Common graminoids may include \emph{Bromus, Calamagrostis, Carex, Elymus, Festuca}, and \emph{Hesperostipa}. Associated forbs may include \emph{Achillea, Eucephalus, Delphinium, Geranium, Heracleum, Ligusticum, Lupinus, Osmorhiza, Pteridium, Rudbeckia, Thalictrum, Valeriana, Wyethia}, and many
others (LandFire 2007b).


\subsubsection{Distribution}
\textbf{Subalpine Conifer (SCN)} The elevational distribution of subalpine forest communities varies with latitude. In the northern Sierra Nevada, such stands begin around 2,450 m and extend up to treeline at 2,750 m to 3,100 m (9,000 ft to 11,000 ft). Both upper and lower limits of subalpine species distributions are driven by a variety of factors, including soil resources, water availability, and climatic limiting factors (Fites-Kaufman 2007).

These forests are characterized by a relatively short growing season with cool temperatures. With the exception of occasional summer thunderstorms, most precipitation falls as snow. Wet years with abundant snowfall can limit growth as these may produce late-lying snowfields that reduce the length of the growing season. Winds can be severe, particularly around exposed ridges. Such wind conditions may produce snow-free winter areas that lower soil temperatures and increase plant water stress (Fites-Kaufman 2007).

Because of the solid granite parent material, areas with deeper soil accumulation can become waterlogged for much of the year. For these reasons, the length of the growing season is a function of not only early season limitation due to low temperatures and snowfields, but also late season limitations due to drought. Studies of the dynamics of alterations of treeline elevation over the past several millennia have reinforced the significance of complex interactions of both temperature and seasonal water availability in determining such changes (Fites-Kaufman 2007). 


\textbf{Subalpine Conifer with Aspen (SCN-ASP)} Sites supporting \emph{P. tremuloides} are associated with added soil moisture, i.e., azonal wet sites. These sites are often close to streams, lakes, and meadows. Other sites include rock reservoirs, springs and seeps. Terrain can be simple to complex. At lower elevations, topographic conditions for this type tends toward positions resulting in relatively colder, wetter conditions within the prevailing climate, e.g., ravines, north slopes, wet depressions, etc. (LandFire 2007b). \emph{P. tremuloides} stands may also be associated with lateral or terminal moraine boulder material, talus-colluvium, rock falls, or lava flows. In addition, pure stands may be found in topographic positions where snow accumulates, mostly at higher north facing elevations, where snow presence means the growing season is too short to support conifers (Shepperd et al. 2006).

\subsection*{Disturbances}

\subsubsection{Wildfire}
\textbf{Subalpine Conifer (SCN)} Most of the subalpine areas of the Sierra Nevada were subjected to repeated glaciation during the Pleistocene, and thus have thin and poorly developed soils with little organic matter. The small amounts of litter accumulation and open stand structure of subalpine forests mean that fire is rare (Fites-Kaufman 2007). It is, however, the major disturbance event of this type (LandFire 2007a). Meyer’s 2013 review suggests that historic and current fire regimes in subalpine forests are normally climate-limited and dominated by surface fires with crown fires occurring occasionally.  

\textbf{Subalpine Conifer with Aspen (SCN-ASP)} Sites supporting \emph{P. tremuloides} are maintained by stand-replacing disturbances that allow regeneration from below-ground suckers. Replacement fire and ground fire are thought to have been common in stable \emph{P. tremuloides} stands historically. Because \emph{P. tremuloides} is associated with mesic conditions, it rarely burns during the normal lightning season. However, during years with little precipitation stands may be more susceptible to burning. Evidence from fire scars and historical studies show that past fires occurred mostly during the spring and fall. These are typically self-perpetuating stands (LandFire)

Estimates of fire rotations for these variants are available from the LandFire project and a few review papers. The LandFire project’s published fire return intervals are based on a series of associated models created using the Vegetation Dynamics Development Tool (VDDT). In VDDT, fires are specified concurrently with the transition that follows them. For example, a replacement fire causes a transition to the early development stage. In the RMLands model, such fires are classified as high mortality. However, in VDDT mixed severity fires may cause a transition to early development, a transition to a more open seral stage, or no transition at all. In this case, we categorize the first example as a high mortality fire, and the second and third examples as a low mortality fire. Based on this approach, we calculated fire rotations and the probability of high mortality fire for each of the SCN and SCN-ASP seral stage (Tables~\ref{tab:scndesc_fire} and \ref{tab:rfr-aspdesc_fire}). We computed overall target fire rotations based on values from Mallek et al. (2013) and Van de Water and Safford (2011) as well as consultations with Meyer, Safford, and Estes (personal communication). 





\begin{table}[!htbp]
\footnotesize
\centering
\caption{Fire rotation index values and probability of high severity fire (at least 75\% overstory tree mortality) probabilities for Subalpine Conifer. The seral stage that is most susceptible to fire (i.e., has the lowest predicted fire rotation) has a fire rotation index value of 1. Higher values correspond with lower likelihoods of experiencing wildfire. The values are relative only within an individual seral stage and should not be compared against other land cover types. Values were derived from VDDT model 0610440 (LandFire 2007), Mallek et al. (2013), and Estes, Safford, and Meyer (personal communication). }
\label{tab:scndesc_fire}
\begin{tabular}{@{}lcc@{}}
\toprule
 \textbf{Seral Stage}    & \textbf{\begin{tabular}[c]{@{}c@{}}Fire Rotation \\ Index\end{tabular}} & \textbf{\begin{tabular}[c]{@{}c@{}}Probability of \\ High Severity Fire\end{tabular}} \\ \hline
Early (All)     		 & 1.65           & 1           \\
Mid--Closed    			 & 1.10           & 0.67        \\
Mid--Moderate  			 & 1.05           & 0.63        \\
Mid--Open      			 & 1.00           & 0.61        \\
Late--Closed   			 & 1.10           & 0.67        \\
Late--Moderate 			 & 1.05           & 0.63        \\
Late--Open     			 & 1.00           & 0.61 	      \\ 
\emph{Target Fire Rotation}    			& \emph{296 years}  &   \\ 
\bottomrule
\end{tabular}
\end{table}

\begin{table}[!htbp]
\footnotesize
\centering
\caption{Fire rotation index values and probability of high severity fire (at least 75\% overstory tree mortality) probabilities for Subalpine Conifer - Aspen type. The seral stage that is most susceptible to fire (i.e., has the lowest predicted fire rotation) has a fire rotation index value of 1. Higher values correspond with lower likelihoods of experiencing wildfire. The values are relative only within an individual seral stage and should not be compared against other land cover types. Values were derived from VDDT model 0610110 (LandFire 2007), Van de Water and Safford (pers. comm. 2013), Safford, and Estes (personal communication).}
\label{tab:scnasp-desc_fire}
\begin{tabular}{@{}lcc@{}}
\toprule
 \textbf{Seral Stage}    & \textbf{\begin{tabular}[c]{@{}c@{}}Fire Rotation \\ Index\end{tabular}} & \textbf{\begin{tabular}[c]{@{}c@{}}Probability of \\ High Severity Fire\end{tabular}} \\ \hline
Early--Aspen        & 2.0           & 1         \\
Mid--Aspen          & 1.0           & 0.67      \\
Late--Conifer-Aspen & 1.3           & 0.63		   \\ 
\emph{Target Fire Rotation}    			& \emph{296 years}  &   \\ 
\bottomrule
\end{tabular}
\end{table}

\subsubsection{Other Disturbance}
Other disturbances are not currently modeled, but may, depending on the seral stage affected and mortality levels, reset patches to early development, maintain existing seral stage, or shift/accelerate succession to a more open seral stage. 

\subsection*{Vegetation Seral Stages}
We recognize seven separate seral stages for SCN: Early Development (ED), Mid Development - Open Canopy Cover (MDO), Mid Development - Moderate Canopy Cover, Mid Development - Closed Canopy Cover (MDC), Late Development - Open Canopy Cover (LDO), Late Development - Moderate Canopy Cover (LDM), and Late Development - Closed Canopy Cover (LDC) (Figure~\ref{transmodel_scn}). The SCN-ASP variant is assigned to three seral stages: Early Development - Aspen (ED-A), Mid Development - Aspen (MD-A), and Late Development - Conifer with Aspen (LD-CA) (Figure~\ref{transmodel_scn-asp}).

Our seral stages are an alternative to ``successional'' classes that imply a linear progression of states and tend not to incorporate disturbance. The seral stages identified here are derived from a combination of successional processes and anthropogenic and natural disturbance, and are intended to represent a composition and structural condition that can be arrived at from multiple other conditions described for that landcover type. Thus our seral stages incorporate age, size, canopy cover, and vegetation composition. In general, the delineation of stages has originated from the LandFire biophysical setting model descriptive of a given landcover type; however, seral stages are not necessarily identical to the classes identified in those models.


\begin{figure}[htbp]
\centering
\includegraphics[width=0.8\textwidth]{/Users/mmallek/Documents/Thesis/statetransmodel/StateTransitionModel/7class.png}
\caption{State and Transition Model for Subalpine Conifer Forest (not inclusive of the aspen variant). Each dark grey box represents one of the seven seral stage for this landcover type. Each column of boxes represents a stage of development: early, middle, and late. Each row of boxes represents a different level of canopy cover: closed (70-100\%), moderate (40-70\%), and open (0-40\%). Transitions between states/seral stage may occur as a result of high mortality fire, low mortality fire, or succession. Specific pathways for each are denoted by the appropriate color line and arrow: red lines relate to high mortality fire, orange lines relate to low mortality fire, and green lines relate to natural succession.} 
\label{transmodel_scn}
\end{figure}

\begin{figure}[htbp]
\centering
\includegraphics[width=0.8\textwidth]{/Users/mmallek/Documents/Thesis/statetransmodel/StateTransitionModel/3class-asp.png}
\caption{State and Transition Model for Subalpine Conifer Forest, Aspen variant. Each dark grey box represents one of the three seral stages for this landcover type. Three seral stages of development are represented: early, middle, and late. Transitions between states/seral stages may occur as a result of high mortality fire, low mortality fire, or succession. Specific pathways for each are denoted by the appropriate color line and arrow: red lines relate to high mortality fire, orange lines relate to low mortality fire, and green lines relate to natural succession.} 
\label{transmodel_scn-asp}
\end{figure}

\paragraph{Early Development (ED)}

\paragraph{Description} The first few years following stand-replacing wildfire are characterized by bare ground, herbs, shrubs, and varying densities of tree seedlings (presumably dependent on seed sources). Dominant species include coniferous tree seedlings, resprouting grasses and shrubs, and invading herbs. Shrubs include \emph{Ribes} spp. Herbs and grasses include \emph{Aster, Pedicularis, Hieracium, Arabis, Erigeron, Carex, Luzula}, and \emph{Poa} (LandFire 2007a).

\paragraph{Succession Transition} In the absence of disturbance, patches in this seral stage will begin transitioning to mid development after 20 years at a rate of 0.4 per time step. Transition to either MDC or MDO can occur, although transition to MDC occurs 90\% of the time. At 80 years, all patches will succeed. On average, patches remain in ED for 33 years.

\paragraph{Wildfire Transition} High mortality wildfire (100\% of fires) recycles the patch through the Early Development seral stage. Low mortality wildfire is not modeled for this seral stage.

\noindent\hrulefill


\paragraph{Mid Development - Open Canopy Cover (MDO)} 

\paragraph{Description} This seral stage represents delayed tree regeneration and long-term domination by shrubs and herbs. Shrubs include \emph{Ribes} spp. Herbs and grasses include \emph{Aster, Pedicularis, Hieracium, Arabis, Erigeron, Carex, Luzula}, and \emph{Poa}. Trees are represented by seedlings and saplings of \emph{T. mertensiana}, \emph{P. contorta} ssp. \emph{murrayana}, and other species (LandFire 2007a).

\paragraph{Succession Transition} Patches in this seral stage will maintain under low mortality disturbance. In the absence of low mortality disturbance, patches in the MDO seral stage will begin transitioning to MDM after 40 years at a rate of 0.3 per timestep. Succession to LDO takes place variably after 60 years since entering a middle development seral stage, at a rate of 0.45 per timestep. All patches succeed by 130 years in mid development.  On average (across all canopy cover seral stages), patches remain in mid development for 71 years.

\paragraph{Wildfire Transition} High mortality wildfire (61\% of fires) recycles the patch through the Early Development seral stage. Low mortality wildfire (39\%) maintains the patch in MDO.

\noindent\hrulefill

\paragraph{Mid Development - Moderate Canopy Cover (MDM)}

\paragraph{Description} This seral stage represents rapid regeneration by \emph{P. contorta} ssp. \emph{murrayana}, with additional conifers coming in, including \emph{T. mertensiana}, \emph{A. magnifica}, and \emph{P. monticola}. Shrubs include \emph{Ribes} spp. Herbs and grasses include \emph{Aster, Pedicularis, Hieracium, Arabis, Erigeron, Carex, Luzula}, and \emph{Poa}. (LandFire 2007a).

\paragraph{Succession Transition} In the absence of low mortality disturbance, patches in the MDM seral stage will begin transitioning to MDC after 40 years at a rate of 0.3 per timestep. Succession to LDM takes place variably after 60 years since entering a middle development seral stage, at a rate of 0.45 per timestep. All patches succeed by 130 years in mid development.  On average (across all canopy cover seral stages), patches remain in mid development for 71 years.
 
\paragraph{Wildfire Transition} High mortality wildfire (63\% of fires) recycles the patch through the Early Development seral stage. Low mortality wildfire (37\%) triggers a transition to MDO.

\noindent\hrulefill

\paragraph{Mid Development - Closed Canopy Cover (MDC)}

\paragraph{Description} This seral stage represents rapid regeneration by \emph{P. contorta} ssp. \emph{murrayana}, with additional conifers coming in, including \emph{T. mertensiana}, \emph{A. magnifica}, and \emph{P. monticola}. Shrubs include \emph{Ribes} spp. Herbs and grasses include \emph{Aster, Pedicularis, Hieracium, Arabis, Erigeron, Carex, Luzula}, and \emph{Poa}. (LandFire 2007a).

\paragraph{Succession Transition} After 60 years without a wildfire-triggered transition, patches in this seral stage will begin transitioning to LDC at a rate of 0.45 per time step. Succession to LDC may occur once the patch age since transition to the mid development stage is at least 60 years. After 130 years, all patches will succeed.

\paragraph{Wildfire Transition} High mortality wildfire (67\% of fires) recycles the patch through the Early Development seral stage. Low mortality wildfire (33\%) triggers a transition to MDM.

\noindent\hrulefill


\paragraph{Late Development - Open Canopy Cover (LDO)}

\paragraph{Description} This seral stage represents late-successional stands with large individuals (DBH greater than 20 in) of \emph{T. mertensiana} and other species. The open stand structure is maintained by mixed severity fire and insect-caused tree mortality (the latter not modeled at this time). Shrubs include \emph{Ribes} spp. Herbs and grasses include \emph{Aster, Pedicularis, Hieracium, Arabis, Erigeron, Carex, Luzula}, and \emph{Poa}. (LandFire 2007a).

\paragraph{Succession Transition} In the absence of any fire, succession to LDM begins at 40 years at a rate of 0.3 per timestep.

\paragraph{Wildfire Transition} High mortality wildfire (61\% of fires) recycles the patch through the Early Development seral stage. Low mortality wildfire (39\%) maintains the patch in LDO.

\noindent\hrulefill

\paragraph{Late Development - Moderate Canopy Cover (LDM)}

\paragraph{Description} This seral stage represents late-successional stands with large individuals (DBH greater than 20 in) of \emph{T. mertensiana} and other species, and advanced regeneration of \emph{T. mertensiana} and other shade tolerant species. The moderately open stand structure is generated by recent low mortality fire and insect-caused tree mortality (the latter not modeled at this time). Shrubs include \emph{Ribes} spp. Herbs and grasses include \emph{Aster, Pedicularis, Hieracium, Arabis, Erigeron, Carex, Luzula}, and \emph{Poa}. (LandFire 2007a).

\paragraph{Succession Transition} In the absence of any fire, succession to LDC begins at 40 years at a rate of 0.3 per timestep.

\paragraph{Wildfire Transition} High mortality wildfire (63\% of fires) recycles the patch through the Early Development seral stage. Low mortality wildfire (37\%) triggers a transition to LDO. 

\noindent\hrulefill

\paragraph{Late Development - Closed Canopy Cover (LDC)}

\paragraph{Description} This seral stage represents late-successional stands with large individuals (DBH greater than 20 in) of \emph{T. mertensiana} and other species, and advanced regeneration of \emph{T. mertensiana} and other shade tolerant species. Shrubs include \emph{Ribes} spp. Herbs and grasses include \emph{Aster, Pedicularis, Hieracium, Arabis, Erigeron, Carex, Luzula}, and \emph{Poa}. (LandFire 2007a).

\paragraph{Succession Transition} Patches in this seral stage will maintain in the absence of disturbance.

\paragraph{Wildfire Transition} High mortality wildfire (67\% of fires) recycles the patch through the Early Development seral stage. Low mortality wildfire (33\%) triggers a transition to LDM. 

\noindent\hrulefill
\noindent\hrulefill

\subsubsection{Aspen Variant}

\paragraph{Early Development - Aspen (ED-A)}

\paragraph{Description} Grasses, forbs, low shrubs, and sparse to moderate cover of tree seedlings/saplings (primarily \emph{P. tremuloides}) with an open canopy. This seral stage is characterized by the recruitment of a new cohort of early successional, shade-intolerant tree species into an open area created by a stand-replacing disturbance. Following disturbance, succession proceeds rapidly from an herbaceous layer to shrubs and trees, which invade together (Verner 1988). \emph{P. tremuloides} suckers over 6ft tall develop within about 10 years (LandFire 2007b). 


\paragraph{Succession Transition} Unless it burns, a patch in the early seral stage persists for 10 years, at which point it transitions to MD-A.

\paragraph{Wildfire Transition} High mortality wildfire (100\% of fires) recycles the patch through the ED-A seral stage. Low mortality wildfire is not modeled for this seral stage.

\noindent\hrulefill


\paragraph{Mid Development - Aspen (MD-A)}

\paragraph{Description} \emph{P. tremuloides} trees 5-16'' DBH. Canopy cover is highly variable, and can range from 40-100\%. These patches range in age from 10 to 110 years (LandFire 2007b).

\paragraph{Succession Transition} Patches in the MD-A seral stage persist for at least 80 years in the absence of fire, at which point they begin transitioning to LD-CA at a rate of 0.3 per timestep. After 200 years since entering MD-A, any remaining patches transition to LD-CA. 

\paragraph{Wildfire Transition} High mortality wildfire (67\% of fires in this seral stage) recycles the patch through the ED-A seral stage. No transition occurs as a result of low mortality fire (33\%).

\noindent\hrulefill



\paragraph{Late Development - Aspen with Conifer (LD-AC)}

\paragraph{Description} These stands have been protected from fire since the last stand-replacing disturbance. \emph{P. tremuloides} trees are predominantly 16'' DBH and greater. Conifers are encroaching and can eventually overtop the aspen (LandFire 2007).

\paragraph{Succession Transition} Patches in this seral stage will maintain in the absence of disturbance.

\paragraph{Wildfire Transition} High mortality wildfire (63\% of fires in this seral stage) returns the patch to ED-A. Low mortality wildfire (33\%) maintains the patch in LD-CA.

\noindent\hrulefill




\subsection*{Seral Stage Classification}
\begin{table}[hbp]
\footnotesize
\centering
\caption{Classification of cover seral stage for SCN. Diameter at Breast Height (DBH) and Cover From Above (CFA) values taken from EVeg polygons. DBH categories are: null, 0-0.9'', 1-4.9'', 5-9.9'', 10-19.9'', 20-29.9'', 30''+. CFA categories are null, 0-10\%, 10-20\%, \dots , 90-100\%. Each row in the table below should be read with a boolean AND across each column.}
\label{scn_classification}
\begin{tabular}{@{}lrrrrr@{}}
\toprule
\textbf{\begin{tabular}[l]{@{}l@{}}Cover \\ Condition\end{tabular}} & \textbf{\begin{tabular}[r]{@{}r@{}}Overstory Tree \\ Diameter 1 \\ (DBH)\end{tabular}} & \textbf{\begin{tabular}[r]{@{}r@{}}Overstory Tree \\ Diameter 2 \\ (DBH)\end{tabular}} & \textbf{\begin{tabular}[r]{@{}r@{}}Total Tree\\ CFA (\%)\end{tabular}} & \textbf{\begin{tabular}[r]{@{}r@{}}Conifer \\ CFA (\%)\end{tabular}} & \textbf{\begin{tabular}[r]{@{}r@{}}Hardwood \\ CFA (\%)\end{tabular}} \\ \midrule
Early All        & null           & any & any    & any    & any  \\
Early All        & 0-4.9''         & any & any    & any    & any  \\
Mid Open         & 5-19.9''        & any & null   & null   & null \\
Mid Open         & 5-19.9''        & any & 0-40   & any    & any  \\
Mid Open         & 5-19.9''        & any & null   & 0-40   & null \\
Mid Moderate     & 5-19.9''        & any & 40-70  & any    & any  \\
Mid Moderate     & 5-19.9''        & any & null   & 40-70  & null \\
Mid Closed       & 5-19.9''        & any & 70-100 & any    & any  \\
Mid Closed       & 5-19.9''        & any & null   & 70-100 & any  \\
Late Closed      & 20''+           & any & 70-100 & any    & any  \\
Late Closed      & 20''+           & any & null   & 70-100 & any  \\
Late Moderate    & 20''+           & any & 40-70  & any    & any  \\
Late Moderate    & 20''+           & any & null   & 40-70  & any  \\
Late Open        & 20''+           & any & null   & null   & null \\
Late Open        & 20''+           & any & 0-40   & any    & any  \\
Late Open        & 20''+           & any & null   & 0-40   & null \\ \bottomrule
\end{tabular}
\end{table}

SCN-ASP seral stages were assigned manually using NAIP 2010 Color IR imagery to assess seral stage.


\clearpage

\subsection*{References}

\begin{hangparas}{.25in}{1} 
\interlinepenalty=10000

Estes, Becky. Personal communication, 15 August 2013.

Fites-Kaufman, Jo Ann, Phil Rundel, Nathan Stephenson, and Dave A. Wixelman. ``Montane and Subalpine Vegetation of the Sierra Nevada and Cascade Ranges.'' In \emph{Terrestrial Vegetation of California, 3rd Edition}, edited by Michael Barbour, Todd Keeler-Wolf, and Allan A. Schoenherr, 456-501. Berkeley and Los Angeles: University of California Press, 2007. 

LandFire. ``Biophysical Setting Models.'' Biophysical Setting 0610440: Northern California Mesic Subalpine Woodland. 2007a. LANDFIRE Project, U.S. Department of Agriculture, Forest Service; U.S. Department of the Interior. \burl{http://www.landfire.gov/national_veg_models_op2.php}. Accessed 9 November 2012.

LandFire. ``Biophysical Setting Models.'' Biophysical Setting 0610610: Inter-Mountain Basins Aspen-Mixed Conifer Forest and Woodland. 2007b. LANDFIRE Project, U.S. Department of Agriculture, Forest Service; U.S. Department of the Interior. \burl{http://www.landfire.gov/national_veg_models_op2.php}. Accessed 7 January 2013.

Meyer, Marc D. ``Natural Range of Variation of Red Fir Forests in the Bioregional Assessment Area'' (unpublished paper, Ecology Group, Pacific Southwest Research Station, 2013).

Safford, Hugh S. Personal communication, 5 May 2013, 15 August 2013.

Shepperd, Wayn De, Paul C. Rogers, David Burton, and Dale L. Bartos. ``Ecology, Biodiversity, Management, and Restoration of Aspen in the Sierra Nevada.'' General Technical Report RMRS-GTR-178. Rocky Mountain Research Station, Forest Service, U.S. Department of Agriculture, 2006.

Van de Water, Kip M. and Safford, Hugh D. ``A Summary of Fire Frequency Estimates for California Vegetation Before Euro-American Settlement.'' \emph{Fire Ecology} 7.3 (2011): 26-57. doi: 10.4996/fireecology.0703026.

Verner, Jared. ``Aspen (ASP).'' \emph{A Guide to Wildlife Habitats of California}. 1988. Mayer, Kenneth E. and Laudenslayer, William F., eds. California Department of Fish and Game. \burl{http://www.dfg.ca.gov/biogeodata/cwhr/pdfs/ASP.pdf}. Accessed 4 December 2012.

\end{hangparas}


%% !TEX root = master.tex
\newpage
\section{Western White Pine (WWP)}
\label{wwp-description}
\subsection*{General Information}

\subsubsection{Cover Type Overview}

\textbf{Western White Pine (WWP)}
\newline
\textbf{Crosswalks}
\begin{itemize}
	\item EVeg: Regional Dominance Type 1
	\begin{itemize}
		\item Western White Pine 
	\end{itemize}

	\item LandFire BpS Model
	\begin{itemize}
		\item 0711720: Sierran-Intermontane Desert Western White Pine-White Fir Woodland
	\end{itemize}

	\item Presettlement Fire Regime Type
	\begin{itemize}
		\item Western White Pine
	\end{itemize}
\end{itemize}

\noindent Reviewed by Becky Estes, Central Sierra Province Ecologist, USDA Forest Service

\subsubsection{Vegetation Description}
\emph{Pinus monticola} is locally abundant in subalpine habitats along the west slope of the Sierra Nevada, where it may occur in small pure stands. More commonly, it mixes with \emph{Pinus contorta} ssp. \emph{murrayana, Pinus jeffreyi, Tsuga mertensiana}, and \emph{Abies magnifica} (particularly on the west side of the Sierra crest) and \emph{Abies concolor} or \emph{Pinus ponderosa} (particularly on the east side) (Fites-Kaufman et al. 2007, LandFire 2007, Estes pers. comm. 2013).

This system tends to be more woodland than forest in character, and the undergrowth is more open and drier, with little shrub or herbaceous cover. Tree regeneration is less prolific than in other mixed-montane conifer systems of the Cascades, Sierras and California Coast Ranges (LandFire 2007). \emph{P. monticola} generally maintains a tree form of growth up nearly to treeline, where it is commonly replaced by other subalpine species on rocky ridges (Fites-Kaufman et al. 2007).

Understories are typically open, with moderately low shrub cover and diversity, and include \emph{Arctostaphylos, Chrysolepis, Ceanothus}, and \emph{Ribes}. Common herbaceous taxa include \emph{Arnica, Festuca, Poa, Carex, Pyrola}, and \emph{Hieracium}. In openings, \emph{Wyethia} can be abundant (LandFire 2007).


\subsubsection{Distribution}
With respect to the focal landscape within the northern Sierra Nevada, these forests and woodlands are found in the upper montane to subalpine zones, at elevations generally over 2000 m (6560 ft). 

It is found on all slopes and aspects, although it occurs more frequently on drier areas. This ecological system generally occurs on basalts, andesite, glacial till, basaltic rubble, colluvium, or volcanic ash-derived soils. These soils have characteristic features of good aeration and drainage, coarse textures, circumneutral to slightly acidic pH, an abundance of mineral material, rockiness, and periods of drought during the growing season. Climatically, this system occurs somewhat in the rain shadow of the Sierras and has a more continental regime, similar to the northern Great Basin (LandFire 2007).


\subsection*{Disturbances}
Most fires in this type are low mortality fires that allow large areas of the landscape to develop mature characteristics. Occasional severe fires are driven by weather extremes (LandFire 2007). Young trees are very susceptible to mortality from fire, but mature \emph{P. monticola} is moderately fire resistant. After a stand-replacing fire, \emph{P. monticola} will seed in from adjacent areas. After a cool to moderate fire that leaves a mosaic of mineral soil and duff, it will reoccupy the site from seed stored in the seed bank. Overall, \emph{P. monticola} is a fire-dependent, seral species. Fire suppression has resulted in decreased stocking levels, mostly due to the increase in White pine blister rust (\emph{Cronartium ribicola}). Periodic, stand-replacing fire or other disturbance is needed to remove competing conifers and allow \emph{P. monticola} to develop (Griffith 1992). 

Estimates of fire rotations for these variants are available from the LandFire project and a few review papers. The LandFire project’s published fire return intervals are based on a series of associated models created using the Vegetation Dynamics Development Tool (VDDT). In VDDT, fires are specified concurrently with the transition that follows them. For example, a replacement fire causes a transition to the early development stage. In the RMLands model, such fires are classified as high mortality. However, in VDDT mixed severity fires may cause a transition to early development, a transition to a more open seral stage, or no transition at all. In this case, we categorize the first example as a high mortality fire, and the second and third examples as a low mortality fire. Based on this approach, we calculated fire rotations and the probability of high mortality fire for each of the WWP and WWP-ASP seral stages (Table~\ref{tab:wwpdesc_fire}). We computed overall target fire rotations based on values from Van de Water and Safford (2011). 

\subsubsection{Wildfire}




\begin{table}[!htbp]
\footnotesize
\centering
\caption{Fire rotation index values and probability of high severity fire (at least 75\% overstory tree mortality) probabilities. The seral stage that is most susceptible to fire (i.e., has the lowest predicted fire rotation) has a fire rotation index value of 1. Higher values correspond with lower likelihoods of experiencing wildfire. The values are relative only within an individual seral stage and should not be compared against other land cover types. Values were derived from VDDT model 0711720 (LandFire 2007) and Van de Water and Safford (2011). }
\label{tab:wwpdesc_fire}
\begin{tabular}{@{}lcc@{}}
\toprule
 \textbf{Seral Stage}    & \textbf{\begin{tabular}[c]{@{}c@{}}Fire Rotation \\ Index\end{tabular}} & \textbf{\begin{tabular}[c]{@{}c@{}}Probability of \\ High Severity Fire\end{tabular}} \\ \hline
Early (All)     		 & 1.8            & 0.17         \\
Mid--Closed    			 & 1.8            & 0.18         \\
Mid--Moderate  			 & 1.3            & 0.12         \\
Mid--Open      			 & 1.0            & 0.09         \\
Late--Closed   			 & 1.8            & 0.17         \\
Late--Moderate 			 & 1.3            & 0.12         \\
Late--Open     			 & 1.0            & 0.09 	    \\ 
\emph{Target Fire Rotation}    			& \emph{88 years}  &   \\ 
\bottomrule
\end{tabular}
\end{table}

\subsubsection{Other Disturbance}
Other disturbances are not currently modeled, but may, depending on the seral stage affected and mortality levels, reset patches to early development, maintain existing seral stages, or shift/accelerate succession to a more open seral stage. 

\subsection*{Vegetation Seral Stages}
We recognize seven separate seral stages for WWP: Early Development (ED), Mid Development - Open Canopy Cover (MDO), Mid Development - Moderate Canopy Cover, Mid Development - Closed Canopy Cover (MDC), Late Development - Open Canopy Cover (LDO), Late Development - Moderate Canopy Cover (LDM), and Late Development - Closed Canopy Cover (LDC) (Figure~\ref{transmodel_wwp}). Our seral stages are an alternative to ``successional'' classes that imply a linear progression of states and tend not to incorporate disturbance. The seral stages identified here are derived from a combination of successional processes and anthropogenic and natural disturbance, and are intended to represent a composition and structural condition that can be arrived at from multiple other conditions described for that landcover type. Thus our seral stages incorporate age, size, canopy cover, and vegetation composition. In general, the delineation of stages has originated from the LandFire biophysical setting model descriptive of a given landcover type; however, seral stages are not necessarily identical to the classes identified in those models.


\begin{figure}[hbp]
\centering
\includegraphics[width=0.8\textwidth]{/Users/mmallek/Documents/Thesis/statetransmodel/StateTransitionModel/7class.png}
\caption{State and Transition Model for Western White Pine Forest. Each dark grey box represents one of the seven seral stages for this landcover type. Each column of boxes represents a stage of development: early, middle, and late. Each row of boxes represents a different level of canopy cover: closed (70-100\%), moderate (40-70\%), and open (0-40\%). Transitions between states/seral stages may occur as a result of high mortality fire, low mortality fire, or succession. Specific pathways for each are denoted by the appropriate color line and arrow: red lines relate to high mortality fire, orange lines relate to low mortality fire, and green lines relate to natural succession.} 
\label{transmodel_wwp}
\end{figure}

\paragraph{Early Development (ED)}

\paragraph{Description} Open stand of \emph{P. monticola}, \emph{A. magnifica}, as well as other tree seedlings mixed with grasses and shrubs. Early seral dominant species include Ceanothus and various grasses. A portion of these stands get into a shrub dominated stage that can persist for for a few decades (LandFire 2007, Estes 2013).

\paragraph{Succession Transition} In the absence of disturbance, patches in this seral stage will begin transitioning to an mid development seral stage after 30 years at a rate of 0.7 per time step. At 70 years, all remaining patches will succeed. The secondary rate of succession to MDO is 0.8, and to MDC is 0.2. On average, patches remain in early development for 43 years.

\paragraph{Wildfire Transition} High mortality wildfire (17\% of fires in this seral stage) recycles the patch through the Early Development seral stage. No transition occurs as a result of low mortality fire.

\noindent\hrulefill


\paragraph{Mid Development - Open Canopy Cover (MDO)}

\paragraph{Description} Open stand of early seral tree species. Heterogeneous ground cover of grasses, forbs, and shrubs. Trees present are pole to medium sized conifers with canopy cover less than 40\%. Conifer species likely present include \emph{P. monticola}, \emph{A. magnifica}, and \emph{P. jeffreyi} (LandFire 2007, Estes 2013).

\paragraph{Succession Transition} Patches in this seral stage will maintain under low mortality disturbance, but after 15 years without fire they begin transitioning to MDM at a rate of 0.8 per time step. Succession to LDO occurs once the patch has been in mid development for 70 years. The rate of succession per time step is 0.4. At 120 years, all remaining patches succeed. On average, patches remain in early development for 83 years.

\paragraph{Wildfire Transition} High mortality wildfire (9\% of fires in this seral stage) recycles the patch through the ED seral stage. Low mortality wildfire (91\%) maintains the patch in MDO.

\noindent\hrulefill

\paragraph{Mid Development - Moderate Canopy Cover (MDM)}

\paragraph{Description} Sparse ground cover of grasses, forbs, and shrubs; moderate to dense cover of trees. Conifers are pole to medium-sized, with canopy cover ranging from 40-70\%. Conifer species likely present include \emph{P. monticola}, \emph{A. magnifica}, and \emph{P. jeffreyi} (LandFire 2007, Estes 2013).

\paragraph{Succession Transition} Patches in this seral stage will maintain under low mortality disturbance, but after 15 years without fire they begin transitioning to MDC at a rate of 0.8 per time step. Succession to LDM occurs once the patch has been in mid development for 70 years. The rate of succession per time step is 0.4. At 120 years, all remaining patches succeed. On average, patches remain in early development for 83 years.

\paragraph{Wildfire Transition} High mortality wildfire (12\% of fires in this seral stage) recycles the patch through the ED seral stage. Low mortality wildfire (88\%) opens the patch up to MDO 40\% of the time; otherwise, the patch remains in MDM.

\noindent\hrulefill

\paragraph{Mid Development - Closed Canopy Cover (MDC)}

\paragraph{Description} Sparse ground cover of grasses, forbs, and shrubs; moderate to dense cover of trees. Conifers are pole to medium-sized, with canopy cover over 70\%. Forests of this type rarely, if ever, exceed 80\% canopy closure even in closed, dense seral stages. Conifer species likely present include \emph{P. monticola}, \emph{A. magnifica}, and \emph{P. jeffreyi} (LandFire 2007, Estes 2013).

\paragraph{Succession Transition} Succession to LDC occurs once the patch has been in mid development for 70 years. The rate of succession per time step is 0.4. At 120 years, all remaining patches succeed. On average, patches remain in early development for 83 years.

\paragraph{Wildfire Transition} High mortality wildfire (18\% of fires in this seral stage) recycles the patch through the Early Development seral stage. Low mortality wildfire (82\%) opens the patch up to MDM 80\% of the time; otherwise, the patch remains in MDC.

\noindent\hrulefill


\paragraph{Late Development - Open Canopy Cover (LDO)}

\paragraph{Description} Open stands of large trees, primarily \emph{P. monticola}, \emph{A. magnifica}, and \emph{P. jeffreyi}. Canopy cover is less than 40\% (LandFire 2007, Estes 2013).

\paragraph{Succession Transition} Patches in this seral stage will maintain under low mortality disturbance, but after 15 years without fire, these patches succeed to LDM at a rate of 0.8 per timestep.

\paragraph{Wildfire Transition} High mortality wildfire (9\% of fires in this seral stage) recycles the patch through the Early Development seral stage. Low mortality wildfire (91\%) maintains the patch in LDO.

\noindent\hrulefill

\paragraph{Late Development - Moderate Canopy Cover (LDM)}

\paragraph{Description} Closed stands of large trees, primarily \emph{P. monticola}, \emph{A. magnifica}, and \emph{P. jeffreyi}. Forests in this landcover type rarely exceed 80\% canopy closure even in closed, dense seral stages. Canopy cover exceeds 70\% (LandFire 2007, Estes 2013).

\paragraph{Succession Transition} Patches in this seral stage will maintain under low mortality disturbance, but after 15 years without fire, these patches succeed to LDC at a rate of 0.8 per timestep.

\paragraph{Wildfire Transition} High mortality wildfire (12\% of fires in this seral stage) recycles the patch through the ED seral stage. Low mortality wildfire (82\%) opens the patch up to LDO 40\% of the time; otherwise, the patch remains in LDC.

\noindent\hrulefill

\paragraph{Late Development - Closed Canopy Cover (LDC)}

\paragraph{Description} Closed stands of large trees, primarily \emph{P. monticola}, \emph{A. magnifica}, and \emph{P. jeffreyi}. Forests in this landcover type rarely exceed 80\% canopy closure even in closed, dense conditions. Canopy cover exceeds 40\% (LandFire 2007, Estes 2013).

\paragraph{Succession Transition} Patches in this seral stage will maintain in the absence of disturbance.

\paragraph{Wildfire Transition} High mortality wildfire (17\% of fires in this seral stage) recycles the patch through the ED seral stage. Low mortality wildfire (83\%) opens the patch up to LDM 80\% of the time; otherwise, the patch remains in LDC.

\noindent\hrulefill




\subsection*{Seral Stage Classification}
\begin{table}[hbp]
\footnotesize
\centering
\caption{Classification of seral stage for WWP. Diameter at Breast Height (DBH) and Cover From Above (CFA) values taken from EVeg polygons. DBH categories are: null, 0-0.9'', 1-4.9'', 5-9.9'', 10-19.9'', 20-29.9'', 30''+. CFA categories are null, 0-10\%, 10-20\%, \dots , 90-100\%. Each row in the table below should be read with a boolean AND across each column.}
\label{wwp_classification}
\begin{tabular}{@{}lrrrrr@{}}
\toprule
\textbf{\begin{tabular}[l]{@{}l@{}}Cover \\ Condition\end{tabular}} & \textbf{\begin{tabular}[r]{@{}r@{}}Overstory Tree \\ Diameter 1 \\ (DBH)\end{tabular}} & \textbf{\begin{tabular}[r]{@{}r@{}}Overstory Tree \\ Diameter 2 \\ (DBH)\end{tabular}} & \textbf{\begin{tabular}[r]{@{}r@{}}Total Tree\\ CFA (\%)\end{tabular}} & \textbf{\begin{tabular}[r]{@{}r@{}}Conifer \\ CFA (\%)\end{tabular}} & \textbf{\begin{tabular}[r]{@{}r@{}}Hardwood \\ CFA (\%)\end{tabular}} \\ \midrule
Early All        & 0-4.9''         & any & any    & any    & any \\
Mid Open         & 5-19.9''        & any & 0-40   & any    & any \\
Mid Moderate     & 5-19.9''        & any & 40-70  & any    & any \\
Mid Closed       & 5-19.9''        & any & null   & 70-100 & any \\
Mid Closed       & 5-19.9''        & any & 70-100 & any    & any \\
Late Open        & 20''+           & any & 0-40   & any    & any \\
Late Moderate    & 20''+           & any & 40-70  & any    & any \\
Late Closed      & 20''+           & any & 70-100 & any    & any \\ \bottomrule
\end{tabular}
\end{table}



\clearpage

\subsection*{References}

\begin{hangparas}{.25in}{1} 
\interlinepenalty=10000
Estes, Becky. Central Sierra Province Ecologist, USDA Forest Service. Personal communication, 3 September 2013.

Griffith, Randy Scott. 1993. ``Pinus monticola.'' In: \emph{Fire Effects Information System}, [Online].  U.S. Department of Agriculture, Forest Service,  Rocky Mountain Research Station, Fire Sciences Laboratory (Producer).  \burl{http://www.fs.fed.us/database/feis/}. [Accessed 4 December 2012].

Fites-Kaufman, Jo Ann, Phil Rundel, Nathan Stephenson, and Dave A. Wixelman. ``Montane and Subalpine Vegetation of the Sierra Nevada and Cascade Ranges.'' In \emph{Terrestrial Vegetation of California, 3rd Edition}, edited by Michael Barbour, Todd Keeler-Wolf, and Allan A. Schoenherr, 456-501. Berkeley and Los Angeles: University of California Press, 2007. 

LandFire. ``Biophysical Setting Models.'' Biophysical Setting 0711720: Sierran-Intermontane Desert Western White Pine-White Fir Woodland. 2007. LANDFIRE Project, U.S. Department of Agriculture, Forest Service; U.S. Department of the Interior. \burl{http://www.landfire.gov/national_veg_models_op2.php}. Accessed 30 November 2012.

Skinner, Carl N. and Chi-Ru Chang. ``Fire Regimes, Past and Present.'' \emph{Sierra Nevada Ecosystem Project: Final report to Congress, vol. II, Assessments and scientific basis for management options}. Davis: University of California, Centers for Water and Wildland Resources, 1996.

Van de Water, Kip M. and Hugh D. Safford. ``A Summary of Fire Frequency Estimates for California Vegetation Before Euro-American Settlement.'' \emph{Fire Ecology} 7.3 (2011): 26-57. doi: 10.4996/fireecology.0703026.

\end{hangparas}


%% !TEX root = master.tex
\newpage
\section{Yellow Pine (YPN)}
\label{ypn-description}

\subsection*{General Information}

\subsubsection*{Cover Type Overview}

\textbf{Yellow Pine (YPN)}
\newline
\textbf{Crosswalks}
\begin{itemize}
	\item EVeg: Regional Dominance Type 1
	\begin{itemize}
		\item Eastside Pine
		\item Jeffrey Pine
		\item Ponderosa Pine
	\end{itemize}

	\item LandFire BpS Model
	\begin{itemize}
		\item Yellow Pine
	\end{itemize}

	\item Presettlement Fire Regime Type
	\begin{itemize}
		\item 0610310 California Montane Jeffrey Pine (-Ponderosa Pine) Woodland
	\end{itemize}

	\item Only occurs on the east side of the Sierra crest.
\end{itemize}

\noindent \textbf{Yellow Pine with Aspen (YPN-ASP)}
\newline
This type is created by overlaying the NRIS TERRA Inventory of Aspen on top of the EVeg layer. Where it intersects with YPN it is assigned to YPN-ASP.

\noindent Reviewed by Hugh Safford, Regional Ecologist, USDA Forest Service

\subsubsection*{Vegetation Description}
\textbf{Yellow Pine (YPN)}	This landcover type is characterized by yellow pine species such as \emph{Pinus ponderosa} or \emph{Pinus jeffreyi} that occur on the east side of the Sierra crest (LandFire 2007a). Relatively pure stands of yellow pine may occur, or they may mix with other tree species including \emph{Abies concolor, Juniperus occidentalis, Pinus contorta} ssp. \emph{murrayana}, and \emph{Quercus kelloggi} (Fites-Kaufman et al. 2007, Fitzhugh 1988). Their understory may include both montane forest and Great Basin shrubs, including but not limited to \emph{Ceanothus, Arctostaphylos, Symphoricarpos, Artemisia tridentata, Purshia tridentata, Ericameria nauseosa, Cercocarpus}, and \emph{Holodiscus}. Herbaceous plants and grasses may include \emph{Wyethia, Balsamorhiza sagittata, Festuca, Calamagrostis}, and \emph{Elymus} (LandFire 2007a, Fitzhugh 1988).

Without disturbance, except for naturally occurring fire, a mosaic of uneven-aged patches develops, with open spaces and dense sapling stands (Safford 2013). \emph{Q. kelloggi} or \emph{Juniperus occidentalis} may form an understory, but pure stands of pine also are found. An open stand of low shrubs, and a grassy herb layer are typical. Crowns of pines are open, allowing light, wind and rain to penetrate, whereas other associated trees provide more dense foliage (Fitzhugh 1988).

\textbf{Yellow Pine with Aspen (YPN-ASP)} These are upland forests and woodlands dominated by \emph{Populus tremuloides} without a significant conifer component, often termed ``stable aspen.'' The understory structure may be complex with multiple shrub and herbaceous layers, or simple with just an herbaceous layer. The herbaceous layer may be dense or sparse, dominated by graminoids or forbs. Common shrubs include \emph{Acer, Amelanchier, Artemisia, Juniperus, Prunus, Rosa, Shepherdia, Symphoricarpos}, and the dwarf-shrubs \emph{Mahonia} and \emph{Vaccinium}. Common graminoids may include \emph{Bromus, Calamagrostis, Carex, Elymus, Festuca}, and \emph{Hesperostipa}. Associated forbs may include \emph{Achillea, Eucephalus, Delphinium, Geranium, Heracleum, Ligusticum, Lupinus, Osmorhiza, Pteridium, Rudbeckia, Thalictrum, Valeriana, Wyethia}, and many others (LandFire 2007b).


\subsubsection*{Distribution}
\textbf{Yellow Pine}	This landcover type occurs on all aspects from about 1200 m to 1980 m (4000-6500 ft) in elevation, east of the Sierra Nevada crest (Fitzhugh 1988). It is usually found on volcanic and granitic substrates, in shallow soils with a frigid soil temperature regime (LandFire 2007a).

\textbf{Yellow Pine with Aspen}	Sites supporting \emph{P. tremuloides} are associated with added soil moisture, i.e., azonal wet sites. These sites are often close to streams, lakes, and meadows. Other sites include rock reservoirs, springs and seeps. Terrain can be simple to complex. At lower elevations, topographic conditions for this type tends toward positions resulting in relatively colder, wetter conditions within the prevailing climate, e.g., ravines, north slopes, wet depressions, etc. (LandFire 2007b). \emph{P. tremuloides} stands may also be associated with lateral or terminal moraine boulder material, talus-colluvium, rock falls, or lava flows. In addition, pure stands may be found in topographic positions where snow accumulates, mostly at higher north facing elevations, where snow presence means the growing season is too short to support conifers (Shepperd et al. 2006). 


\subsection*{Disturbances}

\subsubsection*{Wildfire}
\textbf{Yellow Pine} Wildfires are common and frequent; mortality depends on vegetation vulnerability and wildfire intensity. Low mortality fires kill small trees and consume above-ground portions of shrubs and herbs, but do not kill large trees or below-ground organs of most shrubs and herbs which promptly re-sprout. High mortality fires kill large as well as small trees, and may kill many of the shrubs and herbs as well. Fire kills the above-ground portions of the shrubs and herbs, but most shrubs and herbs resprout from surviving below-ground organs. Wildfires may trigger transitions between developmental seral stages.

The relatively long needles of yellow pines and relatively open structure of theses stands make for dry surface and ground fuels that burn readily. Thus, fires in these stands burn more frequently than those in adjacent forests (Fites-Kaufman et al. 2007). In fact, fire is an integral part of the ecology of yellow pines. Fire has allowed yellow pines to dominate sites where it is the potential climax as well as sites where it would otherwise be seral to more shade-tolerant tree species. \emph{P. ponderosa} and \emph{P. jeffreyi} have evolved with a thick bark and open crown structure that allows them to survive most fires. Mature trees will self-prune, leaving a smooth bole which reduces aerial fire spread. Also, fire creates favorable seedbeds for seedling establishment (Habeck 1992). 

\medskip
\noindent \textbf{Yellow Pine with Aspen}	Sites supporting \emph{P. tremuloides} are maintained by stand-replacing disturbances that allow regeneration from below-ground suckers. Replacement fire and ground fire are thought to have been common in stable \emph{P. tremuloides} stands historically. Because \emph{P. tremuloides} is associated with mesic conditions, it rarely burns during the normal lightning season. However, during years with little precipitation stands may be more susceptible to burning. Evidence from fire scars and historical studies show that past fires occurred mostly during the spring and fall. These are typically self-perpetuating stands (LandFire 2007b)

Van de Water and Safford (2011) found a mean fire return interval of 19 years, median of 20 years, mean min interval of 10 years and mean max of 90 years for Aspen. The LandFire model for northern Sierra Nevada ``stable aspen'' predicts a mean FRI of 31 years. Replacement FRI has a mean of 68 years with a range of 50-300 years, while mixed severity FRI has a mean of 57 years with a range of 20-60 years, and low severity fire is not modeled (LandFire 2007b). We recalculated these numbers using seral stage-specific information and using only high and low mortality fire categories, which resulted in an interval of 38 years for high mortality fire, 111 years for low mortality fire, and 29 years for any fire.

Estimates of fire rotations for these variants are available from the LandFire project and a few review papers. The LandFire project’s published fire return intervals are based on a series of associated models created using the Vegetation Dynamics Development Tool (VDDT). In VDDT, fires are specified concurrently with the transition that follows them. For example, a replacement fire causes a transition to the early development stage. In the RMLands model, such fires are classified as high mortality. However, in VDDT mixed severity fires may cause a transition to early development, a transition to a more open seral stage, or no transition at all. In this case, we categorize the first example as a high mortality fire, and the second and third examples as a low mortality fire. Based on this approach, we calculated fire rotations and the probability of high mortality fire for each of the YPN and YPN-ASP seral stages (Tables~\ref{tab:ypndesc_fire} and \ref{tab:ypn-aspdesc_fire}). We computed overall target fire rotations based on values from Mallek et al. (2013) and Van de Water and Safford (2011). 




\begin{table}[!htbp]
\footnotesize
\centering
\caption{Fire rotation index values and probability of high severity fire (at least 75\% overstory tree mortality) probabilities for Yellow Pine. The seral stage that is most susceptible to fire (i.e., has the lowest predicted fire rotation) has a fire rotation index value of 1. Higher values correspond with lower susceptibility to wildfire. The values are relative only within an individual seral stage and should not be compared against other land cover types. Values were derived from VDDT model 0610581 (LandFire 2007), Mallek et al. (2013), and Safford (personal communication). }
\label{tab:ypndesc_fire}
\begin{tabular}{@{}lcc@{}}
\toprule
 \textbf{Seral Stage}    & \textbf{\begin{tabular}[c]{@{}c@{}}Fire Rotation \\ Index\end{tabular}} & \textbf{\begin{tabular}[c]{@{}c@{}}Probability of \\ High Severity Fire\end{tabular}} \\ \hline
Early (All)     		 & 3.8            & 1           \\
Mid--Closed    			 & 1.4            & 0.26        \\
Mid--Moderate  			 & 1.2            & 0.14        \\
Mid--Open      			 & 1.0             & 0.05        \\
Late--Closed   			 & 1.9            & 0.20        \\
Late--Moderate 			 & 1.3             & 0.08        \\
Late--Open     			 & 1.0             & 0.01     	\\ 
\emph{Target Fire Rotation}    			& \emph{21 years}  &   \\ 
\bottomrule
\end{tabular}
\end{table}

\begin{table}[!htbp]
\footnotesize
\centering
\caption{Fire rotation index values and probability of high severity fire (at least 75\% overstory tree mortality) probabilities for Yellow Pine - Aspen type. The seral stage that is most susceptible to fire (i.e., has the lowest predicted fire rotation) has a fire rotation index value of 1. Higher values correspond with lower susceptibility to wildfire. The values are relative only within an individual seral stage and should not be compared against other land cover types. Values were derived from VDDT model 0610110 (LandFire 2007) and Safford (personal communication).}
\label{tab:ypn-aspdesc_fire}
\begin{tabular}{@{}lcc@{}}
\toprule
 \textbf{Seral Stage}    & \textbf{\begin{tabular}[c]{@{}c@{}}Fire Rotation \\ Index\end{tabular}} & \textbf{\begin{tabular}[c]{@{}c@{}}Probability of \\ High Severity Fire\end{tabular}} \\ \hline
Early--Aspen        & 2.9            & 1           \\
Mid--Aspen          & 1.1            & 0.26        \\
Late--Conifer-Aspen & 1.0            & 0.08  		 \\ 
\emph{Target Fire Rotation}    			& \emph{21 years}  &   \\ 
\bottomrule
\end{tabular}
\end{table}

\subsubsection*{Other Disturbance}
Other disturbances are not currently modeled, but may, depending on the seral stage affected and mortality levels, reset patches to early development, maintain existing seral stages, or shift/accelerate succession to a more open seral stage. 

\subsection*{Vegetation Seral Stages}
We recognize seven separate seral stages for YPN: Early Development (ED), Mid Development - Open Canopy Cover (MDO), Mid Development - Moderate Canopy Cover, Mid Development - Closed Canopy Cover (MDC), Late Development - Open Canopy Cover (LDO), Late Development - Moderate Canopy Cover (LDM), and Late Development - Closed Canopy Cover (LDC) (Figure~\ref{transmodel_ypn}).  The YPN -ASP variant is assigned to three seral stages: Early Development - Aspen (ED-A), Mid Development - Aspen (MD-A), and Late Development - Conifer with Aspen (LD-CA) (Figure~\ref{transmodel_ypn-asp}).

Our seral stages are an alternative to ``successional'' classes that imply a linear progression of states and tend not to incorporate disturbance. The seral stages identified here are derived from a combination of successional processes and anthropogenic and natural disturbance, and are intended to represent a composition and structural condition that can be arrived at from multiple other conditions described for that landcover type. Thus our seral stages incorporate age, size, canopy cover, and vegetation composition. In general, the delineation of stages has originated from the LandFire biophysical setting model descriptive of a given landcover type; however, seral stages are not necessarily identical to the classes identified in those models.

\begin{figure}[htbp]
\centering
\includegraphics[width=0.8\textwidth]{/Users/mmallek/Documents/Thesis/statetransmodel/StateTransitionModel/7class.png}
\caption{State and Transition Model for Yellow Pine Forest and Woodland (not inclusive of the aspen variant). Each dark grey box represents one of the seven seral stages for this landcover type. Each column of boxes represents a stage of development: early, middle, and late. Each row of boxes represents a different level of canopy cover: closed (70-100\%), moderate (40-70\%), and open (0-40\%). Transitions between states/seral stages may occur as a result of high mortality fire, low mortality fire, or succession. Specific pathways for each are denoted by the appropriate color line and arrow: red lines relate to high mortality fire, orange lines relate to low mortality fire, and green lines relate to natural succession.} 
\label{transmodel_ypn}
\end{figure}

\begin{figure}[htbp]
\centering
\includegraphics[width=0.8\textwidth]{/Users/mmallek/Documents/Thesis/statetransmodel/StateTransitionModel/3class-asp.png}
\caption{State and Transition Model for Yellow Pine Forest and Woodland, Aspen variant. Each dark grey box represents one of the three seral stages for this landcover type. Three seral stages of development are represented: early, middle, and late. Transitions between states/seral stages may occur as a result of high mortality fire, low mortality fire, or succession. Specific pathways for each are denoted by the appropriate color line and arrow: red lines relate to high mortality fire, orange lines relate to low mortality fire, and green lines relate to natural succession.} 
\label{transmodel_ypn-asp}
\end{figure}

\paragraph*{Early Development (ED)}

\paragraph*{Description} Grasses, forbs, low shrubs, and sparse to moderate cover of trees (primarily \emph{P. ponderosa} or \emph{P. jeffreyi}) seedlings/saplings with an open canopy. This seral stage is characterized by the recruitment of a new cohort of early successional, shade-intolerant tree species into an open area created by a stand-replacing disturbance. Following such disturbance, some sites are dominated by dense shrub stands composed of \emph{P. tridentata}, \emph{Arctostaphylos}, and/or \emph{Ceanothus}, depending on location. Other postfire sites are more open and dominated by dense pine seedlings, bunchgrasses and forbs.

\paragraph*{Succession Transition} In the absence of disturbance, patches in this seral stage will begin transitioning to MDC or MDO after 40 years at a rate of 0.7 per timestep. The transition to MDO is twice as likely as transition to MDC.  At 80 years, all remaining patches will succeed to either MDC or MDO. 

\paragraph*{Wildfire Transition} High mortality wildfire (100\% of fires in this seral stage) recycles the patch through the Early Development seral stage. Low mortality wildfire is not modeled for this seral stage. 

\noindent\hrulefill


\paragraph*{Mid Development - Open Canopy Cover (MDO)}

\paragraph*{Description} Open mid-development forest with diverse herbaceous understory and scattered woody shrubs. Conifers, primarily \emph{P. ponderosa} or \emph{P. jeffreyi}, are medium sized. Herbs and other species gradually decline as growing trees begin to shade understory. Maintained by frequent burning. Canopy cover is less than 40\% (LandFire 2007a). 

\paragraph*{Succession Transition} Patches in this seral stage will maintain under low mortality disturbance, but after 20 years without fire they begin transitioning to MDM at a rate of 0.8 per time step. Succession to LDO occurs once the patch has been in mid development for 170 years. The rate of succession per time step is 0.4. After 230 years, all patches will have succeeded.

\paragraph*{Wildfire Transition} High mortality wildfire (5\% of fires in this seral stage) recycles the patch through the Early Development seral stage. Low mortality wildfire (95\%) maintains the patch in MDO.

\noindent\hrulefill

\paragraph*{Mid Development - Moderate Canopy Cover (MDM)}

\paragraph*{Description} Mid-development forest with moderate canopy cover. Somewhat ``overstocked'' pole to large pole size trees, primarily \emph{P. ponderosa} or \emph{P. jeffreyi}, susceptible to stagnation. Marginal understory associated with limited site resources. Develops where fire frequency is too low to thin small trees. Canopy cover is 40-70\% (LandFire 2007a).

\paragraph*{Succession Transition} Patches in this seral stage may maintain under low mortality disturbance, but after 20 years without fire they begin transitioning to MDC at a rate of 0.8 per time step. At 130 years since succession to a mid development seral stage, these patches will begin transitioning to LDC. The rate of succession per time step is 0.3. After 230 years, all patches will have succeeded.

\paragraph*{Wildfire Transition} High mortality wildfire (14\% of fires in this seral stage) recycles the patch through the Early Development seral stage. Low mortality wildfire (86\%) opens the stand up to MDO 32\% of the time; otherwise, the patch remains in MDC.

\noindent\hrulefill

\paragraph*{Mid Development - Closed Canopy Cover (MDC)}

\paragraph*{Description} Dense mid-development forest. ``Overstocked'' pole to large pole size trees, primarily \emph{P. ponderosa} or \emph{P. jeffreyi}, susceptible to stagnation. Marginal understory associated with limited site resources. Develops where fire frequency is too low to thin small trees. Canopy cover is over 70\% (LandFire 2007a).

\paragraph*{Succession Transition} At 100 years since succession to a mid development seral stage, these patches will begin transitioning to LDC. The rate of succession per time step is 0.2. After 200 years, all patches will have succeeded.

\paragraph*{Wildfire Transition} High mortality wildfire (26\% of fires in this seral stage) recycles the patch through the Early Development seral stage. Low mortality wildfire (74\%) opens the stand up to MDM 60\% of the time; otherwise, the patch remains in MDC.

\noindent\hrulefill


\paragraph*{Late Development - Open Canopy Cover (LDO)}

\paragraph*{Description} Open late-development forest with large and very large trees, primarily \emph{P. ponderosa} or \emph{P. jeffreyi}. Trees grow in often widely spaced clumps and the understory is open and often diverse. Surface fuels are limited due to frequent burning. Canopy cover is less than 40\% (LandFire 2007a, Safford 2013).

\paragraph*{Succession Transition} Patches in this seral stage will maintain under low mortality disturbance, but after 25 years without fire, these patches succeed to LDM at a rate of 0.7 per timestep.

\paragraph*{Wildfire Transition} High mortality wildfire (1\% of fires in this seral stage) recycles the patch through the Early Development seral stage. Low mortality wildfire (99\%) maintains the patch in LDO.

\noindent\hrulefill

\paragraph*{Late Development - Moderate Canopy Cover (LDM)}

\paragraph*{Description} Open late-development forest with large and very large trees, primarily \emph{P. ponderosa} or \emph{P. jeffreyi}. Trees grow in often widely spaced clumps, although they are becoming more dense, and the understory is fairly open and often diverse. Surface fuels are accumulating. Canopy cover is 40-70\% (LandFire 2007a, Safford 2013).

\paragraph*{Succession Transition} Patches in this seral stage may maintain under low mortality disturbance, but after 25 years without fire, these patches succeed to LDC at a rate of 0.7 per timestep.

\paragraph*{Wildfire Transition} High mortality wildfire (8\% of fires in this seral stage) recycles the patch through the Early Development seral stage. Low mortality wildfire (92\%) opens the stand up to LDO 18\% of the time; otherwise, the patch remains in LDM.

\noindent\hrulefill

\paragraph*{Late Development - Closed Canopy Cover (LDC)}

\paragraph*{Description} Dense late-development forest, primarily \emph{P. ponderosa} or \emph{P. jeffreyi} with large and very large trees, sometimes with significant within-stand mortality. Substantial surface fuel accumulation and ladder fuels. Canopy cover exceeds 70\% (LandFire 2007a).

\paragraph*{Succession Transition} Patches in this seral stage will maintain in the absence of disturbance.

\paragraph*{Wildfire Transition} High mortality wildfire (20\% of fires in this seral stage) recycles the patch through the Early Development seral stage. Low mortality wildfire (80\%) opens the stand up to LDM 58\% of the time; otherwise, the patch remains in LDC.

\noindent\hrulefill
\noindent\hrulefill

\subsubsection*{Aspen Variant}

\paragraph*{Early Development - Aspen (ED-A)}

\paragraph*{Description} Grasses, forbs, low shrubs, and sparse to moderate cover of tree seedlings/saplings (primarily \emph{P. tremuloides}) with an open canopy. This seral stage is characterized by the recruitment of a new cohort of early successional, shade-intolerant tree species into an open area created by a stand-replacing disturbance. 

Following disturbance, succession proceeds rapidly from an herbaceous layer to shrubs and trees, which invade together (Verner 1988). \emph{P. tremuloides} suckers over 6ft tall develop within about 10 years (LandFire 2007b). 

\paragraph*{Succession Transition} Unless it burns, a patch in the early seral stage persists for 10 years, at which point it transitions to MD-A.

\paragraph*{Wildfire Transition} High mortality wildfire (100\% of fires in this seral stage) recycles the patch through the ED-A seral stage. Low mortality wildfire is not modeled for this seral stage.

\noindent\hrulefill


\paragraph*{Mid Development - Aspen (MD-A)}

\paragraph*{Description} \emph{P. tremuloides} trees 5-16'' DBH. Canopy cover is highly variable, and can range from 40-100\%. These patches range in age from 10 to 110 years. (LandFire 2007b).

\paragraph*{Succession Transition} Patches in the MD-A seral stage persist for at least 80 years in the absence of any fire, after which they begin transitioning to LD-CA at a rate of 0.6 per timestep. After 130 years without fire all remaining MD-A patches transition to LD-CA. 

\paragraph*{Wildfire Transition} High mortality wildfire (26\% of fires in this seral stage) recycles the patch through the ED-A seral stage. No transition occurs as a result of low mortality fire (74\%).

\noindent\hrulefill


\paragraph*{Late Development - Aspen with Conifer (LD-AC)}

\paragraph*{Description} If stands are sufficiently protected from fire such that conifer species overtop \emph{P. tremuloides} and become large, they may be able to withstand some fire that more sensitive \emph{P. tremuloides} cannot. When this occurs, it creates a patch characterized by late development conifers, such as \emph{P. contorta} ssp. \emph{murrayana}, and early seral \emph{P. tremuloides}. 

\paragraph*{Succession Transition} LD-CA persists for 70 years in the absence of any fire, at which point patches transition to LDC. 

\paragraph*{Wildfire Transition} High mortality wildfire (13\% of fires in this seral stage) returns the patch to ED-A. Low mortality wildfire (87\%) maintains the stand in LD-CA. 

\noindent\hrulefill




\subsection*{Seral Stage Classification}
\begin{table}[hbp]
\footnotesize
\centering
\caption{Classification of seral stage for YPN. Diameter at Breast Height (DBH) and Cover From Above (CFA) values taken from EVeg polygons. DBH categories are: null, 0-0.9'', 1-4.9'', 5-9.9'', 10-19.9'', 20-29.9'', 30''+. CFA categories are null, 0-10\%, 10-20\%, \dots , 90-100\%. Each row in the table below should be read with a boolean AND across each column.}
\label{ypn_classification}
\begin{tabular}{@{}lrrrrr@{}}
\toprule
\textbf{\begin{tabular}[l]{@{}l@{}}Cover \\ Condition\end{tabular}} & \textbf{\begin{tabular}[r]{@{}r@{}}Overstory Tree \\ Diameter 1 \\ (DBH)\end{tabular}} & \textbf{\begin{tabular}[r]{@{}r@{}}Overstory Tree \\ Diameter 2 \\ (DBH)\end{tabular}} & \textbf{\begin{tabular}[r]{@{}r@{}}Total Tree\\ CFA (\%)\end{tabular}} & \textbf{\begin{tabular}[r]{@{}r@{}}Conifer \\ CFA (\%)\end{tabular}} & \textbf{\begin{tabular}[r]{@{}r@{}}Hardwood \\ CFA (\%)\end{tabular}} \\ \midrule
Early All        & 0-4.9''         & any & any    & any & any \\
Mid Open         & 5-19.9''        & any & 0-40   & any & any \\
Mid Moderate     & 5-19.9''        & any & 40-70  & any & any \\
Mid Closed       & 5-19.9''        & any & 70-100 & any & any \\
Late Open        & 20-40''+        & any & 0-40   & any & any \\
Late Moderate    & 20-40''+        & any & 40-70  & any & any \\
Late Closed      & 20-40''+        & any & 70-100 & any & any \\ \bottomrule
\end{tabular}
\end{table}

YPN-ASP seral stages were assigned manually using NAIP 2010 Color IR imagery to assess seral stage.



\clearpage

\subsection*{References}

\begin{hangparas}{.25in}{1} 
\interlinepenalty=10000

Fitzhugh, E. Lee. ``Eastside Pine (EPN).'' \emph{A Guide to Wildlife Habitats of California}, edited by Kenneth E. Mayer and William F. Laudenslayer. California Deparment of Fish and Game, 1988. \burl{http://www.dfg.ca.gov/biogeodata/cwhr/pdfs/EPN.pdf}. Accessed 4 December 2012.

Habeck, R. J. ``Pinus ponderosa var. ponderosa.'' \emph{Fire Effects Information System}, U.S. Department of Agriculture, Forest Service,  Rocky Mountain Research Station, Fire Sciences Laboratory, 1992. \burl{http://www.fs.fed.us/database/feis/plants/tree/quekel/all.html}. Accessed 21 December 2012.

LandFire. ``Biophysical Setting Models.'' Biophysical Setting 0610310: California Montane Jeffrey Pine (-Ponderosa Pine) Woodland. 2007a. LANDFIRE Project, U.S. Department of Agriculture, Forest Service; U.S. Department of the Interior. \burl{http://www.landfire.gov/national_veg_models_op2.php}. Accessed 9 November 2012.

LandFire. ``Biophysical Setting Models.'' Biophysical Setting 0610110: Rocky Mountain Aspen Forest and Woodland. 2007b. LANDFIRE Project, U.S. Department of Agriculture, Forest Service; U.S. Department of the Interior. \burl{http://www.landfire.gov/national_veg_models_op2.php}. Accessed 7 January 2013.

Safford, Hugh S. Personal communications, 5 May 2013, 26 July 2013, 15 August 2013.

Shepperd, Wayn De, Paul C. Rogers, David Burton, and Dale L. Bartos. ``Ecology, Biodiversity, Management, and Restoration of Aspen in the Sierra Nevada.'' General Technical Report RMRS-GTR-178. Rocky Mountain Research Station, Forest Service, U.S. Department of Agriculture, 2006.

Skinner, Carl N. and Chi-Ru Chang. ``Fire Regimes, Past and Present.'' \emph{Sierra Nevada Ecosystem Project: Final report to Congress, vol. II, Assessments and scientific basis for management options}. Davis: University of California, Centers for Water and Wildland Resources, 1996.

Van de Water, Kip M. and Hugh D. Safford. ``A Summary of Fire Frequency Estimates for California Vegetation Before Euro-American Settlement.'' \emph{Fire Ecology} 7.3 (2011): 26-57. doi: 10.4996/fireecology.0703026.

Verner, Jared. ``Aspen (ASP).'' \emph{A Guide to Wildlife Habitats of California}, edited by Kenneth E. Mayer and William F. Laudenslayer. California Deparment of Fish and Game, 1988. \burl{http://www.dfg.ca.gov/biogeodata/cwhr/pdfs/ASP.pdf}. Accessed 4 December 2012.

\end{hangparas}




%% !TEX root = master.tex
\chapter{Expanded HRV Results and Analysis}
%\chapter{Disturbance Regime and Seral Stage Dynamics Expanded Results}
\label{app:full-results}

%%%%%%%%%%%%%%%%%%%%%%%%%%%%%%%%%%%%%%%%%%%%%%%%%%%%%%%%%%%%%%%%%%%%%%%%%%%%%%%%
\section{Average Canopy Cover and Topographic Position}
Table~\ref{tab:tpi_cc} shows the relationship between canopy cover and topographic position (as measured by the topographic position index (TPI)) for the 9 most extensive cover types in the study area. Figure~\ref{fig:tpi_cc} shows a linear regression fit to a sample of points from each cover type across the TPI and average canopy cover grids.

% redone with results from 2015-09-04. File saved in /Users/mmallek/Tahoe/RMLands/results/results20150904/tpi

% removing 2016-03-13 because I moved this table, minus ocfw-u, to the methods section of ch2, making it redundant here

%\begin{table}[!htbp]
%\footnotesize
%\caption{For each cover type on the landscape, the percent change in canopy cover from the minimum TPI value for that cover type to the maximum TPI value. For seral stage abbreviations, see Table \ref{condtable}.}
%\label{tab:tpi_cc}
%\rotatebox{90}{
%\begin{tabular}{@{}lrrrrr@{}}
%\toprule 
% \textbf{\begin{tabular}[c]{@{}l@{}}Cover \\ Name\end{tabular}} & \small \textbf{\begin{tabular}[c]{@{}l@{}}Minimum \\ TPI\end{tabular}} & \small \textbf{\begin{tabular}[c]{@{}l@{}}Maximum \\ TPI\end{tabular}} & \small \textbf{\begin{tabular}[c]{@{}l@{}}Average Canopy \\Cover at \\ Minimum TPI\end{tabular}} & \small \textbf{\begin{tabular}[c]{@{}l@{}}Average Canopy \\ Cover at \\ Maximum TPI\end{tabular}}  & \small \textbf{\begin{tabular}[c]{@{}l@{}}Percent \\ Change in \\ Canopy \\ Cover\end{tabular}} \\ \midrule
%\textsc{meg\_m   }    & -300                 & 300   & 73.7       & 67.0     & -9.0      \\
%\textsc{meg\_x   }    & -299                 & 300   & 72.7       & 68.5     & -5.7      \\
%\textsc{ocfw     }    & -300                 & 300   & 50.0       & 45.6     & -8.7       \\
%\textsc{ocfw\_u  }    & -300                 & 300   & 49.0       & 34.9     & -28.8       \\
%\textsc{rfr\_m   }    & -300                 & 300   & 72.1       & 64.0     & -11.2     \\
%\textsc{rfr\_x   }    & -259                 & 300   & 40.2       & 29.1     & -27.6     \\
%\textsc{smc\_m   }    & -300                 & 300   & 55.5       & 50.4     & -9.3       \\
%\textsc{smc\_u   }    & -300                 & 300   & 39.9       & 28.9     & -27.7     \\
%\textsc{smc\_x   }    & -300                 & 300   & 27.6       & 21.9     & -20.5     \\ \bottomrule
%\end{tabular}
%}
%\end{table}

%redone 2015-09-14
\begin{figure}[!htbp]
\centering
\includegraphics[width=\textwidth]{/Users/mmallek/Documents/Thesis/Plots/tpi/hrv-facet-2.png}
\caption{Average canopy cover for the nine focal cover types during the simulated. Each blue point represents one pixel of an individual cover type on the landscape grid. The black line is the result of a linear regression fit to the data. Table \ref{tab:tpi_cc} provides the numerical representation of the shift from minimum to maximum TPI values for each cover type. (a) Mixed Evergreen - Mesic; (b) Mixed Evergreen - Xeric; (c) Oak-Conifer Forest and Woodland; (d) Oak-Conifer Forest and Woodland - Ultramafic; (e) Red Fir - Mesic; (f) Red Fir - Xeric; (g) Sierran Mixed Conifer - Mesic; (h) Sierran Mixed Conifer - Ultramafic; (i) Sierran Mixed Conifer - Xeric.} 
\label{fig:tpi_cc}
\end{figure}

\clearpage

%%%%%%%%%%%%%%%%%%%%%%%%%%%%%%%%%%%%%%%%%%%%%%%%%%%%%%%%%%%%%%%%%%%%%%%%%%%%%%%%%%%%%%%%%%%%%%%%
%%%%%%%%%%%%%%%%%%%%%%%%%%%%%%%%%%%%%%%%%%%%%%%%%%%%%%%%%%%%%%%%%%%%%%%%%%%%%%%%%%%%%%%%%%%%%%%%

%\section{Disturbance Regime}

% Took this section out 2016-02-06 because this image is in the main report. Probably this appendix shouldn't contain repeated results of the full landscape.

%redone 9/13
%\begin{figure}[!htbp]
%\centering
%\includegraphics[width=0.6\textwidth]{/Users/mmallek/Documents/Thesis/Plots/darea/hrv_darea_hist.png}
%\caption{Histogram of percent of landscape disturbed by wildfire during the simulation. The distribution is substantially right-skewed, and most fires burn less than 20\% of the eligible landscape.}
%\label{fig:darea_hist}
%\end{figure}





\section{Fire Rotation}

% redone 2016-02-06; file saved to /Users/mmallek/Documents/Thesis/Results/rotation-hrv.csv
\begin{table}[!htbp]
\caption{Full fire rotation results for all cover types present within the core study area.}
\label{tab:all-rotations}
\footnotesize
\begin{tabular}{@{}lrrrr@{}}
\toprule
	& 		&		\multicolumn{3}{c}{\textbf{Rotation Period (Years)}} \\
\textbf{Cover Type}  & \textbf{Area (ha)} & %
\textbf{\begin{tabular}[c]{@{}l@{}}Low\\ Mortality\end{tabular}} & \textbf{\begin{tabular}[c]{@{}l@{}}High\\ Mortality\end{tabular}} & \textbf{\begin{tabular}[c]{@{}l@{}}Any\\ Mortality\end{tabular}} \\ \midrule
Agriculture                                  & 16       & 1634      &   74    &  71 \\
Curl-leaf Mountain Mahogany                  & 18       &  278      &  139    &  93 \\
Grassland                                    & 1379     &  583      &   66    &  59 \\
Lodgepole Pine                               & 837      &   67      &  290    &  55 \\
Lodgepole Pine with Aspen                    & 8        &   50      &  211    &  40 \\
Meadow                                       & 1201     & 1413      &   57    &  55 \\
Mixed Evergreen - Mesic                      & 7273     &   58      &  493    &  52 \\
Mixed Evergreen - Ultramafic                 & 604      &  145      & 1338    & 131 \\
Mixed Evergreen - Xeric                      & 6768     &   45      &  394    &  41 \\
Montane Riparian                             & 732      &   94      &  110    &  51 \\
Oak Woodland                                 & 19       &   39      &  119    &  29 \\
Oak-Conifer Forest and Woodland              & 23279    &   28      &  105    &  22 \\
\begin{tabular}[c]{@{}l@{}}Oak-Conifer Forest and Woodland\\  - Ultramafic\end{tabular}  & 1060     & 64        & 268      & 51  \\
Red Fir - Mesic                              & 8563     &  104      & 159      &  63 \\
Red Fir - Ultramafic                         & 294      &  181      & 302      & 113 \\
Red Fir - Xeric                              & 7493     &   60      & 101      &  38 \\
Red Fir with Aspen                           & 31       &   79      & 190      &  56 \\
Sierran Mixed Conifer - Mesic                & 57853    &   35      & 113      &  27 \\
Sierran Mixed Conifer - Ultramafic           & 4124     &  100      & 196      &  66 \\
Sierran Mixed Conifer - Xeric                & 52198    &   34      &  72      &  23 \\
Sierran Mixed Conifer with Aspen             & 58       &   41      & 132      &  31 \\
Subalpine Conifer                            & 638      &  866      & 234      & 184 \\
Urban                                        & 114      & 1641      &  79      &  76 \\
Western White Pine                           & 273      &  104      & 563      &  88 \\
\textbf{Total}       			& \textbf{174830}    & \textbf{38}   & \textbf{103}   & \textbf{28}  \\ 
\bottomrule
\end{tabular}
\end{table}


%%%%%%%%%%%%%%%%%%%%%%%%%%%%%%%%%%%%%%%%%%%%%%%%%%%%%%%%%%%%%%%%%%%%%%%%%%%%%%%%%%%%%%%%%%%%%%%%
%%%%%%%%%%%%%%%%%%%%%%%%%%%%%%%%%%%%%%%%%%%%%%%%%%%%%%%%%%%%%%%%%%%%%%%%%%%%%%%%%%%%%%%%%%%%%%%%

\section{Individual Cover Type Results}
\label{sec:indiv_cov_results}
% preturn redone 
% updated 2016-02-07 to adjust departure values
% updated 2016-02-10 to merge App_ExtraCovResults into this document


The discussion that follows focuses on seven of the nine cover types found within the core project area that were treated as dynamic in the model and that occurred over an extent of at least 1000 ha in the project area. For each of these cover types, I briefly describe the simulated disturbance regime (i.e., spatial extent and distribution, frequency and temporal variability) associated with each relevant disturbance process, the vegetation dynamics resulting from the interplay between these disturbance processes and succession, and an examination of the cover type’s current departure from the simulated HRV. The cover types are presented in descending order by total area within the project landscape. Results for Sierran Mixed Conifer Mesic and Xeric can be found in Chapter~\ref{sec:hrvresults}

\subsection{Oak-Conifer Forest and Woodland}
% figures updated 2015-09
\begin{figure}[!htbp]
  \centering
  \subfloat[][]{
    \centering
    \includegraphics[width=0.5\textwidth]{/Users/mmallek/Documents/Thesis/Plots/darea/hrv_ocfw.png}
    }%
  \subfloat[][]{
    \includegraphics[width=0.5\textwidth]{/Users/mmallek/Documents/Thesis/Plots/darea/hrv_hist_ocfw.png}
    }
  \caption{\small (a) Disturbance trajectory for Oak-Conifer Forest and Woodland. High mortality fire in dark blue; low mortality fire in light blue. (b) Histogram of disturbed hectares with density curve overlaid.} 
  \label{fig:darea_ocfw}
\end{figure}

Oak-Conifer Forest and Woodland (\textsc{ocfw}) is the third most common cover type within the core project area, encompassing 23,279 ha and comprising roughly 13\% of the project area. The frequency and extent of simulated wildfires in oak-conifer forests and woodlands varied markedly across the landscape (Figure~\ref{fig:darea_ocfw}). Wildfire was quite prevalent in this cover type. I summarize the disturbance regime in Tables~\ref{tab:darea_ocfw} and \ref{tab:darea_atleast_ocfw}.


% updated 2015-09-28
\begin{table}[!htbp]
\small
\centering
\caption{Disturbed area summary statistics for Oak-Conifer Forest and Woodland. Proportions shown are relative to the total area of Oak-Conifer Forest and Woodland.}
\label{tab:darea_ocfw}
\begin{tabular}{@{}llll@{}}
\toprule
\textbf{\begin{tabular}[c]{@{}l@{}}Summary Statistic \\ (disturbed area/timestep)\end{tabular}} & \textbf{Low Mortality} & \textbf{High Mortality} & \textbf{Any Mortality} \\ \midrule
$5^{\text{th}}$ percentile         & 3.39  & 0.70  & 4.35   \\
$50^{\text{th}}$ percentile        & 13.92 & 3.63  & 17.82  \\
$95^{\text{th}}$ percentile        & 45.42 & 13.61 & 58.63  \\
Mean                               & 17.61 & 4.78  & 22.39  \\
\textbf{Fire Rotation} & 28       & 105       & 22 \\ \bottomrule
\end{tabular}
\end{table}

% updated 2015-09-28
\begin{table}[!htbp]
\small
\centering
\caption{Summary of disturbed area in terms of proportion of the amount of Oak-Conifer Forest and Woodland burned (any level of mortality) during the simulation (after the equilibration period). For each benchmark proportion of the landscape, I list the number of timesteps during the simulation when that extent burned, the proportion of timesteps that represents, the interval in timesteps calculated from the proportion (i.e. approximately every 4 timesteps, at least 25\% of the landscape burned.), and the interval in years calculated from the interval in timesteps (5 years to a timestep).}
\label{tab:darea_atleast_ocfw}
\begin{tabular}{@{}lllll@{}}
                        & at least 1\% & at least 10\% & at least 25\% & at least 50\% \\ \midrule
Number of timesteps     & 460          & 344           & 156           & 42           \\
Proportion of timesteps & 1.00         & 0.75          & 0.34          & 0.09         \\
Interval (timesteps)    & 1.00         & 1.34          & 2.96          & 10.98        \\
Interval (years)        & 5.01         & 6.70          & 14.78         & 54.88        \\ \bottomrule
\end{tabular}
\end{table}

Visualizing the point-specific fire rotation calls attention to the variability in wildfire recurrence across the study area. I use barplots to show the spread and underlying values in the distribution of point-specific fire rotations, and maps to demonstrate the spatial variability in this metric across the study area (Figure~\ref{fig:preturn_ocfw}). 

\begin{figure}[!htbp]
  \centering
  \subfloat[][]{
    \centering
    \includegraphics[width=0.5\textwidth]{/Users/mmallek/Documents/Thesis/Plots/preturn/not-called-preturn/hrv-ocfw.png}
    }%
  \subfloat[][]{
    \includegraphics[width=0.5\textwidth]{/Users/mmallek/Documents/Thesis/Plots/preturn-maps/fri_ocfw.png}
    }
  \caption{(a) Distribution of point-specific fire rotations for Oak-Conifer Forest and Woodland. The point-specific fire rotation is the average interval between fires over the length of the simulation, excluding the equilibration period. (b) Spatially-explicit depiction of these point-specific fire rotations across the landscape. Cover types other than Oak-Conifer Forest and Woodland are partially obscured in grey.}
\label{fig:preturn_ocfw}
\end{figure}

The age structure and dynamics of oak-conifer forests and woodlands illustrates the interaction between disturbance and succession processes. I focus my analysis on the 5$^{\text{th}}$ to 95$^{\text{th}}$ percentile range of variability for the simulation (excluding the equilibration period). %
%
The distribution of area among seral stages within oak-conifer forests and woodlands fluctuated considerably over time, as expected (Figure~\ref{fig:covcond_ocfw}). Surprisingly for a cover type in which fuels are the largest contributor to disturbance and fire is relatively frequent, the Late--Open seral stage was relatively uncommon during the simulated HRV, though it was more prevalent than on the current landscape.
%

The seral stage distribution appeared to be in dynamic equilibrium (i.e., the percentage in each seral stage varied about a stable mean). The calculated current seral stage distribution was never observed under the simulated HRV (Table~\ref{tab:ssdyn_ocfw}). The most notable departure was the shift from the Mid Development stages, which are dominant in the current landscape, to Late Development stages, which are almost nonexistent on the current landscape. The current proportions of all Late Development canopy cover levels are lower than at any point during the HRV.  The Early Development and Mid--Moderate proportions are within the HRV, but the other five stages are completely departed from the HRV.

% figures updated 2015-09, and again 2016-02
\begin{figure}[!htbp]
  \centering
  \subfloat[][]{
    \centering
    \includegraphics[width=0.6\textwidth]{/Users/mmallek/Documents/Thesis/Plots/covcond-dynamics/notcalledcovcond/OCFW.pdf}
    }%
  \subfloat[][]{
    \includegraphics[height=2.65in]{/Users/mmallek/Tahoe/R/Rplots/November2014/covcond_current_ocfw.png}
    }\\
  \subfloat[][]{
    \includegraphics[width=\textwidth]{/Users/mmallek/Documents/Thesis/Plots/covcond-bycover/OCFW-HRV-boxplots-.png}
  }
  \caption{(a) Seral Stage dynamics for Oak-Conifer Forest and Woodland. The black vertical line at 40 timesteps marks the end of the equilibration period used in this study. (b) Current seral stage distribution for Oak-Conifer Forest and Woodland. (c) Boxplots showing the range of variability for each seral stage over the course of the simulation, excluding the equilibration period. Boxplots were modified so that whiskers extend from the $5^{\text{th}} - 95^{\text{th}}$ percentiles of the observed results. Thick black bars in line with the boxplots denote the current proportion of mesic mixed conifer forests in a given seral stage.} 
  \label{fig:covcond_ocfw}
\end{figure}

\begin{table}[!htbp]
\footnotesize
\caption{Range of variation in landscape structure, illustrating the seral stage dynamics for Oak-Conifer Forest and Woodland (\textsc{ocfw}). For seral stage abbreviations, see Table \ref{condtable}.}
\label{tab:ssdyn_ocfw}
\begin{tabular}{@{}lrrrrr|rrr@{}}
\toprule
\textbf{\begin{tabular}[c]{@{}l@{}}Seral \\ Stage\end{tabular}}  &  \textbf{srv5\%} &  \textbf{srv25\%} &  \textbf{srv50\%} &  \textbf{srv75\%} &  \textbf{srv95\%}  &  \textbf{\begin{tabular}[c]{@{}l@{}}Current\\ \%cover\end{tabular}} & \textbf{\begin{tabular}[c]{@{}l@{}}Current\\ \%srv\end{tabular}} & \textbf{\begin{tabular}[c]{@{}l@{}}Departure\end{tabular}} \\ \midrule
\textsc{early\_all}      &  7.7            &  11.2             &  14.18     &  17.75           &  23.31      &  19.97    &  84    &  moderate      \\
\textsc{mid\_cl   }      &  3.33           &  7.35             &  11.3      &  15.03           &  19.39      &  37.36    &  100   &  complete      \\
\textsc{mid\_mod  }      &  7.31           &  9.16             &  10.88     &  12.88           &  16.55      &  14.61    &  89    &  moderate      \\
\textsc{mid\_op   }      &  8.8            &  12.01            &  15.12     &  18.75           &  24.48      &  24.34    &  95    &  complete       \\
\textsc{late\_cl  }      &  5.37           &  11.38            &  17.76     &  23.13           &  31.36      &  1.58     &  0     &  complete      \\
\textsc{late\_mod }      &  13.81          &  16.51            &  18.23     &  20.22           &  22.84      &  1.02     &  0     &  complete      \\
\textsc{late\_op  }      &  5.45           &  7.89             &  10.61     &  14.12           &  19.02      &  1.12     &  0     &  complete      \\
\bottomrule
\end{tabular}
\end{table}


The spatial configuration of seral stages fluctuated markedly over time as well, although there was considerable variation in the magnitude of variability among configuration metrics (Table~\ref{tab:fragclass_ocfw} in Appendix~\ref{app:full-class-results}). Area-weighted patch and core area exhibited the greatest variability over time. Because all the Late Development stages are nearly absent from the current landscape, configuration metrics consistently differ between the current seral stage distribution and that observed during the simulated HRV. While some seral stages and metrics fall completely outside the HRV, others are well within it. The HRV results for class-level metrics are consistent for six of the seven seral stages, in the sense of their deviation from their current proportion (Mid--Closed is the outlier) (Figures~\ref{fig:ocfw_areaam}--\ref{fig:ocfw_clumpy}). For example, patches are currently smaller, with less core area and geometric complexity, compared to the simulated period. Early Development and Mid--Open canopy patches tended to be less aggregated during the HRV, while the other seral stages were more aggregated. Only the Late--Moderate and Late--Open seral stages were outside the HRV for the \textsc{clumpy} metric, however.


% figures updated 2015-09-20
\begin{figure}[!htbp]
\centering
    \includegraphics[width=0.8\textwidth]{/Users/mmallek/Documents/Thesis/Plots/fragclass-bymetrics/HRV/OCFW-AREA_AM-boxplots.png}
  \caption{Fragstats class-level results for Oak-Conifer Forest and Woodland and area-weighted mean patch area. Boxplot whiskers extend to the 5th and 95th percentile of the observed distribution. The thick grey bar denotes the metric value on the current landscape.}
  \label{fig:ocfw_areaam}
\end{figure}


\begin{figure}[!htbp]
\centering
    \includegraphics[width=0.8\textwidth]{/Users/mmallek/Documents/Thesis/Plots/fragclass-bymetrics/HRV/OCFW-CORE_AM-boxplots.png}
  \caption{Fragstats class-level results for Oak-Conifer Forest and Woodland and area-weighted mean core area. Boxplot whiskers extend to the 5th and 95th percentile of the observed distribution. The thick grey bar denotes the metric value on the current landscape.}
  \label{fig:ocfw_coream}
\end{figure}


\begin{figure}[!htbp]
\centering
    \includegraphics[width=0.8\textwidth]{/Users/mmallek/Documents/Thesis/Plots/fragclass-bymetrics/HRV/OCFW-SHAPE_AM-boxplots.png}
  \caption{Fragstats class-level results for Oak-Conifer Forest and Woodland and area-weighted mean shape index. Boxplot whiskers extend to the 5th and 95th percentile of the observed distribution. The thick grey bar denotes the metric value on the current landscape.}
  \label{fig:ocfw_shapeam}
\end{figure}


\begin{figure}[!htbp]
\centering
    \includegraphics[width=0.8\textwidth]{/Users/mmallek/Documents/Thesis/Plots/fragclass-bymetrics/HRV/OCFW-CLUMPY-boxplots.png}
  \caption{Fragstats class-level results for Oak-Conifer Forest and Woodland and clumpiness. Boxplot whiskers extend to the 5th and 95th percentile of the observed distribution. The thick grey bar denotes the metric value on the current landscape.}
  \label{fig:ocfw_clumpy}
\end{figure}


%%%%%%%%%%%%%%%%%%%%%%%%%%%%%%%%%%%%%%%%%%%%%%%%%%%%%%%%%%%%%%%%%%%%%%%%%%%%%
%%%%%%%%%%%%%%%%%%%%%%%%%%%%%%%%%%%%%%%%%%%%%%%%%%%%%%%%%%%%%%%%%%%%%%%%%%%%%
%%%%%%%%%%%%%%%%%%%%%%%%%%%%%%%%%%%%%%%%%%%%%%%%%%%%%%%%%%%%%%%%%%%%%%%%%%%%%
%%%%%%%%%%%%%%%%%%%%%%%%%%%%%%%%%%%%%%%%%%%%%%%%%%%%%%%%%%%%%%%%%%%%%%%%%%%%%
%%%%%%%%%%%%%%%%%%%%%%%%%%%%%%%%%%%%%%%%%%%%%%%%%%%%%%%%%%%%%%%%%%%%%%%%%%%%%

\clearpage
\subsection{Red Fir - Mesic} 
% figures updated 2015-09
\begin{figure}[!htbp]
  \centering
  \subfloat[][]{
    \centering
    \includegraphics[width=0.5\textwidth]{/Users/mmallek/Documents/Thesis/Plots/darea/hrv_rfrm.png}
    }%
  \subfloat[][]{
    \includegraphics[width=0.5\textwidth]{/Users/mmallek/Documents/Thesis/Plots/darea/hrv_hist_rfrm.png}
    }
  \caption{\small (a) Disturbance trajectory for Red Fir - Mesic. High mortality fire in dark blue; low mortality fire in light blue. (b) Histogram of disturbed hectares with density curve overlaid.} 
  \label{fig:darea_rfrm}
\end{figure}

Red Fir - Mesic (\textsc{rfr\_m}) is a somewhat common cover type within the core project area, encompassing 8,563 ha and comprising roughly 5\% of the project area. Wildfire was fairly common in this cover type. The frequency and extent of simulated wildfires in mesic red fir forests varied markedly across the landscape (Figure~\ref{fig:darea_rfrm}). I summarize the disturbance regime in Tables~\ref{tab:darea_rfrm} and \ref{tab:darea_atleast_rfrm}.

% updated 2015-09-28
\begin{table}[!htbp]
\small
\centering
\caption{\small Disturbed area summary statistics for Red Fir - Mesic. Proportions shown are relative to the total area of Red Fir - Mesic.}
\label{tab:darea_rfrm}
\begin{tabular}{@{}llll@{}}
\toprule
\textbf{\begin{tabular}[c]{@{}l@{}}Summary Statistic \\ (disturbed area/timestep)\end{tabular}} & \textbf{Low Mortality} & \textbf{High Mortality} & \textbf{Any Mortality} \\ \midrule
$5^{\text{th}}$ percentile         & 0.20  & 0.06  & 0.30  \\
$50^{\text{th}}$ percentile        & 2.70  & 1.53  & 4.31  \\
$95^{\text{th}}$ percentile        & 18.75 & 12.74 & 31.17 \\
Mean                               & 4.80  & 3.14  & 7.94  \\
\textbf{Fire Rotation} & 104      & 159       & 63 \\ \bottomrule
\end{tabular}
\end{table}

% updated 2015-09-28
\begin{table}[!htbp]
\small
\centering
\caption{Summary of disturbed area in terms of proportion of the amount of \textsc{rfr\_m} burned (any level of mortality) during the simulation (after the equilibration period). For each benchmark proportion of the landscape, I list the number of timesteps during the simulation when that extent burned, the proportion of timesteps that represents, the interval in timesteps calculated from the proportion (i.e. approximately every 4 timesteps, at least 25\% of the landscape burned.), and the interval in years calculated from the interval in timesteps (5 years to a timestep).}
\label{tab:darea_atleast_rfrm}
\begin{tabular}{@{}lllll@{}}
                        & at least 1\% & at least 10\% & at least 25\% & at least 50\% \\ \midrule
Number of timesteps     & 380          & 116           & 36            & 1          \\
Proportion of timesteps & 0.82         & 0.25          & 0.08          & 0          \\
Interval (timesteps)    & 1.21         & 3.97          & 12.81         & 461        \\
Interval (years)        & 6.07         & 19.87         & 64.03         & 2305      \\ \bottomrule
\end{tabular}
\end{table}

Visualizing the point-specific fire rotation calls attention to the variability in wildfire recurrence across the study area. I use barplots to show the spread and underlying values in the distribution of point-specific fire rotations, and maps to demonstrate the spatial variability in this metric across the study area (Figure~\ref{fig:preturn_rfrm}). 

\begin{figure}[!htbp]
  \centering
  \subfloat[][]{
    \centering
    \includegraphics[width=0.5\textwidth]{/Users/mmallek/Documents/Thesis/Plots/preturn/not-called-preturn/hrv-rfrm.png}
    }%
  \subfloat[][]{
    \includegraphics[width=0.5\textwidth]{/Users/mmallek/Documents/Thesis/Plots/preturn-maps/fri_rfrm.png}
    }
  \caption{(a) Distribution of point-specific fire rotations for Red Fir - Mesic. The point-specific fire rotation is the average interval between fires over the length of the simulation, excluding the equilibration period. (b) Spatially-explicit depiction of these point-specific fire rotations across the landscape. Cover types other than Red Fir - Mesic are partially obscured in grey.}
\label{fig:preturn_rfrm}
\end{figure}

The age structure and dynamics of mesic red fir forests illustrates the interaction between disturbance and succession processes. I focus my analysis on the 5$^{\text{th}}$ to 95$^{\text{th}}$ percentile range of variability for the simulation (excluding the equilibration period). %

The distribution of area among seral stages within mesic red fir forests fluctuated over time, but the moderate and open canopy cover seral stages are remarkable for having an extremely small range of variability (Figure~\ref{fig:covcond_rfrm}). As expected for mesic red fir forests (Appendix~\ref{app:covertypedesc}), closed canopies predominated. Early Development, which includes post-fire chaparral fields, was the next most extensive cover type. %

The seral stage distribution appeared to be in dynamic equilibrium (i.e., the percentage in each seral stages varied about a stable mean). Our calculated current seral stage distribution was never observed under the simulated HRV (Table~\ref{tab:ssdyn_rfrm}). The most notable departure was a shift from moderate canopy cover to closed canopy cover. Current levels of moderate canopy cover are much higher, and current levels of closed canopy cover much lower, than during the simulated HRV. Early Development and Late--Open are both moderately departed within the HRV. The other five seral stages are completely departed from the HRV. 

\begin{figure}[!htbp]
  \centering
  \subfloat[][]{
    \centering
    \includegraphics[width=0.6\textwidth]{/Users/mmallek/Documents/Thesis/Plots/covcond-dynamics/notcalledcovcond/RFRM.pdf}
    }%
  \subfloat[][]{
    \includegraphics[height=2.65in]{/Users/mmallek/Tahoe/R/Rplots/November2014/covcond_current_rfrm.png}
    }\\
  \subfloat[][]{
    \includegraphics[width=\textwidth]{/Users/mmallek/Documents/Thesis/Plots/covcond-bycover/RFRM-HRV-boxplots-.png}
  }
  \caption{(a) Seral Stage dynamics for Red Fir - Mesic. The black vertical line at 40 timesteps marks the end of the equilibration period used in this study. (b) Current seral stage distribution for Red Fir - Mesic. (c) Boxplots showing the range of variability for each seral stage over the course of the simulation, excluding the equilibration period. Boxplots were modified so that whiskers extend from the $5^{\text{th}} - 95^{\text{th}}$ percentiles of the observed results. Thick black bars in line with the boxplots denote the current proportion of mesic mixed conifer forests in a given seral stage.} 
  \label{fig:covcond_rfrm}
\end{figure}

\begin{table}[!htbp]
\footnotesize
\caption{Range of variation in landscape structure, illustrating the seral stage dynamics for Red Fir - Mesic (\textsc{rfr\_m}). For seral stage abbreviations, see Table \ref{condtable}.}
\label{tab:ssdyn_rfrm}
\begin{tabular}{@{}rrrrrr|rrr@{}}
\toprule
\textbf{\begin{tabular}[c]{@{}l@{}}Seral \\ Stage\end{tabular}}  &  \textbf{srv5\%} &  \textbf{srv25\%} &  \textbf{srv50\%} &  \textbf{srv75\%} &  \textbf{srv95\%}    &  \textbf{\begin{tabular}[c]{@{}l@{}}Current\\ \%cover\end{tabular}} & \textbf{\begin{tabular}[c]{@{}l@{}}Current\\ \%srv\end{tabular}} & \textbf{\begin{tabular}[c]{@{}l@{}}Departure\end{tabular}} \\ \midrule
                            % 5th         25th        50th         75th                  95th      current value  current %ile  dep. index
\textsc{early\_all}        &   6.47        &  10.49   &  15.61     &  22.55            &  32.82     &  24.21    &  81    &  moderate      \\
\textsc{mid\_cl   }        &   20.6        &  29.15   &  34.73     &  41.06            &  48.77     &  3.63     &  0     &  complete      \\
\textsc{mid\_mod  }        &   0.79        &  1.16    &  1.46      &  1.95             &  2.62      &  18.67    &  100   &  complete      \\
\textsc{mid\_op   }        &   0.36        &  0.64    &  0.91      &  1.32             &  2.17      &  16.7     &  100   &  complete      \\
\textsc{late\_cl  }         &  26.29       &  33.03   &  39.48     &  45.47            &  53.47     &  10.7     &  0     &  complete      \\
\textsc{late\_mod }        &   2.31        &  3.2     &  4.19      &  5.2              &  6.95      &  21.96    &  100   &  complete      \\
\textsc{late\_op  }        &   0.73        &  1.1     &  1.61      &  2.2              &  3.4       &  4.13     &  100   &  complete     \\
\bottomrule
\end{tabular}
\end{table}

The spatial configuration of seral stages fluctuated markedly over time as well, although there was considerable variation in the magnitude of variability among configuration metrics (Table~\ref{tab:fragclass_rfrm} in Appendix~\ref{app:full-class-results}). Several metrics exhibited high variability over time, including the area-weighted patch and core area, edge and patch density, radius of gyration, and contrast-weighted edge density (Figures~\ref{fig:rfrm_areaam}--\ref{fig:rfrm_clumpy}). The narrow range of variability observed in some seral stages is repeated in the configuration metrics. In general, current values for the class metrics are often completely outside or near the extremes of the simulated HRV. The direction of the departure depends on seral stage. The Early Development and closed canopy seral stages tend to be smaller in both area and core area, less aggregated, and less geometrically complex now than during the HRV. In contrast, the moderate and open canopy seral stages tend to be larger, with more core area, more aggregation, and more complexity now than during the HRV. The fact that the closed canopies dominated the HRV cover type-seral stages distribution explains this divergence.

% figures updated 2015-09-20
\begin{figure}[!htbp]
\centering
    \includegraphics[width=0.8\textwidth]{/Users/mmallek/Documents/Thesis/Plots/fragclass-bymetrics/HRV/RFR_M-AREA_AM-boxplots.png}
  \caption{Fragstats class-level results for Red Fir - Mesic and area-weighted mean patch area. Boxplot whiskers extend to the 5th and 95th percentile of the observed distribution. The thick grey bar denotes the metric value on the current landscape.}
  \label{fig:rfrm_areaam}
\end{figure}


\begin{figure}[!htbp]
\centering
    \includegraphics[width=0.8\textwidth]{/Users/mmallek/Documents/Thesis/Plots/fragclass-bymetrics/HRV/RFR_M-CORE_AM-boxplots.png}
  \caption{Fragstats class-level results for Red Fir - Mesic and area-weighted mean core area. Boxplot whiskers extend to the 5th and 95th percentile of the observed distribution. The thick grey bar denotes the metric value on the current landscape.}
  \label{fig:rfrm_coream}
\end{figure}


\begin{figure}[!htbp]
\centering
    \includegraphics[width=0.8\textwidth]{/Users/mmallek/Documents/Thesis/Plots/fragclass-bymetrics/HRV/RFR_M-SHAPE_AM-boxplots.png}
  \caption{Fragstats class-level results for Red Fir - Mesic and area-weighted mean shape index. Boxplot whiskers extend to the 5th and 95th percentile of the observed distribution. The thick grey bar denotes the metric value on the current landscape.}
  \label{fig:rfrm_shapeam}
\end{figure}


\begin{figure}[!htbp]
\centering
    \includegraphics[width=0.8\textwidth]{/Users/mmallek/Documents/Thesis/Plots/fragclass-bymetrics/HRV/RFR_M-CLUMPY-boxplots.png}
  \caption{Fragstats class-level results for Red Fir - Mesic and clumpiness. Boxplot whiskers extend to the 5th and 95th percentile of the observed distribution. The thick grey bar denotes the metric value on the current landscape.}
  \label{fig:rfrm_clumpy}
\end{figure}

%%%%%%%%%%%%%%%%%%%%%%%%%%%%%%%%%%%%%%%%%%%%%%%%%%%%%%%%%%%%%%%%%%%%%%%%%%%%%
%%%%%%%%%%%%%%%%%%%%%%%%%%%%%%%%%%%%%%%%%%%%%%%%%%%%%%%%%%%%%%%%%%%%%%%%%%%%%
%%%%%%%%%%%%%%%%%%%%%%%%%%%%%%%%%%%%%%%%%%%%%%%%%%%%%%%%%%%%%%%%%%%%%%%%%%%%%
%%%%%%%%%%%%%%%%%%%%%%%%%%%%%%%%%%%%%%%%%%%%%%%%%%%%%%%%%%%%%%%%%%%%%%%%%%%%%
%%%%%%%%%%%%%%%%%%%%%%%%%%%%%%%%%%%%%%%%%%%%%%%%%%%%%%%%%%%%%%%%%%%%%%%%%%%%%

\clearpage
\subsection{Red Fir - Xeric} 

% figures updated 2015-09
\begin{figure}[!htbp]
  \centering
  \subfloat[][]{
    \centering
    \includegraphics[width=0.5\textwidth]{/Users/mmallek/Documents/Thesis/Plots/darea/hrv_rfrx.png}
    }%
  \subfloat[][]{
    \includegraphics[width=0.5\textwidth]{/Users/mmallek/Documents/Thesis/Plots/darea/hrv_hist_rfrx.png}
    }
  \caption{\small (a) Disturbance trajectory for Red Fir - Xeric. High mortality fire in dark blue; low mortality fire in light blue. (b) Histogram of disturbed hectares with density curve overlaid.} 
  \label{fig:darea_rfrx}
\end{figure}

Red Fir - Xeric (\textsc{rfr\_x}) is a somewhat common cover type within the core project area, encompassing 7,493 ha and comprising roughly 5\% of the project area. Wildfire was fairly common in this cover type, and occurred more frequently on average than in mesic red fir forests.
The frequency and extent of simulated wildfires in xeric red fir forests varied markedly across the landscape (Figure~\ref{fig:darea_rfrx}). I summarize the disturbance regime in Tables~\ref{tab:darea_rfrx} and \ref{tab:darea_atleast_rfrx}.

% updated 2015-09-28
\begin{table}[!htbp]
\small
\centering
\caption{Disturbed area summary statistics for Red Fir - Xeric. Proportions shown are relative to the total area of Red Fir - Xeric.}
\label{tab:darea_rfrx}
\begin{tabular}{@{}llll@{}}
\toprule
\textbf{\begin{tabular}[c]{@{}l@{}}Summary Statistic \\ (disturbed area/timestep)\end{tabular}} & \textbf{Low Mortality} & \textbf{High Mortality} & \textbf{Any Mortality} \\ \midrule
$5^{\text{th}}$ percentile         & 0.34  & 0.17  & 0.56  \\ 
$50^{\text{th}}$ percentile        & 4.47  & 2.77  & 7.21  \\ 
$95^{\text{th}}$ percentile        & 30.60 & 18.39 & 47.14 \\ 
Mean                               & 8.29  & 4.96  & 13.25  \\
\textbf{Fire Rotation} & 60       & 101       & 38  \\ \bottomrule
\end{tabular}
\end{table}

\begin{table}[!htbp]
\small
\centering
\caption{Summary of disturbed area in terms of proportion of the amount of \textsc{rfr\_x} burned (any level of mortality) during the simulation (after the equilibration period). For each benchmark proportion of the landscape, I list the number of timesteps during the simulation when that extent burned, the proportion of timesteps that represents, the interval in timesteps calculated from the proportion (i.e. approximately every 4 timesteps, at least 25\% of the landscape burned.), and the interval in years calculated from the interval in timesteps (5 years to a timestep).}
\label{tab:darea_atleast_rfrx}
\begin{tabular}{@{}lllll@{}}
                        & at least 1\% & at least 10\% & at least 25\% & at least 50\% \\ \midrule
Number of timesteps     & 422          & 188           & 77            & 21            \\
Proportion of timesteps & 0.92         & 0.41          & 0.17          & 0.05          \\
Interval (timesteps)    & 1.09         & 2.45          & 5.99          & 21.95         \\
Interval (years)        & 5.46         & 12.26         & 29.94         & 109.76       \\ \bottomrule
\end{tabular}
\end{table}

Visualizing the point-specific fire rotation calls attention to the variability in wildfire recurrence across the study area. I use barplots to show the spread and underlying values in the distribution of point-specific fire rotations, and maps to demonstrate the spatial variability in this metric across the study area (Figure~\ref{fig:preturn_rfrx}).

\begin{figure}[!htbp]
  \centering
  \subfloat[][]{
    \centering
    \includegraphics[width=0.5\textwidth]{/Users/mmallek/Documents/Thesis/Plots/preturn/not-called-preturn/hrv-rfrx.png}
    }%
  \subfloat[][]{
    \includegraphics[width=0.5\textwidth]{/Users/mmallek/Documents/Thesis/Plots/preturn-maps/fri_rfrx.png}
    }
  \caption{(a) Distribution of point-specific fire rotations for Red Fir - Xeric. The point-specific fire rotation is the average interval between fires over the length of the simulation, excluding the equilibration period. (b) Spatially-explicit depiction of these point-specific fire rotations across the landscape. Cover types other than Red Fir - Xeric are partially obscured in grey.}
\label{fig:preturn_rfrx}
\end{figure}

The age structure and dynamics of xeric red fir forests illustrates the interaction between disturbance and succession processes. I focus my analysis on the 5$^{\text{th}}$ to 95$^{\text{th}}$ percentile range of variability for the simulation (excluding the equilibration period). %

The distribution of area among seral stages within xeric red fir forests fluctuated over time, but less dramatically than many other cover types (Figure~\ref{fig:covcond_rfrx}). Interestingly, although open canopies dominated during Mid Development, the distribution of the three Late Development seral stages was roughly equal. This shift towards higher canopy closure may be due to an increasing resilience to wildfire disturbances by stands of that age: wildfires may burn the understory without significantly affecting overstory canopy cover. Early Development, which includes post-fire chaparral fields, was the single most extensive cover type. The current proportion of Early Development is somewhat departed from the simulated HRV. %

The seral stage distribution appeared to be in dynamic equilibrium. Our calculated current seral stage distribution was never observed under the simulated HRV (Table~\ref{tab:ssdyn_rfrx}). Although the Late--Closed stage is not currently departed from the HRV, and the Early Development stage is moderately departed within the HRV, the other stages are completely departed from the HRV.

\begin{figure}[!htbp]
  \centering
  \subfloat[][]{
    \centering
    \includegraphics[width=0.6\textwidth]{/Users/mmallek/Documents/Thesis/Plots/covcond-dynamics/notcalledcovcond/RFRX.pdf}
    }%
  \subfloat[][]{
    \includegraphics[height=2.65in]{/Users/mmallek/Tahoe/R/Rplots/November2014/covcond_current_rfrx.png}
    }\\
  \subfloat[][]{
    \includegraphics[width=\textwidth]{/Users/mmallek/Documents/Thesis/Plots/covcond-bycover/RFRX-HRV-boxplots-.png}
  }
  \caption{(a) Seral Stage dynamics for Red Fir - Xeric. The black vertical line at 40 timesteps marks the end of the equilibration period used in this study. (b) Current seral stage distribution for Red Fir - Xeric. (c) Boxplots showing the range of variability for each seral stage over the course of the simulation, excluding the equilibration period. Boxplots were modified so that whiskers extend from the $5^{\text{th}} - 95^{\text{th}}$ percentiles of the observed results. Thick black bars in line with the boxplots denote the current proportion of mesic mixed conifer forests in a given seral stage.} 
  \label{fig:covcond_rfrx}
\end{figure}

\begin{table}[!htbp]
\footnotesize
\caption{Range of variation in landscape structure, illustrating the seral stage dynamics for Red Fir -  Xeric (\textsc{rfr\_x}). For seral stage abbreviations, see Table \ref{condtable}.}
\label{tab:ssdyn_rfrx}
\begin{tabular}{@{}rrrrrr|rrr@{}}
\toprule
\textbf{\begin{tabular}[c]{@{}l@{}}Seral \\ Stage\end{tabular}}  &  \textbf{srv5\%} &  \textbf{srv25\%} &  \textbf{srv50\%} &  \textbf{srv75\%} &  \textbf{srv95\%}    &  \textbf{\begin{tabular}[c]{@{}l@{}}Current\\ \%cover\end{tabular}} & \textbf{\begin{tabular}[c]{@{}l@{}}Current\\ \%srv\end{tabular}} & \textbf{\begin{tabular}[c]{@{}l@{}}Departure\end{tabular}} \\ \midrule
                          % 5th           25th       50th          75th          95th             current value current %ile dep. index
\textsc{early\_all}         &  24.76       &  33.1    &  37        &  41.44            &  45.72     &  32.39    &  23    &  none      \\
\textsc{mid\_cl   }        &   0.24        &  0.5     &  0.88      &  1.5              &  2.73      &  8.26     &  100   &  complete      \\
\textsc{mid\_mod  }        &   3.12        &  5.33    &  7.02      &  9.25             &  12.11     &  18.66    &  100   &  complete      \\
\textsc{mid\_op   }        &   13.47       &  17.52   &  19.98     &  22.6             &  27.2      &  12.58    &  3     &  complete      \\
\textsc{late\_cl  }        &   6.46        &  8.73    &  11.28     &  14.19            &  20.38     &  10.45    &  43    &  none      \\
\textsc{late\_mod }        &   8.83        &  10.31   &  11.7      &  12.96            &  14.6      &  14.57    &  95    &  complete      \\
\textsc{late\_op  }        &   6.2         &  8.92    &  11.04     &  13.38            &  16.26     &  3.1      &  0     &  complete     \\
\bottomrule
\end{tabular}
\end{table}

The spatial configuration of seral stages fluctuated markedly over time as well, although there was considerable variation in the magnitude of variability among configuration metrics (Table~\ref{tab:fragclass_rfrx} in Appendix~\ref{app:full-class-results}). In general most seral stages were not departed from the HRV or moderately departed within the HRV, across metrics. However, the Mid Development and Late--Open stages were often completely departed from the HRV, or otherwise moderately departed. These patches were larger, more aggregated, more geometrically complex, and contained more core area during the simulation than in the current landscape.

In general, current values for the class metrics are often completely outside or near the extremes of the simulated HRV. The direction of the departure depends on seral stage. The Early Development and closed canopy stages tend to be smaller in both area and core area, less aggregated, and less geometrically complex now than during the HRV (Figures~\ref{fig:rfrx_areaam}--\ref{fig:rfrx_clumpy}). In contrast, the moderate and open canopy seral stages tend to be larger, with more core area, more aggregation, and more complexity now than during the HRV. The fact that the closed canopy seral stages dominated the HRV cover type-seral stage distribution explains this divergence. Specifically, current patches tend to be smaller in both area and core area and more numerous, with less complex geometries and more edge than patches during the simulated HRV.


% figures updated 2015-09-20
\begin{figure}[!htbp]
\centering
    \includegraphics[width=0.8\textwidth]{/Users/mmallek/Documents/Thesis/Plots/fragclass-bymetrics/HRV/RFR_X-AREA_AM-boxplots.png}
  \caption{Fragstats class-level results for Red Fir - Xeric and area-weighted mean patch area. Boxplot whiskers extend to the 5th and 95th percentile of the observed distribution. The thick grey bar denotes the metric value on the current landscape.}
  \label{fig:rfrx_areaam}
\end{figure}


\begin{figure}[!htbp]
\centering
    \includegraphics[width=0.8\textwidth]{/Users/mmallek/Documents/Thesis/Plots/fragclass-bymetrics/HRV/RFR_X-CORE_AM-boxplots.png}
  \caption{Fragstats class-level results for Red Fir - Xeric and area-weighted mean core area. Boxplot whiskers extend to the 5th and 95th percentile of the observed distribution. The thick grey bar denotes the metric value on the current landscape.}
  \label{fig:rfrx_coream}
\end{figure}


\begin{figure}[!htbp]
\centering
    \includegraphics[width=0.8\textwidth]{/Users/mmallek/Documents/Thesis/Plots/fragclass-bymetrics/HRV/RFR_X-SHAPE_AM-boxplots.png}
  \caption{Fragstats class-level results for Red Fir - Xeric and area-weighted mean shape index. Boxplot whiskers extend to the 5th and 95th percentile of the observed distribution. The thick grey bar denotes the metric value on the current landscape.}
  \label{fig:rfrx_shapeam}
\end{figure}


\begin{figure}[!htbp]
\centering
    \includegraphics[width=0.8\textwidth]{/Users/mmallek/Documents/Thesis/Plots/fragclass-bymetrics/HRV/RFR_X-CLUMPY-boxplots.png}
  \caption{Fragstats class-level results for Red Fir - Xeric and clumpiness. Boxplot whiskers extend to the 5th and 95th percentile of the observed distribution. The thick grey bar denotes the metric value on the current landscape.}
  \label{fig:rfrx_clumpy}
\end{figure}



%%%%%%%%%%%%%%%%%%%%%%%%%%%%%%%%%%%%%%%%%%%%%%%%%%%%%%%%%%%%%%%%%%%%%%%%%%%%%
%%%%%%%%%%%%%%%%%%%%%%%%%%%%%%%%%%%%%%%%%%%%%%%%%%%%%%%%%%%%%%%%%%%%%%%%%%%%%
%%%%%%%%%%%%%%%%%%%%%%%%%%%%%%%%%%%%%%%%%%%%%%%%%%%%%%%%%%%%%%%%%%%%%%%%%%%%%
%%%%%%%%%%%%%%%%%%%%%%%%%%%%%%%%%%%%%%%%%%%%%%%%%%%%%%%%%%%%%%%%%%%%%%%%%%%%%
%%%%%%%%%%%%%%%%%%%%%%%%%%%%%%%%%%%%%%%%%%%%%%%%%%%%%%%%%%%%%%%%%%%%%%%%%%%%%

\clearpage
\subsection{Mixed Evergreen - Mesic} 
% figures updated 2015-09
\begin{figure}[!htbp]
  \centering
  \subfloat[][]{
    \centering
    \includegraphics[width=0.5\textwidth]{/Users/mmallek/Documents/Thesis/Plots/darea/hrv_megm.png}
    }%
  \subfloat[][]{
    \includegraphics[width=0.5\textwidth]{/Users/mmallek/Documents/Thesis/Plots/darea/hrv_hist_megm.png}
    }
  \caption{(a) \small Disturbance trajectory for Mixed Evergreen - Mesic. High mortality fire in dark blue; low mortality fire in light blue. (b) Histogram of disturbed hectares with density curve overlaid.} 
  \label{fig:darea_megm}
\end{figure}

Mixed Evergreen - Mesic (\textsc{meg\_m}) is a somewhat common cover type within the core project area, encompassing 7,273 ha and comprising roughly 4\% of the project area. Wildfire was prevalent in mesic mixed evergreen forest. The frequency and extent of simulated wildfires  varied markedly across the landscape (Figure~\ref{fig:darea_megm}). I summarize the disturbance regime in Tables~\ref{tab:darea_megm} and \ref{tab:darea_atleast_megm}.

% updated 2015-09-28
\begin{table}[!htbp]
\small
\centering
\caption{Disturbed area summary statistics for Mixed Evergreen - Mesic. Proportions shown are relative to the total area of Mixed Evergreen - Mesic.}
\label{tab:darea_megm}
  \begin{tabular}{@{}llll@{}} 
  \toprule
  \textbf{\begin{tabular}[c]{@{}l@{}}Summary Statistic \\ (disturbed area/timestep)\end{tabular}} & \textbf{\begin{tabular}[c]{@{}l@{}}Low  Mortality\end{tabular}} & \textbf{\begin{tabular}[c]{@{}l@{}}High  Mortality\end{tabular}} & \textbf{\begin{tabular}[c]{@{}l@{}}Any  Mortality\end{tabular}} \\ \midrule
$5^{\text{th}}$ percentile    &  0.59          & 0.04          & 0.61     \\
$50^{\text{th}}$ percentile   &  5.13          & 0.54          & 5.84     \\
$95^{\text{th}}$ percentile   &  27.91         & 3.75          & 31.76    \\
  Mean                        &  8.62          & 1.01          & 9.63     \\
  \textbf{Fire Rotation}  & 58       & 493       & 52 \\  \bottomrule
  \end{tabular}
\end{table}
       
% updated 2015-09-28
\begin{table}[!htbp]
\small
\centering
\caption{Summary of disturbed area in terms of proportion of the amount of \textsc{meg\_m} burned (any level of mortality) during the simulation (after the equilibration period). For each benchmark proportion of the landscape, I list the number of timesteps during the simulation when that extent burned, the proportion of timesteps that represents, the interval in timesteps calculated from the proportion (i.e. approximately every 4 timesteps, at least 25\% of the landscape burned.), and the interval in years calculated from the interval in timesteps (5 years to a timestep).}
\label{tab:darea_atleast_megm}
\begin{tabular}{@{}lllll@{}}
                        & at least 1\% & at least 10\% & at least 25\% & at least 50\% \\ \midrule
Number of timesteps     & 422          & 155           & 43            & 4             \\
Proportion of timesteps & 0.92         & 0.34          & 0.09          & 0.01          \\
Interval (timesteps)    & 1.09         & 2.97          & 10.72         & 115.25        \\
Interval (years)        & 5.46         & 14.87         & 53.60         & 576.25       \\ \bottomrule
\end{tabular}
\end{table}

Visualizing the point-specific fire rotation calls attention to the variability in wildfire recurrence across the study area. I use barplots to show the spread and underlying values in the distribution of point-specific fire rotations, and maps to demonstrate the spatial variability in this metric across the study area (Figure~\ref{fig:preturn_megm}).

\begin{figure}[!htbp]
  \centering
  \subfloat[][]{
    \centering
    \includegraphics[width=0.5\textwidth]{/Users/mmallek/Documents/Thesis/Plots/preturn/not-called-preturn/hrv-megm.png}
    }%
  \subfloat[][]{
    \includegraphics[width=0.5\textwidth]{/Users/mmallek/Documents/Thesis/Plots/preturn-maps/fri_megm.png}
    }
  \caption{(a) Distribution of point-specific fire rotations for Mixed Evergreen - Mesic. The point-specific fire rotation is the average interval between fires over the length of the simulation, excluding the equilibration period. (b) Spatially-explicit depiction of these point-specific fire rotations across the landscape. Cover types other than Mixed Evergreen - Mesic are partially obscured in grey.}
    \label{fig:preturn_megm}
\end{figure}

The age structure and dynamics of mesic mixed evergreen forest illustrates the interaction between disturbance and succession processes. I focus my analysis on the 5$^{\text{th}}$ to 95$^{\text{th}}$ percentile range of variability for the simulation (excluding the equilibration period). %

The distribution of area among seral stages within mesic mixed evergreen forest fluctuated very little over time (Figure~\ref{fig:covcond_megm}). Because high mortality fire is very rare in this cover type, and the time to reaching a Late Development stage is relatively short (Appendix \ref{app:covertypedesc}), the vast majority of the cover type's extent was in the Late--Closed stage during the simulation (Table~\ref{tab:ssdyn_rfrm}). %

The seral stage distribution appeared to be in dynamic equilibrium (i.e., the percentage in each seral stages varied about a stable mean). The most notable departure was the shift from Mid Development to Late Development seral stages. About 52\% of the current landscape is comprised of the mesic mixed evergreen forest in Mid Development seral stages, but the Late Development seral stages were always dominant under the simulated HRV. 

\begin{figure}[!htbp]
  \centering
  \subfloat[][]{
    \centering
    \includegraphics[width=0.6\textwidth]{/Users/mmallek/Documents/Thesis/Plots/covcond-dynamics/notcalledcovcond/MEGM.pdf}
    }%
  \subfloat[][]{
  \centering
  \includegraphics[height=2.65in]{/Users/mmallek/Tahoe/R/Rplots/November2014/covcond_current_megm.png}
    }\\
  \subfloat[][]{
    \includegraphics[width=\textwidth]{/Users/mmallek/Documents/Thesis/Plots/covcond-bycover/MEGM-HRV-boxplots-.png}
  }
  \caption{(a) Seral Stage dynamics for Mixed Evergreen - Mesic. The black vertical line at 40 timesteps marks the end of the equilibration period used in this study. (b) Current seral stage distribution for Mixed Evergreen - Mesic. (c) Boxplots showing the range of variability for each seral stage over the course of the simulation, excluding the equilibration period. Boxplots were modified so that whiskers extend from the $5^{\text{th}} - 95^{\text{th}}$ percentiles of the observed results. Thick black bars in line with the boxplots denote the current proportion of mesic mixed conifer forests in a given seral stage.}
\label{fig:covcond_megm}
\end{figure}

\begin{table}[!htbp]
\footnotesize
\caption{Range of variation in landscape structure, illustrating the seral stage dynamics for Mixed Evergreen - Mesic (\textsc{meg\_m}). For seral stage abbreviations, see Table \ref{condtable}.}
\label{tab:ssdyn_megm}
\begin{tabular}{@{}rrrrrr|rrr@{}}
\toprule
 \textbf{\begin{tabular}[c]{@{}l@{}}Seral \\ Stage\end{tabular}}  &  \textbf{srv5\%} &  \textbf{srv25\%} &  \textbf{srv50\%} &  \textbf{srv75\%} &  \textbf{srv95\%}  &  \textbf{\begin{tabular}[c]{@{}l@{}}Current\\ \%cover\end{tabular}} & \textbf{\begin{tabular}[c]{@{}l@{}}Current\\ \%srv\end{tabular}} & \textbf{\begin{tabular}[c]{@{}l@{}}Departure\end{tabular}} \\ \midrule
\textsc{early\_all}      &  1.19           &  2.22             &  3.55      &  5.04            &  7.58       &  8.21     &  98    &  complete      \\
\textsc{mid\_cl   }      &  0.01           &  0.08             &  0.29      &  0.77            &  2.57       &  36.53    &  100   &  complete      \\
\textsc{mid\_mod  }      &  0.69           &  1.35             &  2.14      &  3.48            &  6.01       &  9.76     &  100   &  complete      \\
\textsc{mid\_op   }      &  0.04           &  0.11             &  0.23      &  0.41            &  0.78       &  6.37     &  100   &  complete      \\
\textsc{late\_cl  }      &  53.97          &  64.51            &  70.81     &  76.28           &  81.97      &  29.31    &  0     &  complete      \\
\textsc{late\_mod }      &  8.16           &  12.19            &  14.49     &  17.64           &  21.7       &  7.31     &  4     &  complete      \\
\textsc{late\_op  }      &  2.66           &  4.96             &  7.16      &  10.38           &  14.99      &  2.5      &  4     &  complete      \\
\bottomrule
\end{tabular}
\end{table}

The spatial configuration of seral stages fluctuated markedly over time as well, although there was considerable variation in the magnitude of variability among configuration metrics (Table~\ref{tab:fragclass_megm} in Appendix~\ref{app:full-class-results}). Because the landscape is so dominated by the Late--Closed and Late--Moderate seral stages, I focus on the configuration metrics for these classes. In general, the current landscape contains fewer, smaller, and more clumped patches than existed under the simulated HRV (Figures~\ref{fig:megm_areaam}--\ref{fig:megm_clumpy}). Current patches in Late--Closed are less geometrically complex and have less area in cores than during the simulated HRV.


% figures updated 2015-09-20
\begin{figure}[!htbp]
\centering
    \includegraphics[width=0.8\textwidth]{/Users/mmallek/Documents/Thesis/Plots/fragclass-bymetrics/HRV/MEG_M-AREA_AM-boxplots.png}
  \caption{Fragstats class-level results for Mixed Evergreen - Mesic and area-weighted mean patch area. Boxplot whiskers extend to the 5th and 95th percentile of the observed distribution. The thick grey bar denotes the metric value on the current landscape.}
  \label{fig:megm_areaam}
\end{figure}


\begin{figure}[!htbp]
\centering
    \includegraphics[width=0.8\textwidth]{/Users/mmallek/Documents/Thesis/Plots/fragclass-bymetrics/HRV/MEG_M-CORE_AM-boxplots.png}
  \caption{Fragstats class-level results for Mixed Evergreen - Mesic and area-weighted mean core area. Boxplot whiskers extend to the 5th and 95th percentile of the observed distribution. The thick grey bar denotes the metric value on the current landscape.}
  \label{fig:megm_coream}
\end{figure}


\begin{figure}[!htbp]
\centering
    \includegraphics[width=0.8\textwidth]{/Users/mmallek/Documents/Thesis/Plots/fragclass-bymetrics/HRV/MEG_M-SHAPE_AM-boxplots.png}
  \caption{Fragstats class-level results for Mixed Evergreen - Mesic and area-weighted mean shape index. Boxplot whiskers extend to the 5th and 95th percentile of the observed distribution. The thick grey bar denotes the metric value on the current landscape.}
  \label{fig:megm_shapeam}
\end{figure}


\begin{figure}[!htbp]
\centering
    \includegraphics[width=0.8\textwidth]{/Users/mmallek/Documents/Thesis/Plots/fragclass-bymetrics/HRV/MEG_M-CLUMPY-boxplots.png}
  \caption{Fragstats class-level results for Mixed Evergreen - Mesic and clumpiness. Boxplot whiskers extend to the 5th and 95th percentile of the observed distribution. The thick grey bar denotes the metric value on the current landscape.}
  \label{fig:megm_clumpy}
\end{figure}

%%%%%%%%%%%%%%%%%%%%%%%%%%%%%%%%%%%%%%%%%%%%%%%%%%%%%%%%%%%%%%%%%%%%%%%%%%%%%
%%%%%%%%%%%%%%%%%%%%%%%%%%%%%%%%%%%%%%%%%%%%%%%%%%%%%%%%%%%%%%%%%%%%%%%%%%%%%
%%%%%%%%%%%%%%%%%%%%%%%%%%%%%%%%%%%%%%%%%%%%%%%%%%%%%%%%%%%%%%%%%%%%%%%%%%%%%
%%%%%%%%%%%%%%%%%%%%%%%%%%%%%%%%%%%%%%%%%%%%%%%%%%%%%%%%%%%%%%%%%%%%%%%%%%%%%
%%%%%%%%%%%%%%%%%%%%%%%%%%%%%%%%%%%%%%%%%%%%%%%%%%%%%%%%%%%%%%%%%%%%%%%%%%%%%

\clearpage
\subsection{Mixed Evergreen - Xeric} 
% figures updated 2015-09
\begin{figure}[!htbp]
  \centering
  \subfloat[][]{
    \centering
    \includegraphics[width=0.5\textwidth]{/Users/mmallek/Documents/Thesis/Plots/darea/hrv_megx.png}
    }%
  \subfloat[][]{
    \includegraphics[width=0.5\textwidth]{/Users/mmallek/Documents/Thesis/Plots/darea/hrv_hist_megx.png}
    }
  \caption{\small (a) Disturbance trajectory for Mixed Evergreen - Xeric. High mortality fire in dark blue; low mortality fire in light blue. (b) Histogram of disturbed hectares with density curve overlaid.} 
  \label{fig:darea_megx}
\end{figure}

Mixed Evergreen - Xeric (\textsc{meg\_x}) is a somewhat common cover type within the core project area, encompassing 6,768 ha and comprising roughly 4\% of the project area. Wildfire was prevalent in xeric mixed evergreen forest. The frequency and extent of simulated wildfires varied markedly across the landscape (Figure~\ref{fig:darea_megx}). I summarize the disturbance regime in Tables~\ref{tab:darea_megx} and \ref{tab:darea_atleast_megx}.

% updated 2015-09-28
\begin{table}[!htbp]
\small
\centering
\caption{Disturbed area summary statistics for Mixed Evergreen - Xeric. Proportions shown are relative to the total area of Mixed Evergreen - Xeric.}
\label{tab:darea_megx}
\begin{tabular}{@{}llll@{}}
\toprule
\textbf{\begin{tabular}[c]{@{}l@{}}Summary Statistic \\ (disturbed area/timestep)\end{tabular}} & \textbf{Low Mortality} & \textbf{High Mortality} & \textbf{Any Mortality} \\ \midrule
$5^{\text{th}}$ percentile     & 0.92          & 0.06          & 1.01  \\ 
$50^{\text{th}}$ percentile    & 7.68          & 0.76          & 8.73  \\ 
$95^{\text{th}}$ percentile    & 29.05         & 3.91          & 32.27 \\ 
Mean                           & 11.03         & 1.27          & 12.30 \\
\textbf{Fire Rotation}  & 45       & 394       & 41 \\ \bottomrule
\end{tabular}
\end{table}
     
% updated 2015-09-28
\begin{table}[!htbp]
\small
\centering
\caption{Summary of disturbed area in terms of proportion of the amount of \textsc{meg\_x} burned (any level of mortality) during the simulation (after the equilibration period). For each benchmark proportion of the landscape, I list the number of timesteps during the simulation when that extent burned, the proportion of timesteps that represents, the interval in timesteps calculated from the proportion (i.e. approximately every 4 timesteps, at least 25\% of the landscape burned.), and the interval in years calculated from the interval in timesteps (5 years to a timestep).}
\label{tab:darea_atleast_megx}
\begin{tabular}{@{}lllll@{}}
                        & at least 1\% & at least 10\% & at least 25\% & at least 50\% \\ \midrule
Number of timesteps     & 439          & 211           & 62            & 5             \\
Proportion of timesteps & 0.95         & 0.46          & 0.13          & 0.01          \\
Interval (timesteps)    & 1.05         & 2.18          & 7.44          & 92.20         \\
Interval (years)        & 5.25         & 10.92         & 37.18         & 461.00       \\ \bottomrule
\end{tabular}
\end{table}

Visualizing the point-specific fire rotation calls attention to the variability in wildfire recurrence across the study area. I use barplots to show the spread and underlying values in the distribution of point-specific fire rotations, and maps to demonstrate the spatial variability in this metric across the study area (Figure~\ref{fig:preturn_megx}).

\begin{figure}[!htbp]
  \centering
  \subfloat[][]{
    \centering
    \includegraphics[width=0.5\textwidth]{/Users/mmallek/Documents/Thesis/Plots/preturn/not-called-preturn/hrv-megx.png}
    }%
  \subfloat[][]{
    \includegraphics[width=0.5\textwidth]{/Users/mmallek/Documents/Thesis/Plots/preturn-maps/fri_megx.png}
    }
  \caption{(a) Distribution of point-specific fire rotations for Mixed Evergreen - Xeric. The point-specific fire rotation is the average interval between fires over the length of the simulation, excluding the equilibration period. (b) Spatially-explicit depiction of these point-specific fire rotations across the landscape. Cover types other than Mixed Evergreen - Xeric are partially obscured in grey.}
\label{fig:preturn_megx}
\end{figure}

The age structure and dynamics of xeric mixed evergreen forest illustrates the interaction between disturbance and succession processes. I focus my analysis on the 5$^{\text{th}}$ to 95$^{\text{th}}$ percentile range of variability for the simulation (excluding the equilibration period). %

The distribution of area among seral stages within xeric mixed evergreen forest fluctuated over time (Figure~\ref{fig:covcond_megx}). Because high mortality fire is very rare in this cover type, and the time to reaching a Late Development stage is relatively short (Appendix~\ref{app:covertypedesc}), the vast majority of the landscape was in the Late--Closed seral stage during the simulation (Table~\ref{tab:ssdyn_megx}).  %

The seral stage distribution appeared to be in dynamic equilibrium. The most notable departure was the shift from Mid--Closed to the Late Development seral stages, especially Late--Closed. The current landscape contains 71\% of the xeric mixed evergreen forest in Mid Development stages, but the Late Development stages were always dominant under the simulated HRV. 

\begin{figure}[!htbp]
  \centering
  \subfloat[][]{
    \centering
    \includegraphics[width=0.6\textwidth]{/Users/mmallek/Documents/Thesis/Plots/covcond-dynamics/notcalledcovcond/MEGX.pdf}
    }%
  \subfloat[][]{
    \includegraphics[height=2.65in]{/Users/mmallek/Tahoe/R/Rplots/November2014/covcond_current_megx.png}
    }\\
  \subfloat[][]{
    \includegraphics[width=\textwidth]{/Users/mmallek/Documents/Thesis/Plots/covcond-bycover/MEGX-HRV-boxplots-.png}
  }
  \caption{(a) Seral Stage dynamics for Mixed Evergreen - Xeric. The black vertical line at 40 timesteps marks the end of the equilibration period used in this study. (b) Current seral stage distribution for Mixed Evergreen - Xeric. (c) Boxplots showing the range of variability for each seral stage over the course of the simulation, excluding the equilibration period. Boxplots were modified so that whiskers extend from the $5^{\text{th}} - 95^{\text{th}}$ percentiles of the observed results. Thick black bars in line with the boxplots denote the current proportion of mesic mixed conifer forests in a given seral stage.} 
  \label{fig:covcond_megx}
\end{figure}

\begin{table}[!htbp]
\footnotesize
\caption{Range of variation in landscape structure, illustrating the seral stage dynamics for Mixed Evergreen - Xeric (\textsc{meg\_x}),. For seral stage abbreviations, see Table \ref{condtable}.}
\label{tab:ssdyn_megx}
\begin{tabular}{@{}rrrrrr|rrr@{}}
\toprule
 \textbf{\begin{tabular}[c]{@{}l@{}}Seral \\ Stage\end{tabular}}  &  \textbf{srv5\%} &  \textbf{srv25\%} &  \textbf{srv50\%} &  \textbf{srv75\%} &  \textbf{srv95\%}  &  \textbf{\begin{tabular}[c]{@{}l@{}}Current\\ \%cover\end{tabular}} & \textbf{\begin{tabular}[c]{@{}l@{}}Current\\ \%srv\end{tabular}} & \textbf{\begin{tabular}[c]{@{}l@{}}Departure\end{tabular}} \\ \midrule
\textsc{early\_all}      &  1.68           &  3.32             &  4.58      &  6.06            &  8.64       &  10.88    &  99    &  complete       \\
\textsc{mid\_cl   }      &  0.01           &  0.08             &  0.31      &  0.87            &  2.37       &  48.8     &  100   &  complete      \\
\textsc{mid\_mod  }      &  1.07           &  1.97             &  2.98      &  4.27            &  6.47       &  9.39     &  100   &  complete      \\
\textsc{mid\_op   }      &  0.06           &  0.19             &  0.36      &  0.57            &  0.97       &  12.87    &  100   &  complete      \\
\textsc{late\_cl  }      &  52.93          &  61.01            &  67.58     &  71.66           &  77.74      &  12.84    &  0     &  complete      \\
\textsc{late\_mod }      &  11.3           &  14.32            &  16.68     &  19.66           &  23.14      &  3.84     &  0     &  complete      \\
\textsc{late\_op  }      &  3.07           &  5.35             &  7.12      &  9.77            &  12.88      &  1.38     &  0     &  complete      \\
\bottomrule
\end{tabular}
\end{table}

The spatial configuration of seral stages fluctuated markedly over time as well, although there was considerable variation in the magnitude of variability among configuration metrics (Table~\ref{tab:fragclass_megx} in Appendix~\ref{app:full-class-results}). Area-weighted patch and core area, patch density, and radius of gyration exhibited the greatest variability over time (Figures~\ref{fig:megx_areaam}--\ref{fig:megx_clumpy}). Because the landscape is so dominated by the Late--Closed seral stage, I use its configuration metrics as a proxy for the cover type as a whole. In general, the current landscape contains fewer, smaller, and more isolated patches than existed under the simulated HRV. Patches in Late--Closed are less geometrically complex and have less area in cores in the current landscape than during the simulated HRV. This stage is completely departed from the HRV for all these metrics.

% figures updated 2015-09-20
\begin{figure}[!htbp]
\centering
    \includegraphics[width=0.8\textwidth]{/Users/mmallek/Documents/Thesis/Plots/fragclass-bymetrics/HRV/MEG_X-AREA_AM-boxplots.png}
  \caption{Fragstats class-level results for Mixed Evergreen - Xeric and area-weighted mean patch area. Boxplot whiskers extend to the 5th and 95th percentile of the observed distribution. The thick grey bar denotes the metric value on the current landscape.}
  \label{fig:megx_areaam}
\end{figure}


\begin{figure}[!htbp]
\centering
    \includegraphics[width=0.8\textwidth]{/Users/mmallek/Documents/Thesis/Plots/fragclass-bymetrics/HRV/MEG_X-CORE_AM-boxplots.png}
  \caption{Fragstats class-level results for Mixed Evergreen - Xeric and area-weighted mean core area. Boxplot whiskers extend to the 5th and 95th percentile of the observed distribution. The thick grey bar denotes the metric value on the current landscape.}
  \label{fig:megx_coream}
\end{figure}


\begin{figure}[!htbp]
\centering
    \includegraphics[width=0.8\textwidth]{/Users/mmallek/Documents/Thesis/Plots/fragclass-bymetrics/HRV/MEG_X-SHAPE_AM-boxplots.png}
  \caption{Fragstats class-level results for Mixed Evergreen - Xeric and area-weighted mean shape index. Boxplot whiskers extend to the 5th and 95th percentile of the observed distribution. The thick grey bar denotes the metric value on the current landscape.}
  \label{fig:megx_shapeam}
\end{figure}


\begin{figure}[!htbp]
\centering
    \includegraphics[width=0.8\textwidth]{/Users/mmallek/Documents/Thesis/Plots/fragclass-bymetrics/HRV/MEG_X-CLUMPY-boxplots.png}
  \caption{Fragstats class-level results for Mixed Evergreen - Xeric and clumpiness. Boxplot whiskers extend to the 5th and 95th percentile of the observed distribution. The thick grey bar denotes the metric value on the current landscape.}
  \label{fig:megx_clumpy}
\end{figure}

%%%%%%%%%%%%%%%%%%%%%%%%%%%%%%%%%%%%%%%%%%%%%%%%%%%%%%%%%%%%%%%%%%%%%%%%%%%%%
%%%%%%%%%%%%%%%%%%%%%%%%%%%%%%%%%%%%%%%%%%%%%%%%%%%%%%%%%%%%%%%%%%%%%%%%%%%%%
%%%%%%%%%%%%%%%%%%%%%%%%%%%%%%%%%%%%%%%%%%%%%%%%%%%%%%%%%%%%%%%%%%%%%%%%%%%%%
%%%%%%%%%%%%%%%%%%%%%%%%%%%%%%%%%%%%%%%%%%%%%%%%%%%%%%%%%%%%%%%%%%%%%%%%%%%%%
%%%%%%%%%%%%%%%%%%%%%%%%%%%%%%%%%%%%%%%%%%%%%%%%%%%%%%%%%%%%%%%%%%%%%%%%%%%%%

\clearpage
\subsection{Sierran Mixed Conifer - Ultramafic} 
% figures updated 2015-09
\begin{figure}[!htbp]
  \centering
  \subfloat[][]{
    \centering
    \includegraphics[width=0.5\textwidth]{/Users/mmallek/Documents/Thesis/Plots/darea/hrv_smcu.png}
    }%
  \subfloat[][]{
    \includegraphics[width=0.5\textwidth]{/Users/mmallek/Documents/Thesis/Plots/darea/hrv_hist_smcu.png}
    }
  \caption{\small (a) Disturbance trajectory for Sierran Mixed Conifer - Ultramafic. High mortality fire in dark blue; low mortality fire in light blue. (b) Histogram of disturbed hectares with density curve overlaid.} 
  \label{fig:darea_smcu}
\end{figure}

Sierran Mixed Conifer - Ultramafic (\textsc{smc\_u}) is a relatively uncommon cover type within the core project area, encompassing 4,124 ha and comprising roughly 2\% of the project area. Wildfire is much less common in this cover type compared to non-ultramafic sierran mixed conifer forests. Ultramafic soils support scattered, but rarely dense stands of trees and shrubs, creating fuel discontinuities that stop fires from spreading easily. The frequency and extent of simulated wildfires in ultramafic sierran mixed conifer forests varied markedly across the landscape (Figure~\ref{fig:darea_smcu}).  I summarize the disturbance regime in Tables~\ref{tab:darea_smcu} and \ref{tab:darea_atleast_smcu}.

% updated 2015-09-28
\begin{table}[!htbp]
\small
\centering
\caption{Disturbed area summary statistics for Sierran Mixed Conifer - Ultramafic. Proportions shown are relative to the total area of Sierran Mixed Conifer - Ultramafic.}
\label{tab:darea_smcu}
\begin{tabular}{@{}llll@{}}
\toprule
\textbf{\begin{tabular}[c]{@{}l@{}}Summary Statistic \\ (disturbed area/timestep)\end{tabular}} & \textbf{Low Mortality} & \textbf{High Mortality} & \textbf{Any Mortality} \\ \midrule
$5^{\text{th}}$ percentile    & 0.08          & 0.04          & 0.14  \\
$50^{\text{th}}$ percentile   & 3.09          & 1.48          & 4.81  \\
$95^{\text{th}}$ percentile   & 15.54         & 9.13          & 24.45 \\
Mean                          & 5.01          & 2.55          & 7.56  \\
\textbf{Fire Rotation} & 100       & 196       & 66 \\  \bottomrule
\end{tabular}
\end{table}
      
% updated 2015-09-28
\begin{table}[!htbp]
\small
\centering
\caption{Summary of disturbed area in terms of proportion of the amount of \textsc{smc\_u} burned (any level of mortality) during the simulation (after the equilibration period). For each benchmark proportion of the landscape, I list the number of timesteps during the simulation when that extent burned, the proportion of timesteps that represents, the interval in timesteps calculated from the proportion (i.e. approximately every 4 timesteps, at least 25\% of the landscape burned.), and the interval in years calculated from the interval in timesteps (5 years to a timestep).}
\label{tab:darea_atleast_smcu}
\begin{tabular}{@{}lllll@{}}
                        & at least 1\% & at least 10\% & at least 25\% & at least 50\% \\ \midrule
Number of timesteps     & 364          & 128           & 21            & 1        \\
Proportion of timesteps & 0.79         & 0.28          & 0.05          & 0        \\
Interval (timesteps)    & 1.27         & 3.60          & 21.95         & 461      \\
Interval (years)        & 6.33         & 18.01         & 109.76        & 2305    \\ \bottomrule
\end{tabular}
\end{table}

Visualizing the point-specific fire rotation calls attention to the variability in wildfire recurrence across the study area. I use barplots to show the spread and underlying values in the distribution of point-specific fire rotations, and maps to demonstrate the spatial variability in this metric across the study area (Figure~\ref{fig:preturn_smcu}).

\begin{figure}[!htbp]
  \centering
  \subfloat[][]{
    \centering
    \includegraphics[width=0.5\textwidth]{/Users/mmallek/Documents/Thesis/Plots/preturn/not-called-preturn/hrv-smcu.png}
    }%
  \subfloat[][]{
    \includegraphics[width=0.5\textwidth]{/Users/mmallek/Documents/Thesis/Plots/preturn-maps/fri_smcu.png}
    }
  \caption{(a) Distribution of point-specific fire rotations for Sierran Mixed Conifer - Ultramafic. The point-specific fire rotation is the average interval between fires over the length of the simulation, excluding the equilibration period. (b) Spatially-explicit depiction of these point-specific fire rotations across the landscape. Cover types other than Sierran Mixed Conifer - Ultramafic are partially obscured in grey.}
\label{fig:preturn_smcu}
\end{figure}

The age structure and dynamics of ultramafic mixed conifer forests illustrates the interaction between disturbance and succession processes. I focus my analysis on the 5$^{\text{th}}$ to 95$^{\text{th}}$ percentile range of variability for the simulation (excluding the equilibration period). %

The distribution of area among seral stages within ultramafic mixed conifer forests fluctuated narrowly over time (Figure~\ref{fig:covcond_smcu}). Interestingly, although open canopy seral stages dominated during Middle Development, the distribution of the three Late Development stages was roughly equal (Table~\ref{tab:ssdyn_smcu}. Ultramafic soils present a challenge to vegetation, which may explain the dominance of open canopies at the Mid Development stage. However, because fire is relatively uncommon, the shift in dominance at the Late Development stage may reflect the additional time available to vegetation to grow into a closed canopy (Appendix~\ref{app:covertypedesc}). %

The seral stage distribution appeared to be in dynamic equilibrium. The most notable departures were the decrease in area classified as Early Development, currently at 49\% of the landscape, and the increase in area classified as Mid--Open, currently at 5\%. The only seral stages not completely departed from the HRV were Mid--Closed and Mid--Moderate.

\begin{figure}[!htbp]
  \centering
  \subfloat[][]{
    \centering
    \includegraphics[width=0.6\textwidth]{/Users/mmallek/Documents/Thesis/Plots/covcond-dynamics/notcalledcovcond/SMCU.pdf}
    }%
  \subfloat[][]{
    \includegraphics[height=2.65in]{/Users/mmallek/Tahoe/R/Rplots/November2014/covcond_current_smcu.png}
    }\\
  \subfloat[][]{
    \includegraphics[width=\textwidth]{/Users/mmallek/Documents/Thesis/Plots/covcond-bycover/SMCU-HRV-boxplots-.png}
  }
  \caption{(a) Seral Stage dynamics for Sierran Mixed Conifer - Ultramafic. The black vertical line at 40 timesteps marks the end of the equilibration period used in this study. (b) Current seral stage distribution for Sierran Mixed Conifer - Ultramafic. (c) Boxplots showing the range of variability for each seral stage over the course of the simulation, excluding the equilibration period. Boxplots were modified so that whiskers extend from the $5^{\text{th}} - 95^{\text{th}}$ percentiles of the observed results. Thick black bars in line with the boxplots denote the current proportion of mesic mixed conifer forests in a given seral stage.} 
  \label{fig:covcond_smcu}
\end{figure}

\begin{table}[!htbp]
\footnotesize
\caption{Range of variation in landscape structure, illustrating the seral stage dynamics for Sierran Mixed Conifer - Ultramafic (\textsc{smc\_u}). For seral stage abbreviations, see Table \ref{condtable}.}
\label{tab:ssdyn_smcu}
\begin{tabular}{@{}rrrrrr|rrr@{}}
\toprule
 \textbf{\begin{tabular}[c]{@{}l@{}}Seral \\ Stage\end{tabular}}  &  \textbf{srv5\%} &  \textbf{srv25\%} &  \textbf{srv50\%} &  \textbf{srv75\%} &  \textbf{srv95\%}    &  \textbf{\begin{tabular}[c]{@{}l@{}}Current\\ \%cover\end{tabular}} & \textbf{\begin{tabular}[c]{@{}l@{}}Current\\ \%srv\end{tabular}} & \textbf{\begin{tabular}[c]{@{}l@{}}Departure\end{tabular}} \\ \midrule
\textsc{early\_all}        &   26.04       &  29.09   &  32.15     &  34.42            &  37.5      &  48.7     &  100   &  complete      \\
\textsc{mid\_cl   }        &   1.27        &  1.84    &  2.23      &  2.69             &  3.51      &  2.99     &  85    &  moderate       \\
\textsc{mid\_mod  }        &   5.76        &  6.77    &  7.68      &  8.78             &  10.93     &  6.77     &  25    &  none    \\
\textsc{mid\_op   }        &   17.82       &  20.95   &  22.69     &  24.83            &  27.45     &  5.33     &  0     &  complete      \\
\textsc{late\_cl  }        &   9           &  11.89   &  14.25     &  16.78            &  21.36     &  24.43    &  99    &  complete      \\
\textsc{late\_mod }        &   9.14        &  10.14   &  11.04     &  11.89            &  13.92     &  8.51     &  1     &  complete      \\
\textsc{late\_op  }        &   5.95        &  7.65    &  9.18      &  10.72            &  13.67     &  3.27     &  0     &  complete      \\
\bottomrule
\end{tabular}
\end{table}

The spatial configuration of seral stages fluctuated markedly over time as well, although there was considerable variation in the magnitude of variability among configuration metrics (Table~\ref{tab:fragclass_smcu} in Appendix~\ref{app:full-class-results}). The class-level metrics for ultramafic mixed conifer forests for Early Development, Mid--Open, Late--Closed, and Late--Moderate typically fall completely outside the simulated HRV (Figures~\ref{fig:smcu_areaam}--\ref{fig:smcu_clumpy}). Of these, Mid--Open is currently characterized by smaller, less complex patches with less core area and less aggregation than the same type during the simulated HRV. The other classes have the opposite result. The remaining classes (Mid--Closed, Mid--Moderate, Late--Open) are either not departed from the HRV or moderately departed within the HRV.


% figures updated 2015-09-20
\begin{figure}[!htbp]
\centering
    \includegraphics[width=0.8\textwidth]{/Users/mmallek/Documents/Thesis/Plots/fragclass-bymetrics/HRV/SMC_U-AREA_AM-boxplots.png}
  \caption{Fragstats class-level results for Sierran Mixed Conifer - Ultramafic and area-weighted mean patch area. Boxplot whiskers extend to the 5th and 95th percentile of the observed distribution. The thick grey bar denotes the metric value on the current landscape.}
  \label{fig:smcu_areaam}
\end{figure}


\begin{figure}[!htbp]
\centering
    \includegraphics[width=0.8\textwidth]{/Users/mmallek/Documents/Thesis/Plots/fragclass-bymetrics/HRV/SMC_U-CORE_AM-boxplots.png}
  \caption{Fragstats class-level results for Sierran Mixed Conifer - Ultramafic and area-weighted mean core area. Boxplot whiskers extend to the 5th and 95th percentile of the observed distribution. The thick grey bar denotes the metric value on the current landscape.}
  \label{fig:smcu_coream}
\end{figure}


\begin{figure}[!htbp]
\centering
    \includegraphics[width=0.8\textwidth]{/Users/mmallek/Documents/Thesis/Plots/fragclass-bymetrics/HRV/SMC_U-SHAPE_AM-boxplots.png}
  \caption{Fragstats class-level results for Sierran Mixed Conifer - Ultramafic and area-weighted mean shape index. Boxplot whiskers extend to the 5th and 95th percentile of the observed distribution. The thick grey bar denotes the metric value on the current landscape.}
  \label{fig:smcu_shapeam}
\end{figure}


\begin{figure}[!htbp]
\centering
    \includegraphics[width=0.8\textwidth]{/Users/mmallek/Documents/Thesis/Plots/fragclass-bymetrics/HRV/SMC_U-CLUMPY-boxplots.png}
  \caption{Fragstats class-level results for Sierran Mixed Conifer - Ultramafic and clumpiness. Boxplot whiskers extend to the 5th and 95th percentile of the observed distribution. The thick grey bar denotes the metric value on the current landscape.}
  \label{fig:smcu_clumpy}
\end{figure}

%%%%%%%%%%%%%%%%%%%%%%%%%%%%%%%%%%%%%%%%%%%%%%%%%%%%%%%%%%%%%%%%%%%%%%%%%%%%%
%%%%%%%%%%%%%%%%%%%%%%%%%%%%%%%%%%%%%%%%%%%%%%%%%%%%%%%%%%%%%%%%%%%%%%%%%%%%%
%%%%%%%%%%%%%%%%%%%%%%%%%%%%%%%%%%%%%%%%%%%%%%%%%%%%%%%%%%%%%%%%%%%%%%%%%%%%%
%%%%%%%%%%%%%%%%%%%%%%%%%%%%%%%%%%%%%%%%%%%%%%%%%%%%%%%%%%%%%%%%%%%%%%%%%%%%%
%%%%%%%%%%%%%%%%%%%%%%%%%%%%%%%%%%%%%%%%%%%%%%%%%%%%%%%%%%%%%%%%%%%%%%%%%%%%%



\clearpage
\subsection{Oak-Conifer Forest and Woodland - Ultramafic} 
% figures updated 2015-09
\begin{figure}[!htbp]
  \centering
  \subfloat[][]{
    \centering
    \includegraphics[width=0.5\textwidth]{/Users/mmallek/Documents/Thesis/Plots/darea/hrv_ocfwu.png}
    }%
  \subfloat[][]{
    \includegraphics[width=0.5\textwidth]{/Users/mmallek/Documents/Thesis/Plots/darea/hrv_hist_ocfwu.png}
    }
  \caption{\small (a) Disturbance trajectory for Oak-Conifer Forest and Woodland - Ultramafic. High mortality fire in dark blue; low mortality fire in light blue. (b) Histogram of disturbed hectares with density curve overlaid.} 
  \label{fig:darea_ocfwu}
\end{figure}

Oak-Conifer Forest and Woodland - Ultramafic (\textsc{ocfw\_u}) is a relatively uncommon cover type within the core project area, encompassing 1,060 ha and comprising roughly 0.6\% of the project area. Wildfire is much less common in this cover type compared to its non-ultramafic oak-conifer forests and woodlands. Ultramafic soils support scattered, but rarely dense stands of trees and shrubs, creating fuel discontinuities that stop fires from spreading easily. The frequency and extent of simulated wildfires in ultramafic oak-conifer forests and woodlands varied markedly across the landscape (Figure~\ref{fig:darea_ocfwu}). I summarize the disturbance regime in Tables~\ref{tab:darea_ocfwu} and \ref{tab:darea_atleast_ocfwu}.

% updated 2015-09-28
\begin{table}[!htbp]
\small
\centering
\caption{Disturbed area summary statistics for Oak-Conifer Forest and Woodland - Ultramafic. Proportions shown are relative to the total area of Oak-Conifer Forest and Woodland - Ultramafic.}
\label{tab:darea_ocfwu}
\begin{tabular}{@{}llll@{}}
\toprule
\textbf{\begin{tabular}[c]{@{}l@{}}Summary Statistic \\ (disturbed area/timestep)\end{tabular}} & \textbf{Low Mortality} & \textbf{High Mortality} & \textbf{Any Mortality} \\ \midrule
$5^{\text{th}}$ percentile    & 0.06          & 0.00          & 0.06     \\
$50^{\text{th}}$ percentile   & 4.16          & 0.77          & 4.74     \\
$95^{\text{th}}$ percentile   & 31.08         & 7.11          & 36.12    \\
Mean                          & 7.86          & 1.87          & 9.72     \\
\textbf{Fire Rotation} & 64       & 268      & 51 \\ \bottomrule
\end{tabular}
\end{table}

% updated 2015-09-28
\begin{table}[!htbp]
\small
\centering
\caption{Summary of disturbed area in terms of proportion of the amount of \textsc{ocfw\_u} burned (any level of mortality) during the simulation (after the equilibration period). For each benchmark proportion of the landscape, I list the number of timesteps during the simulation when that extent burned, the proportion of timesteps that represents, the interval in timesteps calculated from the proportion (i.e. approximately every 4 timesteps, at least 25\% of the landscape burned.), and the interval in years calculated from the interval in timesteps (5 years to a timestep).}
\label{tab:darea_atleast_ocfwu}
\begin{tabular}{@{}lllll@{}}
                        & at least 1\% & at least 10\% & at least 25\% & at least 50\% \\ \midrule
Number of timesteps     & 355          & 149           & 51            & 5         \\
Proportion of timesteps & 0.77         & 0.32          & 0.11          & 0.01      \\
Interval (timesteps)    & 1.30         & 3.09          & 9.04          & 92.20     \\
Interval (years)        & 6.49         & 15.47         & 45.20         & 461   \\ \bottomrule
\end{tabular}
\end{table}

Visualizing the point-specific fire rotation calls attention to the variability in wildfire recurrence across the study area. I use barplots to show the spread and underlying values in the distribution of point-specific fire rotations, and maps to demonstrate the spatial variability in this metric across the study area (Figure~\ref{fig:preturn_ocfwu}).

\begin{figure}[!htbp]
  \centering
  \subfloat[][]{
    \centering
    \includegraphics[width=0.5\textwidth]{/Users/mmallek/Documents/Thesis/Plots/preturn/not-called-preturn/hrv-smcu.png}
    }%
  \subfloat[][]{
    \includegraphics[width=0.5\textwidth]{/Users/mmallek/Documents/Thesis/Plots/preturn-maps/fri_smcu.png}
    }
  \caption{(a) Distribution of point-specific fire rotations for Oak-Conifer Forest and Woodland - Ultramafic. The point-specific fire rotation is the average interval between fires over the length of the simulation, excluding the equilibration period. (b) Spatially-explicit depiction of these point-specific fire rotations across the landscape. Cover types other than Oak-Conifer Forest and Woodland - Ultramafic are partially obscured in grey.}
\label{fig:preturn_ocfwu}
\end{figure}

The age structure and dynamics of ultramafic oak-conifer forests and woodlands illustrates the interaction between disturbance and succession processes. I focus my analysis on the 5$^{\text{th}}$ to 95$^{\text{th}}$ percentile range of variability for the simulation (excluding the equilibration period). %

The distribution of area among seral stages within ultramafic oak-conifer forests and woodlands fluctuated narrowly over time (Figure~\ref{fig:covcond_ocfwu}). Late Development stages are rare on the current landscape, but well represented during the HRV (Table~\ref{tab:ssdyn_ocfwu}). Conversely, mid closed is currently quite common, but is virtually absent during the HRV. The current amount of Early Development vegetation is near the median value during the HRV, and is the only stage not completely departed from the HRV. Ultramafic soils present a challenge to vegetation, which may explain the dominance of open canopies at the Mid Development stage. However, because fire is relatively uncommon, the shift in dominance at the Late Development stage may reflect the additional time available to vegetation to grow into a closed canopy (Appendix~\ref{app:covertypedesc}). %

\begin{figure}[!htbp]
  \centering
  \subfloat[][]{
    \centering
    \includegraphics[width=0.6\textwidth]{/Users/mmallek/Documents/Thesis/Plots/covcond-dynamics/notcalledcovcond/OCFWU.pdf}
    }%
  \subfloat[][]{
    \includegraphics[height=2.65in]{/Users/mmallek/Tahoe/R/Rplots/November2014/covcond_current_ocfwu.png}
    }\\
  \subfloat[][]{
    \includegraphics[width=\textwidth]{/Users/mmallek/Documents/Thesis/Plots/covcond-bycover/OCFWU-HRV-boxplots-.png}
  }
  \caption{(a) Seral Stage dynamics for Oak-Conifer Forest and Woodland - Ultramafic. The black vertical line at 40 timesteps marks the end of the equilibration period used in this study. (b) Current seral stage distribution for Oak-Conifer Forest and Woodland - Ultramafic. (c) Boxplots showing the range of variability for each seral stage over the course of the simulation, excluding the equilibration period. Boxplots were modified so that whiskers extend from the $5^{\text{th}} - 95^{\text{th}}$ percentiles of the observed results. Thick black bars in line with the boxplots denote the current proportion of mesic mixed conifer forests in a given seral stage.} 
  \label{fig:covcond_ocfwu}
\end{figure}



\begin{table}[!htbp]
\footnotesize
\caption{Range of variation in landscape structure, illustrating the seral stage dynamics for Oak-Conifer Forest and Woodland - Ultramafic (\textsc{ocfw\_u}). For seral stage abbreviations, see Table \ref{condtable}.}
\label{tab:ssdyn_ocfwu}
\begin{tabular}{@{}rrrrrr|rrr@{}}
\toprule
 \textbf{\begin{tabular}[c]{@{}l@{}}Seral \\ Stage\end{tabular}}  &  \textbf{srv5\%} &  \textbf{srv25\%} &  \textbf{srv50\%} &  \textbf{srv75\%} &  \textbf{srv95\%}  &  \textbf{\begin{tabular}[c]{@{}l@{}}Current\\ \%cover\end{tabular}} & \textbf{\begin{tabular}[c]{@{}l@{}}Current\\ \%srv\end{tabular}} & \textbf{\begin{tabular}[c]{@{}l@{}}Departure\end{tabular}} \\ \midrule
\textsc{early\_all}      &  9.05           &  13.66            &  16.3      &  21              &  26.17      &  17.76    &  63    &  none       \\
\textsc{mid\_cl   }      &  0.06           &  0.14             &  0.29      &  0.54            &  1.15       &  29.32    &  100   &  complete      \\
\textsc{mid\_mod  }      &  2.55           &  4.11             &  5.48      &  8.31            &  11.38      &  11.54    &  96    &  complete       \\
\textsc{mid\_op   }      &  14.79          &  19.29            &  22.49     &  26.11           &  30.28      &  33.49    &  100   &  complete      \\
\textsc{late\_cl  }      &  7.39           &  15.12            &  21.21     &  27.3            &  36.45      &  5.35     &  1     &  complete      \\
\textsc{late\_mod }      &  15.06          &  18.62            &  20.99     &  23.86           &  29.17      &  2.2      &  0     &  complete    \\
\textsc{late\_op  }      &  3.92           &  7.21             &  10.44     &  13.64           &  18.99      &  0.34     &  0     &  complete      \\
\bottomrule
\end{tabular}
\end{table}

The spatial configuration of seral stages fluctuated markedly over time as well, although there was considerable variation in the magnitude of variability among configuration metrics (Table~\ref{tab:fragclass_ocfwu} in Appendix~\ref{app:full-class-results}). Area-weighted patch and core area, patch density, mean similarity, and radius of gyration all exhibited high variability over time (Figures~\ref{fig:ocfwu_areaam}--\ref{fig:ocfwu_clumpy}). For the most part, the current landscape's values fall within the HRV, although the Mid--Closed and Late--Open patches usually fell outside of it. Patches and their cores are larger, more complex, and more numerous now compared to the simulated HRV. The current landscape also has more aggregated patches. 

% figures updated 2015-09-20
\begin{figure}[!htbp]
\centering
    \includegraphics[width=0.8\textwidth]{/Users/mmallek/Documents/Thesis/Plots/fragclass-bymetrics/HRV/OCFW_U-AREA_AM-boxplots.png}
  \caption{Fragstats class-level results for Oak-Conifer Forest and Woodland - Ultramafic and area-weighted mean patch area. Boxplot whiskers extend to the 5th and 95th percentile of the observed distribution. The thick grey bar denotes the metric value on the current landscape.}
  \label{fig:ocfwu_areaam}
\end{figure}


\begin{figure}[!htbp]
\centering
    \includegraphics[width=0.8\textwidth]{/Users/mmallek/Documents/Thesis/Plots/fragclass-bymetrics/HRV/OCFW_U-CORE_AM-boxplots.png}
  \caption{Fragstats class-level results for Oak-Conifer Forest and Woodland - Ultramafic and area-weighted mean core area. Boxplot whiskers extend to the 5th and 95th percentile of the observed distribution. The thick grey bar denotes the metric value on the current landscape.}
  \label{fig:ocfwu_coream}
\end{figure}


\begin{figure}[!htbp]
\centering
    \includegraphics[width=0.8\textwidth]{/Users/mmallek/Documents/Thesis/Plots/fragclass-bymetrics/HRV/OCFW_U-SHAPE_AM-boxplots.png}
  \caption{Fragstats class-level results for Oak-Conifer Forest and Woodland - Ultramafic and area-weighted mean shape index. Boxplot whiskers extend to the 5th and 95th percentile of the observed distribution. The thick grey bar denotes the metric value on the current landscape.}
  \label{fig:ocfwu_shapeam}
\end{figure}


\begin{figure}[!htbp]
\centering
    \includegraphics[width=0.8\textwidth]{/Users/mmallek/Documents/Thesis/Plots/fragclass-bymetrics/HRV/OCFW_U-CLUMPY-boxplots.png}
  \caption{Fragstats class-level results for Oak-Conifer Forest and Woodland - Ultramafic and clumpiness. Boxplot whiskers extend to the 5th and 95th percentile of the observed distribution. The thick grey bar denotes the metric value on the current landscape.}
  \label{fig:ocfwu_clumpy}
\end{figure}

%%%%%%%%%%%%%%%%%%%%%%%%%%%%%%%%%%%%%%%%%%%%%%%%%%%%%%%%%%%%%%%%%%%%%%%%%%%%%%%%%%%%%%%%%%%%%%%%
%%%%%%%%%%%%%%%%%%%%%%%%%%%%%%%%%%%%%%%%%%%%%%%%%%%%%%%%%%%%%%%%%%%%%%%%%%%%%%%%%%%%%%%%%%%%%%%%
%%%%%%%%%%%%%%%%%%%%%%%%%%%%%%%%%%%%%%%%%%%%%%%%%%%%%%%%%%%%%%%%%%%%%%%%%%%%%%%%%%%%%%%%%%%%%%%%
%%%%%%%%%%%%%%%%%%%%%%%%%%%%%%%%%%%%%%%%%%%%%%%%%%%%%%%%%%%%%%%%%%%%%%%%%%%%%%%%%%%%%%%%%%%%%%%%


%%%%%%%%%%%%%%%%%%%%%%%%%%%%%%%%%%%%%%%%%%%%%%%%%%%%%%%%%%%%%%%%%%%%%%%%%%%%%%%%%%%%%
%%%%%%%%%%%%%%%%%%%%%%%%%%%%%%%%%%%%%%%%%%%%%%%%%%%%%%%%%%%%%%%%%%%%%%%%%%%%%%%%%%%%

%% !TEX root = master.tex
\chapter{Analysis by Cover Type: Additional Results}
\label{app:covtype_analysis}

\todo{Alternative way to organize this would be to report all add'l results by cover type; now some like preturn and rotation are reported in the general 'more results' app.}The discussion that follows focuses on seven of the nine cover types found within the core project area that were treated as dynamic in the model and that occurred over an extent of at least 1000 ha in the project area. For each of these cover types, we briefly describe the simulated disturbance regime (i.e., spatial extent and distribution, frequency and temporal variability) associated with each relevant disturbance process, the vegetation dynamics resulting from the interplay between these disturbance processes and succession, and an examination of the cover type’s current departure from the simulated HRV. The cover types are presented in descending order by total area within the project landscape. Results for Sierran Mixed Conifer Mesic and Xeric can be found in Chapter~\ref{ch:covtype}


\section{Oak-Conifer Forest and Woodland} 
% figures updated 2015-09
\begin{figure}[!htbp]
  \centering
  \subfloat[][]{
    \centering
    \includegraphics[width=0.5\textwidth]{/Users/mmallek/Documents/Thesis/Plots/darea/hrv_ocfw.png}
    }%
  \subfloat[][]{
    \includegraphics[width=0.5\textwidth]{/Users/mmallek/Documents/Thesis/Plots/darea/hrv_hist_ocfw.png}
    }
  \caption{\small (a) Disturbance trajectory for Oak-Conifer Forest and Woodland. High mortality fire in dark blue; low mortality fire in light blue. (b) Histogram of disturbed hectares with density curve overlaid.} 
  \label{fig:darea_ocfw}
\end{figure}

Oak-Conifer Forest and Woodland (\textsc{ocfw})is the third most common cover type within the core project area, encompassing 23,279 ha and comprising roughly 13\% of the project area. The frequency and extent of simulated wildfires in oak-conifer forests and woodlands varied markedly across the landscape (Figure~\ref{fig:darea_ocfw}). Wildfire was quite prevalent in this cover type. We summarize the disturbance regime in Tables~\ref{tab:darea_ocfw} and \ref{tab:darea_atleast_ocfw}.

%Wildfire was quite prevalent in this cover type. At least some area burned every five years, and at least 10\% of the cover type burned in about 75\% of the simulated timesteps. The median amount of land burned during one timestep of the simulation was 16\%; over 25\% burned every 17 years. Wildfires occurred across more than 50\% of the cover type about once every 77 years. The maximum extent burned within the cover type was about 92\% (21,400 ha). High mortality wildfire was about one-third as common as low mortality. %
%
%Under this wildfire regime, the grand mean return interval between fires (of any mortality level) varied widely from 17 years to over 500 years, with a median of 23 years (Figure~\ref{fig:preturn_ocfw}). As expected, median return interval and rotation values are much shorter for this cover type as compared to the mixed evergreen forests, which occupy similar elevations. Like those forests, however, low mortality fire was the dominant type. Oak-conifer forests and woodlands had a low mortality fire rotation of 30 years and a high mortality fire rotation of 115 years. %
%
%In general, return intervals and canopy cover varied spatially across the forest and decreased with increasing TPI, reflecting our parameterization, which was based on the theory that higher, more southerly aspects are drier and more susceptible to fires. Canopy cover decreased by about 6\% when comparing minimum to maximum TPI, from an average of 49.6\% to an average of 46.5\% (Table~\ref{tab:tpi_cc}). %
%
%Finally, when stands of oak-conifer forests and woodland were adjacent to cover types with longer return intervals, they also exhibited longer return intervals, reflecting the importance of landscape context on fire regimes.
%%%

% updated 2015-09-28
\begin{table}[!htbp]
\small
\centering
\caption{ Disturbed area summary statistics for Oak-Conifer Forest and Woodland. Proportions shown are relative to the total area of Oak-Conifer Forest and Woodland.}
\label{tab:darea_ocfw}
\begin{tabular}{@{}llll@{}}
\toprule
\textbf{\begin{tabular}[c]{@{}l@{}}Summary Statistic \\ (disturbed area/timestep)\end{tabular}} & \textbf{Low Mortality} & \textbf{High Mortality} & \textbf{Any Mortality} \\ \midrule
$5^{\text{th}}$ percentile         & 3.39  & 0.70  & 4.35   \\
$50^{\text{th}}$ percentile        & 13.92 & 3.63  & 17.82  \\
$95^{\text{th}}$ percentile        & 45.42 & 13.61 & 58.63  \\
Mean                               & 17.61 & 4.78  & 22.39  \\
\textbf{Fire Rotation} & 28       & 105       & 22 \\ \bottomrule
\end{tabular}
\end{table}

% updated 2015-09-28
\begin{table}[!htbp]
\small
\centering
\caption{Summary of disturbed area in terms of proportion of the amount of \textsc{ocfw} burned (any level of mortality) during the simulation (after the equilibration period). For each benchmark proportion of the landscape, we list the number of timesteps during the simulation when that extent burned, the proportion of timesteps that represents, the interval in timesteps calculated from the proportion (i.e. approximately every 4 timesteps, at least 25\% of the landscape burned.), and the interval in years calculated from the interval in timesteps (5 years to a timestep).}
\label{tab:darea_atleast_ocfw}
\begin{tabular}{@{}lllll@{}}
                        & at least 1\% & at least 10\% & at least 25\% & at least 50\% \\ \midrule
Number of timesteps     & 460          & 344           & 156           & 42           \\
Proportion of timesteps & 1.00         & 0.75          & 0.34          & 0.09         \\
Interval (timesteps)    & 1.00         & 1.34          & 2.96          & 10.98        \\
Interval (years)        & 5.01         & 6.70          & 14.78         & 54.88        \\ \bottomrule
\end{tabular}
\end{table}


The age structure and dynamics of oak-conifer forests and woodlands illustrates the interaction between disturbance and succession processes. We focus our analysis on the 5$^{\text{th}}$ to 95$^{\text{th}}$ percentile range of variability for our simulation (excluding the equilibration period). %

The distribution of area among stand conditions within oak-conifer forests and woodlands fluctuated considerably over time, as expected (Figure~\ref{fig:covcond_ocfw}). %For example, the percentage of oak-conifer forests and woodlands in the Late Development - Closed condition varied from 7\% to 35\%, reflecting the dynamic nature of this cover type (Table~\ref{tab:covcond1}). 
Surprisingly for a cover type in which fuels are the largest contributor to disturbance and fire is relatively frequent, the late development, open canopy conditions were relatively uncommon, though more so than on the current landscape.

The seral-stage distribution appeared to be in dynamic equilibrium (i.e., the percentage in each stand condition varied about a stable mean). Our calculated current seral-stage distribution was never observed under the simulated HRV (Table~\ref{tab:covcond1}). The most notable departure was the shift from Mid Development conditions, which are dominant in the current landscape, to Late Development conditions, which are almost nonexistent on the current landscape. The current proportions of all Late Development canopy cover levels are lower than at any point during the HRV.  The Early Development and Mid Development - Moderate conditions are within the HRV, but the other five stages are completely departed from the HRV.

% figures updated 2015-09
\begin{figure}[!htbp]
  \centering
    \includegraphics[width=\textwidth]{/Users/mmallek/Documents/Thesis/Plots/covcond-bycover/OCFW-HRV-boxplots-.png}
  \caption{Boxplots showing the range of variability for each seral stage over the course of the simulation, excluding the equilibration period. Boxplots were modified so that whiskers extend from the $5^{\text{th}} - 95^{\text{th}}$ percentiles of the observed results. Thick black bars in line with the boxplots denote the current proportion of mesic mixed conifer forests in a given seral stage.} 
  \label{fig:covcond_ocfw_boxplots}
\end{figure}

The spatial configuration of stand conditions fluctuated markedly over time as well, although there was considerable variation in the magnitude of variability among configuration metrics (Appendix \ref{sec:full-class-results}). Area-weighted patch and core area exhibited the greatest variability over time. Because Late Development conditions across canopy cover are nearly absent from the current landscape , configuration metrics consistently differ between current conditions and the simulated HRV. While some conditions and metrics fall completely outside the HRV, others are well within it. The HRV results for class-level metrics are consistent for six of the seven seral stages, in the sense of their deviation from current conditions (Mid Development - Closed is the outlier). For example, patches are currently smaller, with less core area and geometric complexity, compared to the simulated period. Early development and middle development, open canopy patches tended to be less aggregated during the HRV, while the other condition classes were more aggregated. Only late development moderate and open canopy classes were outside the HRV for the textsc{clumpy} metric, however.


% figures updated 2015-09-20
\begin{figure}[!htbp]
\centering
    \includegraphics[width=0.8\textwidth]{/Users/mmallek/Documents/Thesis/Plots/fragclass-bymetrics/HRV/OCFW-AREA_AM-boxplots.png}
  \caption{Fragstats class-level results for Oak-Conifer Forest and Woodland and area-weighted mean patch area. Boxplot whiskers extend to the 5th and 95th percentile of the observed distribution. The thick grey bar denotes the metric value on the current landscape.}
  \label{fig:ocfw_areaam}
\end{figure}


\begin{figure}[!htbp]
\centering
    \includegraphics[width=0.8\textwidth]{/Users/mmallek/Documents/Thesis/Plots/fragclass-bymetrics/HRV/OCFW-CORE_AM-boxplots.png}
  \caption{Fragstats class-level results for Oak-Conifer Forest and Woodland and area-weighted mean core area. Boxplot whiskers extend to the 5th and 95th percentile of the observed distribution. The thick grey bar denotes the metric value on the current landscape.}
  \label{fig:ocfw_coream}
\end{figure}


\begin{figure}[!htbp]
\centering
    \includegraphics[width=0.8\textwidth]{/Users/mmallek/Documents/Thesis/Plots/fragclass-bymetrics/HRV/OCFW-SHAPE_AM-boxplots.png}
  \caption{Fragstats class-level results for Oak-Conifer Forest and Woodland and area-weighted mean shape index. Boxplot whiskers extend to the 5th and 95th percentile of the observed distribution. The thick grey bar denotes the metric value on the current landscape.}
  \label{fig:ocfw_shapeam}
\end{figure}


\begin{figure}[!htbp]
\centering
    \includegraphics[width=0.8\textwidth]{/Users/mmallek/Documents/Thesis/Plots/fragclass-bymetrics/HRV/OCFW-CLUMPY-boxplots.png}
  \caption{Fragstats class-level results for Oak-Conifer Forest and Woodland and clumpiness. Boxplot whiskers extend to the 5th and 95th percentile of the observed distribution. The thick grey bar denotes the metric value on the current landscape.}
  \label{fig:ocfw_clumpy}
\end{figure}


%%%%%%%%%%%%%%%%%%%%%%%%%%%%%%%%%%%%%%%%%%%%%%%%%%%%%%%%%%%%%%%%%%%%%%%%%%%%%
%%%%%%%%%%%%%%%%%%%%%%%%%%%%%%%%%%%%%%%%%%%%%%%%%%%%%%%%%%%%%%%%%%%%%%%%%%%%%
%%%%%%%%%%%%%%%%%%%%%%%%%%%%%%%%%%%%%%%%%%%%%%%%%%%%%%%%%%%%%%%%%%%%%%%%%%%%%
%%%%%%%%%%%%%%%%%%%%%%%%%%%%%%%%%%%%%%%%%%%%%%%%%%%%%%%%%%%%%%%%%%%%%%%%%%%%%
%%%%%%%%%%%%%%%%%%%%%%%%%%%%%%%%%%%%%%%%%%%%%%%%%%%%%%%%%%%%%%%%%%%%%%%%%%%%%

\clearpage
\section{Red Fir - Mesic} 
% figures updated 2015-09
\begin{figure}[!htbp]
  \centering
  \subfloat[][]{
    \centering
    \includegraphics[width=0.5\textwidth]{/Users/mmallek/Documents/Thesis/Plots/darea/hrv_rfrm.png}
    }%
  \subfloat[][]{
    \includegraphics[width=0.5\textwidth]{/Users/mmallek/Documents/Thesis/Plots/darea/hrv_hist_rfrm.png}
    }
  \caption{\small (a) Disturbance trajectory for Red Fir - Mesic. High mortality fire in dark blue; low mortality fire in light blue. (b) Histogram of disturbed hectares with density curve overlaid.} 
  \label{fig:darea_rfrm}
\end{figure}

Red Fir - Mesic (\textsc{rfr\_m})is a somewhat common cover type within the core project area, encompassing 8,563 ha and comprising roughly 5\% of the project area. Wildfire was fairly common in this cover type. The frequency and extent of simulated wildfires in mesic red fir forests varied markedly across the landscape (Figure~\ref{fig:darea_rfrm}). We summarize the disturbance regime in Tables~\ref{tab:darea_rfrm} and \ref{tab:darea_atleast_rfrm}.

%Wildfire was fairly common in this cover type. At least some area in mesic red fir forests burned during each timestep in the simulation, although in 20\% of these less than 1\% of the cover type (86 ha) burned. When fires did occur they were seldom very large. At least 10\% of these forests burned about once every 19 years, and fires burned over 25\% of the cover type once in 66 years. However, over wildfires spread over 50\% of the cover type only once in 768 years. The maximum extent burned was 70\% (5,600 ha). The median and mean area burned was 5\% and 8\%, respectively. Low mortality fire was roughly 1.5 times as likely to occur as high mortality fire. %

%Under this wildfire regime, the grand mean return interval between fires (of any mortality level) varied widely from 20 years to over 500 years, with a median of 68 years (Figure~\ref{fig:preturn_rfrm}). Median return interval and rotation values tend to be longer in red fir forests compared to sierran mixed conifer forests, because their higher elevation corresponds to cooler and moister conditions. Mesic red fir forests had a low mortality fire rotation of 102 years and a high mortality fire rotation of 160 years (Table~\ref{tab:darea_rfrm}); neither high nor low mortality fires dominate the cover type.  %

%In general, return intervals and canopy cover varied spatially across the forest and decreased with increasing TPI, reflecting our parameterization, which was based on the theory that higher, more southerly aspects are drier and more susceptible to fires. Canopy cover decreased by about 10\% when comparing minimum to maximum TPI, from an average of 65\% to an average of 59\% (Table~\ref{tab:tpi_cc}).  %

%Finally, when stands of mesic red fir forests were adjacent to cover types with much shorter or longer return intervals, they also exhibited a directional shift in local return intervals towards that of the adjacent type, reflecting the importance of landscape context on fire regimes.

% updated 2015-09-28
\begin{table}[!htbp]
\small
\centering
\caption{\small Disturbed area summary statistics for Red Fir - Mesic. Proportions shown are relative to the total area of Red Fir - Mesic.}
\label{tab:darea_rfrm}
\begin{tabular}{@{}llll@{}}
\toprule
\textbf{\begin{tabular}[c]{@{}l@{}}Summary Statistic \\ (disturbed area/timestep)\end{tabular}} & \textbf{Low Mortality} & \textbf{High Mortality} & \textbf{Any Mortality} \\ \midrule
$5^{\text{th}}$ percentile         & 0.20  & 0.06  & 0.30  \\
$50^{\text{th}}$ percentile        & 2.70  & 1.53  & 4.31  \\
$95^{\text{th}}$ percentile        & 18.75 & 12.74 & 31.17 \\
Mean                               & 4.80  & 3.14  & 7.94  \\
\textbf{Fire Rotation} & 104      & 159       & 63 \\ \bottomrule
\end{tabular}
\end{table}


% updated 2015-09-28
\begin{table}[!htbp]
\small
\centering
\caption{Summary of disturbed area in terms of proportion of the amount of \textsc{rfr\_m} burned (any level of mortality) during the simulation (after the equilibration period). For each benchmark proportion of the landscape, we list the number of timesteps during the simulation when that extent burned, the proportion of timesteps that represents, the interval in timesteps calculated from the proportion (i.e. approximately every 4 timesteps, at least 25\% of the landscape burned.), and the interval in years calculated from the interval in timesteps (5 years to a timestep).}
\label{tab:darea_atleast_rfrm}
\begin{tabular}{@{}lllll@{}}
                        & at least 1\% & at least 10\% & at least 25\% & at least 50\% \\ \midrule
Number of timesteps     & 380          & 116           & 36            & 1          \\
Proportion of timesteps & 0.82         & 0.25          & 0.08          & 0          \\
Interval (timesteps)    & 1.21         & 3.97          & 12.81         & 461        \\
Interval (years)        & 6.07         & 19.87         & 64.03         & 2305      \\ \bottomrule
\end{tabular}
\end{table}
   
%%%
The age structure and dynamics of mesic red fir forests illustrates the interaction between disturbance and succession processes. We focus our analysis on the 5$^{\text{th}}$ to 95$^{\text{th}}$ percentile range of variability for our simulation (excluding the equilibration period). %

The distribution of area among stand conditions within mesic red fir forests fluctuated over time, but the moderate and open canopy cover seral stages are remarkable for having an extremely small range of variability (Figure~\ref{fig:covcond_rfrm}). As expected for mesic red fir forests (Appendix~\ref{sec:covertypedesc}), closed canopy conditions predominated. Early Development, which includes post-fire chaparral fields, was the next most extensive cover type. %

The seral-stage distribution appeared to be in dynamic equilibrium (i.e., the percentage in each stand condition varied about a stable mean). Our calculated current seral-stage distribution was never observed under the simulated HRV (Table~\ref{tab:covcond2}). The most notable departure was a shift from moderate canopy cover to closed canopy cover. Current levels of moderate canopy cover are much higher, and current levels of closed canopy cover much lower, than during the simulated HRV. Early Development and Late Development - Open are both somewhat departed from the HRV. The other five condition types are completely departed from the HRV. 

% figures updated 2015-09
\begin{figure}[!htbp]
  \centering
    \includegraphics[width=\textwidth]{/Users/mmallek/Documents/Thesis/Plots/covcond-bycover/RFRM-HRV-boxplots-.png}
  \caption{Boxplots showing the range of variability for each seral stage over the course of the simulation, excluding the equilibration period. Boxplots were modified so that whiskers extend from the $5^{\text{th}} - 95^{\text{th}}$ percentiles of the observed results. Thick black bars in line with the boxplots denote the current proportion of mesic mixed conifer forests in a given seral stage.} 
  \label{fig:covcond_RFRM_boxplots}
\end{figure}

The spatial configuration of stand conditions fluctuated markedly over time as well, although there was considerable variation in the magnitude of variability among configuration metrics (Appendix \ref{sec:full-class-results}). Several metrics exhibited high variability over time, including the area-weighted patch and core area, edge and patch density, radius of gyration, and contrast-weighted edge density. The narrow range of variability observed in some seral stages is repeated in the configuration metrics. In general, current values for the class metrics are often completely outside or near the extremes of the simulated HRV. The direction of the departure depends on condition class. The early development and closed canopy conditions tend to be smaller in both area and core area, less aggregated, and less geometrically complex now than during the HRV. In contrast, the moderate and open canopy conditions tend to be larger, with more core area, more aggregation, and more complexity now than during the HRV. The fact that the closed canopy conditions dominated the HRV cover-condition distribution explains this divergence.

% figures updated 2015-09-20
\begin{figure}[!htbp]
\centering
    \includegraphics[width=0.8\textwidth]{/Users/mmallek/Documents/Thesis/Plots/fragclass-bymetrics/HRV/RFR_M-AREA_AM-boxplots.png}
  \caption{Fragstats class-level results for Red Fir - Mesic and area-weighted mean patch area. Boxplot whiskers extend to the 5th and 95th percentile of the observed distribution. The thick grey bar denotes the metric value on the current landscape.}
  \label{fig:rfrm_areaam}
\end{figure}


\begin{figure}[!htbp]
\centering
    \includegraphics[width=0.8\textwidth]{/Users/mmallek/Documents/Thesis/Plots/fragclass-bymetrics/HRV/RFR_M-CORE_AM-boxplots.png}
  \caption{Fragstats class-level results for Red Fir - Mesic and area-weighted mean core area. Boxplot whiskers extend to the 5th and 95th percentile of the observed distribution. The thick grey bar denotes the metric value on the current landscape.}
  \label{fig:rfrm_coream}
\end{figure}


\begin{figure}[!htbp]
\centering
    \includegraphics[width=0.8\textwidth]{/Users/mmallek/Documents/Thesis/Plots/fragclass-bymetrics/HRV/RFR_M-SHAPE_AM-boxplots.png}
  \caption{Fragstats class-level results for Red Fir - Mesic and area-weighted mean shape index. Boxplot whiskers extend to the 5th and 95th percentile of the observed distribution. The thick grey bar denotes the metric value on the current landscape.}
  \label{fig:rfrm_shapeam}
\end{figure}


\begin{figure}[!htbp]
\centering
    \includegraphics[width=0.8\textwidth]{/Users/mmallek/Documents/Thesis/Plots/fragclass-bymetrics/HRV/RFR_M-CLUMPY-boxplots.png}
  \caption{Fragstats class-level results for Red Fir - Mesic and clumpiness. Boxplot whiskers extend to the 5th and 95th percentile of the observed distribution. The thick grey bar denotes the metric value on the current landscape.}
  \label{fig:rfrm_clumpy}
\end{figure}


%%%%%%%%%%%%%%%%%%%%%%%%%%%%%%%%%%%%%%%%%%%%%%%%%%%%%%%%%%%%%%%%%%%%%%%%%%%%%
%%%%%%%%%%%%%%%%%%%%%%%%%%%%%%%%%%%%%%%%%%%%%%%%%%%%%%%%%%%%%%%%%%%%%%%%%%%%%
%%%%%%%%%%%%%%%%%%%%%%%%%%%%%%%%%%%%%%%%%%%%%%%%%%%%%%%%%%%%%%%%%%%%%%%%%%%%%
%%%%%%%%%%%%%%%%%%%%%%%%%%%%%%%%%%%%%%%%%%%%%%%%%%%%%%%%%%%%%%%%%%%%%%%%%%%%%
%%%%%%%%%%%%%%%%%%%%%%%%%%%%%%%%%%%%%%%%%%%%%%%%%%%%%%%%%%%%%%%%%%%%%%%%%%%%%
\clearpage
\section{Red Fir - Xeric} 

% figures updated 2015-09
\begin{figure}[!htbp]
  \centering
  \subfloat[][]{
    \centering
    \includegraphics[width=0.5\textwidth]{/Users/mmallek/Documents/Thesis/Plots/darea/hrv_rfrx.png}
    }%
  \subfloat[][]{
    \includegraphics[width=0.5\textwidth]{/Users/mmallek/Documents/Thesis/Plots/darea/hrv_hist_rfrx.png}
    }
  \caption{\small (a) Disturbance trajectory for Red Fir - Xeric. High mortality fire in dark blue; low mortality fire in light blue. (b) Histogram of disturbed hectares with density curve overlaid.} 
  \label{fig:darea_rfrx}
\end{figure}

Red Fir - Xeric (\textsc{rfr\_x})is a somewhat common cover type within the core project area, encompassing 7,493 ha and comprising roughly 5\% of the project area. Wildfire was fairly common in this cover type, and occurred more frequently on average than in mesic red fir forests.
The frequency and extent of simulated wildfires in xeric red fir forests varied markedly across the landscape (Figure~\ref{fig:darea_rfrx}). We summarize the disturbance regime in Tables~\ref{tab:darea_rfrx} and \ref{tab:darea_atleast_rfrx}.

%Wildfire was fairly common in this cover type, and occurred more frequently on average than in mesic red fir forests. Xeric red fir forests burned during every timestep, and during a typical five-year period 7--13\% of the cover type burned. The disturbed area per timestep varied dramatically, from a minimum of 0.04\% to a maximum of 83\% (about 6,250 ha). More than 10\% of xeric red fir forest extent burned at about a 12 year interval, more than 25\% burned at a 33 year interval, and fires burning over 50\% of the cover type occured once in 110 years. Low mortality fire was twice as likely as high mortality fire. %

%Under this wildfire regime, the grand mean return interval between fires (of any mortality level) varied widely from 19 years to over 500 years, with a median of 41 years (Figure~\ref{fig:preturn_rfrx}). Median return interval and rotation values tend to be longer in red fir forests compared to sierran mixed conifer forests, because their higher elevation corresponds to cooler and moister conditions. Xeric red fir forests had a low mortality fire rotation of 62 years and a high mortality fire rotation of 108 years (Table~\ref{tab:darea_rfrx}). These values are shorter in xeric red fir compared to mesic red fir, but much longer than oak-conifer or sierran mixed conifer forests, which occur at lower elevations. %

%In general, return intervals and canopy cover varied spatially across the forest and decreased with increasing TPI, reflecting our parameterization, which was based on the theory that higher, more southerly aspects are drier and more susceptible to fires. Canopy cover decreased by about 32\% when comparing minimum to maximum TPI, from an average of 49\% to an average of 33\% (Table~\ref{tab:tpi_cc}). %

%Finally, when stands of xeric red fir forests were adjacent to cover types with much shorter or longer return intervals, they also exhibited a directional shift in local return intervals towards that of the adjacent type, reflecting the importance of landscape context on fire regimes.

% updated 2015-09-28
\begin{table}[!htbp]
\small
\centering
\caption{Disturbed area summary statistics for Red Fir - Xeric. Proportions shown are relative to the total area of Red Fir - Xeric.}
\label{tab:darea_rfrx}
\begin{tabular}{@{}llll@{}}
\toprule
\textbf{\begin{tabular}[c]{@{}l@{}}Summary Statistic \\ (disturbed area/timestep)\end{tabular}} & \textbf{Low Mortality} & \textbf{High Mortality} & \textbf{Any Mortality} \\ \midrule
$5^{\text{th}}$ percentile         & 0.34  & 0.17  & 0.56  \\ 
$50^{\text{th}}$ percentile        & 4.47  & 2.77  & 7.21  \\ 
$95^{\text{th}}$ percentile        & 30.60 & 18.39 & 47.14 \\ 
Mean                               & 8.29  & 4.96  & 13.25  \\
\textbf{Fire Rotation} & 60       & 101       & 38  \\ \bottomrule
\end{tabular}
\end{table}

\begin{table}[!htbp]
\small
\centering
\caption{Summary of disturbed area in terms of proportion of the amount of \textsc{rfr\_x} burned (any level of mortality) during the simulation (after the equilibration period). For each benchmark proportion of the landscape, we list the number of timesteps during the simulation when that extent burned, the proportion of timesteps that represents, the interval in timesteps calculated from the proportion (i.e. approximately every 4 timesteps, at least 25\% of the landscape burned.), and the interval in years calculated from the interval in timesteps (5 years to a timestep).}
\label{tab:darea_atleast_rfrx}
\begin{tabular}{@{}lllll@{}}
                        & at least 1\% & at least 10\% & at least 25\% & at least 50\% \\ \midrule
Number of timesteps     & 422          & 188           & 77            & 21            \\
Proportion of timesteps & 0.92         & 0.41          & 0.17          & 0.05          \\
Interval (timesteps)    & 1.09         & 2.45          & 5.99          & 21.95         \\
Interval (years)        & 5.46         & 12.26         & 29.94         & 109.76       \\ \bottomrule
\end{tabular}
\end{table}


%%%
The age structure and dynamics of xeric red fir forests illustrates the interaction between disturbance and succession processes. We focus our analysis on the 5$^{\text{th}}$ to 95$^{\text{th}}$ percentile range of variability for our simulation (excluding the equilibration period). %

The distribution of area among stand conditions within xeric red fir forests fluctuated over time, but less dramatically than many other cover types (Figure~\ref{fig:covcond_rfrx}). Interestingly, although open canopy conditions dominated during middle development, the distribution of the three late development canopy conditions was roughly equal. This shift towards higher canopy closure may be due to an increasing resilience to wildfire disturbances by stands of that age: wildfires may burn the understory without significantly affecting overstory canopy cover. Early Development, which includes post-fire chaparral fields, was the single most extensive cover type. The current proportion of Early Development is somewhat departed from the simulated HRV. %

The seral-stage distribution appeared to be in dynamic equilibrium. Our calculated current seral-stage distribution was never observed under the simulated HRV (Table~\ref{tab:covcond2}). Although the late closed stage is not currently departed from the HRV, and the early development stage is moderately departed, the other stages are completely departed from the HRV.

% figures updated 2015-09
\begin{figure}[!htbp]
  \centering
    \includegraphics[width=\textwidth]{/Users/mmallek/Documents/Thesis/Plots/covcond-bycover/RFRX-HRV-boxplots-.png}
  \caption{Boxplots showing the range of variability for each seral stage over the course of the simulation, excluding the equilibration period. Boxplots were modified so that whiskers extend from the $5^{\text{th}} - 95^{\text{th}}$ percentiles of the observed results. Thick black bars in line with the boxplots denote the current proportion of mesic mixed conifer forests in a given seral stage.} 
  \label{fig:covcond_rfrx_boxplots}
\end{figure}

The spatial configuration of seral stages fluctuated markedly over time as well, although there was considerable variation in the magnitude of variability among configuration metrics (Appendix \ref{sec:full-class-results}). In general most seral stages were not departed or moderately departed from the HRV, across metrics. However, middle development, open canopy stages were often completely departed from the HRV, or otherwise moderately departed. These patches were larger, more aggregated, more geometrically complex, and contained more core area during the simulation than in the current landscape.

In general, current values for the class metrics are often completely outside or near the extremes of the simulated HRV. The direction of the departure depends on condition class. The early development and closed canopy conditions tend to be smaller in both area and core area, less aggregated, and less geometrically complex now than during the HRV. In contrast, the moderate and open canopy conditions tend to be larger, with more core area, more aggregation, and more complexity now than during the HRV. The fact that the closed canopy conditions dominated the HRV cover-condition distribution explains this divergence.

Specifically, current patches tend to be smaller in both area and core area and more numerous, with less complex geometries and more edge than patches during the simulated HRV.


% figures updated 2015-09-20
\begin{figure}[!htbp]
\centering
    \includegraphics[width=0.8\textwidth]{/Users/mmallek/Documents/Thesis/Plots/fragclass-bymetrics/HRV/RFR_X-AREA_AM-boxplots.png}
  \caption{Fragstats class-level results for Red Fir - Xeric and area-weighted mean patch area. Boxplot whiskers extend to the 5th and 95th percentile of the observed distribution. The thick grey bar denotes the metric value on the current landscape.}
  \label{fig:rfrx_areaam}
\end{figure}


\begin{figure}[!htbp]
\centering
    \includegraphics[width=0.8\textwidth]{/Users/mmallek/Documents/Thesis/Plots/fragclass-bymetrics/HRV/RFR_X-CORE_AM-boxplots.png}
  \caption{Fragstats class-level results for Red Fir - Xeric and area-weighted mean core area. Boxplot whiskers extend to the 5th and 95th percentile of the observed distribution. The thick grey bar denotes the metric value on the current landscape.}
  \label{fig:rfrx_coream}
\end{figure}


\begin{figure}[!htbp]
\centering
    \includegraphics[width=0.8\textwidth]{/Users/mmallek/Documents/Thesis/Plots/fragclass-bymetrics/HRV/RFR_X-SHAPE_AM-boxplots.png}
  \caption{Fragstats class-level results for Red Fir - Xeric and area-weighted mean shape index. Boxplot whiskers extend to the 5th and 95th percentile of the observed distribution. The thick grey bar denotes the metric value on the current landscape.}
  \label{fig:rfrx_shapeam}
\end{figure}


\begin{figure}[!htbp]
\centering
    \includegraphics[width=0.8\textwidth]{/Users/mmallek/Documents/Thesis/Plots/fragclass-bymetrics/HRV/RFR_X-CLUMPY-boxplots.png}
  \caption{Fragstats class-level results for Red Fir - Xeric and clumpiness. Boxplot whiskers extend to the 5th and 95th percentile of the observed distribution. The thick grey bar denotes the metric value on the current landscape.}
  \label{fig:rfrx_clumpy}
\end{figure}



%%%%%%%%%%%%%%%%%%%%%%%%%%%%%%%%%%%%%%%%%%%%%%%%%%%%%%%%%%%%%%%%%%%%%%%%%%%%%
%%%%%%%%%%%%%%%%%%%%%%%%%%%%%%%%%%%%%%%%%%%%%%%%%%%%%%%%%%%%%%%%%%%%%%%%%%%%%
%%%%%%%%%%%%%%%%%%%%%%%%%%%%%%%%%%%%%%%%%%%%%%%%%%%%%%%%%%%%%%%%%%%%%%%%%%%%%
%%%%%%%%%%%%%%%%%%%%%%%%%%%%%%%%%%%%%%%%%%%%%%%%%%%%%%%%%%%%%%%%%%%%%%%%%%%%%
%%%%%%%%%%%%%%%%%%%%%%%%%%%%%%%%%%%%%%%%%%%%%%%%%%%%%%%%%%%%%%%%%%%%%%%%%%%%%

\clearpage
\section{Mixed Evergreen - Mesic} 
% figures updated 2015-09
\begin{figure}[!htbp]
  \centering
  \subfloat[][]{
    \centering
    \includegraphics[width=0.5\textwidth]{/Users/mmallek/Documents/Thesis/Plots/darea/hrv_megm.png}
    }%
  \subfloat[][]{
    \includegraphics[width=0.5\textwidth]{/Users/mmallek/Documents/Thesis/Plots/darea/hrv_hist_megm.png}
    }
  \caption{(a) \small Disturbance trajectory for Mixed Evergreen - Mesic. High mortality fire in dark blue; low mortality fire in light blue. (b) Histogram of disturbed hectares with density curve overlaid.} 
  \label{fig:darea_megm}
\end{figure}

Mixed Evergreen - Mesic (\textsc{meg\_m}) is a somewhat common cover type within the core project area, encompassing 7,273 ha and comprising roughly 4\% of the project area. Wildfire was prevalent in mesic mixed evergreen forest. The frequency and extent of simulated wildfires  varied markedly across the landscape (Figure~\ref{fig:darea_megm}). We summarize the disturbance regime in Tables~\ref{tab:darea_megm} and \ref{tab:darea_atleast_megm}.

%While mesic mixed evergreen forests always experienced some wildfire during the simulation, during a typical five year period wildfires burned a small portion of the cover type, but the impacts were mostly low mortality. Low mortality fires were eight times as frequent as high mortality fires. Less than 1\% of the cover type (72 ha) burned during 9\% of the timesteps. Much more common were timesteps in which over 10\% of the landscape burned, which occurred about once in 17 years. The median proportion burned was 6\%. Seldom did large extents burn, at any mortality level, although roughly once every 68 years, more than 25\% of the cover type burned. We never observed a timestep during which more than 50\% of these forests experienced wildfire. The maximum extent burned within the cover type was about 48\% (about 3,500 ha). %

%Under this wildfire regime, the grand mean return interval between fires (of any mortality level) varied widely from 19 years to more than 500 years, with a median of 62 years (Figure~\ref{fig:preturn_megm}). The median return interval and rotation values were influenced primarily by the dominant fire type, low mortality. Mesic mixed evergreen forests had a low mortality fire rotation of 63 years and a high mortality fire rotation of 534 years. %

% updated 2015-09-28
\begin{table}[!htbp]
\small
\centering
\caption{Disturbed area summary statistics for Mixed Evergreen - Mesic. Proportions shown are relative to the total area of Mixed Evergreen - Mesic.}
\label{tab:darea_megm}
  \begin{tabular}{@{}llll@{}} 
  \toprule
  \textbf{\begin{tabular}[c]{@{}l@{}}Summary Statistic \\ (disturbed area/timestep)\end{tabular}} & \textbf{\begin{tabular}[c]{@{}l@{}}Low  Mortality\end{tabular}} & \textbf{\begin{tabular}[c]{@{}l@{}}High  Mortality\end{tabular}} & \textbf{\begin{tabular}[c]{@{}l@{}}Any  Mortality\end{tabular}} \\ \midrule
$5^{\text{th}}$ percentile    &  0.59          & 0.04          & 0.61     \\
$50^{\text{th}}$ percentile   &  5.13          & 0.54          & 5.84     \\
$95^{\text{th}}$ percentile   &  27.91         & 3.75          & 31.76    \\
  Mean                        &  8.62          & 1.01          & 9.63     \\
  \textbf{Fire Rotation}  & 58       & 493       & 52 \\  \bottomrule
  \end{tabular}
\end{table}
       
% updated 2015-09-28
\begin{table}[!htbp]
\small
\centering
\caption{Summary of disturbed area in terms of proportion of the amount of \textsc{meg\_m} burned (any level of mortality) during the simulation (after the equilibration period). For each benchmark proportion of the landscape, we list the number of timesteps during the simulation when that extent burned, the proportion of timesteps that represents, the interval in timesteps calculated from the proportion (i.e. approximately every 4 timesteps, at least 25\% of the landscape burned.), and the interval in years calculated from the interval in timesteps (5 years to a timestep).}
\label{tab:darea_atleast_megm}
\begin{tabular}{@{}lllll@{}}
                        & at least 1\% & at least 10\% & at least 25\% & at least 50\% \\ \midrule
Number of timesteps     & 422          & 155           & 43            & 4             \\
Proportion of timesteps & 0.92         & 0.34          & 0.09          & 0.01          \\
Interval (timesteps)    & 1.09         & 2.97          & 10.72         & 115.25        \\
Interval (years)        & 5.46         & 14.87         & 53.60         & 576.25       \\ \bottomrule
\end{tabular}
\end{table}


%In general, return intervals and canopy cover varied spatially across the forest and increased slightly with increasing TPI, an unexpected result. This is likely related to the fact that during the simulated HRV, the vast majority of these stands were in closed canopy conditions, regardless of topographic position. Canopy cover increased by about 1.5\% when comparing minimum to maximum TPI (Table~\ref{tab:tpi_cc}). %

%Finally, when stands of mesic mixed evergreen forests were adjacent to cover types with shorter return intervals, they also exhibited shorter return intervals, reflecting the importance of landscape context on fire regimes.

%%%
The age structure and dynamics of mesic mixed evergreen forest illustrates the interaction between disturbance and succession processes. We focus our analysis on the 5$^{\text{th}}$ to 95$^{\text{th}}$ percentile range of variability for our simulation (excluding the equilibration period). %

The distribution of area among stand conditions within mesic mixed evergreen forest fluctuated very little over time (Figure~\ref{fig:covcond_megm}). Because high mortality fire is very rare in this cover type, and the time to reaching a Late Development stage is relatively short (Appendix \ref{sec:covertypedesc}), the vast majority of the cover type's extent was in the Late Development - Closed stage during the simulation (Table~\ref{tab:covcond1}). %

The seral-stage distribution appeared to be in dynamic equilibrium (i.e., the percentage in each stand condition varied about a stable mean). The most notable departure was the shift from Mid Development to Late Development condition classes. About 52\% of the current landscape is comprised of the mesic mixed evergreen forest in mid development conditions, but the late development conditions were always dominant under the simulated HRV. 

% figures updated 2015-09
\begin{figure}[!htbp]
  \centering
    \includegraphics[width=\textwidth]{/Users/mmallek/Documents/Thesis/Plots/covcond-bycover/MEGM-HRV-boxplots-.png}
  \caption{Boxplots showing the range of variability for each seral stage over the course of the simulation, excluding the equilibration period. Boxplots were modified so that whiskers extend from the $5^{\text{th}} - 95^{\text{th}}$ percentiles of the observed results. Thick black bars in line with the boxplots denote the current proportion of mesic mixed conifer forests in a given seral stage.} 
  \label{fig:covcond_megm_boxplots}
\end{figure}

The spatial configuration of stand conditions fluctuated markedly over time as well, although there was considerable variation in the magnitude of variability among configuration metrics (Appendix \ref{sec:full-class-results}). Because the landscape is so dominated by the Late Development - Closed and Moderate conditions, we focus on the configuration metrics for these classes. In general, the current landscape contains fewer, smaller, and more clumped patches than existed under the simulated HRV. Current patches in Late Development - Closed are less geometrically complex and have less area in cores than during the simulated HRV.


% figures updated 2015-09-20
\begin{figure}[!htbp]
\centering
    \includegraphics[width=0.8\textwidth]{/Users/mmallek/Documents/Thesis/Plots/fragclass-bymetrics/HRV/MEG_M-AREA_AM-boxplots.png}
  \caption{Fragstats class-level results for Mixed Evergreen - Mesic and area-weighted mean patch area. Boxplot whiskers extend to the 5th and 95th percentile of the observed distribution. The thick grey bar denotes the metric value on the current landscape.}
  \label{fig:megm_areaam}
\end{figure}


\begin{figure}[!htbp]
\centering
    \includegraphics[width=0.8\textwidth]{/Users/mmallek/Documents/Thesis/Plots/fragclass-bymetrics/HRV/MEG_M-CORE_AM-boxplots.png}
  \caption{Fragstats class-level results for Mixed Evergreen - Mesic and area-weighted mean core area. Boxplot whiskers extend to the 5th and 95th percentile of the observed distribution. The thick grey bar denotes the metric value on the current landscape.}
  \label{fig:megm_coream}
\end{figure}


\begin{figure}[!htbp]
\centering
    \includegraphics[width=0.8\textwidth]{/Users/mmallek/Documents/Thesis/Plots/fragclass-bymetrics/HRV/MEG_M-SHAPE_AM-boxplots.png}
  \caption{Fragstats class-level results for Mixed Evergreen - Mesic and area-weighted mean shape index. Boxplot whiskers extend to the 5th and 95th percentile of the observed distribution. The thick grey bar denotes the metric value on the current landscape.}
  \label{fig:megm_shapeam}
\end{figure}


\begin{figure}[!htbp]
\centering
    \includegraphics[width=0.8\textwidth]{/Users/mmallek/Documents/Thesis/Plots/fragclass-bymetrics/HRV/MEG_M-CLUMPY-boxplots.png}
  \caption{Fragstats class-level results for Mixed Evergreen - Mesic and clumpiness. Boxplot whiskers extend to the 5th and 95th percentile of the observed distribution. The thick grey bar denotes the metric value on the current landscape.}
  \label{fig:megm_clumpy}
\end{figure}

%%%%%%%%%%%%%%%%%%%%%%%%%%%%%%%%%%%%%%%%%%%%%%%%%%%%%%%%%%%%%%%%%%%%%%%%%%%%%
%%%%%%%%%%%%%%%%%%%%%%%%%%%%%%%%%%%%%%%%%%%%%%%%%%%%%%%%%%%%%%%%%%%%%%%%%%%%%
%%%%%%%%%%%%%%%%%%%%%%%%%%%%%%%%%%%%%%%%%%%%%%%%%%%%%%%%%%%%%%%%%%%%%%%%%%%%%
%%%%%%%%%%%%%%%%%%%%%%%%%%%%%%%%%%%%%%%%%%%%%%%%%%%%%%%%%%%%%%%%%%%%%%%%%%%%%
%%%%%%%%%%%%%%%%%%%%%%%%%%%%%%%%%%%%%%%%%%%%%%%%%%%%%%%%%%%%%%%%%%%%%%%%%%%%%

\clearpage
\section{Mixed Evergreen - Xeric} 
% figures updated 2015-09
\begin{figure}[!htbp]
  \centering
  \subfloat[][]{
    \centering
    \includegraphics[width=0.5\textwidth]{/Users/mmallek/Documents/Thesis/Plots/darea/hrv_megx.png}
    }%
  \subfloat[][]{
    \includegraphics[width=0.5\textwidth]{/Users/mmallek/Documents/Thesis/Plots/darea/hrv_hist_megx.png}
    }
  \caption{\small (a) Disturbance trajectory for Mixed Evergreen - Xeric. High mortality fire in dark blue; low mortality fire in light blue. (b) Histogram of disturbed hectares with density curve overlaid.} 
  \label{fig:darea_megx}
\end{figure}


Mixed Evergreen - Xeric (\textsc{meg\_x})is a somewhat common cover type within the core project area, encompassing 6,768 ha and comprising roughly 4\% of the project area. Wildfire was prevalent in xeric mixed evergreen forest. The frequency and extent of simulated wildfires varied markedly across the landscape (Figure~\ref{fig:darea_megx}). We summarize the disturbance regime in Tables~\ref{tab:darea_megx} and \ref{tab:darea_atleast_megx}.

%While xeric mixed evergreen forests always experienced some wildfire during the simulation, during a typical five year period wildfires burned across a larger extent on average than the mesic mixed evergreen forest, but the impacts were mostly low mortality. Low mortality fires were several times more frequent as high mortality fires, although proportions varied widely across timesteps. At least some area burned every timestep. Timesteps in which over 10\% of the landscape burned occurred about once in 13 years. The median proportion burned was 8\%. Seldom did large extents burn, at any mortality level, although roughly once every 46 years, more than 25\% of the cover type burned. Over 50\% of the landscape on very rare occasions (once in about 770 years). The maximum extent burned within the cover type was about 56\% (about 3,800 ha). %

%Under this wildfire regime, the return interval between fires (of any mortality level) varied widely from 19 years to over 500 years, with a median of 45 years (Figure~\ref{fig:preturn_megm}). The median return interval and rotation values were influenced primarily by the dominant fire type, low mortality. Xeric mixed evergreen forests had a low mortality fire rotation of 47 years and a high mortality fire rotation of 434 years (Table~\ref{tab:darea_megm}).  %

%In general, return intervals and canopy cover varied spatially across the forest and increased slightly with increasing TPI, an unexpected result. This is likely related to the fact that during the simulated HRV, the vast majority of these stands were in closed canopy conditions, regardless of topographic position. Canopy cover increased by about 3.8\% when comparing minimum to maximum TPI (Table~\ref{tab:tpi_cc}). %

%Finally, when stands of xeric mixed evergreen forests were adjacent to cover types with shorter return intervals, they also exhibited shorter return intervals, reflecting the importance of landscape context on fire regimes.

% updated 2015-09-28
\begin{table}[!htbp]
\small
\centering
\caption{Disturbed area summary statistics for Mixed Evergreen - Xeric. Proportions shown are relative to the total area of Mixed Evergreen - Xeric.}
\label{tab:darea_megx}
\begin{tabular}{@{}llll@{}}
\toprule
\textbf{\begin{tabular}[c]{@{}l@{}}Summary Statistic \\ (disturbed area/timestep)\end{tabular}} & \textbf{Low Mortality} & \textbf{High Mortality} & \textbf{Any Mortality} \\ \midrule
$5^{\text{th}}$ percentile     & 0.92          & 0.06          & 1.01  \\ 
$50^{\text{th}}$ percentile    & 7.68          & 0.76          & 8.73  \\ 
$95^{\text{th}}$ percentile    & 29.05         & 3.91          & 32.27 \\ 
Mean                           & 11.03         & 1.27          & 12.30 \\
\textbf{Fire Rotation}  & 45       & 394       & 41 \\ \bottomrule
\end{tabular}
\end{table}
     
% updated 2015-09-28
\begin{table}[!htbp]
\small
\centering
\caption{Summary of disturbed area in terms of proportion of the amount of \textsc{meg\_x} burned (any level of mortality) during the simulation (after the equilibration period). For each benchmark proportion of the landscape, we list the number of timesteps during the simulation when that extent burned, the proportion of timesteps that represents, the interval in timesteps calculated from the proportion (i.e. approximately every 4 timesteps, at least 25\% of the landscape burned.), and the interval in years calculated from the interval in timesteps (5 years to a timestep).}
\label{tab:darea_atleast_megx}
\begin{tabular}{@{}lllll@{}}
                        & at least 1\% & at least 10\% & at least 25\% & at least 50\% \\ \midrule
Number of timesteps     & 439          & 211           & 62            & 5             \\
Proportion of timesteps & 0.95         & 0.46          & 0.13          & 0.01          \\
Interval (timesteps)    & 1.05         & 2.18          & 7.44          & 92.20         \\
Interval (years)        & 5.25         & 10.92         & 37.18         & 461.00       \\ \bottomrule
\end{tabular}
\end{table}

%%%
The age structure and dynamics of xeric mixed evergreen forest illustrates the interaction between disturbance and succession processes. We focus our analysis on the 5$^{\text{th}}$ to 95$^{\text{th}}$ percentile range of variability for our simulation (excluding the equilibration period). %

The distribution of area among stand conditions within xeric mixed evergreen forest fluctuated over time (Figure~\ref{fig:covcond_megx}). Because high mortality fire is very rare in this cover type, and the time to reaching a Late Development stage is relatively short (Appendix~\ref{sec:covertypedesc}), the vast majority of the landscape was in the Late Development - Closed condition during the simulation (Table~\ref{tab:covcond1}).  %

The seral stage distribution appeared to be in dynamic equilibrium. The most notable departure was the shift from the mid closed stage to Late Development conditions, especially late closed. The current landscape contains 71\% of the xeric mixed evergreen forest in mid development stages, but the late development stages were always dominant under the simulated HRV. 

% figures updated 2015-09
\begin{figure}[!htbp]
  \centering
    \includegraphics[width=\textwidth]{/Users/mmallek/Documents/Thesis/Plots/covcond-bycover/MEGX-HRV-boxplots-.png}
  \caption{Boxplots showing the range of variability for each seral stage over the course of the simulation, excluding the equilibration period. Boxplots were modified so that whiskers extend from the $5^{\text{th}} - 95^{\text{th}}$ percentiles of the observed results. Thick black bars in line with the boxplots denote the current proportion of mesic mixed conifer forests in a given seral stage.} 
  \label{fig:covcond_megx_boxplots}
\end{figure}

The spatial configuration of stand conditions fluctuated markedly over time as well, although there was considerable variation in the magnitude of variability among configuration metrics (Appendix \ref{sec:full-class-results}). Area-weighted patch and core area, patch density, and radius of gyration exhibited the greatest variability over time. Because the landscape is so dominated by the late development, closed canopy cover condition, we use its configuration metrics as a proxy for the cover tyep as a whole. In general, the current landscape contains fewer, smaller, and more isolated patches than existed under the simulated HRV. Patches in Late Development - Closed are less geometrically complex and have less area in cores in the current landscape than during the simulated HRV. This stage is completely departed from the HRV for all these metrics.

% figures updated 2015-09-20
\begin{figure}[!htbp]
\centering
    \includegraphics[width=0.8\textwidth]{/Users/mmallek/Documents/Thesis/Plots/fragclass-bymetrics/HRV/MEG_X-AREA_AM-boxplots.png}
  \caption{Fragstats class-level results for Mixed Evergreen - Xeric and area-weighted mean patch area. Boxplot whiskers extend to the 5th and 95th percentile of the observed distribution. The thick grey bar denotes the metric value on the current landscape.}
  \label{fig:megx_areaam}
\end{figure}


\begin{figure}[!htbp]
\centering
    \includegraphics[width=0.8\textwidth]{/Users/mmallek/Documents/Thesis/Plots/fragclass-bymetrics/HRV/MEG_X-CORE_AM-boxplots.png}
  \caption{Fragstats class-level results for Mixed Evergreen - Xeric and area-weighted mean core area. Boxplot whiskers extend to the 5th and 95th percentile of the observed distribution. The thick grey bar denotes the metric value on the current landscape.}
  \label{fig:megx_coream}
\end{figure}


\begin{figure}[!htbp]
\centering
    \includegraphics[width=0.8\textwidth]{/Users/mmallek/Documents/Thesis/Plots/fragclass-bymetrics/HRV/MEG_X-SHAPE_AM-boxplots.png}
  \caption{Fragstats class-level results for Mixed Evergreen - Xeric and area-weighted mean shape index. Boxplot whiskers extend to the 5th and 95th percentile of the observed distribution. The thick grey bar denotes the metric value on the current landscape.}
  \label{fig:megx_shapeam}
\end{figure}


\begin{figure}[!htbp]
\centering
    \includegraphics[width=0.8\textwidth]{/Users/mmallek/Documents/Thesis/Plots/fragclass-bymetrics/HRV/MEG_X-CLUMPY-boxplots.png}
  \caption{Fragstats class-level results for Mixed Evergreen - Xeric and clumpiness. Boxplot whiskers extend to the 5th and 95th percentile of the observed distribution. The thick grey bar denotes the metric value on the current landscape.}
  \label{fig:megx_clumpy}
\end{figure}

%%%%%%%%%%%%%%%%%%%%%%%%%%%%%%%%%%%%%%%%%%%%%%%%%%%%%%%%%%%%%%%%%%%%%%%%%%%%%
%%%%%%%%%%%%%%%%%%%%%%%%%%%%%%%%%%%%%%%%%%%%%%%%%%%%%%%%%%%%%%%%%%%%%%%%%%%%%
%%%%%%%%%%%%%%%%%%%%%%%%%%%%%%%%%%%%%%%%%%%%%%%%%%%%%%%%%%%%%%%%%%%%%%%%%%%%%
%%%%%%%%%%%%%%%%%%%%%%%%%%%%%%%%%%%%%%%%%%%%%%%%%%%%%%%%%%%%%%%%%%%%%%%%%%%%%
%%%%%%%%%%%%%%%%%%%%%%%%%%%%%%%%%%%%%%%%%%%%%%%%%%%%%%%%%%%%%%%%%%%%%%%%%%%%%

\clearpage
\section{Sierran Mixed Conifer - Ultramafic} 
% figures updated 2015-09
\begin{figure}[!htbp]
  \centering
  \subfloat[][]{
    \centering
    \includegraphics[width=0.5\textwidth]{/Users/mmallek/Documents/Thesis/Plots/darea/hrv_smcu.png}
    }%
  \subfloat[][]{
    \includegraphics[width=0.5\textwidth]{/Users/mmallek/Documents/Thesis/Plots/darea/hrv_hist_smcu.png}
    }
  \caption{\small (a) Disturbance trajectory for Sierran Mixed Conifer - Ultramafic. High mortality fire in dark blue; low mortality fire in light blue. (b) Histogram of disturbed hectares with density curve overlaid.} 
  \label{fig:darea_smcu}
\end{figure}

Sierran Mixed Conifer - Ultramafic (\textsc{smc\_u})is a relatively uncommon cover type within the core project area, encompassing 4,124 ha and comprising roughly 2\% of the project area. Wildfire is much less common in this cover type compared to non-ultramafic sierran mixed conifer forests. Ultramafic soils support scattered, but rarely dense stands of trees and shrubs, creating fuel discontinuities that stop fires from spreading easily. The frequency and extent of simulated wildfires in ultramafic sierran mixed conifer forests varied markedly across the landscape (Figure~\ref{fig:darea_smcu}).  We summarize the disturbance regime in Tables~\ref{tab:darea_smcu} and \ref{tab:darea_atleast_smcu}.

%Wildfire is much less common in this cover type compared to non-ultramafic sierran mixed conifer forests. Ultramafic soils support scattered, but rarely dense stands of trees and shrubs, creating fuel discontinuities that stop fires from spreading easily. 

%Despite this, at least some area in ultramafic mixed conifer forest burned during all the simulated timesteps, likely due in large part to its adjacency to highly susceptible cover types. The burned extent was typically quite small: the median value for any fire was just 5\%, and less than 1\% of these forests (41 ha) burned during about 20\% of the simulated timesteps. High mortality fire was extremely rare, affecting only about 1.5\% of these forests during the median timestep. In general, however, fire was reasonably common, and over  10\% of these forests burned about once every 18 years, which is much less common than for the other mixed conifer variants, as well as the oak-conifer ultramafic type. Even less likely were widespread fires. Over 25\% of these forests burned on a 92 year interval, while more than 50\% of the cover type never burned during the simulation. The maximum extent burned within the cover type was about 46\% (about 1,900 ha). 

%Under this wildfire regime, the grand mean return interval between fires (of any mortality level) varied widely from 21 years to over 500 years, with a median of 68 years (Figure~\ref{fig:preturn_smcu}). As expected, median return interval and rotation values are much longer for this cover type as compared to non-ultramafic mixed conifer forests, which occupy similar elevations. Ultramafic mixed conifer forests had a low mortality fire rotation of 94 years and a high mortality fire rotation of 196 years (Table~\ref{tab:darea_smcu}).  %

%In general, return intervals and canopy cover varied spatially across the forest and decreased with increasing TPI, reflecting our parameterization, which was based on the theory that higher, more southerly aspects are drier and more susceptible to fires. Canopy cover decreased by about 23\% when comparing minimum to maximum TPI, from an average of 37\% to an average of 28\% (Table~\ref{tab:tpi_cc}). 

% updated 2015-09-28
\begin{table}[!htbp]
\small
\centering
\caption{Disturbed area summary statistics for Sierran Mixed Conifer - Ultramafic. Proportions shown are relative to the total area of Sierran Mixed Conifer - Ultramafic.}
\label{tab:darea_smcu}
\begin{tabular}{@{}llll@{}}
\toprule
\textbf{\begin{tabular}[c]{@{}l@{}}Summary Statistic \\ (disturbed area/timestep)\end{tabular}} & \textbf{Low Mortality} & \textbf{High Mortality} & \textbf{Any Mortality} \\ \midrule
$5^{\text{th}}$ percentile    & 0.08          & 0.04          & 0.14  \\
$50^{\text{th}}$ percentile   & 3.09          & 1.48          & 4.81  \\
$95^{\text{th}}$ percentile   & 15.54         & 9.13          & 24.45 \\
Mean                          & 5.01          & 2.55          & 7.56  \\
\textbf{Fire Rotation} & 100       & 196       & 66 \\  \bottomrule
\end{tabular}
\end{table}
      
% updated 2015-09-28
\begin{table}[!htbp]
\small
\centering
\caption{Summary of disturbed area in terms of proportion of the amount of \textsc{smc\_u} burned (any level of mortality) during the simulation (after the equilibration period). For each benchmark proportion of the landscape, we list the number of timesteps during the simulation when that extent burned, the proportion of timesteps that represents, the interval in timesteps calculated from the proportion (i.e. approximately every 4 timesteps, at least 25\% of the landscape burned.), and the interval in years calculated from the interval in timesteps (5 years to a timestep).}
\label{tab:darea_atleast_smcu}
\begin{tabular}{@{}lllll@{}}
                        & at least 1\% & at least 10\% & at least 25\% & at least 50\% \\ \midrule
Number of timesteps     & 364          & 128           & 21            & 1        \\
Proportion of timesteps & 0.79         & 0.28          & 0.05          & 0        \\
Interval (timesteps)    & 1.27         & 3.60          & 21.95         & 461      \\
Interval (years)        & 6.33         & 18.01         & 109.76        & 2305    \\ \bottomrule
\end{tabular}
\end{table}

%%%
The age structure and dynamics of ultramafic mixed conifer forests illustrates the interaction between disturbance and succession processes. We focus our analysis on the 5$^{\text{th}}$ to 95$^{\text{th}}$ percentile range of variability for our simulation (excluding the equilibration period). %

The distribution of area among stand conditions within ultramafic mixed conifer forests fluctuated narrowly over time (Figure~\ref{fig:covcond_smcu_boxplots}). Interestingly, although open canopy conditions dominated during middle development, the distribution of the three late development canopy conditions was roughly equal. Ultramafic soils present a challenge to vegetation, which may explain the dominance of open conditions at the Mid Development stage. However, because fire is relatively uncommon, the shift in dominance at the Late Development stage may reflect the additional time available to vegetation to grow into a closed canopy condition (Appendix~\ref{sec:covertypedesc}). %

The seral-stage distribution appeared to be in dynamic equilibrium. The most notable departures were the decrease in area classified as Early Development, currently at 49\% of the landscape, and the increase in area classified as Mid Development - Open, currently at 5\%. The only seral stages not completely departed from the HRV were Mid Development - Closed and Mid Development - Moderate.

% figures updated 2015-09
\begin{figure}[!htbp]
  \centering
    \includegraphics[width=\textwidth]{/Users/mmallek/Documents/Thesis/Plots/covcond-bycover/SMCU-HRV-boxplots-.png}
  \caption{Boxplots showing the range of variability for each seral stage over the course of the simulation, excluding the equilibration period. Boxplots were modified so that whiskers extend from the $5^{\text{th}} - 95^{\text{th}}$ percentiles of the observed results. Thick black bars in line with the boxplots denote the current proportion of mesic mixed conifer forests in a given seral stage.} 
  \label{fig:covcond_smcu_boxplots}
\end{figure}

The spatial configuration of stand conditions fluctuated markedly over time as well, although there was considerable variation in the magnitude of variability among configuration metrics (Appendix \ref{sec:full-class-results}). The class-level metrics for ultramafic mixed conifer forests for early development, mid development open, and late development closed and moderate typically fall completely outside the simulated HRV. Of these, middle development open is currently characterized by smaller, less complex patches with less core area and less aggregation than the same type during the simulated HRV. The other classes have the opposite result. The remaining classes (middle development closed and moderate canopy, late development open canopy) are either not departed or moderately departed from the HRV.


% figures updated 2015-09-20
\begin{figure}[!htbp]
\centering
    \includegraphics[width=0.8\textwidth]{/Users/mmallek/Documents/Thesis/Plots/fragclass-bymetrics/HRV/SMC_U-AREA_AM-boxplots.png}
  \caption{Fragstats class-level results for Sierran Mixed Conifer - Ultramafic and area-weighted mean patch area. Boxplot whiskers extend to the 5th and 95th percentile of the observed distribution. The thick grey bar denotes the metric value on the current landscape.}
  \label{fig:smcu_areaam}
\end{figure}


\begin{figure}[!htbp]
\centering
    \includegraphics[width=0.8\textwidth]{/Users/mmallek/Documents/Thesis/Plots/fragclass-bymetrics/HRV/SMC_U-CORE_AM-boxplots.png}
  \caption{Fragstats class-level results for Sierran Mixed Conifer - Ultramafic and area-weighted mean core area. Boxplot whiskers extend to the 5th and 95th percentile of the observed distribution. The thick grey bar denotes the metric value on the current landscape.}
  \label{fig:smcu_coream}
\end{figure}


\begin{figure}[!htbp]
\centering
    \includegraphics[width=0.8\textwidth]{/Users/mmallek/Documents/Thesis/Plots/fragclass-bymetrics/HRV/SMC_U-SHAPE_AM-boxplots.png}
  \caption{Fragstats class-level results for Sierran Mixed Conifer - Ultramafic and area-weighted mean shape index. Boxplot whiskers extend to the 5th and 95th percentile of the observed distribution. The thick grey bar denotes the metric value on the current landscape.}
  \label{fig:smcu_shapeam}
\end{figure}


\begin{figure}[!htbp]
\centering
    \includegraphics[width=0.8\textwidth]{/Users/mmallek/Documents/Thesis/Plots/fragclass-bymetrics/HRV/SMC_U-CLUMPY-boxplots.png}
  \caption{Fragstats class-level results for Sierran Mixed Conifer - Ultramafic and clumpiness. Boxplot whiskers extend to the 5th and 95th percentile of the observed distribution. The thick grey bar denotes the metric value on the current landscape.}
  \label{fig:smcu_clumpy}
\end{figure}

%%%%%%%%%%%%%%%%%%%%%%%%%%%%%%%%%%%%%%%%%%%%%%%%%%%%%%%%%%%%%%%%%%%%%%%%%%%%%
%%%%%%%%%%%%%%%%%%%%%%%%%%%%%%%%%%%%%%%%%%%%%%%%%%%%%%%%%%%%%%%%%%%%%%%%%%%%%
%%%%%%%%%%%%%%%%%%%%%%%%%%%%%%%%%%%%%%%%%%%%%%%%%%%%%%%%%%%%%%%%%%%%%%%%%%%%%
%%%%%%%%%%%%%%%%%%%%%%%%%%%%%%%%%%%%%%%%%%%%%%%%%%%%%%%%%%%%%%%%%%%%%%%%%%%%%
%%%%%%%%%%%%%%%%%%%%%%%%%%%%%%%%%%%%%%%%%%%%%%%%%%%%%%%%%%%%%%%%%%%%%%%%%%%%%



\clearpage
\section{Oak-Conifer Forest and Woodland - Ultramafic} 
% figures updated 2015-09
\begin{figure}[!htbp]
  \centering
  \subfloat[][]{
    \centering
    \includegraphics[width=0.5\textwidth]{/Users/mmallek/Documents/Thesis/Plots/darea/hrv_ocfwu.png}
    }%
  \subfloat[][]{
    \includegraphics[width=0.5\textwidth]{/Users/mmallek/Documents/Thesis/Plots/darea/hrv_hist_ocfwu.png}
    }
  \caption{\small (a) Disturbance trajectory for Oak-Conifer Forest and Woodland - Ultramafic. High mortality fire in dark blue; low mortality fire in light blue. (b) Histogram of disturbed hectares with density curve overlaid.} 
  \label{fig:darea_ocfwu}
\end{figure}

Oak-Conifer Forest and Woodland - Ultramafic (\textsc{ocfw\_u})is a relatively uncommon cover type within the core project area, encompassing 1,060 ha and comprising roughly 0.6\% of the project area. Wildfire is much less common in this cover type compared to its non-ultramafic oak-conifer forests and woodlands. Ultramafic soils support scattered, but rarely dense stands of trees and shrubs, creating fuel discontinuities that stop fires from spreading easily. The frequency and extent of simulated wildfires in ultramafic oak-conifer forests and woodlands varied markedly across the landscape (Figure~\ref{fig:darea_ocfwu}). We summarize the disturbance regime in Tables~\ref{tab:darea_ocfwu} and \ref{tab:darea_atleast_ocfwu}.

%Wildfire is much less common in this cover type compared to its non-ultramafic oak-conifer forests and woodlands. Ultramafic soils support scattered, but rarely dense stands of trees and shrubs, creating fuel discontinuities that stop fires from spreading easily. All ultramafic oak-conifer forests and woodlands escaped fire 12 times during the simulation, and less than 1\% (11 ha) of them burned during 19\% of the timesteps. Much of this is due to the extreme rarity of high mortality fire, which never burned more than 11\% of this cover type in a single timestep, and actually averaged 0.37\% during the simulated period. In contrast, low mortality fire was fairly common, and over 10\% of the landscape burned about once every 13 years, predominantly as low mortality fire. The median proportion burned was 6\%. Seldom did large extents burn, although once every 32 years more than 25\% of the cover type burned. Fires spread over 50\% of the cover type only once every 177 years.  %

%Under this wildfire regime, the grand mean return interval between fires (of any mortality level) varied widely from 20 years to over 500 years, with a median of 44 years (Figure~\ref{fig:preturn_ocfwu}). As expected, median return interval and rotation values are much longer for this cover type compared to the non-ultramafic variant (Table~\ref{tab:darea_ocfw}), and were clearly influenced by the dominant fire type, low mortality. Ultramafic oak-conifer forests and woodlands had a low mortality fire rotation of 44 years and a high mortality fire rotation of 1295 years (Table~\ref{tab:darea_ocfwu}).  %

%In general, return intervals and canopy cover varied spatially across the forest and decreased with increasing TPI, reflecting our parameterization, which was based on the theory that higher, more southerly aspects are drier and more susceptible to fires. Canopy cover decreased by 9\% when comparing minimum to maximum TPI, from an average of 25\% to an average of 22.8\% (Table~\ref{tab:tpi_cc}). %

%Return intervals and canopy cover varied spatially across the ultramafic oak-conifer forests and woodlands, as expected, and increased with increasing TPI, which was not predicted. Canopy cover increased by about 3\% when comparing minimum to maximum TPI, the only focal cover type for which we observed an increase (Table~\ref{tab:tpi_cc}). 

% updated 2015-09-28
\begin{table}[!htbp]
\small
\centering
\caption{Disturbed area summary statistics for Oak-Conifer Forest and Woodland - Ultramafic. Proportions shown are relative to the total area of Oak-Conifer Forest and Woodland - Ultramafic.}
\label{tab:darea_ocfwu}
\begin{tabular}{@{}llll@{}}
\toprule
\textbf{\begin{tabular}[c]{@{}l@{}}Summary Statistic \\ (disturbed area/timestep)\end{tabular}} & \textbf{Low Mortality} & \textbf{High Mortality} & \textbf{Any Mortality} \\ \midrule
$5^{\text{th}}$ percentile    & 0.06          & 0.00          & 0.06     \\
$50^{\text{th}}$ percentile   & 4.16          & 0.77          & 4.74     \\
$95^{\text{th}}$ percentile   & 31.08         & 7.11          & 36.12    \\
Mean                          & 7.86          & 1.87          & 9.72     \\
\textbf{Fire Rotation} & 64       & 268      & 51 \\ \bottomrule
\end{tabular}
\end{table}

% updated 2015-09-28
\begin{table}[!htbp]
\small
\centering
\caption{Summary of disturbed area in terms of proportion of the amount of \textsc{ocfw\_u} burned (any level of mortality) during the simulation (after the equilibration period). For each benchmark proportion of the landscape, we list the number of timesteps during the simulation when that extent burned, the proportion of timesteps that represents, the interval in timesteps calculated from the proportion (i.e. approximately every 4 timesteps, at least 25\% of the landscape burned.), and the interval in years calculated from the interval in timesteps (5 years to a timestep).}
\label{tab:darea_atleast_ocfwu}
\begin{tabular}{@{}lllll@{}}
                        & at least 1\% & at least 10\% & at least 25\% & at least 50\% \\ \midrule
Number of timesteps     & 355          & 149           & 51            & 5         \\
Proportion of timesteps & 0.77         & 0.32          & 0.11          & 0.01      \\
Interval (timesteps)    & 1.30         & 3.09          & 9.04          & 92.20     \\
Interval (years)        & 6.49         & 15.47         & 45.20         & 461   \\ \bottomrule
\end{tabular}
\end{table}

%%%
The age structure and dynamics of ultramafic oak-conifer forests and woodlands illustrates the interaction between disturbance and succession processes. We focus our analysis on the 5$^{\text{th}}$ to 95$^{\text{th}}$ percentile range of variability for our simulation (excluding the equilibration period). %

The distribution of area among stand conditions within ultramafic oak-conifer forests and woodlands fluctuated narrowly over time (Figure~\ref{fig:covcond_ocfwu_boxplots}). Late development stages are rare on the current landscape, but well represented during the HRV. Conversely, mid closed is currently quite common, but is virtually absent during the HRV. The current amount of early development vegetation is near the median value during the HRV, and is the only stage not completely departed from the HRV. Ultramafic soils present a challenge to vegetation, which may explain the dominance of open conditions at the Mid Development stage. However, because fire is relatively uncommon, the shift in dominance at the Late Development stage may reflect the additional time available to vegetation to grow into a closed canopy condition (Appendix~\ref{sec:covertypedesc}). %


% figures updated 2015-09
\begin{figure}[!htbp]
  \centering
    \includegraphics[width=\textwidth]{/Users/mmallek/Documents/Thesis/Plots/covcond-bycover/OCFWU-HRV-boxplots-.png}
  \caption{Boxplots showing the range of variability for each seral stage over the course of the simulation, excluding the equilibration period. Boxplots were modified so that whiskers extend from the $5^{\text{th}} - 95^{\text{th}}$ percentiles of the observed results. Thick black bars in line with the boxplots denote the current proportion of mesic mixed conifer forests in a given seral stage.} 
  \label{fig:covcond_ocfwu_boxplots}
\end{figure}

The spatial configuration of stand conditions fluctuated markedly over time as well, although there was considerable variation in the magnitude of variability among configuration metrics (Appendix \ref{sec:full-class-results}). Area-weighted patch and core area, patch density, mean similarity, and radius of gyration all exhibited high variability over time. For the most part, the current landscape's values fall within the HRV, although the middle development, closed canopy and late development, open canopy patches usually fell outside of it. Patches and their cores are larger, more complex, and more numerous now compared to the simulated HRV. The current landscape also has more aggregated patches. However, we caution against drawing firm conclusions, since this type did not reach equilibrium until halfway through the simulation, and is small in extent with less consistent results than the other cover types.

% figures updated 2015-09-20
\begin{figure}[!htbp]
\centering
    \includegraphics[width=0.8\textwidth]{/Users/mmallek/Documents/Thesis/Plots/fragclass-bymetrics/HRV/OCFW_U-AREA_AM-boxplots.png}
  \caption{Fragstats class-level results for Oak-Conifer Forest and Woodland - Ultramafic and area-weighted mean patch area. Boxplot whiskers extend to the 5th and 95th percentile of the observed distribution. The thick grey bar denotes the metric value on the current landscape.}
  \label{fig:ocfwu_areaam}
\end{figure}


\begin{figure}[!htbp]
\centering
    \includegraphics[width=0.8\textwidth]{/Users/mmallek/Documents/Thesis/Plots/fragclass-bymetrics/HRV/OCFW_U-CORE_AM-boxplots.png}
  \caption{Fragstats class-level results for Oak-Conifer Forest and Woodland - Ultramafic and area-weighted mean core area. Boxplot whiskers extend to the 5th and 95th percentile of the observed distribution. The thick grey bar denotes the metric value on the current landscape.}
  \label{fig:ocfwu_coream}
\end{figure}


\begin{figure}[!htbp]
\centering
    \includegraphics[width=0.8\textwidth]{/Users/mmallek/Documents/Thesis/Plots/fragclass-bymetrics/HRV/OCFW_U-SHAPE_AM-boxplots.png}
  \caption{Fragstats class-level results for Oak-Conifer Forest and Woodland - Ultramafic and area-weighted mean shape index. Boxplot whiskers extend to the 5th and 95th percentile of the observed distribution. The thick grey bar denotes the metric value on the current landscape.}
  \label{fig:ocfwu_shapeam}
\end{figure}


\begin{figure}[!htbp]
\centering
    \includegraphics[width=0.8\textwidth]{/Users/mmallek/Documents/Thesis/Plots/fragclass-bymetrics/HRV/OCFW_U-CLUMPY-boxplots.png}
  \caption{Fragstats class-level results for Oak-Conifer Forest and Woodland - Ultramafic and clumpiness. Boxplot whiskers extend to the 5th and 95th percentile of the observed distribution. The thick grey bar denotes the metric value on the current landscape.}
  \label{fig:ocfwu_clumpy}
\end{figure}

%%%%%%%%%%%%%%%%%%%%%%%%%%%%%%%%%%%%%%%%%%%%%%%%%%%%%%%%%%%%%%%%%%%%%%%%%%%%%
%%%%%%%%%%%%%%%%%%%%%%%%%%%%%%%%%%%%%%%%%%%%%%%%%%%%%%%%%%%%%%%%%%%%%%%%%%%%%
%%%%%%%%%%%%%%%%%%%%%%%%%%%%%%%%%%%%%%%%%%%%%%%%%%%%%%%%%%%%%%%%%%%%%%%%%%%%%
%%%%%%%%%%%%%%%%%%%%%%%%%%%%%%%%%%%%%%%%%%%%%%%%%%%%%%%%%%%%%%%%%%%%%%%%%%%%%
%%%%%%%%%%%%%%%%%%%%%%%%%%%%%%%%%%%%%%%%%%%%%%%%%%%%%%%%%%%%%%%%%%%%%%%%%%%%%
%% !TEX root = master.tex
\chapter{\textsc{Fragstats} Metrics: brief descriptions}
\label{app:metricdescriptions}
\begin{enumerate}
	\item Percentage of Landscape (PLAND)\\
	\emph{Landscape}-level: not computed\\
	\emph{Class}-level: the percentage of the landscape comprised of a particular patch type\\
	
	%\item Core Area Percentage of Landscape (CPLAND)\\
	%\emph{Landscape}-level: not computed\\
	%\emph{Class}-level: the sum of all core areas in a given patch type divided by the total landscape area; reported as a percentage%\\
	
	\item Patch Density (PD)\\
	\emph{Landscape}-level: number of patches of all cover types and condition classes divided by the total landscape area for one timestep\\
	\emph{Class}-level: number of patches of a give cover type and condition class divided by the total area occupied by that cover-condition type
\\

	\item Total Edge (TE)\todo{COMPLETE ME} \\
	label{item:TE}
	\emph{Landscape}-level:      \\
	\emph{Class}-level:         \\
	
	\item Edge Density (ED) \\
	\label{item:ED}
	\emph{Landscape}-level: the sum of the lengths of all edge segments divided by the total landscape area\\	
	\emph{Class}-level: the sum of the lengths of all edge segments for a given patch type divided by the total area occuring as that patch type\\
	
	\item Mean Area (AREA\_MN)\\
	\emph{Landscape}-level:  mean patch area across all cover types and condition classes\\
	\emph{Class}-level:  mean patch area across all condition classes for a given cover type\\
	
	\item Area-Weighted Mean Area (AREA\_AM)\\
	\emph{Note}: Area-weighted metrics are used to reduce the influence of the many isolated, single-pixel ``patches'' that are an artifact of the model and do not represent true ecological processes. They reflect the mean metric value for a cell selected at random on the landscape. 	\\
	\emph{Landscape}-level: area-weighted mean patch area across all cover types and condition classes \\
	\emph{Class}-level: area-weighted mean patch area across all condition classes for a given cover type. \\
	
	\item Area-Weighted Mean Radius of Gyration (GYRATE\_AM)\\
	\emph{Landscape}-level: measure of the area-weighted mean length across the landscape a patch extends its reach; in other words, calculate the shortest path between every possible pair of cells within a patch and take the longest of this set, then take the average of these ``longest'' paths for every patch on the landscape\\
	\emph{Class}-level: for a given cover type and condition class, the area-weighted mean average length of a patch on the landscape\\
	
	\item Mean Shape (SHAPE\_MN) measures the complexity of patch shape compared to a square of the same size\\
	\emph{Landscape}-level: length of patch perimeter for all patch type on the landscape divided by the area of the landscape, %adjusted to a square standard, and averaged across all patch types\\
	\emph{Class}-level: length of patch perimeter for each patch type on the landscape divided by the square root of the area in that patch type, adjusted to a square standard \\
	
	\item Area-Weighted Mean Shape (SHAPE\_AM)\\
	\emph{Note}: the theoretical idea behind Shape is to compare a patch to the simplest shape, a square 		\\
	\emph{Landscape}-level: length of patch perimeter for all patch types on the landscape divided by the total landscapearea, adjusted to a square standard, and converted to an area-weighted mean across all patch types \\
	\emph{Class}-level: length of patch perimeter for each patch type on the landscape divided by the square root of the area in that patch type, adjusted to a square standard\\
	
	\item Mean Core Area (CORE\_MN)\\
	\emph{Landscape}-level: total core area on the landscape, divided by landscape area\\
	\emph{Class}-level: total core area within each patch type on the landscape, divided by the total area in the same patch type\\
	
	\item Area-Weighted Mean Core Area (CORE\_AM)\\
	\emph{Note}: Core area is defined as the area within a patch beyond some specified depth-of-edge influence (i.e., edge distance) or buffer width and is important for organisms who specialize in patch interiors 	\\
	\emph{Landscape}-level: total core area within each patch on the landscape, divided by the total landscape area in the same patch type	\\
	\emph{Class}-level: total core area within a given patch type on the landscape, divided by the total area in the same patch type	\\
	
	\item Area-Weighted Mean Core Area Index (CAI\_AM)\\
	\emph{Landscape}-level: core area for the landscape as a percentage of total landscape area \\
	%core area of a patch divided by total area of a patch, multiplied by the area of that patch divided by the area of the landscape; reported as a percentage	\\
	\emph{Class}-level: core area for a given patch type as a percentage of a total area in that patch type
	%core area of a patch divided by total area of a patch, multiplied by the area of that patch divided by the total area of that patch type; reported as a percentage	\\
	
	\item Mean Similarity Index (SIMI\_MN)\\
	\emph{Note}: Similarity distinguishes sparse distributions of small and insular habitat patches from configurations where the habitat forms a complex cluster of larger, hospitable (i.e., similar) patches. A similiarity value is assigned to each possible pair of condition classes before running \textsc{Fragstats}. 	\\
	\emph{Landscape}-level: the average of the similarity value for each patch on the landscape \\
	%the average sum over all neighboring patches with edges within a specified distance of the focal patch, of: neighboring patch area times a similarity coefficient between the focal patch type and the class of the neighboring patch, divided by the nearest edge-to-edge distance squared between the focal patch and the neighboring patch	\\
	\emph{Class}-level: the average of the similarity value for each patch within a given patch type \\
	%the average sum over all neighboring patches with edges within a specified distance of the focal patch, of: neighboring patch area times a similarity coefficient between the focal patch type and the class of the neighboring patch, divided by the nearest edge-to-edge distance squared between the focal patch and the neighboring patch; reported for each patch type separately	\\
	
	\item Contrast-Weighted Edge Density (CWED)\\
	\emph{Note}: This metric is intended to highlight the functional importance of edge	\\
	\emph{Landscape}-level: the sum of the lengths of all edge segments divided by the total landscape area (Edge Density, metric~\ref{item:ED}), multiplied by a contrast weight (metric~\ref{item:TECI})
	\emph{Class}-level: the sum of the lengths of all edge segments for a given patch type divided by the total landscape area, multiplied by the appropriate contrast weight (metric~\ref{item:TECI})  	\\

	\item Total Edge Contrast Index (TECI) \\
	\label{item:TECI}
	\emph{Note}: Contrast refers to the relative difference among patch types. For example, mature forest next to young forest might have a lower-
contrast edge than mature forest adjacent to open field. A contrast weight is assigned to each possible pair of condition classes before 
running \textsc{Fragstats}. 	\\
	\emph{Landscape}-level: the sum of the lengths of all edge segments in the landscape multiplied by the appropriate contrast weight, divided by 
the total length of edge in the landscape\\
	\emph{Class}-level: the sum of the products of the lengths of all edge segments for a given patch type and the appropriate contrast weight, 
divided by the sum of the lengths of all edge segments for a given patch type \\

\todo{FILL ME IN}
	\item Area-weighted Mean Edge Contrast
	\label{item:ECONAM}
	\emph{Landscape}-level:      \\
	\emph{Class}-level:         \\

	\item Mean Edge Contrast
	\label{item:ECONMN}
	\emph{Landscape}-level:      \\
	\emph{Class}-level:         \\

	
	\item Contagion (CONTAG)\\
	\emph{Landscape}-level: a cell-based (as opposed to patch-based) metric that measures the likelihood of a given cell belonging to the same patch type as a randomly chosen adjacent cell and is a common measure of both aggregation and dispersion 	\\
	%1 minus the sum of the proportional abundance of each patch type multiplied by the proportion of adjacencies between cells of that patch type and another patch type, multiplied by the logarithm of the same quantity, summed over each unique adjacency type and each patch type; divided by 2 times the logarithm of the number of patch types; reported as a percent; inversely related to edge density\\
	\emph{Class}-level: not computed \\ 	

		
	\item Clumpiness Index (CLUMPY)\\
	\emph{Landscape}-level: not computed  	\\
	\emph{Class}-level: similar conceptually (though not mathematically) to Contagion, the Clumpiness Index indicates how fragmented or aggregated 
the cells of a given patch type are; values range from -1 (completely dispersed) to 1 (maximally clumped), with 0 representing a completely 
random configuration 	\\
	%the proportional deviation of the proportion of like adjacencies involving the corresponding class from that expected under a spatially 
random distribution. If the proportion of like adjacencies ($G_i$) is $\geq$ the proportion of the landscape comprised of the focal class (
$P_i$), then $\text{CLUMPY} = \frac{G_i - P_i}{1 - P_i}$. If $G_i < P_i \text{and} P_i \geq 0.5, \text{then CLUMPY} = \frac{G_i - Pi}{1 - P_i}$
. If $G_i < Pi \text{and} P_i < 0.5, \text{then CLUMPY} = \frac{P_i - G_i}{-P_i}$.	\\
	
	\item Interspersion and Juxtaposition Index (IJI) \\
	\emph{Landscape}-level: a patch-based metric that represents the observed level of interspersion as a percentage of the maximum possible given 
the total number of patch types; based on the total length of edge in the landscape 	\\
	%similar to Contagion, but patch-based rather than cell-based; ranges from 0 (uneven configuration of patches - low interspersion and 
juxtaposition) to 100 (maximally evenly interspersed or juxtaposed patches) 	\\
	%-1 times the sum of the length of each unique edge type divided by the total landscape edge, multiple by the logarithm of the same quantity, 
summed over each unique edge type;  divided by the logarithm of the number of patch types times the number of patch types minus 1 divided by 2 
	\\
	\emph{Class}-level: a patch-based metric that represents the observed level of interspersion as a percentage of the maximum possible given the 
total number of patch types, based on length of edge between the focal patch type and other patch types 	\\
	%similar to the Clumpiness index, but patch-based rather than cell-based; ranges from 0 (uneven configuration of patches - low interspersion 
and juxtaposition) to 100 (maximally evenly interspersed or juxtaposed patches) based on length of edge between the focal patch type and other 
patch types 	\\
	%-1 times the sum of the length of each unique edge type involving the corresponding patch type divided by the total length of edge involving 
the same type, multiplied by the logarithm of the same quantity, summed over each unique edge type; divided by the logarithm of the number of 
patch types minus 1; reported as a percentage	\\
	
	\item Patch Richness (PR)\\
	\emph{Landscape}-level: the number of patch types present in the landscape	\\
	\emph{Class}-level: not computed \\
	
	\item Simpson's Diversity Index (SIDI)\\
	\emph{Landscape}-level: the probability that any 2 pixels selected at random would be different patch types	\\
	\emph{Class}-level: not computed\\
	
	\item Simpson's Evenness Index (SIEI) \\
	\emph{Landscape}-level:  the observed level of diversity divided by the maximum possible diversity for a given patch richness 	\\
	\emph{Class}-level: not computed\\
	
	\item Aggregation Index (AI) - an area-weighted mean class aggregation index \\
	\emph{Landscape}-level: cell-based metric based on the ratio of the observed number of like adjacencies to the maximum possible number of like 
adjacencies; similar in interpretation to Contagion, but uses different statistical methods
	%the number of cells of the same cover and condition (patch type) adjecent to one another, multiplied by the proportion of the landscape 
comprised of that that patch type, summed over all classes; reported as a percentage	\\
	\emph{Class}-level: the number of cells of a given patch type adjacent to one another divided by the maximum possible number of like 
adjacencies for that patch type; similar in interpretation to the Clumpiness index, but uses different statistical methods	\\
\end{enumerate}

Several of the individual landscape metrics are redundant with one another. For example, \emph{Contagion} and \emph{Edge Density} are inversely related, so it is perhaps helpful, but not necessary, to examine both metrics. We evaluated more metrics than we needed, then narrowed our focus in order to provide a relatively simple and highly interpretable conclusion.
%% !TEX root = master.tex
\chapter{Full \textsc{Fragstats} Landscape Results}
\label{app:full-land-results}

\begin{landscape}
\begin{table}[]
\small
\centering
\caption{All \textsc{Fragstats} metrics computed for the study area during the simulated HRV. See Appendix~\ref{app:metricdescriptions} for descriptions. Abbreviations are:
\textsc{pd} = patch density; 
\textsc{ed} = edge density; 
\textsc{area\_am} = area-weighted mean patch size; 
\textsc{area\_mn} = mean patch size;
\textsc{gyrate\_am} = area-weighted mean patch radius of gyration (correlation length); 
\textsc{shape\_am} = area-weighted mean patch shape index; 
\textsc{shape\_mn} = mean patch shape;
\textsc{core\_am} = area-weighted mean patch core area; 
\textsc{core\_mn} = mean core area;
\textsc{cai\_am} = area-weighted mean patch core area index; 
\textsc{simi\_mn} = mean similarity; 
\textsc{cwed} = contrast-weighted edge density; 
\textsc{teci} = total edge contrast index; 
\textsc{econ\_am} = area-weighted mean edge contrast; 
\textsc{econ\_mn} = mean edge contrast;
\textsc{contag} = contagion; 
\textsc{iji} = interspersion and juxtaposition index; 
\textsc{pr} = patch richness; 
\textsc{sidi} = Simpson's diversity index; 
\textsc{siei} = Simpson's evenness index; 
\textsc{ai} = aggregation index.
}
\label{tab:fragland-all}
\begin{tabular}{@{}lrrrrr|rrr@{}}
\toprule
\textbf{\begin{tabular}[c]{@{}l@{}}Landscape\\ Metric\end{tabular}} & \textbf{srv5\%} & \textbf{srv25\%} & \textbf{srv50\%} & \textbf{srv75\%} & \textbf{srv100\%}  & \textbf{\begin{tabular}[c]{@{}l@{}}Current\\ Value\end{tabular}} & \textbf{\begin{tabular}[c]{@{}l@{}}Current\\ \%SRV\end{tabular}} & \textbf{\begin{tabular}[c]{@{}l@{}}Departure\\ Index\end{tabular}} \\ \midrule
\textsc{pd        } & 19.774   & 20.271   & 20.625   & 20.975   & 21.417   & 19.507   & 2   & -96  \\ 
\textsc{ed        } & 120.581  & 121.88   & 122.903  & 123.691  & 124.813  & 128.875  & 100 & 100  \\
\textsc{area\_mn  } & 4.669    & 4.768    & 4.848    & 4.933    & 5.057    & 5.126    & 99  & 98   \\
\textsc{area\_am  } & 156.549  & 166.016  & 174.884  & 184.448  & 205.209  & 119.985  & 0   & -100 \\
\textsc{gyrate\_am} & 693.361  & 705.323  & 715.921  & 730.824  & 758.915  & 620.951  & 0   & -100 \\
\textsc{shape\_mn } & 1.417    & 1.423    & 1.427    & 1.432    & 1.438    & 1.511    & 100 & 100  \\
\textsc{shape\_am } & 3.56     & 3.621    & 3.667    & 3.727    & 3.847    & 3.243    & 0   & -100 \\
\textsc{core\_mn  } & 2.887    & 2.959    & 3.009    & 3.079    & 3.139    & 3.347    & 100 & 100  \\
\textsc{core\_am  } & 135.146  & 141.964  & 149.582  & 157.587  & 169.545  & 106.71   & 0   & -100 \\
\textsc{cai\_am   } & 60.527   & 61.321   & 61.999   & 62.897   & 64.166   & 65.295   & 100 & 100  \\
\textsc{simi\_mn  } & 2333.717 & 2456.329 & 2531.906 & 2629.83  & 2794.671 & 2095.764 & 0   & -100 \\
\textsc{cwed      } & 40.608   & 41.114   & 41.51    & 41.95    & 42.564   & 36.092   & 0   & -100 \\
\textsc{teci      } & 32.717   & 33.051   & 33.337   & 33.687   & 34.23    & 27.654   & 0   & -100 \\
\textsc{econ\_mn  } & 32.556   & 32.846   & 33.118   & 33.469   & 33.948   & 26.576   & 0   & -100 \\
\textsc{econ\_am  } & 32.793   & 33.163   & 33.458   & 33.833   & 34.401   & 27.756   & 0   & -100 \\
\textsc{contag    } & 53.943   & 54.455   & 54.744   & 55.064   & 55.523   & 51.172   & 0   & -100 \\
\textsc{iji       } & 62.016   & 62.696   & 63.096   & 63.513   & 64.066   & 65.868   & 100 & 100  \\
\textsc{pr        } & 115      & 119      & 120      & 122      & 124      & 117      & 9   & -82  \\
\textsc{sidi      } & 0.938    & 0.942    & 0.944    & 0.945    & 0.948    & 0.962    & 100 & 100  \\
\textsc{siei      } & 0.946    & 0.949    & 0.951    & 0.953    & 0.956    & 0.971    & 100 & 100  \\
\textsc{ai        } & 81.531   & 81.699   & 81.821   & 81.974   & 82.168   & 80.963   & 0   & -100 \\ \bottomrule
\end{tabular}
\end{table}
\end{landscape}
%% !TEX root = master.tex



\pagestyle{empty}
\begin{landscape}

\section{Full \textsc{Fragstats} Class Results} % made section instead of chapter since this is still HRV stuff
\label{app:full-class-results}

\footnotesize
\begin{center}
\begin{footnotesize}
\begin{longtable}{llrrrrr|rrr}
\caption{Unabridged results for class-level metrics for Mixed Evergreen - Mesic (\textsc{meg\_m}) calculated with \textsc{Fragstats}. This table shows the range of variability in landscape structure. Included are the $5^{\text{th}}$ percentile, $25^{\text{th}}$ percentile, $50^{\text{th}}$ percentile, $75^{\text{th}}$ percentile, and $95^{\text{th}}$ percentiles of the distribution, as well as the current class value, the current percentile range of variability (\%RV) for that proportion, and the departure classification. For seral stage abbreviations, see Table~\ref{condtable}.} 
\label{tab:fragclass_megm}\\

\hline 
\textbf{\begin{tabular}[c]{@{}l@{}}Cover-Seral Stage Type\end{tabular}}  &   
\textbf{\begin{tabular}[c]{@{}l@{}}Landscape\\ Metric\end{tabular}}  &   
\textbf{$5^{\text{th}}$ } &   
\textbf{$25^{\text{th}}$ } &   
\textbf{$50^{\text{th}}$ } &   
\textbf{$75^{\text{th}}$ } &   
\textbf{$95^{\text{th}}$ }  &  
\textbf{\begin{tabular}[c]{@{}l@{}}Current\\ Value\end{tabular}} &   
\textbf{\begin{tabular}[c]{@{}l@{}}Current\\ \%RV\end{tabular}} &   
\textbf{\begin{tabular}[c]{@{}l@{}}Departure\end{tabular}} \\  \\ \hline 
\endfirsthead

\multicolumn{10}{c}{{\bfseries \tablename\ \thetable{} -- continued from previous page}} \\
\hline 
\textbf{\begin{tabular}[c]{@{}l@{}}Cover-Seral Stage Type\end{tabular}}  &   
\textbf{\begin{tabular}[c]{@{}l@{}}Landscape\\ Metric\end{tabular}}  &   
\textbf{$5^{\text{th}}$ } &   
\textbf{$25^{\text{th}}$ } &   
\textbf{$50^{\text{th}}$ } &   
\textbf{$75^{\text{th}}$ } &   
\textbf{$95^{\text{th}}$ }  &  
\textbf{\begin{tabular}[c]{@{}l@{}}Current\\ Value\end{tabular}} &   
\textbf{\begin{tabular}[c]{@{}l@{}}Current\\ \%RV\end{tabular}} &   
\textbf{\begin{tabular}[c]{@{}l@{}}Departure\end{tabular}} \\  \\ \hline \endhead

\hline \multicolumn{10}{|l|}{{Continued on next page}} \\ \hline
\endfoot

\hline \hline
\endlastfoot

\textsc{meg\_m\_early\_all} & \textsc{ai        }   & 66.279   & 71.173   & 74.11    & 76.878   & 80.515   & 74.437   & 54  & none     \\
\textsc{meg\_m\_mid\_cl   } & \textsc{ai        }   & 59.317   & 71.154   & 76.562   & 81.477   & 88.267   & 77.044   & 54  & none     \\
\textsc{meg\_m\_mid\_mod  } & \textsc{ai        }   & 66.954   & 72.12    & 75.889   & 78.611   & 81.632   & 76.278   & 56  & none     \\
\textsc{meg\_m\_mid\_op   } & \textsc{ai        }   & 39.222   & 54.414   & 61.176   & 66.071   & 75.755   & 70.857   & 89  & moderate \\
\textsc{meg\_m\_late\_cl  } & \textsc{ai        }   & 79.811   & 80.529   & 80.861   & 81.161   & 81.433   & 81.707   & 100 & complete \\
\textsc{meg\_m\_late\_mod } & \textsc{ai        }   & 57.417   & 60.914   & 63.965   & 65.877   & 67.229   & 79.01    & 100 & complete \\
\textsc{meg\_m\_late\_op  } & \textsc{ai        }   & 50.494   & 54.818   & 56.892   & 58.751   & 62.08    & 73.181   & 100 & complete \\
\textsc{meg\_m\_early\_all} & \textsc{area\_am  }   & 3.628    & 7.228    & 11.097   & 20.436   & 51.409   & 6.956    & 24  & moderate \\
\textsc{meg\_m\_mid\_cl   } & \textsc{area\_am  }   & 0.538    & 1.783    & 4.29     & 10.461   & 28.906   & 30.636   & 96  & complete \\
\textsc{meg\_m\_mid\_mod  } & \textsc{area\_am  }   & 3.015    & 6.298    & 10.5     & 17.722   & 50.199   & 10.37    & 50  & none     \\
\textsc{meg\_m\_mid\_op   } & \textsc{area\_am  }   & 0.18     & 0.728    & 1.019    & 1.367    & 1.919    & 5.305    & 100 & complete \\
\textsc{meg\_m\_late\_cl  } & \textsc{area\_am  }   & 33.204   & 44.3     & 56.89    & 70.504   & 81.454   & 22.709   & 0   & complete \\
\textsc{meg\_m\_late\_mod } & \textsc{area\_am  }   & 1.564    & 2.006    & 2.391    & 2.678    & 4.325    & 13.662   & 100 & complete \\
\textsc{meg\_m\_late\_op  } & \textsc{area\_am  }   & 0.994    & 1.212    & 1.358    & 1.534    & 1.912    & 6.7      & 100 & complete \\
\textsc{meg\_m\_early\_all} & \textsc{area\_mn  }   & 1.055    & 1.512    & 1.916    & 2.441    & 3.609    & 2.308    & 72  & none     \\
\textsc{meg\_m\_mid\_cl   } & \textsc{area\_mn  }   & 0.36     & 1.08     & 1.771    & 2.863    & 5.82     & 3.367    & 83  & moderate \\
\textsc{meg\_m\_mid\_mod  } & \textsc{area\_mn  }   & 0.978    & 1.556    & 2.106    & 2.754    & 4.3      & 2.99     & 82  & moderate \\
\textsc{meg\_m\_mid\_op   } & \textsc{area\_mn  }   & 0.135    & 0.36     & 0.531    & 0.695    & 0.99     & 1.894    & 100 & complete \\
\textsc{meg\_m\_late\_cl  } & \textsc{area\_mn  }   & 4.615    & 4.893    & 5.089    & 5.255    & 5.497    & 5.767    & 100 & complete \\
\textsc{meg\_m\_late\_mod } & \textsc{area\_mn  }   & 0.579    & 0.711    & 0.88     & 1.073    & 1.229    & 4.141    & 100 & complete \\
\textsc{meg\_m\_late\_op  } & \textsc{area\_mn  }   & 0.374    & 0.47     & 0.537    & 0.606    & 0.747    & 3.038    & 100 & complete \\
\textsc{meg\_m\_early\_all} & \textsc{cai\_am   }   & 88.581   & 91.425   & 93.151   & 94.708   & 96.461   & 96.195   & 94  & moderate \\
\textsc{meg\_m\_mid\_cl   } & \textsc{cai\_am   }   & 0        & 13.8     & 29.726   & 43.066   & 68.147   & 45.313   & 80  & moderate \\
\textsc{meg\_m\_mid\_mod  } & \textsc{cai\_am   }   & 47.454   & 56.343   & 62.764   & 69.647   & 76.518   & 67.913   & 70  & none     \\
\textsc{meg\_m\_mid\_op   } & \textsc{cai\_am   }   & 31.414   & 55.725   & 68.681   & 78.647   & 90.587   & 79.077   & 76  & moderate \\
\textsc{meg\_m\_late\_cl  } & \textsc{cai\_am   }   & 23.356   & 29.995   & 34.492   & 38.748   & 43.679   & 42.274   & 92  & moderate \\
\textsc{meg\_m\_late\_mod } & \textsc{cai\_am   }   & 34.103   & 39.93    & 44.579   & 48.01    & 52.168   & 56.478   & 100 & complete \\
\textsc{meg\_m\_late\_op  } & \textsc{cai\_am   }   & 28.152   & 33.405   & 37.427   & 41.956   & 47.17    & 55.901   & 100 & complete \\
\textsc{meg\_m\_early\_all} & \textsc{clumpy    }   & 0.663    & 0.712    & 0.741    & 0.768    & 0.805    & 0.744    & 54  & none     \\
\textsc{meg\_m\_mid\_cl   } & \textsc{clumpy    }   & 0.593    & 0.712    & 0.766    & 0.814    & 0.883    & 0.767    & 51  & none     \\
\textsc{meg\_m\_mid\_mod  } & \textsc{clumpy    }   & 0.67     & 0.721    & 0.759    & 0.786    & 0.816    & 0.762    & 56  & none     \\
\textsc{meg\_m\_mid\_op   } & \textsc{clumpy    }   & 0.392    & 0.544    & 0.612    & 0.661    & 0.758    & 0.708    & 89  & moderate \\
\textsc{meg\_m\_late\_cl  } & \textsc{clumpy    }   & 0.792    & 0.798    & 0.802    & 0.805    & 0.807    & 0.815    & 100 & complete \\
\textsc{meg\_m\_late\_mod } & \textsc{clumpy    }   & 0.574    & 0.608    & 0.639    & 0.658    & 0.671    & 0.79     & 100 & complete \\
\textsc{meg\_m\_late\_op  } & \textsc{clumpy    }   & 0.505    & 0.548    & 0.569    & 0.587    & 0.62     & 0.732    & 100 & complete \\
\textsc{meg\_m\_early\_all} & \textsc{core\_am  }   & 3.105    & 6.828    & 10.221   & 19.6     & 50.349   & 6.877    & 26  & none     \\
\textsc{meg\_m\_mid\_cl   } & \textsc{core\_am  }   & 0        & 0.255    & 1.423    & 4.377    & 19.05    & 18.588   & 95  & complete \\
\textsc{meg\_m\_mid\_mod  } & \textsc{core\_am  }   & 1.955    & 3.957    & 7.53     & 13.811   & 44.024   & 7.149    & 48  & none     \\
\textsc{meg\_m\_mid\_op   } & \textsc{core\_am  }   & 0.09     & 0.471    & 0.7      & 1.023    & 1.556    & 4.598    & 100 & complete \\
\textsc{meg\_m\_late\_cl  } & \textsc{core\_am  }   & 12.447   & 19.746   & 29.169   & 37.601   & 49.705   & 11.566   & 4   & complete \\
\textsc{meg\_m\_late\_mod } & \textsc{core\_am  }   & 0.674    & 0.966    & 1.284    & 1.496    & 2.539    & 8.615    & 100 & complete \\
\textsc{meg\_m\_late\_op  } & \textsc{core\_am  }   & 0.356    & 0.476    & 0.592    & 0.704    & 0.945    & 3.86     & 100 & complete \\
\textsc{meg\_m\_early\_all} & \textsc{core\_mn  }   & 0.943    & 1.411    & 1.793    & 2.27     & 3.395    & 2.22     & 74  & none     \\
\textsc{meg\_m\_mid\_cl   } & \textsc{core\_mn  }   & 0        & 0.168    & 0.504    & 1.068    & 2.696    & 1.526    & 86  & moderate \\
\textsc{meg\_m\_mid\_mod  } & \textsc{core\_mn  }   & 0.487    & 0.963    & 1.28     & 1.795    & 3.027    & 2.031    & 82  & moderate \\
\textsc{meg\_m\_mid\_op   } & \textsc{core\_mn  }   & 0.074    & 0.217    & 0.34     & 0.495    & 0.781    & 1.498    & 100 & complete \\
\textsc{meg\_m\_late\_cl  } & \textsc{core\_mn  }   & 1.083    & 1.517    & 1.763    & 1.98     & 2.252    & 2.438    & 100 & complete \\
\textsc{meg\_m\_late\_mod } & \textsc{core\_mn  }   & 0.212    & 0.288    & 0.39     & 0.507    & 0.62     & 2.339    & 100 & complete \\
\textsc{meg\_m\_late\_op  } & \textsc{core\_mn  }   & 0.116    & 0.164    & 0.2      & 0.247    & 0.316    & 1.698    & 100 & complete \\
\textsc{meg\_m\_early\_all} & \textsc{cpland    }   & 0.037    & 0.076    & 0.126    & 0.187    & 0.292    & 0.304    & 97  & complete \\
\textsc{meg\_m\_mid\_cl   } & \textsc{cpland    }   & 0        & 0        & 0.003    & 0.011    & 0.047    & 0.667    & 100 & complete \\
\textsc{meg\_m\_mid\_mod  } & \textsc{cpland    }   & 0.012    & 0.027    & 0.052    & 0.084    & 0.17     & 0.267    & 100 & complete \\
\textsc{meg\_m\_mid\_op   } & \textsc{cpland    }   & 0        & 0.001    & 0.002    & 0.003    & 0.007    & 0.199    & 100 & complete \\
\textsc{meg\_m\_late\_cl  } & \textsc{cpland    }   & 0.727    & 1.01     & 1.207    & 1.396    & 1.593    & 0.497    & 0   & complete \\
\textsc{meg\_m\_late\_mod } & \textsc{cpland    }   & 0.03     & 0.048    & 0.077    & 0.121    & 0.187    & 0.166    & 91  & moderate \\
\textsc{meg\_m\_late\_op  } & \textsc{cpland    }   & 0.006    & 0.013    & 0.02     & 0.031    & 0.049    & 0.057    & 98  & complete \\
\textsc{meg\_m\_early\_all} & \textsc{cwed      }   & 0.077    & 0.146    & 0.223    & 0.305    & 0.434    & 0.346    & 86  & moderate \\
\textsc{meg\_m\_mid\_cl   } & \textsc{cwed      }   & 0.001    & 0.004    & 0.013    & 0.034    & 0.094    & 1.279    & 100 & complete \\
\textsc{meg\_m\_mid\_mod  } & \textsc{cwed      }   & 0.029    & 0.059    & 0.095    & 0.146    & 0.239    & 0.354    & 100 & complete \\
\textsc{meg\_m\_mid\_op   } & \textsc{cwed      }   & 0.001    & 0.003    & 0.005    & 0.008    & 0.017    & 0.296    & 100 & complete \\
\textsc{meg\_m\_late\_cl  } & \textsc{cwed      }   & 2.956    & 3.048    & 3.115    & 3.188    & 3.305    & 0.863    & 0   & complete \\
\textsc{meg\_m\_late\_mod } & \textsc{cwed      }   & 0.137    & 0.194    & 0.253    & 0.347    & 0.487    & 0.259    & 52  & none     \\
\textsc{meg\_m\_late\_op  } & \textsc{cwed      }   & 0.041    & 0.073    & 0.105    & 0.153    & 0.222    & 0.131    & 66  & none     \\
\textsc{meg\_m\_early\_all} & \textsc{econ\_am  }   & 41.508   & 43.81    & 45.143   & 46.558   & 48.533   & 31.665   & 0   & complete \\
\textsc{meg\_m\_mid\_cl   } & \textsc{econ\_am  }   & 21.534   & 26.864   & 30.245   & 34.405   & 42.426   & 27.653   & 29  & none     \\
\textsc{meg\_m\_mid\_mod  } & \textsc{econ\_am  }   & 29.241   & 31.493   & 32.669   & 33.738   & 35.37    & 27.518   & 1   & complete \\
\textsc{meg\_m\_mid\_op   } & \textsc{econ\_am  }   & 24.101   & 28.876   & 31.625   & 34.026   & 37.128   & 29.371   & 30  & none     \\
\textsc{meg\_m\_late\_cl  } & \textsc{econ\_am  }   & 31.647   & 33.211   & 34.802   & 36.461   & 38.351   & 30.695   & 2   & complete \\
\textsc{meg\_m\_late\_mod } & \textsc{econ\_am  }   & 25.824   & 27.055   & 28.062   & 29.552   & 31.362   & 30.264   & 86  & moderate \\
\textsc{meg\_m\_late\_op  } & \textsc{econ\_am  }   & 30.14    & 31.328   & 32.172   & 32.935   & 34.286   & 34.329   & 96  & complete \\
\textsc{meg\_m\_early\_all} & \textsc{econ\_mn  }   & 42.165   & 43.632   & 44.68    & 45.947   & 47.659   & 29.691   & 0   & complete \\
\textsc{meg\_m\_mid\_cl   } & \textsc{econ\_mn  }   & 23.512   & 28.877   & 31.361   & 34.709   & 41.088   & 28.251   & 20  & moderate \\
\textsc{meg\_m\_mid\_mod  } & \textsc{econ\_mn  }   & 31.408   & 32.697   & 33.424   & 34.158   & 35.246   & 28.015   & 0   & complete \\
\textsc{meg\_m\_mid\_op   } & \textsc{econ\_mn  }   & 27.183   & 31.09    & 33.167   & 34.641   & 37.154   & 29.292   & 12  & moderate \\
\textsc{meg\_m\_late\_cl  } & \textsc{econ\_mn  }   & 32.304   & 33.666   & 35.045   & 36.573   & 38.473   & 28.041   & 0   & complete \\
\textsc{meg\_m\_late\_mod } & \textsc{econ\_mn  }   & 29.902   & 31.172   & 32.039   & 32.966   & 34.011   & 29.722   & 4   & complete \\
\textsc{meg\_m\_late\_op  } & \textsc{econ\_mn  }   & 33.045   & 33.779   & 34.403   & 34.959   & 36.107   & 33.197   & 8   & moderate \\
\textsc{meg\_m\_early\_all} & \textsc{ed        }   & 0.17     & 0.323    & 0.492    & 0.671    & 0.986    & 1.112    & 100 & complete \\
\textsc{meg\_m\_mid\_cl   } & \textsc{ed        }   & 0.003    & 0.015    & 0.043    & 0.109    & 0.317    & 4.586    & 100 & complete \\
\textsc{meg\_m\_mid\_mod  } & \textsc{ed        }   & 0.095    & 0.179    & 0.291    & 0.452    & 0.698    & 1.285    & 100 & complete \\
\textsc{meg\_m\_mid\_op   } & \textsc{ed        }   & 0.002    & 0.009    & 0.017    & 0.027    & 0.055    & 1.009    & 100 & complete \\
\textsc{meg\_m\_late\_cl  } & \textsc{ed        }   & 8.346    & 8.736    & 9.026    & 9.29     & 9.526    & 2.941    & 0   & complete \\
\textsc{meg\_m\_late\_mod } & \textsc{ed        }   & 0.451    & 0.655    & 0.853    & 1.181    & 1.721    & 0.864    & 51  & none     \\
\textsc{meg\_m\_late\_op  } & \textsc{ed        }   & 0.123    & 0.219    & 0.314    & 0.473    & 0.679    & 0.387    & 65  & none     \\
\textsc{meg\_m\_early\_all} & \textsc{gyrate\_am}   & 84.087   & 115.98   & 149.282  & 197.262  & 337.932  & 118.307  & 27  & none     \\
\textsc{meg\_m\_mid\_cl   } & \textsc{gyrate\_am}   & 33.839   & 61.254   & 97.498   & 147.165  & 267.034  & 267.666  & 96  & complete \\
\textsc{meg\_m\_mid\_mod  } & \textsc{gyrate\_am}   & 76.687   & 108.916  & 143.065  & 191.809  & 338.124  & 154.277  & 57  & none     \\
\textsc{meg\_m\_mid\_op   } & \textsc{gyrate\_am}   & 16.771   & 35.411   & 45.113   & 52.971   & 66.268   & 109.964  & 100 & complete \\
\textsc{meg\_m\_late\_cl  } & \textsc{gyrate\_am}   & 269.296  & 316.38   & 364.264  & 397.701  & 434.304  & 228.496  & 0   & complete \\
\textsc{meg\_m\_late\_mod } & \textsc{gyrate\_am}   & 54.289   & 61.363   & 67.218   & 72.109   & 83.84    & 183.996  & 100 & complete \\
\textsc{meg\_m\_late\_op  } & \textsc{gyrate\_am}   & 42.38    & 47.794   & 51.153   & 54.825   & 60.645   & 146.917  & 100 & complete \\
\textsc{meg\_m\_early\_all} & \textsc{iji       }   & 55.663   & 57.731   & 59.045   & 60.002   & 61.395   & 58.339   & 37  & none     \\
\textsc{meg\_m\_mid\_cl   } & \textsc{iji       }   & 21.661   & 40.001   & 48.425   & 53.241   & 57.698   & 59.156   & 99  & complete \\
\textsc{meg\_m\_mid\_mod  } & \textsc{iji       }   & 52.592   & 55.712   & 57.882   & 59.755   & 61.76    & 62.004   & 97  & complete \\
\textsc{meg\_m\_mid\_op   } & \textsc{iji       }   & 27.819   & 43.434   & 50.024   & 54.716   & 59.112   & 62.31    & 100 & complete \\
\textsc{meg\_m\_late\_cl  } & \textsc{iji       }   & 54.831   & 56.852   & 58.156   & 59.496   & 60.673   & 60.085   & 87  & moderate \\
\textsc{meg\_m\_late\_mod } & \textsc{iji       }   & 51.354   & 53.442   & 55.331   & 57.389   & 59.464   & 61.664   & 100 & complete \\
\textsc{meg\_m\_late\_op  } & \textsc{iji       }   & 56.456   & 58.288   & 59.237   & 60.245   & 61.373   & 58.712   & 35  & none     \\
\textsc{meg\_m\_early\_all} & \textsc{pd        }   & 0.03     & 0.046    & 0.067    & 0.089    & 0.124    & 0.137    & 99  & complete \\
\textsc{meg\_m\_mid\_cl   } & \textsc{pd        }   & 0.001    & 0.002    & 0.006    & 0.013    & 0.034    & 0.437    & 100 & complete \\
\textsc{meg\_m\_mid\_mod  } & \textsc{pd        }   & 0.016    & 0.027    & 0.04     & 0.056    & 0.08     & 0.132    & 100 & complete \\
\textsc{meg\_m\_mid\_op   } & \textsc{pd        }   & 0.001    & 0.003    & 0.005    & 0.008    & 0.014    & 0.133    & 100 & complete \\
\textsc{meg\_m\_late\_cl  } & \textsc{pd        }   & 0.623    & 0.659    & 0.688    & 0.716    & 0.741    & 0.204    & 0   & complete \\
\textsc{meg\_m\_late\_mod } & \textsc{pd        }   & 0.126    & 0.165    & 0.203    & 0.248    & 0.321    & 0.071    & 0   & complete \\
\textsc{meg\_m\_late\_op  } & \textsc{pd        }   & 0.042    & 0.074    & 0.099    & 0.139    & 0.189    & 0.034    & 3   & complete \\
\textsc{meg\_m\_early\_all} & \textsc{pland     }   & 0.039    & 0.082    & 0.135    & 0.199    & 0.313    & 0.317    & 96  & complete \\
\textsc{meg\_m\_mid\_cl   } & \textsc{pland     }   & 0        & 0.003    & 0.011    & 0.035    & 0.106    & 1.473    & 100 & complete \\
\textsc{meg\_m\_mid\_mod  } & \textsc{pland     }   & 0.022    & 0.047    & 0.086    & 0.135    & 0.246    & 0.394    & 100 & complete \\
\textsc{meg\_m\_mid\_op   } & \textsc{pland     }   & 0        & 0.001    & 0.003    & 0.005    & 0.01     & 0.251    & 100 & complete \\
\textsc{meg\_m\_late\_cl  } & \textsc{pland     }   & 3.1      & 3.363    & 3.507    & 3.608    & 3.717    & 1.175    & 0   & complete \\
\textsc{meg\_m\_late\_mod } & \textsc{pland     }   & 0.082    & 0.12     & 0.171    & 0.254    & 0.386    & 0.294    & 84  & moderate \\
\textsc{meg\_m\_late\_op  } & \textsc{pland     }   & 0.018    & 0.035    & 0.052    & 0.082    & 0.124    & 0.102    & 88  & moderate \\
\textsc{meg\_m\_early\_all} & \textsc{shape\_am }   & 1.526    & 1.696    & 1.902    & 2.212    & 2.992    & 1.708    & 27  & none     \\
\textsc{meg\_m\_mid\_cl   } & \textsc{shape\_am }   & 1.136    & 1.392    & 1.613    & 1.93     & 2.536    & 2.599    & 97  & complete \\
\textsc{meg\_m\_mid\_mod  } & \textsc{shape\_am }   & 1.469    & 1.648    & 1.828    & 2.135    & 2.977    & 1.96     & 64  & none     \\
\textsc{meg\_m\_mid\_op   } & \textsc{shape\_am }   & 1        & 1.176    & 1.295    & 1.382    & 1.524    & 1.73     & 99  & complete \\
\textsc{meg\_m\_late\_cl  } & \textsc{shape\_am }   & 2.524    & 2.717    & 2.921    & 3.084    & 3.275    & 2.076    & 0   & complete \\
\textsc{meg\_m\_late\_mod } & \textsc{shape\_am }   & 1.349    & 1.405    & 1.445    & 1.483    & 1.563    & 1.995    & 100 & complete \\
\textsc{meg\_m\_late\_op  } & \textsc{shape\_am }   & 1.244    & 1.297    & 1.327    & 1.358    & 1.409    & 2.054    & 100 & complete \\
\textsc{meg\_m\_early\_all} & \textsc{shape\_mn }   & 1.244    & 1.295    & 1.329    & 1.369    & 1.433    & 1.393    & 86  & moderate \\
\textsc{meg\_m\_mid\_cl   } & \textsc{shape\_mn }   & 1.078    & 1.248    & 1.33     & 1.419    & 1.615    & 1.467    & 85  & moderate \\
\textsc{meg\_m\_mid\_mod  } & \textsc{shape\_mn }   & 1.213    & 1.28     & 1.326    & 1.379    & 1.466    & 1.456    & 95  & complete \\
\textsc{meg\_m\_mid\_op   } & \textsc{shape\_mn }   & 1        & 1.087    & 1.142    & 1.197    & 1.282    & 1.418    & 100 & complete \\
\textsc{meg\_m\_late\_cl  } & \textsc{shape\_mn }   & 1.496    & 1.505    & 1.512    & 1.52     & 1.53     & 1.573    & 100 & complete \\
\textsc{meg\_m\_late\_mod } & \textsc{shape\_mn }   & 1.143    & 1.172    & 1.196    & 1.234    & 1.262    & 1.538    & 100 & complete \\
\textsc{meg\_m\_late\_op  } & \textsc{shape\_mn }   & 1.094    & 1.117    & 1.133    & 1.151    & 1.177    & 1.645    & 100 & complete \\
\textsc{meg\_m\_early\_all} & \textsc{simi\_mn  }   & 3044.618 & 4132.53  & 4761.782 & 5369.141 & 6134.154 & 3432.092 & 11  & moderate \\
\textsc{meg\_m\_mid\_cl   } & \textsc{simi\_mn  }   & 494.624  & 1639.289 & 2536.005 & 3307.552 & 5480.93  & 3234.009 & 74  & none     \\
\textsc{meg\_m\_mid\_mod  } & \textsc{simi\_mn  }   & 1885.353 & 2469.185 & 2983.79  & 3441.035 & 4126.301 & 4227.928 & 97  & complete \\
\textsc{meg\_m\_mid\_op   } & \textsc{simi\_mn  }   & 372.719  & 728.632  & 2115.797 & 3601.215 & 6289.332 & 5938.995 & 94  & moderate \\
\textsc{meg\_m\_late\_cl  } & \textsc{simi\_mn  }   & 2008.47  & 2101.485 & 2161.885 & 2233.444 & 2384.24  & 1835.667 & 0   & complete \\
\textsc{meg\_m\_late\_mod } & \textsc{simi\_mn  }   & 1827.638 & 2045.721 & 2199.859 & 2367.319 & 2618.151 & 2707.606 & 99  & complete \\
\textsc{meg\_m\_late\_op  } & \textsc{simi\_mn  }   & 1910.171 & 2287.454 & 2590.877 & 2905.347 & 3439.315 & 4404.244 & 100 & complete \\
\textsc{meg\_m\_early\_all} & \textsc{te  		}   & 30960    & 58620    & 89340    & 121800   & 179040   & 201870   & 100 & complete \\
\textsc{meg\_m\_mid\_cl   } & \textsc{te  		}   & 540      & 2745     & 7890     & 19875    & 57465    & 832620   & 100 & complete \\
\textsc{meg\_m\_mid\_mod  } & \textsc{te 		}   & 17280    & 32490    & 52890    & 82110    & 126660   & 233340   & 100 & complete \\
\textsc{meg\_m\_mid\_op   } & \textsc{te 		}   & 450      & 1680     & 3060     & 4830     & 10065    & 183120   & 100 & complete \\
\textsc{meg\_m\_late\_cl  } & \textsc{te 		}   & 1515240  & 1585950  & 1638750  & 1686570  & 1729410  & 533970   & 0   & complete \\
\textsc{meg\_m\_late\_mod } & \textsc{te 		}   & 81960    & 118860   & 154860   & 214380   & 312360   & 156840   & 51  & none     \\
\textsc{meg\_m\_late\_op  } & \textsc{te 		}   & 22260    & 39780    & 57030    & 85830    & 123330   & 70260    & 65  & none    

\end{longtable}
\end{footnotesize}
\end{center}
\end{landscape}

\pagestyle{headings}

%%%%%%%%%%%%%%%%%%%%%%%%%%%%%%%%%%%%%%%%%%%%%%%%%%%%%%%%%%%%%%%%%%%%%%%%%%%%%%%%%%%%%%%%%%%%%%%%
%%%%%%%%%%%%%%%%%%%%%%%%%%%%%%%%%%%%%%%%%%%%%%%%%%%%%%%%%%%%%%%%%%%%%%%%%%%%%%%%%%%%%%%%%%%%%%%%

\pagestyle{empty}
\begin{landscape}
\footnotesize
\begin{center}
\begin{footnotesize}
\begin{longtable}{llrrrrr|rrr}
\caption{Unabridged results for class-level metrics for Mixed Evergreen - Xeric (\textsc{meg\_x}) calculated with \textsc{Fragstats}. This table shows the range of variability in landscape structure. Included are the $5^{\text{th}}$ percentile, $25^{\text{th}}$ percentile, $50^{\text{th}}$ percentile, $75^{\text{th}}$ percentile, and $95^{\text{th}}$ percentiles of the distribution, as well as the current class value, the current percentile range of variability (\%RV) for that proportion, and the departure classification. For seral stage abbreviations, see Table~\ref{condtable}.} \\
\label{tab:fragclass_megx} \\

\hline 
\textbf{\begin{tabular}[c]{@{}l@{}}Cover-Seral Stage Type\end{tabular}}  &   
\textbf{\begin{tabular}[c]{@{}l@{}}Landscape\\ Metric\end{tabular}}  &   
\textbf{$5^{\text{th}}$ } &   
\textbf{$25^{\text{th}}$ } &   
\textbf{$50^{\text{th}}$ } &   
\textbf{$75^{\text{th}}$ } &   
\textbf{$95^{\text{th}}$ }  &  
\textbf{\begin{tabular}[c]{@{}l@{}}Current\\ Value\end{tabular}} &   
\textbf{\begin{tabular}[c]{@{}l@{}}Current\\ \%RV\end{tabular}} &   
\textbf{\begin{tabular}[c]{@{}l@{}}Departure\end{tabular}} \\  \\ \hline 
\endfirsthead

\multicolumn{10}{c}{{\bfseries \tablename\ \thetable{} -- continued from previous page}} \\
\hline 
\textbf{\begin{tabular}[c]{@{}l@{}}Cover-Seral Stage Type\end{tabular}}  &   
\textbf{\begin{tabular}[c]{@{}l@{}}Landscape\\ Metric\end{tabular}}  &   
\textbf{$5^{\text{th}}$ } &   
\textbf{$25^{\text{th}}$ } &   
\textbf{$50^{\text{th}}$ } &   
\textbf{$75^{\text{th}}$ } &   
\textbf{$95^{\text{th}}$ }  &  
\textbf{\begin{tabular}[c]{@{}l@{}}Current\\ Value\end{tabular}} &   
\textbf{\begin{tabular}[c]{@{}l@{}}Current\\ \%RV\end{tabular}} &   
\textbf{\begin{tabular}[c]{@{}l@{}}Departure\end{tabular}} \\  \\ \hline \endhead

\hline \multicolumn{10}{|l|}{{Continued on next page}} \\ \hline
\endfoot

\hline \hline
\endlastfoot

\textsc{meg\_x\_early\_all} & \textsc{ai        }    & 72.701   & 76.472   & 79.257   & 81.635   & 84.313   & 77.01    & 30  & none     \\
\textsc{meg\_x\_mid\_cl   } & \textsc{ai        }    & 61.111   & 75.203   & 80.353   & 85.185   & 91.541   & 81.911   & 59  & none     \\
\textsc{meg\_x\_mid\_mod  } & \textsc{ai        }    & 73.541   & 77.788   & 80.691   & 83.087   & 86.29    & 78.244   & 29  & none     \\
\textsc{meg\_x\_mid\_op   } & \textsc{ai        }    & 38.611   & 59.077   & 64.706   & 70.643   & 85       & 78.005   & 90  & moderate \\
\textsc{meg\_x\_late\_cl  } & \textsc{ai        }    & 81.468   & 82.13    & 82.558   & 82.864   & 83.311   & 77.31    & 0   & complete \\
\textsc{meg\_x\_late\_mod } & \textsc{ai        }    & 65.011   & 67.168   & 68.426   & 69.883   & 71.961   & 75.246   & 100 & complete \\
\textsc{meg\_x\_late\_op  } & \textsc{ai        }    & 54.225   & 58.989   & 61.803   & 64.382   & 67.663   & 72.5     & 100 & complete \\
\textsc{meg\_x\_early\_all} & \textsc{area\_am  }    & 5.556    & 11.388   & 17.446   & 27.254   & 54.154   & 14.598   & 37  & none     \\
\textsc{meg\_x\_mid\_cl   } & \textsc{area\_am  }    & 0.594    & 2.323    & 6.606    & 17.429   & 40.22    & 24.779   & 87  & moderate \\
\textsc{meg\_x\_mid\_mod  } & \textsc{area\_am  }    & 4.773    & 11.22    & 17.124   & 28.033   & 58.821   & 11.386   & 27  & none     \\
\textsc{meg\_x\_mid\_op   } & \textsc{area\_am  }    & 0.09     & 0.802    & 1.048    & 1.308    & 2.106    & 9.983    & 100 & complete \\
\textsc{meg\_x\_late\_cl  } & \textsc{area\_am  }    & 30.886   & 35.307   & 38.969   & 42.228   & 45.274   & 16.592   & 0   & complete \\
\textsc{meg\_x\_late\_mod } & \textsc{area\_am  }    & 2.354    & 2.887    & 3.623    & 5.475    & 13.176   & 9.979    & 91  & moderate \\
\textsc{meg\_x\_late\_op  } & \textsc{area\_am  }    & 1.13     & 1.412    & 1.719    & 2.201    & 4.506    & 4.586    & 96  & complete \\
\textsc{meg\_x\_early\_all} & \textsc{area\_mn  }    & 1.543    & 2.001    & 2.544    & 3.256    & 4.245    & 2.766    & 58  & none     \\
\textsc{meg\_x\_mid\_cl   } & \textsc{area\_mn  }    & 0.368    & 1.17     & 2.269    & 3.602    & 8.915    & 4.663    & 84  & moderate \\
\textsc{meg\_x\_mid\_mod  } & \textsc{area\_mn  }    & 1.548    & 2.219    & 2.852    & 3.8      & 5.26     & 3.118    & 58  & none     \\
\textsc{meg\_x\_mid\_op   } & \textsc{area\_mn  }    & 0.09     & 0.392    & 0.6      & 0.8      & 1.166    & 3.265    & 100 & complete \\
\textsc{meg\_x\_late\_cl  } & \textsc{area\_mn  }    & 4.218    & 4.434    & 4.566    & 4.696    & 4.957    & 2.351    & 0   & complete \\
\textsc{meg\_x\_late\_mod } & \textsc{area\_mn  }    & 0.902    & 1.083    & 1.225    & 1.382    & 1.539    & 2.212    & 100 & complete \\
\textsc{meg\_x\_late\_op  } & \textsc{area\_mn  }    & 0.406    & 0.525    & 0.629    & 0.736    & 0.917    & 1.786    & 100 & complete \\
\textsc{meg\_x\_early\_all} & \textsc{cai\_am   }    & 89.973   & 92.707   & 94.511   & 95.941   & 98.009   & 94.59    & 52  & none     \\
\textsc{meg\_x\_mid\_cl   } & \textsc{cai\_am   }    & 0        & 13.726   & 29.081   & 42.485   & 65.854   & 42.672   & 76  & moderate \\
\textsc{meg\_x\_mid\_mod  } & \textsc{cai\_am   }    & 47.2     & 56.219   & 61.699   & 66.495   & 72.725   & 72.905   & 96  & complete \\
\textsc{meg\_x\_mid\_op   } & \textsc{cai\_am   }    & 28.377   & 57.631   & 74.242   & 85.714   & 100      & 80.368   & 63  & none     \\
\textsc{meg\_x\_late\_cl  } & \textsc{cai\_am   }    & 25.926   & 30.867   & 35.212   & 38.469   & 42.32    & 41.336   & 91  & moderate \\
\textsc{meg\_x\_late\_mod } & \textsc{cai\_am   }    & 45.063   & 49.907   & 52.519   & 54.813   & 58.285   & 61.586   & 100 & complete \\
\textsc{meg\_x\_late\_op  } & \textsc{cai\_am   }    & 34.659   & 40.809   & 44.918   & 48.721   & 56.289   & 53.854   & 92  & moderate \\
\textsc{meg\_x\_early\_all} & \textsc{clumpy    }    & 0.727    & 0.764    & 0.792    & 0.816    & 0.843    & 0.769    & 30  & none     \\
\textsc{meg\_x\_mid\_cl   } & \textsc{clumpy    }    & 0.61     & 0.75     & 0.803    & 0.852    & 0.915    & 0.816    & 57  & none     \\
\textsc{meg\_x\_mid\_mod  } & \textsc{clumpy    }    & 0.735    & 0.778    & 0.807    & 0.831    & 0.863    & 0.782    & 29  & none     \\
\textsc{meg\_x\_mid\_op   } & \textsc{clumpy    }    & 0.386    & 0.591    & 0.647    & 0.706    & 0.85     & 0.779    & 90  & moderate \\
\textsc{meg\_x\_late\_cl  } & \textsc{clumpy    }    & 0.809    & 0.816    & 0.82     & 0.823    & 0.827    & 0.772    & 0   & complete \\
\textsc{meg\_x\_late\_mod } & \textsc{clumpy    }    & 0.65     & 0.671    & 0.683    & 0.698    & 0.719    & 0.752    & 100 & complete \\
\textsc{meg\_x\_late\_op  } & \textsc{clumpy    }    & 0.542    & 0.59     & 0.618    & 0.644    & 0.676    & 0.725    & 100 & complete \\
\textsc{meg\_x\_early\_all} & \textsc{core\_am  }    & 4.752    & 10.386   & 16.486   & 26.101   & 51.015   & 13.815   & 39  & none     \\
\textsc{meg\_x\_mid\_cl   } & \textsc{core\_am  }    & 0        & 0.342    & 1.988    & 5.957    & 21.245   & 11.958   & 89  & moderate \\
\textsc{meg\_x\_mid\_mod  } & \textsc{core\_am  }    & 2.663    & 7.03     & 11.929   & 20.827   & 48.023   & 8.721    & 32  & none     \\
\textsc{meg\_x\_mid\_op   } & \textsc{core\_am  }    & 0.05     & 0.484    & 0.783    & 1.101    & 1.9      & 8.332    & 100 & complete \\
\textsc{meg\_x\_late\_cl  } & \textsc{core\_am  }    & 11.967   & 16.111   & 18.989   & 22.053   & 24.837   & 8.037    & 0   & complete \\
\textsc{meg\_x\_late\_mod } & \textsc{core\_am  }    & 1.336    & 1.66     & 2.118    & 3.309    & 8.351    & 7.024    & 93  & moderate \\
\textsc{meg\_x\_late\_op  } & \textsc{core\_am  }    & 0.503    & 0.671    & 0.863    & 1.191    & 2.916    & 2.724    & 95  & complete \\
\textsc{meg\_x\_early\_all} & \textsc{core\_mn  }    & 1.432    & 1.9      & 2.423    & 3.088    & 3.987    & 2.617    & 59  & none     \\
\textsc{meg\_x\_mid\_cl   } & \textsc{core\_mn  }    & 0        & 0.173    & 0.606    & 1.289    & 4.176    & 1.99     & 87  & moderate \\
\textsc{meg\_x\_mid\_mod  } & \textsc{core\_mn  }    & 0.757    & 1.302    & 1.737    & 2.483    & 3.685    & 2.273    & 69  & none     \\
\textsc{meg\_x\_mid\_op   } & \textsc{core\_mn  }    & 0.064    & 0.24     & 0.414    & 0.6      & 1.006    & 2.624    & 100 & complete \\
\textsc{meg\_x\_late\_cl  } & \textsc{core\_mn  }    & 1.153    & 1.401    & 1.609    & 1.79     & 1.972    & 0.972    & 1   & complete \\
\textsc{meg\_x\_late\_mod } & \textsc{core\_mn  }    & 0.426    & 0.565    & 0.652    & 0.729    & 0.842    & 1.362    & 100 & complete \\
\textsc{meg\_x\_late\_op  } & \textsc{core\_mn  }    & 0.156    & 0.219    & 0.282    & 0.339    & 0.479    & 0.962    & 100 & complete \\
\textsc{meg\_x\_early\_all} & \textsc{cpland    }    & 0.058    & 0.114    & 0.164    & 0.22     & 0.322    & 0.372    & 99  & complete \\
\textsc{meg\_x\_mid\_cl   } & \textsc{cpland    }    & 0        & 0.001    & 0.004    & 0.013    & 0.038    & 0.78     & 100 & complete \\
\textsc{meg\_x\_mid\_mod  } & \textsc{cpland    }    & 0.02     & 0.045    & 0.072    & 0.112    & 0.176    & 0.257    & 100 & complete \\
\textsc{meg\_x\_mid\_op   } & \textsc{cpland    }    & 0        & 0.001    & 0.002    & 0.003    & 0.005    & 0.387    & 100 & complete \\
\textsc{meg\_x\_late\_cl  } & \textsc{cpland    }    & 0.711    & 0.913    & 1.098    & 1.222    & 1.406    & 0.198    & 0   & complete \\
\textsc{meg\_x\_late\_mod } & \textsc{cpland    }    & 0.059    & 0.103    & 0.131    & 0.175    & 0.238    & 0.088    & 18  & moderate \\
\textsc{meg\_x\_late\_op  } & \textsc{cpland    }    & 0.007    & 0.012    & 0.019    & 0.031    & 0.048    & 0.027    & 68  & none     \\
\textsc{meg\_x\_early\_all} & \textsc{cwed      }    & 0.093    & 0.159    & 0.217    & 0.285    & 0.383    & 0.377    & 95  & complete \\
\textsc{meg\_x\_mid\_cl   } & \textsc{cwed      }    & 0.001    & 0.004    & 0.015    & 0.036    & 0.081    & 1.231    & 100 & complete \\
\textsc{meg\_x\_mid\_mod  } & \textsc{cwed      }    & 0.04     & 0.076    & 0.11     & 0.152    & 0.216    & 0.254    & 99  & complete \\
\textsc{meg\_x\_mid\_op   } & \textsc{cwed      }    & 0        & 0.002    & 0.003    & 0.006    & 0.01     & 0.409    & 100 & complete \\
\textsc{meg\_x\_late\_cl  } & \textsc{cwed      }    & 2.346    & 2.421    & 2.474    & 2.532    & 2.609    & 0.368    & 0   & complete \\
\textsc{meg\_x\_late\_mod } & \textsc{cwed      }    & 0.155    & 0.226    & 0.279    & 0.372    & 0.485    & 0.119    & 1   & complete \\
\textsc{meg\_x\_late\_op  } & \textsc{cwed      }    & 0.027    & 0.046    & 0.069    & 0.101    & 0.15     & 0.058    & 40  & none     \\
\textsc{meg\_x\_early\_all} & \textsc{econ\_am  }    & 40.691   & 42.545   & 43.677   & 44.909   & 47.382   & 29.465   & 0   & complete \\
\textsc{meg\_x\_mid\_cl   } & \textsc{econ\_am  }    & 21.819   & 29.172   & 33.152   & 36.712   & 44.504   & 28.279   & 22  & moderate \\
\textsc{meg\_x\_mid\_mod  } & \textsc{econ\_am  }    & 30.945   & 32.781   & 33.821   & 34.924   & 37.191   & 23.881   & 0   & complete \\
\textsc{meg\_x\_mid\_op   } & \textsc{econ\_am  }    & 16.451   & 24.061   & 28.665   & 31.943   & 37.053   & 27.831   & 46  & none     \\
\textsc{meg\_x\_late\_cl  } & \textsc{econ\_am  }    & 33.238   & 34.297   & 35.253   & 36.258   & 37.897   & 25.799   & 0   & complete \\
\textsc{meg\_x\_late\_mod } & \textsc{econ\_am  }    & 23.774   & 25.105   & 26.105   & 27.086   & 29.196   & 23.608   & 4   & complete \\
\textsc{meg\_x\_late\_op  } & \textsc{econ\_am  }    & 25.878   & 27.809   & 28.774   & 29.653   & 31.185   & 27.951   & 28  & none     \\
\textsc{meg\_x\_early\_all} & \textsc{econ\_mn  }    & 41.399   & 42.601   & 43.601   & 44.565   & 45.946   & 29.867   & 0   & complete \\
\textsc{meg\_x\_mid\_cl   } & \textsc{econ\_mn  }    & 24.215   & 29.583   & 32.143   & 34.987   & 40.428   & 24.424   & 6   & moderate \\
\textsc{meg\_x\_mid\_mod  } & \textsc{econ\_mn  }    & 31.261   & 32.352   & 33.007   & 33.738   & 34.91    & 22.987   & 0   & complete \\
\textsc{meg\_x\_mid\_op   } & \textsc{econ\_mn  }    & 22.294   & 27.856   & 30.518   & 32.936   & 37.322   & 27.694   & 25  & moderate \\
\textsc{meg\_x\_late\_cl  } & \textsc{econ\_mn  }    & 28.899   & 29.962   & 31.174   & 32.229   & 33.914   & 22.778   & 0   & complete \\
\textsc{meg\_x\_late\_mod } & \textsc{econ\_mn  }    & 26.879   & 27.937   & 28.802   & 29.456   & 30.75    & 23.014   & 0   & complete \\
\textsc{meg\_x\_late\_op  } & \textsc{econ\_mn  }    & 29.827   & 30.758   & 31.381   & 31.933   & 33.076   & 28.024   & 0   & complete \\
\textsc{meg\_x\_early\_all} & \textsc{ed        }    & 0.21     & 0.368    & 0.498    & 0.653    & 0.895    & 1.247    & 100 & complete \\
\textsc{meg\_x\_mid\_cl   } & \textsc{ed        }    & 0.003    & 0.014    & 0.044    & 0.108    & 0.24     & 4.503    & 100 & complete \\
\textsc{meg\_x\_mid\_mod  } & \textsc{ed        }    & 0.123    & 0.227    & 0.33     & 0.452    & 0.639    & 1.063    & 100 & complete \\
\textsc{meg\_x\_mid\_op   } & \textsc{ed        }    & 0.001    & 0.007    & 0.013    & 0.02     & 0.035    & 1.464    & 100 & complete \\
\textsc{meg\_x\_late\_cl  } & \textsc{ed        }    & 6.756    & 7.093    & 7.308    & 7.517    & 7.732    & 1.492    & 0   & complete \\
\textsc{meg\_x\_late\_mod } & \textsc{ed        }    & 0.565    & 0.843    & 1.064    & 1.393    & 1.825    & 0.501    & 2   & complete \\
\textsc{meg\_x\_late\_op  } & \textsc{ed        }    & 0.094    & 0.15     & 0.231    & 0.348    & 0.518    & 0.199    & 40  & none     \\
\textsc{meg\_x\_early\_all} & \textsc{gyrate\_am}    & 100.375  & 139.914  & 177.113  & 217.757  & 296.803  & 154.026  & 34  & none     \\
\textsc{meg\_x\_mid\_cl   } & \textsc{gyrate\_am}    & 32.995   & 66.385   & 115.926  & 184.133  & 297.15   & 222.17   & 85  & moderate \\
\textsc{meg\_x\_mid\_mod  } & \textsc{gyrate\_am}    & 92.719   & 143.955  & 180.537  & 226.645  & 312.098  & 143.757  & 25  & moderate \\
\textsc{meg\_x\_mid\_op   } & \textsc{gyrate\_am}    & 15       & 37.013   & 44.645   & 50.684   & 62.365   & 144.494  & 100 & complete \\
\textsc{meg\_x\_late\_cl  } & \textsc{gyrate\_am}    & 243.302  & 257.966  & 273.819  & 286.396  & 296.381  & 186.881  & 0   & complete \\
\textsc{meg\_x\_late\_mod } & \textsc{gyrate\_am}    & 66.103   & 72.627   & 80.754   & 95.276   & 131.516  & 173.458  & 99  & complete \\
\textsc{meg\_x\_late\_op  } & \textsc{gyrate\_am}    & 43.116   & 49.571   & 55.278   & 63.324   & 87.558   & 103.556  & 98  & complete \\
\textsc{meg\_x\_early\_all} & \textsc{iji       }    & 51.99    & 53.861   & 55.094   & 56.149   & 57.753   & 56.952   & 88  & moderate \\
\textsc{meg\_x\_mid\_cl   } & \textsc{iji       }    & 20.396   & 34.508   & 43.845   & 48.869   & 53.786   & 57.886   & 100 & complete \\
\textsc{meg\_x\_mid\_mod  } & \textsc{iji       }    & 50.2     & 53.289   & 54.93    & 56.691   & 59.542   & 56.24    & 70  & none     \\
\textsc{meg\_x\_mid\_op   } & \textsc{iji       }    & 14.555   & 33.191   & 40.202   & 45.951   & 51.446   & 58.958   & 100 & complete \\
\textsc{meg\_x\_late\_cl  } & \textsc{iji       }    & 52.48    & 53.965   & 55.019   & 55.994   & 57.221   & 51.401   & 1   & complete \\
\textsc{meg\_x\_late\_mod } & \textsc{iji       }    & 45.431   & 47.455   & 49.117   & 50.785   & 52.994   & 52.147   & 90  & moderate \\
\textsc{meg\_x\_late\_op  } & \textsc{iji       }    & 50.434   & 52.822   & 53.898   & 54.858   & 55.878   & 53.11    & 31  & none     \\
\textsc{meg\_x\_early\_all} & \textsc{pd        }    & 0.03     & 0.048    & 0.066    & 0.086    & 0.12     & 0.142    & 100 & complete \\
\textsc{meg\_x\_mid\_cl   } & \textsc{pd        }    & 0.001    & 0.002    & 0.006    & 0.013    & 0.03     & 0.392    & 100 & complete \\
\textsc{meg\_x\_mid\_mod  } & \textsc{pd        }    & 0.017    & 0.028    & 0.041    & 0.054    & 0.078    & 0.113    & 100 & complete \\
\textsc{meg\_x\_mid\_op   } & \textsc{pd        }    & 0.001    & 0.002    & 0.004    & 0.006    & 0.009    & 0.148    & 100 & complete \\
\textsc{meg\_x\_late\_cl  } & \textsc{pd        }    & 0.61     & 0.643    & 0.675    & 0.702    & 0.73     & 0.203    & 0   & complete \\
\textsc{meg\_x\_late\_mod } & \textsc{pd        }    & 0.126    & 0.171    & 0.203    & 0.252    & 0.315    & 0.065    & 0   & complete \\
\textsc{meg\_x\_late\_op  } & \textsc{pd        }    & 0.029    & 0.05     & 0.069    & 0.099    & 0.132    & 0.028    & 5   & complete \\
\textsc{meg\_x\_early\_all} & \textsc{pland     }    & 0.06     & 0.121    & 0.176    & 0.232    & 0.341    & 0.393    & 99  & complete \\
\textsc{meg\_x\_mid\_cl   } & \textsc{pland     }    & 0        & 0.003    & 0.015    & 0.037    & 0.093    & 1.829    & 100 & complete \\
\textsc{meg\_x\_mid\_mod  } & \textsc{pland     }    & 0.038    & 0.076    & 0.118    & 0.176    & 0.267    & 0.352    & 100 & complete \\
\textsc{meg\_x\_mid\_op   } & \textsc{pland     }    & 0        & 0.001    & 0.002    & 0.004    & 0.007    & 0.482    & 100 & complete \\
\textsc{meg\_x\_late\_cl  } & \textsc{pland     }    & 2.74     & 2.935    & 3.106    & 3.209    & 3.321    & 0.478    & 0   & complete \\
\textsc{meg\_x\_late\_mod } & \textsc{pland     }    & 0.123    & 0.194    & 0.245    & 0.334    & 0.453    & 0.144    & 11  & moderate \\
\textsc{meg\_x\_late\_op  } & \textsc{pland     }    & 0.016    & 0.027    & 0.043    & 0.068    & 0.106    & 0.05     & 59  & none     \\
\textsc{meg\_x\_early\_all} & \textsc{shape\_am }    & 1.544    & 1.698    & 1.851    & 2.016    & 2.261    & 1.805    & 43  & none     \\
\textsc{meg\_x\_mid\_cl   } & \textsc{shape\_am }    & 1.129    & 1.385    & 1.624    & 1.92     & 2.363    & 2.089    & 86  & moderate \\
\textsc{meg\_x\_mid\_mod  } & \textsc{shape\_am }    & 1.493    & 1.702    & 1.864    & 2.04     & 2.396    & 1.845    & 47  & none     \\
\textsc{meg\_x\_mid\_op   } & \textsc{shape\_am }    & 1        & 1.171    & 1.268    & 1.365    & 1.547    & 1.744    & 99  & complete \\
\textsc{meg\_x\_late\_cl  } & \textsc{shape\_am }    & 2.133    & 2.186    & 2.219    & 2.252    & 2.298    & 1.975    & 0   & complete \\
\textsc{meg\_x\_late\_mod } & \textsc{shape\_am }    & 1.411    & 1.461    & 1.508    & 1.587    & 1.764    & 1.919    & 99  & complete \\
\textsc{meg\_x\_late\_op  } & \textsc{shape\_am }    & 1.222    & 1.282    & 1.326    & 1.397    & 1.578    & 1.611    & 96  & complete \\
\textsc{meg\_x\_early\_all} & \textsc{shape\_mn }    & 1.241    & 1.28     & 1.302    & 1.33     & 1.366    & 1.401    & 100 & complete \\
\textsc{meg\_x\_mid\_cl   } & \textsc{shape\_mn }    & 1.075    & 1.213    & 1.3      & 1.374    & 1.645    & 1.421    & 85  & moderate \\
\textsc{meg\_x\_mid\_mod  } & \textsc{shape\_mn }    & 1.222    & 1.277    & 1.313    & 1.345    & 1.391    & 1.397    & 96  & complete \\
\textsc{meg\_x\_mid\_op   } & \textsc{shape\_mn }    & 1        & 1.086    & 1.148    & 1.216    & 1.376    & 1.419    & 97  & complete \\
\textsc{meg\_x\_late\_cl  } & \textsc{shape\_mn }    & 1.38     & 1.387    & 1.394    & 1.401    & 1.41     & 1.317    & 0   & complete \\
\textsc{meg\_x\_late\_mod } & \textsc{shape\_mn }    & 1.177    & 1.208    & 1.233    & 1.254    & 1.276    & 1.367    & 100 & complete \\
\textsc{meg\_x\_late\_op  } & \textsc{shape\_mn }    & 1.077    & 1.103    & 1.125    & 1.143    & 1.174    & 1.356    & 100 & complete \\
\textsc{meg\_x\_early\_all} & \textsc{simi\_mn  }    & 1953.96  & 2768.605 & 3251.862 & 3817.56  & 4595.013 & 2375.836 & 13  & moderate \\
\textsc{meg\_x\_mid\_cl   } & \textsc{simi\_mn  }    & 332.364  & 1047.116 & 1697.704 & 2398.075 & 4434.031 & 2414.102 & 76  & moderate \\
\textsc{meg\_x\_mid\_mod  } & \textsc{simi\_mn  }    & 1479.729 & 1883.877 & 2230.049 & 2597.385 & 3246.618 & 2552.584 & 72  & none     \\
\textsc{meg\_x\_mid\_op   } & \textsc{simi\_mn  }    & 303.12   & 669.511  & 1290.841 & 3016.288 & 5850.705 & 4552.039 & 90  & moderate \\
\textsc{meg\_x\_late\_cl  } & \textsc{simi\_mn  }    & 1402.492 & 1476.639 & 1562.496 & 1657.648 & 1759.804 & 1000.967 & 0   & complete \\
\textsc{meg\_x\_late\_mod } & \textsc{simi\_mn  }    & 1250.676 & 1389.937 & 1535.017 & 1675.532 & 1877.219 & 1138.751 & 1   & complete \\
\textsc{meg\_x\_late\_op  } & \textsc{simi\_mn  }    & 1099.256 & 1415.769 & 1635.739 & 1855.49  & 2318.216 & 2990.76  & 100 & complete \\
\textsc{meg\_x\_early\_all} & \textsc{te  		}    & 38220    & 66840    & 90420    & 118620   & 162480   & 226410   & 100 & complete \\
\textsc{meg\_x\_mid\_cl   } & \textsc{te  		}    & 600      & 2550     & 7920     & 19680    & 43572    & 817590   & 100 & complete \\
\textsc{meg\_x\_mid\_mod  } & \textsc{te 		}    & 22260    & 41220    & 59880    & 82080    & 115950   & 193020   & 100 & complete \\
\textsc{meg\_x\_mid\_op   } & \textsc{te 		}    & 180      & 1200     & 2280     & 3720     & 6282     & 265800   & 100 & complete \\
\textsc{meg\_x\_late\_cl  } & \textsc{te 		}    & 1226610  & 1287840  & 1326780  & 1364700  & 1403850  & 270960   & 0   & complete \\
\textsc{meg\_x\_late\_mod } & \textsc{te 		}    & 102600   & 153060   & 193110   & 252930   & 331410   & 91020    & 2   & complete \\
\textsc{meg\_x\_late\_op  } & \textsc{te 		}    & 17040    & 27300    & 41880    & 63090    & 94020    & 36180    & 40  & none    


\end{longtable}
\end{footnotesize}
\end{center}
\end{landscape}

\pagestyle{headings}

%%%%%%%%%%%%%%%%%%%%%%%%%%%%%%%%%%%%%%%%%%%%%%%%%%%%%%%%%%%%%%%%%%%%%%%%%%%%%%%%%%%%%%%%%%%%%%%%
%%%%%%%%%%%%%%%%%%%%%%%%%%%%%%%%%%%%%%%%%%%%%%%%%%%%%%%%%%%%%%%%%%%%%%%%%%%%%%%%%%%%%%%%%%%%%%%%

\pagestyle{empty}
\begin{landscape}
\footnotesize
\begin{center}
\begin{footnotesize}
\begin{longtable}{llrrrrr|rrr}
\caption{Unabridged results for class-level metrics for Oak-Conifer Forest and Woodland (\textsc{ocfw}) calculated with \textsc{Fragstats}. This table shows the range of variability in landscape structure. Included are the $5^{\text{th}}$ percentile, $25^{\text{th}}$ percentile, $50^{\text{th}}$ percentile, $75^{\text{th}}$ percentile, and $95^{\text{th}}$ percentiles of the distribution, as well as the current class value, the current percentile range of variability (\%RV) for that proportion, and the departure classification. For seral stage abbreviations, see Table~\ref{condtable}.} \\
\label{tab:fragclass_ocfw} \\

\hline 
\textbf{\begin{tabular}[c]{@{}l@{}}Cover-Seral Stage Type\end{tabular}}  &   
\textbf{\begin{tabular}[c]{@{}l@{}}Landscape\\ Metric\end{tabular}}  &   
\textbf{$5^{\text{th}}$ } &   
\textbf{$25^{\text{th}}$ } &   
\textbf{$50^{\text{th}}$ } &   
\textbf{$75^{\text{th}}$ } &   
\textbf{$95^{\text{th}}$ }  &  
\textbf{\begin{tabular}[c]{@{}l@{}}Current\\ Value\end{tabular}} &   
\textbf{\begin{tabular}[c]{@{}l@{}}Current\\ \%RV\end{tabular}} &   
\textbf{\begin{tabular}[c]{@{}l@{}}Departure\end{tabular}} \\  \\ \hline 
\endfirsthead

\multicolumn{10}{c}{{\bfseries \tablename\ \thetable{} -- continued from previous page}} \\
\hline 
\textbf{\begin{tabular}[c]{@{}l@{}}Cover-Seral Stage Type\end{tabular}}  &   
\textbf{\begin{tabular}[c]{@{}l@{}}Landscape\\ Metric\end{tabular}}  &   
\textbf{$5^{\text{th}}$ } &   
\textbf{$25^{\text{th}}$ } &   
\textbf{$50^{\text{th}}$ } &   
\textbf{$75^{\text{th}}$ } &   
\textbf{$95^{\text{th}}$ }  &  
\textbf{\begin{tabular}[c]{@{}l@{}}Current\\ Value\end{tabular}} &   
\textbf{\begin{tabular}[c]{@{}l@{}}Current\\ \%RV\end{tabular}} &   
\textbf{\begin{tabular}[c]{@{}l@{}}Departure\end{tabular}} \\  \\ \hline \endhead

\hline \multicolumn{10}{|l|}{{Continued on next page}} \\ \hline
\endfoot

\hline \hline
\endlastfoot

\textsc{ocfw\_early\_all} & \textsc{ai        }    & 83.743   & 84.743   & 85.476   & 86.358   & 87.207   & 84.16    & 11  & moderate \\
\textsc{ocfw\_mid\_cl   } & \textsc{ai        }    & 79.887   & 82.463   & 83.893   & 85.034   & 86.359   & 86.993   & 99  & complete \\
\textsc{ocfw\_mid\_mod  } & \textsc{ai        }    & 74.165   & 78.772   & 80.963   & 82.683   & 84.3     & 83.443   & 85  & moderate \\
\textsc{ocfw\_mid\_op   } & \textsc{ai        }    & 83.266   & 84.146   & 84.916   & 85.582   & 86.373   & 83.666   & 14  & moderate \\
\textsc{ocfw\_late\_cl  } & \textsc{ai        }    & 81.445   & 83.041   & 84.198   & 85.402   & 86.862   & 84.087   & 48  & none     \\
\textsc{ocfw\_late\_mod } & \textsc{ai        }    & 79.066   & 80.556   & 81.428   & 82.391   & 83.499   & 85.214   & 100 & complete \\
\textsc{ocfw\_late\_op  } & \textsc{ai        }    & 73.163   & 76.746   & 79.059   & 81.039   & 83.298   & 85.611   & 100 & complete \\
\textsc{ocfw\_early\_all} & \textsc{area\_am  }    & 36.191   & 53.924   & 84.126   & 116.693  & 221.992  & 45.113   & 13  & moderate \\
\textsc{ocfw\_mid\_cl   } & \textsc{area\_am  }    & 16.781   & 40.243   & 68.741   & 108.702  & 197.579  & 117.148  & 80  & moderate \\
\textsc{ocfw\_mid\_mod  } & \textsc{area\_am  }    & 8.168    & 23.826   & 42.571   & 71.524   & 147.182  & 21.968   & 24  & moderate \\
\textsc{ocfw\_mid\_op   } & \textsc{area\_am  }    & 35.861   & 58.376   & 78.148   & 111.791  & 180.013  & 38.05    & 8   & moderate \\
\textsc{ocfw\_late\_cl  } & \textsc{area\_am  }    & 23.744   & 46.719   & 74.542   & 106.801  & 203.062  & 12.611   & 1   & complete \\
\textsc{ocfw\_late\_mod } & \textsc{area\_am  }    & 21.236   & 33.094   & 53.019   & 79.653   & 142.244  & 12.107   & 0   & complete \\
\textsc{ocfw\_late\_op  } & \textsc{area\_am  }    & 7.743    & 15.564   & 29.464   & 57.291   & 157.312  & 11.03    & 12  & moderate \\
\textsc{ocfw\_early\_all} & \textsc{area\_mn  }    & 5.756    & 6.594    & 7.322    & 7.986    & 9.159    & 7.353    & 51  & none     \\
\textsc{ocfw\_mid\_cl   } & \textsc{area\_mn  }    & 3.969    & 5.245    & 6.358    & 7.288    & 8.79     & 13.989   & 100 & complete \\
\textsc{ocfw\_mid\_mod  } & \textsc{area\_mn  }    & 2.467    & 3.53     & 4.323    & 5.363    & 6.764    & 7.231    & 98  & complete \\
\textsc{ocfw\_mid\_op   } & \textsc{area\_mn  }    & 6.172    & 7.038    & 7.853    & 8.677    & 9.825    & 7.416    & 36  & none     \\
\textsc{ocfw\_late\_cl  } & \textsc{area\_mn  }    & 4.868    & 6.147    & 7.307    & 8.579    & 10.408   & 7.36     & 53  & none     \\
\textsc{ocfw\_late\_mod } & \textsc{area\_mn  }    & 4.071    & 4.779    & 5.415    & 6.153    & 7.252    & 6.776    & 89  & moderate \\
\textsc{ocfw\_late\_op  } & \textsc{area\_mn  }    & 2.45     & 3.132    & 3.817    & 4.836    & 6.446    & 6.525    & 96  & complete \\
\textsc{ocfw\_early\_all} & \textsc{cai\_am   }    & 94.433   & 95.396   & 95.977   & 96.54    & 97.225   & 97.426   & 98  & complete \\
\textsc{ocfw\_mid\_cl   } & \textsc{cai\_am   }    & 30.556   & 38.715   & 45.618   & 50.448   & 56.599   & 42.932   & 39  & none     \\
\textsc{ocfw\_mid\_mod  } & \textsc{cai\_am   }    & 43.319   & 56.561   & 62.02    & 66.406   & 71.413   & 69.712   & 91  & moderate \\
\textsc{ocfw\_mid\_op   } & \textsc{cai\_am   }    & 76.97    & 79.2     & 80.872   & 82.419   & 84.453   & 85.213   & 98  & complete \\
\textsc{ocfw\_late\_cl  } & \textsc{cai\_am   }    & 27.036   & 33.38    & 38.563   & 44.193   & 49.098   & 35.559   & 35  & none     \\
\textsc{ocfw\_late\_mod } & \textsc{cai\_am   }    & 42.36    & 46.771   & 48.948   & 51.82    & 54.616   & 63.036   & 100 & complete \\
\textsc{ocfw\_late\_op  } & \textsc{cai\_am   }    & 48.803   & 53.037   & 55.534   & 58.682   & 64.415   & 62       & 91  & moderate \\
\textsc{ocfw\_early\_all} & \textsc{clumpy    }    & 0.835    & 0.844    & 0.851    & 0.86     & 0.869    & 0.837    & 9   & moderate \\
\textsc{ocfw\_mid\_cl   } & \textsc{clumpy    }    & 0.797    & 0.822    & 0.836    & 0.847    & 0.861    & 0.863    & 97  & complete \\
\textsc{ocfw\_mid\_mod  } & \textsc{clumpy    }    & 0.74     & 0.786    & 0.807    & 0.824    & 0.84     & 0.831    & 85  & moderate \\
\textsc{ocfw\_mid\_op   } & \textsc{clumpy    }    & 0.829    & 0.838    & 0.845    & 0.852    & 0.86     & 0.831    & 9   & moderate \\
\textsc{ocfw\_late\_cl  } & \textsc{clumpy    }    & 0.812    & 0.827    & 0.838    & 0.85     & 0.864    & 0.84     & 53  & none     \\
\textsc{ocfw\_late\_mod } & \textsc{clumpy    }    & 0.787    & 0.802    & 0.81     & 0.82     & 0.831    & 0.852    & 100 & complete \\
\textsc{ocfw\_late\_op  } & \textsc{clumpy    }    & 0.73     & 0.766    & 0.788    & 0.808    & 0.83     & 0.856    & 100 & complete \\
\textsc{ocfw\_early\_all} & \textsc{core\_am  }    & 35.269   & 52.398   & 80.382   & 110.851  & 214.599  & 43.871   & 14  & moderate \\
\textsc{ocfw\_mid\_cl   } & \textsc{core\_am  }    & 8.087    & 23.09    & 42.86    & 69.866   & 129.331  & 57.319   & 66  & none     \\
\textsc{ocfw\_mid\_mod  } & \textsc{core\_am  }    & 5.324    & 17.707   & 33.578   & 57.587   & 122.488  & 16.386   & 23  & moderate \\
\textsc{ocfw\_mid\_op   } & \textsc{core\_am  }    & 30.492   & 49.472   & 67.283   & 97.388   & 156.204  & 34.474   & 10  & moderate \\
\textsc{ocfw\_late\_cl  } & \textsc{core\_am  }    & 10.128   & 22.629   & 38.776   & 63.481   & 128.421  & 4.78     & 1   & complete \\
\textsc{ocfw\_late\_mod } & \textsc{core\_am  }    & 12.506   & 19.867   & 32.648   & 49.024   & 88.879   & 8.055    & 1   & complete \\
\textsc{ocfw\_late\_op  } & \textsc{core\_am  }    & 4.615    & 9.892    & 19.824   & 40.218   & 116.712  & 7.746    & 17  & moderate \\
\textsc{ocfw\_early\_all} & \textsc{core\_mn  }    & 5.555    & 6.334    & 7.031    & 7.679    & 8.736    & 7.164    & 57  & none     \\
\textsc{ocfw\_mid\_cl   } & \textsc{core\_mn  }    & 1.307    & 2.086    & 2.902    & 3.588    & 4.706    & 6.006    & 100 & complete \\
\textsc{ocfw\_mid\_mod  } & \textsc{core\_mn  }    & 1.031    & 2        & 2.721    & 3.5      & 4.659    & 5.041    & 98  & complete \\
\textsc{ocfw\_mid\_op   } & \textsc{core\_mn  }    & 4.949    & 5.634    & 6.312    & 7.109    & 8.057    & 6.32     & 51  & none     \\
\textsc{ocfw\_late\_cl  } & \textsc{core\_mn  }    & 1.38     & 2.022    & 2.736    & 3.716    & 5.148    & 2.617    & 45  & none     \\
\textsc{ocfw\_late\_mod } & \textsc{core\_mn  }    & 1.801    & 2.267    & 2.681    & 3.067    & 3.818    & 4.271    & 100 & complete \\
\textsc{ocfw\_late\_op  } & \textsc{core\_mn  }    & 1.34     & 1.707    & 2.156    & 2.743    & 3.742    & 4.045    & 97  & complete \\
\textsc{ocfw\_early\_all} & \textsc{cpland    }    & 1.131    & 1.669    & 2.123    & 2.65     & 3.503    & 2.49     & 69  & none     \\
\textsc{ocfw\_mid\_cl   } & \textsc{cpland    }    & 0.14     & 0.358    & 0.652    & 0.897    & 1.282    & 2.061    & 100 & complete \\
\textsc{ocfw\_mid\_mod  } & \textsc{cpland    }    & 0.241    & 0.521    & 0.76     & 1.03     & 1.585    & 1.302    & 89  & moderate \\
\textsc{ocfw\_mid\_op   } & \textsc{cpland    }    & 1.126    & 1.533    & 1.937    & 2.436    & 3.153    & 2.659    & 86  & moderate \\
\textsc{ocfw\_late\_cl  } & \textsc{cpland    }    & 0.212    & 0.503    & 0.83     & 1.233    & 1.908    & 0.072    & 1   & complete \\
\textsc{ocfw\_late\_mod } & \textsc{cpland    }    & 0.634    & 0.851    & 1.028    & 1.211    & 1.486    & 0.082    & 0   & complete \\
\textsc{ocfw\_late\_op  } & \textsc{cpland    }    & 0.177    & 0.303    & 0.454    & 0.677    & 1.133    & 0.089    & 1   & complete \\
\textsc{ocfw\_early\_all} & \textsc{cwed      }    & 0.958    & 1.319    & 1.636    & 1.95     & 2.482    & 1.905    & 74  & none     \\
\textsc{ocfw\_mid\_cl   } & \textsc{cwed      }    & 0.306    & 0.623    & 0.908    & 1.185    & 1.494    & 2.684    & 100 & complete \\
\textsc{ocfw\_mid\_mod  } & \textsc{cwed      }    & 0.457    & 0.683    & 0.859    & 1.075    & 1.527    & 1.112    & 79  & moderate \\
\textsc{ocfw\_mid\_op   } & \textsc{cwed      }    & 0.859    & 1.098    & 1.387    & 1.719    & 2.218    & 2.101    & 93  & moderate \\
\textsc{ocfw\_late\_cl  } & \textsc{cwed      }    & 0.542    & 1.033    & 1.576    & 1.978    & 2.563    & 0.153    & 0   & complete \\
\textsc{ocfw\_late\_mod } & \textsc{cwed      }    & 1.116    & 1.395    & 1.599    & 1.818    & 2.167    & 0.083    & 0   & complete \\
\textsc{ocfw\_late\_op  } & \textsc{cwed      }    & 0.263    & 0.412    & 0.587    & 0.821    & 1.191    & 0.103    & 1   & complete \\
\textsc{ocfw\_early\_all} & \textsc{econ\_am  }    & 34.982   & 36.305   & 37.315   & 38.511   & 39.985   & 34.313   & 2   & complete \\
\textsc{ocfw\_mid\_cl   } & \textsc{econ\_am  }    & 26.869   & 28.567   & 29.897   & 31.475   & 34.497   & 31.44    & 75  & moderate \\
\textsc{ocfw\_mid\_mod  } & \textsc{econ\_am  }    & 24.799   & 26.433   & 27.595   & 28.876   & 30.578   & 26.44    & 26  & none     \\
\textsc{ocfw\_mid\_op   } & \textsc{econ\_am  }    & 26.737   & 27.697   & 28.416   & 28.966   & 29.813   & 29.909   & 96  & complete \\
\textsc{ocfw\_late\_cl  } & \textsc{econ\_am  }    & 29.557   & 31.062   & 32.11    & 33.302   & 35.519   & 31.927   & 47  & none     \\
\textsc{ocfw\_late\_mod } & \textsc{econ\_am  }    & 28.495   & 29.824   & 30.784   & 31.577   & 32.976   & 28.405   & 4   & complete \\
\textsc{ocfw\_late\_op  } & \textsc{econ\_am  }    & 22.549   & 24.249   & 25.769   & 27.177   & 29.067   & 31.546   & 100 & complete \\
\textsc{ocfw\_early\_all} & \textsc{econ\_mn  }    & 34.114   & 35.293   & 36.156   & 37.12    & 38.486   & 34.587   & 11  & moderate \\
\textsc{ocfw\_mid\_cl   } & \textsc{econ\_mn  }    & 26.372   & 27.537   & 28.586   & 30.142   & 32.828   & 32.486   & 95  & complete \\
\textsc{ocfw\_mid\_mod  } & \textsc{econ\_mn  }    & 23.649   & 25.247   & 26.335   & 27.552   & 29.637   & 26.828   & 61  & none     \\
\textsc{ocfw\_mid\_op   } & \textsc{econ\_mn  }    & 27.524   & 28.2     & 28.654   & 29.039   & 29.663   & 31.56    & 100 & complete \\
\textsc{ocfw\_late\_cl  } & \textsc{econ\_mn  }    & 31.631   & 32.996   & 34.195   & 35.629   & 39.405   & 33.507   & 37  & none     \\
\textsc{ocfw\_late\_mod } & \textsc{econ\_mn  }    & 28.211   & 29.917   & 31.189   & 32.405   & 34.029   & 31.224   & 51  & none     \\
\textsc{ocfw\_late\_op  } & \textsc{econ\_mn  }    & 21.719   & 23.466   & 24.753   & 26.117   & 27.723   & 35.579   & 100 & complete \\
\textsc{ocfw\_early\_all} & \textsc{ed        }    & 2.575    & 3.532    & 4.388    & 5.29     & 6.83     & 5.472    & 79  & moderate \\
\textsc{ocfw\_mid\_cl   } & \textsc{ed        }    & 1.001    & 2.043    & 3.175    & 4.084    & 5.291    & 8.444    & 100 & complete \\
\textsc{ocfw\_mid\_mod  } & \textsc{ed        }    & 1.868    & 2.625    & 3.163    & 3.847    & 5.184    & 4.184    & 84  & moderate \\
\textsc{ocfw\_mid\_op   } & \textsc{ed        }    & 2.993    & 3.898    & 4.852    & 6.078    & 7.733    & 6.876    & 87  & moderate \\
\textsc{ocfw\_late\_cl  } & \textsc{ed        }    & 1.563    & 3.154    & 4.896    & 6.223    & 7.968    & 0.466    & 0   & complete \\
\textsc{ocfw\_late\_mod } & \textsc{ed        }    & 3.762    & 4.57     & 5.266    & 5.891    & 6.872    & 0.286    & 0   & complete \\
\textsc{ocfw\_late\_op  } & \textsc{ed        }    & 1.09     & 1.702    & 2.357    & 3.175    & 4.65     & 0.306    & 0   & complete \\
\textsc{ocfw\_early\_all} & \textsc{gyrate\_am}    & 241.305  & 287.935  & 345.873  & 413.84   & 558.696  & 268.668  & 16  & moderate \\
\textsc{ocfw\_mid\_cl   } & \textsc{gyrate\_am}    & 166.648  & 259.106  & 330.025  & 406.981  & 566.236  & 474.172  & 87  & moderate \\
\textsc{ocfw\_mid\_mod  } & \textsc{gyrate\_am}    & 112.87   & 186.367  & 249.66   & 320.729  & 464.922  & 207.612  & 33  & none     \\
\textsc{ocfw\_mid\_op   } & \textsc{gyrate\_am}    & 242.133  & 302.666  & 347.503  & 413.445  & 527.323  & 259.389  & 10  & moderate \\
\textsc{ocfw\_late\_cl  } & \textsc{gyrate\_am}    & 200.668  & 284.284  & 354.679  & 426.939  & 583.711  & 165.148  & 1   & complete \\
\textsc{ocfw\_late\_mod } & \textsc{gyrate\_am}    & 186.516  & 231.792  & 286.419  & 352.176  & 456.958  & 149.439  & 1   & complete \\
\textsc{ocfw\_late\_op  } & \textsc{gyrate\_am}    & 113.293  & 155.821  & 211.351  & 291.278  & 520.517  & 141.368  & 18  & moderate \\
\textsc{ocfw\_early\_all} & \textsc{iji       }    & 58.774   & 60.244   & 61.407   & 62.706   & 64.294   & 60.61    & 31  & none     \\
\textsc{ocfw\_mid\_cl   } & \textsc{iji       }    & 54.078   & 56.247   & 57.924   & 59.384   & 62.131   & 66.454   & 100 & complete \\
\textsc{ocfw\_mid\_mod  } & \textsc{iji       }    & 53.049   & 57.586   & 59.296   & 60.493   & 61.909   & 62.976   & 100 & complete \\
\textsc{ocfw\_mid\_op   } & \textsc{iji       }    & 58.82    & 60.135   & 60.996   & 61.941   & 63.421   & 67.401   & 100 & complete \\
\textsc{ocfw\_late\_cl  } & \textsc{iji       }    & 57.388   & 59.766   & 61.044   & 62.286   & 64.312   & 59.475   & 20  & moderate \\
\textsc{ocfw\_late\_mod } & \textsc{iji       }    & 57.879   & 59.896   & 60.868   & 62.005   & 63.366   & 61.683   & 70  & none     \\
\textsc{ocfw\_late\_op  } & \textsc{iji       }    & 43.858   & 47.87    & 50.042   & 52.733   & 55.553   & 60.477   & 100 & complete \\
\textsc{ocfw\_early\_all} & \textsc{pd        }    & 0.19     & 0.25     & 0.303    & 0.362    & 0.452    & 0.348    & 71  & none     \\
\textsc{ocfw\_mid\_cl   } & \textsc{pd        }    & 0.078    & 0.158    & 0.228    & 0.285    & 0.36     & 0.343    & 91  & moderate \\
\textsc{ocfw\_mid\_mod  } & \textsc{pd        }    & 0.206    & 0.246    & 0.28     & 0.32     & 0.388    & 0.258    & 35  & none     \\
\textsc{ocfw\_mid\_op   } & \textsc{pd        }    & 0.194    & 0.248    & 0.304    & 0.372    & 0.453    & 0.421    & 91  & moderate \\
\textsc{ocfw\_late\_cl  } & \textsc{pd        }    & 0.118    & 0.218    & 0.315    & 0.374    & 0.468    & 0.028    & 0   & complete \\
\textsc{ocfw\_late\_mod } & \textsc{pd        }    & 0.298    & 0.347    & 0.383    & 0.418    & 0.477    & 0.019    & 0   & complete \\
\textsc{ocfw\_late\_op  } & \textsc{pd        }    & 0.111    & 0.161    & 0.211    & 0.275    & 0.367    & 0.022    & 0   & complete \\
\textsc{ocfw\_early\_all} & \textsc{pland     }    & 1.187    & 1.747    & 2.212    & 2.762    & 3.639    & 2.556    & 67  & none     \\
\textsc{ocfw\_mid\_cl   } & \textsc{pland     }    & 0.382    & 0.879    & 1.429    & 1.922    & 2.533    & 4.8      & 100 & complete \\
\textsc{ocfw\_mid\_mod  } & \textsc{pland     }    & 0.554    & 0.916    & 1.225    & 1.569    & 2.362    & 1.868    & 86  & moderate \\
\textsc{ocfw\_mid\_op   } & \textsc{pland     }    & 1.379    & 1.885    & 2.403    & 3.032    & 3.995    & 3.121    & 80  & moderate \\
\textsc{ocfw\_late\_cl  } & \textsc{pland     }    & 0.691    & 1.442    & 2.246    & 2.986    & 4.185    & 0.203    & 0   & complete \\
\textsc{ocfw\_late\_mod } & \textsc{pland     }    & 1.394    & 1.766    & 2.087    & 2.437    & 2.968    & 0.131    & 0   & complete \\
\textsc{ocfw\_late\_op  } & \textsc{pland     }    & 0.348    & 0.534    & 0.817    & 1.203    & 1.906    & 0.144    & 1   & complete \\
\textsc{ocfw\_early\_all} & \textsc{shape\_am }    & 1.957    & 2.092    & 2.273    & 2.478    & 3.03     & 2.069    & 21  & moderate \\
\textsc{ocfw\_mid\_cl   } & \textsc{shape\_am }    & 1.849    & 2.144    & 2.429    & 2.716    & 3.432    & 3.006    & 87  & moderate \\
\textsc{ocfw\_mid\_mod  } & \textsc{shape\_am }    & 1.625    & 1.937    & 2.251    & 2.523    & 3.137    & 1.959    & 28  & none     \\
\textsc{ocfw\_mid\_op   } & \textsc{shape\_am }    & 2.121    & 2.369    & 2.576    & 2.85     & 3.311    & 2.101    & 4   & complete \\
\textsc{ocfw\_late\_cl  } & \textsc{shape\_am }    & 2.027    & 2.354    & 2.658    & 2.902    & 3.599    & 1.738    & 1   & complete \\
\textsc{ocfw\_late\_mod } & \textsc{shape\_am }    & 2.047    & 2.298    & 2.615    & 2.969    & 3.592    & 1.558    & 0   & complete \\
\textsc{ocfw\_late\_op  } & \textsc{shape\_am }    & 1.704    & 1.889    & 2.193    & 2.654    & 3.825    & 1.456    & 0   & complete \\
\textsc{ocfw\_early\_all} & \textsc{shape\_mn }    & 1.422    & 1.444    & 1.459    & 1.473    & 1.492    & 1.579    & 100 & complete \\
\textsc{ocfw\_mid\_cl   } & \textsc{shape\_mn }    & 1.416    & 1.46     & 1.488    & 1.514    & 1.552    & 1.676    & 100 & complete \\
\textsc{ocfw\_mid\_mod  } & \textsc{shape\_mn }    & 1.375    & 1.404    & 1.427    & 1.45     & 1.484    & 1.557    & 100 & complete \\
\textsc{ocfw\_mid\_op   } & \textsc{shape\_mn }    & 1.458    & 1.482    & 1.499    & 1.512    & 1.535    & 1.582    & 100 & complete \\
\textsc{ocfw\_late\_cl  } & \textsc{shape\_mn }    & 1.448    & 1.492    & 1.521    & 1.544    & 1.573    & 1.552    & 83  & moderate \\
\textsc{ocfw\_late\_mod } & \textsc{shape\_mn }    & 1.448    & 1.471    & 1.489    & 1.51     & 1.537    & 1.481    & 36  & none     \\
\textsc{ocfw\_late\_op  } & \textsc{shape\_mn }    & 1.404    & 1.434    & 1.457    & 1.483    & 1.525    & 1.448    & 39  & none     \\
\textsc{ocfw\_early\_all} & \textsc{simi\_mn  }    & 2891.659 & 3356.395 & 3625.608 & 3944.445 & 4382.326 & 2409.316 & 1   & complete \\
\textsc{ocfw\_mid\_cl   } & \textsc{simi\_mn  }    & 2047.961 & 2288.238 & 2474.895 & 2699.749 & 3158.397 & 2713.299 & 77  & moderate \\
\textsc{ocfw\_mid\_mod  } & \textsc{simi\_mn  }    & 2024.62  & 2239.683 & 2437.881 & 2602.99  & 2927.032 & 3037.674 & 97  & complete \\
\textsc{ocfw\_mid\_op   } & \textsc{simi\_mn  }    & 2507.647 & 2806.578 & 3008.049 & 3276.525 & 3651.946 & 2893.278 & 37  & none     \\
\textsc{ocfw\_late\_cl  } & \textsc{simi\_mn  }    & 1782.809 & 1928.235 & 2071.727 & 2227.438 & 2538.956 & 2342.631 & 88  & moderate \\
\textsc{ocfw\_late\_mod } & \textsc{simi\_mn  }    & 1882.5   & 2035.755 & 2170.165 & 2303.575 & 2546.419 & 2592.142 & 97  & complete \\
\textsc{ocfw\_late\_op  } & \textsc{simi\_mn  }    & 1859.94  & 2243.606 & 2474.975 & 2736.722 & 3174.316 & 4186.441 & 100 & complete \\
\textsc{ocfw\_early\_all} & \textsc{te  	  }    & 467430   & 641190   & 796710   & 960450   & 1240020  & 993510   & 79  & moderate \\
\textsc{ocfw\_mid\_cl   } & \textsc{te        }    & 181740   & 370860   & 576510   & 741450   & 960600   & 1533060  & 100 & complete \\
\textsc{ocfw\_mid\_mod  } & \textsc{te        }    & 339150   & 476550   & 574290   & 698490   & 941220   & 759570   & 84  & moderate \\
\textsc{ocfw\_mid\_op   } & \textsc{te        }    & 543330   & 707730   & 880920   & 1103430  & 1403940  & 1248390  & 87  & moderate \\
\textsc{ocfw\_late\_cl  } & \textsc{te        }    & 283860   & 572550   & 888930   & 1129770  & 1446630  & 84540    & 0   & complete \\
\textsc{ocfw\_late\_mod } & \textsc{te        }    & 683010   & 829770   & 956100   & 1069470  & 1247640  & 52020    & 0   & complete \\
\textsc{ocfw\_late\_op  } & \textsc{te        }    & 197970   & 309030   & 427830   & 576510   & 844290   & 55620    & 0   & complete

\end{longtable}
\end{footnotesize}
\end{center}
\end{landscape}

\pagestyle{headings}

%%%%%%%%%%%%%%%%%%%%%%%%%%%%%%%%%%%%%%%%%%%%%%%%%%%%%%%%%%%%%%%%%%%%%%%%%%%%%%%%%%%%%%%%%%%%%%%%
%%%%%%%%%%%%%%%%%%%%%%%%%%%%%%%%%%%%%%%%%%%%%%%%%%%%%%%%%%%%%%%%%%%%%%%%%%%%%%%%%%%%%%%%%%%%%%%%

\pagestyle{empty}
\begin{landscape}
\footnotesize
\begin{center}
\begin{footnotesize}
\begin{longtable}{llrrrrr|rrr}
\caption{Unabridged results for class-level metrics for Oak-Conifer Forest and Woodland - Ultramafic (\textsc{ocfw\_u}) calculated with \textsc{Fragstats}. This table shows the range of variability in landscape structure. Included are the $5^{\text{th}}$ percentile, $25^{\text{th}}$ percentile, $50^{\text{th}}$ percentile, $75^{\text{th}}$ percentile, and $95^{\text{th}}$ percentiles of the distribution, as well as the current class value, the current percentile range of variability (\%RV) for that proportion, and the departure classification. For seral stage abbreviations, see Table~\ref{condtable}.} \\
\label{tab:fragclass_ocfwu} \\

\hline 
\textbf{\begin{tabular}[c]{@{}l@{}}Cover-Seral Stage Type\end{tabular}}  &   
\textbf{\begin{tabular}[c]{@{}l@{}}Landscape\\ Metric\end{tabular}}  &   
\textbf{$5^{\text{th}}$ } &   
\textbf{$25^{\text{th}}$ } &   
\textbf{$50^{\text{th}}$ } &   
\textbf{$75^{\text{th}}$ } &   
\textbf{$95^{\text{th}}$ }  &  
\textbf{\begin{tabular}[c]{@{}l@{}}Current\\ Value\end{tabular}} &   
\textbf{\begin{tabular}[c]{@{}l@{}}Current\\ \%RV\end{tabular}} &   
\textbf{\begin{tabular}[c]{@{}l@{}}Departure\end{tabular}} \\  \\ \hline 
\endfirsthead

\multicolumn{10}{c}{{\bfseries \tablename\ \thetable{} -- continued from previous page}} \\
\hline 
\textbf{\begin{tabular}[c]{@{}l@{}}Cover-Seral Stage Type\end{tabular}}  &   
\textbf{\begin{tabular}[c]{@{}l@{}}Landscape\\ Metric\end{tabular}}  &   
\textbf{$5^{\text{th}}$ } &   
\textbf{$25^{\text{th}}$ } &   
\textbf{$50^{\text{th}}$ } &   
\textbf{$75^{\text{th}}$ } &   
\textbf{$95^{\text{th}}$ }  &  
\textbf{\begin{tabular}[c]{@{}l@{}}Current\\ Value\end{tabular}} &   
\textbf{\begin{tabular}[c]{@{}l@{}}Current\\ \%RV\end{tabular}} &   
\textbf{\begin{tabular}[c]{@{}l@{}}Departure\end{tabular}} \\  \\ \hline \endhead

\hline \multicolumn{10}{|l|}{{Continued on next page}} \\ \hline
\endfoot

\hline \hline
\endlastfoot


\textsc{ocfw\_u\_early\_all} & \textsc{ai        }    & 76.685   & 78.374   & 79.474   & 81.104   & 84.231   & 82.851   & 92  & moderate \\
\textsc{ocfw\_u\_mid\_cl   } & \textsc{ai        }    & 75       & 81.034   & 86.207   & 90.315   & 100      & 83.623   & 37  & none     \\
\textsc{ocfw\_u\_mid\_mod  } & \textsc{ai        }    & 71.756   & 76.812   & 79.921   & 82.186   & 86.69    & 80.972   & 64  & none     \\
\textsc{ocfw\_u\_mid\_op   } & \textsc{ai        }    & 79.088   & 80.657   & 81.75    & 83.348   & 85.137   & 81.935   & 55  & none     \\
\textsc{ocfw\_u\_late\_cl  } & \textsc{ai        }    & 76.004   & 79.171   & 80.923   & 82.637   & 84.761   & 90.835   & 100 & complete \\
\textsc{ocfw\_u\_late\_mod } & \textsc{ai        }    & 74.115   & 76.897   & 78.503   & 80.528   & 82.691   & 84.849   & 100 & complete \\
\textsc{ocfw\_u\_late\_op  } & \textsc{ai        }    & 69.881   & 72.454   & 73.973   & 75.946   & 80.657   & 79.105   & 92  & moderate \\
\textsc{ocfw\_u\_early\_all} & \textsc{area\_am  }    & 4.895    & 7.013    & 9.496    & 14.801   & 33.829   & 16.277   & 80  & moderate \\
\textsc{ocfw\_u\_mid\_cl   } & \textsc{area\_am  }    & 0.18     & 1.08     & 1.71     & 2.79     & 9.761    & 13.708   & 98  & complete \\
\textsc{ocfw\_u\_mid\_mod  } & \textsc{area\_am  }    & 1.795    & 3.445    & 5.679    & 10.488   & 26.554   & 6.819    & 59  & none     \\
\textsc{ocfw\_u\_mid\_op   } & \textsc{area\_am  }    & 6.904    & 10.092   & 13.403   & 18.892   & 37.553   & 14.595   & 58  & none     \\
\textsc{ocfw\_u\_late\_cl  } & \textsc{area\_am  }    & 4.288    & 8.132    & 12.014   & 19.35    & 38.005   & 18.655   & 73  & none     \\
\textsc{ocfw\_u\_late\_mod } & \textsc{area\_am  }    & 3.588    & 5.594    & 8.385    & 13.411   & 25.639   & 10.678   & 66  & none     \\
\textsc{ocfw\_u\_late\_op  } & \textsc{area\_am  }    & 1.786    & 2.274    & 2.934    & 4.082    & 10.735   & 1.714    & 4   & complete \\
\textsc{ocfw\_u\_early\_all} & \textsc{area\_mn  }    & 2.6      & 3.01     & 3.432    & 4.077    & 5.174    & 5.547    & 98  & complete \\
\textsc{ocfw\_u\_mid\_cl   } & \textsc{area\_mn  }    & 0.18     & 0.99     & 1.65     & 2.507    & 6.818    & 6.355    & 95  & complete \\
\textsc{ocfw\_u\_mid\_mod  } & \textsc{area\_mn  }    & 1.35     & 2.085    & 2.808    & 3.638    & 6.139    & 3.867    & 82  & moderate \\
\textsc{ocfw\_u\_mid\_op   } & \textsc{area\_mn  }    & 3.264    & 3.994    & 4.556    & 5.297    & 6.638    & 5.4      & 77  & moderate \\
\textsc{ocfw\_u\_late\_cl  } & \textsc{area\_mn  }    & 2.389    & 3.442    & 4.18     & 5.468    & 6.986    & 11.556   & 100 & complete \\
\textsc{ocfw\_u\_late\_mod } & \textsc{area\_mn  }    & 2.155    & 2.763    & 3.243    & 4.058    & 5.784    & 5.94     & 97  & complete \\
\textsc{ocfw\_u\_late\_op  } & \textsc{area\_mn  }    & 1.321    & 1.633    & 1.886    & 2.139    & 3.036    & 1.2      & 2   & complete \\
\textsc{ocfw\_u\_early\_all} & \textsc{cai\_am   }    & 89.601   & 91.71    & 93.043   & 94.439   & 96.076   & 89.233   & 5   & complete \\
\textsc{ocfw\_u\_mid\_cl   } & \textsc{cai\_am   }    & 0        & 0        & 0        & 16.912   & 67.059   & 25.289   & 80  & moderate \\
\textsc{ocfw\_u\_mid\_mod  } & \textsc{cai\_am   }    & 19.277   & 36.033   & 47.33    & 55.334   & 64.926   & 44.582   & 43  & none     \\
\textsc{ocfw\_u\_mid\_op   } & \textsc{cai\_am   }    & 62.926   & 65.635   & 67.765   & 70.982   & 74.725   & 66.894   & 42  & none     \\
\textsc{ocfw\_u\_late\_cl  } & \textsc{cai\_am   }    & 3.163    & 9.792    & 14.61    & 19.236   & 26.611   & 28.505   & 97  & complete \\
\textsc{ocfw\_u\_late\_mod } & \textsc{cai\_am   }    & 26.955   & 32.074   & 35.916   & 40.06    & 45.566   & 37.5     & 59  & none     \\
\textsc{ocfw\_u\_late\_op  } & \textsc{cai\_am   }    & 21.212   & 32.165   & 38.472   & 46.292   & 55.294   & 80       & 100 & complete \\
\textsc{ocfw\_u\_early\_all} & \textsc{clumpy    }    & 0.767    & 0.784    & 0.794    & 0.811    & 0.842    & 0.828    & 92  & moderate \\
\textsc{ocfw\_u\_mid\_cl   } & \textsc{clumpy    }    & 0.75     & 0.81     & 0.862    & 0.903    & 1        & 0.836    & 37  & none     \\
\textsc{ocfw\_u\_mid\_mod  } & \textsc{clumpy    }    & 0.718    & 0.768    & 0.799    & 0.822    & 0.867    & 0.81     & 65  & none     \\
\textsc{ocfw\_u\_mid\_op   } & \textsc{clumpy    }    & 0.791    & 0.806    & 0.817    & 0.833    & 0.851    & 0.819    & 55  & none     \\
\textsc{ocfw\_u\_late\_cl  } & \textsc{clumpy    }    & 0.76     & 0.791    & 0.809    & 0.826    & 0.847    & 0.908    & 100 & complete \\
\textsc{ocfw\_u\_late\_mod } & \textsc{clumpy    }    & 0.741    & 0.769    & 0.785    & 0.805    & 0.827    & 0.848    & 100 & complete \\
\textsc{ocfw\_u\_late\_op  } & \textsc{clumpy    }    & 0.699    & 0.724    & 0.74     & 0.759    & 0.806    & 0.791    & 92  & moderate \\
\textsc{ocfw\_u\_early\_all} & \textsc{core\_am  }    & 4.481    & 6.448    & 8.583    & 13.74    & 29.63    & 13.934   & 77  & moderate \\
\textsc{ocfw\_u\_mid\_cl   } & \textsc{core\_am  }    & 0        & 0        & 0        & 0.36     & 7.123    & 4.579    & 94  & moderate \\
\textsc{ocfw\_u\_mid\_mod  } & \textsc{core\_am  }    & 0.448    & 1.444    & 3.3      & 6.352    & 17.756   & 3.042    & 48  & none     \\
\textsc{ocfw\_u\_mid\_op   } & \textsc{core\_am  }    & 4.696    & 7.391    & 9.992    & 13.482   & 27.703   & 10.963   & 59  & none     \\
\textsc{ocfw\_u\_late\_cl  } & \textsc{core\_am  }    & 0.201    & 1.372    & 2.727    & 4.914    & 13.099   & 7.048    & 85  & moderate \\
\textsc{ocfw\_u\_late\_mod } & \textsc{core\_am  }    & 1.31     & 2.212    & 3.37     & 6.744    & 12.517   & 3.434    & 52  & none     \\
\textsc{ocfw\_u\_late\_op  } & \textsc{core\_am  }    & 0.486    & 0.843    & 1.21     & 1.85     & 5.146    & 1.397    & 60  & none     \\
\textsc{ocfw\_u\_early\_all} & \textsc{core\_mn  }    & 2.421    & 2.805    & 3.181    & 3.792    & 4.609    & 4.95     & 97  & complete \\
\textsc{ocfw\_u\_mid\_cl   } & \textsc{core\_mn  }    & 0        & 0        & 0        & 0.308    & 3.285    & 1.607    & 90  & moderate \\
\textsc{ocfw\_u\_mid\_mod  } & \textsc{core\_mn  }    & 0.294    & 0.773    & 1.361    & 2.046    & 3.388    & 1.724    & 65  & none     \\
\textsc{ocfw\_u\_mid\_op   } & \textsc{core\_mn  }    & 2.156    & 2.703    & 3.155    & 3.617    & 4.706    & 3.612    & 75  & moderate \\
\textsc{ocfw\_u\_late\_cl  } & \textsc{core\_mn  }    & 0.088    & 0.35     & 0.614    & 1.022    & 1.776    & 3.294    & 100 & complete \\
\textsc{ocfw\_u\_late\_mod } & \textsc{core\_mn  }    & 0.692    & 0.924    & 1.129    & 1.636    & 2.257    & 2.228    & 95  & complete \\
\textsc{ocfw\_u\_late\_op  } & \textsc{core\_mn  }    & 0.35     & 0.553    & 0.728    & 0.938    & 1.35     & 0.96     & 77  & moderate \\
\textsc{ocfw\_u\_early\_all} & \textsc{cpland    }    & 0.058    & 0.089    & 0.11     & 0.143    & 0.178    & 0.09     & 27  & none     \\
\textsc{ocfw\_u\_mid\_cl   } & \textsc{cpland    }    & 0        & 0        & 0        & 0        & 0.005    & 0.043    & 100 & complete \\
\textsc{ocfw\_u\_mid\_mod  } & \textsc{cpland    }    & 0.002    & 0.005    & 0.012    & 0.023    & 0.038    & 0.03     & 87  & moderate \\
\textsc{ocfw\_u\_mid\_op   } & \textsc{cpland    }    & 0.068    & 0.088    & 0.11     & 0.129    & 0.154    & 0.131    & 79  & moderate \\
\textsc{ocfw\_u\_late\_cl  } & \textsc{cpland    }    & 0.002    & 0.009    & 0.018    & 0.029    & 0.056    & 0.009    & 24  & moderate \\
\textsc{ocfw\_u\_late\_mod } & \textsc{cpland    }    & 0.017    & 0.027    & 0.035    & 0.049    & 0.071    & 0.005    & 0   & complete \\
\textsc{ocfw\_u\_late\_op  } & \textsc{cpland    }    & 0.002    & 0.006    & 0.01     & 0.015    & 0.027    & 0.002    & 5   & complete \\
\textsc{ocfw\_u\_early\_all} & \textsc{cwed      }    & 0.075    & 0.107    & 0.127    & 0.148    & 0.178    & 0.076    & 6   & moderate \\
\textsc{ocfw\_u\_mid\_cl   } & \textsc{cwed      }    & 0        & 0.001    & 0.002    & 0.003    & 0.008    & 0.146    & 100 & complete \\
\textsc{ocfw\_u\_mid\_mod  } & \textsc{cwed      }    & 0.008    & 0.016    & 0.027    & 0.038    & 0.056    & 0.058    & 97  & complete \\
\textsc{ocfw\_u\_mid\_op   } & \textsc{cwed      }    & 0.088    & 0.115    & 0.132    & 0.148    & 0.17     & 0.173    & 97  & complete \\
\textsc{ocfw\_u\_late\_cl  } & \textsc{cwed      }    & 0.054    & 0.108    & 0.144    & 0.187    & 0.23     & 0.025    & 1   & complete \\
\textsc{ocfw\_u\_late\_mod } & \textsc{cwed      }    & 0.055    & 0.081    & 0.106    & 0.128    & 0.175    & 0.014    & 0   & complete \\
\textsc{ocfw\_u\_late\_op  } & \textsc{cwed      }    & 0.009    & 0.018    & 0.025    & 0.036    & 0.053    & 0.003    & 1   & complete \\
\textsc{ocfw\_u\_early\_all} & \textsc{econ\_am  }    & 33.578   & 35.199   & 36.568   & 38.161   & 40.862   & 30.024   & 0   & complete \\
\textsc{ocfw\_u\_mid\_cl   } & \textsc{econ\_am  }    & 16.263   & 25.625   & 32.21    & 40.833   & 55.846   & 35.233   & 59  & none     \\
\textsc{ocfw\_u\_mid\_mod  } & \textsc{econ\_am  }    & 26.932   & 30.108   & 32.318   & 34.639   & 37.935   & 30.869   & 33  & none     \\
\textsc{ocfw\_u\_mid\_op   } & \textsc{econ\_am  }    & 30.411   & 31.664   & 32.498   & 33.456   & 34.992   & 35.106   & 96  & complete \\
\textsc{ocfw\_u\_late\_cl  } & \textsc{econ\_am  }    & 36.448   & 38.884   & 41.055   & 42.986   & 46.113   & 46.059   & 95  & complete \\
\textsc{ocfw\_u\_late\_mod } & \textsc{econ\_am  }    & 28.477   & 31.868   & 34.123   & 35.73    & 38.2     & 40.531   & 100 & complete \\
\textsc{ocfw\_u\_late\_op  } & \textsc{econ\_am  }    & 21.441   & 24.27    & 26.959   & 29.208   & 33.053   & 36.104   & 100 & complete \\
\textsc{ocfw\_u\_early\_all} & \textsc{econ\_mn  }    & 32.722   & 34.234   & 35.577   & 37.075   & 39.165   & 28.317   & 0   & complete \\
\textsc{ocfw\_u\_mid\_cl   } & \textsc{econ\_mn  }    & 16.429   & 26.331   & 33.125   & 40.769   & 55.038   & 37.187   & 62  & none     \\
\textsc{ocfw\_u\_mid\_mod  } & \textsc{econ\_mn  }    & 27.199   & 29.719   & 31.645   & 33.708   & 37.093   & 28.898   & 17  & moderate \\
\textsc{ocfw\_u\_mid\_op   } & \textsc{econ\_mn  }    & 29.921   & 30.932   & 31.693   & 32.553   & 33.677   & 33.652   & 95  & complete \\
\textsc{ocfw\_u\_late\_cl  } & \textsc{econ\_mn  }    & 39.702   & 41.864   & 43.459   & 45.279   & 48.116   & 44.162   & 61  & none     \\
\textsc{ocfw\_u\_late\_mod } & \textsc{econ\_mn  }    & 29.39    & 32.091   & 34.406   & 36.231   & 38.326   & 35.107   & 59  & none     \\
\textsc{ocfw\_u\_late\_op  } & \textsc{econ\_mn  }    & 22.718   & 25.283   & 27.5     & 29.631   & 33.263   & 38.194   & 100 & complete \\
\textsc{ocfw\_u\_early\_all} & \textsc{ed        }    & 0.198    & 0.286    & 0.352    & 0.412    & 0.501    & 0.254    & 15  & moderate \\
\textsc{ocfw\_u\_mid\_cl   } & \textsc{ed        }    & 0.001    & 0.003    & 0.006    & 0.01     & 0.025    & 0.407    & 100 & complete \\
\textsc{ocfw\_u\_mid\_mod  } & \textsc{ed        }    & 0.027    & 0.049    & 0.083    & 0.116    & 0.173    & 0.189    & 98  & complete \\
\textsc{ocfw\_u\_mid\_op   } & \textsc{ed        }    & 0.275    & 0.358    & 0.411    & 0.459    & 0.531    & 0.506    & 92  & moderate \\
\textsc{ocfw\_u\_late\_cl  } & \textsc{ed        }    & 0.122    & 0.252    & 0.357    & 0.455    & 0.552    & 0.054    & 1   & complete \\
\textsc{ocfw\_u\_late\_mod } & \textsc{ed        }    & 0.175    & 0.245    & 0.312    & 0.375    & 0.497    & 0.036    & 0   & complete \\
\textsc{ocfw\_u\_late\_op  } & \textsc{ed        }    & 0.03     & 0.067    & 0.099    & 0.136    & 0.205    & 0.009    & 1   & complete \\
\textsc{ocfw\_u\_early\_all} & \textsc{gyrate\_am}    & 97.515   & 114.246  & 136.644  & 163.976  & 260.959  & 186.593  & 87  & moderate \\
\textsc{ocfw\_u\_mid\_cl   } & \textsc{gyrate\_am}    & 15       & 44.028   & 59.583   & 79.244   & 128.623  & 171.396  & 98  & complete \\
\textsc{ocfw\_u\_mid\_mod  } & \textsc{gyrate\_am}    & 56.761   & 79.575   & 97.097   & 133.848  & 220.106  & 117.46   & 64  & none     \\
\textsc{ocfw\_u\_mid\_op   } & \textsc{gyrate\_am}    & 110.14   & 138.59   & 158.274  & 190.64   & 269.705  & 178.228  & 69  & none     \\
\textsc{ocfw\_u\_late\_cl  } & \textsc{gyrate\_am}    & 93.431   & 125.182  & 155.544  & 194.942  & 277.949  & 202.133  & 79  & moderate \\
\textsc{ocfw\_u\_late\_mod } & \textsc{gyrate\_am}    & 82.872   & 102.894  & 126.043  & 152.941  & 222.291  & 191.121  & 87  & moderate \\
\textsc{ocfw\_u\_late\_op  } & \textsc{gyrate\_am}    & 56.477   & 65.774   & 74.294   & 87.479   & 140.476  & 57.455   & 7   & moderate \\
\textsc{ocfw\_u\_early\_all} & \textsc{iji       }    & 54.005   & 55.603   & 56.701   & 57.797   & 59.263   & 52.768   & 2   & complete \\
\textsc{ocfw\_u\_mid\_cl   } & \textsc{iji       }    & 6.568    & 18.879   & 28.093   & 35.961   & 43.428   & 54.818   & 100 & complete \\
\textsc{ocfw\_u\_mid\_mod  } & \textsc{iji       }    & 42.44    & 49.441   & 52.065   & 54.186   & 56.682   & 51.763   & 47  & none     \\
\textsc{ocfw\_u\_mid\_op   } & \textsc{iji       }    & 54.041   & 55.906   & 57.456   & 58.577   & 60.792   & 56.976   & 41  & none     \\
\textsc{ocfw\_u\_late\_cl  } & \textsc{iji       }    & 51.846   & 55.034   & 56.394   & 57.569   & 59.162   & 46.059   & 0   & complete \\
\textsc{ocfw\_u\_late\_mod } & \textsc{iji       }    & 50.44    & 53.926   & 55.092   & 56.525   & 58.761   & 46.348   & 1   & complete \\
\textsc{ocfw\_u\_late\_op  } & \textsc{iji       }    & 38.364   & 44.893   & 48.612   & 51.6     & 54.491   & 31.045   & 1   & complete \\
\textsc{ocfw\_u\_early\_all} & \textsc{pd        }    & 0.022    & 0.029    & 0.035    & 0.039    & 0.045    & 0.018    & 1   & complete \\
\textsc{ocfw\_u\_mid\_cl   } & \textsc{pd        }    & 0.001    & 0.001    & 0.001    & 0.001    & 0.002    & 0.027    & 100 & complete \\
\textsc{ocfw\_u\_mid\_mod  } & \textsc{pd        }    & 0.004    & 0.006    & 0.009    & 0.012    & 0.017    & 0.018    & 97  & complete \\
\textsc{ocfw\_u\_mid\_op   } & \textsc{pd        }    & 0.026    & 0.03     & 0.034    & 0.038    & 0.043    & 0.036    & 67  & none     \\
\textsc{ocfw\_u\_late\_cl  } & \textsc{pd        }    & 0.014    & 0.023    & 0.03     & 0.035    & 0.042    & 0.003    & 0   & complete \\
\textsc{ocfw\_u\_late\_mod } & \textsc{pd        }    & 0.02     & 0.026    & 0.03     & 0.034    & 0.04     & 0.002    & 0   & complete \\
\textsc{ocfw\_u\_late\_op  } & \textsc{pd        }    & 0.005    & 0.009    & 0.013    & 0.018    & 0.025    & 0.002    & 1   & complete \\
\textsc{ocfw\_u\_early\_all} & \textsc{pland     }    & 0.062    & 0.095    & 0.117    & 0.153    & 0.195    & 0.101    & 31  & none     \\
\textsc{ocfw\_u\_mid\_cl   } & \textsc{pland     }    & 0        & 0.001    & 0.001    & 0.003    & 0.008    & 0.172    & 100 & complete \\
\textsc{ocfw\_u\_mid\_mod  } & \textsc{pland     }    & 0.007    & 0.014    & 0.026    & 0.041    & 0.069    & 0.068    & 95  & complete \\
\textsc{ocfw\_u\_mid\_op   } & \textsc{pland     }    & 0.097    & 0.131    & 0.16     & 0.188    & 0.219    & 0.196    & 82  & moderate \\
\textsc{ocfw\_u\_late\_cl  } & \textsc{pland     }    & 0.037    & 0.088    & 0.13     & 0.177    & 0.238    & 0.032    & 4   & complete \\
\textsc{ocfw\_u\_late\_mod } & \textsc{pland     }    & 0.049    & 0.075    & 0.101    & 0.132    & 0.192    & 0.013    & 0   & complete \\
\textsc{ocfw\_u\_late\_op  } & \textsc{pland     }    & 0.007    & 0.016    & 0.025    & 0.036    & 0.058    & 0.002    & 1   & complete \\
\textsc{ocfw\_u\_early\_all} & \textsc{shape\_am }    & 1.477    & 1.569    & 1.685    & 1.797    & 2.015    & 1.898    & 90  & moderate \\
\textsc{ocfw\_u\_mid\_cl   } & \textsc{shape\_am }    & 1        & 1.17     & 1.294    & 1.483    & 1.75     & 1.801    & 97  & complete \\
\textsc{ocfw\_u\_mid\_mod  } & \textsc{shape\_am }    & 1.287    & 1.387    & 1.497    & 1.641    & 1.999    & 1.534    & 58  & none     \\
\textsc{ocfw\_u\_mid\_op   } & \textsc{shape\_am }    & 1.522    & 1.669    & 1.78     & 1.868    & 2.102    & 1.844    & 71  & none     \\
\textsc{ocfw\_u\_late\_cl  } & \textsc{shape\_am }    & 1.472    & 1.651    & 1.812    & 1.996    & 2.38     & 1.556    & 13  & moderate \\
\textsc{ocfw\_u\_late\_mod } & \textsc{shape\_am }    & 1.426    & 1.564    & 1.698    & 1.888    & 2.273    & 1.937    & 79  & moderate \\
\textsc{ocfw\_u\_late\_op  } & \textsc{shape\_am }    & 1.274    & 1.37     & 1.442    & 1.524    & 1.741    & 1.378    & 30  & none     \\
\textsc{ocfw\_u\_early\_all} & \textsc{shape\_mn }    & 1.329    & 1.371    & 1.398    & 1.426    & 1.461    & 1.555    & 100 & complete \\
\textsc{ocfw\_u\_mid\_cl   } & \textsc{shape\_mn }    & 1        & 1.162    & 1.282    & 1.444    & 1.705    & 1.54     & 86  & moderate \\
\textsc{ocfw\_u\_mid\_mod  } & \textsc{shape\_mn }    & 1.249    & 1.305    & 1.346    & 1.394    & 1.47     & 1.391    & 75  & moderate \\
\textsc{ocfw\_u\_mid\_op   } & \textsc{shape\_mn }    & 1.349    & 1.403    & 1.431    & 1.452    & 1.492    & 1.532    & 99  & complete \\
\textsc{ocfw\_u\_late\_cl  } & \textsc{shape\_mn }    & 1.328    & 1.401    & 1.446    & 1.486    & 1.55     & 1.478    & 69  & none     \\
\textsc{ocfw\_u\_late\_mod } & \textsc{shape\_mn }    & 1.315    & 1.373    & 1.409    & 1.453    & 1.513    & 1.75     & 100 & complete \\
\textsc{ocfw\_u\_late\_op  } & \textsc{shape\_mn }    & 1.215    & 1.285    & 1.328    & 1.373    & 1.45     & 1.259    & 15  & moderate \\
\textsc{ocfw\_u\_early\_all} & \textsc{simi\_mn  }    & 1743.398 & 2686.561 & 3248.354 & 3953.011 & 4846.893 & 6939.216 & 100 & complete \\
\textsc{ocfw\_u\_mid\_cl   } & \textsc{simi\_mn  }    & 223.229  & 408.096  & 716.987  & 4037.735 & 8847.346 & 3413.653 & 71  & none     \\
\textsc{ocfw\_u\_mid\_mod  } & \textsc{simi\_mn  }    & 657.785  & 1743.136 & 2523.49  & 3477.813 & 5385.981 & 2665.766 & 55  & none     \\
\textsc{ocfw\_u\_mid\_op   } & \textsc{simi\_mn  }    & 2218.689 & 2733.611 & 3114.17  & 3531.203 & 4139.423 & 5220.69  & 100 & complete \\
\textsc{ocfw\_u\_late\_cl  } & \textsc{simi\_mn  }    & 1253.019 & 1459.079 & 1582.768 & 1770.012 & 2006.432 & 1216.445 & 5   & complete \\
\textsc{ocfw\_u\_late\_mod } & \textsc{simi\_mn  }    & 1283.078 & 1749.884 & 2018.401 & 2307.27  & 2682.885 & 1261.947 & 5   & complete \\
\textsc{ocfw\_u\_late\_op  } & \textsc{simi\_mn  }    & 750.864  & 1661.292 & 2176.823 & 2770.5   & 3823.903 & 537.273  & 3   & complete \\
\textsc{ocfw\_u\_early\_all} & \textsc{te  	     }    & 36000    & 52020    & 63840    & 74760    & 91020    & 46170    & 15  & moderate \\
\textsc{ocfw\_u\_mid\_cl   } & \textsc{te        }    & 180      & 600      & 1020     & 1800     & 4452     & 73920    & 100 & complete \\
\textsc{ocfw\_u\_mid\_mod  } & \textsc{te        }    & 4980     & 8940     & 15000    & 21000    & 31380    & 34350    & 98  & complete \\
\textsc{ocfw\_u\_mid\_op   } & \textsc{te        }    & 49860    & 65010    & 74640    & 83280    & 96450    & 91830    & 92  & moderate \\
\textsc{ocfw\_u\_late\_cl  } & \textsc{te        }    & 22200    & 45660    & 64860    & 82560    & 100200   & 9840     & 1   & complete \\
\textsc{ocfw\_u\_late\_mod } & \textsc{te        }    & 31830    & 44520    & 56730    & 68100    & 90180    & 6480     & 0   & complete \\
\textsc{ocfw\_u\_late\_op  } & \textsc{te        }    & 5520     & 12240    & 18000    & 24720    & 37200    & 1620     & 1   & complete


\end{longtable}
\end{footnotesize}
\end{center}
\end{landscape}

\pagestyle{headings}

%%%%%%%%%%%%%%%%%%%%%%%%%%%%%%%%%%%%%%%%%%%%%%%%%%%%%%%%%%%%%%%%%%%%%%%%%%%%%%%%%%%%%%%%%%%%%%%%
%%%%%%%%%%%%%%%%%%%%%%%%%%%%%%%%%%%%%%%%%%%%%%%%%%%%%%%%%%%%%%%%%%%%%%%%%%%%%%%%%%%%%%%%%%%%%%%%

\pagestyle{empty}
\begin{landscape}
\footnotesize
\begin{center}
\begin{footnotesize}
\begin{longtable}{llrrrrr|rrr}

\caption{Unabridged results for class-level metrics for Red Fir - Mesic (\textsc{rfr\_m}) calculated with \textsc{Fragstats}. This table shows the range of variability in landscape structure. Included are the $5^{\text{th}}$ percentile, $25^{\text{th}}$ percentile, $50^{\text{th}}$ percentile, $75^{\text{th}}$ percentile, and $95^{\text{th}}$ percentiles of the distribution, as well as the current class value, the current percentile range of variability (\%RV) for that proportion, and the departure classification. For seral stage abbreviations, see Table~\ref{condtable}.} \\
\label{tab:fragclass_rfrm} \\

\hline 
\textbf{\begin{tabular}[c]{@{}l@{}}Cover-Seral Stage Type\end{tabular}}  &   
\textbf{\begin{tabular}[c]{@{}l@{}}Landscape\\ Metric\end{tabular}}  &   
\textbf{$5^{\text{th}}$ } &   
\textbf{$25^{\text{th}}$ } &   
\textbf{$50^{\text{th}}$ } &   
\textbf{$75^{\text{th}}$ } &   
\textbf{$95^{\text{th}}$ }  &  
\textbf{\begin{tabular}[c]{@{}l@{}}Current\\ Value\end{tabular}} &   
\textbf{\begin{tabular}[c]{@{}l@{}}Current\\ \%RV\end{tabular}} &   
\textbf{\begin{tabular}[c]{@{}l@{}}Departure\end{tabular}} \\  \\ \hline 
\endfirsthead

\multicolumn{10}{c}{{\bfseries \tablename\ \thetable{} -- continued from previous page}} \\
\hline 
\textbf{\begin{tabular}[c]{@{}l@{}}Cover-Seral Stage Type\end{tabular}}  &   
\textbf{\begin{tabular}[c]{@{}l@{}}Landscape\\ Metric\end{tabular}}  &   
\textbf{$5^{\text{th}}$ } &   
\textbf{$25^{\text{th}}$ } &   
\textbf{$50^{\text{th}}$ } &   
\textbf{$75^{\text{th}}$ } &   
\textbf{$95^{\text{th}}$ }  &  
\textbf{\begin{tabular}[c]{@{}l@{}}Current\\ Value\end{tabular}} &   
\textbf{\begin{tabular}[c]{@{}l@{}}Current\\ \%RV\end{tabular}} &   
\textbf{\begin{tabular}[c]{@{}l@{}}Departure\end{tabular}} \\  \\ \hline \endhead

\hline \multicolumn{10}{|l|}{{Continued on next page}} \\ \hline
\endfoot

\hline \hline
\endlastfoot

\textsc{rfr\_m\_early\_all} & \textsc{ai        }   & 75.722  & 77.749  & 79.671  & 81.287   & 83.845   & 78.365  & 32  & none     \\
\textsc{rfr\_m\_mid\_cl   } & \textsc{ai        }   & 80.61   & 81.821  & 82.796  & 84.068   & 85.491   & 75.19   & 0   & complete \\
\textsc{rfr\_m\_mid\_mod  } & \textsc{ai        }   & 35.652  & 46.191  & 50.336  & 53.349   & 57.766   & 75.737  & 100 & complete \\
\textsc{rfr\_m\_mid\_op   } & \textsc{ai        }   & 22.368  & 39.56   & 47.15   & 52.976   & 59.53    & 77.17   & 100 & complete \\
\textsc{rfr\_m\_late\_cl  } & \textsc{ai        }   & 81.554  & 82.97   & 83.825  & 84.961   & 85.7     & 78.708  & 0   & complete \\
\textsc{rfr\_m\_late\_mod } & \textsc{ai        }   & 61.976  & 65.712  & 67.617  & 69.945   & 72.793   & 79.567  & 100 & complete \\
\textsc{rfr\_m\_late\_op  } & \textsc{ai        }   & 23.034  & 51.878  & 60.465  & 66.072   & 75.657   & 76.018  & 97  & complete \\
\textsc{rfr\_m\_early\_all} & \textsc{area\_am  }   & 11.868  & 21.97   & 35.166  & 68.483   & 214.984  & 15.669  & 11  & moderate \\
\textsc{rfr\_m\_mid\_cl   } & \textsc{area\_am  }   & 32.076  & 49.072  & 72.092  & 145.357  & 270.35   & 8.264   & 0   & complete \\
\textsc{rfr\_m\_mid\_mod  } & \textsc{area\_am  }   & 0.408   & 0.649   & 0.787   & 0.925    & 1.119    & 10.406  & 100 & complete \\
\textsc{rfr\_m\_mid\_op   } & \textsc{area\_am  }   & 0.18    & 0.45    & 0.673   & 0.859    & 1.285    & 12.773  & 100 & complete \\
\textsc{rfr\_m\_late\_cl  } & \textsc{area\_am  }   & 34.015  & 52.755  & 83.46   & 149.106  & 240.122  & 13.681  & 0   & complete \\
\textsc{rfr\_m\_late\_mod } & \textsc{area\_am  }   & 1.395   & 1.98    & 2.38    & 3.301    & 4.8      & 16.203  & 100 & complete \\
\textsc{rfr\_m\_late\_op  } & \textsc{area\_am  }   & 0.15    & 0.738   & 1.029   & 1.472    & 2.414    & 6.52    & 100 & complete \\
\textsc{rfr\_m\_early\_all} & \textsc{area\_mn  }   & 2.276   & 2.747   & 3.358   & 4.026    & 5.405    & 3.98    & 74  & none     \\
\textsc{rfr\_m\_mid\_cl   } & \textsc{area\_mn  }   & 4.312   & 5.135   & 5.839   & 6.772    & 8.484    & 2.312   & 0   & complete \\
\textsc{rfr\_m\_mid\_mod  } & \textsc{area\_mn  }   & 0.204   & 0.28    & 0.332   & 0.386    & 0.464    & 3.298   & 100 & complete \\
\textsc{rfr\_m\_mid\_op   } & \textsc{area\_mn  }   & 0.135   & 0.216   & 0.275   & 0.34     & 0.441    & 3.886   & 100 & complete \\
\textsc{rfr\_m\_late\_cl  } & \textsc{area\_mn  }   & 4.894   & 5.878   & 6.796   & 7.71     & 8.771    & 4.173   & 0   & complete \\
\textsc{rfr\_m\_late\_mod } & \textsc{area\_mn  }   & 0.663   & 0.891   & 1.066   & 1.285    & 1.586    & 5.074   & 100 & complete \\
\textsc{rfr\_m\_late\_op  } & \textsc{area\_mn  }   & 0.129   & 0.27    & 0.382   & 0.57     & 0.9      & 2.989   & 100 & complete \\
\textsc{rfr\_m\_early\_all} & \textsc{cai\_am   }   & 90.269  & 92.144  & 93.175  & 93.866   & 94.948   & 88.159  & 1   & complete \\
\textsc{rfr\_m\_mid\_cl   } & \textsc{cai\_am   }   & 21.35   & 27.638  & 30.906  & 34.46    & 39.055   & 29.957  & 41  & none     \\
\textsc{rfr\_m\_mid\_mod  } & \textsc{cai\_am   }   & 3.738   & 8.929   & 15.094  & 21.053   & 31.818   & 53.317  & 100 & complete \\
\textsc{rfr\_m\_mid\_op   } & \textsc{cai\_am   }   & 14.706  & 30.909  & 40      & 47.771   & 59.575   & 73.177  & 100 & complete \\
\textsc{rfr\_m\_late\_cl  } & \textsc{cai\_am   }   & 17.424  & 21.982  & 26.222  & 31.014   & 35.889   & 22.106  & 26  & none     \\
\textsc{rfr\_m\_late\_mod } & \textsc{cai\_am   }   & 26.203  & 33.094  & 38.408  & 45.088   & 53.568   & 46.455  & 80  & moderate \\
\textsc{rfr\_m\_late\_op  } & \textsc{cai\_am   }   & 0       & 7.143   & 19.54   & 31.579   & 47.917   & 34.695  & 82  & moderate \\
\textsc{rfr\_m\_early\_all} & \textsc{clumpy    }   & 0.756   & 0.776   & 0.795   & 0.812    & 0.836    & 0.781   & 30  & none     \\
\textsc{rfr\_m\_mid\_cl   } & \textsc{clumpy    }   & 0.804   & 0.815   & 0.825   & 0.837    & 0.851    & 0.752   & 0   & complete \\
\textsc{rfr\_m\_mid\_mod  } & \textsc{clumpy    }   & 0.356   & 0.462   & 0.503   & 0.533    & 0.578    & 0.755   & 100 & complete \\
\textsc{rfr\_m\_mid\_op   } & \textsc{clumpy    }   & 0.224   & 0.396   & 0.472   & 0.53     & 0.595    & 0.77    & 100 & complete \\
\textsc{rfr\_m\_late\_cl  } & \textsc{clumpy    }   & 0.813   & 0.827   & 0.835   & 0.846    & 0.854    & 0.786   & 0   & complete \\
\textsc{rfr\_m\_late\_mod } & \textsc{clumpy    }   & 0.62    & 0.657   & 0.676   & 0.699    & 0.728    & 0.794   & 100 & complete \\
\textsc{rfr\_m\_late\_op  } & \textsc{clumpy    }   & 0.23    & 0.519   & 0.605   & 0.661    & 0.757    & 0.76    & 96  & complete \\
\textsc{rfr\_m\_early\_all} & \textsc{core\_am  }   & 11.107  & 20      & 32.644  & 63.639   & 200.939  & 13.424  & 9   & moderate \\
\textsc{rfr\_m\_mid\_cl   } & \textsc{core\_am  }   & 11.966  & 20.07   & 34.057  & 72.244   & 135.988  & 2.399   & 0   & complete \\
\textsc{rfr\_m\_mid\_mod  } & \textsc{core\_am  }   & 0.015   & 0.085   & 0.155   & 0.242    & 0.42     & 5.792   & 100 & complete \\
\textsc{rfr\_m\_mid\_op   } & \textsc{core\_am  }   & 0.029   & 0.18    & 0.317   & 0.471    & 0.818    & 9.807   & 100 & complete \\
\textsc{rfr\_m\_late\_cl  } & \textsc{core\_am  }   & 9.706   & 16.68   & 35.325  & 67.875   & 120.077  & 3.666   & 0   & complete \\
\textsc{rfr\_m\_late\_mod } & \textsc{core\_am  }   & 0.472   & 0.794   & 1.131   & 1.56     & 2.857    & 9.217   & 100 & complete \\
\textsc{rfr\_m\_late\_op  } & \textsc{core\_am  }   & 0       & 0.053   & 0.266   & 0.475    & 1.005    & 2.412   & 100 & complete \\
\textsc{rfr\_m\_early\_all} & \textsc{core\_mn  }   & 2.117   & 2.542   & 3.123   & 3.73     & 5.035    & 3.508   & 67  & none     \\
\textsc{rfr\_m\_mid\_cl   } & \textsc{core\_mn  }   & 1.021   & 1.488   & 1.811   & 2.245    & 2.763    & 0.693   & 1   & complete \\
\textsc{rfr\_m\_mid\_mod  } & \textsc{core\_mn  }   & 0.01    & 0.027   & 0.05    & 0.074    & 0.113    & 1.758   & 100 & complete \\
\textsc{rfr\_m\_mid\_op   } & \textsc{core\_mn  }   & 0.026   & 0.072   & 0.111   & 0.15     & 0.217    & 2.844   & 100 & complete \\
\textsc{rfr\_m\_late\_cl  } & \textsc{core\_mn  }   & 0.919   & 1.3     & 1.781   & 2.406    & 2.998    & 0.922   & 6   & moderate \\
\textsc{rfr\_m\_late\_mod } & \textsc{core\_mn  }   & 0.208   & 0.319   & 0.42    & 0.532    & 0.731    & 2.357   & 100 & complete \\
\textsc{rfr\_m\_late\_op  } & \textsc{core\_mn  }   & 0       & 0.022   & 0.076   & 0.154    & 0.303    & 1.037   & 100 & complete \\
\textsc{rfr\_m\_early\_all} & \textsc{cpland    }   & 0.292   & 0.474   & 0.738   & 1.106    & 1.573    & 0.991   & 68  & none     \\
\textsc{rfr\_m\_mid\_cl   } & \textsc{cpland    }   & 0.228   & 0.435   & 0.573   & 0.707    & 0.891    & 0.049   & 0   & complete \\
\textsc{rfr\_m\_mid\_mod  } & \textsc{cpland    }   & 0       & 0       & 0.001   & 0.002    & 0.003    & 0.481   & 100 & complete \\
\textsc{rfr\_m\_mid\_op   } & \textsc{cpland    }   & 0       & 0.001   & 0.001   & 0.002    & 0.005    & 0.572   & 100 & complete \\
\textsc{rfr\_m\_late\_cl  } & \textsc{cpland    }   & 0.24    & 0.35    & 0.514   & 0.706    & 0.925    & 0.112   & 1   & complete \\
\textsc{rfr\_m\_late\_mod } & \textsc{cpland    }   & 0.004   & 0.009   & 0.014   & 0.02     & 0.033    & 0.487   & 100 & complete \\
\textsc{rfr\_m\_late\_op  } & \textsc{cpland    }   & 0       & 0       & 0.001   & 0.001    & 0.004    & 0.066   & 100 & complete \\
\textsc{rfr\_m\_early\_all} & \textsc{cwed      }   & 0.357   & 0.55    & 0.787   & 1.04     & 1.429    & 0.957   & 68  & none     \\
\textsc{rfr\_m\_mid\_cl   } & \textsc{cwed      }   & 0.92    & 1.263   & 1.491   & 1.67     & 1.972    & 0.138   & 0   & complete \\
\textsc{rfr\_m\_mid\_mod  } & \textsc{cwed      }   & 0.007   & 0.012   & 0.016   & 0.022    & 0.031    & 0.721   & 100 & complete \\
\textsc{rfr\_m\_mid\_op   } & \textsc{cwed      }   & 0.002   & 0.005   & 0.008   & 0.012    & 0.02     & 0.576   & 100 & complete \\
\textsc{rfr\_m\_late\_cl  } & \textsc{cwed      }   & 1.252   & 1.506   & 1.68    & 1.891    & 2.141    & 0.397   & 0   & complete \\
\textsc{rfr\_m\_late\_mod } & \textsc{cwed      }   & 0.019   & 0.034   & 0.048   & 0.066    & 0.099    & 0.794   & 100 & complete \\
\textsc{rfr\_m\_late\_op  } & \textsc{cwed      }   & 0.002   & 0.004   & 0.006   & 0.009    & 0.017    & 0.18    & 100 & complete \\
\textsc{rfr\_m\_early\_all} & \textsc{econ\_am  }   & 32.868  & 34.361  & 35.609  & 36.979   & 39.396   & 29.035  & 0   & complete \\
\textsc{rfr\_m\_mid\_cl   } & \textsc{econ\_am  }   & 30.721  & 33.755  & 35.084  & 37.251   & 39.275   & 25.026  & 0   & complete \\
\textsc{rfr\_m\_mid\_mod  } & \textsc{econ\_am  }   & 29.544  & 31.673  & 33.068  & 34.698   & 37.005   & 24.266  & 0   & complete \\
\textsc{rfr\_m\_mid\_op   } & \textsc{econ\_am  }   & 25.179  & 27.696  & 29.351  & 30.936   & 33.478   & 23.355  & 2   & complete \\
\textsc{rfr\_m\_late\_cl  } & \textsc{econ\_am  }   & 35.873  & 37.745  & 39.81   & 41.658   & 44.799   & 27.341  & 0   & complete \\
\textsc{rfr\_m\_late\_mod } & \textsc{econ\_am  }   & 22.621  & 25.58   & 27.742  & 29.725   & 32.255   & 26.967  & 40  & none     \\
\textsc{rfr\_m\_late\_op  } & \textsc{econ\_am  }   & 25.097  & 29.018  & 31.618  & 34.285   & 40.125   & 28.112  & 19  & moderate \\
\textsc{rfr\_m\_early\_all} & \textsc{econ\_mn  }   & 30.471  & 31.961  & 33.03   & 34.324   & 35.891   & 27.699  & 0   & complete \\
\textsc{rfr\_m\_mid\_cl   } & \textsc{econ\_mn  }   & 32.282  & 34.341  & 35.999  & 38.046   & 40.093   & 23.522  & 0   & complete \\
\textsc{rfr\_m\_mid\_mod  } & \textsc{econ\_mn  }   & 31.904  & 33.622  & 34.818  & 36.064   & 37.49    & 23.454  & 0   & complete \\
\textsc{rfr\_m\_mid\_op   } & \textsc{econ\_mn  }   & 26.232  & 28.253  & 29.396  & 30.65    & 32.888   & 24.264  & 1   & complete \\
\textsc{rfr\_m\_late\_cl  } & \textsc{econ\_mn  }   & 36.522  & 38.441  & 40.417  & 42.303   & 45.597   & 25.033  & 0   & complete \\
\textsc{rfr\_m\_late\_mod } & \textsc{econ\_mn  }   & 29.38   & 31.9    & 33.766  & 35.383   & 37.6     & 27.467  & 1   & complete \\
\textsc{rfr\_m\_late\_op  } & \textsc{econ\_mn  }   & 29.434  & 32.185  & 34.423  & 36.924   & 41.591   & 28.176  & 2   & complete \\
\textsc{rfr\_m\_early\_all} & \textsc{ed        }   & 0.972   & 1.558   & 2.202   & 2.998    & 4.191    & 3.28    & 82  & moderate \\
\textsc{rfr\_m\_mid\_cl   } & \textsc{ed        }   & 2.577   & 3.568   & 4.155   & 4.753    & 5.407    & 0.563   & 0   & complete \\
\textsc{rfr\_m\_mid\_mod  } & \textsc{ed        }   & 0.022   & 0.035   & 0.046   & 0.063    & 0.089    & 2.97    & 100 & complete \\
\textsc{rfr\_m\_mid\_op   } & \textsc{ed        }   & 0.008   & 0.016   & 0.026   & 0.04     & 0.069    & 2.413   & 100 & complete \\
\textsc{rfr\_m\_late\_cl  } & \textsc{ed        }   & 3.073   & 3.719   & 4.252   & 4.71     & 5.523    & 1.49    & 0   & complete \\
\textsc{rfr\_m\_late\_mod } & \textsc{ed        }   & 0.064   & 0.118   & 0.168   & 0.227    & 0.339    & 2.914   & 100 & complete \\
\textsc{rfr\_m\_late\_op  } & \textsc{ed        }   & 0.005   & 0.011   & 0.019   & 0.029    & 0.058    & 0.634   & 100 & complete \\
\textsc{rfr\_m\_early\_all} & \textsc{gyrate\_am}   & 140.819 & 191.431 & 246.937 & 338.964  & 612.961  & 181.32  & 21  & moderate \\
\textsc{rfr\_m\_mid\_cl   } & \textsc{gyrate\_am}   & 239.064 & 302.366 & 368.053 & 495.961  & 665.156  & 126.366 & 0   & complete \\
\textsc{rfr\_m\_mid\_mod  } & \textsc{gyrate\_am}   & 25.25   & 31.98   & 36.054  & 39.594   & 45.014   & 146.7   & 100 & complete \\
\textsc{rfr\_m\_mid\_op   } & \textsc{gyrate\_am}   & 16.906  & 26.828  & 32.825  & 38.196   & 45.465   & 162.066 & 100 & complete \\
\textsc{rfr\_m\_late\_cl  } & \textsc{gyrate\_am}   & 253.954 & 317.393 & 412.206 & 506.07   & 678.233  & 167.222 & 0   & complete \\
\textsc{rfr\_m\_late\_mod } & \textsc{gyrate\_am}   & 49.628  & 59.678  & 65.83   & 76.233   & 93.299   & 177.453 & 100 & complete \\
\textsc{rfr\_m\_late\_op  } & \textsc{gyrate\_am}   & 15.725  & 33.043  & 41.462  & 49.657   & 67.021   & 115.594 & 100 & complete \\
\textsc{rfr\_m\_early\_all} & \textsc{iji       }   & 55.602  & 56.958  & 57.881  & 58.843   & 60.208   & 58.127  & 57  & none     \\
\textsc{rfr\_m\_mid\_cl   } & \textsc{iji       }   & 55.894  & 57.155  & 58.124  & 58.847   & 60.024   & 59.957  & 95  & complete \\
\textsc{rfr\_m\_mid\_mod  } & \textsc{iji       }   & 45.854  & 50.142  & 52.778  & 55.047   & 57.763   & 61.602  & 100 & complete \\
\textsc{rfr\_m\_mid\_op   } & \textsc{iji       }   & 38.513  & 45.79   & 49.109  & 52.327   & 55.827   & 61.212  & 100 & complete \\
\textsc{rfr\_m\_late\_cl  } & \textsc{iji       }   & 56.181  & 57.904  & 59.044  & 60.549   & 62.069   & 57.207  & 16  & moderate \\
\textsc{rfr\_m\_late\_mod } & \textsc{iji       }   & 39.132  & 43.391  & 46.32   & 48.717   & 52.697   & 60.007  & 100 & complete \\
\textsc{rfr\_m\_late\_op  } & \textsc{iji       }   & 35.416  & 43.181  & 47.146  & 50.244   & 54.063   & 57.371  & 100 & complete \\
\textsc{rfr\_m\_early\_all} & \textsc{pd        }   & 0.122   & 0.175   & 0.24    & 0.294    & 0.388    & 0.283   & 72  & none     \\
\textsc{rfr\_m\_mid\_cl   } & \textsc{pd        }   & 0.204   & 0.266   & 0.305   & 0.337    & 0.38     & 0.07    & 0   & complete \\
\textsc{rfr\_m\_mid\_mod  } & \textsc{pd        }   & 0.01    & 0.015   & 0.02    & 0.025    & 0.034    & 0.274   & 100 & complete \\
\textsc{rfr\_m\_mid\_op   } & \textsc{pd        }   & 0.004   & 0.008   & 0.013   & 0.018    & 0.029    & 0.201   & 100 & complete \\
\textsc{rfr\_m\_late\_cl  } & \textsc{pd        }   & 0.234   & 0.266   & 0.288   & 0.311    & 0.344    & 0.122   & 0   & complete \\
\textsc{rfr\_m\_late\_mod } & \textsc{pd        }   & 0.016   & 0.025   & 0.034   & 0.042    & 0.06     & 0.207   & 100 & complete \\
\textsc{rfr\_m\_late\_op  } & \textsc{pd        }   & 0.003   & 0.005   & 0.008   & 0.01     & 0.017    & 0.063   & 100 & complete \\
\textsc{rfr\_m\_early\_all} & \textsc{pland     }   & 0.314   & 0.513   & 0.797   & 1.197    & 1.694    & 1.125   & 72  & none     \\
\textsc{rfr\_m\_mid\_cl   } & \textsc{pland     }   & 1.033   & 1.528   & 1.812   & 2.125    & 2.575    & 0.163   & 0   & complete \\
\textsc{rfr\_m\_mid\_mod  } & \textsc{pland     }   & 0.003   & 0.005   & 0.006   & 0.009    & 0.014    & 0.903   & 100 & complete \\
\textsc{rfr\_m\_mid\_op   } & \textsc{pland     }   & 0.001   & 0.002   & 0.004   & 0.005    & 0.01     & 0.781   & 100 & complete \\
\textsc{rfr\_m\_late\_cl  } & \textsc{pland     }   & 1.286   & 1.606   & 1.975   & 2.278    & 2.715    & 0.508   & 0   & complete \\
\textsc{rfr\_m\_late\_mod } & \textsc{pland     }   & 0.012   & 0.025   & 0.036   & 0.052    & 0.084    & 1.048   & 100 & complete \\
\textsc{rfr\_m\_late\_op  } & \textsc{pland     }   & 0       & 0.002   & 0.003   & 0.005    & 0.011    & 0.189   & 100 & complete \\
\textsc{rfr\_m\_early\_all} & \textsc{shape\_am }   & 1.818   & 2.025   & 2.362   & 2.786    & 3.826    & 2.153   & 36  & none     \\
\textsc{rfr\_m\_mid\_cl   } & \textsc{shape\_am }   & 2.259   & 2.659   & 3.001   & 3.463    & 4.218    & 1.776   & 0   & complete \\
\textsc{rfr\_m\_mid\_mod  } & \textsc{shape\_am }   & 1.093   & 1.15    & 1.194   & 1.239    & 1.304    & 2.005   & 100 & complete \\
\textsc{rfr\_m\_mid\_op   } & \textsc{shape\_am }   & 1.022   & 1.111   & 1.172   & 1.219    & 1.306    & 2.097   & 100 & complete \\
\textsc{rfr\_m\_late\_cl  } & \textsc{shape\_am }   & 2.327   & 2.604   & 3.048   & 3.437    & 4.231    & 2.042   & 0   & complete \\
\textsc{rfr\_m\_late\_mod } & \textsc{shape\_am }   & 1.295   & 1.368   & 1.434   & 1.519    & 1.642    & 2.079   & 100 & complete \\
\textsc{rfr\_m\_late\_op  } & \textsc{shape\_am }   & 1       & 1.132   & 1.217   & 1.294    & 1.461    & 1.725   & 100 & complete \\
\textsc{rfr\_m\_early\_all} & \textsc{shape\_mn }   & 1.317   & 1.352   & 1.378   & 1.409    & 1.448    & 1.496   & 100 & complete \\
\textsc{rfr\_m\_mid\_cl   } & \textsc{shape\_mn }   & 1.429   & 1.462   & 1.484   & 1.504    & 1.537    & 1.37    & 1   & complete \\
\textsc{rfr\_m\_mid\_mod  } & \textsc{shape\_mn }   & 1.032   & 1.054   & 1.073   & 1.092    & 1.121    & 1.495   & 100 & complete \\
\textsc{rfr\_m\_mid\_op   } & \textsc{shape\_mn }   & 1.009   & 1.037   & 1.057   & 1.076    & 1.109    & 1.539   & 100 & complete \\
\textsc{rfr\_m\_late\_cl  } & \textsc{shape\_mn }   & 1.448   & 1.475   & 1.493   & 1.516    & 1.552    & 1.527   & 86  & moderate \\
\textsc{rfr\_m\_late\_mod } & \textsc{shape\_mn }   & 1.141   & 1.188   & 1.222   & 1.256    & 1.305    & 1.597   & 100 & complete \\
\textsc{rfr\_m\_late\_op  } & \textsc{shape\_mn }   & 1       & 1.04    & 1.071   & 1.105    & 1.167    & 1.436   & 100 & complete \\
\textsc{rfr\_m\_early\_all} & \textsc{simi\_mn  }   & 752.532 & 847.56  & 948.509 & 1075.977 & 1287.228 & 809.215 & 16  & moderate \\
\textsc{rfr\_m\_mid\_cl   } & \textsc{simi\_mn  }   & 796.585 & 886.113 & 986.131 & 1126.76  & 1415.017 & 445.371 & 0   & complete \\
\textsc{rfr\_m\_mid\_mod  } & \textsc{simi\_mn  }   & 357.509 & 480.875 & 598.787 & 767.223  & 1039.057 & 463.176 & 22  & moderate \\
\textsc{rfr\_m\_mid\_op   } & \textsc{simi\_mn  }   & 322.959 & 456.615 & 561.452 & 727.217  & 1268.018 & 547.025 & 45  & none     \\
\textsc{rfr\_m\_late\_cl  } & \textsc{simi\_mn  }   & 745.485 & 852.004 & 958.581 & 1067.189 & 1366.464 & 412.672 & 0   & complete \\
\textsc{rfr\_m\_late\_mod } & \textsc{simi\_mn  }   & 525.049 & 639.516 & 768.712 & 962.985  & 1728.973 & 544.149 & 9   & moderate \\
\textsc{rfr\_m\_late\_op  } & \textsc{simi\_mn  }   & 293.969 & 396.311 & 507.572 & 647.663  & 1153.883 & 663.665 & 77  & moderate \\
\textsc{rfr\_m\_early\_all} & \textsc{te  	    }   & 176370  & 282900  & 399780  & 544380   & 760800   & 595560  & 82  & moderate \\
\textsc{rfr\_m\_mid\_cl   } & \textsc{te        }   & 467790  & 647790  & 754320  & 863010   & 981630   & 102150  & 0   & complete \\
\textsc{rfr\_m\_mid\_mod  } & \textsc{te        }   & 4020    & 6360    & 8430    & 11430    & 16140    & 539160  & 100 & complete \\
\textsc{rfr\_m\_mid\_op   } & \textsc{te        }   & 1500    & 2940    & 4770    & 7200     & 12600    & 438000  & 100 & complete \\
\textsc{rfr\_m\_late\_cl  } & \textsc{te        }   & 557880  & 675270  & 771870  & 855210   & 1002750  & 270510  & 0   & complete \\
\textsc{rfr\_m\_late\_mod } & \textsc{te        }   & 11580   & 21420   & 30480   & 41220    & 61530    & 529050  & 100 & complete \\
\textsc{rfr\_m\_late\_op  } & \textsc{te        }   & 900     & 2070    & 3420    & 5220     & 10500    & 115140  & 100 & complete

\end{longtable}
\end{footnotesize}
\end{center}
\end{landscape}


\pagestyle{headings}

%%%%%%%%%%%%%%%%%%%%%%%%%%%%%%%%%%%%%%%%%%%%%%%%%%%%%%%%%%%%%%%%%%%%%%%%%%%%%%%%%%%%%%%%%%%%%%%%
%%%%%%%%%%%%%%%%%%%%%%%%%%%%%%%%%%%%%%%%%%%%%%%%%%%%%%%%%%%%%%%%%%%%%%%%%%%%%%%%%%%%%%%%%%%%%%%%

\pagestyle{empty}
\begin{landscape}
\footnotesize
\begin{center}
\begin{footnotesize}
\begin{longtable}{llrrrrr|rrr}
\caption{Unabridged results for class-level metrics for Red Fir - Xeric (\textsc{rfr\_x}) calculated with \textsc{Fragstats}. This table shows the range of variability in landscape structure. Included are the $5^{\text{th}}$ percentile, $25^{\text{th}}$ percentile, $50^{\text{th}}$ percentile, $75^{\text{th}}$ percentile, and $95^{\text{th}}$ percentiles of the distribution, as well as the current class value, the current percentile range of variability (\%RV) for that proportion, and the departure classification. For seral stage abbreviations, see Table~\ref{condtable}.} \\
\label{tab:fragclass_rfrx} \\

\hline 
\textbf{\begin{tabular}[c]{@{}l@{}}Cover-Seral Stage Type\end{tabular}}  &   
\textbf{\begin{tabular}[c]{@{}l@{}}Landscape\\ Metric\end{tabular}}  &   
\textbf{$5^{\text{th}}$ } &   
\textbf{$25^{\text{th}}$ } &   
\textbf{$50^{\text{th}}$ } &   
\textbf{$75^{\text{th}}$ } &   
\textbf{$95^{\text{th}}$ }  &  
\textbf{\begin{tabular}[c]{@{}l@{}}Current\\ Value\end{tabular}} &   
\textbf{\begin{tabular}[c]{@{}l@{}}Current\\ \%RV\end{tabular}} &   
\textbf{\begin{tabular}[c]{@{}l@{}}Departure\end{tabular}} \\  \\ \hline 
\endfirsthead

\multicolumn{10}{c}{{\bfseries \tablename\ \thetable{} -- continued from previous page}} \\
\hline 
\textbf{\begin{tabular}[c]{@{}l@{}}Cover-Seral Stage Type\end{tabular}}  &   
\textbf{\begin{tabular}[c]{@{}l@{}}Landscape\\ Metric\end{tabular}}  &   
\textbf{$5^{\text{th}}$ } &   
\textbf{$25^{\text{th}}$ } &   
\textbf{$50^{\text{th}}$ } &   
\textbf{$75^{\text{th}}$ } &   
\textbf{$95^{\text{th}}$ }  &  
\textbf{\begin{tabular}[c]{@{}l@{}}Current\\ Value\end{tabular}} &   
\textbf{\begin{tabular}[c]{@{}l@{}}Current\\ \%RV\end{tabular}} &   
\textbf{\begin{tabular}[c]{@{}l@{}}Departure\end{tabular}} \\  \\ \hline \endhead

\hline \multicolumn{10}{|l|}{{Continued on next page}} \\ \hline
\endfoot

\hline \hline
\endlastfoot

\textsc{rfr\_x\_early\_all} & \textsc{ai        }   & 80.119  & 81.124  & 81.826   & 82.608   & 83.884   & 82.016  & 56  & none     \\
\textsc{rfr\_x\_mid\_cl   } & \textsc{ai        }   & 67.426  & 72.449  & 75.894   & 78.623   & 82.482   & 80.956  & 88  & moderate \\
\textsc{rfr\_x\_mid\_mod  } & \textsc{ai        }   & 73.288  & 76.799  & 78.428   & 80.144   & 82.215   & 78.611  & 54  & none     \\
\textsc{rfr\_x\_mid\_op   } & \textsc{ai        }   & 78.933  & 79.957  & 80.9     & 82.034   & 83.288   & 75.934  & 0   & complete \\
\textsc{rfr\_x\_late\_cl  } & \textsc{ai        }   & 69.164  & 73.021  & 74.843   & 77.159   & 79.424   & 79.354  & 95  & complete \\
\textsc{rfr\_x\_late\_mod } & \textsc{ai        }   & 72.823  & 74.971  & 76.707   & 78.221   & 80.197   & 77.075  & 56  & none     \\
\textsc{rfr\_x\_late\_op  } & \textsc{ai        }   & 74.826  & 76.689  & 78.075   & 79.286   & 81.677   & 75.456  & 11  & moderate \\
\textsc{rfr\_x\_early\_all} & \textsc{area\_am  }   & 27.455  & 36.459  & 46.672   & 60.996   & 86.757   & 36.886  & 26  & none     \\
\textsc{rfr\_x\_mid\_cl   } & \textsc{area\_am  }   & 1.439   & 2.421   & 4.654    & 8.158    & 22.054   & 16.84   & 92  & moderate \\
\textsc{rfr\_x\_mid\_mod  } & \textsc{area\_am  }   & 4.275   & 8.554   & 12.565   & 20.198   & 40.276   & 18.934  & 72  & none     \\
\textsc{rfr\_x\_mid\_op   } & \textsc{area\_am  }   & 13.994  & 18.191  & 25.388   & 36.275   & 57.918   & 9.215   & 0   & complete \\
\textsc{rfr\_x\_late\_cl  } & \textsc{area\_am  }   & 3.822   & 6.295   & 9.512    & 16.967   & 37.715   & 10.145  & 54  & none     \\
\textsc{rfr\_x\_late\_mod } & \textsc{area\_am  }   & 3.998   & 6.202   & 9.194    & 15.483   & 29.766   & 12.915  & 66  & none     \\
\textsc{rfr\_x\_late\_op  } & \textsc{area\_am  }   & 5.226   & 8.658   & 13.292   & 21.389   & 45.545   & 5.644   & 10  & moderate \\
\textsc{rfr\_x\_early\_all} & \textsc{area\_mn  }   & 4.231   & 4.889   & 5.336    & 5.829    & 6.526    & 4.522   & 10  & moderate \\
\textsc{rfr\_x\_mid\_cl   } & \textsc{area\_mn  }   & 0.758   & 1.264   & 1.81     & 2.495    & 3.634    & 5.607   & 100 & complete \\
\textsc{rfr\_x\_mid\_mod  } & \textsc{area\_mn  }   & 1.885   & 2.57    & 3.014    & 3.603    & 4.614    & 4.123   & 90  & moderate \\
\textsc{rfr\_x\_mid\_op   } & \textsc{area\_mn  }   & 3.437   & 3.901   & 4.278    & 4.737    & 5.58     & 3.326   & 3   & complete \\
\textsc{rfr\_x\_late\_cl  } & \textsc{area\_mn  }   & 1.061   & 1.497   & 1.846    & 2.407    & 3.139    & 4.238   & 99  & complete \\
\textsc{rfr\_x\_late\_mod } & \textsc{area\_mn  }   & 1.67    & 2.138   & 2.481    & 2.967    & 3.537    & 3.295   & 89  & moderate \\
\textsc{rfr\_x\_late\_op  } & \textsc{area\_mn  }   & 2.076   & 2.58    & 3.03     & 3.559    & 4.379    & 2.662   & 31  & none     \\
\textsc{rfr\_x\_early\_all} & \textsc{cai\_am   }   & 94.478  & 94.853  & 95.154   & 95.453   & 95.895   & 91.818  & 0   & complete \\
\textsc{rfr\_x\_mid\_cl   } & \textsc{cai\_am   }   & 7.619   & 17.391  & 26.41    & 35.632   & 49.333   & 42.444  & 88  & moderate \\
\textsc{rfr\_x\_mid\_mod  } & \textsc{cai\_am   }   & 33.922  & 44.551  & 51.896   & 58.895   & 65.984   & 64.445  & 93  & moderate \\
\textsc{rfr\_x\_mid\_op   } & \textsc{cai\_am   }   & 63.742  & 68.262  & 71.739   & 74.999   & 80.093   & 74.558  & 74  & none     \\
\textsc{rfr\_x\_late\_cl  } & \textsc{cai\_am   }   & 6.007   & 11.906  & 17.154   & 22.372   & 30.975   & 29.709  & 94  & moderate \\
\textsc{rfr\_x\_late\_mod } & \textsc{cai\_am   }   & 23.894  & 30.508  & 34.499   & 39.289   & 45.047   & 49.003  & 98  & complete \\
\textsc{rfr\_x\_late\_op  } & \textsc{cai\_am   }   & 23.182  & 28.89   & 31.404   & 34.898   & 42.362   & 40.964  & 93  & moderate \\
\textsc{rfr\_x\_early\_all} & \textsc{clumpy    }   & 0.797   & 0.808   & 0.814    & 0.822    & 0.835    & 0.818   & 62  & none     \\
\textsc{rfr\_x\_mid\_cl   } & \textsc{clumpy    }   & 0.674   & 0.724   & 0.759    & 0.786    & 0.824    & 0.809   & 88  & moderate \\
\textsc{rfr\_x\_mid\_mod  } & \textsc{clumpy    }   & 0.733   & 0.768   & 0.784    & 0.801    & 0.822    & 0.784   & 51  & none     \\
\textsc{rfr\_x\_mid\_op   } & \textsc{clumpy    }   & 0.788   & 0.798   & 0.807    & 0.818    & 0.831    & 0.758   & 0   & complete \\
\textsc{rfr\_x\_late\_cl  } & \textsc{clumpy    }   & 0.691   & 0.73    & 0.748    & 0.771    & 0.794    & 0.793   & 95  & complete \\
\textsc{rfr\_x\_late\_mod } & \textsc{clumpy    }   & 0.728   & 0.749   & 0.766    & 0.781    & 0.801    & 0.769   & 55  & none     \\
\textsc{rfr\_x\_late\_op  } & \textsc{clumpy    }   & 0.748   & 0.766   & 0.78     & 0.792    & 0.816    & 0.754   & 11  & moderate \\
\textsc{rfr\_x\_early\_all} & \textsc{core\_am  }   & 26.346  & 35.123  & 44.459   & 58.587   & 82.829   & 32.922  & 19  & moderate \\
\textsc{rfr\_x\_mid\_cl   } & \textsc{core\_am  }   & 0.102   & 0.454   & 1.237    & 3.364    & 11.788   & 9.022   & 93  & moderate \\
\textsc{rfr\_x\_mid\_mod  } & \textsc{core\_am  }   & 1.846   & 4.844   & 8.051    & 14.89    & 32.992   & 13.48   & 73  & none     \\
\textsc{rfr\_x\_mid\_op   } & \textsc{core\_am  }   & 10.684  & 14.584  & 20.405   & 30.02    & 47.112   & 7.09    & 0   & complete \\
\textsc{rfr\_x\_late\_cl  } & \textsc{core\_am  }   & 0.245   & 1.176   & 2.504    & 5.528    & 17.349   & 3.631   & 63  & none     \\
\textsc{rfr\_x\_late\_mod } & \textsc{core\_am  }   & 1.312   & 2.645   & 4.266    & 8.455    & 18.737   & 8.041   & 74  & none     \\
\textsc{rfr\_x\_late\_op  } & \textsc{core\_am  }   & 1.64    & 3.397   & 5.903    & 9.832    & 25.167   & 2.803   & 20  & moderate \\
\textsc{rfr\_x\_early\_all} & \textsc{core\_mn  }   & 3.995   & 4.654   & 5.087    & 5.537    & 6.215    & 4.152   & 7   & moderate \\
\textsc{rfr\_x\_mid\_cl   } & \textsc{core\_mn  }   & 0.082   & 0.242   & 0.438    & 0.802    & 1.56     & 2.38    & 100 & complete \\
\textsc{rfr\_x\_mid\_mod  } & \textsc{core\_mn  }   & 0.643   & 1.191   & 1.561    & 2.132    & 2.887    & 2.657   & 93  & moderate \\
\textsc{rfr\_x\_mid\_op   } & \textsc{core\_mn  }   & 2.321   & 2.71    & 3.067    & 3.48     & 4.244    & 2.48    & 12  & moderate \\
\textsc{rfr\_x\_late\_cl  } & \textsc{core\_mn  }   & 0.074   & 0.188   & 0.31     & 0.496    & 0.943    & 1.259   & 99  & complete \\
\textsc{rfr\_x\_late\_mod } & \textsc{core\_mn  }   & 0.428   & 0.67    & 0.873    & 1.143    & 1.525    & 1.615   & 98  & complete \\
\textsc{rfr\_x\_late\_op  } & \textsc{core\_mn  }   & 0.514   & 0.782   & 0.949    & 1.188    & 1.714    & 1.09    & 66  & none     \\
\textsc{rfr\_x\_early\_all} & \textsc{cpland    }   & 1.265   & 1.684   & 1.897    & 2.1      & 2.349    & 1.233   & 5   & complete \\
\textsc{rfr\_x\_mid\_cl   } & \textsc{cpland    }   & 0       & 0.002   & 0.006    & 0.019    & 0.046    & 0.146   & 100 & complete \\
\textsc{rfr\_x\_mid\_mod  } & \textsc{cpland    }   & 0.034   & 0.083   & 0.145    & 0.216    & 0.335    & 0.495   & 100 & complete \\
\textsc{rfr\_x\_mid\_op   } & \textsc{cpland    }   & 0.393   & 0.529   & 0.644    & 0.775    & 0.98     & 0.382   & 5   & complete \\
\textsc{rfr\_x\_late\_cl  } & \textsc{cpland    }   & 0.008   & 0.022   & 0.046    & 0.086    & 0.203    & 0.128   & 85  & moderate \\
\textsc{rfr\_x\_late\_mod } & \textsc{cpland    }   & 0.033   & 0.066   & 0.096    & 0.128    & 0.18     & 0.296   & 100 & complete \\
\textsc{rfr\_x\_late\_op  } & \textsc{cpland    }   & 0.036   & 0.068   & 0.098    & 0.131    & 0.187    & 0.052   & 13  & moderate \\
\textsc{rfr\_x\_early\_all} & \textsc{cwed      }   & 1.328   & 1.584   & 1.753    & 1.907    & 2.11     & 0.964   & 1   & complete \\
\textsc{rfr\_x\_mid\_cl   } & \textsc{cwed      }   & 0.005   & 0.014   & 0.025    & 0.043    & 0.08     & 0.191   & 100 & complete \\
\textsc{rfr\_x\_mid\_mod  } & \textsc{cwed      }   & 0.085   & 0.161   & 0.22     & 0.289    & 0.394    & 0.455   & 99  & complete \\
\textsc{rfr\_x\_mid\_op   } & \textsc{cwed      }   & 0.425   & 0.566   & 0.664    & 0.75     & 0.922    & 0.352   & 1   & complete \\
\textsc{rfr\_x\_late\_cl  } & \textsc{cwed      }   & 0.15    & 0.228   & 0.305    & 0.402    & 0.59     & 0.281   & 42  & none     \\
\textsc{rfr\_x\_late\_mod } & \textsc{cwed      }   & 0.156   & 0.214   & 0.266    & 0.317    & 0.385    & 0.426   & 100 & complete \\
\textsc{rfr\_x\_late\_op  } & \textsc{cwed      }   & 0.138   & 0.215   & 0.285    & 0.356    & 0.482    & 0.109   & 1   & complete \\
\textsc{rfr\_x\_early\_all} & \textsc{econ\_am  }   & 32.308  & 34.616  & 36.04    & 37.483   & 39.254   & 29.03   & 1   & complete \\
\textsc{rfr\_x\_mid\_cl   } & \textsc{econ\_am  }   & 20.35   & 22.785  & 24.546   & 27.464   & 32.88    & 20.676  & 7   & moderate \\
\textsc{rfr\_x\_mid\_mod  } & \textsc{econ\_am  }   & 24.527  & 25.708  & 26.634   & 27.741   & 29.333   & 19.958  & 0   & complete \\
\textsc{rfr\_x\_mid\_op   } & \textsc{econ\_am  }   & 26.499  & 27.214  & 27.705   & 28.273   & 29.296   & 20.549  & 0   & complete \\
\textsc{rfr\_x\_late\_cl  } & \textsc{econ\_am  }   & 26.183  & 28.537  & 30.649   & 33.806   & 37.816   & 23.034  & 0   & complete \\
\textsc{rfr\_x\_late\_mod } & \textsc{econ\_am  }   & 27.08   & 28.93   & 30.349   & 31.627   & 33.496   & 22.064  & 0   & complete \\
\textsc{rfr\_x\_late\_op  } & \textsc{econ\_am  }   & 28.627  & 29.958  & 30.747   & 31.503   & 32.238   & 23.833  & 0   & complete \\
\textsc{rfr\_x\_early\_all} & \textsc{econ\_mn  }   & 32.762  & 35.511  & 36.965   & 38.553   & 40.474   & 28.267  & 0   & complete \\
\textsc{rfr\_x\_mid\_cl   } & \textsc{econ\_mn  }   & 21.708  & 24.297  & 26.311   & 28.733   & 34.379   & 21.113  & 4   & complete \\
\textsc{rfr\_x\_mid\_mod  } & \textsc{econ\_mn  }   & 26.009  & 27.187  & 27.955   & 28.97    & 30.597   & 20.74   & 0   & complete \\
\textsc{rfr\_x\_mid\_op   } & \textsc{econ\_mn  }   & 27.553  & 28.152  & 28.631   & 29.162   & 30.086   & 21.411  & 0   & complete \\
\textsc{rfr\_x\_late\_cl  } & \textsc{econ\_mn  }   & 28.741  & 30.499  & 32.406   & 35.104   & 38.956   & 22.652  & 0   & complete \\
\textsc{rfr\_x\_late\_mod } & \textsc{econ\_mn  }   & 29.442  & 31.282  & 32.503   & 33.891   & 35.393   & 22.565  & 0   & complete \\
\textsc{rfr\_x\_late\_op  } & \textsc{econ\_mn  }   & 29.996  & 31.178  & 31.92    & 32.72    & 33.53    & 26.171  & 0   & complete \\
\textsc{rfr\_x\_early\_all} & \textsc{ed        }   & 3.465   & 4.316   & 4.804    & 5.277    & 5.713    & 3.247   & 4   & complete \\
\textsc{rfr\_x\_mid\_cl   } & \textsc{ed        }   & 0.018   & 0.053   & 0.098    & 0.17     & 0.33     & 0.907   & 100 & complete \\
\textsc{rfr\_x\_mid\_mod  } & \textsc{ed        }   & 0.31    & 0.581   & 0.804    & 1.073    & 1.486    & 2.221   & 100 & complete \\
\textsc{rfr\_x\_mid\_op   } & \textsc{ed        }   & 1.521   & 2.016   & 2.342    & 2.656    & 3.208    & 1.678   & 10  & moderate \\
\textsc{rfr\_x\_late\_cl  } & \textsc{ed        }   & 0.435   & 0.667   & 0.944    & 1.33     & 2.056    & 1.221   & 69  & none     \\
\textsc{rfr\_x\_late\_mod } & \textsc{ed        }   & 0.477   & 0.699   & 0.872    & 1.033    & 1.238    & 1.88    & 100 & complete \\
\textsc{rfr\_x\_late\_op  } & \textsc{ed        }   & 0.448   & 0.703   & 0.917    & 1.13     & 1.496    & 0.436   & 5   & complete \\
\textsc{rfr\_x\_early\_all} & \textsc{gyrate\_am}   & 222.061 & 259.65  & 293.181  & 332.329  & 388.565  & 259.32  & 25  & moderate \\
\textsc{rfr\_x\_mid\_cl   } & \textsc{gyrate\_am}   & 50.261  & 65.802  & 88.343   & 118.417  & 197.46   & 186.667 & 95  & complete \\
\textsc{rfr\_x\_mid\_mod  } & \textsc{gyrate\_am}   & 88.62   & 122.327 & 146.836  & 181.421  & 253.373  & 194.415 & 80  & moderate \\
\textsc{rfr\_x\_mid\_op   } & \textsc{gyrate\_am}   & 153.601 & 177.409 & 205.347  & 240.433  & 298.962  & 137.65  & 0   & complete \\
\textsc{rfr\_x\_late\_cl  } & \textsc{gyrate\_am}   & 81.786  & 104.844 & 127.782  & 168.139  & 243.478  & 140.872 & 59  & none     \\
\textsc{rfr\_x\_late\_mod } & \textsc{gyrate\_am}   & 84.615  & 104.671 & 124.882  & 159.071  & 206.467  & 151.483 & 71  & none     \\
\textsc{rfr\_x\_late\_op  } & \textsc{gyrate\_am}   & 94.71   & 121.132 & 148.828  & 185.574  & 258.962  & 102.617 & 10  & moderate \\
\textsc{rfr\_x\_early\_all} & \textsc{iji       }   & 54.784  & 55.918  & 56.727   & 57.385   & 58.126   & 59.991  & 100 & complete \\
\textsc{rfr\_x\_mid\_cl   } & \textsc{iji       }   & 34.72   & 42.239  & 45.968   & 48.089   & 51.333   & 55.871  & 100 & complete \\
\textsc{rfr\_x\_mid\_mod  } & \textsc{iji       }   & 46.321  & 48.917  & 50.812   & 52.282   & 54.372   & 57.333  & 100 & complete \\
\textsc{rfr\_x\_mid\_op   } & \textsc{iji       }   & 48.169  & 50.405  & 52.519   & 54.17    & 55.733   & 55.806  & 96  & complete \\
\textsc{rfr\_x\_late\_cl  } & \textsc{iji       }   & 47.787  & 49.425  & 50.654   & 52.043   & 53.711   & 55.276  & 100 & complete \\
\textsc{rfr\_x\_late\_mod } & \textsc{iji       }   & 47.141  & 49.533  & 50.726   & 52.047   & 53.836   & 54.945  & 100 & complete \\
\textsc{rfr\_x\_late\_op  } & \textsc{iji       }   & 44.025  & 45.971  & 47.639   & 49.559   & 51.874   & 55.844  & 100 & complete \\
\textsc{rfr\_x\_early\_all} & \textsc{pd        }   & 0.29    & 0.344   & 0.37     & 0.399    & 0.418    & 0.297   & 7   & moderate \\
\textsc{rfr\_x\_mid\_cl   } & \textsc{pd        }   & 0.004   & 0.009   & 0.015    & 0.023    & 0.041    & 0.061   & 100 & complete \\
\textsc{rfr\_x\_mid\_mod  } & \textsc{pd        }   & 0.044   & 0.068   & 0.089    & 0.111    & 0.148    & 0.186   & 100 & complete \\
\textsc{rfr\_x\_mid\_op   } & \textsc{pd        }   & 0.143   & 0.187   & 0.212    & 0.234    & 0.267    & 0.154   & 9   & moderate \\
\textsc{rfr\_x\_late\_cl  } & \textsc{pd        }   & 0.091   & 0.12    & 0.147    & 0.181    & 0.226    & 0.101   & 14  & moderate \\
\textsc{rfr\_x\_late\_mod } & \textsc{pd        }   & 0.073   & 0.09    & 0.105    & 0.121    & 0.137    & 0.183   & 100 & complete \\
\textsc{rfr\_x\_late\_op  } & \textsc{pd        }   & 0.06    & 0.08    & 0.1      & 0.117    & 0.144    & 0.048   & 1   & complete \\
\textsc{rfr\_x\_early\_all} & \textsc{pland     }   & 1.323   & 1.766   & 1.99     & 2.202    & 2.477    & 1.343   & 6   & moderate \\
\textsc{rfr\_x\_mid\_cl   } & \textsc{pland     }   & 0.004   & 0.013   & 0.027    & 0.056    & 0.112    & 0.343   & 100 & complete \\
\textsc{rfr\_x\_mid\_mod  } & \textsc{pland     }   & 0.091   & 0.188   & 0.284    & 0.383    & 0.547    & 0.768   & 100 & complete \\
\textsc{rfr\_x\_mid\_op   } & \textsc{pland     }   & 0.582   & 0.768   & 0.896    & 1.047    & 1.305    & 0.513   & 3   & complete \\
\textsc{rfr\_x\_late\_cl  } & \textsc{pland     }   & 0.104   & 0.182   & 0.274    & 0.423    & 0.713    & 0.43    & 77  & moderate \\
\textsc{rfr\_x\_late\_mod } & \textsc{pland     }   & 0.131   & 0.208   & 0.273    & 0.34     & 0.429    & 0.604   & 100 & complete \\
\textsc{rfr\_x\_late\_op  } & \textsc{pland     }   & 0.132   & 0.223   & 0.309    & 0.399    & 0.541    & 0.128   & 5   & complete \\
\textsc{rfr\_x\_early\_all} & \textsc{shape\_am }   & 2.228   & 2.429   & 2.575    & 2.783    & 2.972    & 2.133   & 2   & complete \\
\textsc{rfr\_x\_mid\_cl   } & \textsc{shape\_am }   & 1.251   & 1.367   & 1.475    & 1.648    & 2.019    & 2.07    & 96  & complete \\
\textsc{rfr\_x\_mid\_mod  } & \textsc{shape\_am }   & 1.491   & 1.673   & 1.825    & 1.972    & 2.28     & 2.197   & 93  & moderate \\
\textsc{rfr\_x\_mid\_op   } & \textsc{shape\_am }   & 1.811   & 1.93    & 2.041    & 2.199    & 2.563    & 1.905   & 21  & moderate \\
\textsc{rfr\_x\_late\_cl  } & \textsc{shape\_am }   & 1.475   & 1.599   & 1.749    & 1.95     & 2.362    & 1.839   & 62  & none     \\
\textsc{rfr\_x\_late\_mod } & \textsc{shape\_am }   & 1.475   & 1.595   & 1.712    & 1.902    & 2.124    & 1.923   & 78  & moderate \\
\textsc{rfr\_x\_late\_op  } & \textsc{shape\_am }   & 1.533   & 1.673   & 1.844    & 2.025    & 2.369    & 1.653   & 21  & moderate \\
\textsc{rfr\_x\_early\_all} & \textsc{shape\_mn }   & 1.437   & 1.465   & 1.478    & 1.491    & 1.506    & 1.428   & 2   & complete \\
\textsc{rfr\_x\_mid\_cl   } & \textsc{shape\_mn }   & 1.144   & 1.214   & 1.261    & 1.316    & 1.391    & 1.595   & 100 & complete \\
\textsc{rfr\_x\_mid\_mod  } & \textsc{shape\_mn }   & 1.285   & 1.329   & 1.354    & 1.383    & 1.42     & 1.514   & 100 & complete \\
\textsc{rfr\_x\_mid\_op   } & \textsc{shape\_mn }   & 1.371   & 1.395   & 1.416    & 1.435    & 1.461    & 1.51    & 100 & complete \\
\textsc{rfr\_x\_late\_cl  } & \textsc{shape\_mn }   & 1.185   & 1.223   & 1.257    & 1.299    & 1.36     & 1.5     & 100 & complete \\
\textsc{rfr\_x\_late\_mod } & \textsc{shape\_mn }   & 1.243   & 1.295   & 1.317    & 1.348    & 1.385    & 1.446   & 100 & complete \\
\textsc{rfr\_x\_late\_op  } & \textsc{shape\_mn }   & 1.293   & 1.331   & 1.365    & 1.394    & 1.431    & 1.412   & 88  & moderate \\
\textsc{rfr\_x\_early\_all} & \textsc{simi\_mn  }   & 793.869 & 887.356 & 963.867  & 1076.966 & 1227.852 & 597.985 & 0   & complete \\
\textsc{rfr\_x\_mid\_cl   } & \textsc{simi\_mn  }   & 537.477 & 797.143 & 1046.845 & 1354.73  & 2072.974 & 466.925 & 3   & complete \\
\textsc{rfr\_x\_mid\_mod  } & \textsc{simi\_mn  }   & 762.781 & 911.128 & 1035.652 & 1189.188 & 1424.022 & 415.603 & 0   & complete \\
\textsc{rfr\_x\_mid\_op   } & \textsc{simi\_mn  }   & 847.187 & 937.351 & 1039.225 & 1154.195 & 1321.19  & 531.867 & 0   & complete \\
\textsc{rfr\_x\_late\_cl  } & \textsc{simi\_mn  }   & 706.635 & 863.937 & 1002.353 & 1187.668 & 1468.429 & 421.915 & 0   & complete \\
\textsc{rfr\_x\_late\_mod } & \textsc{simi\_mn  }   & 686.042 & 788.304 & 870.405  & 977.949  & 1163.576 & 444.247 & 0   & complete \\
\textsc{rfr\_x\_late\_op  } & \textsc{simi\_mn  }   & 671.611 & 791.293 & 857.295  & 922.837  & 1075.795 & 537.628 & 0   & complete \\
\textsc{rfr\_x\_early\_all} & \textsc{te  	    }   & 629130  & 783540  & 872280   & 957990   & 1037130  & 589410  & 4   & complete \\
\textsc{rfr\_x\_mid\_cl   } & \textsc{te        }   & 3270    & 9600    & 17880    & 30960    & 59910    & 164670  & 100 & complete \\
\textsc{rfr\_x\_mid\_mod  } & \textsc{te        }   & 56280   & 105480  & 145980   & 194880   & 269700   & 403290  & 100 & complete \\
\textsc{rfr\_x\_mid\_op   } & \textsc{te        }   & 276090  & 365940  & 425280   & 482130   & 582420   & 304680  & 10  & moderate \\
\textsc{rfr\_x\_late\_cl  } & \textsc{te        }   & 78900   & 121140  & 171450   & 241500   & 373260   & 221640  & 69  & none     \\
\textsc{rfr\_x\_late\_mod } & \textsc{te        }   & 86640   & 126840  & 158370   & 187590   & 224670   & 341250  & 100 & complete \\
\textsc{rfr\_x\_late\_op  } & \textsc{te        }   & 81420   & 127560  & 166530   & 205230   & 271650   & 79140   & 5   & complete

\end{longtable}
\end{footnotesize}
\end{center}
\end{landscape}

\pagestyle{headings}

%%%%%%%%%%%%%%%%%%%%%%%%%%%%%%%%%%%%%%%%%%%%%%%%%%%%%%%%%%%%%%%%%%%%%%%%%%%%%%%%%%%%%%%%%%%%%%%%
%%%%%%%%%%%%%%%%%%%%%%%%%%%%%%%%%%%%%%%%%%%%%%%%%%%%%%%%%%%%%%%%%%%%%%%%%%%%%%%%%%%%%%%%%%%%%%%%

\pagestyle{empty}
\begin{landscape}
\footnotesize
\begin{center}
\begin{footnotesize}
\begin{longtable}{llrrrrr|rrr}
\caption{Unabridged results for class-level metrics for Sierran Mixed Conifer - Mesic (\textsc{smc\_m}) calculated with \textsc{Fragstats}. This table shows the range of variability in landscape structure. Included are the $5^{\text{th}}$ percentile, $25^{\text{th}}$ percentile, $50^{\text{th}}$ percentile, $75^{\text{th}}$ percentile, and $95^{\text{th}}$ percentiles of the distribution, as well as the current class value, the current percentile range of variability (\%RV) for that proportion, and the departure classification. For seral stage abbreviations, see Table~\ref{condtable}.} \\
\label{tab:fragclass_smcm} \\

\hline 
\textbf{\begin{tabular}[c]{@{}l@{}}Cover-Seral Stage Type\end{tabular}}  &   
\textbf{\begin{tabular}[c]{@{}l@{}}Landscape\\ Metric\end{tabular}}  &   
\textbf{$5^{\text{th}}$ } &   
\textbf{$25^{\text{th}}$ } &   
\textbf{$50^{\text{th}}$ } &   
\textbf{$75^{\text{th}}$ } &   
\textbf{$95^{\text{th}}$ }  &  
\textbf{\begin{tabular}[c]{@{}l@{}}Current\\ Value\end{tabular}} &   
\textbf{\begin{tabular}[c]{@{}l@{}}Current\\ \%RV\end{tabular}} &   
\textbf{\begin{tabular}[c]{@{}l@{}}Departure\end{tabular}} \\  \\ \hline 
\endfirsthead

\multicolumn{10}{c}{{\bfseries \tablename\ \thetable{} -- continued from previous page}} \\
\hline 
\textbf{\begin{tabular}[c]{@{}l@{}}Cover-Seral Stage Type\end{tabular}}  &   
\textbf{\begin{tabular}[c]{@{}l@{}}Landscape\\ Metric\end{tabular}}  &   
\textbf{$5^{\text{th}}$ } &   
\textbf{$25^{\text{th}}$ } &   
\textbf{$50^{\text{th}}$ } &   
\textbf{$75^{\text{th}}$ } &   
\textbf{$95^{\text{th}}$ }  &  
\textbf{\begin{tabular}[c]{@{}l@{}}Current\\ Value\end{tabular}} &   
\textbf{\begin{tabular}[c]{@{}l@{}}Current\\ \%RV\end{tabular}} &   
\textbf{\begin{tabular}[c]{@{}l@{}}Departure\end{tabular}} \\  \\ \hline \endhead

\hline \multicolumn{10}{|l|}{{Continued on next page}} \\ \hline
\endfoot

\hline \hline
\endlastfoot

\textsc{smc\_m\_early\_all} & \textsc{ai        }   & 79.298   & 80.933   & 81.704   & 82.546   & 83.916   & 79.656   & 9   & moderate \\
\textsc{smc\_m\_mid\_cl   } & \textsc{ai        }   & 82.175   & 82.989   & 83.503   & 83.967   & 84.585   & 79.661   & 0   & complete \\
\textsc{smc\_m\_mid\_mod  } & \textsc{ai        }   & 71.437   & 73.924   & 75.664   & 77.051   & 79.391   & 80.133   & 99  & complete \\
\textsc{smc\_m\_mid\_op   } & \textsc{ai        }   & 76.568   & 78.098   & 79.019   & 80.052   & 81.806   & 80.403   & 82  & moderate \\
\textsc{smc\_m\_late\_cl  } & \textsc{ai        }   & 75.718   & 77.823   & 79.261   & 80.646   & 82.217   & 83.088   & 100 & complete \\
\textsc{smc\_m\_late\_mod } & \textsc{ai        }   & 73.629   & 74.489   & 75.331   & 76.683   & 78.674   & 81.421   & 100 & complete \\
\textsc{smc\_m\_late\_op  } & \textsc{ai        }   & 75.422   & 76.986   & 78.37    & 79.785   & 81.173   & 77.793   & 41  & none     \\
\textsc{smc\_m\_early\_all} & \textsc{area\_am  }   & 34.173   & 58.751   & 82.522   & 129.865  & 284.111  & 28.587   & 3   & complete \\
\textsc{smc\_m\_mid\_cl   } & \textsc{area\_am  }   & 83.715   & 118.32   & 157.31   & 217.611  & 336.249  & 24.753   & 0   & complete \\
\textsc{smc\_m\_mid\_mod  } & \textsc{area\_am  }   & 6.339    & 12.339   & 20.539   & 38.295   & 113.763  & 40.315   & 77  & moderate \\
\textsc{smc\_m\_mid\_op   } & \textsc{area\_am  }   & 15.031   & 26.286   & 39.578   & 58.067   & 141.207  & 61.287   & 77  & moderate \\
\textsc{smc\_m\_late\_cl  } & \textsc{area\_am  }   & 17.957   & 35.036   & 55.35    & 87.353   & 180.983  & 64.791   & 59  & none     \\
\textsc{smc\_m\_late\_mod } & \textsc{area\_am  }   & 8.141    & 11.294   & 15.152   & 26.394   & 52.181   & 53.663   & 96  & complete \\
\textsc{smc\_m\_late\_op  } & \textsc{area\_am  }   & 10.434   & 16.53    & 28.255   & 44.772   & 80.657   & 11.076   & 9   & moderate \\
\textsc{smc\_m\_early\_all} & \textsc{area\_mn  }   & 3.837    & 4.543    & 5.08     & 5.539    & 6.461    & 4.388    & 19  & moderate \\
\textsc{smc\_m\_mid\_cl   } & \textsc{area\_mn  }   & 5.419    & 6.159    & 6.55     & 6.934    & 7.692    & 4.454    & 0   & complete \\
\textsc{smc\_m\_mid\_mod  } & \textsc{area\_mn  }   & 1.626    & 2.032    & 2.359    & 2.667    & 3.206    & 5.593    & 100 & complete \\
\textsc{smc\_m\_mid\_op   } & \textsc{area\_mn  }   & 2.615    & 3.056    & 3.352    & 3.693    & 4.256    & 5.486    & 100 & complete \\
\textsc{smc\_m\_late\_cl  } & \textsc{area\_mn  }   & 2.622    & 3.227    & 3.845    & 4.447    & 5.243    & 6.71     & 100 & complete \\
\textsc{smc\_m\_late\_mod } & \textsc{area\_mn  }   & 2.283    & 2.554    & 2.743    & 3.003    & 3.487    & 5.937    & 100 & complete \\
\textsc{smc\_m\_late\_op  } & \textsc{area\_mn  }   & 2.577    & 3.055    & 3.453    & 3.881    & 4.522    & 3.5      & 53  & none     \\
\textsc{smc\_m\_early\_all} & \textsc{cai\_am   }   & 94.388   & 94.803   & 95.074   & 95.343   & 95.949   & 96.461   & 100 & complete \\
\textsc{smc\_m\_mid\_cl   } & \textsc{cai\_am   }   & 20.257   & 25.071   & 28.086   & 30.777   & 34.664   & 42.237   & 100 & complete \\
\textsc{smc\_m\_mid\_mod  } & \textsc{cai\_am   }   & 39.946   & 48.652   & 53.15    & 57.69    & 62.319   & 60.034   & 88  & moderate \\
\textsc{smc\_m\_mid\_op   } & \textsc{cai\_am   }   & 69.497   & 72.306   & 74.273   & 76.494   & 79.696   & 80.096   & 97  & complete \\
\textsc{smc\_m\_late\_cl  } & \textsc{cai\_am   }   & 17.189   & 24.238   & 28.12    & 31.963   & 36.482   & 43.109   & 100 & complete \\
\textsc{smc\_m\_late\_mod } & \textsc{cai\_am   }   & 29.303   & 33.09    & 36.691   & 40.946   & 46.159   & 57.853   & 100 & complete \\
\textsc{smc\_m\_late\_op  } & \textsc{cai\_am   }   & 37.941   & 41.435   & 44.754   & 47.485   & 51.788   & 57.995   & 100 & complete \\
\textsc{smc\_m\_early\_all} & \textsc{clumpy    }   & 0.782    & 0.795    & 0.803    & 0.81     & 0.825    & 0.786    & 10  & moderate \\
\textsc{smc\_m\_mid\_cl   } & \textsc{clumpy    }   & 0.801    & 0.806    & 0.812    & 0.817    & 0.822    & 0.79     & 0   & complete \\
\textsc{smc\_m\_mid\_mod  } & \textsc{clumpy    }   & 0.712    & 0.736    & 0.753    & 0.766    & 0.789    & 0.789    & 95  & complete \\
\textsc{smc\_m\_mid\_op   } & \textsc{clumpy    }   & 0.761    & 0.775    & 0.783    & 0.794    & 0.811    & 0.793    & 74  & none     \\
\textsc{smc\_m\_late\_cl  } & \textsc{clumpy    }   & 0.754    & 0.772    & 0.785    & 0.797    & 0.812    & 0.817    & 99  & complete \\
\textsc{smc\_m\_late\_mod } & \textsc{clumpy    }   & 0.731    & 0.739    & 0.747    & 0.76     & 0.781    & 0.805    & 100 & complete \\
\textsc{smc\_m\_late\_op  } & \textsc{clumpy    }   & 0.75     & 0.766    & 0.778    & 0.791    & 0.806    & 0.775    & 44  & none     \\
\textsc{smc\_m\_early\_all} & \textsc{core\_am  }   & 32.445   & 56.37    & 80.426   & 123.405  & 272.209  & 27.758   & 3   & complete \\
\textsc{smc\_m\_mid\_cl   } & \textsc{core\_am  }   & 31.499   & 52.158   & 72.983   & 111.169  & 172.681  & 13.032   & 0   & complete \\
\textsc{smc\_m\_mid\_mod  } & \textsc{core\_am  }   & 3.466    & 8.235    & 14.526   & 29.864   & 90.613   & 27.273   & 73  & none     \\
\textsc{smc\_m\_mid\_op   } & \textsc{core\_am  }   & 11.909   & 21.645   & 33.233   & 49.475   & 123.795  & 50.621   & 77  & moderate \\
\textsc{smc\_m\_late\_cl  } & \textsc{core\_am  }   & 6.082    & 15.068   & 26.48    & 46.624   & 93.601   & 35.675   & 64  & none     \\
\textsc{smc\_m\_late\_mod } & \textsc{core\_am  }   & 3.506    & 5.315    & 7.742    & 15.799   & 33.809   & 35.619   & 96  & complete \\
\textsc{smc\_m\_late\_op  } & \textsc{core\_am  }   & 5.316    & 9.864    & 17.164   & 27.878   & 51.218   & 7.329    & 15  & moderate \\
\textsc{smc\_m\_early\_all} & \textsc{core\_mn  }   & 3.652    & 4.32     & 4.806    & 5.263    & 6.115    & 4.232    & 21  & moderate \\
\textsc{smc\_m\_mid\_cl   } & \textsc{core\_mn  }   & 1.152    & 1.561    & 1.838    & 2.089    & 2.599    & 1.881    & 55  & none     \\
\textsc{smc\_m\_mid\_mod  } & \textsc{core\_mn  }   & 0.668    & 0.998    & 1.275    & 1.501    & 1.894    & 3.357    & 100 & complete \\
\textsc{smc\_m\_mid\_op   } & \textsc{core\_mn  }   & 1.889    & 2.234    & 2.495    & 2.749    & 3.299    & 4.394    & 100 & complete \\
\textsc{smc\_m\_late\_cl  } & \textsc{core\_mn  }   & 0.475    & 0.782    & 1.065    & 1.41     & 1.894    & 2.893    & 100 & complete \\
\textsc{smc\_m\_late\_mod } & \textsc{core\_mn  }   & 0.692    & 0.86     & 0.993    & 1.203    & 1.559    & 3.435    & 100 & complete \\
\textsc{smc\_m\_late\_op  } & \textsc{core\_mn  }   & 1.022    & 1.307    & 1.544    & 1.791    & 2.168    & 2.03     & 91  & moderate \\
\textsc{smc\_m\_early\_all} & \textsc{cpland    }   & 3.378    & 5.136    & 6.453    & 7.859    & 10.418   & 4.567    & 17  & moderate \\
\textsc{smc\_m\_mid\_cl   } & \textsc{cpland    }   & 1.895    & 2.734    & 3.446    & 4.031    & 5.016    & 1.311    & 1   & complete \\
\textsc{smc\_m\_mid\_mod  } & \textsc{cpland    }   & 0.329    & 0.572    & 0.79     & 1.059    & 1.539    & 3.443    & 100 & complete \\
\textsc{smc\_m\_mid\_op   } & \textsc{cpland    }   & 1.121    & 1.715    & 2.127    & 2.554    & 3.354    & 4.11     & 100 & complete \\
\textsc{smc\_m\_late\_cl  } & \textsc{cpland    }   & 0.252    & 0.542    & 0.921    & 1.42     & 2.227    & 3.217    & 100 & complete \\
\textsc{smc\_m\_late\_mod } & \textsc{cpland    }   & 0.505    & 0.678    & 0.848    & 1.064    & 1.529    & 2.641    & 100 & complete \\
\textsc{smc\_m\_late\_op  } & \textsc{cpland    }   & 0.471    & 0.747    & 1.06     & 1.406    & 1.957    & 0.658    & 18  & moderate \\
\textsc{smc\_m\_early\_all} & \textsc{cwed      }   & 3.057    & 4.347    & 5.327    & 6.357    & 8.026    & 3.67     & 13  & moderate \\
\textsc{smc\_m\_mid\_cl   } & \textsc{cwed      }   & 7.306    & 8.417    & 9.154    & 9.801    & 10.682   & 1.919    & 0   & complete \\
\textsc{smc\_m\_mid\_mod  } & \textsc{cwed      }   & 0.728    & 0.948    & 1.162    & 1.437    & 1.953    & 3.336    & 100 & complete \\
\textsc{smc\_m\_mid\_op   } & \textsc{cwed      }   & 1.073    & 1.531    & 1.855    & 2.265    & 2.909    & 2.89     & 95  & complete \\
\textsc{smc\_m\_late\_cl  } & \textsc{cwed      }   & 1.152    & 2.002    & 2.7      & 3.582    & 4.784    & 4.36     & 91  & moderate \\
\textsc{smc\_m\_late\_mod } & \textsc{cwed      }   & 1.522    & 1.851    & 2.083    & 2.354    & 2.864    & 2.795    & 94  & moderate \\
\textsc{smc\_m\_late\_op  } & \textsc{cwed      }   & 0.873    & 1.351    & 1.743    & 2.344    & 2.998    & 0.854    & 4   & complete \\
\textsc{smc\_m\_early\_all} & \textsc{econ\_am  }   & 30.578   & 31.643   & 32.435   & 33.115   & 34.525   & 28.562   & 0   & complete \\
\textsc{smc\_m\_mid\_cl   } & \textsc{econ\_am  }   & 31.207   & 32.642   & 33.727   & 34.85    & 36.548   & 22.152   & 0   & complete \\
\textsc{smc\_m\_mid\_mod  } & \textsc{econ\_am  }   & 21.916   & 22.965   & 23.867   & 24.666   & 26.087   & 21.259   & 2   & complete \\
\textsc{smc\_m\_mid\_op   } & \textsc{econ\_am  }   & 21.971   & 22.816   & 23.503   & 24.205   & 25.278   & 21.088   & 0   & complete \\
\textsc{smc\_m\_late\_cl  } & \textsc{econ\_am  }   & 26.419   & 27.612   & 28.582   & 29.614   & 31.782   & 25.742   & 2   & complete \\
\textsc{smc\_m\_late\_mod } & \textsc{econ\_am  }   & 24.495   & 25.732   & 26.755   & 27.604   & 29.333   & 24.583   & 7   & moderate \\
\textsc{smc\_m\_late\_op  } & \textsc{econ\_am  }   & 23.672   & 25.1     & 26.029   & 26.897   & 28.213   & 24.884   & 20  & moderate \\
\textsc{smc\_m\_early\_all} & \textsc{econ\_mn  }   & 30.137   & 30.942   & 31.627   & 32.382   & 33.484   & 27.489   & 0   & complete \\
\textsc{smc\_m\_mid\_cl   } & \textsc{econ\_mn  }   & 33.614   & 34.702   & 35.949   & 36.889   & 38.203   & 23.042   & 0   & complete \\
\textsc{smc\_m\_mid\_mod  } & \textsc{econ\_mn  }   & 24.927   & 25.828   & 26.497   & 27.194   & 28.323   & 21.733   & 0   & complete \\
\textsc{smc\_m\_mid\_op   } & \textsc{econ\_mn  }   & 22.769   & 23.397   & 23.812   & 24.367   & 25.163   & 21.533   & 0   & complete \\
\textsc{smc\_m\_late\_cl  } & \textsc{econ\_mn  }   & 31.091   & 32.245   & 33.222   & 34.493   & 37.459   & 25.293   & 0   & complete \\
\textsc{smc\_m\_late\_mod } & \textsc{econ\_mn  }   & 28.359   & 29.622   & 30.477   & 31.25    & 32.574   & 23.675   & 0   & complete \\
\textsc{smc\_m\_late\_op  } & \textsc{econ\_mn  }   & 25.929   & 26.656   & 27.277   & 27.939   & 28.687   & 24.838   & 0   & complete \\
\textsc{smc\_m\_early\_all} & \textsc{ed        }   & 9.55     & 13.266   & 16.448   & 20.175   & 25.638   & 12.896   & 23  & moderate \\
\textsc{smc\_m\_mid\_cl   } & \textsc{ed        }   & 20.169   & 24.13    & 26.949   & 29.029   & 31.958   & 8.506    & 0   & complete \\
\textsc{smc\_m\_mid\_mod  } & \textsc{ed        }   & 3        & 3.888    & 4.789    & 5.853    & 8.066    & 15.278   & 100 & complete \\
\textsc{smc\_m\_mid\_op   } & \textsc{ed        }   & 4.52     & 6.505    & 8.011    & 9.62     & 12.556   & 13.462   & 99  & complete \\
\textsc{smc\_m\_late\_cl  } & \textsc{ed        }   & 3.685    & 6.736    & 9.246    & 12.537   & 16.49    & 17.012   & 97  & complete \\
\textsc{smc\_m\_late\_mod } & \textsc{ed        }   & 5.559    & 6.701    & 7.675    & 8.674    & 10.576   & 11.444   & 99  & complete \\
\textsc{smc\_m\_late\_op  } & \textsc{ed        }   & 3.464    & 5.314    & 6.765    & 8.927    & 11.611   & 3.432    & 5   & complete \\
\textsc{smc\_m\_early\_all} & \textsc{gyrate\_am}   & 232.183  & 303      & 356.702  & 431.808  & 586.991  & 235.351  & 6   & moderate \\
\textsc{smc\_m\_mid\_cl   } & \textsc{gyrate\_am}   & 367.257  & 450.299  & 503.859  & 580.649  & 727.346  & 226.378  & 0   & complete \\
\textsc{smc\_m\_mid\_mod  } & \textsc{gyrate\_am}   & 101.053  & 138.715  & 173.452  & 216.607  & 363.092  & 285.145  & 90  & moderate \\
\textsc{smc\_m\_mid\_op   } & \textsc{gyrate\_am}   & 155.604  & 199.006  & 232.438  & 280.876  & 409.782  & 349.297  & 90  & moderate \\
\textsc{smc\_m\_late\_cl  } & \textsc{gyrate\_am}   & 173.698  & 237.135  & 299.935  & 373.237  & 505.218  & 375.073  & 76  & moderate \\
\textsc{smc\_m\_late\_mod } & \textsc{gyrate\_am}   & 120.495  & 140.749  & 161.13   & 193.808  & 271.475  & 333.055  & 99  & complete \\
\textsc{smc\_m\_late\_op  } & \textsc{gyrate\_am}   & 134.082  & 168.345  & 205.644  & 252.826  & 338.196  & 147.8    & 13  & moderate \\
\textsc{smc\_m\_early\_all} & \textsc{iji       }   & 55.242   & 56.696   & 57.714   & 58.805   & 60.502   & 56.964   & 31  & none     \\
\textsc{smc\_m\_mid\_cl   } & \textsc{iji       }   & 60.814   & 62.044   & 62.782   & 63.438   & 64.424   & 58.408   & 0   & complete \\
\textsc{smc\_m\_mid\_mod  } & \textsc{iji       }   & 52.22    & 54.576   & 55.848   & 57.215   & 59.179   & 58.347   & 89  & moderate \\
\textsc{smc\_m\_mid\_op   } & \textsc{iji       }   & 52.348   & 54.258   & 55.721   & 57.233   & 59.159   & 59.599   & 97  & complete \\
\textsc{smc\_m\_late\_cl  } & \textsc{iji       }   & 52.647   & 55.759   & 57.715   & 59.297   & 61.769   & 55.973   & 28  & none     \\
\textsc{smc\_m\_late\_mod } & \textsc{iji       }   & 53.226   & 55.631   & 57.105   & 58.181   & 59.464   & 58.223   & 77  & moderate \\
\textsc{smc\_m\_late\_op  } & \textsc{iji       }   & 51.135   & 52.752   & 53.957   & 55.166   & 56.76    & 60.641   & 100 & complete \\
\textsc{smc\_m\_early\_all} & \textsc{pd        }   & 0.832    & 1.083    & 1.34     & 1.576    & 1.946    & 1.079    & 24  & moderate \\
\textsc{smc\_m\_mid\_cl   } & \textsc{pd        }   & 1.508    & 1.721    & 1.851    & 1.965    & 2.106    & 0.697    & 0   & complete \\
\textsc{smc\_m\_mid\_mod  } & \textsc{pd        }   & 0.442    & 0.535    & 0.626    & 0.725    & 0.92     & 1.026    & 99  & complete \\
\textsc{smc\_m\_mid\_op   } & \textsc{pd        }   & 0.531    & 0.691    & 0.845    & 1.005    & 1.262    & 0.935    & 66  & none     \\
\textsc{smc\_m\_late\_cl  } & \textsc{pd        }   & 0.387    & 0.66     & 0.864    & 1.113    & 1.382    & 1.112    & 75  & moderate \\
\textsc{smc\_m\_late\_mod } & \textsc{pd        }   & 0.652    & 0.766    & 0.843    & 0.928    & 1.06     & 0.769    & 26  & none     \\
\textsc{smc\_m\_late\_op  } & \textsc{pd        }   & 0.371    & 0.54     & 0.684    & 0.85     & 1.074    & 0.324    & 3   & complete \\
\textsc{smc\_m\_early\_all} & \textsc{pland     }   & 3.534    & 5.416    & 6.766    & 8.267    & 10.96    & 4.734    & 16  & moderate \\
\textsc{smc\_m\_mid\_cl   } & \textsc{pland     }   & 8.838    & 10.817   & 12.221   & 13.336   & 14.859   & 3.103    & 0   & complete \\
\textsc{smc\_m\_mid\_mod  } & \textsc{pland     }   & 0.802    & 1.153    & 1.46     & 1.876    & 2.67     & 5.736    & 100 & complete \\
\textsc{smc\_m\_mid\_op   } & \textsc{pland     }   & 1.51     & 2.287    & 2.856    & 3.482    & 4.559    & 5.131    & 100 & complete \\
\textsc{smc\_m\_late\_cl  } & \textsc{pland     }   & 1.194    & 2.269    & 3.262    & 4.696    & 6.506    & 7.462    & 100 & complete \\
\textsc{smc\_m\_late\_mod } & \textsc{pland     }   & 1.597    & 2.02     & 2.337    & 2.714    & 3.5      & 4.565    & 100 & complete \\
\textsc{smc\_m\_late\_op  } & \textsc{pland     }   & 1.121    & 1.734    & 2.329    & 3.203    & 4.256    & 1.134    & 6   & moderate \\
\textsc{smc\_m\_early\_all} & \textsc{shape\_am }   & 2.229    & 2.538    & 2.783    & 3.107    & 3.765    & 2.295    & 7   & moderate \\
\textsc{smc\_m\_mid\_cl   } & \textsc{shape\_am }   & 2.842    & 3.229    & 3.513    & 3.774    & 4.408    & 2.186    & 0   & complete \\
\textsc{smc\_m\_mid\_mod  } & \textsc{shape\_am }   & 1.624    & 1.82     & 1.978    & 2.196    & 2.739    & 2.701    & 94  & moderate \\
\textsc{smc\_m\_mid\_op   } & \textsc{shape\_am }   & 1.873    & 2.043    & 2.2      & 2.402    & 2.808    & 2.995    & 97  & complete \\
\textsc{smc\_m\_late\_cl  } & \textsc{shape\_am }   & 2.096    & 2.398    & 2.698    & 3.054    & 3.695    & 2.84     & 61  & none     \\
\textsc{smc\_m\_late\_mod } & \textsc{shape\_am }   & 1.774    & 1.933    & 2.058    & 2.228    & 2.537    & 2.702    & 98  & complete \\
\textsc{smc\_m\_late\_op  } & \textsc{shape\_am }   & 1.797    & 2.008    & 2.177    & 2.37     & 2.689    & 1.781    & 4   & complete \\
\textsc{smc\_m\_early\_all} & \textsc{shape\_mn }   & 1.417    & 1.44     & 1.456    & 1.473    & 1.494    & 1.487    & 93  & moderate \\
\textsc{smc\_m\_mid\_cl   } & \textsc{shape\_mn }   & 1.455    & 1.474    & 1.49     & 1.502    & 1.518    & 1.506    & 83  & moderate \\
\textsc{smc\_m\_mid\_mod  } & \textsc{shape\_mn }   & 1.262    & 1.294    & 1.314    & 1.335    & 1.362    & 1.588    & 100 & complete \\
\textsc{smc\_m\_mid\_op   } & \textsc{shape\_mn }   & 1.326    & 1.347    & 1.361    & 1.376    & 1.395    & 1.557    & 100 & complete \\
\textsc{smc\_m\_late\_cl  } & \textsc{shape\_mn }   & 1.349    & 1.398    & 1.426    & 1.45     & 1.476    & 1.549    & 100 & complete \\
\textsc{smc\_m\_late\_mod } & \textsc{shape\_mn }   & 1.337    & 1.358    & 1.373    & 1.391    & 1.42     & 1.569    & 100 & complete \\
\textsc{smc\_m\_late\_op  } & \textsc{shape\_mn }   & 1.337    & 1.371    & 1.388    & 1.41     & 1.435    & 1.476    & 100 & complete \\
\textsc{smc\_m\_early\_all} & \textsc{simi\_mn  }   & 3141.482 & 3536.523 & 3805.807 & 4087.435 & 4473.53  & 2306.563 & 0   & complete \\
\textsc{smc\_m\_mid\_cl   } & \textsc{simi\_mn  }   & 2086.884 & 2217.507 & 2319.176 & 2416.893 & 2559.776 & 2036.063 & 2   & complete \\
\textsc{smc\_m\_mid\_mod  } & \textsc{simi\_mn  }   & 2092.872 & 2289.425 & 2449.622 & 2659.203 & 3010.308 & 2042.245 & 4   & complete \\
\textsc{smc\_m\_mid\_op   } & \textsc{simi\_mn  }   & 2306.673 & 2573.109 & 2760.159 & 2981.778 & 3390.646 & 2065.211 & 0   & complete \\
\textsc{smc\_m\_late\_cl  } & \textsc{simi\_mn  }   & 1762.392 & 1948.249 & 2100.299 & 2310.14  & 2810.149 & 1675.049 & 2   & complete \\
\textsc{smc\_m\_late\_mod } & \textsc{simi\_mn  }   & 1922.087 & 2060.249 & 2177.855 & 2312.647 & 2545.558 & 2094.397 & 33  & none     \\
\textsc{smc\_m\_late\_op  } & \textsc{simi\_mn  }   & 2175.489 & 2438.887 & 2631.499 & 2784.548 & 3008.115 & 3085.473 & 98  & complete \\
\textsc{smc\_m\_early\_all} & \textsc{te  	    }   & 1733790  & 2408460  & 2986230  & 3662850  & 4654590  & 2341320  & 23  & moderate \\
\textsc{smc\_m\_mid\_cl   } & \textsc{te        }   & 3661770  & 4380840  & 4892670  & 5270370  & 5802090  & 1544220  & 0   & complete \\
\textsc{smc\_m\_mid\_mod  } & \textsc{te        }   & 544680   & 705840   & 869400   & 1062630  & 1464390  & 2773800  & 100 & complete \\
\textsc{smc\_m\_mid\_op   } & \textsc{te        }   & 820590   & 1180920  & 1454520  & 1746510  & 2279580  & 2444070  & 99  & complete \\
\textsc{smc\_m\_late\_cl  } & \textsc{te        }   & 669060   & 1222950  & 1678650  & 2276190  & 2993850  & 3088590  & 97  & complete \\
\textsc{smc\_m\_late\_mod } & \textsc{te        }   & 1009230  & 1216560  & 1393440  & 1574850  & 1920180  & 2077740  & 99  & complete \\
\textsc{smc\_m\_late\_op  } & \textsc{te        }   & 628950   & 964770   & 1228140  & 1620690  & 2108040  & 623040   & 5   & complete

\end{longtable}
\end{footnotesize}
\end{center}
\end{landscape}

\pagestyle{headings}


%%%%%%%%%%%%%%%%%%%%%%%%%%%%%%%%%%%%%%%%%%%%%%%%%%%%%%%%%%%%%%%%%%%%%%%%%%%%%%%%%%%%%%%%%%%%%%%%
%%%%%%%%%%%%%%%%%%%%%%%%%%%%%%%%%%%%%%%%%%%%%%%%%%%%%%%%%%%%%%%%%%%%%%%%%%%%%%%%%%%%%%%%%%%%%%%%

\pagestyle{empty}
\begin{landscape}
\footnotesize
\begin{center}
\begin{footnotesize}
\begin{longtable}{llrrrrr|rrr}
\caption{Unabridged results for class-level metrics for Sierran Mixed Conifer - Ultramafic (\textsc{smc\_u}) calculated with \textsc{Fragstats}. This table shows the range of variability in landscape structure. Included are the $5^{\text{th}}$ percentile, $25^{\text{th}}$ percentile, $50^{\text{th}}$ percentile, $75^{\text{th}}$ percentile, and $95^{\text{th}}$ percentiles of the distribution, as well as the current class value, the current percentile range of variability (\%RV) for that proportion, and the departure classification. For seral stage abbreviations, see Table~\ref{condtable}.} \\
\label{tab:fragclass_smcu} \\

\hline 
\textbf{\begin{tabular}[c]{@{}l@{}}Cover-Seral Stage Type\end{tabular}}  &   
\textbf{\begin{tabular}[c]{@{}l@{}}Landscape\\ Metric\end{tabular}}  &   
\textbf{$5^{\text{th}}$ } &   
\textbf{$25^{\text{th}}$ } &   
\textbf{$50^{\text{th}}$ } &   
\textbf{$75^{\text{th}}$ } &   
\textbf{$95^{\text{th}}$ }  &  
\textbf{\begin{tabular}[c]{@{}l@{}}Current\\ Value\end{tabular}} &   
\textbf{\begin{tabular}[c]{@{}l@{}}Current\\ \%RV\end{tabular}} &   
\textbf{\begin{tabular}[c]{@{}l@{}}Departure\end{tabular}} \\  \\ \hline 
\endfirsthead

\multicolumn{10}{c}{{\bfseries \tablename\ \thetable{} -- continued from previous page}} \\
\hline 
\textbf{\begin{tabular}[c]{@{}l@{}}Cover-Seral Stage Type\end{tabular}}  &   
\textbf{\begin{tabular}[c]{@{}l@{}}Landscape\\ Metric\end{tabular}}  &   
\textbf{$5^{\text{th}}$ } &   
\textbf{$25^{\text{th}}$ } &   
\textbf{$50^{\text{th}}$ } &   
\textbf{$75^{\text{th}}$ } &   
\textbf{$95^{\text{th}}$ }  &  
\textbf{\begin{tabular}[c]{@{}l@{}}Current\\ Value\end{tabular}} &   
\textbf{\begin{tabular}[c]{@{}l@{}}Current\\ \%RV\end{tabular}} &   
\textbf{\begin{tabular}[c]{@{}l@{}}Departure\end{tabular}} \\  \\ \hline \endhead

\hline \multicolumn{10}{|l|}{{Continued on next page}} \\ \hline
\endfoot

\hline \hline
\endlastfoot

\textsc{smc\_u\_early\_all} & \textsc{ai        }   & 81.186   & 82.555   & 83.355   & 83.954   & 85.007   & 90.77    & 100 & complete \\
\textsc{smc\_u\_mid\_cl   } & \textsc{ai        }   & 72.124   & 74.619   & 76.8     & 78.853   & 82.176   & 82.364   & 96  & complete \\
\textsc{smc\_u\_mid\_mod  } & \textsc{ai        }   & 75.168   & 76.761   & 78.205   & 80.397   & 83.316   & 84.429   & 99  & complete \\
\textsc{smc\_u\_mid\_op   } & \textsc{ai        }   & 80.166   & 81.484   & 82.496   & 83.296   & 84.996   & 80.079   & 5   & complete \\
\textsc{smc\_u\_late\_cl  } & \textsc{ai        }   & 76.856   & 78.232   & 80.144   & 81.595   & 83.997   & 88.154   & 100 & complete \\
\textsc{smc\_u\_late\_mod } & \textsc{ai        }   & 73.673   & 75.422   & 76.635   & 78.092   & 80.337   & 85.809   & 100 & complete \\
\textsc{smc\_u\_late\_op  } & \textsc{ai        }   & 74.95    & 76.592   & 77.647   & 78.735   & 83.265   & 83.304   & 96  & complete \\
\textsc{smc\_u\_early\_all} & \textsc{area\_am  }   & 23.563   & 30.699   & 37.853   & 50.947   & 63.073   & 178.043  & 100 & complete \\
\textsc{smc\_u\_mid\_cl   } & \textsc{area\_am  }   & 1.843    & 2.565    & 3.604    & 6.566    & 12.713   & 9.921    & 89  & moderate \\
\textsc{smc\_u\_mid\_mod  } & \textsc{area\_am  }   & 3.967    & 5.461    & 8.059    & 13.117   & 30.272   & 15.779   & 81  & moderate \\
\textsc{smc\_u\_mid\_op   } & \textsc{area\_am  }   & 12.239   & 18.369   & 22.238   & 28.451   & 47.915   & 8.289    & 0   & complete \\
\textsc{smc\_u\_late\_cl  } & \textsc{area\_am  }   & 5.175    & 8.22     & 12.065   & 19.225   & 41.683   & 87.059   & 98  & complete \\
\textsc{smc\_u\_late\_mod } & \textsc{area\_am  }   & 3.173    & 4.461    & 6.333    & 8.419    & 18.676   & 28.024   & 98  & complete \\
\textsc{smc\_u\_late\_op  } & \textsc{area\_am  }   & 3.426    & 4.964    & 6.349    & 9.247    & 24.953   & 8.763    & 74  & none     \\
\textsc{smc\_u\_early\_all} & \textsc{area\_mn  }   & 5.518    & 6.581    & 7.287    & 8.06     & 9.056    & 18.228   & 100 & complete \\
\textsc{smc\_u\_mid\_cl   } & \textsc{area\_mn  }   & 1.426    & 1.763    & 2.148    & 2.736    & 3.694    & 4.375    & 100 & complete \\
\textsc{smc\_u\_mid\_mod  } & \textsc{area\_mn  }   & 2.262    & 2.71     & 3.154    & 3.903    & 4.988    & 5.782    & 99  & complete \\
\textsc{smc\_u\_mid\_op   } & \textsc{area\_mn  }   & 4.341    & 5.024    & 5.568    & 6.125    & 7.297    & 4.498    & 7   & moderate \\
\textsc{smc\_u\_late\_cl  } & \textsc{area\_mn  }   & 2.71     & 3.274    & 4.016    & 4.717    & 6.35     & 14.89    & 100 & complete \\
\textsc{smc\_u\_late\_mod } & \textsc{area\_mn  }   & 2.052    & 2.441    & 2.771    & 3.194    & 3.898    & 7.672    & 100 & complete \\
\textsc{smc\_u\_late\_op  } & \textsc{area\_mn  }   & 2.207    & 2.628    & 2.962    & 3.264    & 5.066    & 4.419    & 93  & moderate \\
\textsc{smc\_u\_early\_all} & \textsc{cai\_am   }   & 93.431   & 93.826   & 94.211   & 94.594   & 95.282   & 90.974   & 0   & complete \\
\textsc{smc\_u\_mid\_cl   } & \textsc{cai\_am   }   & 0.99     & 4.749    & 9.191    & 15.132   & 28.335   & 36.37    & 100 & complete \\
\textsc{smc\_u\_mid\_mod  } & \textsc{cai\_am   }   & 25.33    & 33.856   & 40.629   & 46.424   & 56.885   & 73.184   & 100 & complete \\
\textsc{smc\_u\_mid\_op   } & \textsc{cai\_am   }   & 60.715   & 65.409   & 68.982   & 72.015   & 75.673   & 84.974   & 100 & complete \\
\textsc{smc\_u\_late\_cl  } & \textsc{cai\_am   }   & 3.615    & 7.475    & 10.615   & 13.969   & 27.773   & 39.893   & 100 & complete \\
\textsc{smc\_u\_late\_mod } & \textsc{cai\_am   }   & 19.901   & 26.724   & 31.824   & 37.739   & 46.977   & 60.877   & 100 & complete \\
\textsc{smc\_u\_late\_op  } & \textsc{cai\_am   }   & 22.89    & 29.004   & 34.545   & 40.216   & 55.135   & 73.931   & 100 & complete \\
\textsc{smc\_u\_early\_all} & \textsc{clumpy    }   & 0.81     & 0.824    & 0.832    & 0.838    & 0.848    & 0.907    & 100 & complete \\
\textsc{smc\_u\_mid\_cl   } & \textsc{clumpy    }   & 0.721    & 0.746    & 0.768    & 0.788    & 0.822    & 0.824    & 96  & complete \\
\textsc{smc\_u\_mid\_mod  } & \textsc{clumpy    }   & 0.752    & 0.767    & 0.782    & 0.804    & 0.833    & 0.844    & 99  & complete \\
\textsc{smc\_u\_mid\_op   } & \textsc{clumpy    }   & 0.8      & 0.814    & 0.824    & 0.832    & 0.849    & 0.8      & 5   & complete \\
\textsc{smc\_u\_late\_cl  } & \textsc{clumpy    }   & 0.768    & 0.782    & 0.801    & 0.816    & 0.839    & 0.881    & 100 & complete \\
\textsc{smc\_u\_late\_mod } & \textsc{clumpy    }   & 0.736    & 0.754    & 0.766    & 0.78     & 0.803    & 0.858    & 100 & complete \\
\textsc{smc\_u\_late\_op  } & \textsc{clumpy    }   & 0.749    & 0.766    & 0.776    & 0.787    & 0.832    & 0.833    & 96  & complete \\
\textsc{smc\_u\_early\_all} & \textsc{core\_am  }   & 22.458   & 29.426   & 36.492   & 48.723   & 60.827   & 159.264  & 100 & complete \\
\textsc{smc\_u\_mid\_cl   } & \textsc{core\_am  }   & 0.022    & 0.12     & 0.414    & 1.177    & 4.8      & 3.615    & 94  & moderate \\
\textsc{smc\_u\_mid\_mod  } & \textsc{core\_am  }   & 1.391    & 2.314    & 4.003    & 7.277    & 21.528   & 12.118   & 86  & moderate \\
\textsc{smc\_u\_mid\_op   } & \textsc{core\_am  }   & 8.501    & 13.16    & 16.514   & 23.255   & 42.093   & 7.34     & 3   & complete \\
\textsc{smc\_u\_late\_cl  } & \textsc{core\_am  }   & 0.288    & 0.967    & 2.031    & 4.188    & 17.222   & 46.47    & 99  & complete \\
\textsc{smc\_u\_late\_mod } & \textsc{core\_am  }   & 0.913    & 1.569    & 2.462    & 4.115    & 11.394   & 17.985   & 98  & complete \\
\textsc{smc\_u\_late\_op  } & \textsc{core\_am  }   & 1.056    & 1.839    & 2.737    & 4.519    & 16.805   & 7.282    & 89  & moderate \\
\textsc{smc\_u\_early\_all} & \textsc{core\_mn  }   & 5.163    & 6.186    & 6.87     & 7.606    & 8.547    & 16.583   & 100 & complete \\
\textsc{smc\_u\_mid\_cl   } & \textsc{core\_mn  }   & 0.018    & 0.09     & 0.198    & 0.375    & 0.881    & 1.591    & 100 & complete \\
\textsc{smc\_u\_mid\_mod  } & \textsc{core\_mn  }   & 0.631    & 0.938    & 1.276    & 1.782    & 2.679    & 4.232    & 100 & complete \\
\textsc{smc\_u\_mid\_op   } & \textsc{core\_mn  }   & 2.841    & 3.307    & 3.834    & 4.327    & 5.445    & 3.822    & 50  & none     \\
\textsc{smc\_u\_late\_cl  } & \textsc{core\_mn  }   & 0.102    & 0.251    & 0.423    & 0.639    & 1.719    & 5.94     & 100 & complete \\
\textsc{smc\_u\_late\_mod } & \textsc{core\_mn  }   & 0.427    & 0.675    & 0.885    & 1.191    & 1.756    & 4.67     & 100 & complete \\
\textsc{smc\_u\_late\_op  } & \textsc{core\_mn  }   & 0.537    & 0.771    & 1.016    & 1.293    & 2.797    & 3.267    & 97  & complete \\
\textsc{smc\_u\_early\_all} & \textsc{cpland    }   & 0.616    & 0.82     & 0.926    & 1.005    & 1.132    & 1.005    & 75  & moderate \\
\textsc{smc\_u\_mid\_cl   } & \textsc{cpland    }   & 0        & 0.001    & 0.002    & 0.005    & 0.013    & 0.024    & 98  & complete \\
\textsc{smc\_u\_mid\_mod  } & \textsc{cpland    }   & 0.019    & 0.033    & 0.052    & 0.074    & 0.123    & 0.112    & 94  & moderate \\
\textsc{smc\_u\_mid\_op   } & \textsc{cpland    }   & 0.288    & 0.34     & 0.381    & 0.435    & 0.506    & 0.103    & 0   & complete \\
\textsc{smc\_u\_late\_cl  } & \textsc{cpland    }   & 0.006    & 0.016    & 0.029    & 0.047    & 0.121    & 0.222    & 100 & complete \\
\textsc{smc\_u\_late\_mod } & \textsc{cpland    }   & 0.018    & 0.032    & 0.045    & 0.064    & 0.094    & 0.118    & 98  & complete \\
\textsc{smc\_u\_late\_op  } & \textsc{cpland    }   & 0.018    & 0.033    & 0.044    & 0.058    & 0.112    & 0.054    & 70  & none     \\
\textsc{smc\_u\_early\_all} & \textsc{cwed      }   & 0.567    & 0.755    & 0.815    & 0.877    & 0.931    & 0.621    & 9   & moderate \\
\textsc{smc\_u\_mid\_cl   } & \textsc{cwed      }   & 0.013    & 0.019    & 0.026    & 0.036    & 0.054    & 0.044    & 86  & moderate \\
\textsc{smc\_u\_mid\_mod  } & \textsc{cwed      }   & 0.064    & 0.086    & 0.103    & 0.124    & 0.166    & 0.066    & 7   & moderate \\
\textsc{smc\_u\_mid\_op   } & \textsc{cwed      }   & 0.274    & 0.341    & 0.366    & 0.395    & 0.438    & 0.084    & 0   & complete \\
\textsc{smc\_u\_late\_cl  } & \textsc{cwed      }   & 0.185    & 0.247    & 0.292    & 0.344    & 0.423    & 0.32     & 67  & none     \\
\textsc{smc\_u\_late\_mod } & \textsc{cwed      }   & 0.092    & 0.117    & 0.134    & 0.156    & 0.195    & 0.098    & 9   & moderate \\
\textsc{smc\_u\_late\_op  } & \textsc{cwed      }   & 0.062    & 0.085    & 0.107    & 0.127    & 0.18     & 0.045    & 1   & complete \\
\textsc{smc\_u\_early\_all} & \textsc{econ\_am  }   & 33.186   & 34.511   & 35.683   & 36.791   & 38.683   & 43.39    & 100 & complete \\
\textsc{smc\_u\_mid\_cl   } & \textsc{econ\_am  }   & 24.953   & 28.163   & 30.851   & 33.489   & 37.397   & 25.371   & 7   & moderate \\
\textsc{smc\_u\_mid\_mod  } & \textsc{econ\_am  }   & 23.814   & 25.496   & 26.849   & 28.264   & 29.921   & 18.358   & 0   & complete \\
\textsc{smc\_u\_mid\_op   } & \textsc{econ\_am  }   & 25.081   & 25.963   & 26.637   & 27.335   & 28.304   & 23.735   & 0   & complete \\
\textsc{smc\_u\_late\_cl  } & \textsc{econ\_am  }   & 32.879   & 37.184   & 38.937   & 40.481   & 43.264   & 31.662   & 3   & complete \\
\textsc{smc\_u\_late\_mod } & \textsc{econ\_am  }   & 25.311   & 27.703   & 29.165   & 30.848   & 33.33    & 23.954   & 1   & complete \\
\textsc{smc\_u\_late\_op  } & \textsc{econ\_am  }   & 23.868   & 25.144   & 25.942   & 26.948   & 28.344   & 22.941   & 1   & complete \\
\textsc{smc\_u\_early\_all} & \textsc{econ\_mn  }   & 34.667   & 36.042   & 36.846   & 37.655   & 39.132   & 39.912   & 99  & complete \\
\textsc{smc\_u\_mid\_cl   } & \textsc{econ\_mn  }   & 26.279   & 29.603   & 31.645   & 33.882   & 37.35    & 23.118   & 1   & complete \\
\textsc{smc\_u\_mid\_mod  } & \textsc{econ\_mn  }   & 24.942   & 26.354   & 27.416   & 28.38    & 29.883   & 19.567   & 0   & complete \\
\textsc{smc\_u\_mid\_op   } & \textsc{econ\_mn  }   & 25.729   & 26.374   & 26.898   & 27.529   & 28.686   & 24.842   & 0   & complete \\
\textsc{smc\_u\_late\_cl  } & \textsc{econ\_mn  }   & 35.051   & 38.204   & 39.567   & 40.801   & 43.315   & 37.613   & 20  & moderate \\
\textsc{smc\_u\_late\_mod } & \textsc{econ\_mn  }   & 26.849   & 28.485   & 29.806   & 31.166   & 33.335   & 24.58    & 1   & complete \\
\textsc{smc\_u\_late\_op  } & \textsc{econ\_mn  }   & 24.428   & 25.744   & 26.791   & 27.642   & 29.162   & 26.772   & 50  & none     \\
\textsc{smc\_u\_early\_all} & \textsc{ed        }   & 1.509    & 2.056    & 2.239    & 2.452    & 2.643    & 1.444    & 3   & complete \\
\textsc{smc\_u\_mid\_cl   } & \textsc{ed        }   & 0.035    & 0.06     & 0.084    & 0.121    & 0.178    & 0.179    & 96  & complete \\
\textsc{smc\_u\_mid\_mod  } & \textsc{ed        }   & 0.236    & 0.319    & 0.382    & 0.456    & 0.624    & 0.333    & 29  & none     \\
\textsc{smc\_u\_mid\_op   } & \textsc{ed        }   & 1.057    & 1.268    & 1.352    & 1.468    & 1.641    & 0.339    & 0   & complete \\
\textsc{smc\_u\_late\_cl  } & \textsc{ed        }   & 0.452    & 0.63     & 0.755    & 0.905    & 1.13     & 0.943    & 81  & moderate \\
\textsc{smc\_u\_late\_mod } & \textsc{ed        }   & 0.308    & 0.394    & 0.464    & 0.539    & 0.684    & 0.395    & 26  & none     \\
\textsc{smc\_u\_late\_op  } & \textsc{ed        }   & 0.231    & 0.335    & 0.413    & 0.488    & 0.676    & 0.184    & 2   & complete \\
\textsc{smc\_u\_early\_all} & \textsc{gyrate\_am}   & 220.699  & 248.711  & 278.103  & 315.922  & 365.359  & 640.292  & 100 & complete \\
\textsc{smc\_u\_mid\_cl   } & \textsc{gyrate\_am}   & 57.442   & 67.105   & 81.838   & 106.909  & 144.433  & 132.35   & 90  & moderate \\
\textsc{smc\_u\_mid\_mod  } & \textsc{gyrate\_am}   & 83.518   & 99.496   & 118.926  & 151.114  & 226.478  & 170.304  & 84  & moderate \\
\textsc{smc\_u\_mid\_op   } & \textsc{gyrate\_am}   & 150.966  & 180.606  & 198.396  & 225.322  & 278.027  & 135.764  & 1   & complete \\
\textsc{smc\_u\_late\_cl  } & \textsc{gyrate\_am}   & 96.563   & 120.305  & 146.321  & 182.948  & 282.379  & 454.655  & 100 & complete \\
\textsc{smc\_u\_late\_mod } & \textsc{gyrate\_am}   & 75.607   & 90.138   & 105.768  & 122.47   & 177.76   & 223.005  & 98  & complete \\
\textsc{smc\_u\_late\_op  } & \textsc{gyrate\_am}   & 79.302   & 95.9     & 107.79   & 124.127  & 215.737  & 125.479  & 77  & moderate \\
\textsc{smc\_u\_early\_all} & \textsc{iji       }   & 58.879   & 60.142   & 61.241   & 62.013   & 63.28    & 70.354   & 100 & complete \\
\textsc{smc\_u\_mid\_cl   } & \textsc{iji       }   & 40.41    & 45.293   & 48.955   & 52.201   & 55.001   & 57.596   & 100 & complete \\
\textsc{smc\_u\_mid\_mod  } & \textsc{iji       }   & 44.488   & 47.413   & 50.138   & 52.547   & 55.182   & 55.788   & 97  & complete \\
\textsc{smc\_u\_mid\_op   } & \textsc{iji       }   & 48.105   & 51.41    & 53.918   & 56.255   & 59.287   & 60.512   & 100 & complete \\
\textsc{smc\_u\_late\_cl  } & \textsc{iji       }   & 46.083   & 49.2     & 51.524   & 53.66    & 57.07    & 64.339   & 100 & complete \\
\textsc{smc\_u\_late\_mod } & \textsc{iji       }   & 45.233   & 47.976   & 49.671   & 51.132   & 53.313   & 57.938   & 100 & complete \\
\textsc{smc\_u\_late\_op  } & \textsc{iji       }   & 43.564   & 46.445   & 48.589   & 50.809   & 53.417   & 51.73    & 88  & moderate \\
\textsc{smc\_u\_early\_all} & \textsc{pd        }   & 0.104    & 0.126    & 0.135    & 0.14     & 0.151    & 0.061    & 0   & complete \\
\textsc{smc\_u\_mid\_cl   } & \textsc{pd        }   & 0.005    & 0.008    & 0.011    & 0.014    & 0.02     & 0.015    & 80  & moderate \\
\textsc{smc\_u\_mid\_mod  } & \textsc{pd        }   & 0.026    & 0.034    & 0.039    & 0.046    & 0.058    & 0.026    & 6   & moderate \\
\textsc{smc\_u\_mid\_op   } & \textsc{pd        }   & 0.078    & 0.095    & 0.102    & 0.11     & 0.12     & 0.027    & 0   & complete \\
\textsc{smc\_u\_late\_cl  } & \textsc{pd        }   & 0.048    & 0.061    & 0.068    & 0.077    & 0.087    & 0.038    & 2   & complete \\
\textsc{smc\_u\_late\_mod } & \textsc{pd        }   & 0.038    & 0.046    & 0.051    & 0.056    & 0.066    & 0.025    & 0   & complete \\
\textsc{smc\_u\_late\_op  } & \textsc{pd        }   & 0.026    & 0.036    & 0.044    & 0.052    & 0.061    & 0.016    & 0   & complete \\
\textsc{smc\_u\_early\_all} & \textsc{pland     }   & 0.646    & 0.869    & 0.985    & 1.067    & 1.2      & 1.104    & 83  & moderate \\
\textsc{smc\_u\_mid\_cl   } & \textsc{pland     }   & 0.009    & 0.016    & 0.023    & 0.038    & 0.057    & 0.068    & 98  & complete \\
\textsc{smc\_u\_mid\_mod  } & \textsc{pland     }   & 0.069    & 0.101    & 0.127    & 0.159    & 0.225    & 0.153    & 73  & none     \\
\textsc{smc\_u\_mid\_op   } & \textsc{pland     }   & 0.421    & 0.506    & 0.564    & 0.627    & 0.716    & 0.121    & 0   & complete \\
\textsc{smc\_u\_late\_cl  } & \textsc{pland     }   & 0.143    & 0.213    & 0.273    & 0.35     & 0.481    & 0.558    & 99  & complete \\
\textsc{smc\_u\_late\_mod } & \textsc{pland     }   & 0.087    & 0.115    & 0.142    & 0.173    & 0.226    & 0.194    & 85  & moderate \\
\textsc{smc\_u\_late\_op  } & \textsc{pland     }   & 0.065    & 0.104    & 0.132    & 0.159    & 0.238    & 0.073    & 7   & moderate \\
\textsc{smc\_u\_early\_all} & \textsc{shape\_am }   & 2.08     & 2.235    & 2.371    & 2.55     & 2.832    & 2.741    & 89  & moderate \\
\textsc{smc\_u\_mid\_cl   } & \textsc{shape\_am }   & 1.273    & 1.342    & 1.413    & 1.559    & 1.834    & 1.576    & 77  & moderate \\
\textsc{smc\_u\_mid\_mod  } & \textsc{shape\_am }   & 1.421    & 1.536    & 1.654    & 1.795    & 2.115    & 1.724    & 64  & none     \\
\textsc{smc\_u\_mid\_op   } & \textsc{shape\_am }   & 1.755    & 1.886    & 1.962    & 2.083    & 2.307    & 1.681    & 1   & complete \\
\textsc{smc\_u\_late\_cl  } & \textsc{shape\_am }   & 1.495    & 1.627    & 1.774    & 1.944    & 2.284    & 2.695    & 100 & complete \\
\textsc{smc\_u\_late\_mod } & \textsc{shape\_am }   & 1.401    & 1.489    & 1.574    & 1.694    & 1.944    & 2.004    & 97  & complete \\
\textsc{smc\_u\_late\_op  } & \textsc{shape\_am }   & 1.401    & 1.496    & 1.558    & 1.665    & 2.111    & 1.444    & 11  & moderate \\
\textsc{smc\_u\_early\_all} & \textsc{shape\_mn }   & 1.505    & 1.547    & 1.577    & 1.612    & 1.651    & 1.578    & 51  & none     \\
\textsc{smc\_u\_mid\_cl   } & \textsc{shape\_mn }   & 1.218    & 1.266    & 1.301    & 1.334    & 1.41     & 1.451    & 100 & complete \\
\textsc{smc\_u\_mid\_mod  } & \textsc{shape\_mn }   & 1.306    & 1.343    & 1.374    & 1.41     & 1.457    & 1.424    & 84  & moderate \\
\textsc{smc\_u\_mid\_op   } & \textsc{shape\_mn }   & 1.416    & 1.445    & 1.466    & 1.485    & 1.51     & 1.536    & 100 & complete \\
\textsc{smc\_u\_late\_cl  } & \textsc{shape\_mn }   & 1.331    & 1.367    & 1.398    & 1.436    & 1.494    & 1.704    & 100 & complete \\
\textsc{smc\_u\_late\_mod } & \textsc{shape\_mn }   & 1.298    & 1.327    & 1.356    & 1.395    & 1.458    & 1.511    & 100 & complete \\
\textsc{smc\_u\_late\_op  } & \textsc{shape\_mn }   & 1.3      & 1.332    & 1.352    & 1.385    & 1.453    & 1.404    & 86  & moderate \\
\textsc{smc\_u\_early\_all} & \textsc{simi\_mn  }   & 3463.681 & 4065.997 & 4456.346 & 4947.922 & 5636.337 & 6879.449 & 100 & complete \\
\textsc{smc\_u\_mid\_cl   } & \textsc{simi\_mn  }   & 1149.893 & 1609.314 & 2043.472 & 2730.371 & 3714.328 & 1294.872 & 11  & moderate \\
\textsc{smc\_u\_mid\_mod  } & \textsc{simi\_mn  }   & 1648.438 & 2031.756 & 2316.05  & 2654.391 & 3433.663 & 1519.229 & 2   & complete \\
\textsc{smc\_u\_mid\_op   } & \textsc{simi\_mn  }   & 2360.737 & 2686.072 & 2928.899 & 3281.598 & 3916.711 & 2995.686 & 55  & none     \\
\textsc{smc\_u\_late\_cl  } & \textsc{simi\_mn  }   & 1204.708 & 1350.053 & 1475.401 & 1609.901 & 1799.524 & 2521.58  & 100 & complete \\
\textsc{smc\_u\_late\_mod } & \textsc{simi\_mn  }   & 1207.403 & 1484.277 & 1688.796 & 1889.374 & 2190.724 & 2029.212 & 88  & moderate \\
\textsc{smc\_u\_late\_op  } & \textsc{simi\_mn  }   & 1235.752 & 1643.704 & 2018.649 & 2382.293 & 2893.308 & 3101.898 & 98  & complete \\
\textsc{smc\_u\_early\_all} & \textsc{te  	    }   & 273930   & 373230   & 406500   & 445230   & 479850   & 262110   & 3   & complete \\
\textsc{smc\_u\_mid\_cl   } & \textsc{te        }   & 6360     & 10920    & 15210    & 21960    & 32280    & 32460    & 96  & complete \\
\textsc{smc\_u\_mid\_mod  } & \textsc{te        }   & 42840    & 57930    & 69420    & 82800    & 113220   & 60480    & 29  & none     \\
\textsc{smc\_u\_mid\_op   } & \textsc{te        }   & 191910   & 230130   & 245400   & 266550   & 297900   & 61590    & 0   & complete \\
\textsc{smc\_u\_late\_cl  } & \textsc{te        }   & 82020    & 114360   & 137010   & 164340   & 205110   & 171180   & 81  & moderate \\
\textsc{smc\_u\_late\_mod } & \textsc{te        }   & 55920    & 71460    & 84150    & 97800    & 124260   & 71640    & 26  & none     \\
\textsc{smc\_u\_late\_op  } & \textsc{te        }   & 41880    & 60780    & 75030    & 88620    & 122760   & 33360    & 2   & complete


\end{longtable}
\end{footnotesize}
\end{center}
\end{landscape}

\pagestyle{headings}

%%%%%%%%%%%%%%%%%%%%%%%%%%%%%%%%%%%%%%%%%%%%%%%%%%%%%%%%%%%%%%%%%%%%%%%%%%%%%%%%%%%%%%%%%%%%%%%%
%%%%%%%%%%%%%%%%%%%%%%%%%%%%%%%%%%%%%%%%%%%%%%%%%%%%%%%%%%%%%%%%%%%%%%%%%%%%%%%%%%%%%%%%%%%%%%%%

\pagestyle{empty}
\begin{landscape}
\footnotesize
\begin{center}
\begin{footnotesize}
\begin{longtable}{llrrrrr|rrr}
\caption{Unabridged results for class-level metrics for Sierran Mixed Conifer - Xeric (\textsc{smc\_x}) calculated with \textsc{Fragstats}. This table shows the range of variability in landscape structure. Included are the $5^{\text{th}}$ percentile, $25^{\text{th}}$ percentile, $50^{\text{th}}$ percentile, $75^{\text{th}}$ percentile, and $95^{\text{th}}$ percentiles of the distribution, as well as the current class value, the current percentile range of variability (\%RV) for that proportion, and the departure classification. For seral stage abbreviations, see Table~\ref{condtable}.} \\
\label{tab:fragclass_smcx} \\

\hline 
\textbf{\begin{tabular}[c]{@{}l@{}}Cover-Seral Stage Type\end{tabular}}  &   
\textbf{\begin{tabular}[c]{@{}l@{}}Landscape\\ Metric\end{tabular}}  &   
\textbf{$5^{\text{th}}$ } &   
\textbf{$25^{\text{th}}$ } &   
\textbf{$50^{\text{th}}$ } &   
\textbf{$75^{\text{th}}$ } &   
\textbf{$95^{\text{th}}$ }  &  
\textbf{\begin{tabular}[c]{@{}l@{}}Current\\ Value\end{tabular}} &   
\textbf{\begin{tabular}[c]{@{}l@{}}Current\\ \%RV\end{tabular}} &   
\textbf{\begin{tabular}[c]{@{}l@{}}Departure\end{tabular}} \\  \\ \hline 
\endfirsthead

\multicolumn{10}{c}{{\bfseries \tablename\ \thetable{} -- continued from previous page}} \\
\hline 
\textbf{\begin{tabular}[c]{@{}l@{}}Cover-Seral Stage Type\end{tabular}}  &   
\textbf{\begin{tabular}[c]{@{}l@{}}Landscape\\ Metric\end{tabular}}  &   
\textbf{$5^{\text{th}}$ } &   
\textbf{$25^{\text{th}}$ } &   
\textbf{$50^{\text{th}}$ } &   
\textbf{$75^{\text{th}}$ } &   
\textbf{$95^{\text{th}}$ }  &  
\textbf{\begin{tabular}[c]{@{}l@{}}Current\\ Value\end{tabular}} &   
\textbf{\begin{tabular}[c]{@{}l@{}}Current\\ \%RV\end{tabular}} &   
\textbf{\begin{tabular}[c]{@{}l@{}}Departure\end{tabular}} \\  \\ \hline \endhead

\hline \multicolumn{10}{|l|}{{Continued on next page}} \\ \hline
\endfoot

\hline \hline
\endlastfoot

\textsc{smc\_x\_early\_all} & \textsc{ai        }   & 81.186   & 82.555   & 83.355   & 83.954   & 85.007   & 90.77    & 100 & complete \\
\textsc{smc\_x\_mid\_cl   } & \textsc{ai        }   & 72.124   & 74.619   & 76.8     & 78.853   & 82.176   & 82.364   & 96  & complete \\
\textsc{smc\_x\_mid\_mod  } & \textsc{ai        }   & 75.168   & 76.761   & 78.205   & 80.397   & 83.316   & 84.429   & 99  & complete \\
\textsc{smc\_x\_mid\_op   } & \textsc{ai        }   & 80.166   & 81.484   & 82.496   & 83.296   & 84.996   & 80.079   & 5   & complete \\
\textsc{smc\_x\_late\_cl  } & \textsc{ai        }   & 76.856   & 78.232   & 80.144   & 81.595   & 83.997   & 88.154   & 100 & complete \\
\textsc{smc\_x\_late\_mod } & \textsc{ai        }   & 73.673   & 75.422   & 76.635   & 78.092   & 80.337   & 85.809   & 100 & complete \\
\textsc{smc\_x\_late\_op  } & \textsc{ai        }   & 74.95    & 76.592   & 77.647   & 78.735   & 83.265   & 83.304   & 96  & complete \\
\textsc{smc\_x\_early\_all} & \textsc{area\_am  }   & 23.563   & 30.699   & 37.853   & 50.947   & 63.073   & 178.043  & 100 & complete \\
\textsc{smc\_x\_mid\_cl   } & \textsc{area\_am  }   & 1.843    & 2.565    & 3.604    & 6.566    & 12.713   & 9.921    & 89  & moderate \\
\textsc{smc\_x\_mid\_mod  } & \textsc{area\_am  }   & 3.967    & 5.461    & 8.059    & 13.117   & 30.272   & 15.779   & 81  & moderate \\
\textsc{smc\_x\_mid\_op   } & \textsc{area\_am  }   & 12.239   & 18.369   & 22.238   & 28.451   & 47.915   & 8.289    & 0   & complete \\
\textsc{smc\_x\_late\_cl  } & \textsc{area\_am  }   & 5.175    & 8.22     & 12.065   & 19.225   & 41.683   & 87.059   & 98  & complete \\
\textsc{smc\_x\_late\_mod } & \textsc{area\_am  }   & 3.173    & 4.461    & 6.333    & 8.419    & 18.676   & 28.024   & 98  & complete \\
\textsc{smc\_x\_late\_op  } & \textsc{area\_am  }   & 3.426    & 4.964    & 6.349    & 9.247    & 24.953   & 8.763    & 74  & none     \\
\textsc{smc\_x\_early\_all} & \textsc{area\_mn  }   & 5.518    & 6.581    & 7.287    & 8.06     & 9.056    & 18.228   & 100 & complete \\
\textsc{smc\_x\_mid\_cl   } & \textsc{area\_mn  }   & 1.426    & 1.763    & 2.148    & 2.736    & 3.694    & 4.375    & 100 & complete \\
\textsc{smc\_x\_mid\_mod  } & \textsc{area\_mn  }   & 2.262    & 2.71     & 3.154    & 3.903    & 4.988    & 5.782    & 99  & complete \\
\textsc{smc\_x\_mid\_op   } & \textsc{area\_mn  }   & 4.341    & 5.024    & 5.568    & 6.125    & 7.297    & 4.498    & 7   & moderate \\
\textsc{smc\_x\_late\_cl  } & \textsc{area\_mn  }   & 2.71     & 3.274    & 4.016    & 4.717    & 6.35     & 14.89    & 100 & complete \\
\textsc{smc\_x\_late\_mod } & \textsc{area\_mn  }   & 2.052    & 2.441    & 2.771    & 3.194    & 3.898    & 7.672    & 100 & complete \\
\textsc{smc\_x\_late\_op  } & \textsc{area\_mn  }   & 2.207    & 2.628    & 2.962    & 3.264    & 5.066    & 4.419    & 93  & moderate \\
\textsc{smc\_x\_early\_all} & \textsc{cai\_am   }   & 93.431   & 93.826   & 94.211   & 94.594   & 95.282   & 90.974   & 0   & complete \\
\textsc{smc\_x\_mid\_cl   } & \textsc{cai\_am   }   & 0.99     & 4.749    & 9.191    & 15.132   & 28.335   & 36.37    & 100 & complete \\
\textsc{smc\_x\_mid\_mod  } & \textsc{cai\_am   }   & 25.33    & 33.856   & 40.629   & 46.424   & 56.885   & 73.184   & 100 & complete \\
\textsc{smc\_x\_mid\_op   } & \textsc{cai\_am   }   & 60.715   & 65.409   & 68.982   & 72.015   & 75.673   & 84.974   & 100 & complete \\
\textsc{smc\_x\_late\_cl  } & \textsc{cai\_am   }   & 3.615    & 7.475    & 10.615   & 13.969   & 27.773   & 39.893   & 100 & complete \\
\textsc{smc\_x\_late\_mod } & \textsc{cai\_am   }   & 19.901   & 26.724   & 31.824   & 37.739   & 46.977   & 60.877   & 100 & complete \\
\textsc{smc\_x\_late\_op  } & \textsc{cai\_am   }   & 22.89    & 29.004   & 34.545   & 40.216   & 55.135   & 73.931   & 100 & complete \\
\textsc{smc\_x\_early\_all} & \textsc{clumpy    }   & 0.81     & 0.824    & 0.832    & 0.838    & 0.848    & 0.907    & 100 & complete \\
\textsc{smc\_x\_mid\_cl   } & \textsc{clumpy    }   & 0.721    & 0.746    & 0.768    & 0.788    & 0.822    & 0.824    & 96  & complete \\
\textsc{smc\_x\_mid\_mod  } & \textsc{clumpy    }   & 0.752    & 0.767    & 0.782    & 0.804    & 0.833    & 0.844    & 99  & complete \\
\textsc{smc\_x\_mid\_op   } & \textsc{clumpy    }   & 0.8      & 0.814    & 0.824    & 0.832    & 0.849    & 0.8      & 5   & complete \\
\textsc{smc\_x\_late\_cl  } & \textsc{clumpy    }   & 0.768    & 0.782    & 0.801    & 0.816    & 0.839    & 0.881    & 100 & complete \\
\textsc{smc\_x\_late\_mod } & \textsc{clumpy    }   & 0.736    & 0.754    & 0.766    & 0.78     & 0.803    & 0.858    & 100 & complete \\
\textsc{smc\_x\_late\_op  } & \textsc{clumpy    }   & 0.749    & 0.766    & 0.776    & 0.787    & 0.832    & 0.833    & 96  & complete \\
\textsc{smc\_x\_early\_all} & \textsc{core\_am  }   & 22.458   & 29.426   & 36.492   & 48.723   & 60.827   & 159.264  & 100 & complete \\
\textsc{smc\_x\_mid\_cl   } & \textsc{core\_am  }   & 0.022    & 0.12     & 0.414    & 1.177    & 4.8      & 3.615    & 94  & moderate \\
\textsc{smc\_x\_mid\_mod  } & \textsc{core\_am  }   & 1.391    & 2.314    & 4.003    & 7.277    & 21.528   & 12.118   & 86  & moderate \\
\textsc{smc\_x\_mid\_op   } & \textsc{core\_am  }   & 8.501    & 13.16    & 16.514   & 23.255   & 42.093   & 7.34     & 3   & complete \\
\textsc{smc\_x\_late\_cl  } & \textsc{core\_am  }   & 0.288    & 0.967    & 2.031    & 4.188    & 17.222   & 46.47    & 99  & complete \\
\textsc{smc\_x\_late\_mod } & \textsc{core\_am  }   & 0.913    & 1.569    & 2.462    & 4.115    & 11.394   & 17.985   & 98  & complete \\
\textsc{smc\_x\_late\_op  } & \textsc{core\_am  }   & 1.056    & 1.839    & 2.737    & 4.519    & 16.805   & 7.282    & 89  & moderate \\
\textsc{smc\_x\_early\_all} & \textsc{core\_mn  }   & 5.163    & 6.186    & 6.87     & 7.606    & 8.547    & 16.583   & 100 & complete \\
\textsc{smc\_x\_mid\_cl   } & \textsc{core\_mn  }   & 0.018    & 0.09     & 0.198    & 0.375    & 0.881    & 1.591    & 100 & complete \\
\textsc{smc\_x\_mid\_mod  } & \textsc{core\_mn  }   & 0.631    & 0.938    & 1.276    & 1.782    & 2.679    & 4.232    & 100 & complete \\
\textsc{smc\_x\_mid\_op   } & \textsc{core\_mn  }   & 2.841    & 3.307    & 3.834    & 4.327    & 5.445    & 3.822    & 50  & none     \\
\textsc{smc\_x\_late\_cl  } & \textsc{core\_mn  }   & 0.102    & 0.251    & 0.423    & 0.639    & 1.719    & 5.94     & 100 & complete \\
\textsc{smc\_x\_late\_mod } & \textsc{core\_mn  }   & 0.427    & 0.675    & 0.885    & 1.191    & 1.756    & 4.67     & 100 & complete \\
\textsc{smc\_x\_late\_op  } & \textsc{core\_mn  }   & 0.537    & 0.771    & 1.016    & 1.293    & 2.797    & 3.267    & 97  & complete \\
\textsc{smc\_x\_early\_all} & \textsc{cpland    }   & 0.616    & 0.82     & 0.926    & 1.005    & 1.132    & 1.005    & 75  & moderate \\
\textsc{smc\_x\_mid\_cl   } & \textsc{cpland    }   & 0        & 0.001    & 0.002    & 0.005    & 0.013    & 0.024    & 98  & complete \\
\textsc{smc\_x\_mid\_mod  } & \textsc{cpland    }   & 0.019    & 0.033    & 0.052    & 0.074    & 0.123    & 0.112    & 94  & moderate \\
\textsc{smc\_x\_mid\_op   } & \textsc{cpland    }   & 0.288    & 0.34     & 0.381    & 0.435    & 0.506    & 0.103    & 0   & complete \\
\textsc{smc\_x\_late\_cl  } & \textsc{cpland    }   & 0.006    & 0.016    & 0.029    & 0.047    & 0.121    & 0.222    & 100 & complete \\
\textsc{smc\_x\_late\_mod } & \textsc{cpland    }   & 0.018    & 0.032    & 0.045    & 0.064    & 0.094    & 0.118    & 98  & complete \\
\textsc{smc\_x\_late\_op  } & \textsc{cpland    }   & 0.018    & 0.033    & 0.044    & 0.058    & 0.112    & 0.054    & 70  & none     \\
\textsc{smc\_x\_early\_all} & \textsc{cwed      }   & 0.567    & 0.755    & 0.815    & 0.877    & 0.931    & 0.621    & 9   & moderate \\
\textsc{smc\_x\_mid\_cl   } & \textsc{cwed      }   & 0.013    & 0.019    & 0.026    & 0.036    & 0.054    & 0.044    & 86  & moderate \\
\textsc{smc\_x\_mid\_mod  } & \textsc{cwed      }   & 0.064    & 0.086    & 0.103    & 0.124    & 0.166    & 0.066    & 7   & moderate \\
\textsc{smc\_x\_mid\_op   } & \textsc{cwed      }   & 0.274    & 0.341    & 0.366    & 0.395    & 0.438    & 0.084    & 0   & complete \\
\textsc{smc\_x\_late\_cl  } & \textsc{cwed      }   & 0.185    & 0.247    & 0.292    & 0.344    & 0.423    & 0.32     & 67  & none     \\
\textsc{smc\_x\_late\_mod } & \textsc{cwed      }   & 0.092    & 0.117    & 0.134    & 0.156    & 0.195    & 0.098    & 9   & moderate \\
\textsc{smc\_x\_late\_op  } & \textsc{cwed      }   & 0.062    & 0.085    & 0.107    & 0.127    & 0.18     & 0.045    & 1   & complete \\
\textsc{smc\_x\_early\_all} & \textsc{econ\_am  }   & 33.186   & 34.511   & 35.683   & 36.791   & 38.683   & 43.39    & 100 & complete \\
\textsc{smc\_x\_mid\_cl   } & \textsc{econ\_am  }   & 24.953   & 28.163   & 30.851   & 33.489   & 37.397   & 25.371   & 7   & moderate \\
\textsc{smc\_x\_mid\_mod  } & \textsc{econ\_am  }   & 23.814   & 25.496   & 26.849   & 28.264   & 29.921   & 18.358   & 0   & complete \\
\textsc{smc\_x\_mid\_op   } & \textsc{econ\_am  }   & 25.081   & 25.963   & 26.637   & 27.335   & 28.304   & 23.735   & 0   & complete \\
\textsc{smc\_x\_late\_cl  } & \textsc{econ\_am  }   & 32.879   & 37.184   & 38.937   & 40.481   & 43.264   & 31.662   & 3   & complete \\
\textsc{smc\_x\_late\_mod } & \textsc{econ\_am  }   & 25.311   & 27.703   & 29.165   & 30.848   & 33.33    & 23.954   & 1   & complete \\
\textsc{smc\_x\_late\_op  } & \textsc{econ\_am  }   & 23.868   & 25.144   & 25.942   & 26.948   & 28.344   & 22.941   & 1   & complete \\
\textsc{smc\_x\_early\_all} & \textsc{econ\_mn  }   & 34.667   & 36.042   & 36.846   & 37.655   & 39.132   & 39.912   & 99  & complete \\
\textsc{smc\_x\_mid\_cl   } & \textsc{econ\_mn  }   & 26.279   & 29.603   & 31.645   & 33.882   & 37.35    & 23.118   & 1   & complete \\
\textsc{smc\_x\_mid\_mod  } & \textsc{econ\_mn  }   & 24.942   & 26.354   & 27.416   & 28.38    & 29.883   & 19.567   & 0   & complete \\
\textsc{smc\_x\_mid\_op   } & \textsc{econ\_mn  }   & 25.729   & 26.374   & 26.898   & 27.529   & 28.686   & 24.842   & 0   & complete \\
\textsc{smc\_x\_late\_cl  } & \textsc{econ\_mn  }   & 35.051   & 38.204   & 39.567   & 40.801   & 43.315   & 37.613   & 20  & moderate \\
\textsc{smc\_x\_late\_mod } & \textsc{econ\_mn  }   & 26.849   & 28.485   & 29.806   & 31.166   & 33.335   & 24.58    & 1   & complete \\
\textsc{smc\_x\_late\_op  } & \textsc{econ\_mn  }   & 24.428   & 25.744   & 26.791   & 27.642   & 29.162   & 26.772   & 50  & none     \\
\textsc{smc\_x\_early\_all} & \textsc{ed        }   & 1.509    & 2.056    & 2.239    & 2.452    & 2.643    & 1.444    & 3   & complete \\
\textsc{smc\_x\_mid\_cl   } & \textsc{ed        }   & 0.035    & 0.06     & 0.084    & 0.121    & 0.178    & 0.179    & 96  & complete \\
\textsc{smc\_x\_mid\_mod  } & \textsc{ed        }   & 0.236    & 0.319    & 0.382    & 0.456    & 0.624    & 0.333    & 29  & none     \\
\textsc{smc\_x\_mid\_op   } & \textsc{ed        }   & 1.057    & 1.268    & 1.352    & 1.468    & 1.641    & 0.339    & 0   & complete \\
\textsc{smc\_x\_late\_cl  } & \textsc{ed        }   & 0.452    & 0.63     & 0.755    & 0.905    & 1.13     & 0.943    & 81  & moderate \\
\textsc{smc\_x\_late\_mod } & \textsc{ed        }   & 0.308    & 0.394    & 0.464    & 0.539    & 0.684    & 0.395    & 26  & none     \\
\textsc{smc\_x\_late\_op  } & \textsc{ed        }   & 0.231    & 0.335    & 0.413    & 0.488    & 0.676    & 0.184    & 2   & complete \\
\textsc{smc\_x\_early\_all} & \textsc{gyrate\_am}   & 220.699  & 248.711  & 278.103  & 315.922  & 365.359  & 640.292  & 100 & complete \\
\textsc{smc\_x\_mid\_cl   } & \textsc{gyrate\_am}   & 57.442   & 67.105   & 81.838   & 106.909  & 144.433  & 132.35   & 90  & moderate \\
\textsc{smc\_x\_mid\_mod  } & \textsc{gyrate\_am}   & 83.518   & 99.496   & 118.926  & 151.114  & 226.478  & 170.304  & 84  & moderate \\
\textsc{smc\_x\_mid\_op   } & \textsc{gyrate\_am}   & 150.966  & 180.606  & 198.396  & 225.322  & 278.027  & 135.764  & 1   & complete \\
\textsc{smc\_x\_late\_cl  } & \textsc{gyrate\_am}   & 96.563   & 120.305  & 146.321  & 182.948  & 282.379  & 454.655  & 100 & complete \\
\textsc{smc\_x\_late\_mod } & \textsc{gyrate\_am}   & 75.607   & 90.138   & 105.768  & 122.47   & 177.76   & 223.005  & 98  & complete \\
\textsc{smc\_x\_late\_op  } & \textsc{gyrate\_am}   & 79.302   & 95.9     & 107.79   & 124.127  & 215.737  & 125.479  & 77  & moderate \\
\textsc{smc\_x\_early\_all} & \textsc{iji       }   & 58.879   & 60.142   & 61.241   & 62.013   & 63.28    & 70.354   & 100 & complete \\
\textsc{smc\_x\_mid\_cl   } & \textsc{iji       }   & 40.41    & 45.293   & 48.955   & 52.201   & 55.001   & 57.596   & 100 & complete \\
\textsc{smc\_x\_mid\_mod  } & \textsc{iji       }   & 44.488   & 47.413   & 50.138   & 52.547   & 55.182   & 55.788   & 97  & complete \\
\textsc{smc\_x\_mid\_op   } & \textsc{iji       }   & 48.105   & 51.41    & 53.918   & 56.255   & 59.287   & 60.512   & 100 & complete \\
\textsc{smc\_x\_late\_cl  } & \textsc{iji       }   & 46.083   & 49.2     & 51.524   & 53.66    & 57.07    & 64.339   & 100 & complete \\
\textsc{smc\_x\_late\_mod } & \textsc{iji       }   & 45.233   & 47.976   & 49.671   & 51.132   & 53.313   & 57.938   & 100 & complete \\
\textsc{smc\_x\_late\_op  } & \textsc{iji       }   & 43.564   & 46.445   & 48.589   & 50.809   & 53.417   & 51.73    & 88  & moderate \\
\textsc{smc\_x\_early\_all} & \textsc{pd        }   & 0.104    & 0.126    & 0.135    & 0.14     & 0.151    & 0.061    & 0   & complete \\
\textsc{smc\_x\_mid\_cl   } & \textsc{pd        }   & 0.005    & 0.008    & 0.011    & 0.014    & 0.02     & 0.015    & 80  & moderate \\
\textsc{smc\_x\_mid\_mod  } & \textsc{pd        }   & 0.026    & 0.034    & 0.039    & 0.046    & 0.058    & 0.026    & 6   & moderate \\
\textsc{smc\_x\_mid\_op   } & \textsc{pd        }   & 0.078    & 0.095    & 0.102    & 0.11     & 0.12     & 0.027    & 0   & complete \\
\textsc{smc\_x\_late\_cl  } & \textsc{pd        }   & 0.048    & 0.061    & 0.068    & 0.077    & 0.087    & 0.038    & 2   & complete \\
\textsc{smc\_x\_late\_mod } & \textsc{pd        }   & 0.038    & 0.046    & 0.051    & 0.056    & 0.066    & 0.025    & 0   & complete \\
\textsc{smc\_x\_late\_op  } & \textsc{pd        }   & 0.026    & 0.036    & 0.044    & 0.052    & 0.061    & 0.016    & 0   & complete \\
\textsc{smc\_x\_early\_all} & \textsc{pland     }   & 0.646    & 0.869    & 0.985    & 1.067    & 1.2      & 1.104    & 83  & moderate \\
\textsc{smc\_x\_mid\_cl   } & \textsc{pland     }   & 0.009    & 0.016    & 0.023    & 0.038    & 0.057    & 0.068    & 98  & complete \\
\textsc{smc\_x\_mid\_mod  } & \textsc{pland     }   & 0.069    & 0.101    & 0.127    & 0.159    & 0.225    & 0.153    & 73  & none     \\
\textsc{smc\_x\_mid\_op   } & \textsc{pland     }   & 0.421    & 0.506    & 0.564    & 0.627    & 0.716    & 0.121    & 0   & complete \\
\textsc{smc\_x\_late\_cl  } & \textsc{pland     }   & 0.143    & 0.213    & 0.273    & 0.35     & 0.481    & 0.558    & 99  & complete \\
\textsc{smc\_x\_late\_mod } & \textsc{pland     }   & 0.087    & 0.115    & 0.142    & 0.173    & 0.226    & 0.194    & 85  & moderate \\
\textsc{smc\_x\_late\_op  } & \textsc{pland     }   & 0.065    & 0.104    & 0.132    & 0.159    & 0.238    & 0.073    & 7   & moderate \\
\textsc{smc\_x\_early\_all} & \textsc{shape\_am }   & 2.08     & 2.235    & 2.371    & 2.55     & 2.832    & 2.741    & 89  & moderate \\
\textsc{smc\_x\_mid\_cl   } & \textsc{shape\_am }   & 1.273    & 1.342    & 1.413    & 1.559    & 1.834    & 1.576    & 77  & moderate \\
\textsc{smc\_x\_mid\_mod  } & \textsc{shape\_am }   & 1.421    & 1.536    & 1.654    & 1.795    & 2.115    & 1.724    & 64  & none     \\
\textsc{smc\_x\_mid\_op   } & \textsc{shape\_am }   & 1.755    & 1.886    & 1.962    & 2.083    & 2.307    & 1.681    & 1   & complete \\
\textsc{smc\_x\_late\_cl  } & \textsc{shape\_am }   & 1.495    & 1.627    & 1.774    & 1.944    & 2.284    & 2.695    & 100 & complete \\
\textsc{smc\_x\_late\_mod } & \textsc{shape\_am }   & 1.401    & 1.489    & 1.574    & 1.694    & 1.944    & 2.004    & 97  & complete \\
\textsc{smc\_x\_late\_op  } & \textsc{shape\_am }   & 1.401    & 1.496    & 1.558    & 1.665    & 2.111    & 1.444    & 11  & moderate \\
\textsc{smc\_x\_early\_all} & \textsc{shape\_mn }   & 1.505    & 1.547    & 1.577    & 1.612    & 1.651    & 1.578    & 51  & none     \\
\textsc{smc\_x\_mid\_cl   } & \textsc{shape\_mn }   & 1.218    & 1.266    & 1.301    & 1.334    & 1.41     & 1.451    & 100 & complete \\
\textsc{smc\_x\_mid\_mod  } & \textsc{shape\_mn }   & 1.306    & 1.343    & 1.374    & 1.41     & 1.457    & 1.424    & 84  & moderate \\
\textsc{smc\_x\_mid\_op   } & \textsc{shape\_mn }   & 1.416    & 1.445    & 1.466    & 1.485    & 1.51     & 1.536    & 100 & complete \\
\textsc{smc\_x\_late\_cl  } & \textsc{shape\_mn }   & 1.331    & 1.367    & 1.398    & 1.436    & 1.494    & 1.704    & 100 & complete \\
\textsc{smc\_x\_late\_mod } & \textsc{shape\_mn }   & 1.298    & 1.327    & 1.356    & 1.395    & 1.458    & 1.511    & 100 & complete \\
\textsc{smc\_x\_late\_op  } & \textsc{shape\_mn }   & 1.3      & 1.332    & 1.352    & 1.385    & 1.453    & 1.404    & 86  & moderate \\
\textsc{smc\_x\_early\_all} & \textsc{simi\_mn  }   & 3463.681 & 4065.997 & 4456.346 & 4947.922 & 5636.337 & 6879.449 & 100 & complete \\
\textsc{smc\_x\_mid\_cl   } & \textsc{simi\_mn  }   & 1149.893 & 1609.314 & 2043.472 & 2730.371 & 3714.328 & 1294.872 & 11  & moderate \\
\textsc{smc\_x\_mid\_mod  } & \textsc{simi\_mn  }   & 1648.438 & 2031.756 & 2316.05  & 2654.391 & 3433.663 & 1519.229 & 2   & complete \\
\textsc{smc\_x\_mid\_op   } & \textsc{simi\_mn  }   & 2360.737 & 2686.072 & 2928.899 & 3281.598 & 3916.711 & 2995.686 & 55  & none     \\
\textsc{smc\_x\_late\_cl  } & \textsc{simi\_mn  }   & 1204.708 & 1350.053 & 1475.401 & 1609.901 & 1799.524 & 2521.58  & 100 & complete \\
\textsc{smc\_x\_late\_mod } & \textsc{simi\_mn  }   & 1207.403 & 1484.277 & 1688.796 & 1889.374 & 2190.724 & 2029.212 & 88  & moderate \\
\textsc{smc\_x\_late\_op  } & \textsc{simi\_mn  }   & 1235.752 & 1643.704 & 2018.649 & 2382.293 & 2893.308 & 3101.898 & 98  & complete \\
\textsc{smc\_x\_early\_all} & \textsc{te  	    }   & 273930   & 373230   & 406500   & 445230   & 479850   & 262110   & 3   & complete \\
\textsc{smc\_x\_mid\_cl   } & \textsc{te        }   & 6360     & 10920    & 15210    & 21960    & 32280    & 32460    & 96  & complete \\
\textsc{smc\_x\_mid\_mod  } & \textsc{te        }   & 42840    & 57930    & 69420    & 82800    & 113220   & 60480    & 29  & none     \\
\textsc{smc\_x\_mid\_op   } & \textsc{te        }   & 191910   & 230130   & 245400   & 266550   & 297900   & 61590    & 0   & complete \\
\textsc{smc\_x\_late\_cl  } & \textsc{te        }   & 82020    & 114360   & 137010   & 164340   & 205110   & 171180   & 81  & moderate \\
\textsc{smc\_x\_late\_mod } & \textsc{te        }   & 55920    & 71460    & 84150    & 97800    & 124260   & 71640    & 26  & none     \\
\textsc{smc\_x\_late\_op  } & \textsc{te        }   & 41880    & 60780    & 75030    & 88620    & 122760   & 33360    & 2   & complete

\end{longtable}
\end{footnotesize}
\end{center}
\end{landscape}

\pagestyle{headings}


\backmatter 
%\include{glossary} 
%\include{notat} 
\bibliographystyle{humannat}%amsalpha} %The style you want to use for references. 
\interlinepenalty=10000
\bibliography{bibliography.bib} %The files containing all the articles and books you ever referenced. 
\setcitestyle{notesep={:},aysep={}} 

%\printindex %Make an index AUTOMATICALLY 

\end{document} 

% Tags for Thesis
%		\documentclass[12pt]{article}
%		
%		%\usepackage{geometry}
%		%\geometry{verbose,letterpaper,tmargin=2.54cm,bmargin=2.54cm,lmargin=2.54cm,rmargin=2.54cm}
%		
%		\usepackage[left=1in,right=2in,top=1in,bottom=1in]{geometry}
%		
%		\usepackage{booktabs, colortbl}
%		\usepackage[table,xcdraw,dvipsnames]{xcolor}
%		%\usepackage[utf8]{inputenc}
%		\usepackage{amsmath} % need this to put plain text in math mode
%		%\usepackage{graphicx}
%		\usepackage[textwidth=1.8in]{todonotes}
%		\usepackage{setspace}
%		%\usepackage{listings}
%		\usepackage{natbib}
%		\usepackage{wrapfig,subfig,graphicx}
%		
%		% from hrv report
%		\usepackage{pdflscape}
%		\usepackage{longtable}
%		\usepackage{latexsym}
%		%\usepackage{listings}
%		%\usepackage{textcomp}
%		%\usepackage{sidecap}
%		\usepackage{multirow}
%		%\usepackage{changepage}
%		%\usepackage{siunitx}
%		\usepackage[figuresright]{rotating}
%		\usepackage[format=hang,labelfont=bf,size=small]{caption}
%		\usepackage{breakurl}
%		
%		%\usepackage[modulo]{lineno}
%		
%		
%		\usepackage{hyperref}
%		%\usepackage{cleverref}
%		\newcommand\myshade{85}
%		\colorlet{mylinkcolor}{violet}
%		\colorlet{mycitecolor}{Aquamarine}
%		\colorlet{myurlcolor}{YellowOrange}
%		
%		\hypersetup{colorlinks=true, citecolor=mycitecolor!\myshade!black, linkcolor=mylinkcolor!\myshade!black, urlcolor = myurlcolor!\myshade!black}
%		
%		%\doublespacing
%		%\linenumbers
%		
%		\title{Effect of climate change on future landscape composition and configuration, Tahoe National Forest, California, USA}
%		\author{ Maritza Mallek }
%		\date{\today}
%		
%		\begin{document}
%		\maketitle
%		%\begin{spacing}{1.9}
%		\begin{spacing}{1}
%		bibliographystyle{humannat}%amsalpha} %The style you want to use for references. 
%		\bibliography{bibliography} %The files containing all the articles and books you ever referenced. 
%		\setcitestyle{notesep={:},aysep={}} 

