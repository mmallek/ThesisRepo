% !TEX root = master.tex
\newpage
\section{Lodgepole Pine (LPN)}

\subsection*{General Information}

\subsubsection{Cover Type Overview}

\textbf{Lodgepole Pine (LPN)}
\newline
Crosswalks
\begin{itemize}
	\item EVeg: Regional Dominance Type 1
	\begin{itemize}
		\item Lodgepole Pine
	\end{itemize}

	\item LandFire BpS Model
	\begin{itemize}
		\item 0610581 Sierra Nevada Subalpine Lodgepole Pine Forest and Woodland - Wet
		\item 0610582 Sierra Nevada Subalpine Lodgepole Pine Forest and Woodland - Dry

	\end{itemize}

	\item Presettlement Fire Regime Type
	\begin{itemize}
		\item Lodgepole Pine
	\end{itemize}
\end{itemize}

\paragraph{Lodgepole Pine with Aspen (LPN-ASP)}
This type is created by overlaying the NRIS TERRA Inventory of Aspen on top of the EVeg layer. Where it intersects with LPN it is assigned to LPN-ASP.
\newline

\noindent Reviewed by Shana Gross, Ecologist, USDA Forest Service

\subsubsection{Vegetation Description}
\paragraph{Lodgepole Pine (LPN)} \emph{P. contorta} ssp. \emph{murrayana} is the overwhelming dominant within its forest community, mixing occasionally with \emph{Abies magnifica}, and with scattered \emph{Pinus jeffreyi}  and \emph{Pinus monticola}, and \emph{Tsuga mertensiana} at higher elevations (Fites-Kaufman et al. 2007). Mature Sierran stands often contain significant seedlings and saplings. Understory characteristics are influenced by proximity to meadow and stream margins. \emph{Arctostaphylos} and \emph{Ribes} are common shrubs. Stands associated with meadow edges and streams may have a rich herbaceous layer consisting of grasses, forbs, and sedges. Species associations are likely very location specific. Plants present may include but are not limited to \emph{Cassiope}, \emph{Vaccinium}, \emph{Phyllodoce}, \emph{Kalmia}, \emph{Ceanothus}, \emph{Chrysolepis}, and \emph{Carex}. Elsewhere, the understory may be virtually absent, consisting of scattered shrubs such as \emph{Quercus vaccinifolia}, and herbs like \emph{Antennaria}, \emph{Arabis}, \emph{Eriogonum}, and \emph{Gayophytum}. Fast-moving streams within the cover type are generally characterized by relatively dense populations of \emph{Salix} (Bartolome 1988, Fites-Kaufman et al. 2007, LandFire 2007a, LandFire 2007b).  

\paragraph{Lodgepole Pine with Aspen (LPN-ASP)}	When \emph{Populus tremuloides} co-occurs with LPN on the west side of the Sierran crest, it is typically found in smaller patches, often less than 2 ha (5 acres) in size. Mature stands in which \emph{P. tremuloides} are still dominant are usually relatively open. Average canopy closures range from 60 to 100 percent in young and intermediate-aged stands and from 25 to 60 percent in mature stands. The open nature of the stands results in substantial light penetration to the ground (Verner 1988).

\subsubsection{Distribution}
\paragraph{Lodgepole Pine (LPN)}	Open stands of \emph{P. contorta} ssp. \emph{murrayana}, which make up a widespread upper montane forest/woodland, tolerating both rocky soils and semisaturated meadow edges, in an elevational belt within and above the \emph{A. magnifica zone}. These forests, strongly dominated by \emph{P. contorta} ssp. \emph{murrayana}, generally occur at elevations of about 1,830 m to 2,400 m (6000 ft to 7875 ft) in the northern Sierra Nevada. Stands of \emph{P. contorta} ssp. \emph{murrayana} may reach much lower, however, with cold air drainage down glacial canyons (Fites-Kaufman et al. 2007, Anderson 1996). On infertile soils, \emph{P. contorta} ssp. \emph{murrayana} is often the only tree species that will grow (Lotan and Critchfield 1990).
More than any other Sierran conifer, \emph{P. contorta} ssp. \emph{murrayana} is relatively tolerant of poor soil aeration, and thus grows well around the margins of wet meadows and other moist areas. Many upper montane and subalpine meadows in the Sierra Nevada exhibit invasion of young \emph{P. contorta} ssp. \emph{murrayana} moving inward from their drier margins. It is not clear how much this process has been influenced by changes in fire frequency or grazing over the last 150 years (Fites-Kaufman et al. 2007).

\paragraph{Lodgepole Pine with Aspen (LPN-ASP)}		Sites supporting \emph{P. tremuloides} are associated with added soil moisture, i.e., azonal wet sites. These sites are found throughout the LPN zone, often close to streams, lakes, and meadows. Other sites include rock reservoirs, springs and seeps. Terrain can be simple to complex (LandFire 2007c). 


\subsection*{Disturbances}

\subsubsection{Wildfire}

\paragraph{Lodgepole Pine (LPN)} 	Wildfires tend to be high mortality, stand-replacing fires that initiate a process of post-fire forest succession. High mortality fires kill large as well as small trees, and may kill many of the shrubs and herbs as well, although below-ground organs of at least some individual shrubs and herbs survive and resprout. Low mortality fires tend to only kill small seedlings and depend on the herbaceous layer to carry fire.

Unlike the Rocky Mountain subspecies of \emph{P. contorta} (ssp. \emph{latifolia}), \emph{P. contorta} ssp. \emph{murrayana} does not have serotinous cones (Fites-Kaufman et al. 2007). Following high mortality fire, it initially establishes in even-aged stands, but small-scale disturbances and the ability of the subspecies to regenerate in the absence of fire promote uneven-aged structure (Cope 1993, Gross 2013).

High mortality fire occurs at long intervals. Mixed severity fire is related to fire behavior across the often moist areas where \emph{P. contorta} ssp. \emph{murrayana} is found. Surface fires are more common on drier sites, although in general sparse fuels limit fire ignition and spread. Most fires are small (less than 1 ha) but very large fires covering hundreds of hectares do occur (LandFire 2007a, LandFire 2007b). This is due in part to the high susceptibility to fire mortality by \emph{P. contorta} ssp. \emph{murrayana} because of its thin bark and shallower roots. Postfire conditions provide an ideal seedbed, and \emph{P. contorta} ssp. \emph{murrayana} is an early post-fire colonizer (Cope 1993).

\paragraph{Lodgepole Pine with Aspen (LPN-ASP)}	Sites supporting \emph{P. tremuloides} are maintained by stand-replacing disturbances that allow regeneration from below-ground suckers. Upland clones are impaired or suppressed by conifer ingrowth and overtopping and intensive grazing that inhibits growth. In a reference condition scenario, a few stands will advance toward conifer dominance, but in the current landscape scenario where fire has been reduced from reference conditions there are many more conifer-dominated mixed aspen stands (LandFire 2007c, Verner 1988). 

Estimates of fire rotations for these variants are available from the LandFire project and a few review papers. The LandFire project’s published fire return intervals are based on a series of associated models created using the Vegetation Dynamics Development Tool (VDDT). In VDDT, fires are specified concurrently with the transition that follows them. For example, a replacement fire causes a transition to the early development stage. In the RMLands model, such fires are classified as high mortality. However, in VDDT mixed severity fires may cause a transition to early development, a transition to a more open seral stage, or no transition at all. In this case, we categorize the first example as a high mortality fire, and the second and third examples as a low mortality fire. Based on this approach, we calculated fire rotations and the probability of high mortality fire for each of the LPN and LPN-ASP seral stages (Tables~\ref{tab:lpndesc_fire} and \ref{tab:lpnaspdesc_fire}). We computed overall target fire rotations based on values from Mallek et al. (2013) and Van de Water and Safford (2011). 



\begin{table}[]
\centering
\caption{Fire rotation (years) and proportion of high (versus low) mortality fires for Lodgepole Pine type. Values were derived from VDDT model 0610790 (LandFire 2007), Mallek et al. (2013), and Estes (personal communication). }
\label{tab:lpndesc_fire}
\begin{tabular}{@{}lcc@{}}
\toprule
\textbf{Condition}          & \textbf{Fire Rotation} & \multicolumn{1}{l}{\textbf{\begin{tabular}[c]{@{}l@{}}Proportion \\ High Mortality\end{tabular}}} \\ \midrule
Target                      & 52    & n/a        \\
Early Development - All     & 29    & 0.03       \\
Mid Development - Closed    & 59    & 0.41       \\
Mid Development - Moderate  & 27    & 0.15       \\
Mid Development - Open      & 18    & 0.07       \\
Late Development - Closed   & 37    & 0.26       \\
Late Development - Moderate & 24    & 0.13       \\
Late Development - Open     & 18    & 0.07       \\ \bottomrule
\end{tabular}
\end{table}

\begin{table}[]
\centering
\caption{Fire rotation (years) and proportion of high (versus low) mortality fires for Lodgepole Pine - Aspen type. Values were derived from VDDT model 0610790 (LandFire 2007) and Van de Water and Safford (pers. comm. 2013).}
\label{tab:lpnaspdesc_fire}
\begin{tabular}{@{}lcc@{}}
\toprule
\textbf{Condition}               & \textbf{Fire Rotation} & \multicolumn{1}{l}{\textbf{\begin{tabular}[c]{@{}l@{}}Proportion \\ High Mortality\end{tabular}}} \\ \midrule
Target                           & 52     & n/a        \\
Early Development - Aspen        & 29     & 0.03       \\
Mid Development - Aspen          & 59     & 0.41       \\
Mid Development - Aspen-Conifer  & 27     & 0.15       \\
Late Development - Conifer-Aspen & 24     & 0.13       \\
Late Development - Closed        & 37     & 0.26       \\ \bottomrule
\end{tabular}
\end{table}

\subsubsection{Other Disturbance}
Other disturbances are not currently modeled, but may, depending on the seral stage affected and mortality levels, reset patches to early development, maintain existing seral stages, or shift/accelerate succession to a more open seral stage. 

\subsection*{Vegetation Seral Stages}
We recognize seven separate seral stages for LPN: Early Development (ED), Mid Development - Open Canopy Cover (MDO), Mid Development - Moderate Canopy Cover, Mid Development - Closed Canopy Cover (MDC), Late Development - Open Canopy Cover (LDO), Late Development - Moderate Canopy Cover (LDM), and Late Development - Closed Canopy Cover (LDC) (Figure~\ref{transmodel_lpn}). The LPN-ASP variant is assigned to five seral stages: Early Development - Aspen (EDA), Mid Development - Aspen (MDA), Mid Development - Aspen with Conifer (MDAC), Late Development - Conifer with Aspen (LDCA), and Late Development - Closed Canopy Cover (LDC) (Figure~\ref{transmodel_lpn-asp}).

Our seral stages are an alternative to ``successional'' classes that imply a linear progression of states and tend not to incorporate disturbance. The seral stages identified here are derived from a combination of successional processes and anthropogenic and natural disturbance, and are intended to represent a composition and structural condition that can be arrived at from multiple other conditions described for that landcover type. Thus our seral stages incorporate age, size, canopy cover, and vegetation composition. In general, the delineation of stages has originated from the LandFire biophysical setting model descriptive of a given landcover type; however, seral stages are not necessarily identical to the classes identified in those models.


\begin{figure}[htbp]
\centering
\includegraphics[width=0.8\textwidth]{/Users/mmallek/Documents/Thesis/statetransmodel/StateTransitionModel/7class.png}
\caption{State and Transition Model for Lodgepole Pine Forest (not inclusive of the aspen variant). Each dark grey box represents one of the seven seral stages for this landcover type. Each column of boxes represents a stage of development: early, middle, and late. Each row of boxes represents a different level of canopy cover: closed (70-100\%), moderate (40-70\%), and open (0-40\%). Transitions between states/seral stages may occur as a result of high mortality fire, low mortality fire, or succession. Specific pathways for each are denoted by the appropriate color line and arrow: red lines relate to high mortality fire, orange lines relate to low mortality fire, and green lines relate to natural succession.} 
\label{transmodel_lpn}
\end{figure}

\begin{figure}[htbp]
\centering
\includegraphics[width=0.8\textwidth]{/Users/mmallek/Documents/Thesis/statetransmodel/StateTransitionModel/5class-asp.png}
\caption{State and Transition Model for Lodgepole Pine Forest - Aspen variant. Each dark grey box represents one of the seven seral stages for this landcover type. Each column of boxes represents a stage of development: early, middle, and late. Transitions between states/seral stages may occur as a result of high mortality fire, low mortality fire, or succession. Specific pathways for each are denoted by the appropriate color line and arrow: red lines relate to high mortality fire, orange lines relate to low mortality fire, and green lines relate to natural succession.} 
\label{transmodel_lpn-asp}
\end{figure}


\subsubsection{Lodgepole Pine}

\paragraph{Early Development (ED)}

\paragraph{Description} Grasses, forbs, low shrubs, and sparse to moderate cover of trees (primarily \emph{P. contorta} ssp. \emph{murrayana}) seedlings/saplings with an open canopy. This seral stage is characterized by the recruitment of a new cohort of early successional, shade-intolerant tree species into an open area created by a stand-replacing disturbance. 


A short period of herbaceous productivity precedes closure of the tree canopy on productive sites. The prolific seed output, establishment, and seedling growth of \emph{P. contorta} ssp. \emph{murrayana} makes the period of herbaceous production short (Bartolome 1988). \emph{P. contorta} ssp. \emph{murrayana} regeneration density ranges from moderate to dog hair thickets (LandFire 2007a).


\paragraph{Succession Transition} In the absence of disturbance, patches in this seral stage will begin transitioning to MDC at 10 years at a rate of 0.6 per time step. At 40 years, all patches will succeed. On average, patches remain in early development for 18 years.

\paragraph{Wildfire Transition} High mortality wildfire (100\% of fires in this seral stage) recycles the patch through the Early Development seral stage. No transition occurs as a result of low mortality fire.

\noindent\hrulefill


\paragraph{Mid Development - Open Canopy Cover (MDO)}

\paragraph{Description} Sparse ground cover of grasses, forbs, and shrubs. Mid-maturity \emph{P. contorta} ssp. \emph{murrayana} where surface fire or other disturbance has opened the stand. Canopy cover ranges from 10-50\% (LandFire 2007a).
Continued recruitment into stands produces overstocking and slow growth of the overcrowded trees. This overcrowding may make them susceptible to insects, although others have argued that the more vigorously growing trees are more likely to be attacked. Beetle infestation creates large quantities of fuel that increase the probability of wildfire (Bartolome 1988).


\paragraph{Succession Transition} Patches in this seral stage may stay in this seral stage under low mortality disturbance, but after 10 years without fire they begin transitioning to MDM at a rate of 0.8 per time step. Succession to LDO occurs once the patch has been in mid development for 50 years. The rate of succession per time step is 0.5. At 100 years, all stands will succeed to LDO. On average, patches remain in mid development for 54 years.

\paragraph{Wildfire Transition} High mortality wildfire (7\% of fires in this seral stage) recycles the patch through the Early Development seral stage. Low mortality wildfire (93\%) maintains the patch in MDO.

\noindent\hrulefill

\paragraph{Mid Development - Moderate Canopy Cover (MDM)}

\paragraph{Description} Sparse ground cover of grasses, forbs, and shrubs. Mid-maturity \emph{P. contorta} ssp. \emph{murrayana} where surface fire or other disturbance has opened the stand. Canopy cover ranges from 10-50\% (LandFire 2007a).

Continued recruitment into stands produces overstocking and slow growth of the overcrowded trees. This overcrowding may make them susceptible to insects, although others have argued that the more vigorously growing trees are more likely to be attacked. Beetle infestation creates large quantities of fuel that increase the probability of wildfire (Bartolome 1988).


\paragraph{Succession Transition} Patches in this seral stage may stay in this seral stage under low mortality disturbance, but after 10 years without fire they begin transitioning to MDC at a rate of 0.8 per time step. Succession to LDM occurs once the patch has been in mid development for 45 years. The rate of succession per time step is 0.55. At 90 years, all stands will succeed to LDM.

\paragraph{Wildfire Transition} High mortality wildfire (15\% of fires in this seral stage) recycles the patch through the Early Development seral stage. Low mortality wildfire (85\%) triggers a transition to MDO 68\% of the time; otherwise, it remains in MDM.

\noindent\hrulefill

\paragraph{Mid Development - Closed Canopy Cover (MDC)}

\paragraph{Description} Sparse ground cover of grasses, forbs, and shrubs; mid-maturity \emph{P. contorta} ssp. \emph{murrayana} undergoing intrinsic stand thinning. Considerable surface fuel from tree mortality from previous fire. Canopy cover is greater than 50\% (LandFire 2007a).

Continued recruitment into stands produces overstocking and slow growth of the overcrowded trees. This overcrowding may make them susceptible to insects, although others have argued that the more vigorously growing trees are more likely to be attacked. Beetle infestation creates large quantities of fuel that increase the probability of wildfire. (Bartolome 1988).


\paragraph{Succession Transition} After 40 years in a MD seral stage without a wildfire-triggered transition, patches in this seral stage will begin transitioning to LDC. The rate of succession per time step is 0.6. At 80 years, all patches succeed to LDC.

\paragraph{Wildfire Transition} High mortality wildfire (41\% of fires in this seral stage) recycles the patch through the Early Development seral stage. Low mortality wildfire (59\%) triggers a transition to MDM.

\noindent\hrulefill


\paragraph{Late Development - Open Canopy Cover (LDO)}

\paragraph{Description} Areas that have experienced one or more low severity understory fires that had reduced stand density or old stands that have not experienced fire but have been thinned by other processes (tree falls, etc.). Stands are uneven aged. Canopy cover ranges from 10-50\% (LandFire 2007a).

\paragraph{Succession Transition} Patches in this seral stage may maintain under low mortality disturbance, but after 25 years without fire, these patches succeed to LDM at a rate of 0.7 per timestep.

\paragraph{Wildfire Transition} High mortality wildfire (7\% of fires in this seral stage) recycles the patch through the Early Development seral stage. Low mortality wildfire (93\%) maintains the patch in LDO.

\noindent\hrulefill

\paragraph{Late Development - Moderate Canopy Cover (LDM)}

\paragraph{Description} Sparse ground cover of grasses, forbs, and shrubs. Mid-maturity \emph{P. contorta} ssp. \emph{murrayana} where surface fire or other disturbance has opened the stand. Canopy cover ranges from 10-50\% (LandFire 2007a).

Continued recruitment into stands produces overstocking and slow growth of the overcrowded trees. This overcrowding may make them susceptible to insects, although others have argued that the more vigorously growing trees are more likely to be attacked. Beetle infestation creates large quantities of fuel that increase the probability of wildfire (Bartolome 1988).


\paragraph{Succession Transition} Patches in this seral stage may stay in this seral stage under low mortality disturbance, but after 25 years without fire they begin transitioning to LDC at a rate of 0.7 per time step. 

\paragraph{Wildfire Transition} High mortality wildfire (13\% of fires in this seral stage) recycles the patch through the Early Development seral stage. Low mortality wildfire (87\%) triggers a transition to LDO 73\% of the time; otherwise, it remains in LDM.

\noindent\hrulefill

\paragraph{Late Development - Closed Canopy Cover (LDC)}

\paragraph{Description} Old \emph{P. contorta} ssp. \emph{murrayana} stands where fire has had minimal influence. Canopy cover exceeds 50\%.

\paragraph{Succession Transition} This class will maintain in the absence of disturbance.

\paragraph{Wildfire Transition} High mortality wildfire (26\% of fires in this seral stage) recycles the patch through the Early Development seral stage. Low mortality wildfire (73\%) triggers a transition to LDM.

\noindent\hrulefill
\noindent\hrulefill

\subsubsection{Aspen Variant}

\paragraph{Early Development - Aspen (ED-A)}

\paragraph{Description} Grasses, forbs, low shrubs, and sparse to moderate cover of tree seedlings/saplings (primarily \emph{P. tremuloides}) with an open canopy. This seral stage is characterized by the recruitment of a new cohort of early successional, shade-intolerant tree species into an open area created by a stand-replacing disturbance. 

Following disturbance, succession proceeds rapidly from an herbaceous layer to shrubs and trees, which invade together (Verner 1988). \emph{P. tremuloides} suckers over 6ft tall develop within about 10 years (LandFire 2007c). 


\paragraph{Succession Transition} Unless it burns, a patch in the early seral stage persists for 10 years, at which point it transitions to MD-A.

\paragraph{Wildfire Transition} High mortality wildfire (100\% of fires in this seral stage) recycles the patch through the Early Development seral stage. No transition occurs as a result of low mortality fire.

\noindent\hrulefill


\paragraph{Mid Development - Aspen (MD-A)}

\paragraph{Description} \emph{P. tremuloides} trees 5-16'' DBH. Canopy cover is highly variable, and can range from 40-100\%. These patches range in age from 10 to 110 years. Some understory conifers, predominantly \emph{P. contorta} ssp. \emph{murrayana}, are encroaching, but \emph{P. tremuloides} is still the dominant component of the stand (LandFire 2007c).

\paragraph{Succession Transition} MD-A persists for at least 50 years in the absence of fire, after which patches begin transitioning to MD-AC at a rate of 0.6 per timestep. After 100 years all remaining MD-A patches transition to MD-AC. 

\paragraph{Wildfire Transition} High mortality wildfire (41\% of fires in this seral stage) recycles the patch through the Early Development seral stage. No transition occurs as a result of low mortality fire.

\noindent\hrulefill

\paragraph{Mid Development - Aspen with Conifer (MD-AC)}

\paragraph{Description} These stands have been protected from fire since the last stand-replacing disturbance. \emph{P. tremuloides} trees are predominantly 16'' DBH and greater. Conifers (predominantly \emph{P. contorta} ssp. \emph{murrayana}) are present and becoming increasingly dominant over the \emph{P. tremuloides}. Conifers are pole to medium-sized, and conifer cover is at least 40\% (LandFire 2007c).

\paragraph{Succession Transition} MD-AC persists for 100 years in the absence of high mortality fire, after which patches transition to LDC. 

\paragraph{Wildfire Transition} High mortality wildfire (15\% of fires in this seral stage) returns the patch to ED-A. Low mortality wildfire (85\%) maintains the patch in MD-AC.

\noindent\hrulefill

\paragraph{Late Development - Closed (LDC)}

\paragraph{Description} Some \emph{P. tremuloides} continue to be present in the understory, but large\emph{ P. contorta} ssp. \emph{murrayana} are now the dominant tree species, having overtopped the \emph{P. tremuloides}. Smaller conifers are present in the midstory as well (LandFire 2007a). This seral stage is analogous to the LDC seral stage for the LPN variant.

\paragraph{Succession Transition} Patches in this seral stage will maintain in the absence of disturbance.

\paragraph{Wildfire Transition} High mortality wildfire (26\% of fires in this seral stage) will return the patch to ED-A. Low mortality wildfire (74\%) opens the stand up to LD-CA.

\noindent\hrulefill


\paragraph{Late Development - Conifer with Aspen (LD-CA)}

\paragraph{Description} If stands are sufficiently protected from fire such that conifer species overtop \emph{P. tremuloides} and become large, they may be able to withstand some fire that more sensitive \emph{P. tremuloides} cannot. When this occurs, it creates a patch characterized by late development conifers, such as \emph{P. contorta} ssp. \emph{murrayana}, and early seral \emph{P. tremuloides}. 

\paragraph{Succession Transition} LD-CA persists for 70 years in the absence of any fire, at which point patches transition to LDC. 

\paragraph{Wildfire Transition} High mortality wildfire (13\% of fires in this seral stage) returns the patch to ED-A. Low mortality wildfire (87\%) maintains the stand in LD-CA. 

\noindent\hrulefill

\newpage
\subsection*{Condition Classification}

\begin{table}[hbp]
\small
\centering
\caption{Classification of seral stage for LPN. Diameter at Breast Height (DBH) and Cover From Above (CFA) values taken from EVeg polygons. DBH categories are: null, 0-0.9'', 1-4.9'', 5-9.9'', 10-19.9'', 20-29.9'', 30''+. CFA categories are null, 0-10\%, 10-20\%, \dots , 90-100\%. Each row in the table below should be read with a boolean AND across each column.}
\label{lpn_classification}
\begin{tabular}{@{}lrrrrr@{}}
\toprule
\textbf{\begin{tabular}[l]{@{}l@{}}Cover \\ Condition\end{tabular}} & \textbf{\begin{tabular}[r]{@{}r@{}}Overstory Tree \\ Diameter 1 \\ (DBH)\end{tabular}} & \textbf{\begin{tabular}[r]{@{}r@{}}Overstory Tree \\ Diameter 2 \\ (DBH)\end{tabular}} & \textbf{\begin{tabular}[r]{@{}r@{}}Total Tree\\ CFA (\%)\end{tabular}} & \textbf{\begin{tabular}[r]{@{}r@{}}Conifer \\ CFA (\%)\end{tabular}} & \textbf{\begin{tabular}[r]{@{}r@{}}Hardwood \\ CFA (\%)\end{tabular}} \\ \midrule
Early All & 0-4.9'' & any & any & any & any \\
Mid Open & 5-9.9'' & any & 0-40 & any & any \\
Mid Moderate & 5-9.9'' & any & 40-70 & any & any \\
Mid Closed & 5-9.9'' & any & 70-100 & any & any \\
Late Open & 10''+ & any & 0-40 & any & any \\
Late Moderate & 10''+ & any & 40-70 & any & any \\
Late Closed & 10''+ & any & 70-100 & any & any \\ \bottomrule
\end{tabular}
\end{table}

LPN-ASP seral stages were assigned manually using NAIP 2010 Color IR imagery to assess seral stage.



\clearpage
\subsection*{References}
\begin{hangparas}{.25in}{1} 
Bartolome, James W. ``Lodgepole Pine (LPN).'' \emph{A Guide to Wildlife Habitats of California}, edited by Mayer, Kenneth E. and William F. Laudenslayer. California Deparment of Fish and Game. 1988. \burl{http://www.dfg.ca.gov/biogeodata/cwhr/pdfs/LPN.pdf}. Accessed 4 December 2012.

``CalVeg Zone 1.'' Vegetation Descriptions. \emph{Vegetation Classification and Mapping}.  11 December 2008. U.S. Forest Service. \burl{http://www.fs.usda.gov/Internet/FSE\_DOCUMENTS/fsbdev3\_046448.pdf}. Accessed 2 April 2013.

Cope, Amy B. 1993. ``Pinus contorta var. murrayana.'' In: Fire Effects Information System, [Online].  U.S. Department of Agriculture, Forest Service,  Rocky Mountain Research Station, Fire Sciences Laboratory (Producer).  \burl{http://www.fs.fed.us/database/feis/} [Accessed 4 December 2012].

Fites-Kaufman, Jo Ann, Phil Rundel, Nathan Stephenson, and Dave A. Wixelman. ``Montane and Subalpine Vegetation of the Sierra Nevada and Cascade Ranges.'' In \emph{Terrestrial Vegetation of California, 3rd Edition}, edited by Michael Barbour, Todd Keeler-Wolf, and Allan A. Schoenherr, 456-501. Berkeley and Los Angeles: University of California Press, 2007. 

Gross, Shana. Ecologist, USDA Forest Service. Personal communication, 3 July 2013.

Lotan, James E. and William B. Critchfield. ``Lodgepole Pine.'' Russell M. Burns and Barbara H. Honkala, tech. coords. Silvics of North America, vol 1. Conifers; Glossary. Agriculture handbook no.654. Washington, D.C.: U.S. Dept. of Agriculture, Forest Service, 1990. 

LandFire. ``Biophysical Setting Models.'' Biophysical Setting 0610581: Sierra Nevada Subalpine Lodgepole Pine Forest and Woodland. 2007a. LANDFIRE Project, U.S. Department of Agriculture, Forest Service; U.S. Department of the Interior. \burl{http://www.landfire.gov/national\_veg\_models\_op2.php}. Accessed 9 November 2012.

LandFire. ``Biophysical Setting Models.'' Biophysical Setting 0610582: Sierra Nevada Subalpine Lodgepole Pine Forest and Woodland. 2007b. LANDFIRE Project, U.S. Department of Agriculture, Forest Service; U.S. Department of the Interior. \burl{http://www.landfire.gov/national\_veg\_models\_op2.php}. Accessed 9 November 2012.

LandFire. ``Biophysical Setting Models.'' Biophysical Setting 0610610: Inter-Mountain Basins Aspen-Mixed Conifer Forest and Woodland. 2007c. LANDFIRE Project, U.S. Department of Agriculture, Forest Service; U.S. Department of the Interior. \burl{http://www.landfire.gov/national\_veg\_models\_op2.php}. Accessed 7 January 2013.

Safford, Hugh S. Regional Ecologist, USDA Forest Service. Personal communication, 5 May 2013.

Skinner, Carl N. and Chi-Ru Chang. ``Fire Regimes, Past and Present.'' \emph{Sierra Nevada Ecosystem Project: Final report to Congress, vol. II, Assessments and scientific basis for management options}. Davis: University of California, Centers for Water and Wildland Resources, 1996.

Van de Water, Kip M. and Hugh D. Safford. ``A Summary of Fire Frequency Estimates for California Vegetation Before Euro-American Settlement.'' \emph{Fire Ecology} 7.3 (2011): 26-57. doi: 10.4996/fireecology.0703026.

Verner, Jared. ``Aspen (ASP).'' \emph{A Guide to Wildlife Habitats of California}, edited by Kenneth E. Mayer and William F. Laudenslayer. California Deparment of Fish and Game, 1988. \burl{http://www.dfg.ca.gov/biogeodata/cwhr/pdfs/ASP.pdf}. Accessed 4 December 2012.

\end{hangparas}

