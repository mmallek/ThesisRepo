% !TEX root = master.tex
\newpage
\section{Western White Pine (WWP)}

\subsection*{General Information}

\subsubsection{Cover Type Overview}

\textbf{Western White Pine (WWP)}
\newline
Crosswalks
\begin{itemize}
	\item EVeg: Regional Dominance Type 1
	\begin{itemize}
		\item Western White Pine 
	\end{itemize}

	\item LandFire BpS Model
	\begin{itemize}
		\item 0711720: Sierran-Intermontane Desert Western White Pine-White Fir Woodland
	\end{itemize}

	\item Presettlement Fire Regime Type
	\begin{itemize}
		\item Western White Pine
	\end{itemize}
\end{itemize}

\noindent Reviewed by Becky Estes, Central Sierra Province Ecologist, USDA Forest Service

\subsubsection{Vegetation Description}
\emph{Pinus monticola} is locally abundant in subalpine habitats along the west slope of the Sierra Nevada, where it may occur in small pure stands. More commonly, it mixes with \emph{Pinus contorta} ssp. \emph{murrayana, Pinus jeffreyi, Tsuga mertensiana}, and \emph{Abies magnifica} (particularly on the west side of the Sierra crest) and \emph{Abies concolor} or \emph{Pinus ponderosa} (particularly on the east side) (Fites-Kaufman et al. 2007, LandFire 2007, Estes pers. comm. 2013).

This system tends to be more woodland than forest in character, and the undergrowth is more open and drier, with little shrub or herbaceous cover. Tree regeneration is less prolific than in other mixed-montane conifer systems of the Cascades, Sierras and California Coast Ranges (LandFire 2007). \emph{P. monticola} generally maintains a tree form of growth up nearly to treeline, where it is commonly replaced by other subalpine species on rocky ridges (Fites-Kaufman et al. 2007).

Understories are typically open, with moderately low shrub cover and diversity, and include \emph{Arctostaphylos, Chrysolepis, Ceanothus}, and \emph{Ribes}. Common herbaceous taxa include \emph{Arnica, Festuca, Poa, Carex, Pyrola}, and \emph{Hieracium}. In openings, \emph{Wyethia} can be abundant (LandFire 2007).


\subsubsection{Distribution}
With respect to the focal landscape within the northern Sierra Nevada, these forests and woodlands are found in the upper montane to subalpine zones, at elevations generally over 2000 m (6560 ft). 

It is found on all slopes and aspects, although it occurs more frequently on drier areas. This ecological system generally occurs on basalts, andesite, glacial till, basaltic rubble, colluvium, or volcanic ash-derived soils. These soils have characteristic features of good aeration and drainage, coarse textures, circumneutral to slightly acidic pH, an abundance of mineral material, rockiness, and periods of drought during the growing season. Climatically, this system occurs somewhat in the rain shadow of the Sierras and has a more continental regime, similar to the northern Great Basin (LandFire 2007).


\subsection*{Disturbances}
Most fires in this type are low mortality fires that allow large areas of the landscape to develop mature characteristics. Occasional severe fires are driven by weather extremes (LandFire 2007). Young trees are very susceptible to mortality from fire, but mature \emph{P. monticola} is moderately fire resistant. After a stand-replacing fire, \emph{P. monticola} will seed in from adjacent areas. After a cool to moderate fire that leaves a mosaic of mineral soil and duff, it will reoccupy the site from seed stored in the seed bank. Overall, \emph{P. monticola} is a fire-dependent, seral species. Fire suppression has resulted in decreased stocking levels, mostly due to the increase in White pine blister rust (\emph{Cronartium ribicola}). Periodic, stand-replacing fire or other disturbance is needed to remove competing conifers and allow \emph{P. monticola} to develop (Griffith 1992). 

Estimates of fire rotations for these variants are available from the LandFire project and a few review papers. The LandFire project’s published fire return intervals are based on a series of associated models created using the Vegetation Dynamics Development Tool (VDDT). In VDDT, fires are specified concurrently with the transition that follows them. For example, a replacement fire causes a transition to the early development stage. In the RMLands model, such fires are classified as high mortality. However, in VDDT mixed severity fires may cause a transition to early development, a transition to a more open seral stage, or no transition at all. In this case, we categorize the first example as a high mortality fire, and the second and third examples as a low mortality fire. Based on this approach, we calculated fire rotations and the probability of high mortality fire for each of the WWP and WWP-ASP seral stages (Table~\ref{tab:wwpdesc_fire}). We computed overall target fire rotations based on values from Van de Water and Safford (2011). 

\subsubsection{Wildfire}




\begin{table}[]
\small
\centering
\caption{Fire rotation (years) and proportion of high (versus low) mortality fires for Lodgepole Pine type. Values were derived from VDDT model 0711720 (LandFire 2007) and Van de Water and Safford (2011). }
\label{tab:wwpdesc_fire}
\begin{tabular}{@{}lcc@{}}
\toprule
\textbf{Condition}         & \multicolumn{1}{l}{\textbf{Fire Rotation}} & \multicolumn{1}{l}{\textbf{\begin{tabular}[c]{@{}l@{}}Proportion \\ High Mortality\end{tabular}}} \\ \midrule
Target                      & 88            & n/a                           \\
Early Development - All     & 33            & 0.17                          \\
Mid Development - Closed    & 33            & 0.18                          \\
Mid Development - Moderate  & 24            & 0.12                          \\
Mid Development - Open      & 18            & 0.09                          \\
Late Development - Closed   & 33            & 0.17                          \\
Late Development - Moderate & 24            & 0.12                          \\
Late Development - Open     & 18            & 0.09 					      \\ \bottomrule
\end{tabular}
\end{table}

\subsubsection{Other Disturbance}
Other disturbances are not currently modeled, but may, depending on the seral stage affected and mortality levels, reset patches to early development, maintain existing seral stages, or shift/accelerate succession to a more open seral stage. 

\subsection*{Vegetation Seral Stages}
We recognize seven separate seral stages for WWP: Early Development (ED), Mid Development - Open Canopy Cover (MDO), Mid Development - Moderate Canopy Cover, Mid Development - Closed Canopy Cover (MDC), Late Development - Open Canopy Cover (LDO), Late Development - Moderate Canopy Cover (LDM), and Late Development - Closed Canopy Cover (LDC) (Figure~\ref{transmodel_wwp}). Our seral stages are an alternative to ``successional'' classes that imply a linear progression of states and tend not to incorporate disturbance. The seral stages identified here are derived from a combination of successional processes and anthropogenic and natural disturbance, and are intended to represent a composition and structural condition that can be arrived at from multiple other conditions described for that landcover type. Thus our seral stages incorporate age, size, canopy cover, and vegetation composition. In general, the delineation of stages has originated from the LandFire biophysical setting model descriptive of a given landcover type; however, seral stages are not necessarily identical to the classes identified in those models.


\begin{figure}[hbp]
\centering
\includegraphics[width=0.8\textwidth]{/Users/mmallek/Documents/Thesis/statetransmodel/StateTransitionModel/7class.png}
\caption{State and Transition Model for Western White Pine Forest. Each dark grey box represents one of the seven seral stages for this landcover type. Each column of boxes represents a stage of development: early, middle, and late. Each row of boxes represents a different level of canopy cover: closed (70-100\%), moderate (40-70\%), and open (0-40\%). Transitions between states/seral stages may occur as a result of high mortality fire, low mortality fire, or succession. Specific pathways for each are denoted by the appropriate color line and arrow: red lines relate to high mortality fire, orange lines relate to low mortality fire, and green lines relate to natural succession.} 
\label{transmodel_wwp}
\end{figure}

\paragraph{Early Development (ED)}

\paragraph{Description} Open stand of \emph{P. monticola}, \emph{A. magnifica}, as well as other tree seedlings mixed with grasses and shrubs. Early seral dominant species include Ceanothus and various grasses. A portion of these stands get into a shrub dominated stage that can persist for for a few decades (LandFire 2007, Estes 2013).

\paragraph{Succession Transition} In the absence of disturbance, patches in this seral stage will begin transitioning to an mid development seral stage after 30 years at a rate of 0.7 per time step. At 70 years, all remaining patches will succeed. The secondary rate of succession to MDO is 0.8, and to MDC is 0.2. On average, patches remain in early development for 43 years.

\paragraph{Wildfire Transition} High mortality wildfire (17\% of fires in this seral stage) recycles the patch through the Early Development seral stage. No transition occurs as a result of low mortality fire.

\noindent\hrulefill


\paragraph{Mid Development - Open Canopy Cover (MDO)}

\paragraph{Description} Open stand of early seral tree species. Heterogeneous ground cover of grasses, forbs, and shrubs. Trees present are pole to medium sized conifers with canopy cover less than 40\%. Conifer species likely present include \emph{P. monticola}, \emph{A. magnifica}, and \emph{P. jeffreyi} (LandFire 2007, Estes 2013).

\paragraph{Succession Transition} Patches in this seral stage will maintain under low mortality disturbance, but after 15 years without fire they begin transitioning to MDM at a rate of 0.8 per time step. Succession to LDO occurs once the patch has been in mid development for 70 years. The rate of succession per time step is 0.4. At 120 years, all remaining patches succeed. On average, patches remain in early development for 83 years.

\paragraph{Wildfire Transition} High mortality wildfire (9\% of fires in this seral stage) recycles the patch through the ED seral stage. Low mortality wildfire (91\%) maintains the patch in MDO.

\noindent\hrulefill

\paragraph{Mid Development - Moderate Canopy Cover (MDM)}

\paragraph{Description} Sparse ground cover of grasses, forbs, and shrubs; moderate to dense cover of trees. Conifers are pole to medium-sized, with canopy cover ranging from 40-70\%. Conifer species likely present include \emph{P. monticola}, \emph{A. magnifica}, and \emph{P. jeffreyi} (LandFire 2007, Estes 2013).

\paragraph{Succession Transition} Patches in this seral stage will maintain under low mortality disturbance, but after 15 years without fire they begin transitioning to MDC at a rate of 0.8 per time step. Succession to LDM occurs once the patch has been in mid development for 70 years. The rate of succession per time step is 0.4. At 120 years, all remaining patches succeed. On average, patches remain in early development for 83 years.

\paragraph{Wildfire Transition} High mortality wildfire (12\% of fires in this seral stage) recycles the patch through the ED seral stage. Low mortality wildfire (88\%) opens the patch up to MDO 40\% of the time; otherwise, the patch remains in MDM.

\noindent\hrulefill

\paragraph{Mid Development - Closed Canopy Cover (MDC)}

\paragraph{Description} Sparse ground cover of grasses, forbs, and shrubs; moderate to dense cover of trees. Conifers are pole to medium-sized, with canopy cover over 70\%. Forests of this type rarely, if ever, exceed 80\% canopy closure even in closed, dense seral stages. Conifer species likely present include \emph{P. monticola}, \emph{A. magnifica}, and \emph{P. jeffreyi} (LandFire 2007, Estes 2013).

\paragraph{Succession Transition} Succession to LDC occurs once the patch has been in mid development for 70 years. The rate of succession per time step is 0.4. At 120 years, all remaining patches succeed. On average, patches remain in early development for 83 years.

\paragraph{Wildfire Transition} High mortality wildfire (18\% of fires in this seral stage) recycles the patch through the Early Development seral stage. Low mortality wildfire (82\%) opens the patch up to MDM 80\% of the time; otherwise, the patch remains in MDC.

\noindent\hrulefill


\paragraph{Late Development - Open Canopy Cover (LDO)}

\paragraph{Description} Open stands of large trees, primarily \emph{P. monticola}, \emph{A. magnifica}, and \emph{P. jeffreyi}. Canopy cover is less than 40\% (LandFire 2007, Estes 2013).

\paragraph{Succession Transition} Patches in this seral stage will maintain under low mortality disturbance, but after 15 years without fire, these patches succeed to LDM at a rate of 0.8 per timestep.

\paragraph{Wildfire Transition} High mortality wildfire (9\% of fires in this seral stage) recycles the patch through the Early Development seral stage. Low mortality wildfire (91\%) maintains the patch in LDO.

\noindent\hrulefill

\paragraph{Late Development - Moderate Canopy Cover (LDM)}

\paragraph{Description} Closed stands of large trees, primarily \emph{P. monticola}, \emph{A. magnifica}, and \emph{P. jeffreyi}. Forests in this landcover type rarely exceed 80\% canopy closure even in closed, dense seral stages. Canopy cover exceeds 70\% (LandFire 2007, Estes 2013).

\paragraph{Succession Transition} Patches in this seral stage will maintain under low mortality disturbance, but after 15 years without fire, these patches succeed to LDC at a rate of 0.8 per timestep.

\paragraph{Wildfire Transition} High mortality wildfire (12\% of fires in this seral stage) recycles the patch through the ED seral stage. Low mortality wildfire (82\%) opens the patch up to LDO 40\% of the time; otherwise, the patch remains in LDC.

\noindent\hrulefill

\paragraph{Late Development - Closed Canopy Cover (LDC)}

\paragraph{Description} Closed stands of large trees, primarily \emph{P. monticola}, \emph{A. magnifica}, and \emph{P. jeffreyi}. Forests in this landcover type rarely exceed 80\% canopy closure even in closed, dense conditions. Canopy cover exceeds 40\% (LandFire 2007, Estes 2013).

\paragraph{Succession Transition} Patches in this seral stage will maintain in the absence of disturbance.

\paragraph{Wildfire Transition} High mortality wildfire (17\% of fires in this seral stage) recycles the patch through the ED seral stage. Low mortality wildfire (83\%) opens the patch up to LDM 80\% of the time; otherwise, the patch remains in LDC.

\noindent\hrulefill




\subsection*{Seral Stage Classification}
\begin{table}[hbp]
\small
\centering
\caption{Classification of seral stage for WWP. Diameter at Breast Height (DBH) and Cover From Above (CFA) values taken from EVeg polygons. DBH categories are: null, 0-0.9'', 1-4.9'', 5-9.9'', 10-19.9'', 20-29.9'', 30''+. CFA categories are null, 0-10\%, 10-20\%, \dots , 90-100\%. Each row in the table below should be read with a boolean AND across each column.}
\label{wwp_classification}
\begin{tabular}{@{}lrrrrr@{}}
\toprule
\textbf{\begin{tabular}[l]{@{}l@{}}Cover \\ Condition\end{tabular}} & \textbf{\begin{tabular}[r]{@{}r@{}}Overstory Tree \\ Diameter 1 \\ (DBH)\end{tabular}} & \textbf{\begin{tabular}[r]{@{}r@{}}Overstory Tree \\ Diameter 2 \\ (DBH)\end{tabular}} & \textbf{\begin{tabular}[r]{@{}r@{}}Total Tree\\ CFA (\%)\end{tabular}} & \textbf{\begin{tabular}[r]{@{}r@{}}Conifer \\ CFA (\%)\end{tabular}} & \textbf{\begin{tabular}[r]{@{}r@{}}Hardwood \\ CFA (\%)\end{tabular}} \\ \midrule
Early All        & 0-4.9''         & any & any    & any    & any \\
Mid Open         & 5-19.9''        & any & 0-40   & any    & any \\
Mid Moderate     & 5-19.9''        & any & 40-70  & any    & any \\
Mid Closed       & 5-19.9''        & any & null   & 70-100 & any \\
Mid Closed       & 5-19.9''        & any & 70-100 & any    & any \\
Late Open        & 20''+           & any & 0-40   & any    & any \\
Late Moderate    & 20''+           & any & 40-70  & any    & any \\
Late Closed      & 20''+           & any & 70-100 & any    & any \\ \bottomrule
\end{tabular}
\end{table}



\clearpage

\subsection*{References}
\begin{hangparas}{.25in}{1} 
Estes, Becky. Central Sierra Province Ecologist, USDA Forest Service. Personal communication, 3 September 2013.

Griffith, Randy Scott. 1993. ``Pinus monticola.'' In: \emph{Fire Effects Information System}, [Online].  U.S. Department of Agriculture, Forest Service,  Rocky Mountain Research Station, Fire Sciences Laboratory (Producer).  \burl{http://www.fs.fed.us/database/feis/}. [Accessed 4 December 2012].

Fites-Kaufman, Jo Ann, Phil Rundel, Nathan Stephenson, and Dave A. Wixelman. ``Montane and Subalpine Vegetation of the Sierra Nevada and Cascade Ranges.'' In \emph{Terrestrial Vegetation of California, 3rd Edition}, edited by Michael Barbour, Todd Keeler-Wolf, and Allan A. Schoenherr, 456-501. Berkeley and Los Angeles: University of California Press, 2007. 

LandFire. ``Biophysical Setting Models.'' Biophysical Setting 0711720: Sierran-Intermontane Desert Western White Pine-White Fir Woodland. 2007. LANDFIRE Project, U.S. Department of Agriculture, Forest Service; U.S. Department of the Interior. \burl{http://www.landfire.gov/national_veg_models_op2.php}. Accessed 30 November 2012.

Skinner, Carl N. and Chi-Ru Chang. ``Fire Regimes, Past and Present.'' \emph{Sierra Nevada Ecosystem Project: Final report to Congress, vol. II, Assessments and scientific basis for management options}. Davis: University of California, Centers for Water and Wildland Resources, 1996.

Van de Water, Kip M. and Hugh D. Safford. ``A Summary of Fire Frequency Estimates for California Vegetation Before Euro-American Settlement.'' \emph{Fire Ecology} 7.3 (2011): 26-57. doi: 10.4996/fireecology.0703026.

\end{hangparas}

