% !TEX root = master.tex

\section{Methods}
\label{sec:hrvmethods}

% removed study area, it's now in the Introduction Chapter

\subsection{Modeling Framework}
\label{sec:modelframe}

A partial introduction to \textsc{RMLands} is included in Chapter~\ref{CH1}, but here I provide a more detailed description of the model and how I used it. 
% intro to RMLands in Introduction Chapter (modeling framework, methodological limitations)

\subsubsection*{Input Layers}
\label{subsec:hrvinputlayers}

All input layers to \textsc{RMLands} must be custom-built to work with the software. For technical details on the data structure requirements of \textsc{RMLands}, see Appendix \ref{app:inputs}. A brief overview of each input layer is included below.

\paragraph*{Cover} Cover type is based on the potential or current natural vegetation of a site and includes both natural and anthropogenic cover types. For example, cover types include not only Lodgepole Pine, Sierran Mixed Conifer, and Red Fir, but also Barren and Agriculture. Succession pathways are defined uniquely for each cover type, susceptibility to natural disturbances varies among cover types, and suitability or eligibility for various vegetation treatments varies among cover types. Cover is a static (constant) grid and therefore provides a fixed template upon which disturbance and succession processes play out over time. 

The source for the cover layer is the Region 5 Existing Vegetation Layer (``EVeg''), first mapped to the \textsc{calveg} classification developed by the Region's Ecology Program in 1978. When deciding on land cover types, including determining xeric and mesic subtypes, our focus was to best represent the study area and the surrounding landscape. I used the \textsc{calveg} Mapping Zone boundary for the ``North Sierra'' (Figure~\ref{calveg}) as our focus for defining vegetation and disturbance, including susceptibility, response to fire, and fire size and distribution. Within the study area, the EVeg layer was developed based on three separate efforts: a satellite-based imagery analyses in 2000, and two orthoimagery analysis completed by contracting firms in 2005. Generally, specific cover type names were derived from the California Fire Return Interval Departure (FRID) report by \citet{VandeWater2011}. I also considered information from \emph{A Guide to Wildlife Habitats of California}, popularly known as the ``Wildlife Habitat Relationship (WHR)'' cover types \citep{WHR1988}. 

\begin{wrapfigure}{R}{0.5\textwidth} % use a capital R to allow figure to float
\includegraphics[width=0.5\textwidth]{/Users/mmallek/Tahoe/Report3/images/CALVEGmappingzones.png}
\caption{\small CALVEG Mapping Zones. These zones meet U.S. Forest Service standard at national and regional levels. These ecological provinces are associated with dozens of vegetation alliances, which are used to classify vegetation in spatial data products. I used vegetation alliance definitions for the North Sierra zone to classify the land cover spatial data shared by the U.S. Forest Service.} 
\label{calveg}
\end{wrapfigure}

\subparagraph*{Alternative Cover Layers}
The original intent of our team was to utilize two separate cover layers: one for the historical reference period, and one for the current period to be used in projections of future scenarios. Two layers were identified as potentially suitable for the historic analysis: a map created from forest survey and inventory efforts under Albert Wieslander conducted between 1928 and 1940 (``Wieslander'') \citep{Thorne2006}, and a map of Potential Natural Vegetation (PNV) created by a Forest Service Enterprise Team for the Tahoe National Forest in the 2000s (Forest Service internal GIS data). Our intent was to use the PNV, Wieslander, or a combination thereof to derive the land cover layer for the HRV phase of the project. 

In order to validate the historical maps, I needed to develop a crosswalk between the vegetation type methodologies for the EVeg, PNV, and Wieslander maps. I also examined the spatial consistency in cover types across the maps. With significant assistance from the Tahoe National Forest, I attempted to create a crosswalk from these maps to the set of land cover types to be used in the project. However, I was unable to develop a consistent and comprehensive set of rules for this purpose. A major reason for this is that both the PNV and Wieslander maps used species lists, rather than assemblages (as in \textsc{calveg} and LandFire). For example, Sierran mixed conifer forests do not appear as a dominant ``cover type'' in the PNV map. The Wieslander maps may contain an internal crosswalk to a mixed conifer alliance, but in my study area I observed this to be true only rarely. 

In addition, the PNV map contained a more significant error: I learned that, for the purposes of the modeling used to create the PNV map, ``potential natural vegetation'' was defined as the so-called ``climax'' community that would develop in the complete absence of all disturbance, including natural disturbances like wildfire \citep{Fites1993}. Since my goal was to mimic the natural historic range of variability, I decided to discard this layer. The Wieslander map had its own issues. Most problematic was the non-systematic spatial error of up to 300 meters, which meant it would not be suitable for comparing specific locations \citep{Thorne2006}. In addition, crosswalking precisely was impossible because coded vegetation was not necessarily in order of most prevalent vegetation, but instead prioritized tree species over shrubs, and commercially important trees over others. As an example from the handbook states, a plot consisting of 75\% \emph{Quercus kelloggi} (black oak), 15\% \emph{Pinus ponderosa} (ponderosa pine), and 10\% \emph{Pinus lambertiana} (grey pine) would be coded as ponderosa pine, grey pine, black oak \citep{Thorne2006}. Finally, the Wieslander maps were developed from surveys done in the 1930s, decades after the huge influx of settlers in the 1850s; by the 1930s, vegetation patterns may have already been significantly altered \citep{Thorne2006}. Consequently, the Wieslander map is also not a reliable predictor of land cover type without extensive review of the original data and maps, which would be beyond the scope of this project. 

To further confirm these issues, I examined the overlap in land cover types between different maps in ArcGIS. In general, the overlap between EVeg and either the PNV or the Wieslander layers was no better than random, and in many cases it was worse. I decided, in conjunction with Tahoe National Forest staff, to proceed using only the EVeg map. The calibration period of the model was omitted from the HRV analysis in order to reduce or eliminate the influence of the current landscape characteristics on the results.



%This ensured that our analysis of future management scenarios and comparison of spatial metrics between those results and the HRV results was credible.

% in retrospect I wonder if we should have analyzed the configuration more. in the end the biggest problem was probably the lack of crosswalk, since a precise spatial equivalence wasn't assumed.

\subparagraph*{Selection of Specific Cover Types}
In the early stages of this project, the team created a suite of land cover types based roughly on the Wildlife Habitat Relationships (WHR) types used in California and by Forest Service managers and planners. These consisted of the WHR types with a few additional types where additional specificity or refinement was desired. For example, Red Fir was split up into two subtypes. The original concept was to begin with the WHR types and modify them as needed based on other attributes in the EVeg layer. However, creating a crosswalk from WHR to the project-specific types also proved problematic. First, I realized that the WHR values were actually derived from the \textsc{calveg} species alliances included in the EVeg layer, but the methodology used was unavailable or missing. The crosswalks I did find were not mutually exclusive and all-inclusive, and do not always make ecological sense \citep{Keeler-Wolf2007,DeBecker1988,Game2005}. This is probably due in part to the fact that WHR is not a mapping classification. It is always derived secondarily. So, I was unable to create consistent rules for mapping from WHR to other types. In addition, the WHR types are explicitly used to map current vegetation in a way that is relevant to wildlife biologists. Compared to succession processes and susceptibility and response to fire, which were my focus in this project, the WHR classification simply has a very different purpose that rendered it less useful for my needs. Others have encountered similar issues:
%
\begin{quote}
WHR has been less successful in differentiating between vegetation types. Because the habitat types are \emph{inconsistently defined}, a broad familiarity with its detailed descriptions is needed to differentiate among types of similar structure. Although mappers have constructed rules for discriminating among types, difficulties still remain because \emph{species dominance varies substantially within some types and broad overlaps in dominant plants occur among types}. Other problems arise due to the small number of classes and the \emph{inconsistencies in scale} among them \citep[23, emphasis added]{Keeler-Wolf2007}
\end{quote}
%
In collaboration with National Forest staff I decided to instead base our land cover types on, at the first order, the Presettlement Fire Regime (PFR) types as defined in the Fire Return Interval Departure (FRID) report by \citet{VandeWater2011}. The PFR types, as part of the FRID, were developed through a combination of literature and expert workshops. Peer review was solicited during these workshops, and the framework was then subjected to additional review via the academic publication process. The PFR was also useful for this project specifically because it grouped vegetation types based on their relationship to wildfire, which is the disturbance type simulated in this study \citep{VandeWater2011}. Using the FRID methodology provided an opportunity to avoid using the second-order WHR classification and trying to reverse-engineer it to fit into my custom land cover types. 

Thus I created a new structure of cover types in a nested regime. At the coarsest level are the PFR types, created by aggregating \textsc{calveg} as described above. Some of these are then subdivided using the Biophysical Settings from LandFire. Finally, a few types are further refined, ultimately generating a set of land cover types specific to the Yuba River Watershed, but applicable to the northern Sierra Nevada in general. A mutually exclusive and all-inclusive crosswalk for each land cover type used in this analysis to a single LandFire Biophysical Setting and Presettlement Fire Regime type thus exists. Appendix~\ref{app:covertypedesc} includes specifics on how this applies to individual cover types..

I used Python scripts and ArcGIS to conduct the geoprocessing necessary to prepare the EVeg layer for use in \textsc{RMLands}. All processing was done after converting shapefiles (vector data) to the raster format. Land cover types were differentiated based on spatial location, presence of aspen stands, presence of ultramaifc soils, and position along a xeric to mesic gradient. 

First, I created Aspen variants of forested land cover types by overlaying an aspen layer onto the vegetation layer and using ArcGIS tools and Python scripts to create combined types (``[type] - Aspen'') where appropriate. Second, areas mapped as a vegetation type characteristic of early successional forest (e.g. chaparral) were remapped using ArcGIS tools and Python scripts to an appropriate forest cover type, based on the land cover types in the area immediately adjacent to the patch determined to be in an early successional stage. Next, vegetation data and elevation data were analyzed together to distinguish east- and west-side (of the Sierran crest) areas from one another. This information was used to appropriately identify land cover types that, in my application, are mapped only on the east-side. Yellow Pine and stable Aspen variants of forested land cover types were confined to the east-side in my application. Ultramafic\footnote{Ultramafic soils are those created from the weathering of igneous rocks, brought to the earth's surface as magma, where they then cooled. Ultramafic soils are typically shallow, rocky, and nutrient deficient, with high levels of metals uncommon in other soils. Only a few species of plants have evolved to live on them, many of which are endemic to such soils. Plants that do grow mature more slowly and cover the land less continuously than the same plant would on better soil. In the study area, the most common ultramafic rock is serpentine \citep{Safford2004}.} land cover types were mapped by overlaying a geology layer obtained from the Tahoe National Forest (1:100,000 scale) onto the vegetation layer and using ArGIS tools and Python scripts to create ``[type] - Ultramafic''. 

Next, the Sierran Mixed Conifer, Red Fir, and Mixed Evergreen cover types, which cover broad swaths of land across elevation and aspect, were subclassified into either a mesic or xeric variant. When present, aspen or ultramafic soils supersede xeric-mesic classification. Although the WHR classification system does not divide, for example, Sierran Mixed Conifer, into xeric or mesic types, other classification systems often do. In some cases this division is recognized at the PFR level (e.g., Sierran Mixed Conifer), while in others the refinement occurs at the Biophysical Setting level (e.g., Red Fir). The PFR method does crosswalk directly from \textsc{calveg} assignments, so for certain cases it would be possible to simply use this classification strategy. However, PFRs are based on existing vegetation only and do not incorporate abiotic factors \citep{Safford2014}. In addition, when I showed a map based on the PFR classification to local experts, they felt that the distribution of xeric versus mesic types did not accurately represent the study area. Consequently, I explored some biophysical indicators related to moisture that could be used to designate and separate the mesic subtype from the xeric subtype.

Then, together with the team, I chose four metrics to comprise the mesic-xeric index. All metrics consist of modeled values. Climatic water deficit (CWD) is the annual evaporative demand that exceeds available water, measured annually in the summer. It is derived by subtracting actual evapotranspiration from potential evapotranspiration. The second metric, the topographic wetness index (TWI), measures topographic moisture. It is a function of slope and the catchment area of a particular point. Soil water storage (STOR) is the average amount of water stored in the soil annually. It is derived from precipitation, snowmelt rates, actual evapotranspiration, groundwater recharge rates, and surface water runoff rates. The final metric is the result of precipitation minus potential evapotranspiration (PPET), a measure of climatic moisture.

These variables were standardized by z-score such that higher values correspond to more mesic environments. Thus, potential evapotranspiration was inverted for this purpose. The mean for each metric is zero and the units are in terms of standard deviation. To combine the metrics, I combined the z-score value raster grids with equal weights. In conjunction with local experts, a break point in the resulting xeric-mesic gradient was selected and then applied using the ArcGIS tool Raster Calculator, creating ``[type] - Mesic'' and ``[type] - Xeric''. For the Sierran Mixed Conifer and Red Fir cover types, index values from the negative end of the range up to $-1/4$ standard deviations below the mean (zero) were used to create xeric variants, while the remaining portion of the spectrum was used to designate the mesic variants. For the Mixed Evergreen cover type, the break point along the gradient was $-1/2$ standard deviations below the mean. 


Ultimately, 31 cover types were generated for the buffered study area, as listed in Table~\ref{covertable} and shown in Figure~\ref{fig:covermap}.\footnote{Larger images of all of the input layers are included in Appendix \ref{app:inputs}.}. %A thorough description of geoprocessing steps necessary to recreate this data layer will be available soon. 
As Table~\ref{covertable} demonstrates, most cover types occupy a small extent of the study area. The cover types with an extent of less than 1000 ha within the core study area may have statistically unreliable results; this problem increases as the extent of given cover type decreases. I caution against attempting to make inferences for these less common cover types. However, because the nine cover types that do occur over at least 1000 ha represent approximately 93\% of the core study area, I have high confidence in the landscape-level results. These nine cover types are considered the focal cover types, and were all fully analyzed as part of the historical range of variability assessment. For space and continuity, in the main body of this thesis I discuss in detail only the two most common cover types, Sierran Mixed Conifer - Mesic and Serrian Mixed Conifer - Xeric, which comprise the bulk of the land in the study area actively managed by the Tahoe National Forest. Results for the other seven cover types are included in Appendix~\ref{app:full-results}. 




%%%%%%%%%%%%%%%%%%%%%%
%%% COVER TABLE %%%%%%
%%%%%%%%%%%%%%%%%%%%%%

\begin{table}[!htbp]
\footnotesize
\centering
\caption{List of land cover types developed for this project. Included are the cover type abbreviation, full cover type name, and total area in the buffered study area in both acres and hectares. Cover types are listed in descending order based on area within the core study area only. Cover types that undergo succession appear in the first group, while static cover types appear in the second group. I use the cover type abbreviation in tables and figures for space.}
\label{covertable}
\begin{tabular}{@{}llrr@{}}
\toprule
 \textbf{\begin{tabular}[c]{@{}l@{}}Land Cover \\ Abbreviation\end{tabular}} & \textbf{Land Cover Name}    & \textbf{\begin{tabular}[c]{@{}l@{}}Area \\ Core Only\\ (Hectares)\end{tabular}} & \textbf{\begin{tabular}[c]{@{}l@{}}Area\\ Core+Buffer\\ (Hectares)\end{tabular}} \\ \midrule
                        \textsc{smc\_m  }     & Sierran Mixed Conifer - Mesic                & 57,853      & 133,920    \\
\rowcolor[HTML]{CAD6BA} \textsc{smc\_x  }     & Sierran Mixed Conifer - Xeric                & 52,198      & 91,443     \\
                        \textsc{ocfw    }     & Oak-Conifer Forest and Woodland              & 23,729      & 56,941     \\
\rowcolor[HTML]{CAD6BA} \textsc{rfr\_m  }     & Red Fir - Mesic                              & 8,563       & 19,626     \\
                        \textsc{rfr\_x  }     & Red Fir - Xeric                              & 7,493       & 9,989      \\
\rowcolor[HTML]{CAD6BA} \textsc{meg\_m  }     & Mixed Evergreen - Mesic                      & 7,273       & 13,547     \\
                        \textsc{meg\_x  }     & Mixed Evergreen - Xeric                      & 6,768       & 13,771     \\
\rowcolor[HTML]{CAD6BA} \textsc{smc\_u  }     & Sierran Mixed Conifer - Ultramafic           & 4,124       & 9,774      \\
                        \textsc{ocfw\_u }     & \begin{tabular}[c]{@{}l@{}}Oak-Conifer Forest and \\ Woodland -  Ultramafic\end{tabular} & 1,060   & 2,185   \\
\rowcolor[HTML]{CAD6BA} \textsc{lpn     }     & Lodgepole Pine                               & 837         & 2,816      \\
                        \textsc{mrip    }     & Montane Riparian                             & 732         & 2,216      \\
\rowcolor[HTML]{CAD6BA} \textsc{scn     }     & Subalpine Conifer                            & 638         & 12,543     \\
                        \textsc{meg\_u  }     & Mixed Evergreen - Ultramafic                 & 604         & 1,655      \\
\rowcolor[HTML]{CAD6BA} \textsc{rfr\_u  }     & Red Fir - Ultramafic                         & 294         & 321        \\
                        \textsc{wwp     }     & Western White Pine                           & 273         & 510        \\
\rowcolor[HTML]{CAD6BA} \textsc{smc\_asp}     & Sierran Mixed Conifer with Aspen             & 58          & 121        \\
                        \textsc{oak     }     & Oak Woodland                                 & 19          & 4,186      \\
\rowcolor[HTML]{CAD6BA} \textsc{cmm     }     & Curl-leaf Mountain Mahogany                  & 18          & 41         \\
                        \textsc{lpn\_asp}     & Lodgepole Pine with Aspen                    & 8           & 31         \\
\rowcolor[HTML]{CAD6BA} \textsc{ypn     }     & Yellow Pine                                  & 0           & 10,499     \\
                        \textsc{sage    }     & Big Sagebrush                                & 0           & 1,600      \\
\rowcolor[HTML]{CAD6BA} \textsc{rfr\_asp}     & Red Fir with Aspen                           & 0           & 34         \\
                        \textsc{scn\_asp}     & Subalpine Conifer with Aspen                 & 0           & 6          \\
\rowcolor[HTML]{CAD6BA} \textsc{lsg     }     & Black and Low Sagebrush                      & 0           & 5          \\
                        \textsc{ypn\_asp}     & Yellow Pine with Aspen                       & 0           & 3          \\ 
\midrule
                        \textsc{wat     }     & Water                                        & 4,058       & 8,212      \\
\rowcolor[HTML]{CAD6BA} \textsc{bar     }     & Barren                                       & 2,665       & 8,751      \\
                        \textsc{grass   }     & Grassland                                    & 1,379       & 4,617      \\
\rowcolor[HTML]{CAD6BA} \textsc{med     }     & Meadow                                       & 1,201       & 3,435      \\
                        \textsc{urb     }     & Urban                                        & 114         & 782        \\
\rowcolor[HTML]{CAD6BA} \textsc{agr     }     & Agriculture                                  & 16          & 5,416      \\
\bottomrule

\end{tabular}
\end{table}
%\normalsize

\paragraph*{Seral Stage}
Seral stage classes combine developmental stage and canopy cover, and are defined for all cover types that undergo succession. Seral stages in this application are based on LandFire structural classes, and were further modified in collaboration with local experts on the Tahoe National Forest. In \textsc{RMLands}, susceptibility to and mortality from natural disturbances varies among seral stages. Unlike the cover grid, the seral stage grid changes dynamically over time in response to simulated succession and disturbance events. The combination of cover type and seral stage forms the basis for characterizing vegetation patterns and dynamics.

The source for the seral stage layer is the Region 5 Existing Vegetation Layer, mapped to the \textsc{calveg} classification. The \textsc{calveg} classification was developed by the Region's Ecology Program in 1978. Within the study area, the Existing Vegetation Layer was developed based on three separate efforts: a satellite-based imagery analysis in 2000, and two orthoimagery analysis completed by contracting firms in 2005. All members of the team discussed potential attributes to be used for this classification, and identified attributes for tree diameter at breast height and cover from above to classify pixels into early, middle, or late development, and open, moderate, and closed canopy. In this application, aspen and shrub types have seral stages that differ from that of the remaining forest types. The other forested types use a consistent set of seral stages.

Extensive geoprocessing was required to prepare this layer for \textsc{RMLands}. Beyond converting the vector data to a raster format, further analysis was required to update the layer to a year 2010 condition. Spatial data on wildfire and timber management history was used to provide a more accurate assessment of seral stage based on estimated stand age. In addition, areas currently mapped as chaparral in the Existing Vegetation Layer were assigned to the early development stage. The full set of seral stages is provided in Table~\ref{condtable} and depicted in Figure~\ref{fig:conditionmap}.

%%%%%%%%%%%%%%%%%%%%%%
%%% CONDITION TABLE %%
%%%%%%%%%%%%%%%%%%%%%%

\begin{table}[!htbp]
\footnotesize
\centering
\caption{List of seral stages developed for this project. Seral stages describe developmental stage (e.g. ``early'') and canopy closure (e.g. ``open''). The non-seral ``stage'' applies to land cover types for which I do not simulate succession (Barren, Grassland, Urban, Agriculture, Water, and Meadow). Included are the seral stage abbreviations, and full names.}
\label{condtable}
\begin{tabular}{@{}ll@{}}
\toprule
\textbf{\begin{tabular}[c]{@{}l@{}}Seral Stage \\ Abbreviation\end{tabular}} & \textbf{\begin{tabular}[c]{@{}l@{}}Seral Stage  \\ Name\end{tabular}} \\ \midrule
\rowcolor[HTML]{CAD6BA}   \textsc{ns}             & Non-Seral                     			    \\
                          \textsc{early\_all }    & Early 					                     \\
\rowcolor[HTML]{CAD6BA}   \textsc{mid\_cl    }    & Mid--Closed                    \\
                          \textsc{mid\_mod   }    & Mid--Moderate                  \\
\rowcolor[HTML]{CAD6BA}   \textsc{mid\_op    }    & Mid--Open                      \\
                          \textsc{late\_cl   }    & Late--Closed                   \\
\rowcolor[HTML]{CAD6BA}   \textsc{late\_mod   }   & Late--Moderate                 \\
                          \textsc{late\_cl     }  & Late--Open                     \\
\rowcolor[HTML]{CAD6BA}   \textsc{early\_asp  }   & Early--Aspen                   \\
                          \textsc{mid\_asp   }    & Mid--Aspen                     \\
\rowcolor[HTML]{CAD6BA}   \textsc{mid\_ac    }    & Mid--Aspen Conifer             \\
                          \textsc{late\_ca    }   & Late--Conifer Aspen            \\ \bottomrule
\end{tabular}
\end{table}



\paragraph*{Age}
Age represents the number of years since the last stand-replacing disturbance (high mortality wildfire). Because the characteristic species of a given cover type may not immediately establish after a stand-replacing fire, it is likely that the age value is larger than the actual age of the oldest individuals in a stand. Several of the cover types in this area may go through a chaparral-dominated early development stage; in those cases the oldest trees in the stand could be decades older than the formal stand age. In \textsc{RMLands}, age is used to trigger potential successional transitions and to calculate susceptibility to disturbance. In this application, I rounded all modeled and derived ages to the nearest five years (the length of one timestep). In the HRV analysis, the initial age value assigned to a given cell is not necessarily important to the outcome of the simulation, due to the exclusion of an equilibration period consisting of the first 40 timesteps from the analyzed results.

In this application, I used data from stand exams dating to the 1960s and from recent Forest Service Region 5 Ecology group survey plots to estimate stand age across the buffered study area. I then interpolated that information across the landscape. Due to insufficient data, I was unable to disaggregate the data below the landscape scale to cover type or another more finely resolved classification. I also acknowledge that the stand exam and modern veg plots do not constitute a true sample and were conducted almost exclusively in mid-mature and mature stands of commercially viable trees, thus skewing the results to some unquantifiable degree.

I updated the interpolated data with wildfire and timber management history, and assigned ages to types coded as chaparral in the Existing Vegetation layer to the midpoint of the age spread of Early Development for the forest cover type to which it was converted. Individual raster cells with age values out of compliance with allowed ages for the corresponding seral stage of a given cell were modified to be in compliance (see Appendix~\ref{app:covertypedesc}). I modified the age layer rather than the seral stage layer based on the assumption that the seral stage assignment was more accurate that the interpolated age information. The input Age layer, showing the map at timestep 0, is shown in Figure~\ref{fig:agemap}.


\paragraph*{Seral stage-Age}
Seral stage-Age represents the age since transitioning to the current seral stage. In \textsc{RMLands} it affects most transitions between seral stages: typically there is a threshold seral stage-age below which transitions do not occur. After creating both the seral stage and age layers, I used a Python function to derive seral stage-age based on the youngest possible age for a cell of that cover and seral stage. For example, if a particular cell on the landscape was defined as having a cover type of Lodgepole Pine, with a seral stage of Mid--Closed, and an age of 50 years, I took the minimum age for that cover type-seral stage combination (according to the Lodgepole Pine cover type description in Appendix~\ref{lpn-description}, the earliest age at which a Lodgepole Pine cell can transition to Mid--Closed is 10 years), and subtracted it from the derived age to arrive at a seral stage-age of 40 years. The same caveats and assumptions that apply to the seral stage and age layers also apply to the seral stage-age layer. The final map for the initial seral stage-age is shown in Figure~\ref{fig:condagemap}.


\paragraph*{Topographic Position Index}
The topographic position index (TPI) combines heat load, which is based on aspect and slope, with slope position (Figure~\ref{fig:tpimap}), and ranges from -300 to 300. High values for TPI are correlated with locations on steep, south and west-facing, upper slopes. Low values are correlated with locations on gentle, north and east-facing, valley bottoms. Intermediate values occur along a gradient of these characteristics. The TPI is scaled to the study area and the region immediately surrounding it, and is therefore a local index only, although it could be derived based on any spatial boundary. The purpose of incorporating the TPI was to better mimic fire behavior on the landscape. Past research has indicated that fires generally burn more frequently on southerly, steep slopes than on gentle, northerly slopes. Because they are drier and more exposed, and because of the way fuels allow preheating and ready ignition of vegetation as fire travels uphill \citep{Rothermel1983}, the likelihood of overstory tree mortality increases with increasing TPI (these statements may also be interpreted as the inverse, wetter valley bottoms are less likely to produce a crown fire) \citep{North2012,Taylor2003a}. I used TPI to adjust vegetation susceptibility and mortality in \textsc{RMLands}, as described in the model parameterization section \ref{subsec:hrvmodelparam}. As part of model evaluation, I ploted the average canopy cover (over the simulated period) against TPI and calculate the proportional effect of TPI on cover (section \ref{subsec:modelassessment}).



\paragraph*{Elevation} 
Elevation represents the height above sea level in meters. In \textsc{RMLands}, the elevation layer affects disturbance spread. The elevation grid used in this analysis was a digital elevation model (DEM) provided by the Tahoe National Forest GIS staff and rescaled from 10 m$^2$ to 30 m$^2$ pixels. It is shown as a map in Figure~\ref{fig:elevationmap}.



\paragraph*{Slope} 
Slope represents the steepness of a cell as measured in percent and is derived from the elevation layer. Slope is used in GIS preprocessing to define cover types. Within \textsc{RMLands}, slope affects disturbance spread. The slope for the study area was derived from the elevation layer described above, and is shown in Figure~\ref{fig:slopemap}.


\paragraph*{Aspect} Aspect represents the direction a cell is facing in terms of eight cardinal directions. Flat aspects are also recognized. Within \textsc{RMLands}, aspect affects disturbance spread. The aspect for the study area was derived from the elevation layer described above, and is shown in Figure~\ref{fig:aspectmap}. 


\paragraph*{Streams} 
Streams represents linear hydrological features, classified as small, medium or large based on stream order. In this application, the streams layer was created from a line coverage containing hydrography data, including an attribute for stream size or order, by converting to a grid based on the stream size attribute. Streams may inhibit the spread of wildfire in \textsc{RMLands}, depending on both stream size and potential wildfire size. In order to function as a barrier, cells in the stream raster coded as streams much share a side, rather than only a vertex. The stream layer was used to overwrite any cells not coded as ``Water'' in the cover type layer, such that all ``large'' (first order) streams are represented as water in the cover type layer. The final streams input layer is shown in Figure~\ref{fig:streamsmap}.


\paragraph*{Buffer/Core} 
This layer identifies and distinguishes the ``core'' study area from the 10 km ``buffer,'' which allows wildfires to initiate outside of and burn beyond the formal study area. Without a buffer, edge effects would alter results for all aspects of the disturbance regime, as well as resulting landscape composition and configuration. A 10 km buffer was selected arbitrarily, but has in the past been sufficient to offset edge effects \citep{McGarigal2005,McGarigal2005a}. Because the simulation plays out on the full extent of the core plus the buffer, all input grids are developed to that larger extent as well. To create this raster, the original study area polygon was buffered in ArcGIS by 10 km, then converted to raster using the same procedure as for other layers. The buffer and core are easily distinguished in the subfigures of Figure~\ref{fig:inputlayermaps}: the core is the interior area delineated by a thick black line, while the buffer is the area outside of this line, displayed at a decreased brightness level.


\begin{figure}[!htbp]
  \centering
  \subfloat[][]{
    \centering
	\includegraphics[width=0.4\textwidth]{/Users/mmallek/Tahoe/Report3/images/cover_resized.png}
    \label{fig:covermap}
  } \qquad
  \subfloat[][]{
    \includegraphics[width=0.4\textwidth]{/Users/mmallek/Tahoe/Report3/images/condition_resized.png}
    \label{fig:conditionmap}
  } %\\
  %
	\subfloat[][]{
    \includegraphics[width=0.4\textwidth]{/Users/mmallek/Tahoe/Report3/images/age_resized.png}
    \label{fig:agemap}
  } \qquad
	\subfloat[][]{
    \includegraphics[width=0.4\textwidth]{/Users/mmallek/Tahoe/Report3/images/condage_resized.png}
    \label{fig:condagemap}
  } 
  %
  \subfloat[][]{
    \centering
	\includegraphics[width=0.4\textwidth]{/Users/mmallek/Tahoe/Report3/images/tpi_resized.png}
    \label{fig:tpimap}
  } \qquad
  \subfloat[][]{
    \includegraphics[width=0.4\textwidth]{/Users/mmallek/Tahoe/Report3/images/elevation_resized.png}
    \label{fig:elevationmap}
  } %\\
  %
	\subfloat[][]{
    \includegraphics[width=0.4\textwidth]{/Users/mmallek/Tahoe/Report3/images/slope_resized.png}
    \label{fig:slopemap}
  } \qquad
	\subfloat[][]{
    \includegraphics[width=0.4\textwidth]{/Users/mmallek/Tahoe/Report3/images/aspect_11oct.png}
    \label{fig:aspectmap}
  } 
  %\\
  	\subfloat[][]{
    \includegraphics[width=0.4\textwidth]{/Users/mmallek/Tahoe/Report3/images/streams_resized.png}
    \label{fig:streamsmap}
  } 

  \caption{RMLands input layers. (a) Cover type map (b) Seral stage map (c) Age map at Timestep 0 (d) Seral stage-Age map at Timestep 0 (e) Topographic Position Index (f) Elevation (g) Slope (h) Aspect (i) Streams}
  \label{fig:inputlayermaps}
\end{figure}


\subsection{Model Parameterization}
\label{subsec:hrvmodelparam}

\subsubsection*{State and Transition Models}
I have created a detailed cover type description document for each cover type in the simulated landscape that experiences transitions between cover class (Appendix \ref{app:covertypedesc}). These documents describe crosswalks to other data layers, detailed accounts of the multiple species characteristic of the cover type, the cover type distribution, relationship and response to wildfire, predicted fire rotations, as well as descriptions of each seral stage present within the cover type and their succession and wildfire transition conditions and rates. Each detailed document can be summarized in part as a state and transition model for a particular cover type, which is implemented in the model by specifying susceptibility to wildfire, rules for vegetational succession, and rules for transitions after a fire event. Figure~\ref{transmodel} shows a generic example state and transition model for the forested cover types.

\begin{figure}[htbp]
\centering
\includegraphics[width=0.8\textwidth]{/Users/mmallek/Tahoe/Report3/images/state_trans_model.pdf}
\caption{Generic state and transition model for all non-shrub seral cover types. Each dark grey box represents one of the seven seral stages for this landcover type. Each column of boxes represents a stage of development: early, middle, and late. Each row of boxes represents a different level of canopy cover: closed (70-100\%), moderate (40-70\%), and open (0-40\%). Transitions between states/seral stages may occur as a result of high mortality fire, low mortality fire, or succession. Specific pathways for each are denoted by the appropriate color line and arrow: red lines relate to high mortality fire, orange lines relate to low mortality fire, and green lines relate to natural succession.} 
\label{transmodel}
\end{figure}


An important characteristic of \textsc{RMLands} is that it treats fire somewhat differently from other landscape succession and disturbance models, which affected how I created and specified the state and transition models. In \textsc{RMLands}, fires spread probabilistically based on the susceptibility of an individual cell. It does not contain a mechanistic fire model and fuels are not directly incorporated into fire spread. In addition, I do not classify individual fires as a whole to a \emph{low, mixed, or high severity} status. Some fire ecologists combine fire attributes such as flame length and fire size into their interpretation of the relative \emph{severity} of a particular fire \citep{Agee1993}.   Ecologists working at other scales and not working with models often describe \emph{mixed severity} regimes \citep[e.g.,][]{Kane2013}, which \citet{Collins2010} define as ``stand-replacing patches within a matrix of low to moderate fire-induced effects.'' If I were to adopt that definition, nearly all fires would be classified as \emph{mixed severity} due to the 30 m cell size and resolution at which fire mortality is defined, rendering this perspective moot for my study. Instead, I focus on defining conditions under which transitions among potential states within a given cover type occur or not (Figure~\ref{fig:mosaic}).
\begin{wrapfigure}{R}{0.5\textwidth} % use a capital R to allow figure to float
\includegraphics[width=0.5\textwidth]{/Users/mmallek/Documents/Thesis/Seminar/rimfirehillside.jpeg}
\caption{Aftermath of the 2013 Rim Fire in the Sierra Nevada. As in my model, post-fire, the landscape can be sorted into high mortality and low mortality areas. Photo from USFS Region 5.} 
\label{fig:mosaic}
\end{wrapfigure} 
I evaluate and classify fire by its effects on individual cells. First, I evaluate whether a cell burned. Next, all burned cells are evaluated probabilistically and assigned either a high severity (``high mortality'') outcome or low mortality outcome. If a cell burns at high severity, then it is deemed to have had a high mortality outcome and transitions to the Early Development seral stage. Recently, some researchers have differed on whether 75\% or 95\% overstory tree mortality is a more appropriate cutoff point for defining a ``stand-replacing'' event \citep{Fule2014,Mallek2013}. In this paper, I use 75\% as the cutoff, which is widely accepted in the literature \citep{Agee1993,Agee2007,Miller2009,Baker2014}.

To derive probabilities for post-fire transitions, both for transitions to the Early seral stage or to a more open seral stage, I used the Vegetation Dynamics Development Tool (VDDT) models associated with the Biophysical Settings Models from the LandFire project. From the VDDT models, I used the probabilities of a transition to the early seral stage, a more open canopy seral stage, or of no transition. I ignored the classified type of fire (as replacement, mixed, or low severity), focusing instead on the outcome from fire in terms of the seral stage, if any, to which a cell transitioned after wildfire. High severity fire occurs when the result is converstion to early seral (regardless of whether the fire is labeled ``replacement'' or ``mixed'' in the VDDT model). All other fires are less than high severity\todo{still need a way to describe fire properly - ask lee?}, and considered low mortality. The probability of a high mortality outcome from fire was calculated by dividing the summed probabilities of fire leading to a transition to the Early Development seral stage by the summed probabilities of all fires.  I also used LandFire data to derive probabilities of succession. I then evaluated and refined these probabilities with input from local experts to capture subtle changes in succession and transition relevant to the scale of the study area.

To illustrate the parameterization, in the following tables I present values for the Sierran Mixed Conifer - Mesic cover type model. The target fire rotation for this cover type is 29 years. A fire rotation index is used as the parameter controlling the relative susceptibility to fire of the seral stages within an individual cover type (low values correspond to higher susceptibility). In addition, the probability of high severity fire leading to at least 75\% overstory mortality is specified for each seral stage (Table~\ref{smcm_fri_phm})\todo{I like this phrasing}. I also specified transition probabilities for natural succession between the early, middle, and late stages of development, as well as between closed, moderate, and open canopy cover. This type of succession also depends on the time in the current seral stage both in terms of the early-middle-late sequence (\emph{Development-Age}) and the specific stage-canopy cover combination (\emph{Seral Stage-Age}) (Table~\ref{smcm_vegtrans}). Finally, probabilities are specified for vegetation transitions after less than high severity wildfire (Table~\ref{smcm_firetrans}).


% edited 2015-9
\begin{table}[htbp]
\footnotesize
\centering
\caption{Fire rotation index values and probability of high severity fire (at least 75\% overstory tree mortality) for Sierran Mixed Conifer - Mesic. The seral stage that is most susceptible to fire (i.e., has the lowest predicted fire rotation) has a fire rotation index value of 1. Higher values correspond with lower susceptibility to wildfire. The values are relative only within an individual seral stage and should not be compared against other land cover types.}
\label{smcm_fri_phm}
\begin{tabular}{lcc}
\hline
 \textbf{Seral Stage}    & \textbf{\begin{tabular}[c]{@{}c@{}}Fire Rotation \\ Index\end{tabular}} & \textbf{\begin{tabular}[c]{@{}c@{}}Probability of \\ High Severity Fire\end{tabular}} \\ \hline
Early (All)     			& 5.4        & 1.0                 \\
Mid--Closed    				& 2.4        & 0.23              \\
Mid--Moderate  				& 1.6        & 0.17              \\
Mid--Open      				& 1.3        & 0.14              \\
Late--Closed   				& 4.3        & 0.37              \\
Late--Moderate 				& 1.6        & 0.14              \\
Late--Open     				& 1.0          & 0.09              \\ 
\emph{Target Fire Rotation}    		& \emph{29 years}  &   \\ \hline
\end{tabular}
\end{table}

\begin{table}[!htbp]
\footnotesize
\centering
\caption{Timeframes for transitions between seral stages in \textsc{RMLands} for Sierran Mixed Conifer~-~Mesic. ``Early to Mid'' and ``Mid to Late'' times are based on the time in a developmental stage, regardless of disturbance history. ``Open to Moderate'' and ``Moderate to Closed'' times are based on the time in a seral stage since the last recorded fire.}
\label{smcm_vegtrans}
\begin{tabular}{cccc}
\hline
\textbf{\begin{tabular}[c]{@{}c@{}}Seral Stage \\ Transition\end{tabular}} & \textbf{Minimum (years)} & \textbf{Average (years)} & \textbf{Maximum (years)} \\ \hline
Early to Mid 	& 20      & 26      & 40      \\
Mid to Late 	& 100     & 113     & 150     \\
\begin{tabular}[c]{@{}c@{}}Open to Moderate or\\ Moderate to Closed\end{tabular}  & 15      & 21      &    ---     \\ \hline
\end{tabular}

\end{table}


\begin{table}[!htbp]
\footnotesize
\centering
\caption{Transition probabilities for Sierran Mixed Conifer - Mesic following low mortality fire.}
\label{smcm_firetrans}
\begin{tabular}{lcc}
\hline
\textbf{Seral Stage Transition} & \textbf{Probability}\\
\hline
Mid--Closed to Mid--Moderate    & 0.17 \\ %0.53 
Mid--Moderate to Mid--Open    	& 0.24 \\ %0.36
Late--Closed to Late--Moderate	& 0.54 \\
Late--Moderate to Late--Open    & 0.24 \\
\hline
\end{tabular}
\end{table}

Transitions between Early and Middle Development, and between Middle and Late Development are governed by the time in the Early or Middle stage (canopy cover usually does not affect these probabilities). These transitions may begin at the minimum time in a specified \emph{Development-Age}, and proceed at rates that vary across cover types. Table~\ref{smcm_vegtrans} displays the average \emph{Seral Stage-Age} of transition. If a cell reaches the maximum stage-age listed, its probability of transitioning goes to 1. 

Transitions between the canopy cover levels occur within one developmental stage: i.e., between Mid--Open and Mid--Moderate, but not between Mid--Open and Late--Moderate. These transitions are governed by the time in a specific seral stage since the last wildfire. This means that the ``years since'' value may be affected by a low mortality fire, a transition between developmental stages, or a transition between canopy cover levels. Similarly to the developmental transitions, the shift from, for example, Mid--Open to Mid--Closed, may begin when the minimum time is reached, and also proceeds at rates that vary across cover types. No maximum age is specified for this type of transition.

\subsubsection*{Disturbance Parameters} 
\label{subsubsec:distparams}

RMLands works by simulating fires and succession, one timestep at a time. The first part of every timestep is simulating fire and the second part simulates succession. To simulate fire, fires are \emph{initiated} randomly across the landscape at the cell level (that is, one pixel in the raster). The initial cell burns, or doesn't, based on its susceptibility value at that time, which is resolved probabilistically. \emph{Spread} is a step where the spatially-explicit aspect of the model comes into play. Fire can spread from a burned cell to adjacent cells. This process also incorporates spatial information on aspect and elevation, among other attributes. Whether or not the fire spreads is also based on the \emph{susceptibility} of surrounding cells and the probability of having a larger \emph{potential size}. After fire, each cell is assigned to ahigh or low \emph{mortality outcome} probabilistically. Based on this designation, cells are evaluated to determine whether or not they will \emph{transition}. Figure~\ref{fig:rmlands-fire-steps} illustrates these steps.

\begin{figure}[htbp]
\centering
\includegraphics[height=0.3\textheight]{/Users/mmallek/Documents/Thesis/Seminar/rmlands-fire.png}
\caption{Steps in fire simulation in the \textsc{RMLands} software.} 
\label{fig:rmlands-fire-steps}
\end{figure}

In \textsc{RMLands}, parameter specification is grouped under five headings: climate, initiation, susceptibility, spread, and mortality.


%\begin{adjustwidth}{5ex}{0pt}
\begin{itemize}
\item \emph{Climate:} The climate parameters are based on a rescaling of the Palmer Drought Severity Index (PDSI). PDSI is a long-term measure of drought, on the scale of months to years. It is based on precipitation and temperature and incorporates soil moisture. Resconstructed PDSI values for summer months during the historic period of this project (1550-1850) are available from the National Oceanic and Atmospheric Administration (\burl{http://www.ncdc.noaa.gov/paleo/pdsi.html}). I used datasets from \citet{Zhangetal.2004} and \citet{Cook2004} that included spatially-explicit PDSI values for North America during the historical period. These data are summarized at large scales; for example, the \citet{Cook2004} data are calculated for a grid with points spaced at 2.5\textdegree. I selected the five closest points to the center of the study area from these two datasets and calculated the inverse distance-weighted mean of the values. I then converted the yearly data into five-year averages to align with the five-year timesteps in our model. PDSI normally scales from -6 (extreme drought) to 6 (extremely wet). Values of 0 are considered normal. By recentering the mean value around 1 and then taking the inverse, I created a dataset in which a value of 1 is neither wetter nor dryer than average, values between 0 and 1 represent wetter-than-normal timesteps, and values greater than 1 represent dryer-than-normal timesteps (Figure~\ref{pdsi}). Climate interacts with other disturbance parameters in \textsc{RMLands}, including initiation, susceptibility, and spread.

\begin{figure}[htbp]
\centering
\includegraphics[height=0.3\textheight]{/Users/mmallek/Documents/Thesis/Plots/pdsi/hrv-bigtext.png}
\caption{Palmer Drought Severity Index, rescaled, inverted, and presented as a 5-year average for the ``historical'' period in this study (1550-1850).} 
\label{pdsi}
\end{figure}

%\medskip

%\noindent 
\item \emph{Initiation:} In \textsc{RMLands}, the ignition calibration coefficient is typically used as a calibration parameter. The ignition calibration coefficient refers to the number of attempted ignitions per 100,000 ha per year. For the HRV simulation, I set this coefficient at 42. I applied the coefficient evenly across the landscape based on local expert knowledge of lighting strike locations in the area. Fires may be initiated anywhere within the study area or the 10 km buffer around it. The total area cover within that boundary is 409,411 ha, so up to 860 fire starts were possible during each 5-year timestep in our simulation (not all potential ignitions result in fire). Climate also influences initiation. The probability of wildfire initiation is a function of its susceptibility to wildfire and the climate modifier value for that timestep, and is applied at the cell level.

%\medskip

%\noindent 
\item \emph{Susceptibility:} Cover type and seral stage are both inputs to susceptibility. Topographic position is also an input, whose influence is specified at the cover type level. Cover type modifies susceptibility via the ability to specify the influence of topographic position on susceptibility (Table~\ref{covtpi}). The magnitude of this effect is estimated as a potential reduction in susceptibility of 30\% between the minimum and maximum Topographic Position Index (TPI) values used in the model. I implemented this by using a logistic function to convert the TPI grid values into an appropriate multiplier within the susceptibility equation:

$$\text{TPI Susceptibility Factor} = L + \frac{R-L}{1+e^{k(x_0-x)}}$$

in which $L= 0.7$, $R=1$, slope $k=1$, inflection point $x_0=0$, and $x=\text{TPI}$. %at a given cell
It therefore simplifies to 

$$\text{TPI Susceptibility Factor} = 0.7 + \frac{0.3}{1+e^{-x}}$$

In this way, when the TPI Susceptibility Factor = 1, TPI has no effect. This happens only when the TPI value at an individual cell is zero. The effect of this is that areas with low TPI (generally north-facing and flatter slopes) burn less frequently than areas with high TPI (generally south-facing and steeper slopes).


\begin{table}[htbp]
\footnotesize
\centering
\caption{Cover types whose susceptibility is modified by Topographic Position Index. All cover types are modified in the same way.}
\label{covtpi}
\begin{tabular}{ll}
\hline
\multicolumn{2}{c}{\textbf{Cover Types modified by TPI Susceptibility Factor}} \\
\hline
Grassland     					& Red Fir - Mesic   			\\
Lodgepole Pine    				& Red Fir - Ultramafic			\\
Mixed Evergreen - Mesic				& Red Fir - Xeric    			\\
Mixed Evergreen - Ultramafic     		& Sierran Mixed Conifer - Mesic    	\\
Mixed Evergreen - Xeric 			& Sierran Mixed Conifer - Ultramafic 	\\
Montane Riparian				& Sierran Mixed Conifer - Xeric 	\\
Oak Woodland 					& Western White Pine			\\
Oak-Conifer Forest and Woodland 		& Yellow Pine 				\\
Oak-Conifer Forest and Woodland - Ultramafic 	&					\\
\hline
\end{tabular}

\end{table}

Seral stage further modifies susceptibility. I used the Weibull cumulative distribution function and specify a scale parameter $\lambda$ (``mean return interval''), shape parameter $k$, and the reset point for the function (\emph{age since high mortality disturbance} or \emph{age since any disturbance}) \citep{Johnson1985}. The fire rotation index for the seral stage is used as the basis for $\lambda$ and treated as a calibration parameter. These values were initially set as equal to the mean return interval values provided in the analogous LandFire Biophysical Setting types \citep{Landfire2007}. Some modifications were made based on consultation with Forest Service staff. All values of $\lambda$ within a cover type were modified as a group and kept relative to one another (that is, the rotation index ratios were preserved) even as the magnitude of the parameters were adjusted. The fire rotation index values for Sierra Mixed Conifer - Mesic are shown in Table~\ref{smcm_fri_phm}; these values for each cover type are included in the cover type description documents (Appendix \ref{app:covertypedesc}). I set $k=3$ for all cover types and seral stages. I selected between (\emph{age since high mortality disturbance} and \emph{age since any disturbance}) based on whether wildfires in that cover type are climate-driven (in which case I selected the former) or fuels-driven (in which case I selected the latter) (Figure~\ref{howdriven}).

\begin{table}[htbp]
\footnotesize
\centering
\caption{Cover types sorted by whether wildfire disturbance in them is characterized by fuels present or overarching climatic conditions. If the likelihood of wildfire depends on the accumulation of fuels, the value of $x$ (``time since'') reverts to 0 after any disturbance. If the likelihood of wildfire depends primarily on climate and weather conditions, the value of $x$ reverts to 0 only after a high mortality disturbance.}
\label{howdriven}
\begin{tabular}{ll}
\hline
 \textbf{Fuel-Driven Cover Types} 	& \textbf{Climate-Driven Cover Types}	\\
\hline
Curl-leaf Mountain Mahogany 	        & Agriculture   			\\
Grassland     			        & Big Sagebrush 			\\
Lodgepole Pine                          & Black and Low Sagebrush		\\
Meadow				        & Lodgepole Pine with Aspen 		\\
Mixed Evergreen - Mesic			& Montane Riparian			\\
Mixed Evergreen - Ultramafic            & Red Fir with Aspen   			\\
Mixed Evergreen - Xeric 		& Red Fir - Mesic    			\\
Oak Woodland 				& Red Fir - Ultramafic 			\\
Oak-Conifer Forest and Woodland 	& Red Fir - Xeric 			\\
Oak-Conifer Forest and Woodland - Ultramafic 	& Subalpine Conifer 		\\
Sierran Mixed Conifer - Ultramafic 	& Subalpine Conifer with Aspen 		\\
Sierran Mixed Conifer - Xeric 		& Sierran Mixed Conifer with Aspen 	\\
Urban 					& Sierran Mixed Conifer - Mesic 	\\
Yellow Pine 				& Western White Pine 			\\
					& Yellow Pine with Aspen 		\\
\hline
\end{tabular}
\end{table}


%\medskip



%\noindent 
\item \emph{Spread:} The probability of fire spread in \textsc{RMLands} is a function of climate, susceptibility to wildfire, potential wildfire size, wind, spotting, relative elevation, and presence of streams. The first two are described above. The disturbance size distribution that regulates potential fire size was created by analyzing the size distribution of all mapped fires in the Northern Sierra \textsc{calveg} mapping zone and west of the Sierran crest, available from the U.S. Geological Survey and the California Department of Forestry and Fire Protection \citep{calfire2012,usgs-fire-data2012}, which together go back to approximately 1900. 

\begin{minipage}{\linewidth}
 \centering
 \includegraphics[width=10cm]{/Users/mmallek/Tahoe/Report3/images/weather.png}
  \captionof{figure}{Weather stations used to inform wind direction parameters. Weather stations are denoted by red circles. A black boundary line identifies the study area.}
  \label{weather}
\end{minipage}

Wind is incorporated in two parts. First, a prevailing \emph{wind direction} for the fire is selected probabilistically from the eight cardinal directions. To compute the wind distribution values, I first consulted local experts to determine the dates of fire season (May 15 to October 15) and burning period times (1000 hours to 1800 hours). I then downloaded all available historic wind direction data from 6 local weather stations (Rice Canyon, Saddleback, Downieville, White Cloud, Emigrant Gap, and Blue Canyon, Figure~\ref{weather}). Data from all weather stations was weighted equally. After the wind direction is selected, fires are able to grow in all directions, but are relatively more likely to spread with wind than against it. I parameterized the influence of \emph{relative wind} as a reduction in spread likelihood. Thus, spread in the same direction as wind has a neutral effect, spread at $\ang{45}$ angles is reduced by 30\%, spread at $\ang{90}$  angles is reduced by 70\%, spread at $\ang{135}$ angles is reduced by 90\%, and spread opposite the prevailing wind direction is reduced by 95\%. 

\emph{Relative elevation} also modifies spreading potential. I parameterized the model such that spread downhill is extremely unlikely. \emph{Spotting} and the extent to which streams act as \emph{barriers to spread} are affected by the fire size. As fires become larger, their probability of spotting and spotting distance increases. Similarly, streams function as a barrier to smaller fires, but large fires are able to spread past streams regardless of size. This decision is based on the idea that large fires are more influenced by wind and climatic conditions. Stream size does impact smaller fires; the largest streams and rivers are usually an effective barrier to smaller fires, although even fairly small fires often spread past intermittent and small perennial streams. 


\item \emph{Mortality:} Cover type and seral stage are both inputs to \emph{mortality}. Cover type modifies susceptibility via the ability to specify the influence of topographic position on mortality (Table~\ref{covtpi_mort}). The magnitude of this effect is estimated as a potential reduction in mortality of 30\% between the minimum and maximum TPI values used in the model. I implemented this by using a logistic function to convert the TPI grid values into an appropriate multiplier within the mortality equation:


$$\text{TPI Mortality Factor} = L + \frac{R-L}{1+e^{k(x_0-x)}}$$

in which $L= 0.7$, $R=1$, slope $k=1$, inflection point $x_0=0$, and $x=\text{TPI}$. %at a given cell
It therefore simplifies to 

$$\text{TPI Mortality Factor} = 0.7 + \frac{0.3}{1+e^{-x}}$$

In this way, when the TPI Mortality Factor = 1, TPI has no effect. This happens only when the TPI value at an individual cell is zero. The effect of this is that areas with low TPI (generally north-facing and flatter slopes) are less likely to experience high severity wildfire leading to over 75\% overstory mortality than areas with high TPI (generally south-facing and steeper slopes).

\begin{table}[htbp]
\footnotesize
\centering
\caption{Cover types whose mortality is modified by Topographic Position Index.}
\label{covtpi_mort}
\begin{tabular}{ll}
\hline
\multicolumn{2}{c}{\textbf{Cover Types Mdified by TPI Mortality Factor}} \\
\hline
Grassland     					& Red Fir - Mesic   			\\
Lodgepole Pine    				& Red Fir - Ultramafic			\\
Mixed Evergreen - Mesic				& Red Fir - Xeric    			\\
Mixed Evergreen - Ultramafic     		& Sierran Mixed Conifer - Mesic    	\\
Mixed Evergreen - Xeric 			& Sierran Mixed Conifer - Ultramafic 	\\
Montane Riparian				& Sierran Mixed Conifer - Xeric 	\\
Oak Woodland 					& Western White Pine			\\
Oak-Conifer Forest and Woodland 		& Yellow Pine 				\\
Oak-Conifer Forest and Woodland - Ultramafic 	&					\\
\hline
\end{tabular}
\end{table}

Seral stage further modifies mortality. I extracted the likelihood of mortality from the VDDT models built during the LandFire project, as described at the beginning of section~\ref{subsec:hrvmodelparam}. As an example, these probabilities for Sierran Mixed Conifer - Mesic are provided in Table~\ref{smcm_fri_phm}.



\end{itemize}
%\end{adjustwidth}

\subsection{Model Calibration}
Although \textsc{RMLands} is a process-based model with parameters sourced from the literature, the team and I had greater confidence in some parameters than others, especially with respect to how they function within the \textsc{RMLands} framework. Consequently, I calibrated, or verified, the model by iteratively adjusting certain parameters in which there was less confidence about the appropriate values until the outputs were tuned to a set of parameters in which the team had high confidence. Specifically, I manipulated the ignition calibration coefficient and the fire rotation index and measured calibration success based on conformity to pre-specified rotation values at the cover type level. Fire rotation index values were changed by a constant multiplier across all seral stages of a given cover type. That is, cover types were modified as groups but the index ratios within them were maintained. The calibration target was defined as the rotation values for the nine focal cover types\footnote{Mixed Evergreen - Mesic, Mixed Evergreen - Xeric, Oak-Conifer Forest and Woodland, Oak-Conifer Forest and Woodland - Ultramafic, Red Fir - Mesic, Red Fir - Xeric, Sierran Mixed Conifer - Mesic, Sierran Mixed Conifer - Ultramafic, Sierran Mixed Conifer - Xeric.}  $\pm 10$\% of the original target rotations. I focused on these nine types because they all extend across more than 1,000 ha, and are thus statistically stable from simulation to simulation. Target values were based on published empirical values and refined with input from local experts. I chose rotation as the calibration target because targets were available from the literature and because fire rotation is a fundamental measurement that \textsc{RMLands} was designed to capture. In addition, using rotation ties calibration to a parameter that is relateable to Forest Service staff and that can be used at the landscape scale as a target by managers in various programs. To illustrate the calibration process, I describe it for Sierran Mixed Conifer - Mesic. The target fire rotation was 29 years. I adjusted the input seral stage fire return index by multiplying it by different constants, eventually arriving at an increase by a factor of 25 from the original calculated ratio values. That is, each initial scale parameter value was multiplied by 25 in order to modify its susceptibility to fire without changing the relative susceptibility among its seral stages (see Table~\ref{smcm_fri_phm} for the final relative susceptibility values). 


%%%
%in a nutshell, we have to calibrate the model because we it's not completely mechanistic and we're making guesses on a lot of things
%so we pick a few model parameters that we have high confidence in, and decide  not to change those
%and then we pick a few model outputs that we have confidence in
%because our goal is to simulate a regime we believe we already understand
%so we want it to look "right"
%so we figure if we can make the model outputs agree with the numbers we're pretty sure are right, then we trust the other outputs where we weren't totally sure what to expect
%and we do this by adjusting the parameters that we have lower confidence we got right hte first time, or that don't relate to the real world direclty and mechanistically



\subsubsection*{Model Execution}
During the calibration phase of the model, a typical simulation run consisted of three iterations of the model lasting 200 timesteps each. The equilibration period of 40 timesteps was chosen based on visual analysis of the seral stage distribution plots. The cutoff period was chosen at 40 timesteps because by this point the nine focal cover types had reached a distribution characteristic of a foregoing, stable distrbution oscillating around a mean value. I exclude this equilibration period from most of my statistical results because they are an artifact of the starting conditions. Once calibration was complete, I conducted one run of 500 timesteps in order to capture multiple disturbance and succession cycles across the most common cover types. Each timestep represents five years. The five-year timestep was chosen based on the short fire return intervals (how frequently fire recurred in the same location) recorded from dendrochronology analysis in the literature and my desire to capture these very short rotations in the simulation.

\subsection{Data Analysis}
\label{subsec:dataanalysis}

Sierra Nevada vegetation is extremely diverse and complex, both ecologically and spatially. In the body of this thesis I limit my results to an evaluation of the full landscape and of the xeric and mesic mixed conifer forests, which together comprise 63\% of the study area. Results for the next seven most extensive types are included in Appendices~\ref{app:full-results}. In general, my confidence in the results decline as the extent of a cover type declines, because the results are statistical and large samples are needed. I do not include results for cover types that extend across less than 1,000 ha of the study area.

\subsubsection*{Disturbance Regime} I quantified the following overall temporal and spatial characteristics of the wildfire disturbance regime:
\begin{itemize}
	\item \emph{Disturbed Area:} I calculated disturbed area for each timestep, divided into low mortality and high mortality disturbance, and summed to produce an ``any mortality'' statistic. I summarize the results for the $5^{\text{th}}$ percentile, $50^{\text{th}}$ percentile (median), $95^{\text{th}}$ percentile, and mean area disturbed as a proportion of the total area eligible for disturbance for the full simulation excluding the equilibration period (460 timesteps, or 2300 years). Because it can be difficult to visualize what our quantitative results look like, I include several maps that illustrate the results, demonstrating that model results are spatially-explicit and realistic. To do this, I include maps of the landscape illustrating the $5^{\text{th}}$ percentile, $50^{\text{th}}$ percentile, $95^{\text{th}}$ percentile, and mean area burned during the simulation, plus a example 4-timestep sequence illustrating changes to the seral stage pattern for mesic mixed conifer forests due to successional and disturbance procceses. Finally, I use a histogram to display the distribution of wildfire extents during the simulation, excluding the equilibration period.
	\item \emph{Disturbance Frequency:} I calculated the number of years between disturbances exceeding a particular threshold in total disturbed area. I report the frequency of timesteps during which thresholds of at least 10\%, 25\%, or 50\% of the landscape experienced wildfire. I also translate this into the proportion of timesteps in the simulation, and the interval between such occurrences in both years and timesteps. The purpose of this redundancy is to facilitate different ways of conceptualizing fire frequency, which will be useful to managers from diverse backgrounds.
	\item \emph{Climate Effect:} Climate interacts with several components of the model. I present plots illustrating the value of the climate parameter by timesteps concurrently with the area disturbed per timestep. It is not practical to further illustrate its effect everywhere, and in some cases its influence is not easily separated from the other inputs to the model. 
	\item \emph{Rotation Period:} I calculated the rotation period---the number of years required to burn an area equivalent to the total eligible area---for each cover type within the study area and the study area as a whole. I report the rotation values for low mortality fire, high mortality fire, and any fire for the full landscape, Sierran Mixed Conifer - Mesic, Sierran Mixed Conifer - Xeric. Results for the other seven focal cover types are available in Appendix~\ref{app:full-results}.
	\item \emph{Return Interval:} I summarized the cell-specific fire rotation---the average number of years between disturbances at a single cell---and present it as the distribution of the percentage of eligible cells that experienced each possible mean return interval. I use histograms to visualize the distribution of this return interval for low mortality fire, high mortality fire, and any fire, along with their median values. The median is equivalent to the cell-specific grand mean return interval for a given cover type across the landscape. I also display this result spatially as a map showing the cell-specific fire rotation for each raster cell across the landscape. 
\end{itemize}

\subsubsection*{Vegetation Response} 

\emph{Landscape Composition:} I quantified the distribution and dynamics of landscape composition by cover type. For the single 2500 year simulation (with 200 year equilibration period), I summarized the results in a table and graphically. For the tabular results, I present the $5^{\text{th}}$ percentile, $25^{\text{th}}$ percentile, $50^{\text{th}}$ percentile, $75^{\text{th}}$ percentile, and $95^{\text{th}}$ percentile of the distribution. I compared the current landscape seral stage distribution to this simulated historic range of variability to determine whether the current landscape deviates, and to what degree, from the HRV. 

Using a stacked bar plot, I visualized the proportion of the total area of a given cover type occurring at each seral stage, for each timestep in the model. In addition, I used a bar plot of the current seral stage distribution to allow a visual comparison between current conditions and the historical range of variability in the distribution of the seral stages. While the bar plots are useful for visualizing the cover type-seral stage dynamics, box plots facilitate a visual comparison of the $5^{\text{th}}-95^{\text{th}}$ percentile distribution (the HRV) to the current landscape values.

\emph{Landscape Configuration:} I used \textsc{Fragstats} \citep{Fragstats2012} to compute several landscape-level and class-level metrics that summarize landscape structure and pattern over the course of the simulation. I present the results in a series of tables and figures. A general introduction to  the \textsc{Fragstats} metrics are included as Appendix~\ref{app:metricdescriptions}. For a much more detailed and mathematical description of all \textsc{Fragstats} metrics, see the \href{http://www.umass.edu/landeco/research/fragstats/documents/fragstats.help.4.2.pdf}{documentation}. Each metric is computed on the study area for a single timestep, and the results are displayed in tabular format by quantiles and in graphical format with line graphs and boxplots. Table~\ref{tab:fragland-desc} summarizes the \textsc{Fragstats} metrics selected as focal metrics to provide a simple and understandable explanation of the characteristics of landscape structure during the simulated HRV. I selected metrics to represent commonly identified groups of landscape metrics: patch area and edge, patch shape complexity, core area, aggregation, and diversity \citep{McGarigal2015}. It is fairly intuitive to understand how these metrics may be affected by natural disturbance and human management efforts, thus allowing us to describe the HRV and develop suggestions tying management actions to results for these metrics.



\begin{table}[!htbp]
\footnotesize
\centering
\caption{A subset of \textsc{Fragstats} metrics I selected to emphasize in order to provide a parsimonious explanation of the variability in landscape structure during the simulated HRV. An `X' in the landscape or class column denotes whether a metric is calculated at that level. Abbreviations are included because they are used in tables and figures later in the document and in the appendices to conserve space.} 
\label{tab:fragland-desc}
%
\begin{tabular}{@{}llccc@{}}
\toprule
{\bf Metric}                    & {\bf Abbreviation} & {\bf \begin{tabular}[c]{@{}c@{}}Landscape-\\ level\end{tabular}} & {\bf \begin{tabular}[c]{@{}c@{}}Class-\\ level\end{tabular}} & {\bf Category} \\ 
\midrule
Edge Density                    & \textsc{ed} 			& X        & X     & area and edge metric		\\ 
Area-Weighted Mean Area         & \textsc{area\_am}  	& X        & X     & area and edge metric		\\
Area-Weighted Mean Shape        & \textsc{shape\_am} 	& X        & X     & shape metric 				\\
Area-Weighted Mean Core Area    & \textsc{core\_am}  	& X        & X     & core area metric		\\
Contagion                       & \textsc{contag} 		& X        & --    & aggregation metric		\\
Clumpiness Index                & \textsc{clumpy} 		& --       & X     & aggregation metric		\\
Simpson’s Evenness Index        & \textsc{siei}      	& X        & --    & diversity metric		\\
\bottomrule
\end{tabular}
\end{table}

I summarized the 90\% range of variability for both the composition and pattern metrics, and from this inferred the extent to which the current landscape departs from that range of variability. Thus I use both a quantitative and a qualitative assessment to determine how and to what extent the landscape has changed. Based on these results, I analyze different potential causes of the results, with an emphasis on understanding how human activities may be the primary factor related to any change. 

For both the composition and pattern metrics, I quantified the current landscape's departure from the HRV conditions by summarizing the distribution of each \textsc{Fragstats} metric calculated over the length of the simulation, minus the equilibration period. I computed the $5^{\text{th}}$, $25^{\text{th}}$, $50^{\text{th}}$, $75^{\text{th}}$, and $95^{\text{th}}$ percentiles of the distribution of observed values. I calculated a current percentile of the range of variability value (\%RV) by computing where along the $0^{\text{th}}-100^{\text{th}}$ percentile range of variability for the simulated historical period the current landscape metric value falls. 

To assess landscape composition and configuration, I compared the current landscape to the HRV, and report departure based on the following standards. If the current landscape metric value falls within the $25^{\text{th}}-75^{\text{th}}$ percentile range (the box in our boxplots), it is considered not departed. If it falls within the $5^{\text{th}}-25^{\text{th}}$ percentile range or the $75^{\text{th}}-95^{\text{th}}$ percentile range (the whiskers in our boxplots), it is moderately departed. If it falls outside that range, it is completely departed.And if it falls to its death, it is dearly departed. Thus, for the landscape metric \emph{Patch Density}, 19.507 is equivalent to the 32$^{\text{nd}}$ percentile of observations during the simluated HRV, and this metric is therefore currently within the HRV for the landscape. However, the landscape metric \emph{Edge Density} has a current value of 128.875. Because $128.875 > 125.316$, and 125.316 is the largest value observed during the simluated HRV, edge density at the landscape level is currently completely departed from the HRV. 

\clearpage

%%%%%%%%%%%%%%%%%%%%%%%%%%%%%%%%%%%%%%%%%%%%%%%%%%%%%%%%%%%%%%%%%%%%%%%%%%%%%%%%%%%%%%%%%%%%%%%%%%%%%%%%%%%%
\subsection{Model Assessment}
\label{subsec:modelassessment}

\subsubsection*{Sensitivity Analysis} Ideally, a sensitivity analysis would be performed to assess the sensitivity of the input parameters to the model, and subsequently indicate areas for future research. In this case, I did not complete a rigorous sensitivity analysis due primarily to practical constraints. The chief constraints were related to time, for rerunning the model many times under varying parameter sets, and disk space, due to the large amount of data generated by each run of the model. Despite this, as part of the model calibration process, I gained insight into the relative sensitivity of some parameters. Thus I can offer a qualitative sensitivity analysis. First, the ignition parameter is quite sensitive; changing it by a few interval values changes model outcomes for most analysis measures. Presumably this happens because increasing the number of potential fire starts increases the odds of a fire initiating on a susceptible cell (A formal evaluation of this effect is outside the scope of this project.). In comparison, the fire return index is relatively insensitive; I often modified it by more than an order of magnitude in order to effect a small change in the rotation outcome. Third, the probability of high mortality fire at the seral stage level is fairly sensitive. This is logical because the conversion of forest to early seral conditions directly impacts most of the metrics by which I evaluate landscape structure and composition and because high mortality fire results in a transition to Early Development 100\% of the time, so its impact is more direct and thus directly measurable..

Of the parameters observed to be more sensitive, the seral stage-level likelihood of high mortality fire is the one whose effects on model outcomes have the most important implications for my results, and are most important to invest further research effort on. Currently, probabilities of high mortality fire at the seral stage level are extremely difficult to find in the literature because no record can be taken from a tree completely consumed in a fire. Only a few researchers have attempted to infer high severity fire based on factors such as later reports of dense young conifer or shrub cover \citep{Collins2011,Baker2014,Stephens2015}.\todo{could maybe still think here about what else I can say about the baker paper} Most studies of stand-replacing fire have relied on satellite imagery to confirm ``stand-replacement'' effects \citep[e.g.,][]{Collins2010,Mallek2013}. Even if legacy trees\footnote{trees that are much older than the overall stand and that are presumed to have been left standing after a prior stand-replacing disturbance} existed and could be sampled to infer dates of past high severity fires, it would be difficult to determine when and over what extent past stand-replacing fires burned because of low sample sizes associated with legacy trees, as well as other factors such as post-fire drought \citep{Minnich2000,Baker2014}. Furthermore, the studies that derive percent high severity based on imagery produce overall cover type estimates, rather than estimates based on seral stage. Because it is the seral stage estimates that are needed to improve the model, research into this area would fill a gap in our ecological understanding and enhance the value of \textsc{RMLands}-based results.

\subsubsection*{Uncertainty Analysis} The model calibration process involved systematically varying certain input parameters and testing resultant model outcomes, which allowed me to complete a rudimentary and qualitative sensitivity analysis. An analogous process did not occur that might approach an uncertainty analysis. Conducting an uncertainty analysis on the parameter set could theoretically be accomplished, but the same constraints exist as for the sensitivity analysis. In addition to these, established uncertainties were not available for all input parameters (such as probability of high mortality fire at the seral stage level) and were often used to parameterize the stochastic part of the model when available (such as succession rates by land cover type and seral stage). Although not conducted as part of this study, an uncertainty analysis would add value to the study results.

\subsubsection*{Model Validation} I did not conduct a formal model validation, defined as testing the model outputs against independent data. One way to do this would have been to test the parameter set on a neighboring geography with a similar ecological composition. The comparison would then be done either on the results of a hindcasting exercise or on the results of recreating the historical conditions for the next few hundred years. Clearly, the latter exercise is impossible, so model validation could not be accomplished that way. Unfortunately, a hindcasting exercise is also not implementable. First, the model only simulates wildfires; many other small-scale disturbances also occur in the study area and affect landscape pattern. Second, attempts to recreate the current conditions on the landscape would be confounded with the history of vegetation management from the last 150 years. At the time of this study, \textsc{RMLands} functionality for simulating vegetation treatments in the Sierra Nevada was still under development. Even if it was available, however, the existing descriptions of past vegetation treatments are not sufficiently detailed to use in a model validation exercise. A final potential method for validating the model would be to use an old land cover type and seral stage map, and compare its composition and configuration to the HRV results. However, no such map exists, and if it had, I would have used it as my starting condition and eliminated the need for model equilibration.

In addition, it is important to understand that this model can never be fully validated because, while useful, it is like all models an abstract and simplified representation of reality. \textsc{RMLands} was set up to simulate wildfires, but there are many other disturbance processes that exist at varying scales that are not simulated here, including insects and disease, wind-throw, wild ungulate and beaver herbivory, avalanches, and other forms of soil movement. The complex interactions among them that characterize real landscapes are also, as a result, omitted from consideration.

Finally, to parameterize the mdoel, I used local empirical data wherever possible. However, I also drew on relevant scientific studies, often from other geographic locations, and relied heavily on expert opinion when scientific studies and local empirical data were not available. As a consequence, the pool of potentially available independent data is limited. 

\subsubsection*{Model Evaluation} \label{subsubsec:modelevaluation} Because true model validation was not possible for this study, a secondary method for validation is to test whether the model outputs make sense ecologically and based on available empirical data. These strategies bleed into model evaluation, or the degree to which the model outputs line up with empirical observations. As outlined in the previous paragraph, the most straightforward method for model evaluation would be to employ a hindcasting strategy, but this is not practicable. 

To some extent, the fact that model calibration was highly successful, in that output fire rotations were within 10\% or less of target values (Table~\ref{rotation-diff}), provides a positive form of model evaluation. I used rotation values as the calibration target because targets were available from the literature and because fire rotation is a fundamental measurement that \textsc{RMLands} was designed to capture. In addition, using rotation ties calibration to a parameter that is relatable to Forest Service staff and that can be used as a target by managers in various programs. 

% fixed 2016-02-06
\begin{table}[!htbp]
\centering
\footnotesize
\caption{Comparison of the target versus actual overall fire rotations recorded during the simulated historical range of variability. Includes calculated final percent difference.}
\label{rotation-diff}
\begin{tabular}{@{}lrrr@{}}
 \toprule
 \textbf{\begin{tabular}[c]{@{}l@{}}Land Cover Type\end{tabular}} &
 \textbf{\begin{tabular}[c]{@{}l@{}}Target \\ Rotation\end{tabular}} &
 \textbf{\begin{tabular}[c]{@{}l@{}}Actual \\ Rotation\end{tabular}} &
 \textbf{\begin{tabular}[c]{@{}l@{}}Percent \\ Difference\end{tabular}} \\
\midrule
Mixed Evergreen - Mesic            & 50    & 52    & 4\%   \\
Mixed Evergreen - Xeric            & 40    & 41    & \textless 1\%   \\
Oak-Conifer Forest and Woodland    & 21    & 22    & 5\%   \\
Red Fir - Mesic                    & 60    & 63    & 5\%   \\
Red Fir - Xeric                    & 40    & 38    & 5\%   \\
Sierran Mixed Conifer - Mesic      & 29    & 27    & 7\%   \\
Sierran Mixed Conifer - Ultramafic & 60    & 66    & 10\%  \\
Sierran Mixed Conifer - Xeric      & 22    & 23    & 5\%   \\
\bottomrule
\end{tabular}
\end{table}


A second method of model evaluation was a visual inspection of the output grids demonstrating wildfire extents to verify that they were similar to actual wildfire perimeters. In addition, I plotted the actual disturbance size distribution against the expected distribution (Figure~\ref{fig:dsize}). 


% updated 9/13
\begin{figure}[!htbp]
  \centering
    \centering
    \includegraphics[height=0.3\textheight]{/Users/mmallek/Documents/Thesis/Plots/dsize/hrv-ggplot.png}
  \caption{Side by side barplot of the observed and target wildfire size distribution for our 500-timestep long run of the model.}
  \label{fig:dsize}
\end{figure}

As a further effort toward model evaluation, I examined the results of implementating the topographic position index (TPI). The TPI value for a given cell acts as an input into the susceptibility and mortality values otherwise defined for that cover type and seral stage combination. Early development and open canopy seral stages tend to result from fire, and I predicted that an increase in fires and in the likelihood of high mortality fire would lead to a decrease in the average canopy cover values for cells with large TPI values. Table~\ref{tab:tpi_cc} in Appendix \ref{app:full-results} displays the results for this simulation for the nine most common cover types. All show decreased average canopy cover as TPI increases. Figure \ref{fig:tpi_cc_smc} shows the plotted data and fitted linear regression line for mesic and xeric sierran mixed conifer forests. Figure~\ref{fig:averagecc} is a map displaying average canopy cover across the landscape for the full simulated HRV timeframe, excluding the equilibration period. In general, rotations and canopy cover varied spatially across the forest and decreased with increasing TPI, reflecting empirical observations that higher, more southerly aspects are drier and more susceptible to fires. In mesic mixed conifer forests, canopy cover decreased by about 9\% when comparing minimum to maximum TPI, from an average of 55.5\% to an average of 50.4\%. In xeric mixed conifer forests, canopy cover decreased by 20.5\% when comparing minimum to maximum TPI, from an average of 27.6\% to an average of 21.9\% (Table~\ref{tab:tpi_cc_smcs}).


% figure redone
\begin{figure}[!htbp]
\centering
\includegraphics[width=.8\textwidth]{/Users/mmallek/Documents/Thesis/Plots/tpi/hrv-facet-smc.png}
\caption{Average canopy cover for Sierran Mixed Conifer Mesic and Xeric during the simulated HRV. Each blue point represents one pixel of an individual cover type on the landscape grid. The black line is the result of a linear regression fit to the data. Table \ref{tab:tpi_cc} provides the numerical representation of the shift from minimum to maximum TPI values for each cover type. (a) Sierran Mixed Conifer - Mesic; (b) Sierran Mixed Conifer - Xeric.}
\label{fig:tpi_cc_smc}
\end{figure}

% figure redone
\begin{figure}[!htbp]
\centering
\includegraphics[width=0.8\textwidth]{/Users/mmallek/Documents/Thesis/maps/averagecanopycover.pdf}
\caption{Smoothed visualization of the average canopy cover across the study area over the course of the simulation. Higher percent cover is shown in dark blue, transitioning to red where average percent cover was low. Land cover types Water and Barren have no canopy cover value and appear as grey.}
\label{fig:averagecc}
\end{figure}

%redone 9/15
%\begin{table}[!htbp]
%\footnotesize
%\centering
%\caption{The percent change in canopy cover from the minimum TPI value for that cover type to the maximum TPI value. Results for Sierran Mixed Conifer Mesic and Xeric shown here; results for other focal cover types available in Appendix~\ref{app:full-results}}.
%\label{tab:tpi_cc_smcs}
%\begin{tabular}{@{}lrrrrr@{}}
%\toprule
% \small \textbf{\begin{tabular}[c]{@{}l@{}}Cover \\ Name\end{tabular}} & \small \textbf{\begin{tabular}[c]{@{}l@{}}Minimum \\ TPI\end{tabular}} & \small \textbf{\begin{tabular}[c]{@{}l@{}}Maximum \\ TPI\end{tabular}} & \small \textbf{\begin{tabular}[c]{@{}l@{}}Average Canopy \\Cover at \\ Minimum TPI\end{tabular}} & \small \textbf{\begin{tabular}[c]{@{}l@{}}Average Canopy \\ Cover at \\ Maximum TPI\end{tabular}}  & \small \textbf{\begin{tabular}[c]{@{}l@{}}Percent \\ Change in \\ Canopy \\ Cover\end{tabular}} \\ \midrule
%\textsc{smc\_m   }    & -300                 & 300                  & 55.5       & 50.4              & -9.3      \\
%\textsc{smc\_x   }    & -300                 & 300                  & 27.6       & 21.9              & -20.5     \\ \bottomrule
%\end{tabular}
%\end{table}

\begin{table}[!htbp]
\footnotesize
\caption{For each cover type on the landscape, the percent change in canopy cover from the minimum TPI value for that cover type to the maximum TPI value. For seral stage abbreviations, see Table \ref{condtable}.}
\label{tab:tpi_cc}
%\rotatebox{90}{
\begin{tabular}{@{}lrrrrr@{}}
\toprule 
 \textbf{\begin{tabular}[c]{@{}l@{}}Cover \\ Name\end{tabular}} & \small \textbf{\begin{tabular}[c]{@{}l@{}}Minimum \\ TPI\end{tabular}} & \small \textbf{\begin{tabular}[c]{@{}l@{}}Maximum \\ TPI\end{tabular}} & \small \textbf{\begin{tabular}[c]{@{}l@{}}Average Canopy \\Cover at \\ Minimum TPI\end{tabular}} & \small \textbf{\begin{tabular}[c]{@{}l@{}}Average Canopy \\ Cover at \\ Maximum TPI\end{tabular}}  & \small \textbf{\begin{tabular}[c]{@{}l@{}}Percent \\ Change in \\ Canopy \\ Cover\end{tabular}} \\ \midrule
\textsc{meg\_m   }    & -300                 & 300   & 73.7       & 67.0     & -9.0      \\
\textsc{meg\_x   }    & -299                 & 300   & 72.7       & 68.5     & -5.7      \\
\textsc{ocfw     }    & -300                 & 300   & 50.0       & 45.6     & -8.7       \\
\textsc{rfr\_m   }    & -300                 & 300   & 72.1       & 64.0     & -11.2     \\
\textsc{rfr\_x   }    & -259                 & 300   & 40.2       & 29.1     & -27.6     \\
\textsc{smc\_m   }    & -300                 & 300   & 55.5       & 50.4     & -9.3       \\
\textsc{smc\_u   }    & -300                 & 300   & 39.9       & 28.9     & -27.7     \\
\textsc{smc\_x   }    & -300                 & 300   & 27.6       & 21.9     & -20.5     \\ \bottomrule
\end{tabular}
%}
\end{table}

\clearpage