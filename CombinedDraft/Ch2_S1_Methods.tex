% !TEX root = master.tex

\section{Methods}
\label{sec:hrvmethods}

% removed study area, it's now in the Introduction Chapter

\subsection{Modeling Framework}
\label{sec:modelframe}

A partial introduction to \textsc{RMLands} is included in Chapter~\ref{CH1}, but we provide a more detailed description of the model and how we used it here.
% intro to RMLands in Introduction Chapter (modeling framework, methodological limitations)

\subsubsection{Input Layers}
\label{subsec:hrvinputlayers}

All input layers to \textsc{RMLands} must be custom-built to work with the software. For technical details on the data structure requirements of \textsc{RMLands}, see Appendix \ref{app:inputs}. A brief overview of each input layer is included below.

\paragraph{Cover} Cover type is based on the potential or current natural vegetation of a site and includes both natural and anthropogenic cover types. For example, cover types include not only Lodgepole Pine, Sierran Mixed Conifer, and Red Fir, but also Barren and Agriculture. Succession pathways are defined uniquely for each cover type, susceptibility to natural disturbances varies among cover types, and suitability or eligibility for various vegetation treatments varies among cover types. Cover is a static (constant) grid and therefore provides a fixed template upon which disturbance and succession processes play out over time. 

The source for the cover layer is the Region 5 Existing Vegetation Layer (``EVeg''), first mapped to the \textsc{calveg} classification developed by the Region's Ecology Program in 1978. When deciding on land cover types, including determining xeric and mesic subtypes, our focus was to best represent the project area and the surrounding landscape. We used the \textsc{calveg} Mapping Zone boundary for the ``North Sierra'' (Figure~\ref{calveg}) as our focus for defining vegetation and disturbance, including susceptibility, response to fire, and fire size and distribution. Within the project area, the EVeg layer was developed based on three separate efforts: a satellite-based imagery analyses in 2000, and two orthoimagery analysis completed by contracting firms in 2005. Generally, specific cover type names were derived from the California Fire Return Interval Departure (FRID) report by \citet{VandeWater2011}. We also considered information from \emph{A Guide to Wildlife Habitats of California}, popularly known as the ``Wildlife Habitat Relationship (WHR)'' cover types. 

\begin{wrapfigure}{R}{0.5\textwidth} % use a capital R to allow figure to float
\includegraphics[width=0.48\textwidth]{/Users/mmallek/Tahoe/Report3/images/CALVEGmappingzones.png}
\caption{\small CALVEG Mapping Zones. These zones meet U.S. Forest Service standard at national and regional levels. These ecological provinces are associated with dozens of vegetation alliances, which are used to classify vegetation in spatial data products. We used vegetation alliance definitions for the North Sierra zone to classify the land cover spatial data shared by the U.S. Forest Service.} 
\label{calveg}
\end{wrapfigure}

\subparagraph{Alternative Cover Layers}
The original intent of our team was to utilize two separate cover layers: one for the historical reference period, and one for the current period to be used in projections of future scenarios. Two layers were identified as potentially suitable for the historic analysis: a map created from forest survey and inventory efforts under Albert Wieslander conducted between 1928 and 1940 (``Wieslander'') \citep{Thorne2006}, and a map of Potential Natural Vegetation created by a Forest Service Enterprise Team for the Tahoe National Forest in the 2000s (Forest Service internal GIS data). Our intent was to use the PNV, Wieslander, or a combination thereof to derive the land cover layer for the HRV phase of the project. 

In order to validate the historical maps, we needed to develop a crosswalk between the vegetation type methodologies for the EVeg, PNV, and Wieslander maps. We also examined the spatial consistency in cover types across the maps. With significant assistance from the Tahoe National Forest, we attempted to create a crosswalk from these maps to the set of land cover types to be used in the project. However, we were unable to develop a consistent and comprehensive set of rules for this purpose. A major reason for this is that both the PNV and Wieslander maps used species lists, rather than assemblages (as in \textsc{calveg} and LandFire). For example, Sierran mixed conifer forests do not appear as a dominant ``cover type'' in the PNV map. The Wieslander maps do contain an internal crosswalk to a mixed conifer alliance, but only rarely. 

In addition, the PNV map contained a more significant error: we learned that, for the purposes of the modeling used to create the PNV map, ``potential natural vegetation'' meant the so-called ``climax'' community that would develop in the complete absence of disturbance, regardless of whether that disturbance was human-caused or natural. Since we are seeking to mimic the natural historic range of variability, we decided to discard this layer. The Wieslander map had its own issues. Most problematic was the non-systematic spatial error of up to 300 meters, which meant it would not be suitable for comparing specific locations. In addition, crosswalking precisely was impossible because coded vegetation was not necessarily in order of most prevalent vegetation, but instead prioritized tree species over shrubs, and commercially important trees over others. As an example from the handbook states, a plot consisting of 75\% \emph{Quercus kelloggi} (black oak), 15\% \emph{Pinus ponderosa} (ponderosa pine), and 10\% \emph{Pinus lambertiana} (grey pine) would be coded as ponderosa pine, grey pine, black oak. Finally, the Wieslander maps were developed from surveys done in the 1930s, decades after the huge influx of settlers in the 1850s; by the 1930s, vegetation patterns may have already been significantly altered \citep{Thorne2006}. Consequently, the Wieslander map is also not a reliable predictor of land cover type without extensive review of the original data and maps, which would be beyond the scope of this project. 

To confirm these problems, we examined the overlap in land cover types between different maps in ArcGIS. In general, the overlap between EVeg and either the PNV or the Wieslander layers was no better than random, and in many cases it was worse. We decided, in conjunction with Tahoe National Forest staff, to proceed using only the EVeg map, and omit the calibration period of the model from our analysis of the characteristics of the HRV. %This ensured that our analysis of future management scenarios and comparison of spatial metrics between those results and the HRV results was credible.
% in retrospect I wonder if we should have analyzed the configuration more. in the end the biggest problem was probably the lack of crosswalk, since a precise spatial equivalence wasn't assumed.

\subparagraph{Selection of Specific Cover Types}
In the early stages of this project, the team created a suite of land cover types based roughly on the Wildlife Habitat Relationships (WHR) types used in California and by Forest Service managers and planners. These consisted of the WHR types with a few additional types where additional specificity or refinement was desired. For example, Red Fir was split up into two subtypes. The original concept was to begin with the WHR types and modify them as needed based on other attributes in the EVeg layer. However, creating a crosswalk from WHR to the project-specific types also proved problematic. First, we realized that the WHR values were actually derived from the \textsc{calveg} species alliances included in the EVeg layer, but the methodology used was unavailable or missing. The crosswalks we did find were not mutually exclusive and all-inclusive, and do not always make ecological sense \citep{Keeler-Wolf2007,DeBecker1988,Game2005}. This is probably due in part to the fact that WHR is not a mapping classification. It is always derived secondarily. So, we were unable to create consistent rules for mapping from WHR to other types. Others have encountered similar issues:
%
\begin{quote}
WHR has been less successful in differentiating between vegetation types. Because the habitat types are inconsistently defined, a broad familiarity with its detailed descriptions is needed to differentiate among types of similar structure. Although mappers have constructed rules for discriminating among types, difficulties still remain because species dominance varies substantially within some types and broad overlaps in dominant plants occur among types. Other problems arise due to the small number of classes and the inconsistencies in scale among them \citep[p.~23]{Keeler-Wolf2007}
\end{quote}
%
In collaboration with National Forest staff we decided to instead base our land cover types on, at the first order, Presettlement Fire Regime (PFR) types as defined in the Fire Return Interval Departure (FRID) report by \citet{VandeWater2011}. The PFR types, as part of the FRID, were developed through the scientific process and underwent peer review. We used the methodology from the FRID rather than using the second-order WHR classification and trying to reverse-engineer it to fit into our custom land cover types. Thus we created a new structure of cover types in a nested regime, moving from PFR (the coarsest aggregation of \textsc{calveg} types, which included a direct crosswalk from them to PFR types), to Biophysical Settings from LandFire (which were also crosswalked to PFR types in the FRID report), and finally to various local types not otherwise represented, such as xeric and mesic variants of cover types like Mixed Evergreen, and aspen variants, such as Red Fir - Aspen. A mutually exclusive and all-inclusive crosswalk for each land cover type used in this analysis to a single LandFire Biophysical Setting and Presettlement Fire Regime type thus exists.

Extensive geoprocessing was required to prepare the EVeg layer for use in \textsc{RMLands}. Beyond converting the vector data to a raster format, further analysis was required to distinguish east- and west-side areas from one another, and generate the cover type modifications that the team agreed on. Aspen types were created by overlaying an aspen layer onto the vegetation layer and creating combined types (``[type] - Aspen'')where appropriate. Areas mapped as a vegetation type characteristic of early seral (e.g. chaparral) were analyzed and assigned an appropriate forested cover type. Ultramafic\footnote{Ultramafic soils are those created from the weathering of igneous rocks, brought to the earth's surface as magma, where they then cooled. Ultramafic soils are typically shallow, rocky, and nutrient deficient, with high levels of metals uncommon in other soils. Only a few species of plants have evolved to live on them, many of which are endemic to such soils. Plants that do grow mature more slowly and cover the land less continuously than the same plant would on better soil. In the study area, the most common ultramafic rock is serpentine \citep{Safford2004}.} types were created by overlaying a geology layer onto the vegetation layer and performing a similar processing step to create ``[type] - Ultramafic''. Finally, for the Sierran Mixed Conifer and Red Fir cover types, which cover broad swaths of land across elevation and aspect, a xeric to mesic gradient was developed in conjunction with local experts and applied, creating ``[type] - Mesic'' and ``[type] - Xeric''. 

Ultimately, 31 cover types were generated for the buffered project area, as listed in Table~\ref{covertable} and shown in Figure~\ref{fig:covermap}.\footnote{Larger images of all of the input layers are included in Appendix \ref{app:inputs}.}. %A thorough description of geoprocessing steps necessary to recreate this data layer will be available soon. 
As Table~\ref{covertable} demonstrates, most cover types occupy a small extent of the project area. The cover types with an extent of less than 1000 ha within the core project area may have statistically unreliable results; this problem increases as the extent of given cover type decreases. We caution against attempting to make inferences for these less common cover types. However, because the nine cover types that do occur over at least 1000 ha represent approximately 93\% of the core project area, we have high confidence in the landscape-level results. These nine cover types were considered our focal cover types, and were all fully analyzed as part of the historical range of variability assessment. For space and continuity, in the main body of this thesis we discuss only the two most common cover types, Sierran Mixed Conifer - Mesic and Serrian Mixed Conifer - Xeric. Results for the other seven cover types are included in the appropriate appendices. 

%%%%%%%%%%%%%%%%%%%%%%
%%% COVER TABLE %%%%%%
%%%%%%%%%%%%%%%%%%%%%%

\begin{table}[!htbp]
\small
\centering
\caption{List of land cover types developed for this project. Included are the cover type abbreviation, full cover type name, and total area in the buffered project landscape in both acres and hectares. We use the cover type abbreviation in tables and figures for space.}
\label{covertable}
\begin{tabular}{@{}llrr@{}}
\toprule
 \textbf{\begin{tabular}[c]{@{}l@{}}Land Cover \\ Abbreviation\end{tabular}} & \textbf{Land Cover Name}    & \textbf{\begin{tabular}[c]{@{}l@{}}Area \\ Core Only\\ (Hectares)\end{tabular}} & \textbf{\begin{tabular}[c]{@{}l@{}}Area\\ Core+Buffer\\ (Hectares)\end{tabular}} \\ \midrule
\rowcolor[HTML]{CAD6BA} \textsc{agr     }     & Agriculture                                  & 16          & 5,416     \\
                        \textsc{bar     }     & Barren                                       & 2,665        & 8,751     \\
\rowcolor[HTML]{CAD6BA} \textsc{cmm     }     & Curl-leaf Mountain Mahogany                  & 18          & 41        \\
                        \textsc{grass   }     & Grassland                                    & 1,379        & 4,617     \\
\rowcolor[HTML]{CAD6BA} \textsc{lpn     }     & Lodgepole Pine                               & 837         & 2,816     \\
                        \textsc{lpn\_asp}     & Lodgepole Pine with Aspen      & 8             & 31   \\
\rowcolor[HTML]{CAD6BA} \textsc{lsg     }     & Black and Low Sagebrush                      & 0           & 5         \\
                        \textsc{med     }     & Meadow                                       & 1,201        & 3,435     \\
\rowcolor[HTML]{CAD6BA} \textsc{meg\_m  }     & Mixed Evergreen - Mesic                      & 7,273        & 13,547    \\
                        \textsc{meg\_u  }     & Mixed Evergreen - Ultramafic                 & 604         & 1,655     \\
\rowcolor[HTML]{CAD6BA} \textsc{meg\_x  }     & Mixed Evergreen - Xeric                      & 6,768        & 13,771    \\
                        \textsc{mrip    }     & Montane Riparian                             & 732         & 2,216     \\
\rowcolor[HTML]{CAD6BA} \textsc{oak     }     & Oak Woodland                                 & 19          & 4,186     \\
                        \textsc{ocfw    }     & Oak-Conifer Forest and Woodland              & 23,729       & 56,941    \\
\rowcolor[HTML]{CAD6BA} \textsc{ocfw\_u }     & \begin{tabular}[c]{@{}l@{}}Oak-Conifer Forest and \\ Woodland -  Ultramafic\end{tabular} & 1060   & 2,185   \\
                        \textsc{rfr\_asp}     & Red Fir with Aspen                           & 0     		    & 34             \\
\rowcolor[HTML]{CAD6BA} \textsc{rfr\_m  }     & Red Fir - Mesic                              & 8,563  	      & 19,626         \\
                        \textsc{rfr\_u  }     & Red Fir - Ultramafic                         & 294   		    & 321            \\
\rowcolor[HTML]{CAD6BA} \textsc{rfr\_x  }     & Red Fir - Xeric                              & 7,493  	      & 9,989          \\
                        \textsc{sage    }     & Big Safebrush                                & 0     		    & 1,600          \\
\rowcolor[HTML]{CAD6BA} \textsc{scn     }     & Subalpine Conifer                            & 638   		    & 12,543         \\
                        \textsc{scn\_asp}     & Subalpine Conifer with Aspen                 & 0     		    & 6              \\
\rowcolor[HTML]{CAD6BA} \textsc{smc\_asp}     & Sierran Mixed Conifer with Aspen             & 58    		    & 121            \\
                        \textsc{smc\_m  }     & Sierran Mixed Conifer - Mesic                & 57,853 	      & 133,920        \\
\rowcolor[HTML]{CAD6BA} \textsc{smc\_u  }     & Sierran Mixed Conifer - Ultramafic           & 4,124  	      & 9,774          \\
                        \textsc{smc\_x  }     & Sierran Mixed Conifer - Xeric                & 52,198 	      & 91,443         \\
\rowcolor[HTML]{CAD6BA} \textsc{urb     }     & Urban                                        & 114   		    & 782            \\
                        \textsc{wat     }     & Water                                        & 4,058  	      & 8,212          \\
\rowcolor[HTML]{CAD6BA} \textsc{wwp     }     & Western White Pine                           & 273   		    & 510            \\
                        \textsc{ypn     }     & Yellow Pine                                  & 0     		    & 10,499         \\
\rowcolor[HTML]{CAD6BA} \textsc{ypn\_asp}     & Yellow Pine with Aspen                       & 0     		    & 3              \\ \bottomrule
\end{tabular}
\end{table}
\normalsize

\paragraph{Seral Stage}
Seral stage classes combine developmental stage and canopy cover, and are defined for all cover types that undergo succession. Seral stages in this application are based on LandFire structural classes, and were further modified in collaboration with local experts on the Tahoe National Forest. In \textsc{RMLands}, susceptibility to and mortality from natural disturbances varies among seral stages. Unlike the cover grid, the seral stage grid changes dynamically over time in response to simulated succession and disturbance events. The combination of cover type and seral stage forms the basis for characterizing vegetation patterns and dynamics.

The source for the seral stage layer is the Region 5 Existing Vegetation Layer, mapped to the \textsc{calveg} classification. The \textsc{calveg} classification was developed by the Region's Ecology Program in 1978. Within the project area, the Existing Vegetation Layer was developed based on three separate efforts: a satellite-based imagery analysis in 2000, and two orthoimagery analysis completed by contracting firms in 2005. All members of the team discussed potential attributes to be used for this classification, and identified attributes for tree diameter at breast height and cover from above to classify pixels into early, middle, or late development, and open, moderate, and closed canopy. In this application, aspen and shrub types have seral stages that differ from that of the remaining forest types. The other forested types use a consistent set of seral stages.

Extensive geoprocessing was required to prepare this layer for \textsc{RMLands}. Beyond converting the vector data to a raster format, further analysis was required to update the layer to a year 2010 condition. Spatial data on wildfire and timber management history was used to provide a more accurate assessment of seral stage based on estimated stand age. In addition, areas currently mapped as chaparral in the Existing Vegetation Layer were assigned to the early development stage. The full set of seral stages is provided in Table~\ref{condtable} and depicted in Figure~\ref{fig:conditionmap}.

%%%%%%%%%%%%%%%%%%%%%%
%%% CONDITION TABLE %%
%%%%%%%%%%%%%%%%%%%%%%

\begin{table}[!htbp]
\small
\centering
\caption{List of seral stages developed for this project. Seral stages describe developmental stage (e.g. ``early'') and canopy closure (e.g. ``open''). Included are the seral stage abbreviations, and full names.}
\label{condtable}
\begin{tabular}{@{}ll@{}}
\toprule
\textbf{\begin{tabular}[c]{@{}l@{}}Seral Stage \\ Abbreviation\end{tabular}} & \textbf{\begin{tabular}[c]{@{}l@{}}Seral Stage  \\ Name\end{tabular}} \\ \midrule
\rowcolor[HTML]{CAD6BA}   \textsc{ns}             & Non-Seral                     			    \\
                          \textsc{early\_all }    & Early 					                     \\
\rowcolor[HTML]{CAD6BA}   \textsc{mid\_cl    }    & Mid--Closed                    \\
                          \textsc{mid\_mod   }    & Mid--Moderate                  \\
\rowcolor[HTML]{CAD6BA}   \textsc{mid\_op    }    & Mid--Open                      \\
                          \textsc{late\_cl   }    & Late--Closed                   \\
\rowcolor[HTML]{CAD6BA}   \textsc{late\_mod   }   & Late--Moderate                 \\
                          \textsc{late\_cl     }  & Late--Open                     \\
\rowcolor[HTML]{CAD6BA}   \textsc{early\_asp  }   & Early--Aspen                   \\
                          \textsc{mid\_asp   }    & Mid--Aspen                     \\
\rowcolor[HTML]{CAD6BA}   \textsc{mid\_ac    }    & Mid--Aspen Conifer             \\
                          \textsc{late\_ca    }   & Late--Conifer Aspen            \\ \bottomrule
\end{tabular}
\end{table}



\paragraph{Age}
Age represents the number of years since the last stand-replacing disturbance (high mortality wildfire). Because the characteristic species of a given cover type may not immediately establish after a stand-replacing fire, it is likely that the age value is larger than the actual age of the oldest individuals in a stand. Several of the cover types in this area may go through a chaparral-dominated early development stage; in those cases the oldest trees in the stand could be decades older than the formal stand age. In \textsc{RMLands}, age is used to trigger potential successional transitions and to calculate susceptibility to disturbance. In this application, we rounded all modeled and derived ages to the nearest five years (the length of one timestep).

In the HRV analysis, the initial age value assigned to a given cell is not necessarily important to the outcome of the simulation, due to the exclusion of the first (in our case) 40 timesteps from the analyzed results. In the future scenario analysis, the initial age value carries more weight, because the total simulation length is only 18 timesteps.

In this application, we used data from stand exams dating to the 1960s and from recent Forest Service Region 5 Ecology group survey plots to estimate stand age across the buffered project area. We then interpolated that information across the landscape. Due to insufficient data, we were unable to disaggregate the data below the landscape scale to cover type or another more finely resolved classification. We also acknowledge that the stand exam and modern veg plots do not constitute a true sample and were conducted almost exclusively in mid-mature and mature stands of commercially viable trees, thus skewing the results to some unquantifiable degree.

We updated the interpolated data with wildfire and timber management history, and assigned ages to types coded as chaparral in the Existing Vegetation layer to the midpoint of the age spread of early development for the forest cover type to which it was converted. Remaining ages out of compliance with allowed ages for the corresponding seral stage of a given cell were modified to be in compliance, based on the assumption that the seral stage assignment was more accurate that the interpolated age information. The input Age layer, showing the map at timestep 0, is shown in Figure~\ref{fig:agemap}.


\paragraph{Seral stage-Age}
Seral stage-Age represents the age since transitioning to the current seral stage. In \textsc{RMLands} it affects most transitions between seral stages: typically there is a threshold seral stage-age below which transitions do not occur. After creating both the seral stage and age layers, we used a Python function to derive seral stage-age based on the youngest possible age for a cell of that cover and seral stage. For example, if we determine that a particular cell on the landscape has a cover type of Lodgepole Pine, seral stage of Mid Development Closed, and age of 50 years, we take the minimum age for that cover-seral stage combination (10 years old), and subtract it from the age to arrive at a seral stage-age of 40. The same caveats and assumptions that apply to the seral stage and age layers also apply to the seral stage-age layer. The final, original map for the initial seral stage-age is shown in Figure~\ref{fig:condagemap}.


\paragraph{Topographic Position Index}
Our topographic position index (TPI) combines heat load, which is based on aspect and slope, with slope position (Figure~\ref{fig:tpimap}), and ranges from -300 to 300. High values for TPI are correlated with locations on steep, south and west-facing, upper slopes. Low values are correlated with locations on gentle, north and east-facing, valley bottoms. Values in between occur along a gradient of these characteristics. The TPI is scaled to the project area and the region immediately surrounding it, and is therefore a local index only. We implemented this in order to better mimic fire behavior on the landscape. Past research has indicated that fires generally burn more frequently on southerly, steep slopes than on gentle, northerly slopes. Because they are drier and more exposed, and because of the way fuels allow preheating and ready ignition of vegetation as fire travels uphill, the likelihood of overstory tree mortality increases with increasing TPI (these statements may also be interpreted as the inverse, wetter valley bottoms are less likely to produce a crown fire) \citep{North2012,Taylor2003a}. We use TPI to adjust vegetation susceptibility and mortality, as described in the model parameterization section. As part of model calibration, we plot the average canopy cover (over the simulated period) against TPI and calculate the proportional effect of TPI on cover.



\paragraph{Elevation} 
Elevation represents the height above sea level in meters. In \textsc{RMLands}, the elevation layer affects disturbance spread. The elevation grid used in this analysis was a digital elevation model (DEM) provided by the Tahoe National Forest GIS staff and rescaled from 10 m$^2$ to 30 m$^2$ pixels. It is shown as a map in Figure~\ref{fig:elevationmap}.



\paragraph{Slope} 
Slope represents the steepness of a cell as measured in percent and is derived from the elevation layer. Slope is used in GIS preprocessing to define cover types and eligibility for various vegetation treatments. Within \textsc{RMLands}, slope affects disturbance spread. The slope for the study area was derived from the elevation layer described above, and is shown in Figure~\ref{fig:slopemap}.


\paragraph{Aspect} Aspect represents the direction a cell is facing in terms of eight cardinal directions. Flat aspects are also recognized. Within \textsc{RMLands}, aspect affects disturbance spread. The aspect for the study area was derived from the elevation layer described above, and is shown in Figure~\ref{fig:aspectmap}. 


\paragraph{Streams} 
Streams represents linear hydrological features, classified as small, medium or large based on stream order. In this application, the streams layer was created from a line coverage containing hydrography data, including an attribute for stream size or order, by converting to a grid based on the stream size attribute. Streams may inhibit the spread of wildfire in \textsc{RMLands}, depending on both stream size and potential wildfire size. In order to function as a barrier, cells in the stream raster coded as streams much share a side, rather than only a vertex. The stream layer was used to overwrite any cells not coded as ``Water'' in the cover type layer, such that all ``large'' (first order) streams are represented as water in the cover type layer. The final streams input layer is shown in Figure~\ref{fig:streamsmap}.


\paragraph{Buffer/Core} 
This layer identifies and distinguishes the ``core'' project area from the 10 km ``buffer'' added to allow wildfires to initiate outside of and burn beyond the formal project area. Without a buffer, edge effects would alter results for all aspects of the disturbance regime, as well as resulting landscape composition and configuration. A 10 km buffer was selected arbitrarily, but has in the past been sufficient to offset edge effects (McGarigal, personal communication). Because the simulation plays out on the full extent of the core plus the buffer, all input grids are developed to that larger extent as well. To create this raster, the original project polygon was buffered in ArcMap by 10 km, then converted to raster using the same procedure as for other layers. The buffer and core are easily distinguished in Figure~\ref{fig:inputlayermaps}: the core is the interior area delinated by a thick black line, while the buffer is area outside of this line, displayed at a decreased brightness level.


\begin{figure}[!htbp]
  \centering
  \subfloat[][]{
    \centering
	\includegraphics[width=0.4\textwidth]{/Users/mmallek/Tahoe/Report3/images/cover_resized.png}
    \label{fig:covermap}
  } \qquad
  \subfloat[][]{
    \includegraphics[width=0.4\textwidth]{/Users/mmallek/Tahoe/Report3/images/condition_resized.png}
    \label{fig:conditionmap}
  } %\\
  %
	\subfloat[][]{
    \includegraphics[width=0.4\textwidth]{/Users/mmallek/Tahoe/Report3/images/age_resized.png}
    \label{fig:agemap}
  } \qquad
	\subfloat[][]{
    \includegraphics[width=0.4\textwidth]{/Users/mmallek/Tahoe/Report3/images/condage_resized.png}
    \label{fig:condagemap}
  } 
  %
  \subfloat[][]{
    \centering
	\includegraphics[width=0.4\textwidth]{/Users/mmallek/Tahoe/Report3/images/tpi_resized.png}
    \label{fig:tpimap}
  } \qquad
  \subfloat[][]{
    \includegraphics[width=0.4\textwidth]{/Users/mmallek/Tahoe/Report3/images/elevation_resized.png}
    \label{fig:elevationmap}
  } %\\
  %
	\subfloat[][]{
    \includegraphics[width=0.4\textwidth]{/Users/mmallek/Tahoe/Report3/images/slope_resized.png}
    \label{fig:slopemap}
  } \qquad
	\subfloat[][]{
    \includegraphics[width=0.4\textwidth]{/Users/mmallek/Tahoe/Report3/images/aspect_11oct.png}
    \label{fig:aspectmap}
  } 
  %\\
  	\subfloat[][]{
    \includegraphics[width=0.4\textwidth]{/Users/mmallek/Tahoe/Report3/images/streams_resized.png}
    \label{fig:streamsmap}
  } 

  \caption{RMLands input layers. (a) Cover type map (b) Seral stage map (c) Age map at Timestep 0 (d) Seral stage-Age map at Timestep 0 (e) Topographic Position Index (f) Elevation (g) Slope (h) Aspect (i) Streams}
  \label{fig:inputlayermaps}
\end{figure}


\subsection{Model Parameterization}
\label{subsec:hrvmodelparam}

\paragraph{State and Transition Models}
We have created a detailed cover type description document for each cover type in the simulated landscape that experiences transitions between cover class. These documents describe crosswalks to other data layers, detailed accounts of the multiple species characteristic of the cover type, cover type distribution, relationship and response to wildfire, predicted fire return intervals, plus descriptions of each seral stage present within the cover type and their succession and wildfire transition conditions and rates (Appendix \ref{app:covertypedesc}). Each detailed document can be summarized as a state and transition model for a particular cover type, which is implemented in the model by specifying susceptibility to wildfire, rules for vegetational succession, and rules for transitions after a fire event. Figure~\ref{transmodel} shows a generic example state and transition model for the forested cover types.

An important characteristic of \textsc{RMLands} is that it treats fire somewhat different from other landscape succession and disturbance models, which affected how we created and specified the state and transition models. In \textsc{RMLands}, fires spread probabilistically based on the susceptibility of an individual cell. It does not contain a fire model and fuels are not directly incorporated into fire spread. In addition, we do not classify individual fires as a whole to a ``low,'' ``mixed,'' or ``high'' severity status. Some fire ecologists combine fire attributes such as flame length and fire size into their interpretation of the relative ``severity'' of a particular fire \citep{Agee1993}.   Ecologists working at other scales and not working with models often describe ``mixed severity'' regimes \citep[e.g.,]{Kane2013}, which \citet{Collins2010} define as ``stand-replacing patches within a matrix of low to moderate fire-induced effects.'' Because at the 30 m cell size of our model, nearly all fires would be classified as ``mixed severity'' by the prior definition, it becomes moot. Instead, we focus on defining conditions under which transitions among potential states within a given cover type occur or not. All burned cells are evaluated probabilitistically and assigned to a high severity--``high mortality'' outcome or not. All non-high mortality outcomes are considered low mortality. If a cell burns at high severity, then it transitions to the Early seral stage. Recently, some researchers have differed on whether 75\% or 95\% overstory tree mortality is a more appropriate cutoff point for defining a ``stand-replacing'' event \citep{Fule2014,Mallek2013}. In this paper, we use 75\% as our cutoff, which is widely accepted in the literature \citep{Miller2009,Baker2014,Agee2007,Agee1993}. 

To derive probabilities for post-fire transitions, both for transitions to the Early seral stage or to a more open seral stage, we used the VDDT models associated with the Biophysical Settings Models from the LandFire project. From the VDDT models, we used the probabilities of a transition to the early seral stage, a more open canopy seral stage, or of no transition. We ignored the classified type of fire (as replacement, mixed, or low severity), focusing instead on the outcome from fire in terms of the seral stage, if any, to which a cell transitioned after wildfire. High mortality fires are those that result in conversion to early seral (regardless of whether they are called ``replacement'' or ``mixed''). All other fires are considered low mortality. The probability of a high mortality outcome from fire was calculated by dividing the summed probabilities of high mortality fires as defined above by the summed probabilities of all fires.  We also used LandFire data to derive probabilities of succession. We then evaluated and refined the probabilities with input from local experts to capture subtle changes in succession and transition at the project scale.


\begin{figure}[htbp]
\centering
\includegraphics[width=0.8\textwidth]{/Users/mmallek/Tahoe/Report3/images/state_trans_model.pdf}
\caption{Generic state and transition model for all non-shrub seral cover types. Each dark grey box represents one of the seven seral stages for this landcover type. Each column of boxes represents a stage of development: early, middle, and late. Each row of boxes represents a different level of canopy cover: closed (70-100\%), moderate (40-70\%), and open (0-40\%). Transitions between states/seral stages may occur as a result of high mortality fire, low mortality fire, or succession. Specific pathways for each are denoted by the appropriate color line and arrow: red lines relate to high mortality fire, orange lines relate to low mortality fire, and green lines relate to natural succession.} 
\label{transmodel}
\end{figure}

To illustrate the parameterization, in the following tables we present values for the Sierran Mixed Conifer - Mesic cover type model. The target fire return interval for this cover type is 29 years. A fire return interval index is used as the parameter controlling the relative susceptibility to fire of the seral stages within an individual cover type. In addition, the probability of high severity fire leading to at least 75\% overstory mortality is specified for each seral stage (Table~\ref{smcm_fri_phm}). We also specified transition probabilities for natural succession between the early, middle, and late stages of development, as well as between closed, moderate, and open canopy cover. This type of succession also depends on the time in the current seral stage both in terms of the early-middle-late sequence (\emph{Development-Age}) and the specific stage-canopy cover combination (\emph{Seral Stage-Age}) (Table~\ref{smcm_vegtrans}). Finally, probabilities are specified for vegetation transitions after wildfire (Table~\ref{smcm_firetrans}).



% edited 2015-9
\begin{table}[htbp]
\small
\centering
\caption{Relative susceptibility to fire and proportion of high severity fire (at least 75\% overstory tree mortality) probabilities for Sierran Mixed Conifer - Mesic.}
\label{smcm_fri_phm}
\begin{tabular}{lcc}
\hline
\textbf{Seral Stage}    & \textbf{\begin{tabular}[c]{@{}c@{}}Relative Susceptibility \\ to Fire\end{tabular}} & \textbf{\begin{tabular}[c]{@{}c@{}}Probability of \\ High Severity Fire\end{tabular}} \\ \hline
Early (All)     			& 5.5        & 1                 \\
Mid--Closed    				& 2.4        & 0.23              \\
Mid--Moderate  				& 1.6        & 0.17              \\
Mid--Open      				& 1.3        & 0.14              \\
Late--Closed   				& 4.3        & 0.37              \\
Late--Moderate 				& 1.6        & 0.14              \\
Late--Open     				& 1          & 0.09              \\ 
\emph{Target Fire Return Interval}    			& \emph{29 years}  &   \\ \hline
\end{tabular}

\end{table}

\begin{table}[!htbp]
\small
\centering
\caption{Timeframes for transitions between seral stages in \textsc{RMLands} for Sierran Mixed Conifer - Mesic. ``Early to Mid'' and ``Mid to Late'' times are based on the time in a developmental stage, regardless of disturbance history. ``Open to Moderate'' and ``Moderate to Closed'' times are based on the time in a seral stage since a disturbance.}
\label{smcm_vegtrans}
\begin{tabular}{cccc}
\hline
\textbf{\begin{tabular}[c]{@{}c@{}}Seral Stage \\ Transition\end{tabular}} & \textbf{Minimum (years)} & \textbf{Average (years)} & \textbf{Maximum (years)} \\ \hline
Early to Mid 	& 20      & 26      & 40      \\
Mid to Late 	& 100     & 113     & 150     \\
\begin{tabular}[c]{@{}c@{}}Open to Moderate or\\ Moderate to Closed\end{tabular}  & 15      & 21      &    ---     \\ \hline
\end{tabular}

\end{table}


\begin{table}[!htbp]
\small
\centering
\caption{Transition probabilities for Sierran Mixed Conifer - Mesic following low mortality fire.}
\label{smcm_firetrans}
\begin{tabular}{lcc}
\hline
\textbf{Seral Stage Transition} & \textbf{Probability}\\
\hline
Mid--Closed to Mid--Moderate     	& 0.53   	\\
Mid--Moderate to Mid--Open    	& 0.36		\\
Late--Closed to Late--Moderate	& 0.54    \\
Late--Moderate to Late--Open     	& 0.24    \\
\hline
\end{tabular}
\end{table}

Transitions between Early and Middle Development, and between Middle and Late Development are governed by the time in the Early or Middle stage (canopy cover usually does not affect these probabilities). These transitions may begin at the minimum time in a specified \emph{Development-Age}, and proceed at rates that vary across cover types. Table~\ref{smcm_vegtrans} displays the average \emph{Seral Stage-Age} of transition. If a cell reaches the maximum stage-age listed, its probability of transitioning goes to 1. 

Transitions between the canopy cover types occur within one developmental stage: i.e., between Mid--Open and Mid--Moderate, but not between Mid--Open and Late--Moderate. These transitions are governed by the time in the full seral stage specified since the last disturbance. This ``years since'' value may be affected by a low mortality fire, a transition between developmental stages, or a transition between canopy cover levels. Similarly to the developmental transitions, the shift from, for example, Mid--Open to Mid--Closed may begin when the minimum time is reached, and also proceeds at rates that vary across cover types. No maximum age is specified for this type of transition.

\paragraph{Disturbance Parameters} 
\label{subsubsec:distparams}

%\begin{adjustwidth}{5ex}{0pt}
\begin{itemize}
\item \emph{Climate:} The climate parameters are based on a rescaling of the Palmer Drought Severity Index (PDSI). PDSI is a long-term measure of drought, on the scale of months to years. It is based on precipitation and temperature and incorporates soil moisture. Resconstructed PDSI values for summer months during the historic period of this project (1550-1850) are available from the National Oceanic and Atmospheric Administration (\burl{http://www.ncdc.noaa.gov/paleo/pdsi.html}). We used datasets from \citet{Zhangetal.2004} and \citet{Cook2004}. These data are summarized at large scales; for example, the \citet{Cook2004} data are calculated for a grid with points spaced at 2.5\textdegree. We selected the five closest points to the center of the project area from these two datasets and calculated the inverse distance-weighted mean of the values. We then converted the yearly data into five-year averages to align with the five-year timesteps in our model. By recentering the mean value around 1 and then taking the inverse, we create a dataset in which a value of 1 is neither wetter nor dryer than average, values between 0 and 1 represent wetter-than-normal timesteps, and values greater than 1 represent dryer-than-normal timesteps (Figure~\ref{pdsi}). Climate interacts with other disturbance parameters in \textsc{RMLands}, including initiation, susceptibility, and spread.

\begin{figure}[htbp]
\centering
\includegraphics[height=0.3\textheight]{/Users/mmallek/Documents/Thesis/Plots/pdsi/hrv-bigtext.png}
\caption{Palmer Drought Severity Index, rescaled, inverted, and presented as a 5-year average for the ``historical'' period in this study (1550-1850).} 
\label{pdsi}
\end{figure}

%\medskip

%\noindent 
\item \emph{Susceptibility:} Cover and seral stage are both inputs to susceptibility. Cover modifies susceptibility via the ability to specify the influence of topographic position on susceptibility (Table~\ref{covtpi}). The magnitude of this effect is estimated as a potential reduction in susceptibility of 30\% between the minimum and maximum TPI values used in the model. We implement this by using a logistic function to convert the TPI grid values into an appropriate multiplier within the susceptibility equation:

$$\text{TPI Susceptibility Factor} = L + \frac{R-L}{1+e^{k(x_0-x)}}$$

in which $L= 0.7$, $R=1$, slope $k=1$, inflection point $x_0=0$, and $x=\text{TPI}$.%at a given cell
It therefore simplifies to 

$$\text{TPI Susceptibility Factor} = 0.7 + \frac{0.3}{1+e^{-x}}$$

In this way, when the TPI Susceptibility Factor = 1, TPI has no effect. This happens only when the TPI value at an individual cell is zero. The effect of this is that areas with low TPI (generally north-facing and flatter slopes) burn less frequently than areas with high TPI (generally south-facing and steeper slopes).


\begin{table}[htbp]
\small
\centering
\caption{Cover types whose susceptibility is modified by Topographic Position Index. All cover types are modified in the same way.}
\label{covtpi}
\begin{tabular}{ll}
\hline
\multicolumn{2}{c}{\textbf{Cover Types with TPI Adjustment}} \\
\hline
Grassland     									& Red Fir - Mesic   					\\
Lodgepole Pine    								& Red Fir - Ultramafic					\\
Mixed Evergreen - Mesic							& Red Fir - Xeric    					\\
Mixed Evergreen - Ultramafic     				& Sierran Mixed Conifer - Mesic    		\\
Mixed Evergreen - Xeric 						& Sierran Mixed Conifer - Ultramafic 	\\
Montane Riparian								& Sierran Mixed Conifer - Xeric 		\\
Oak Woodland 									& Western White Pine					\\
Oak-Conifer Forest and Woodland 				& Yellow Pine 							\\
Oak-Conifer Forest and Woodland - Ultramafic 	&										\\
\hline
\end{tabular}

\end{table}

Seral stage further modifies susceptibility. We use the Weibull cumulative distribution function and specify a scale parameter $\lambda$ (mean return interval), shape parameter $k$, and the reset point for the function (\emph{age since high mortality disturbance} or \emph{age since any disturbance}). The fire return index for the seral stage is used as a calibration parameter and was initially set as equal to the mean return interval values provided in analogous LandFire Biophysical Setting types \citep{Landfire2007}. Some modifications were made based on consultation with Forest Service staff. All fire index values within a cover type were modified as a group and kept relative to one another even as the magnitude of the index was adjusted. The relative difference for Sierra Mixed Conifer - Mesic is shown in Table~\ref{smcm_fri_phm}; these values for each cover type are included in the cover type description documents (Appendix \ref{app:covertypedesc}). We set $k=3$ for all cover types and seral stages. We selected between (\emph{age since high mortality disturbance} and \emph{age since any disturbance}) based on whether wildfires in that cover type are climate-driven (in which case we select the former) or fuels-driven (in which case we select the latter) (Figure~\ref{howdriven}).

\begin{table}[htbp]
\small
\centering
\caption{Cover types sorted by whether wildfire disturbance in them is characterized by fuels present or overarching climatic conditions. If the likelihood of wildfire depends on the accumulation of fuels, the value of $x$ (``time since'') reverts to 0 after any disturbance. If the likelihood of wildfire depends primarily on climate and weather conditions, the value of $x$ reverts to 0 only after a high mortality disturbance.}
\label{howdriven}
\begin{tabular}{ll}
\hline
\textbf{Fuel-Driven Cover Types} 				& \textbf{Climate-Driven Cover Types}	\\
\hline
Curl-leaf Mountain Mahogany 					& Agriculture   						\\
Grassland     									& Big Sagebrush 						\\
Lodgepole Pine    								& Black and Low Sagebrush				\\
Meadow											& Lodgepole Pine with Aspen 			\\
Mixed Evergreen - Mesic							& Montane Riparian						\\
Mixed Evergreen - Ultramafic     				& Red Fir with Aspen   					\\
Mixed Evergreen - Xeric 						& Red Fir - Mesic    					\\
Oak Woodland 									& Red Fir - Ultramafic 					\\
Oak-Conifer Forest and Woodland 				& Red Fir - Xeric 						\\ 	
Oak-Conifer Forest and Woodland - Ultramafic 	& Subalpine Conifer 					\\
Sierran Mixed Conifer - Ultramafic 				& Subalpine Conifer with Aspen 			\\
Sierran Mixed Conifer - Xeric 					& Sierran Mixed Conifer with Aspen 		\\
Urban 											& Sierran Mixed Conifer - Mesic 		\\
Yellow Pine 									& Western White Pine 					\\
												& Yellow Pine with Aspen 				\\
\hline
\end{tabular}
\end{table}


%\medskip

%\noindent 
\item \emph{Initiation:} In \textsc{RMLands}, parameters for initiation are used as calibration parameters. The probability of wildfire initiation is a function of its susceptibility to wildfire and the climate modifier value for that timestep, and is applied at the cell level. The ignition calibration coefficient refers to the number of attempted ignitions per 100,000 ha per year. For the HRV simulation, we set this coefficient at 42. We applied the coefficient evenly across the landscape based on local expert knowledge of lighting strike locations in the area. Fires may be initiated anywhere within the project area or the 10 km buffer around it. The total area cover within that boundary is 409,411 ha, so up to 860 fire starts were possible during each 5-year timestep in our simulation (not all potential ignitions result in fire). Climate also influences initiation.

%\medskip

%\noindent 
\item \emph{Spread:} The probability of fire spread in \textsc{RMLands} is a function of climate, susceptibility to wildfire, potential wildfire size, wind, spotting, relative elevation, and presence of streams. The first two are described above. The disturbance size distribution that regulates potential fire size was created by analyzing the size distribution of all mapped fires in the Northern Sierra \textsc{calveg} mapping zone and west of the Sierran crest, available from the Forest Service and the California Department of Forestry and Fire Protection, which goes back to approximately 1900. 

Wind is incorporated in two parts. First, a prevailing wind direction for the fire is selected probabilistically from the eight cardinal directions. To compute the wind distribution values, we first consulted local experts to determine the dates of fire season (May 15 to October 15) and burning period times (1000 hours to 1800 hours). We then downloaded all available historic wind direction data from 6 local weather stations (Rice Canyon, Saddleback, Downieville, White Cloud, Emigrant Gap, and Blue Canyon, Figure~\ref{weather}). Data from all weather stations was weighted equally. After the wind direction is selected, fires are able to grow in all directions, but are relatively more likely to spread with wind than against it. We parameterized the influence of \emph{relative wind} as a reduction in spread likelihood. Thus, spread in the same direction as wind has a neutral effect, spread at $\ang{45}$ angles is reduced by 30\%, spread at $\ang{90}$  angles is reduced by 70\%, spread at $\ang{135}$ angles is reduced by 90\%, and spread opposite the prevailing wind direction is reduced by 95\%. 

\begin{figure}[htbp]
\centering
\includegraphics[width=0.3\textheight]{/Users/mmallek/Tahoe/Report3/images/weather.png}
\caption{Weather stations used to inform wind direction parameters. Weather stations are denoted by red circles. A black boundary line identifies the study area.}
\label{weather}
\end{figure}

Relative elevation also modifies spreading potential. We parameterized the model such that spread downhill is extremely unlikely. Spotting and the extent to which streams act as barriers to spread are affected by the fire size. As fires become larger, their probability of spotting and spotting distance increases. Similarly, streams function as a barrier to smaller fires, but large fires are able to spread past streams regardless of size. This decision is based on the idea that large fires are more influenced by wind and climatic conditions. Stream size does impact smaller fires; the largest streams and rivers are usually an effective barrier to smaller fires, although even fairly small fires often spread past intermittent and small perennial streams. 


\item \emph{Mortality:} Cover and seral stage are both inputs to mortality. Cover modifies susceptibility via the ability to specify the influence of topographic position on mortality (Table~\ref{covtpi_mort}). The magnitude of this effect is estimated as a potential reduction in mortality of 30\% between the minimum and maximum TPI values used in the model. We implement this by using a logistic function to convert the TPI grid values into an appropriate multiplier within the mortality equation:


$$\text{TPI Mortality Factor} = L + \frac{R-L}{1+e^{k(x_0-x)}}$$

in which $L= 0.7$, $R=1$, slope $k=1$, inflection point $x_0=0$, and $x=\text{TPI}$.%at a given cell
It therefore simplifies to 

$$\text{TPI Mortality Factor} = 0.7 + \frac{0.3}{1+e^{-x}}$$

In this way, when the TPI Mortality Factor = 1, TPI has no effect. This happens only when the TPI value at an individual cell is zero. The effect of this is that areas with low TPI (generally north-facing and flatter slopes) are less likely to be high severity than areas with high TPI (generally south-facing and steeper slopes).

\begin{table}[htbp]
\small
\centering
\caption{Cover types whose mortality is modified by Topographic Position Index.}
\label{covtpi_mort}
\begin{tabular}{ll}
\hline
\multicolumn{2}{c}{\textbf{Cover Types with TPI Adjustment}} \\
\hline
Grassland     									& Red Fir - Mesic   					\\
Lodgepole Pine    								& Red Fir - Ultramafic					\\
Mixed Evergreen - Mesic							& Red Fir - Xeric    					\\
Mixed Evergreen - Ultramafic     				& Sierran Mixed Conifer - Mesic    		\\
Mixed Evergreen - Xeric 						& Sierran Mixed Conifer - Ultramafic 	\\
Montane Riparian								& Sierran Mixed Conifer - Xeric 		\\
Oak Woodland 									& Western White Pine					\\
Oak-Conifer Forest and Woodland 				& Yellow Pine 							\\
Oak-Conifer Forest and Woodland - Ultramafic 	&										\\
\hline
\end{tabular}
\end{table}

Seral stage further modifies mortality. We extracted the likelihood of mortality from the VDDT models built during the LandFire project, as described at the beginning of section~\ref{subsec:hrvmodelparam}. As an example, these probabilities for Sierran Mixed Conifer - Mesic are provided in Table~\ref{smcm_fri_phm}.



\end{itemize}
%\end{adjustwidth}

\subsection{Model Calibration}
Although \textsc{RMLands} is a process-based model with parameters sourced from the literature, our team had greater confidence in some parameters than others, especially as to how they function within the \textsc{RMLands} framework. Consequently, we calibrated our model by iteratively adjusting certain parameters in which we had less confidence about the appropriate values until the outputs were tuned to a set of parameters in which we had high confidence. Specifically, we manipulated the ignition calibration coefficient and the fire return index and measured calibration success based on conformity to pre-specified rotation values at the cover type level. Fire return index values were changed by a constant multiplier across all seral stages of a given cover type; cover types were modified as groups but the index ratios within them were maintained. We set our calibration target as rotation values for the nine focal cover types within 10\% of the original target rotations. Target values were based on empirical published values and local expert opinion. We chose rotation as the calibration target because targets were available from the literature and because fire rotation is a fundamental measurement that \textsc{RMLands} was designed to capture. In addition, using rotation ties calibration to a parameter that is relateable to Forest Service staff and that can be used at the landscape scale as a target by managers in various programs.

For example, while calibrating, the target rotation for Sierran Mixed Conifer - Mesic was 29 years. We adjusted the input seral stage fire return index by multiplying it by different constants, eventually arriving at an increase by a factor of 24 from the original calculated ratio values. That is, each initial scale parameter value was multiplied by 24 in order to modify its susceptibility to fire without changing the relative susceptibility among its seral stages. 

We verified that the model was functioning properly by evaluating a few different outputs. First, we visually inspected output grids demonstrating wildfire extents to verify that they were similar to actual wildfire perimeters. Second, we plotted the actual disturbance size distribution against the expected distribution (Table~\ref{fig:dsize}). Last, we examined the results of our implementation of the topographic position index.

The topographic position index value for a given cell acts as an input into the susceptibility and mortality values otherwise defined for that cover type and seral stage combination. Early development and open canopy seral stages tend to result from fire, and we predicted that an increase in fires and in the likelihood of high mortality fire would lead to a decrease in the average canopy cover values for cells with large TPI values. Table~\ref{tab:tpi_cc} in Appendix \ref{app:full-results} displays the results for this simulation for the nine most common cover types. All show decreased average canopy cover as TPI increases, with the decrease ranging from 5.7\% in xeric mixed evergreen forest to 28.8\% in ultramafic oak-conifer forests. Figure \ref{fig:tpi_cc_smc} shows the plotted data and fitted linear regression line for mesic and xeric sierran mixed conifer forests. Figure \ref{fig:averagecc} is a map displaying average canopy cover across the landscape for the full simulated HRV timeframe, excluding the equilibration period. In general, return intervals and canopy cover varied spatially across the forest and decreased with increasing TPI, reflecting our parameterization, which was based on the theory that higher, more southerly aspects are drier and more susceptible to fires. In mesic mixed conifer forests, canopy cover decreased by about 13\% when comparing minimum to maximum TPI, from an average of 49\% to an average of 43\%. In xeric mixed conifer forests, canopy cover decreased by about 25\% when comparing minimum to maximum TPI, from an average of 36\% to an average of 27\% (Table~\ref{tab:tpi_cc_smcs}).

% figure redone
\todo{this figure should have grey for water/barren; looks like high canopy cover now}
\begin{figure}[!htbp]
\centering
\includegraphics[width=0.8\textwidth]{/Users/mmallek/Documents/Thesis/maps/hrv_tpi.pdf}
\caption{Smoothed visualization of the average canopy cover across the project area over the course of the simulation. Higher percent cover is shown in dark blue, transitioning to red where average percent cover was low. Water is shown in blue; barren is shown in grey.}
\label{fig:averagecc}
\end{figure}

% figure redone
\begin{figure}[!htbp]
\centering
\includegraphics[width=.8\textwidth]{/Users/mmallek/Documents/Thesis/Plots/tpi/hrv-facet-smc.png}
\caption{Average canopy cover for Sierran Mixed Conifer Mesic and Xeric during the simulated HRV. Each blue point represents one pixel of an individual cover type on the landscape grid. The black line is the result of a linear regression fit to the data. Table \ref{tab:tpi_cc} provides the numerical representation of the shift from minimum to maximum TPI values for each cover type. (a) Sierran Mixed Conifer - Mesic; (b) Sierran Mixed Conifer - Xeric.} 
\label{fig:tpi_cc_smc}
\end{figure}

%redone 9/15
\begin{table}[!htbp]
\centering
\caption{The percent change in canopy cover from the minimum TPI value for that cover type to the maximum TPI value. Results for Sierran Mixed Conifer Mesic and Xeric shown here; results for other focal cover types available in Appendix~\ref{app:full-results}}.
\label{tab:tpi_cc_smcs}
\begin{tabular}{@{}lrrrrr@{}}
\toprule
\small \textbf{\begin{tabular}[c]{@{}l@{}}Cover \\ Name\end{tabular}} & \small \textbf{\begin{tabular}[c]{@{}l@{}}Minimum \\ TPI\end{tabular}} & \small \textbf{\begin{tabular}[c]{@{}l@{}}Maximum \\ TPI\end{tabular}} & \small \textbf{\begin{tabular}[c]{@{}l@{}}Average Canopy \\Cover at \\ Minimum TPI\end{tabular}} & \small \textbf{\begin{tabular}[c]{@{}l@{}}Average Canopy \\ Cover at \\ Maximum TPI\end{tabular}}  & \small \textbf{\begin{tabular}[c]{@{}l@{}}Percent \\ Change in \\ Canopy \\ Cover\end{tabular}} \\ \midrule
\textsc{smc\_m   }    & -300                 & 300                  & 55.5       & 50.4              & -9.3      \\
\textsc{smc\_x   }    & -300                 & 300                  & 27.6       & 21.9              & -20.5     \\ \bottomrule
\end{tabular}
\end{table}








%%%
%in a nutshell, we have to calibrate the model because we it's not completely mechanistic and we're making guesses on a lot of things
%so we pick a few model parameters that we have high confidence in, and decide  not to change those
%and then we pick a few model outputs that we have confidence in
%because our goal is to simulate a regime we believe we already understand
%so we want it to look "right"
%so we figure if we can make the model outputs agree with the numbers we're pretty sure are right, then we trust the other outputs where we weren't totally sure what to expect
%and we do this by adjusting the parameters that we have lower confidence we got right hte first time, or that don't relate to the real world direclty and mechanistically


% updated 9/13
\begin{figure}[!htbp]
  \centering
    \centering
    \includegraphics[height=0.3\textheight]{/Users/mmallek/Documents/Thesis/Plots/dsize/hrv-ggplot.png}
  \caption{Side by side barplot of the observed and target wildfire size distribution for our 500-timestep long run of the model.}
  \label{fig:dsize}
\end{figure}

\subsubsection{Model Execution}
During the calibration phase of the model, a typical simulation was three runs of 200 timesteps each. The equilibration period of 40 timesteps was chosen based on visual analysis of the disturbed area and rotation plots from the combined runs. Once calibration was complete, we conducted one run of 500 timesteps in order to capture multiple disturbance and succession cycles across the most common cover types. Each timestep represents five years. The five-year timestep was chosen based on the short fire return intervals recorded from dendrochronology analysis in the literature and our desire to capture these very short return intervals in the simulation.

\subsection{Data Analysis}
\label{subsec:dataanalysis}

\paragraph{Disturbance Regime} We quantified the following overall temporal and spatial characteristics of the wildfire disturbance regime:
\begin{itemize}
	\item \emph{Disturbed Area:} We calculated disturbed area for each timestep, divided into low mortality and high mortality disturbance, and summed to produce an ``any mortality'' statistic. We summarize the results for minimum, maximum, mean, and median area disturbed as a proportion of the total area eligible for disturbance for the full simulation excluding the equilibration period (460 timesteps, or 2300 years). Because it can be difficult to visualize what our quantitative results look like, we include several maps that illustrate the results, demonstrating that model results are spatially-explicit and realistic. To do this, we include maps of the landscape illustrating the $5^{\text{th}}$, $50^{\text{th}}$, $95^{\text{th}}$, and mean area burned during the simulation, plus a example 4-timestep sequence illustrating changes to the seral stage pattern for mesic mixed conifer forests due to successional and disturbance procceses. Finally, we use a histogram to display the distribution of wildfire extents during the simulation, excluding the equilibration period.
	\item \emph{Disturbance Frequency:} We calculated the number of years between disturbances exceeding a particular threshold in total disturbed area. We report the frequency of timesteps during which thresholds of at least 10\%, 25\%, or 50\% of the landscape experienced wildfire.
	\item \emph{Climate Effect:} Climate interacts with several components of the model. We present plots illustrating the value of the climate parameter by timesteps concurrently with the area disturbed per timestep. It is not practical to further illustrate its effect everywhere, and in some cases its influence is not easily separated from the other inputs to the model. 
	\item \emph{Rotation Period:} We calculated the rotation period---the number of years required to burn an area equivalent to the total eligible area---for each cover type within the project area. We report the rotation values for low mortality fire, high mortality fire, and any fire for each of the nine focal cover types individually and the study area as a whole.
	\item \emph{Return Interval:} We summarized the cell-specific population mean return interval---the average number of years between disturbances at a single cell---and present it as the distribution of the percentage of eligible cells that experienced each possible mean return interval. We use histograms to visualize the distribution of this return interval for low mortality fire, high mortality fire, and any fire, along with their median values. This method is based on the full landscape results and is equivalent to the cell-specific grand mean return interval for a given cover type across the landscape. We also display this result spatially as a map showing the population fire return interval across the landscape.
\end{itemize}

\paragraph{Vegetation Response} 

\par \emph{Landscape Composition:} We quantified the distribution and dynamics of landscape composition by cover type. For our single 2500 year simulation (with 200 year equilibration period), we summarize the results in a table and graphically. For the tabular results, we present $5^{\text{th}}$, $25^{\text{th}}$, $50^{\text{th}}$, $75^{\text{th}}$, and $95^{\text{th}}$ percentiles of the distribution. We compared the current landscape seral stage distribution to this simulated historic range of variability to determine whether the current landscape deviates, and to what degree, from the HRV. 

Using a stacked bar plot, we visualize the proportion of the total area of a given cover type occurring at each seral stage, for each timestep in the model. In addition, we show a bar plot of the current seral stage distribution, allowing a visual comparison between current conditions and the historical range of variability in the distribution of the seral stages. While the bar plots are useful for visualizing the cover-seral stage dynamics, box plots more simply $5^{\text{th}}--95^{\text{th}}$ percentile distribution of interest and compare that to the current landscape values.






\emph{Landscape Structure and Patterns:} We used \textsc{Fragstats} \citep{Fragstats2012} to compute several landscape-level and class-level metrics that summarize landscape structure over the course of the simulation. We present the results in a series of tables and figures. 

The descriptions in Appendix~\ref{app:metricdescriptions} are intended as a general introduction to the \textsc{Fragstats} metrics; for a much more detailed and mathematical description of all \textsc{Fragstats} metrics, see the \href{http://www.umass.edu/landeco/research/fragstats/documents/fragstats.help.4.2.pdf}{documentation}. Each metric is computed on the study area for a single timestep and the results are displayed in tabular format by quantiles and in graphical format with line graphs and boxplots. Table~\ref{tab:fragland-desc} summarizes the \textsc{Fragstats} metrics selected as focal metrics to provide a simple and understandable explanation of the characteristics of landscape structure during the simulated HRV. We selected metrics to represent commonly identified groups of landscape metrics: patch area and edge, patch shape complexity, core area, aggregation, and diversity \citep{McGarigal2015}. It is fairly intuitive to understand how these metrics may be affected by natural disturbance and human management efforts, thus allowing us to describe the HRV and develop suggestions tying management actions to results for these metrics.

One of the principal purposes of gaining a better quantitative understanding of the historic reference period is to know whether recent human activities have caused landscapes to move outside their historical range of variability. 


\begin{table}[!htbp]
\footnotesize
\centering
\caption{A subset of \textsc{Fragstats} metrics we selected to emphasize in order to provide a parsimonious explanation of the variability in landscape structure during the simulated HRV. An `X' in the landscape or class column denotes whether that metric is calculated at that level. Abbreviations are included because they are used in tables and figures later in the document and appendices for space.} 
\label{tab:fragland-desc}
%
\begin{tabular}{@{}llccc@{}}
\toprule
{\bf Metric}                    & {\bf Abbreviation} & {\bf \begin{tabular}[c]{@{}c@{}}Landscape-\\ level\end{tabular}} & {\bf \begin{tabular}[c]{@{}c@{}}Class-\\ level\end{tabular}} & {\bf Category} \\ 
\midrule
Edge Density                    & \textsc{ed} 			& X        & X     & area and edge metric		\\ 
Area-Weighted Mean Area         & \textsc{area\_am}  	& X        & X     & area and edge metric		\\
Area-Weighted Mean Shape        & \textsc{shape\_am} 	& X        & X     & shape metric 				\\
Area-Weighted Mean Core Area    & \textsc{core\_am}  	& X        & X     & core area metric		\\
Contagion                       & \textsc{contag} 		& X        & --    & aggregation metric		\\
Clumpiness Index                & \textsc{clumpy} 		& --       & X     & aggregation metric		\\
Simpson’s Evenness Index        & \textsc{siei}      	& X        & --    & diversity metric		\\
\bottomrule
\end{tabular}
\end{table}



For both the composition and pattern metrics, we quantified the current landscape's departure from the HRV conditions by calculating a departure index for each cover-seral stage type and each \textsc{Fragstats} metric. We summarized the distribution of each metric calculated over the length of the simulation, minus the equilibration period. We computed the $5^{\text{th}}$, $25^{\text{th}}$, $50^{\text{th}}$, $75^{\text{th}}$, and $95^{\text{th}}$ percentiles of the distribution of observed values. We calculated a current percentile of the range of variability value (\%RV) by computing where along the $0^{\text{th}}-100^{\text{th}}$ percentile range of variability for the simulated historical period the current landscape metric value falls. If the current landscape is outside of the HRV, its current \%RV value is noted as either 0 or 100. The departure index indicates the distance from the 50$^{\text{th}}$ percentile value for a given metric. A value of 0 means that the current value is identical to the median from the simulated HRV, and a value of either less than -95 or greater than 95 means that the current value is below or above and outside the simulated HRV. This departure index is computed by subtracting 50 from the \%RV, then dividing by 50 and multiplying by 100. Thus, for the landscape metric \emph{Patch Density}, 19.507 is equivalent to the 32$^{\text{nd}}$ percentile of observations during the HRV simulation, and the departure index is $(39-50)/50*100 = -22$). This value is within the HRV for the landscape. However, the landscape metric \emph{Edge Density} is 100, because $128.875 > 125.316$, the largest value observed during the HRV simulation. Therefore, edge density at the landscape level is outside the HRV.

\todo{I think I'd like to use the same standards I developed for FRV, actually. Based on interquartile range (within), secondary range (moderate), or outside (full).}

\clearpage
