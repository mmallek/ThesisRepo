% !TEX root = master.tex

\chapter{Historical Range of Variability}
\label{ch:hrv}
\setcitestyle{notesep={:},aysep={}}

In this chapter I present the methods, results, and discussion for the historical range of variability analysis. The historical period defined for this project was 1550 to 1850. Because I did not have consistent and complete data on wildfire and vegetation growth and pattern during this period, I used simulations to incorporate the data that do exist, and generate new datasets of otherwise unobservable landscape trajectories. Many landscape disturbance and succession models exist, with different input requirements, purposes, intended geographic applicability, etc. Typically, each timestep of the model produces a set of GIS layers, which can then be analyzed to quantify trajectories and patterns; in other words, to describe the range of variability. In this study I used the Rocky Mountain Landscape Simulator (RMLands). Originally developed for use in the Rocky Mountains, an early phase of this project was to adapt it to the Sierra Nevada. RMLands is disturbance and succession model software that is stochastic, spatially explicit, and raster-based. The model was parameterized to simulate passive management of fire and vegetation. That is, I did not simulate vegetation treatments, nor did I attempt to emulate additional fire suppression efforts.

% !TEX root = master.tex

\section{Methods}
\label{sec:hrvmethods}

% removed study area, it's now in the Introduction Chapter

\subsection{Modeling Framework}
\label{sec:modelframe}

A partial introduction to \textsc{RMLands} is included in Chapter~\ref{CH1}, but here I provide a more detailed description of the model and how I used it. 
% intro to RMLands in Introduction Chapter (modeling framework, methodological limitations)

\subsubsection*{Input Layers}
\label{subsec:hrvinputlayers}

All input layers to \textsc{RMLands} must be custom-built to work with the software. For technical details on the data structure requirements of \textsc{RMLands}, see Appendix \ref{app:inputs}. A brief overview of each input layer is included below.

\paragraph*{Cover} Cover type is based on the potential or current natural vegetation of a site and includes both natural and anthropogenic cover types. For example, cover types include not only Lodgepole Pine, Sierran Mixed Conifer, and Red Fir, but also Barren and Agriculture. Succession pathways are defined uniquely for each cover type, susceptibility to natural disturbances varies among cover types, and suitability or eligibility for various vegetation treatments varies among cover types. Cover is a static (constant) grid and therefore provides a fixed template upon which disturbance and succession processes play out over time. 

The source for the cover layer is the Region 5 Existing Vegetation Layer (``EVeg''), first mapped to the \textsc{calveg} classification developed by the Region's Ecology Program in 1978. When deciding on land cover types, including determining xeric and mesic subtypes, our focus was to best represent the study area and the surrounding landscape. I used the \textsc{calveg} Mapping Zone boundary for the ``North Sierra'' (Figure~\ref{calveg}) as our focus for defining vegetation and disturbance, including susceptibility, response to fire, and fire size and distribution. Within the study area, the EVeg layer was developed based on three separate efforts: a satellite-based imagery analyses in 2000, and two orthoimagery analysis completed by contracting firms in 2005. Generally, specific cover type names were derived from the California Fire Return Interval Departure (FRID) report by \citet{VandeWater2011}. I also considered information from \emph{A Guide to Wildlife Habitats of California}, popularly known as the ``Wildlife Habitat Relationship (WHR)'' cover types \citep{WHR1988}. 

\begin{wrapfigure}{R}{0.5\textwidth} % use a capital R to allow figure to float
\includegraphics[width=0.5\textwidth]{/Users/mmallek/Tahoe/Report3/images/CALVEGmappingzones.png}
\caption{\small CALVEG Mapping Zones. These zones meet U.S. Forest Service standard at national and regional levels. These ecological provinces are associated with dozens of vegetation alliances, which are used to classify vegetation in spatial data products. I used vegetation alliance definitions for the North Sierra zone to classify the land cover spatial data shared by the U.S. Forest Service.} 
\label{calveg}
\end{wrapfigure}

\subparagraph*{Alternative Cover Layers}
The original intent of our team was to utilize two separate cover layers: one for the historical reference period, and one for the current period to be used in projections of future scenarios. Two layers were identified as potentially suitable for the historic analysis: a map created from forest survey and inventory efforts under Albert Wieslander conducted between 1928 and 1940 (``Wieslander'') \citep{Thorne2006}, and a map of Potential Natural Vegetation (PNV) created by a Forest Service Enterprise Team for the Tahoe National Forest in the 2000s (Forest Service internal GIS data). Our intent was to use the PNV, Wieslander, or a combination thereof to derive the land cover layer for the HRV phase of the project. 

In order to validate the historical maps, I needed to develop a crosswalk between the vegetation type methodologies for the EVeg, PNV, and Wieslander maps. I also examined the spatial consistency in cover types across the maps. With significant assistance from the Tahoe National Forest, I attempted to create a crosswalk from these maps to the set of land cover types to be used in the project. However, I was unable to develop a consistent and comprehensive set of rules for this purpose. A major reason for this is that both the PNV and Wieslander maps used species lists, rather than assemblages (as in \textsc{calveg} and LandFire). For example, Sierran mixed conifer forests do not appear as a dominant ``cover type'' in the PNV map. The Wieslander maps may contain an internal crosswalk to a mixed conifer alliance, but in my study area I observed this to be true only rarely. 

In addition, the PNV map contained a more significant error: I learned that, for the purposes of the modeling used to create the PNV map, ``potential natural vegetation'' was defined as the so-called ``climax'' community that would develop in the complete absence of all disturbance, including natural disturbances like wildfire \citep{Fites1993}. Since my goal was to mimic the natural historic range of variability, I decided to discard this layer. The Wieslander map had its own issues. Most problematic was the non-systematic spatial error of up to 300 meters, which meant it would not be suitable for comparing specific locations \citep{Thorne2006}. In addition, crosswalking precisely was impossible because coded vegetation was not necessarily in order of most prevalent vegetation, but instead prioritized tree species over shrubs, and commercially important trees over others. As an example from the handbook states, a plot consisting of 75\% \emph{Quercus kelloggi} (black oak), 15\% \emph{Pinus ponderosa} (ponderosa pine), and 10\% \emph{Pinus lambertiana} (grey pine) would be coded as ponderosa pine, grey pine, black oak \citep{Thorne2006}. Finally, the Wieslander maps were developed from surveys done in the 1930s, decades after the huge influx of settlers in the 1850s; by the 1930s, vegetation patterns may have already been significantly altered \citep{Thorne2006}. Consequently, the Wieslander map is also not a reliable predictor of land cover type without extensive review of the original data and maps, which would be beyond the scope of this project. 

To further confirm these issues, I examined the overlap in land cover types between different maps in ArcGIS. In general, the overlap between EVeg and either the PNV or the Wieslander layers was no better than random, and in many cases it was worse. I decided, in conjunction with Tahoe National Forest staff, to proceed using only the EVeg map. The calibration period of the model was omitted from the HRV analysis in order to reduce or eliminate the influence of the current landscape characteristics on the results.



%This ensured that our analysis of future management scenarios and comparison of spatial metrics between those results and the HRV results was credible.

% in retrospect I wonder if we should have analyzed the configuration more. in the end the biggest problem was probably the lack of crosswalk, since a precise spatial equivalence wasn't assumed.

\subparagraph*{Selection of Specific Cover Types}
In the early stages of this project, the team created a suite of land cover types based roughly on the Wildlife Habitat Relationships (WHR) types used in California and by Forest Service managers and planners. These consisted of the WHR types with a few additional types where additional specificity or refinement was desired. For example, Red Fir was split up into two subtypes. The original concept was to begin with the WHR types and modify them as needed based on other attributes in the EVeg layer. However, creating a crosswalk from WHR to the project-specific types also proved problematic. First, I realized that the WHR values were actually derived from the \textsc{calveg} species alliances included in the EVeg layer, but the methodology used was unavailable or missing. The crosswalks I did find were not mutually exclusive and all-inclusive, and do not always make ecological sense \citep{Keeler-Wolf2007,DeBecker1988,Game2005}. This is probably due in part to the fact that WHR is not a mapping classification. It is always derived secondarily. So, I was unable to create consistent rules for mapping from WHR to other types. In addition, the WHR types are explicitly used to map current vegetation in a way that is relevant to wildlife biologists. Compared to succession processes and susceptibility and response to fire, which were my focus in this project, the WHR classification simply has a very different purpose that rendered it less useful for my needs. Others have encountered similar issues:
%
\begin{quote}
WHR has been less successful in differentiating between vegetation types. Because the habitat types are \emph{inconsistently defined}, a broad familiarity with its detailed descriptions is needed to differentiate among types of similar structure. Although mappers have constructed rules for discriminating among types, difficulties still remain because \emph{species dominance varies substantially within some types and broad overlaps in dominant plants occur among types}. Other problems arise due to the small number of classes and the \emph{inconsistencies in scale} among them \citep[23, emphasis added]{Keeler-Wolf2007}
\end{quote}
%
In collaboration with National Forest staff I decided to instead base our land cover types on, at the first order, the Presettlement Fire Regime (PFR) types as defined in the Fire Return Interval Departure (FRID) report by \citet{VandeWater2011}. The PFR types, as part of the FRID, were developed through a combination of literature and expert workshops. Peer review was solicited during these workshops, and the framework was then subjected to additional review via the academic publication process. The PFR was also useful for this project specifically because it grouped vegetation types based on their relationship to wildfire, which is the disturbance type simulated in this study \citep{VandeWater2011}. Using the FRID methodology provided an opportunity to avoid using the second-order WHR classification and trying to reverse-engineer it to fit into my custom land cover types. 

Thus I created a new structure of cover types in a nested regime. At the coarsest level are the PFR types, created by aggregating \textsc{calveg} as described above. Some of these are then subdivided using the Biophysical Settings from LandFire. Finally, a few types are further refined, ultimately generating a set of land cover types specific to the Yuba River Watershed, but applicable to the northern Sierra Nevada in general. A mutually exclusive and all-inclusive crosswalk for each land cover type used in this analysis to a single LandFire Biophysical Setting and Presettlement Fire Regime type thus exists. Appendix~\ref{app:covertypedesc} includes specifics on how this applies to individual cover types..

I used Python scripts and ArcGIS to conduct the geoprocessing necessary to prepare the EVeg layer for use in \textsc{RMLands}. All processing was done after converting shapefiles (vector data) to the raster format. Land cover types were differentiated based on spatial location, presence of aspen stands, presence of ultramaifc soils, and position along a xeric to mesic gradient. 

First, I created Aspen variants of forested land cover types by overlaying an aspen layer onto the vegetation layer and using ArcGIS tools and Python scripts to create combined types (``[type] - Aspen'') where appropriate. Second, areas mapped as a vegetation type characteristic of early successional forest (e.g. chaparral) were remapped using ArcGIS tools and Python scripts to an appropriate forest cover type, based on the land cover types in the area immediately adjacent to the patch determined to be in an early successional stage. Next, vegetation data and elevation data were analyzed together to distinguish east- and west-side (of the Sierran crest) areas from one another. This information was used to appropriately identify land cover types that, in my application, are mapped only on the east-side. Yellow Pine and stable Aspen variants of forested land cover types were confined to the east-side in my application. Ultramafic\footnote{Ultramafic soils are those created from the weathering of igneous rocks, brought to the earth's surface as magma, where they then cooled. Ultramafic soils are typically shallow, rocky, and nutrient deficient, with high levels of metals uncommon in other soils. Only a few species of plants have evolved to live on them, many of which are endemic to such soils. Plants that do grow mature more slowly and cover the land less continuously than the same plant would on better soil. In the study area, the most common ultramafic rock is serpentine \citep{Safford2004}.} land cover types were mapped by overlaying a geology layer obtained from the Tahoe National Forest (1:100,000 scale) onto the vegetation layer and using ArGIS tools and Python scripts to create ``[type] - Ultramafic''. 

Next, the Sierran Mixed Conifer, Red Fir, and Mixed Evergreen cover types, which cover broad swaths of land across elevation and aspect, were subclassified into either a mesic or xeric variant. When present, aspen or ultramafic soils supersede xeric-mesic classification. Although the WHR classification system does not divide, for example, Sierran Mixed Conifer, into xeric or mesic types, other classification systems often do. In some cases this division is recognized at the PFR level (e.g., Sierran Mixed Conifer), while in others the refinement occurs at the Biophysical Setting level (e.g., Red Fir). The PFR method does crosswalk directly from \textsc{calveg} assignments, so for certain cases it would be possible to simply use this classification strategy. However, PFRs are based on existing vegetation only and do not incorporate abiotic factors \citep{Safford2014}. In addition, when I showed a map based on the PFR classification to local experts, they felt that the distribution of xeric versus mesic types did not accurately represent the study area. Consequently, I explored some biophysical indicators related to moisture that could be used to designate and separate the mesic subtype from the xeric subtype.

Then, together with the team, I chose four metrics to comprise the mesic-xeric index. All metrics consist of modeled values. Climatic water deficit (CWD) is the annual evaporative demand that exceeds available water, measured annually in the summer. It is derived by subtracting actual evapotranspiration from potential evapotranspiration. The second metric, the topographic wetness index (TWI), measures topographic moisture. It is a function of slope and the catchment area of a particular point. Soil water storage (STOR) is the average amount of water stored in the soil annually. It is derived from precipitation, snowmelt rates, actual evapotranspiration, groundwater recharge rates, and surface water runoff rates. The final metric is the result of precipitation minus potential evapotranspiration (PPET), a measure of climatic moisture.

These variables were standardized by z-score such that higher values correspond to more mesic environments. Thus, potential evapotranspiration was inverted for this purpose. The mean for each metric is zero and the units are in terms of standard deviation. To combine the metrics, I combined the z-score value raster grids with equal weights. In conjunction with local experts, a break point in the resulting xeric-mesic gradient was selected and then applied using the ArcGIS tool Raster Calculator, creating ``[type] - Mesic'' and ``[type] - Xeric''. For the Sierran Mixed Conifer and Red Fir cover types, index values from the negative end of the range up to $-1/4$ standard deviations below the mean (zero) were used to create xeric variants, while the remaining portion of the spectrum was used to designate the mesic variants. For the Mixed Evergreen cover type, the break point along the gradient was $-1/2$ standard deviations below the mean. 


Ultimately, 31 cover types were generated for the buffered study area, as listed in Table~\ref{covertable} and shown in Figure~\ref{fig:covermap}.\footnote{Larger images of all of the input layers are included in Appendix \ref{app:inputs}.}. %A thorough description of geoprocessing steps necessary to recreate this data layer will be available soon. 
As Table~\ref{covertable} demonstrates, most cover types occupy a small extent of the study area. The cover types with an extent of less than 1000 ha within the core study area may have statistically unreliable results; this problem increases as the extent of given cover type decreases. I caution against attempting to make inferences for these less common cover types. However, because the nine cover types that do occur over at least 1000 ha represent approximately 93\% of the core study area, I have high confidence in the landscape-level results. These nine cover types are considered the focal cover types, and were all fully analyzed as part of the historical range of variability assessment. For space and continuity, in the main body of this thesis I discuss in detail only the two most common cover types, Sierran Mixed Conifer - Mesic and Serrian Mixed Conifer - Xeric, which comprise the bulk of the land in the study area actively managed by the Tahoe National Forest. Results for the other seven cover types are included in Appendix~\ref{app:full-results}. 




%%%%%%%%%%%%%%%%%%%%%%
%%% COVER TABLE %%%%%%
%%%%%%%%%%%%%%%%%%%%%%

\begin{table}[!htbp]
\footnotesize
\centering
\caption{List of land cover types developed for this project. Included are the cover type abbreviation, full cover type name, and total area in the buffered study area in both acres and hectares. Cover types are listed in descending order based on area within the core study area only. Cover types that undergo succession appear in the first group, while static cover types appear in the second group. I use the cover type abbreviation in tables and figures for space.}
\label{covertable}
\begin{tabular}{@{}llrr@{}}
\toprule
 \textbf{\begin{tabular}[c]{@{}l@{}}Land Cover \\ Abbreviation\end{tabular}} & \textbf{Land Cover Name}    & \textbf{\begin{tabular}[c]{@{}l@{}}Area \\ Core Only\\ (Hectares)\end{tabular}} & \textbf{\begin{tabular}[c]{@{}l@{}}Area\\ Core+Buffer\\ (Hectares)\end{tabular}} \\ \midrule
                        \textsc{smc\_m  }     & Sierran Mixed Conifer - Mesic                & 57,853      & 133,920    \\
\rowcolor[HTML]{CAD6BA} \textsc{smc\_x  }     & Sierran Mixed Conifer - Xeric                & 52,198      & 91,443     \\
                        \textsc{ocfw    }     & Oak-Conifer Forest and Woodland              & 23,729      & 56,941     \\
\rowcolor[HTML]{CAD6BA} \textsc{rfr\_m  }     & Red Fir - Mesic                              & 8,563       & 19,626     \\
                        \textsc{rfr\_x  }     & Red Fir - Xeric                              & 7,493       & 9,989      \\
\rowcolor[HTML]{CAD6BA} \textsc{meg\_m  }     & Mixed Evergreen - Mesic                      & 7,273       & 13,547     \\
                        \textsc{meg\_x  }     & Mixed Evergreen - Xeric                      & 6,768       & 13,771     \\
\rowcolor[HTML]{CAD6BA} \textsc{smc\_u  }     & Sierran Mixed Conifer - Ultramafic           & 4,124       & 9,774      \\
                        \textsc{ocfw\_u }     & \begin{tabular}[c]{@{}l@{}}Oak-Conifer Forest and \\ Woodland -  Ultramafic\end{tabular} & 1,060   & 2,185   \\
\rowcolor[HTML]{CAD6BA} \textsc{lpn     }     & Lodgepole Pine                               & 837         & 2,816      \\
                        \textsc{mrip    }     & Montane Riparian                             & 732         & 2,216      \\
\rowcolor[HTML]{CAD6BA} \textsc{scn     }     & Subalpine Conifer                            & 638         & 12,543     \\
                        \textsc{meg\_u  }     & Mixed Evergreen - Ultramafic                 & 604         & 1,655      \\
\rowcolor[HTML]{CAD6BA} \textsc{rfr\_u  }     & Red Fir - Ultramafic                         & 294         & 321        \\
                        \textsc{wwp     }     & Western White Pine                           & 273         & 510        \\
\rowcolor[HTML]{CAD6BA} \textsc{smc\_asp}     & Sierran Mixed Conifer with Aspen             & 58          & 121        \\
                        \textsc{oak     }     & Oak Woodland                                 & 19          & 4,186      \\
\rowcolor[HTML]{CAD6BA} \textsc{cmm     }     & Curl-leaf Mountain Mahogany                  & 18          & 41         \\
                        \textsc{lpn\_asp}     & Lodgepole Pine with Aspen                    & 8           & 31         \\
\rowcolor[HTML]{CAD6BA} \textsc{ypn     }     & Yellow Pine                                  & 0           & 10,499     \\
                        \textsc{sage    }     & Big Sagebrush                                & 0           & 1,600      \\
\rowcolor[HTML]{CAD6BA} \textsc{rfr\_asp}     & Red Fir with Aspen                           & 0           & 34         \\
                        \textsc{scn\_asp}     & Subalpine Conifer with Aspen                 & 0           & 6          \\
\rowcolor[HTML]{CAD6BA} \textsc{lsg     }     & Black and Low Sagebrush                      & 0           & 5          \\
                        \textsc{ypn\_asp}     & Yellow Pine with Aspen                       & 0           & 3          \\ 
\midrule
                        \textsc{wat     }     & Water                                        & 4,058       & 8,212      \\
\rowcolor[HTML]{CAD6BA} \textsc{bar     }     & Barren                                       & 2,665       & 8,751      \\
                        \textsc{grass   }     & Grassland                                    & 1,379       & 4,617      \\
\rowcolor[HTML]{CAD6BA} \textsc{med     }     & Meadow                                       & 1,201       & 3,435      \\
                        \textsc{urb     }     & Urban                                        & 114         & 782        \\
\rowcolor[HTML]{CAD6BA} \textsc{agr     }     & Agriculture                                  & 16          & 5,416      \\
\bottomrule

\end{tabular}
\end{table}
%\normalsize

\paragraph*{Seral Stage}
Seral stage classes combine developmental stage and canopy cover, and are defined for all cover types that undergo succession. Seral stages in this application are based on LandFire structural classes, and were further modified in collaboration with local experts on the Tahoe National Forest. In \textsc{RMLands}, susceptibility to and mortality from natural disturbances varies among seral stages. Unlike the cover grid, the seral stage grid changes dynamically over time in response to simulated succession and disturbance events. The combination of cover type and seral stage forms the basis for characterizing vegetation patterns and dynamics.

The source for the seral stage layer is the Region 5 Existing Vegetation Layer, mapped to the \textsc{calveg} classification. The \textsc{calveg} classification was developed by the Region's Ecology Program in 1978. Within the study area, the Existing Vegetation Layer was developed based on three separate efforts: a satellite-based imagery analysis in 2000, and two orthoimagery analysis completed by contracting firms in 2005. All members of the team discussed potential attributes to be used for this classification, and identified attributes for tree diameter at breast height and cover from above to classify pixels into early, middle, or late development, and open, moderate, and closed canopy. In this application, aspen and shrub types have seral stages that differ from that of the remaining forest types. The other forested types use a consistent set of seral stages.

Extensive geoprocessing was required to prepare this layer for \textsc{RMLands}. Beyond converting the vector data to a raster format, further analysis was required to update the layer to a year 2010 condition. Spatial data on wildfire and timber management history was used to provide a more accurate assessment of seral stage based on estimated stand age. In addition, areas currently mapped as chaparral in the Existing Vegetation Layer were assigned to the early development stage. The full set of seral stages is provided in Table~\ref{condtable} and depicted in Figure~\ref{fig:conditionmap}.

%%%%%%%%%%%%%%%%%%%%%%
%%% CONDITION TABLE %%
%%%%%%%%%%%%%%%%%%%%%%

\begin{table}[!htbp]
\footnotesize
\centering
\caption{List of seral stages developed for this project. Seral stages describe developmental stage (e.g. ``early'') and canopy closure (e.g. ``open''). The non-seral ``stage'' applies to land cover types for which I do not simulate succession (Barren, Grassland, Urban, Agriculture, Water, and Meadow). Included are the seral stage abbreviations, and full names.}
\label{condtable}
\begin{tabular}{@{}ll@{}}
\toprule
\textbf{\begin{tabular}[c]{@{}l@{}}Seral Stage \\ Abbreviation\end{tabular}} & \textbf{\begin{tabular}[c]{@{}l@{}}Seral Stage  \\ Name\end{tabular}} \\ \midrule
\rowcolor[HTML]{CAD6BA}   \textsc{ns}             & Non-Seral                     			    \\
                          \textsc{early\_all }    & Early 					                     \\
\rowcolor[HTML]{CAD6BA}   \textsc{mid\_cl    }    & Mid--Closed                    \\
                          \textsc{mid\_mod   }    & Mid--Moderate                  \\
\rowcolor[HTML]{CAD6BA}   \textsc{mid\_op    }    & Mid--Open                      \\
                          \textsc{late\_cl   }    & Late--Closed                   \\
\rowcolor[HTML]{CAD6BA}   \textsc{late\_mod   }   & Late--Moderate                 \\
                          \textsc{late\_cl     }  & Late--Open                     \\
\rowcolor[HTML]{CAD6BA}   \textsc{early\_asp  }   & Early--Aspen                   \\
                          \textsc{mid\_asp   }    & Mid--Aspen                     \\
\rowcolor[HTML]{CAD6BA}   \textsc{mid\_ac    }    & Mid--Aspen Conifer             \\
                          \textsc{late\_ca    }   & Late--Conifer Aspen            \\ \bottomrule
\end{tabular}
\end{table}



\paragraph*{Age}
Age represents the number of years since the last stand-replacing disturbance (high mortality wildfire). Because the characteristic species of a given cover type may not immediately establish after a stand-replacing fire, it is likely that the age value is larger than the actual age of the oldest individuals in a stand. Several of the cover types in this area may go through a chaparral-dominated early development stage; in those cases the oldest trees in the stand could be decades older than the formal stand age. In \textsc{RMLands}, age is used to trigger potential successional transitions and to calculate susceptibility to disturbance. In this application, I rounded all modeled and derived ages to the nearest five years (the length of one timestep). In the HRV analysis, the initial age value assigned to a given cell is not necessarily important to the outcome of the simulation, due to the exclusion of an equilibration period consisting of the first 40 timesteps from the analyzed results.

In this application, I used data from stand exams dating to the 1960s and from recent Forest Service Region 5 Ecology group survey plots to estimate stand age across the buffered study area. I then interpolated that information across the landscape. Due to insufficient data, I was unable to disaggregate the data below the landscape scale to cover type or another more finely resolved classification. I also acknowledge that the stand exam and modern veg plots do not constitute a true sample and were conducted almost exclusively in mid-mature and mature stands of commercially viable trees, thus skewing the results to some unquantifiable degree.

I updated the interpolated data with wildfire and timber management history, and assigned ages to types coded as chaparral in the Existing Vegetation layer to the midpoint of the age spread of Early Development for the forest cover type to which it was converted. Individual raster cells with age values out of compliance with allowed ages for the corresponding seral stage of a given cell were modified to be in compliance (see Appendix~\ref{app:covertypedesc}). I modified the age layer rather than the seral stage layer based on the assumption that the seral stage assignment was more accurate that the interpolated age information. The input Age layer, showing the map at timestep 0, is shown in Figure~\ref{fig:agemap}.


\paragraph*{Seral stage-Age}
Seral stage-Age represents the age since transitioning to the current seral stage. In \textsc{RMLands} it affects most transitions between seral stages: typically there is a threshold seral stage-age below which transitions do not occur. After creating both the seral stage and age layers, I used a Python function to derive seral stage-age based on the youngest possible age for a cell of that cover and seral stage. For example, if a particular cell on the landscape was defined as having a cover type of Lodgepole Pine, with a seral stage of Mid--Closed, and an age of 50 years, I took the minimum age for that cover type-seral stage combination (according to the Lodgepole Pine cover type description in Appendix~\ref{lpn-description}, the earliest age at which a Lodgepole Pine cell can transition to Mid--Closed is 10 years), and subtracted it from the derived age to arrive at a seral stage-age of 40 years. The same caveats and assumptions that apply to the seral stage and age layers also apply to the seral stage-age layer. The final map for the initial seral stage-age is shown in Figure~\ref{fig:condagemap}.


\paragraph*{Topographic Position Index}
The topographic position index (TPI) combines heat load, which is based on aspect and slope, with slope position (Figure~\ref{fig:tpimap}), and ranges from -300 to 300. High values for TPI are correlated with locations on steep, south and west-facing, upper slopes. Low values are correlated with locations on gentle, north and east-facing, valley bottoms. Intermediate values occur along a gradient of these characteristics. The TPI is scaled to the study area and the region immediately surrounding it, and is therefore a local index only, although it could be derived based on any spatial boundary. The purpose of incorporating the TPI was to better mimic fire behavior on the landscape. Past research has indicated that fires generally burn more frequently on southerly, steep slopes than on gentle, northerly slopes. Because they are drier and more exposed, and because of the way fuels allow preheating and ready ignition of vegetation as fire travels uphill \citep{Rothermel1983}, the likelihood of overstory tree mortality increases with increasing TPI (these statements may also be interpreted as the inverse, wetter valley bottoms are less likely to produce a crown fire) \citep{North2012,Taylor2003a}. I used TPI to adjust vegetation susceptibility and mortality in \textsc{RMLands}, as described in the model parameterization section \ref{subsec:hrvmodelparam}. As part of model evaluation, I ploted the average canopy cover (over the simulated period) against TPI and calculate the proportional effect of TPI on cover (section \ref{subsec:modelassessment}).



\paragraph*{Elevation} 
Elevation represents the height above sea level in meters. In \textsc{RMLands}, the elevation layer affects disturbance spread. The elevation grid used in this analysis was a digital elevation model (DEM) provided by the Tahoe National Forest GIS staff and rescaled from 10 m$^2$ to 30 m$^2$ pixels. It is shown as a map in Figure~\ref{fig:elevationmap}.



\paragraph*{Slope} 
Slope represents the steepness of a cell as measured in percent and is derived from the elevation layer. Slope is used in GIS preprocessing to define cover types. Within \textsc{RMLands}, slope affects disturbance spread. The slope for the study area was derived from the elevation layer described above, and is shown in Figure~\ref{fig:slopemap}.


\paragraph*{Aspect} Aspect represents the direction a cell is facing in terms of eight cardinal directions. Flat aspects are also recognized. Within \textsc{RMLands}, aspect affects disturbance spread. The aspect for the study area was derived from the elevation layer described above, and is shown in Figure~\ref{fig:aspectmap}. 


\paragraph*{Streams} 
Streams represents linear hydrological features, classified as small, medium or large based on stream order. In this application, the streams layer was created from a line coverage containing hydrography data, including an attribute for stream size or order, by converting to a grid based on the stream size attribute. Streams may inhibit the spread of wildfire in \textsc{RMLands}, depending on both stream size and potential wildfire size. In order to function as a barrier, cells in the stream raster coded as streams much share a side, rather than only a vertex. The stream layer was used to overwrite any cells not coded as ``Water'' in the cover type layer, such that all ``large'' (first order) streams are represented as water in the cover type layer. The final streams input layer is shown in Figure~\ref{fig:streamsmap}.


\paragraph*{Buffer/Core} 
This layer identifies and distinguishes the ``core'' study area from the 10 km ``buffer,'' which allows wildfires to initiate outside of and burn beyond the formal study area. Without a buffer, edge effects would alter results for all aspects of the disturbance regime, as well as resulting landscape composition and configuration. A 10 km buffer was selected arbitrarily, but has in the past been sufficient to offset edge effects \citep{McGarigal2005,McGarigal2005a}. Because the simulation plays out on the full extent of the core plus the buffer, all input grids are developed to that larger extent as well. To create this raster, the original study area polygon was buffered in ArcGIS by 10 km, then converted to raster using the same procedure as for other layers. The buffer and core are easily distinguished in the subfigures of Figure~\ref{fig:inputlayermaps}: the core is the interior area delineated by a thick black line, while the buffer is the area outside of this line, displayed at a decreased brightness level.


\begin{figure}[!htbp]
  \centering
  \subfloat[][]{
    \centering
	\includegraphics[width=0.4\textwidth]{/Users/mmallek/Tahoe/Report3/images/cover_resized.png}
    \label{fig:covermap}
  } \qquad
  \subfloat[][]{
    \includegraphics[width=0.4\textwidth]{/Users/mmallek/Tahoe/Report3/images/condition_resized.png}
    \label{fig:conditionmap}
  } %\\
  %
	\subfloat[][]{
    \includegraphics[width=0.4\textwidth]{/Users/mmallek/Tahoe/Report3/images/age_resized.png}
    \label{fig:agemap}
  } \qquad
	\subfloat[][]{
    \includegraphics[width=0.4\textwidth]{/Users/mmallek/Tahoe/Report3/images/condage_resized.png}
    \label{fig:condagemap}
  } 
  %
  \subfloat[][]{
    \centering
	\includegraphics[width=0.4\textwidth]{/Users/mmallek/Tahoe/Report3/images/tpi_resized.png}
    \label{fig:tpimap}
  } \qquad
  \subfloat[][]{
    \includegraphics[width=0.4\textwidth]{/Users/mmallek/Tahoe/Report3/images/elevation_resized.png}
    \label{fig:elevationmap}
  } %\\
  %
	\subfloat[][]{
    \includegraphics[width=0.4\textwidth]{/Users/mmallek/Tahoe/Report3/images/slope_resized.png}
    \label{fig:slopemap}
  } \qquad
	\subfloat[][]{
    \includegraphics[width=0.4\textwidth]{/Users/mmallek/Tahoe/Report3/images/aspect_11oct.png}
    \label{fig:aspectmap}
  } 
  %\\
  	\subfloat[][]{
    \includegraphics[width=0.4\textwidth]{/Users/mmallek/Tahoe/Report3/images/streams_resized.png}
    \label{fig:streamsmap}
  } 

  \caption{RMLands input layers. (a) Cover type map (b) Seral stage map (c) Age map at Timestep 0 (d) Seral stage-Age map at Timestep 0 (e) Topographic Position Index (f) Elevation (g) Slope (h) Aspect (i) Streams}
  \label{fig:inputlayermaps}
\end{figure}


\subsection{Model Parameterization}
\label{subsec:hrvmodelparam}

\subsubsection*{State and Transition Models}
I have created a detailed cover type description document for each cover type in the simulated landscape that experiences transitions between cover class (Appendix \ref{app:covertypedesc}). These documents describe crosswalks to other data layers, detailed accounts of the multiple species characteristic of the cover type, the cover type distribution, relationship and response to wildfire, predicted fire rotations, as well as descriptions of each seral stage present within the cover type and their succession and wildfire transition conditions and rates. Each detailed document can be summarized in part as a state and transition model for a particular cover type, which is implemented in the model by specifying susceptibility to wildfire, rules for vegetational succession, and rules for transitions after a fire event. Figure~\ref{transmodel} shows a generic example state and transition model for the forested cover types.

\begin{figure}[htbp]
\centering
\includegraphics[width=0.8\textwidth]{/Users/mmallek/Tahoe/Report3/images/state_trans_model.pdf}
\caption{Generic state and transition model for all non-shrub seral cover types. Each dark grey box represents one of the seven seral stages for this landcover type. Each column of boxes represents a stage of development: early, middle, and late. Each row of boxes represents a different level of canopy cover: closed (70-100\%), moderate (40-70\%), and open (0-40\%). Transitions between states/seral stages may occur as a result of high mortality fire, low mortality fire, or succession. Specific pathways for each are denoted by the appropriate color line and arrow: red lines relate to high mortality fire, orange lines relate to low mortality fire, and green lines relate to natural succession.} 
\label{transmodel}
\end{figure}


An important characteristic of \textsc{RMLands} is that it treats fire somewhat differently from other landscape succession and disturbance models, which affected how I created and specified the state and transition models. In \textsc{RMLands}, fires spread probabilistically based on the susceptibility of an individual cell. It does not contain a mechanistic fire model and fuels are not directly incorporated into fire spread. In addition, I do not classify individual fires as a whole to a \emph{low, mixed, or high severity} status. Some fire ecologists combine fire attributes such as flame length and fire size into their interpretation of the relative \emph{severity} of a particular fire \citep{Agee1993}.   Ecologists working at other scales and not working with models often describe \emph{mixed severity} regimes \citep[e.g.,][]{Kane2013}, which \citet{Collins2010} define as ``stand-replacing patches within a matrix of low to moderate fire-induced effects.'' If I were to adopt that definition, nearly all fires would be classified as \emph{mixed severity} due to the 30 m cell size and resolution at which fire mortality is defined, rendering this perspective moot for my study. Instead, I focus on defining conditions under which transitions among potential states within a given cover type occur or not (Figure~\ref{fig:mosaic}).
\begin{wrapfigure}{R}{0.5\textwidth} % use a capital R to allow figure to float
\includegraphics[width=0.5\textwidth]{/Users/mmallek/Documents/Thesis/Seminar/rimfirehillside.jpeg}
\caption{Aftermath of the 2013 Rim Fire in the Sierra Nevada. As in my model, post-fire, the landscape can be sorted into high mortality and low mortality areas. Photo from USFS Region 5.} 
\label{fig:mosaic}
\end{wrapfigure} 
I evaluate and classify fire by its effects on individual cells. First, I evaluate whether a cell burned. Next, all burned cells are evaluated probabilistically and assigned either a high severity (``high mortality'') outcome or low mortality outcome. If a cell burns at high severity, then it is deemed to have had a high mortality outcome and transitions to the Early Development seral stage. Recently, some researchers have differed on whether 75\% or 95\% overstory tree mortality is a more appropriate cutoff point for defining a ``stand-replacing'' event \citep{Fule2014,Mallek2013}. In this paper, I use 75\% as the cutoff, which is widely accepted in the literature \citep{Agee1993,Agee2007,Miller2009,Baker2014}.

To derive probabilities for post-fire transitions, both for transitions to the Early seral stage or to a more open seral stage, I used the Vegetation Dynamics Development Tool (VDDT) models associated with the Biophysical Settings Models from the LandFire project. From the VDDT models, I used the probabilities of a transition to the early seral stage, a more open canopy seral stage, or of no transition. I ignored the classified type of fire (as replacement, mixed, or low severity), focusing instead on the outcome from fire in terms of the seral stage, if any, to which a cell transitioned after wildfire. High severity fire occurs when the result is converstion to early seral (regardless of whether the fire is labeled ``replacement'' or ``mixed'' in the VDDT model). All other fires are less than high severity\todo{still need a way to describe fire properly - ask lee?}, and considered low mortality. The probability of a high mortality outcome from fire was calculated by dividing the summed probabilities of fire leading to a transition to the Early Development seral stage by the summed probabilities of all fires.  I also used LandFire data to derive probabilities of succession. I then evaluated and refined these probabilities with input from local experts to capture subtle changes in succession and transition relevant to the scale of the study area.

To illustrate the parameterization, in the following tables I present values for the Sierran Mixed Conifer - Mesic cover type model. The target fire rotation for this cover type is 29 years. A fire rotation index is used as the parameter controlling the relative susceptibility to fire of the seral stages within an individual cover type (low values correspond to higher susceptibility). In addition, the probability of high severity fire leading to at least 75\% overstory mortality is specified for each seral stage (Table~\ref{smcm_fri_phm})\todo{I like this phrasing}. I also specified transition probabilities for natural succession between the early, middle, and late stages of development, as well as between closed, moderate, and open canopy cover. This type of succession also depends on the time in the current seral stage both in terms of the early-middle-late sequence (\emph{Development-Age}) and the specific stage-canopy cover combination (\emph{Seral Stage-Age}) (Table~\ref{smcm_vegtrans}). Finally, probabilities are specified for vegetation transitions after less than high severity wildfire (Table~\ref{smcm_firetrans}).


% edited 2015-9
\begin{table}[htbp]
\footnotesize
\centering
\caption{Fire rotation index values and probability of high severity fire (at least 75\% overstory tree mortality) for Sierran Mixed Conifer - Mesic. The seral stage that is most susceptible to fire (i.e., has the lowest predicted fire rotation) has a fire rotation index value of 1. Higher values correspond with lower susceptibility to wildfire. The values are relative only within an individual seral stage and should not be compared against other land cover types.}
\label{smcm_fri_phm}
\begin{tabular}{lcc}
\hline
 \textbf{Seral Stage}    & \textbf{\begin{tabular}[c]{@{}c@{}}Fire Rotation \\ Index\end{tabular}} & \textbf{\begin{tabular}[c]{@{}c@{}}Probability of \\ High Severity Fire\end{tabular}} \\ \hline
Early (All)     			& 5.4        & 1.0                 \\
Mid--Closed    				& 2.4        & 0.23              \\
Mid--Moderate  				& 1.6        & 0.17              \\
Mid--Open      				& 1.3        & 0.14              \\
Late--Closed   				& 4.3        & 0.37              \\
Late--Moderate 				& 1.6        & 0.14              \\
Late--Open     				& 1.0          & 0.09              \\ 
\emph{Target Fire Rotation}    		& \emph{29 years}  &   \\ \hline
\end{tabular}
\end{table}

\begin{table}[!htbp]
\footnotesize
\centering
\caption{Timeframes for transitions between seral stages in \textsc{RMLands} for Sierran Mixed Conifer~-~Mesic. ``Early to Mid'' and ``Mid to Late'' times are based on the time in a developmental stage, regardless of disturbance history. ``Open to Moderate'' and ``Moderate to Closed'' times are based on the time in a seral stage since the last recorded fire.}
\label{smcm_vegtrans}
\begin{tabular}{cccc}
\hline
\textbf{\begin{tabular}[c]{@{}c@{}}Seral Stage \\ Transition\end{tabular}} & \textbf{Minimum (years)} & \textbf{Average (years)} & \textbf{Maximum (years)} \\ \hline
Early to Mid 	& 20      & 26      & 40      \\
Mid to Late 	& 100     & 113     & 150     \\
\begin{tabular}[c]{@{}c@{}}Open to Moderate or\\ Moderate to Closed\end{tabular}  & 15      & 21      &    ---     \\ \hline
\end{tabular}

\end{table}


\begin{table}[!htbp]
\footnotesize
\centering
\caption{Transition probabilities for Sierran Mixed Conifer - Mesic following low mortality fire.}
\label{smcm_firetrans}
\begin{tabular}{lcc}
\hline
\textbf{Seral Stage Transition} & \textbf{Probability}\\
\hline
Mid--Closed to Mid--Moderate    & 0.17 \\ %0.53 
Mid--Moderate to Mid--Open    	& 0.24 \\ %0.36
Late--Closed to Late--Moderate	& 0.54 \\
Late--Moderate to Late--Open    & 0.24 \\
\hline
\end{tabular}
\end{table}

Transitions between Early and Middle Development, and between Middle and Late Development are governed by the time in the Early or Middle stage (canopy cover usually does not affect these probabilities). These transitions may begin at the minimum time in a specified \emph{Development-Age}, and proceed at rates that vary across cover types. Table~\ref{smcm_vegtrans} displays the average \emph{Seral Stage-Age} of transition. If a cell reaches the maximum stage-age listed, its probability of transitioning goes to 1. 

Transitions between the canopy cover levels occur within one developmental stage: i.e., between Mid--Open and Mid--Moderate, but not between Mid--Open and Late--Moderate. These transitions are governed by the time in a specific seral stage since the last wildfire. This means that the ``years since'' value may be affected by a low mortality fire, a transition between developmental stages, or a transition between canopy cover levels. Similarly to the developmental transitions, the shift from, for example, Mid--Open to Mid--Closed, may begin when the minimum time is reached, and also proceeds at rates that vary across cover types. No maximum age is specified for this type of transition.

\subsubsection*{Disturbance Parameters} 
\label{subsubsec:distparams}

RMLands works by simulating fires and succession, one timestep at a time. The first part of every timestep is simulating fire and the second part simulates succession. To simulate fire, fires are \emph{initiated} randomly across the landscape at the cell level (that is, one pixel in the raster). The initial cell burns, or doesn't, based on its susceptibility value at that time, which is resolved probabilistically. \emph{Spread} is a step where the spatially-explicit aspect of the model comes into play. Fire can spread from a burned cell to adjacent cells. This process also incorporates spatial information on aspect and elevation, among other attributes. Whether or not the fire spreads is also based on the \emph{susceptibility} of surrounding cells and the probability of having a larger \emph{potential size}. After fire, each cell is assigned to ahigh or low \emph{mortality outcome} probabilistically. Based on this designation, cells are evaluated to determine whether or not they will \emph{transition}. Figure~\ref{fig:rmlands-fire-steps} illustrates these steps.

\begin{figure}[htbp]
\centering
\includegraphics[height=0.3\textheight]{/Users/mmallek/Documents/Thesis/Seminar/rmlands-fire.png}
\caption{Steps in fire simulation in the \textsc{RMLands} software.} 
\label{fig:rmlands-fire-steps}
\end{figure}

In \textsc{RMLands}, parameter specification is grouped under five headings: climate, initiation, susceptibility, spread, and mortality.


%\begin{adjustwidth}{5ex}{0pt}
\begin{itemize}
\item \emph{Climate:} The climate parameters are based on a rescaling of the Palmer Drought Severity Index (PDSI). PDSI is a long-term measure of drought, on the scale of months to years. It is based on precipitation and temperature and incorporates soil moisture. Resconstructed PDSI values for summer months during the historic period of this project (1550-1850) are available from the National Oceanic and Atmospheric Administration (\burl{http://www.ncdc.noaa.gov/paleo/pdsi.html}). I used datasets from \citet{Zhangetal.2004} and \citet{Cook2004} that included spatially-explicit PDSI values for North America during the historical period. These data are summarized at large scales; for example, the \citet{Cook2004} data are calculated for a grid with points spaced at 2.5\textdegree. I selected the five closest points to the center of the study area from these two datasets and calculated the inverse distance-weighted mean of the values. I then converted the yearly data into five-year averages to align with the five-year timesteps in our model. PDSI normally scales from -6 (extreme drought) to 6 (extremely wet). Values of 0 are considered normal. By recentering the mean value around 1 and then taking the inverse, I created a dataset in which a value of 1 is neither wetter nor dryer than average, values between 0 and 1 represent wetter-than-normal timesteps, and values greater than 1 represent dryer-than-normal timesteps (Figure~\ref{pdsi}). Climate interacts with other disturbance parameters in \textsc{RMLands}, including initiation, susceptibility, and spread.

\begin{figure}[htbp]
\centering
\includegraphics[height=0.3\textheight]{/Users/mmallek/Documents/Thesis/Plots/pdsi/hrv-bigtext.png}
\caption{Palmer Drought Severity Index, rescaled, inverted, and presented as a 5-year average for the ``historical'' period in this study (1550-1850).} 
\label{pdsi}
\end{figure}

%\medskip

%\noindent 
\item \emph{Initiation:} In \textsc{RMLands}, the ignition calibration coefficient is typically used as a calibration parameter. The ignition calibration coefficient refers to the number of attempted ignitions per 100,000 ha per year. For the HRV simulation, I set this coefficient at 42. I applied the coefficient evenly across the landscape based on local expert knowledge of lighting strike locations in the area. Fires may be initiated anywhere within the study area or the 10 km buffer around it. The total area cover within that boundary is 409,411 ha, so up to 860 fire starts were possible during each 5-year timestep in our simulation (not all potential ignitions result in fire). Climate also influences initiation. The probability of wildfire initiation is a function of its susceptibility to wildfire and the climate modifier value for that timestep, and is applied at the cell level.

%\medskip

%\noindent 
\item \emph{Susceptibility:} Cover type and seral stage are both inputs to susceptibility. Topographic position is also an input, whose influence is specified at the cover type level. Cover type modifies susceptibility via the ability to specify the influence of topographic position on susceptibility (Table~\ref{covtpi}). The magnitude of this effect is estimated as a potential reduction in susceptibility of 30\% between the minimum and maximum Topographic Position Index (TPI) values used in the model. I implemented this by using a logistic function to convert the TPI grid values into an appropriate multiplier within the susceptibility equation:

$$\text{TPI Susceptibility Factor} = L + \frac{R-L}{1+e^{k(x_0-x)}}$$

in which $L= 0.7$, $R=1$, slope $k=1$, inflection point $x_0=0$, and $x=\text{TPI}$. %at a given cell
It therefore simplifies to 

$$\text{TPI Susceptibility Factor} = 0.7 + \frac{0.3}{1+e^{-x}}$$

In this way, when the TPI Susceptibility Factor = 1, TPI has no effect. This happens only when the TPI value at an individual cell is zero. The effect of this is that areas with low TPI (generally north-facing and flatter slopes) burn less frequently than areas with high TPI (generally south-facing and steeper slopes).


\begin{table}[htbp]
\footnotesize
\centering
\caption{Cover types whose susceptibility is modified by Topographic Position Index. All cover types are modified in the same way.}
\label{covtpi}
\begin{tabular}{ll}
\hline
\multicolumn{2}{c}{\textbf{Cover Types modified by TPI Susceptibility Factor}} \\
\hline
Grassland     					& Red Fir - Mesic   			\\
Lodgepole Pine    				& Red Fir - Ultramafic			\\
Mixed Evergreen - Mesic				& Red Fir - Xeric    			\\
Mixed Evergreen - Ultramafic     		& Sierran Mixed Conifer - Mesic    	\\
Mixed Evergreen - Xeric 			& Sierran Mixed Conifer - Ultramafic 	\\
Montane Riparian				& Sierran Mixed Conifer - Xeric 	\\
Oak Woodland 					& Western White Pine			\\
Oak-Conifer Forest and Woodland 		& Yellow Pine 				\\
Oak-Conifer Forest and Woodland - Ultramafic 	&					\\
\hline
\end{tabular}

\end{table}

Seral stage further modifies susceptibility. I used the Weibull cumulative distribution function and specify a scale parameter $\lambda$ (``mean return interval''), shape parameter $k$, and the reset point for the function (\emph{age since high mortality disturbance} or \emph{age since any disturbance}) \citep{Johnson1985}. The fire rotation index for the seral stage is used as the basis for $\lambda$ and treated as a calibration parameter. These values were initially set as equal to the mean return interval values provided in the analogous LandFire Biophysical Setting types \citep{Landfire2007}. Some modifications were made based on consultation with Forest Service staff. All values of $\lambda$ within a cover type were modified as a group and kept relative to one another (that is, the rotation index ratios were preserved) even as the magnitude of the parameters were adjusted. The fire rotation index values for Sierra Mixed Conifer - Mesic are shown in Table~\ref{smcm_fri_phm}; these values for each cover type are included in the cover type description documents (Appendix \ref{app:covertypedesc}). I set $k=3$ for all cover types and seral stages. I selected between (\emph{age since high mortality disturbance} and \emph{age since any disturbance}) based on whether wildfires in that cover type are climate-driven (in which case I selected the former) or fuels-driven (in which case I selected the latter) (Figure~\ref{howdriven}).

\begin{table}[htbp]
\footnotesize
\centering
\caption{Cover types sorted by whether wildfire disturbance in them is characterized by fuels present or overarching climatic conditions. If the likelihood of wildfire depends on the accumulation of fuels, the value of $x$ (``time since'') reverts to 0 after any disturbance. If the likelihood of wildfire depends primarily on climate and weather conditions, the value of $x$ reverts to 0 only after a high mortality disturbance.}
\label{howdriven}
\begin{tabular}{ll}
\hline
 \textbf{Fuel-Driven Cover Types} 	& \textbf{Climate-Driven Cover Types}	\\
\hline
Curl-leaf Mountain Mahogany 	        & Agriculture   			\\
Grassland     			        & Big Sagebrush 			\\
Lodgepole Pine                          & Black and Low Sagebrush		\\
Meadow				        & Lodgepole Pine with Aspen 		\\
Mixed Evergreen - Mesic			& Montane Riparian			\\
Mixed Evergreen - Ultramafic            & Red Fir with Aspen   			\\
Mixed Evergreen - Xeric 		& Red Fir - Mesic    			\\
Oak Woodland 				& Red Fir - Ultramafic 			\\
Oak-Conifer Forest and Woodland 	& Red Fir - Xeric 			\\
Oak-Conifer Forest and Woodland - Ultramafic 	& Subalpine Conifer 		\\
Sierran Mixed Conifer - Ultramafic 	& Subalpine Conifer with Aspen 		\\
Sierran Mixed Conifer - Xeric 		& Sierran Mixed Conifer with Aspen 	\\
Urban 					& Sierran Mixed Conifer - Mesic 	\\
Yellow Pine 				& Western White Pine 			\\
					& Yellow Pine with Aspen 		\\
\hline
\end{tabular}
\end{table}


%\medskip



%\noindent 
\item \emph{Spread:} The probability of fire spread in \textsc{RMLands} is a function of climate, susceptibility to wildfire, potential wildfire size, wind, spotting, relative elevation, and presence of streams. The first two are described above. The disturbance size distribution that regulates potential fire size was created by analyzing the size distribution of all mapped fires in the Northern Sierra \textsc{calveg} mapping zone and west of the Sierran crest, available from the U.S. Geological Survey and the California Department of Forestry and Fire Protection \citep{calfire2012,usgs-fire-data2012}, which together go back to approximately 1900. 

\begin{minipage}{\linewidth}
 \centering
 \includegraphics[width=10cm]{/Users/mmallek/Tahoe/Report3/images/weather.png}
  \captionof{figure}{Weather stations used to inform wind direction parameters. Weather stations are denoted by red circles. A black boundary line identifies the study area.}
  \label{weather}
\end{minipage}

Wind is incorporated in two parts. First, a prevailing \emph{wind direction} for the fire is selected probabilistically from the eight cardinal directions. To compute the wind distribution values, I first consulted local experts to determine the dates of fire season (May 15 to October 15) and burning period times (1000 hours to 1800 hours). I then downloaded all available historic wind direction data from 6 local weather stations (Rice Canyon, Saddleback, Downieville, White Cloud, Emigrant Gap, and Blue Canyon, Figure~\ref{weather}). Data from all weather stations was weighted equally. After the wind direction is selected, fires are able to grow in all directions, but are relatively more likely to spread with wind than against it. I parameterized the influence of \emph{relative wind} as a reduction in spread likelihood. Thus, spread in the same direction as wind has a neutral effect, spread at $\ang{45}$ angles is reduced by 30\%, spread at $\ang{90}$  angles is reduced by 70\%, spread at $\ang{135}$ angles is reduced by 90\%, and spread opposite the prevailing wind direction is reduced by 95\%. 

\emph{Relative elevation} also modifies spreading potential. I parameterized the model such that spread downhill is extremely unlikely. \emph{Spotting} and the extent to which streams act as \emph{barriers to spread} are affected by the fire size. As fires become larger, their probability of spotting and spotting distance increases. Similarly, streams function as a barrier to smaller fires, but large fires are able to spread past streams regardless of size. This decision is based on the idea that large fires are more influenced by wind and climatic conditions. Stream size does impact smaller fires; the largest streams and rivers are usually an effective barrier to smaller fires, although even fairly small fires often spread past intermittent and small perennial streams. 


\item \emph{Mortality:} Cover type and seral stage are both inputs to \emph{mortality}. Cover type modifies susceptibility via the ability to specify the influence of topographic position on mortality (Table~\ref{covtpi_mort}). The magnitude of this effect is estimated as a potential reduction in mortality of 30\% between the minimum and maximum TPI values used in the model. I implemented this by using a logistic function to convert the TPI grid values into an appropriate multiplier within the mortality equation:


$$\text{TPI Mortality Factor} = L + \frac{R-L}{1+e^{k(x_0-x)}}$$

in which $L= 0.7$, $R=1$, slope $k=1$, inflection point $x_0=0$, and $x=\text{TPI}$. %at a given cell
It therefore simplifies to 

$$\text{TPI Mortality Factor} = 0.7 + \frac{0.3}{1+e^{-x}}$$

In this way, when the TPI Mortality Factor = 1, TPI has no effect. This happens only when the TPI value at an individual cell is zero. The effect of this is that areas with low TPI (generally north-facing and flatter slopes) are less likely to experience high severity wildfire leading to over 75\% overstory mortality than areas with high TPI (generally south-facing and steeper slopes).

\begin{table}[htbp]
\footnotesize
\centering
\caption{Cover types whose mortality is modified by Topographic Position Index.}
\label{covtpi_mort}
\begin{tabular}{ll}
\hline
\multicolumn{2}{c}{\textbf{Cover Types Mdified by TPI Mortality Factor}} \\
\hline
Grassland     					& Red Fir - Mesic   			\\
Lodgepole Pine    				& Red Fir - Ultramafic			\\
Mixed Evergreen - Mesic				& Red Fir - Xeric    			\\
Mixed Evergreen - Ultramafic     		& Sierran Mixed Conifer - Mesic    	\\
Mixed Evergreen - Xeric 			& Sierran Mixed Conifer - Ultramafic 	\\
Montane Riparian				& Sierran Mixed Conifer - Xeric 	\\
Oak Woodland 					& Western White Pine			\\
Oak-Conifer Forest and Woodland 		& Yellow Pine 				\\
Oak-Conifer Forest and Woodland - Ultramafic 	&					\\
\hline
\end{tabular}
\end{table}

Seral stage further modifies mortality. I extracted the likelihood of mortality from the VDDT models built during the LandFire project, as described at the beginning of section~\ref{subsec:hrvmodelparam}. As an example, these probabilities for Sierran Mixed Conifer - Mesic are provided in Table~\ref{smcm_fri_phm}.



\end{itemize}
%\end{adjustwidth}

\subsection{Model Calibration}
Although \textsc{RMLands} is a process-based model with parameters sourced from the literature, the team and I had greater confidence in some parameters than others, especially with respect to how they function within the \textsc{RMLands} framework. Consequently, I calibrated, or verified, the model by iteratively adjusting certain parameters in which there was less confidence about the appropriate values until the outputs were tuned to a set of parameters in which the team had high confidence. Specifically, I manipulated the ignition calibration coefficient and the fire rotation index and measured calibration success based on conformity to pre-specified rotation values at the cover type level. Fire rotation index values were changed by a constant multiplier across all seral stages of a given cover type. That is, cover types were modified as groups but the index ratios within them were maintained. The calibration target was defined as the rotation values for the nine focal cover types\footnote{Mixed Evergreen - Mesic, Mixed Evergreen - Xeric, Oak-Conifer Forest and Woodland, Oak-Conifer Forest and Woodland - Ultramafic, Red Fir - Mesic, Red Fir - Xeric, Sierran Mixed Conifer - Mesic, Sierran Mixed Conifer - Ultramafic, Sierran Mixed Conifer - Xeric.}  $\pm 10$\% of the original target rotations. I focused on these nine types because they all extend across more than 1,000 ha, and are thus statistically stable from simulation to simulation. Target values were based on published empirical values and refined with input from local experts. I chose rotation as the calibration target because targets were available from the literature and because fire rotation is a fundamental measurement that \textsc{RMLands} was designed to capture. In addition, using rotation ties calibration to a parameter that is relateable to Forest Service staff and that can be used at the landscape scale as a target by managers in various programs. To illustrate the calibration process, I describe it for Sierran Mixed Conifer - Mesic. The target fire rotation was 29 years. I adjusted the input seral stage fire return index by multiplying it by different constants, eventually arriving at an increase by a factor of 25 from the original calculated ratio values. That is, each initial scale parameter value was multiplied by 25 in order to modify its susceptibility to fire without changing the relative susceptibility among its seral stages (see Table~\ref{smcm_fri_phm} for the final relative susceptibility values). 


%%%
%in a nutshell, we have to calibrate the model because we it's not completely mechanistic and we're making guesses on a lot of things
%so we pick a few model parameters that we have high confidence in, and decide  not to change those
%and then we pick a few model outputs that we have confidence in
%because our goal is to simulate a regime we believe we already understand
%so we want it to look "right"
%so we figure if we can make the model outputs agree with the numbers we're pretty sure are right, then we trust the other outputs where we weren't totally sure what to expect
%and we do this by adjusting the parameters that we have lower confidence we got right hte first time, or that don't relate to the real world direclty and mechanistically



\subsubsection*{Model Execution}
During the calibration phase of the model, a typical simulation run consisted of three iterations of the model lasting 200 timesteps each. The equilibration period of 40 timesteps was chosen based on visual analysis of the seral stage distribution plots. The cutoff period was chosen at 40 timesteps because by this point the nine focal cover types had reached a distribution characteristic of a foregoing, stable distrbution oscillating around a mean value. I exclude this equilibration period from most of my statistical results because they are an artifact of the starting conditions. Once calibration was complete, I conducted one run of 500 timesteps in order to capture multiple disturbance and succession cycles across the most common cover types. Each timestep represents five years. The five-year timestep was chosen based on the short fire return intervals (how frequently fire recurred in the same location) recorded from dendrochronology analysis in the literature and my desire to capture these very short rotations in the simulation.

\subsection{Data Analysis}
\label{subsec:dataanalysis}

Sierra Nevada vegetation is extremely diverse and complex, both ecologically and spatially. In the body of this thesis I limit my results to an evaluation of the full landscape and of the xeric and mesic mixed conifer forests, which together comprise 63\% of the study area. Results for the next seven most extensive types are included in Appendices~\ref{app:full-results}. In general, my confidence in the results decline as the extent of a cover type declines, because the results are statistical and large samples are needed. I do not include results for cover types that extend across less than 1,000 ha of the study area.

\subsubsection*{Disturbance Regime} I quantified the following overall temporal and spatial characteristics of the wildfire disturbance regime:
\begin{itemize}
	\item \emph{Disturbed Area:} I calculated disturbed area for each timestep, divided into low mortality and high mortality disturbance, and summed to produce an ``any mortality'' statistic. I summarize the results for the $5^{\text{th}}$ percentile, $50^{\text{th}}$ percentile (median), $95^{\text{th}}$ percentile, and mean area disturbed as a proportion of the total area eligible for disturbance for the full simulation excluding the equilibration period (460 timesteps, or 2300 years). Because it can be difficult to visualize what our quantitative results look like, I include several maps that illustrate the results, demonstrating that model results are spatially-explicit and realistic. To do this, I include maps of the landscape illustrating the $5^{\text{th}}$ percentile, $50^{\text{th}}$ percentile, $95^{\text{th}}$ percentile, and mean area burned during the simulation, plus a example 4-timestep sequence illustrating changes to the seral stage pattern for mesic mixed conifer forests due to successional and disturbance procceses. Finally, I use a histogram to display the distribution of wildfire extents during the simulation, excluding the equilibration period.
	\item \emph{Disturbance Frequency:} I calculated the number of years between disturbances exceeding a particular threshold in total disturbed area. I report the frequency of timesteps during which thresholds of at least 10\%, 25\%, or 50\% of the landscape experienced wildfire. I also translate this into the proportion of timesteps in the simulation, and the interval between such occurrences in both years and timesteps. The purpose of this redundancy is to facilitate different ways of conceptualizing fire frequency, which will be useful to managers from diverse backgrounds.
	\item \emph{Climate Effect:} Climate interacts with several components of the model. I present plots illustrating the value of the climate parameter by timesteps concurrently with the area disturbed per timestep. It is not practical to further illustrate its effect everywhere, and in some cases its influence is not easily separated from the other inputs to the model. 
	\item \emph{Rotation Period:} I calculated the rotation period---the number of years required to burn an area equivalent to the total eligible area---for each cover type within the study area and the study area as a whole. I report the rotation values for low mortality fire, high mortality fire, and any fire for the full landscape, Sierran Mixed Conifer - Mesic, Sierran Mixed Conifer - Xeric. Results for the other seven focal cover types are available in Appendix~\ref{app:full-results}.
	\item \emph{Return Interval:} I summarized the cell-specific fire rotation---the average number of years between disturbances at a single cell---and present it as the distribution of the percentage of eligible cells that experienced each possible mean return interval. I use histograms to visualize the distribution of this return interval for low mortality fire, high mortality fire, and any fire, along with their median values. The median is equivalent to the cell-specific grand mean return interval for a given cover type across the landscape. I also display this result spatially as a map showing the cell-specific fire rotation for each raster cell across the landscape. 
\end{itemize}

\subsubsection*{Vegetation Response} 

\emph{Landscape Composition:} I quantified the distribution and dynamics of landscape composition by cover type. For the single 2500 year simulation (with 200 year equilibration period), I summarized the results in a table and graphically. For the tabular results, I present the $5^{\text{th}}$ percentile, $25^{\text{th}}$ percentile, $50^{\text{th}}$ percentile, $75^{\text{th}}$ percentile, and $95^{\text{th}}$ percentile of the distribution. I compared the current landscape seral stage distribution to this simulated historic range of variability to determine whether the current landscape deviates, and to what degree, from the HRV. 

Using a stacked bar plot, I visualized the proportion of the total area of a given cover type occurring at each seral stage, for each timestep in the model. In addition, I used a bar plot of the current seral stage distribution to allow a visual comparison between current conditions and the historical range of variability in the distribution of the seral stages. While the bar plots are useful for visualizing the cover type-seral stage dynamics, box plots facilitate a visual comparison of the $5^{\text{th}}-95^{\text{th}}$ percentile distribution (the HRV) to the current landscape values.

\emph{Landscape Configuration:} I used \textsc{Fragstats} \citep{Fragstats2012} to compute several landscape-level and class-level metrics that summarize landscape structure and pattern over the course of the simulation. I present the results in a series of tables and figures. A general introduction to  the \textsc{Fragstats} metrics are included as Appendix~\ref{app:metricdescriptions}. For a much more detailed and mathematical description of all \textsc{Fragstats} metrics, see the \href{http://www.umass.edu/landeco/research/fragstats/documents/fragstats.help.4.2.pdf}{documentation}. Each metric is computed on the study area for a single timestep, and the results are displayed in tabular format by quantiles and in graphical format with line graphs and boxplots. Table~\ref{tab:fragland-desc} summarizes the \textsc{Fragstats} metrics selected as focal metrics to provide a simple and understandable explanation of the characteristics of landscape structure during the simulated HRV. I selected metrics to represent commonly identified groups of landscape metrics: patch area and edge, patch shape complexity, core area, aggregation, and diversity \citep{McGarigal2015}. It is fairly intuitive to understand how these metrics may be affected by natural disturbance and human management efforts, thus allowing us to describe the HRV and develop suggestions tying management actions to results for these metrics.



\begin{table}[!htbp]
\footnotesize
\centering
\caption{A subset of \textsc{Fragstats} metrics I selected to emphasize in order to provide a parsimonious explanation of the variability in landscape structure during the simulated HRV. An `X' in the landscape or class column denotes whether a metric is calculated at that level. Abbreviations are included because they are used in tables and figures later in the document and in the appendices to conserve space.} 
\label{tab:fragland-desc}
%
\begin{tabular}{@{}llccc@{}}
\toprule
{\bf Metric}                    & {\bf Abbreviation} & {\bf \begin{tabular}[c]{@{}c@{}}Landscape-\\ level\end{tabular}} & {\bf \begin{tabular}[c]{@{}c@{}}Class-\\ level\end{tabular}} & {\bf Category} \\ 
\midrule
Edge Density                    & \textsc{ed} 			& X        & X     & area and edge metric		\\ 
Area-Weighted Mean Area         & \textsc{area\_am}  	& X        & X     & area and edge metric		\\
Area-Weighted Mean Shape        & \textsc{shape\_am} 	& X        & X     & shape metric 				\\
Area-Weighted Mean Core Area    & \textsc{core\_am}  	& X        & X     & core area metric		\\
Contagion                       & \textsc{contag} 		& X        & --    & aggregation metric		\\
Clumpiness Index                & \textsc{clumpy} 		& --       & X     & aggregation metric		\\
Simpson’s Evenness Index        & \textsc{siei}      	& X        & --    & diversity metric		\\
\bottomrule
\end{tabular}
\end{table}

I summarized the 90\% range of variability for both the composition and pattern metrics, and from this inferred the extent to which the current landscape departs from that range of variability. Thus I use both a quantitative and a qualitative assessment to determine how and to what extent the landscape has changed. Based on these results, I analyze different potential causes of the results, with an emphasis on understanding how human activities may be the primary factor related to any change. 

For both the composition and pattern metrics, I quantified the current landscape's departure from the HRV conditions by summarizing the distribution of each \textsc{Fragstats} metric calculated over the length of the simulation, minus the equilibration period. I computed the $5^{\text{th}}$, $25^{\text{th}}$, $50^{\text{th}}$, $75^{\text{th}}$, and $95^{\text{th}}$ percentiles of the distribution of observed values. I calculated a current percentile of the range of variability value (\%RV) by computing where along the $0^{\text{th}}-100^{\text{th}}$ percentile range of variability for the simulated historical period the current landscape metric value falls. 

To assess landscape composition and configuration, I compared the current landscape to the HRV, and report departure based on the following standards. If the current landscape metric value falls within the $25^{\text{th}}-75^{\text{th}}$ percentile range (the box in our boxplots), it is considered not departed. If it falls within the $5^{\text{th}}-25^{\text{th}}$ percentile range or the $75^{\text{th}}-95^{\text{th}}$ percentile range (the whiskers in our boxplots), it is moderately departed. If it falls outside that range, it is completely departed.And if it falls to its death, it is dearly departed. Thus, for the landscape metric \emph{Patch Density}, 19.507 is equivalent to the 32$^{\text{nd}}$ percentile of observations during the simluated HRV, and this metric is therefore currently within the HRV for the landscape. However, the landscape metric \emph{Edge Density} has a current value of 128.875. Because $128.875 > 125.316$, and 125.316 is the largest value observed during the simluated HRV, edge density at the landscape level is currently completely departed from the HRV. 

\clearpage

%%%%%%%%%%%%%%%%%%%%%%%%%%%%%%%%%%%%%%%%%%%%%%%%%%%%%%%%%%%%%%%%%%%%%%%%%%%%%%%%%%%%%%%%%%%%%%%%%%%%%%%%%%%%
\subsection{Model Assessment}
\label{subsec:modelassessment}

\subsubsection*{Sensitivity Analysis} Ideally, a sensitivity analysis would be performed to assess the sensitivity of the input parameters to the model, and subsequently indicate areas for future research. In this case, I did not complete a rigorous sensitivity analysis due primarily to practical constraints. The chief constraints were related to time, for rerunning the model many times under varying parameter sets, and disk space, due to the large amount of data generated by each run of the model. Despite this, as part of the model calibration process, I gained insight into the relative sensitivity of some parameters. Thus I can offer a qualitative sensitivity analysis. First, the ignition parameter is quite sensitive; changing it by a few interval values changes model outcomes for most analysis measures. Presumably this happens because increasing the number of potential fire starts increases the odds of a fire initiating on a susceptible cell (A formal evaluation of this effect is outside the scope of this project.). In comparison, the fire return index is relatively insensitive; I often modified it by more than an order of magnitude in order to effect a small change in the rotation outcome. Third, the probability of high mortality fire at the seral stage level is fairly sensitive. This is logical because the conversion of forest to early seral conditions directly impacts most of the metrics by which I evaluate landscape structure and composition and because high mortality fire results in a transition to Early Development 100\% of the time, so its impact is more direct and thus directly measurable..

Of the parameters observed to be more sensitive, the seral stage-level likelihood of high mortality fire is the one whose effects on model outcomes have the most important implications for my results, and are most important to invest further research effort on. Currently, probabilities of high mortality fire at the seral stage level are extremely difficult to find in the literature because no record can be taken from a tree completely consumed in a fire. Only a few researchers have attempted to infer high severity fire based on factors such as later reports of dense young conifer or shrub cover \citep{Collins2011,Baker2014,Stephens2015}.\todo{could maybe still think here about what else I can say about the baker paper} Most studies of stand-replacing fire have relied on satellite imagery to confirm ``stand-replacement'' effects \citep[e.g.,][]{Collins2010,Mallek2013}. Even if legacy trees\footnote{trees that are much older than the overall stand and that are presumed to have been left standing after a prior stand-replacing disturbance} existed and could be sampled to infer dates of past high severity fires, it would be difficult to determine when and over what extent past stand-replacing fires burned because of low sample sizes associated with legacy trees, as well as other factors such as post-fire drought \citep{Minnich2000,Baker2014}. Furthermore, the studies that derive percent high severity based on imagery produce overall cover type estimates, rather than estimates based on seral stage. Because it is the seral stage estimates that are needed to improve the model, research into this area would fill a gap in our ecological understanding and enhance the value of \textsc{RMLands}-based results.

\subsubsection*{Uncertainty Analysis} The model calibration process involved systematically varying certain input parameters and testing resultant model outcomes, which allowed me to complete a rudimentary and qualitative sensitivity analysis. An analogous process did not occur that might approach an uncertainty analysis. Conducting an uncertainty analysis on the parameter set could theoretically be accomplished, but the same constraints exist as for the sensitivity analysis. In addition to these, established uncertainties were not available for all input parameters (such as probability of high mortality fire at the seral stage level) and were often used to parameterize the stochastic part of the model when available (such as succession rates by land cover type and seral stage). Although not conducted as part of this study, an uncertainty analysis would add value to the study results.

\subsubsection*{Model Validation} I did not conduct a formal model validation, defined as testing the model outputs against independent data. One way to do this would have been to test the parameter set on a neighboring geography with a similar ecological composition. The comparison would then be done either on the results of a hindcasting exercise or on the results of recreating the historical conditions for the next few hundred years. Clearly, the latter exercise is impossible, so model validation could not be accomplished that way. Unfortunately, a hindcasting exercise is also not implementable. First, the model only simulates wildfires; many other small-scale disturbances also occur in the study area and affect landscape pattern. Second, attempts to recreate the current conditions on the landscape would be confounded with the history of vegetation management from the last 150 years. At the time of this study, \textsc{RMLands} functionality for simulating vegetation treatments in the Sierra Nevada was still under development. Even if it was available, however, the existing descriptions of past vegetation treatments are not sufficiently detailed to use in a model validation exercise. A final potential method for validating the model would be to use an old land cover type and seral stage map, and compare its composition and configuration to the HRV results. However, no such map exists, and if it had, I would have used it as my starting condition and eliminated the need for model equilibration.

In addition, it is important to understand that this model can never be fully validated because, while useful, it is like all models an abstract and simplified representation of reality. \textsc{RMLands} was set up to simulate wildfires, but there are many other disturbance processes that exist at varying scales that are not simulated here, including insects and disease, wind-throw, wild ungulate and beaver herbivory, avalanches, and other forms of soil movement. The complex interactions among them that characterize real landscapes are also, as a result, omitted from consideration.

Finally, to parameterize the mdoel, I used local empirical data wherever possible. However, I also drew on relevant scientific studies, often from other geographic locations, and relied heavily on expert opinion when scientific studies and local empirical data were not available. As a consequence, the pool of potentially available independent data is limited. 

\subsubsection*{Model Evaluation} \label{subsubsec:modelevaluation} Because true model validation was not possible for this study, a secondary method for validation is to test whether the model outputs make sense ecologically and based on available empirical data. These strategies bleed into model evaluation, or the degree to which the model outputs line up with empirical observations. As outlined in the previous paragraph, the most straightforward method for model evaluation would be to employ a hindcasting strategy, but this is not practicable. 

To some extent, the fact that model calibration was highly successful, in that output fire rotations were within 10\% or less of target values (Table~\ref{rotation-diff}), provides a positive form of model evaluation. I used rotation values as the calibration target because targets were available from the literature and because fire rotation is a fundamental measurement that \textsc{RMLands} was designed to capture. In addition, using rotation ties calibration to a parameter that is relatable to Forest Service staff and that can be used as a target by managers in various programs. 

% fixed 2016-02-06
\begin{table}[!htbp]
\centering
\footnotesize
\caption{Comparison of the target versus actual overall fire rotations recorded during the simulated historical range of variability. Includes calculated final percent difference.}
\label{rotation-diff}
\begin{tabular}{@{}lrrr@{}}
 \toprule
 \textbf{\begin{tabular}[c]{@{}l@{}}Land Cover Type\end{tabular}} &
 \textbf{\begin{tabular}[c]{@{}l@{}}Target \\ Rotation\end{tabular}} &
 \textbf{\begin{tabular}[c]{@{}l@{}}Actual \\ Rotation\end{tabular}} &
 \textbf{\begin{tabular}[c]{@{}l@{}}Percent \\ Difference\end{tabular}} \\
\midrule
Mixed Evergreen - Mesic            & 50    & 52    & 4\%   \\
Mixed Evergreen - Xeric            & 40    & 41    & \textless 1\%   \\
Oak-Conifer Forest and Woodland    & 21    & 22    & 5\%   \\
Red Fir - Mesic                    & 60    & 63    & 5\%   \\
Red Fir - Xeric                    & 40    & 38    & 5\%   \\
Sierran Mixed Conifer - Mesic      & 29    & 27    & 7\%   \\
Sierran Mixed Conifer - Ultramafic & 60    & 66    & 10\%  \\
Sierran Mixed Conifer - Xeric      & 22    & 23    & 5\%   \\
\bottomrule
\end{tabular}
\end{table}


A second method of model evaluation was a visual inspection of the output grids demonstrating wildfire extents to verify that they were similar to actual wildfire perimeters. In addition, I plotted the actual disturbance size distribution against the expected distribution (Figure~\ref{fig:dsize}). 


% updated 9/13
\begin{figure}[!htbp]
  \centering
    \centering
    \includegraphics[height=0.3\textheight]{/Users/mmallek/Documents/Thesis/Plots/dsize/hrv-ggplot.png}
  \caption{Side by side barplot of the observed and target wildfire size distribution for our 500-timestep long run of the model.}
  \label{fig:dsize}
\end{figure}

As a further effort toward model evaluation, I examined the results of implementating the topographic position index (TPI). The TPI value for a given cell acts as an input into the susceptibility and mortality values otherwise defined for that cover type and seral stage combination. Early development and open canopy seral stages tend to result from fire, and I predicted that an increase in fires and in the likelihood of high mortality fire would lead to a decrease in the average canopy cover values for cells with large TPI values. Table~\ref{tab:tpi_cc} in Appendix \ref{app:full-results} displays the results for this simulation for the nine most common cover types. All show decreased average canopy cover as TPI increases. Figure \ref{fig:tpi_cc_smc} shows the plotted data and fitted linear regression line for mesic and xeric sierran mixed conifer forests. Figure~\ref{fig:averagecc} is a map displaying average canopy cover across the landscape for the full simulated HRV timeframe, excluding the equilibration period. In general, rotations and canopy cover varied spatially across the forest and decreased with increasing TPI, reflecting empirical observations that higher, more southerly aspects are drier and more susceptible to fires. In mesic mixed conifer forests, canopy cover decreased by about 9\% when comparing minimum to maximum TPI, from an average of 55.5\% to an average of 50.4\%. In xeric mixed conifer forests, canopy cover decreased by 20.5\% when comparing minimum to maximum TPI, from an average of 27.6\% to an average of 21.9\% (Table~\ref{tab:tpi_cc_smcs}).


% figure redone
\begin{figure}[!htbp]
\centering
\includegraphics[width=.8\textwidth]{/Users/mmallek/Documents/Thesis/Plots/tpi/hrv-facet-smc.png}
\caption{Average canopy cover for Sierran Mixed Conifer Mesic and Xeric during the simulated HRV. Each blue point represents one pixel of an individual cover type on the landscape grid. The black line is the result of a linear regression fit to the data. Table \ref{tab:tpi_cc} provides the numerical representation of the shift from minimum to maximum TPI values for each cover type. (a) Sierran Mixed Conifer - Mesic; (b) Sierran Mixed Conifer - Xeric.}
\label{fig:tpi_cc_smc}
\end{figure}

% figure redone
\begin{figure}[!htbp]
\centering
\includegraphics[width=0.8\textwidth]{/Users/mmallek/Documents/Thesis/maps/averagecanopycover.pdf}
\caption{Smoothed visualization of the average canopy cover across the study area over the course of the simulation. Higher percent cover is shown in dark blue, transitioning to red where average percent cover was low. Land cover types Water and Barren have no canopy cover value and appear as grey.}
\label{fig:averagecc}
\end{figure}

%redone 9/15
%\begin{table}[!htbp]
%\footnotesize
%\centering
%\caption{The percent change in canopy cover from the minimum TPI value for that cover type to the maximum TPI value. Results for Sierran Mixed Conifer Mesic and Xeric shown here; results for other focal cover types available in Appendix~\ref{app:full-results}}.
%\label{tab:tpi_cc_smcs}
%\begin{tabular}{@{}lrrrrr@{}}
%\toprule
% \small \textbf{\begin{tabular}[c]{@{}l@{}}Cover \\ Name\end{tabular}} & \small \textbf{\begin{tabular}[c]{@{}l@{}}Minimum \\ TPI\end{tabular}} & \small \textbf{\begin{tabular}[c]{@{}l@{}}Maximum \\ TPI\end{tabular}} & \small \textbf{\begin{tabular}[c]{@{}l@{}}Average Canopy \\Cover at \\ Minimum TPI\end{tabular}} & \small \textbf{\begin{tabular}[c]{@{}l@{}}Average Canopy \\ Cover at \\ Maximum TPI\end{tabular}}  & \small \textbf{\begin{tabular}[c]{@{}l@{}}Percent \\ Change in \\ Canopy \\ Cover\end{tabular}} \\ \midrule
%\textsc{smc\_m   }    & -300                 & 300                  & 55.5       & 50.4              & -9.3      \\
%\textsc{smc\_x   }    & -300                 & 300                  & 27.6       & 21.9              & -20.5     \\ \bottomrule
%\end{tabular}
%\end{table}

\begin{table}[!htbp]
\footnotesize
\caption{For each cover type on the landscape, the percent change in canopy cover from the minimum TPI value for that cover type to the maximum TPI value. For seral stage abbreviations, see Table \ref{condtable}.}
\label{tab:tpi_cc}
%\rotatebox{90}{
\begin{tabular}{@{}lrrrrr@{}}
\toprule 
 \textbf{\begin{tabular}[c]{@{}l@{}}Cover \\ Name\end{tabular}} & \small \textbf{\begin{tabular}[c]{@{}l@{}}Minimum \\ TPI\end{tabular}} & \small \textbf{\begin{tabular}[c]{@{}l@{}}Maximum \\ TPI\end{tabular}} & \small \textbf{\begin{tabular}[c]{@{}l@{}}Average Canopy \\Cover at \\ Minimum TPI\end{tabular}} & \small \textbf{\begin{tabular}[c]{@{}l@{}}Average Canopy \\ Cover at \\ Maximum TPI\end{tabular}}  & \small \textbf{\begin{tabular}[c]{@{}l@{}}Percent \\ Change in \\ Canopy \\ Cover\end{tabular}} \\ \midrule
\textsc{meg\_m   }    & -300                 & 300   & 73.7       & 67.0     & -9.0      \\
\textsc{meg\_x   }    & -299                 & 300   & 72.7       & 68.5     & -5.7      \\
\textsc{ocfw     }    & -300                 & 300   & 50.0       & 45.6     & -8.7       \\
\textsc{rfr\_m   }    & -300                 & 300   & 72.1       & 64.0     & -11.2     \\
\textsc{rfr\_x   }    & -259                 & 300   & 40.2       & 29.1     & -27.6     \\
\textsc{smc\_m   }    & -300                 & 300   & 55.5       & 50.4     & -9.3       \\
\textsc{smc\_u   }    & -300                 & 300   & 39.9       & 28.9     & -27.7     \\
\textsc{smc\_x   }    & -300                 & 300   & 27.6       & 21.9     & -20.5     \\ \bottomrule
\end{tabular}
%}
\end{table}

\clearpage
% !TEX root = master.tex
\section{Results}
\label{sec:hrvresults}
\subsection{Disturbance Regime}

\subsubsection*{Burned Area and Wildfire Frequency}

% 174830 eligible hectares
% 181553 hectares in core
% check math using Wildfire_darea_trajectory.csv
% redone 9/15

In this section the same results are presented in multiple ways; although this is obviously redundant, it is done with the intent of being as inclusive as possible with respect to the potential audience. The HRV analysis presented here forms the backbone of a formal report to be submitted to the U.S. Forest Service for use in understanding and planning future management actions. In my experience, variations in the training and background of individuals affects how they understand and interpret different values. Some prefer probabilistic results while others find results tied to real terms most useful. In order to not privilege either perspective, I have included a range of information translated in different ways. My goal is to facilitate understanding of the results for both academic and professional audiences.

Approximately 96\% of the landscape was eligible for wildfire disturbance (all cover types except Barren and Water).\footnote{In this section I report values based on percent of eligible landscape. There are 181,550 hectares in the core study area, and 174,830 remain after excluding Barren and Water.} As expected, the frequency and extent of simulated wildfires varied across timesteps. This variability is illustrated in Figure~\ref{fig:darea-a}. The same data is also represented as a histogram (Figure~\ref{fig:darea-b}), which highlights the rarity of extremely expansive wildfire damage. It was more common during the simulation for fire to burn across 5--20\% of the landscape during a timestep.

As a result, the specified rotation interval and percent mortality expected over time on this landscape, large proportions of the study area burned each (5-year) timestep. As detailed in Table \ref{tab:darea_atleast}, not only did fire occur in every timestep of the simulation, it also extended across over 10\% of the landscape during most timesteps. In the median timestep (Table \ref{tab:darea}), about 14\% of the landscape burned, which translates into 24,500 hectares of burned land, 6,600 hectares of which burned at high mortality (though not necessarily contiguously). High mortality fires do include the burning of Early Development vegetation, including chaparral, when it resets the successional process. Further details on the percent of the full landscape that burned, and the breakdown of low versus high mortality outcomes, are included in Tables \ref{tab:darea_atleast} and \ref{tab:darea}.

% created 9/17
\begin{table}[!htbp]
\footnotesize
\centering
\caption{Summary of disturbed area in terms of proportion of the landscape burned during the simulation (after the equilibration period). For each benchmark proportion ($>1$\%, $>10$\%, $>25$\%, or $>50$\%) of the landscape that burns, I provide a few representations of the data intended to support different ways of considering the results, that I hope will serve both modelers' and managers' perspectives. First, I counted the number of timesteps during the simulation when that extent burned at either high or low mortality ($n$). Then I calculated the proportion of timesteps that represents ($p = n/500$). The inverse of this is the interval in timesteps, i.e., approximately every 4 timesteps, at least 25\% of the landscape burned. ($t = 1/p$). Finally, I converted the interval in timesteps to the interval in years ($y = 5t$).}
\label{tab:darea_atleast}
\begin{tabular}{@{}lllll@{}}
\toprule
 \textbf{Proportion of Landscape Burned} & \textbf{$\mathbf{>1}$\%}     & \textbf{$\mathbf{>10}$\%}    & \textbf{$\mathbf{>25}$\%}    & \textbf{$\mathbf{>50}$\%} \\ \midrule
Number of timesteps ($n$)        & 459              & 313              & 115              & 13            \\
Proportion of timesteps ($p = n/500$)    & 1.00             & 0.68             & 0.25             & 0.03          \\
Interval (timesteps) ($t = 1/p$)      & 1.00             & 1.47             & 4.01             & 35.46         \\
Interval (years)    ($y = 5t$)       & 5.02             & 7.36             & 20.04            & 177.31        \\ \bottomrule
\end{tabular}
\end{table}

% redone 9/16
\begin{table}[!htbp]
\footnotesize
\centering
\caption{Summary statistics for wildfire frequency by area disturbed during the simulation. Values are expressed as percentage and areal extent (in hectares) of the landscape eligible for disturbance that was actually burned.}
\label{tab:darea}
\begin{tabular}{@{}llll@{}}
\toprule
\textbf{\begin{tabular}[c]{@{}l@{}}Summary Statistic \\ (burned area/timestep)\end{tabular}}    & \textbf{Low Mortality}   & \textbf{High Mortality}    & \textbf{Any Mortality}   \\
\midrule                      %Low                      %high                 %any
$5^{\text{th}}$ percentile         &   2.72 (4,763)        & 0.71 (1,244)     &    3.54 (6,184)         \\
$50^{\text{th}}$ percentile        &   10.47 (18,300)      & 3.75 (6,563)     &    14.04 (24,544)         \\
$95^{\text{th}}$ percentile        &   31.29 (54,703)      & 21.43 (37,461)   &    45.88 (80,209)          \\
   Mean                            &   13.20 (23,079)      & 4.87 (8,512)     &    18.07 (31,592)         \\
\bottomrule
\end{tabular}
\end{table}

%redone 9/13


\begin{figure}[!htbp]
  \centering
  \subfloat[][]{
    \centering
    \includegraphics[width=0.5\textwidth]{/Users/mmallek/Documents/Thesis/Plots/darea/hrv_all.png} 
    \label{fig:darea-a}
    }%
  \subfloat[][]{
    \includegraphics[width=0.5\textwidth]{/Users/mmallek/Documents/Thesis/Plots/darea/hrv_newhist_all.png}
    \label{fig:darea-b}
    }
  \caption{(a) Disturbance trajectory for wildfire during the simulation, excluding the equilibration period. Red bars represent high mortality fire, while green bars represent low mortality fire and are stacked on top of high mortality. (b) Histogram of the percent of the landscape burned per timestep.} 
  \label{fig:darea}
\end{figure}

While the tables and figures above represent the data aspatially, it can also be helpful to look at individual timesteps during the simulation. Because \textsc{RMLands} is spatially explicit in its wildfire generation processes, the burned areas generated within the model look like fire perimeter maps created following actual wildfire events, and occur in the areas on the landscape anticipated to be the most susceptible to wildfire. The extent of burned area observed during the simulation is far above what has been observed in recent decades \citep{calfire2012,usgs-fire-data2012}. Beyond saying it is much greater, however, it can be difficult to imagine and conceptualize what fire covering, for example, 50\% of the study area looks like. To facilitate understanding of the statistical outcomes of this work, I created maps displaying the mortality outcomes from wildfires at four key timesteps: the minimum, maximum, median, and mean area burned\footnote{Because the exact mean area burned value did not correspond to any timestep, I display the one with the closest value for the ``any mortality'' category.} in Figures~\ref{fig:darea_min_map}--\ref{fig:darea_mean_map}. During timesteps when a large amount of fire is recorded, multiple individual fires can ``run together'' on the landscape. Each timestep in the model represents five years, and the model does not differentiate between individual fire seasons below this level. Thus the included individual fire map (right-hand figures) offers a visualization of fires over the short term.

\newpage

% background color 24, 15, 41, 0
\begin{figure}[!htbp]
  \centering
  \subfloat[][]{
    \centering
    \includegraphics[width=0.5\textwidth]{/Users/mmallek/Documents/Thesis/maps/hrv-wfmort-5th.pdf}
    \label{fig:darea_min}
  }%
  \subfloat[][]{
    \includegraphics[width=0.5\textwidth]{/Users/mmallek/Documents/Thesis/maps/hrv-distid-5th.pdf}
    \label{fig:distid_min}
  }
  \caption{Maps of area burned during the timestep in the \textbf{$\mathbf{5}^{\text{th}}$ percentile for area burned (3.54\%)} during the simulation. (a) Map by mortality level. Red indicates high mortality fire, while orange indicates low mortality fire. (b) Map showing each individual fire in a different color.}
  \label{fig:darea_min_map}
\end{figure}

\begin{figure}[!htbp]
  \centering
  \subfloat[][]{
    \centering
    \includegraphics[width=0.5\textwidth]{/Users/mmallek/Documents/Thesis/maps/hrv-wfmort-median.pdf}
    \label{fig:darea_median}
  }%
  \subfloat[][]{
    \includegraphics[width=0.5\textwidth]{/Users/mmallek/Documents/Thesis/maps/hrv-distid-median.pdf}
    \label{fig:distid_median}
  }
  \caption{Maps of area burned during the timestep with the \textbf{$\mathbf{50}^{\text{th}}$ percentile for area burned (14.04\%)} during the simulation. (a) Map by mortality level. Red indicates high mortality fire, while orange indicates low mortality fire. (b) Map showing each individual fire in a different color.}
  \label{fig:darea_median_map}
\end{figure}

\begin{figure}[!htbp]
  \centering
  \subfloat[][]{
    \centering
    \includegraphics[width=0.5\textwidth]{/Users/mmallek/Documents/Thesis/maps/hrv-wfmort-95th.pdf}
    \label{fig:darea_max}
  }%
  \subfloat[][]{
    \includegraphics[width=0.5\textwidth]{/Users/mmallek/Documents/Thesis/maps/hrv-distid-95th.pdf}
    \label{fig:distid_max}
  }
  \caption{Maps of area burned during the timestep with the \textbf{$\mathbf{95}^{\text{th}}$ percentile for area burned (45.88\%)} during the simulation. (a) Map by mortality level. Red indicates high mortality fire, while orange indicates low mortality fire. (b) Map showing each individual fire in a different color.}
  \label{fig:darea_max_map}
\end{figure}

\begin{figure}[!htbp]
  \centering
  \subfloat[][]{
    \centering
    \includegraphics[width=0.5\textwidth]{/Users/mmallek/Documents/Thesis/maps/hrv-wfmort-mean.pdf}
    \label{fig:darea_mean}
  }%
  \subfloat[][]{
    \includegraphics[width=0.5\textwidth]{/Users/mmallek/Documents/Thesis/maps/hrv-distid-mean.pdf}
    \label{fig:distid_mean}
  }
  \caption{Maps of area burned during the timestep with the \textbf{mean total area burned (18.07\%)} during the simulation. (a) Map by mortality level. Red indicates high mortality fire, while orange indicates low mortality fire. (b) Map showing each individual fire in a different color.}
  \label{fig:darea_mean_map}
\end{figure}

\clearpage

%%%%%%%%%%%%%%%%%%%%%%%%%%%%%%%%%%%%%%%%%%%%%%%%%%%%%%%%%%%%%%%%%%%%%%%%%%%%%%%%%%%%%%%%%%%%%%%%%%%%%%%%%%%%%%%%%%%%%%%%%%%%%%%%%%%%

\paragraph*{Sierran Mixed Conifer - Mesic}
Sierran Mixed Conifer - Mesic (\textsc{smc\_m}) is the dominant cover type within the core study area, encompassing 57,853 ha and comprising roughly 32\% of the study area. Wildfire was prevalent in this cover type. I again present figures and tables that incorporate some redundancy in order to facilitate understanding by a broad audience, as described in the beginning of this section (Figure \ref{fig:darea_smcm}). I summarize the disturbance regime in Tables~\ref{tab:darea_smcm} and \ref{tab:darea_atleast_smcm}. The frequency and extent of burned area is similar to that for the landscape as a whole.

% plots redone
\begin{figure}[!htbp]
  \centering
  \subfloat[][]{
    \centering
    \includegraphics[width=0.5\textwidth]{/Users/mmallek/Documents/Thesis/Plots/darea/hrv_smcm.png}
    }%
  \subfloat[][]{
    \includegraphics[width=0.5\textwidth]{/Users/mmallek/Documents/Thesis/Plots/darea/hrv_newhist_smcm.png}
    }
  \caption{\small (a) Disturbance trajectory for Sierran Mixed Conifer - Mesic. High mortality fire in red; low mortality fire in green. (b) Histogram of the percent of the landscape burned per timestep.} 
  \label{fig:darea_smcm}
\end{figure}

% updated 2015-09-21
\begin{table}[!htbp]
\footnotesize
\centering
\caption{Disturbed area summary statistics for Sierran Mixed Conifer - Mesic (\textsc{smc\_m}). Proportions shown are relative to the total area of \textsc{smc\_m}.}
\label{tab:darea_smcm}
\begin{tabular}{@{}lrrr@{}}
\toprule
\textbf{\begin{tabular}[c]{@{}l@{}}Summary Statistic \\ (burned \textsc{SMC\_M}/timestep)\end{tabular}} & \textbf{Low Mortality} & \textbf{High Mortality} & \textbf{Any Mortality} \\ \midrule
$5^{\text{th}}$ percentile        & 2.60  & 0.47  & 3.17  \\
$50^{\text{th}}$ percentile       & 11.45 & 3.35  & 14.89 \\
$95^{\text{th}}$ percentile       & 34.17 & 11.57 & 45.27 \\
Mean                              & 14.42 & 4.42  & 18.83 \\
\bottomrule
\end{tabular}
\end{table}


\begin{table}[!htbp]
\footnotesize
\centering
\caption{Summary of disturbed area in terms of proportion of the amount of Sierran Mixed Conifer - Mesic (\textsc{smc\_m}). See Table~\ref{tab:darea_atleast} caption for details.}
\label{tab:darea_atleast_smcm}
\begin{tabular}{@{}lllll@{}}
\toprule
 \textbf{Proportion of SMC\_M Burned} & \textbf{$\mathbf{>1}$\%}     & \textbf{$\mathbf{>10}$\%}    & \textbf{$\mathbf{>25}$\%}    & \textbf{$\mathbf{>50}$\%} \\ \midrule
Number of timesteps ($n$)    & 458          & 311           & 126           & 15            \\
Proportion of timesteps ($p = n/500$) & 0.99         & 0.67          & 0.27          & 0.03          \\
Interval (timesteps) ($t = 1/p$)   & 1.01         & 1.48          & 3.66          & 30.73         \\
Interval (years) ($y = 5t$)       & 5.03         & 7.41          & 18.29         & 153.67       \\ \bottomrule
\end{tabular}
\end{table}



%\clearpage
%%%%%%%%%%%%%%%%%%%%%%%%%%%%%%%%%%%%%%%%%%%%%%%%%%%%%%%%%%%%%%%%%%%%%%%%%%%%%%%%%%%%%%%%%%%%%%%%%%%%%%%%%%%%%%%%%%%%%%%%%%%%%%%%%%%%


\paragraph*{Sierran Mixed Conifer - Xeric}
Sierran Mixed Conifer - Xeric (\textsc{smc\_x}) is the second most dominant cover type within the core study area, encompassing 52,198 ha and comprising roughly 29\% of the study area. Wildfire was prevalent in this cover type. I again present figures and tables that incorporate some redundancy in order to facilitate understanding by a broad audience (Figure \ref{fig:darea_smcx}). I summarize the disturbance regime in Tables \ref{tab:darea_smcx} and \ref{tab:darea_atleast_smcx}. Low and high mortality fires were more frequent and extensive on the xeric mixed conifer forests than in mesic mixed conifer forests or the study area as a whole. High mortality wildfire on xeric mixed conifer forests extended over a larger mean and median proportion compared to the overall landscape, although the $95^{\text{th}}$ percentile value for high mortality fire extent in xeric mixed conifer forests was 3.2\% percentiles less than that for mesic mixed conifer forests.

% updated 2015-09
\begin{figure}[!hbp]
  \centering
  \subfloat[][]{
    \centering
    \includegraphics[width=0.5\textwidth]{/Users/mmallek/Documents/Thesis/Plots/darea/hrv_smcx.png}
    }%
  \subfloat[][]{
    \includegraphics[width=0.5\textwidth]{/Users/mmallek/Documents/Thesis/Plots/darea/hrv_newhist_smcx.png}
    }
  \caption{\small (a) Disturbance trajectory for Sierran Mixed Conifer - Xeric. High mortality fire in red; low mortality fire in green. (b) Histogram of the percent of the landscape burned per timestep.} 
  \label{fig:darea_smcx}
\end{figure}

% updated 2015-09-21
\begin{table}[!htbp]
\footnotesize
\centering
\caption{Disturbed area summary statistics for Sierran Mixed Conifer - Xeric (\textsc{smc\_x}). Proportions shown are relative to the total area of \textsc{smc\_x}.}
\label{tab:darea_smcx}
\begin{tabular}{@{}llll@{}}
\toprule
\textbf{\begin{tabular}[c]{@{}l@{}}Summary Statistic \\ (disturbed SMC\_X/timestep)\end{tabular}} & \textbf{Low Mortality} & \textbf{High Mortality} & \textbf{Any Mortality} \\ \midrule
$5^{\text{th}}$ percentile    & 3.30  & 1.00  & 4.50  \\
$50^{\text{th}}$ percentile   & 11.92 & 5.17  & 17.55 \\
$95^{\text{th}}$ percentile   & 36.02 & 18.20 & 54.28 \\
Mean                          & 14.88 & 6.95  & 21.83 \\  \bottomrule
\end{tabular}
\end{table}


\begin{table}[!htbp]
\footnotesize
\centering
\caption{Summary of disturbed area in terms of proportion of the amount of Sierran Mixed Conifer - Xeric (\textsc{smc\_x}) burned during the simulation. See Table~\ref{tab:darea_atleast} caption for details.}
\label{tab:darea_atleast_smcx}
\begin{tabular}{@{}lllll@{}}
 \toprule
\textbf{Proportion of SMC\_X Burned} & \textbf{$\mathbf{>1}$\%}     & \textbf{$\mathbf{>10}$\%}    & \textbf{$\mathbf{>25}$\%}    & \textbf{$\mathbf{>50}$\%} \\ \midrule
Number of timesteps ($n$)             & 461          & 347           & 148           & 27            \\
Proportion of timesteps ($p = n/500$) & 1.00         & 0.75          & 0.32          & 0.06          \\
Interval (timesteps) ($t = 1/p$)      & 1.00         & 1.33          & 3.11          & 17.07         \\
Interval (years)    ($y = 5t$)        & 5.00         & 6.64          & 15.57         & 85.37         \\ \bottomrule
\end{tabular}
\end{table}

%%%%%%%%%%%%%%%%%%%%%%%%%%%%%%%%%%%%%%%%%%%%%%%%%%%%%%%%%%%%%%%%%%%%%%%%%%%%%%%%%%%%%%%%%%%%%%%%%%%%%%%%%%%%%%%%%%%%%%%%%%%%%%%%%%%%
%%%%%%%%%%%%%%%%%%%%%%%%%%%%%%%%%%%%%%%%%%%%%%%%%%%%%%%%%%%%%%%%%%%%%%%%%%%%%%%%%%%%%%%%%%%%%%%%%%%%%%%%%%%%%%%%%%%%%%%%%%%%%%%%%%%%

\subsubsection*{Climate Effect} 
% fire size sentence is discussion
Climate has a positive relationship with disturbed area. The climate parameter is regressed against disturbed area in Figure \ref{fig:climate_darea}, but note the heteroskedastic variance about the mean. The relationship is weakly positive, in that as climate shifts from wet to drought, disturbed area increases. The climate parameter is defined such that 1 is the average value over the historical period. During wetter-than-average years, less area was disturbed. For example, no more than 20\% of the landscape burned in any of the timesteps during which the climate parameter was below 0.63. However, over 50\% of the landscape burned in several timesteps when the climate parameter was less than 1 (wet periods). Figure \ref{fig:compare_clim_darea} illustrates the climate parameter values and disturbed area proportion of the landscape for a subset of timesteps during the simulation to illustrate that in some years, a high climate parameter occurs with a higher disturbed area percentage, but in other years a low climate parameter occurs with a higher disturbed area percentage that in mesic mixed conifer forests. Therefore, while a correlation certainly exists between climate and disturbed area, it is not associated with a firm ceiling or floor.

\begin{figure}[!htbp]
  \centering
  \subfloat[][]{
    \centering
  \includegraphics[width=0.5\textwidth]{/Users/mmallek/Documents/Thesis/Plots/darea/hrv_climdarea.png}
    \label{fig:climate_darea}
  } 
  \subfloat[][]{
    \includegraphics[width=0.5\textwidth]{/Users/mmallek/Documents/Thesis/Plots/darea/climate_darea_vert.png}
    \label{fig:compare_clim_darea}
  } 
  \caption{(a) Scatterplot of the climate parameter and disturbed area value for each timestep of the simulation (excluding the equilibration period). A linear model has been fit to the data and is shown as a blue line; the grey shaded area represents the 95\% confidence interval around the mean. (b) Climate parameter and proportion of eligible landscape disturbed by wildfire for timesteps 250 to 310 of the simulation, illustrating the wide variability in both climate parameter values and disturbed area per timestep. The purple shaded area helps map the climate parameter value and proportion of landscape burned during the same individual timestep.}
  \label{fig:climate_darea_combofig}
\end{figure}

\clearpage

\newpage
\subsubsection*{Rotation Period} 
As described in Chapter \ref{sec:hrvmethods}, I calibrated the model by adjusting seral stage-specific susceptibility values until the nine cover types with more than 1000 ha extent across the study area were within 10\% of their target fire rotation. I present here the results for the two focal cover types and the full study area. Full results for all cover types in the study area are presented in Table~\ref{tab:all-rotations}. 

% updated 2015-09
\begin{table}[!htbp]
\footnotesize
\centering
\caption{Fire rotation for Sierran Mixed Conifer - Mesic, Sierran Mixed Conifer - Xeric, and the full landscape.}
\label{tab:ch2-rotations}
\begin{tabular}{@{}lrrr@{}}
\toprule
\begin{tabular}[c]{@{}l@{}}Land Cover \\ Type\end{tabular}     & \begin{tabular}[c]{@{}l@{}}Low Mortality \\ Fire Rotation\end{tabular} & \begin{tabular}[c]{@{}l@{}}High Mortality \\ Fire Rotation\end{tabular} & \begin{tabular}[c]{@{}l@{}}All Fires \\ Rotation\end{tabular} \\ \midrule
\textsc{smc\_m   }      & 35  & 113   & 27    \\
\textsc{smc\_x   }      & 34  & 72    & 23    \\
Full Landscape          & 38  & 103   & 28    \\ \bottomrule
\end{tabular}
\end{table}


\subsubsection*{Point-specific Return Interval}
Visualizing the point-specific fire rotation (return interval) for an individual grid cell calls attention to the variability in wildfire recurrence across the study area. Barplots show the spread and underlying values in the distribution of point-specific fire rotations, and maps demonstrate the spatial variability in this metric across the study area. Overall, the point-specific fire rotation for an individual cell ranged from 17 years to \textgreater 2500 years (cells that never burned during the simulation) for both classes of wildfire mortality (Figure \ref{fig:preturn}). The grand mean return interval across all cover types was 42 years for low mortality fire, 111 year for high mortality fire, and 29 years for any fire. The point-specific fire rotation plots and maps specific to Sierran Mixed Conifer Mesic and Xeric follow (Figures~\ref{fig:preturn_smcm} and \ref{fig:preturn_smcx}). Under this wildfire regime, the point-specific fire rotation for an individual point between fires (of any mortality level) for both of these mixed conifer forest types varied widely from about 17 years to over 500 years, with grand means of 28 years (for mesic) and 23 years (for xeric) (Figures~\ref{fig:preturn_smcm} and \ref{fig:preturn_smcx}). Results for the other seven focal cover types are included in Appendix~\ref{sec:indiv_cov_results}. 



% first plot redone 9/13
% second plot not redone yet
\begin{figure}[!htbp]
  \centering
  \subfloat[][]{
    \centering
    \includegraphics[height=.4\textheight]{/Users/mmallek/Documents/Thesis/Plots/preturn/not-called-preturn/hrv-total.png}
    \label{fig:preturn_plot}
  }%
  \qquad
  \subfloat[][]{
    \includegraphics[height=.4\textheight]{/Users/mmallek/Documents/Thesis/Plots/preturn-maps/fri_all.png}
    \label{fig:preturn_map}
  }
  \caption{(a) Distribution of point-specific fire rotations for the full landscape under study. The ``full landscape'' includes each cell in the raster with a cover type eligible to burn. The point-specific fire rotation is the average interval between fires over the length of the simulation, excluding the equilibration period, at each individual grid cell. (b) Spatially-explicit depiction of these point-specific fire rotations across the landscape, for all cover types.}
  \label{fig:preturn}
\end{figure}

% first plot updated 9/13
\begin{figure}[!htbp]
  \centering
  \subfloat[][]{
    \centering
    \includegraphics[width=0.5\textwidth]{/Users/mmallek/Documents/Thesis/Plots/preturn/not-called-preturn/hrv-smcm.png}
    }%
  \subfloat[][]{
    \includegraphics[width=0.5\textwidth]{/Users/mmallek/Documents/Thesis/Plots/preturn-maps/fri_smcm.png}
    }
  \caption{(a) Distribution of point-specific fire rotations for Sierran Mixed Conifer - Mesic. The point-specific fire rotation is the average interval between fires over the length of the simulation, excluding the equilibration period, at each individual grid cell. (b) Spatially-explicit depiction of these point-specific fire rotations across the landscape. Cover types other than Sierran Mixed Conifer - Mesic are partially obscured in grey.}
\label{fig:preturn_smcm}
\end{figure}

%first plot redone 9/13
\begin{figure}[!htbp]
  \centering
  \subfloat[][]{
    \centering
    \includegraphics[width=0.5\textwidth]{/Users/mmallek/Documents/Thesis/Plots/preturn/not-called-preturn/hrv-smcx.png}
    }%
  \subfloat[][]{
    \includegraphics[width=0.5\textwidth]{/Users/mmallek/Documents/Thesis/Plots/preturn-maps/fri_smcx.png}
    }
  \caption{(a) Distribution of point-specific fire rotations for Sierran Mixed Conifer - Xeric. The point-specific fire rotation is the average interval between fires over the length of the simulation, excluding the equilibration period, at each individual grid cell. (b) Spatially-explicit depiction of these point-specific fire rotations across the landscape. Cover types other than Sierran Mixed Conifer - Xeric are partially obscured in grey.}
\label{fig:preturn_smcx}
\end{figure}

\clearpage


%%%%%%%%%%%%%%%%%%%%%%%%%%%%%%%%
%%%%%%%%%%%%%%%%%%%%%%%%%%%%%%%%
%%%%%%%%%%%%%%%%%%%%%%%%%%%%%%%%

%\pagebreak[4]
\subsection{Vegetation Response}
\label{subsec:HRVvegresponse}


\subsubsection*{Landscape Composition} 

% fixed plots - equilibration line is hard coded in. ocfwu calibration changed; now seems okay by ts 40. potentially could even have cut off equilibration at ts 20 but it's arbitrary. good to keep in mind for future stuff though.
The seral stage distribution for each cover type varied over time, but did appear to be in dynamic equilibrium \citep{Diamond1969}. Evidence of both high mortality fire, which triggers a transition to the Early Development seral stage for all cover types, and low mortality fire, which can thin a stand and cause a transition to a more open canopy seral stage (within the same development level), are visible in examining the output grids. Figure \ref{fig:covcondmaps} illustrates these changes for a sequence of four timesteps during the simulation. The seral stage dynamics and current seral stage distribution plots specific to Sierran Mixed Conifer - Mesic and Sierran Mixed Conifer - Xeric follow (Figures~\ref{fig:hrv-covcond_smcm} and \ref{fig:hrv-covcond_smcx}). I compare the current landscape's seral stage distribution to the simulated distribution and assess the current landscape's departure from the HRV in Tables~\ref{tab:covcond_smcm} and \ref{tab:covcond_smcx}. Plots and tabular results for the other seven focal types are included in Appendix~\ref{sec:indiv_cov_results}.

% new plots 2015-09-18
\begin{figure}[!htbp]
  \centering
  \subfloat[][]{
    \includegraphics[width=0.5\textwidth]{/Users/mmallek/Documents/Thesis/maps/hrv-covcondseq-5.pdf}
  }%
  \subfloat[][]{
    \includegraphics[width=0.5\textwidth]{/Users/mmallek/Documents/Thesis/maps/hrv-covcondseq-6.pdf}
  }\\%
  \subfloat[][]{
    \includegraphics[width=0.5\textwidth]{/Users/mmallek/Documents/Thesis/maps/hrv-covcondseq-7.pdf}
    }
  \subfloat[][]{
    \centering
    \includegraphics[width=0.5\textwidth]{/Users/mmallek/Documents/Thesis/maps/hrv-covcondseq-8.pdf}
  }%
  \caption{A sequence of four timesteps during the middle of the simulation, showing changes in seral stages over time. Here I highlight the dominant cover type, Sierran Mixed Conifer - Mesic, and its classes, in order to illustrate the dynamics that play out over many years. (a) Timestep 1 (b) Timestep 2 (c) Timestep 3 (d) Timestep 4. Patches in shades of brown and tan belong to other cover types.}
  \label{fig:covcondmaps}
\end{figure}


\clearpage

\paragraph*{Sierran Mixed Conifer - Mesic}

% hrv plot updated 2015-09
\begin{figure}[!htbp]
  \centering
  \subfloat[][]{
    \centering
    \includegraphics[width=0.6\textwidth]{/Users/mmallek/Documents/Thesis/Plots/covcond-dynamics/notcalledcovcond/SMCM.pdf}
  }%
  \subfloat[][]{
    \includegraphics[height=2.65in]{/Users/mmallek/Tahoe/R/Rplots/November2014/covcond_current_smcm.png}
  } \\
  \subfloat[][]{
    \includegraphics[width=\textwidth]{/Users/mmallek/Documents/Thesis/Plots/covcond-bycover/SMCM-HRV-boxplots-.png}
  }
  \caption{(a) Cover type-Seral stage dynamics for Sierran Mixed Conifer - Mesic. The black vertical line at 40 timesteps marks the end of the equilibration period used in this study. (b) Current seral stage distribution for Sierran Mixed Conifer - Mesic. (c) Boxplots showing the range of variability for each seral stage over the course of the simulation, excluding the equilibration period. Boxplots were modified so that whiskers extend from the $5^{\text{th}} - 95^{\text{th}}$ percentiles of the observed results. Thick black bars in line with the boxplots denote the current proportion of mesic mixed conifer forests in a given seral stage.} 
  \label{fig:hrv-covcond_smcm}
\end{figure}

The distribution of area among stand seral stages within mesic mixed conifer forests fluctuated over time, but appeared to be in dynamic equilibrium (Figure~\ref{fig:hrv-covcond_smcm}). The percentage of mesic mixed conifer forests in the Early Development seral stage varied from approximately 8\%--25\%. This seral stage is currently within the simulated HRV (48$^{\text{th}}$ percentile). The Late--Moderate seral stage is currently moderately departed within the HRV, but at the edge (95$^{\text{th}}$ percentile). Current proportions of the other five seral stages are completely departed from the simulated HRV.  The current seral-stage distribution was never observed under the simulated HRV (Table~\ref{tab:covcond_smcm}). The current landscape contains more Mid--Moderate, Mid--Open, and Late--Closed, and less Mid--CLosed and Late--Open, than the simulated HRV.


%\begin{landscape}
% table updated 2015-09
\begin{table}[!htbp]
\footnotesize
\centering
\caption{Range of variability in landscape structure, illustrating the cover type-seral stage class dynamics for Sierran Mixed Conifer - Mesic. Included are the $5^{\text{th}}$ percentile, $25^{\text{th}}$ percentile, $50^{\text{th}}$ percentile, $75^{\text{th}}$ percentile, and $95^{\text{th}}$ percentiles of the distribution, as well as the current landscape proportion, the current percentile range of variability (\%RV) for that proportion, and the departure classification. For seral stage abbreviations, see Table \ref{condtable}. Departure is classified as follows: if the current landscape metric value falls within the $25^{\text{th}}-75^{\text{th}}$ percentile range (the box in the boxplots), it is considered not departed (Departure is ``none'' in the table). If it falls within the $5^{\text{th}}-25^{\text{th}}$ percentile range or the $75^{\text{th}}-95^{\text{th}}$ percentile range (the whiskers in the boxplots), it is moderately departed (Departure is ``moderate'' in the table). If it falls outside that range, it is completely departed (Departure is ``complete'' in the table).}
\label{tab:covcond_smcm}
\begin{tabular}{@{}lrrrrr|rrr@{}}
\toprule
\textbf{\begin{tabular}[c]{@{}l@{}}Seral \\ Stage\end{tabular}}  &  \textbf{$\mathbf{5}^{\text{th}}$} &   \textbf{$\mathbf{25}^{\text{th}}$} &   \textbf{$\mathbf{50}^{\text{th}}$} &   \textbf{$\mathbf{75}^{\text{th}}$} &   \textbf{$\mathbf{95}^{\text{th}}$}  &  \textbf{\begin{tabular}[c]{@{}l@{}}Current\\ \%cover\end{tabular}} & \textbf{\begin{tabular}[c]{@{}l@{}}Current\\ \%RV\end{tabular}} & \textbf{\begin{tabular}[c]{@{}l@{}}Departure\end{tabular}} \\ \midrule
\textsc{early\_all}        &   7.75        &  12.34   &  15.11     &  18.68   &  24.74     &  14.98    &  48    &  none      \\
\textsc{mid\_cl   }        &   21.52       &  26.15   &  29.69     &  32.58   &  37.01     &  9.74     &  0     &  complete     \\
\textsc{mid\_mod  }        &   6.8         &  7.98    &  9.03      &  10.3    &  12.63     &  17.97    &  100   &  complete     \\
\textsc{mid\_op   }        &   6.68        &  9.2     &  11.21     &  13.08   &  16.15     &  16.29    &  96    &  complete     \\
\textsc{late\_cl  }        &   5.31        &  9.54    &  12.87     &  17.2    &  22.91     &  23.23    &  97    &  complete      \\
\textsc{late\_mod }        &   8.56        &  10.32   &  11.24     &  12.56   &  14.41     &  14.18    &  95    &  moderate      \\
\textsc{late\_op  }        &   4.96        &  7.39    &  9.26      &  12.12   &  14.95     &  3.6      &  1     &  complete      \\
\bottomrule
\end{tabular}
\end{table}





%\end{landscape}
%\clearpage
%%%%%%%%%%%%%%%%%%%%%%%%%%%%%%%%%%%%%%%%%%%%%%%%%%%%%%%%%%%%%%%%%%%%%%%%%%%%%%%%%%%%%%%%%%%%%%%%
\paragraph*{Sierran Mixed Conifer - Xeric}

% plot updated 2015-09
\begin{figure}[!htbp]
  \centering
  \subfloat[][]{
    \centering
    \includegraphics[width=0.6\textwidth]{/Users/mmallek/Documents/Thesis/Plots/covcond-dynamics/notcalledcovcond/SMCX.pdf}
  }%
  \subfloat[][]{
    \includegraphics[height=2.65in]{/Users/mmallek/Tahoe/R/Rplots/November2014/covcond_current_smcx.png}
  } \\
  \subfloat[][]{
    \includegraphics[width=\textwidth]{/Users/mmallek/Documents/Thesis/Plots/covcond-bycover/SMCX-HRV-boxplots-.png}
  }
  \caption{(a) Cover type-Seral stage dynamics for Sierran Mixed Conifer - Xeric. The black vertical line at 40 timesteps marks the end of the equilibration period used in this study. (b) Current seral stage distribution for Sierran Mixed Conifer - Xeric. (c) Boxplots showing the range of variability for each seral stage over the course of the simulation, excluding the equilibration period. Boxplots were modified so that whiskers extend from the $5^{\text{th}} - 95^{\text{th}}$ percentiles of the observed results. Thick black bars in line with the boxplots denote the current proportion of xeric mixed conifer forests in a given seral stage.}  
  \label{fig:hrv-covcond_smcx}
\end{figure}

The distribution of area among seral stages within xeric mixed conifer forests fluctuated over time, but appeared to be in dynamic equilibrium (Figure~\ref{fig:covcond_smcx}). The percentage of xeric mixed conifer forests in the Early Development varied from approximately 25\% to 43\% (Table~\ref{tab:covcond_smcx}). During the simulation, Early Development (which includes post-fire chaparral fields) and Mid--Open seral stages dominated, in contrast to the current distribution, which is somewhat even across classes (although Late--Open is currently quite rare).
%
The current seral stage distribution was never observed under the simulated HRV, and all of the seral stages were fully departed from the HRV (Table~\ref{tab:covcond_smcx}). The current landscape contains more closed and moderate canopy forest, and less Early Development and open canopy forest, than the simulated HRV.


% table updated 2015-09
\begin{table}[!htbp]
\footnotesize
\centering
\caption{Range of variability in landscape structure, illustrating the cover type-seral stage dynamics for Sierran Mixed Conifer - Xeric. Included are the $5^{\text{th}}$ percentile, $25^{\text{th}}$ percentile, $50^{\text{th}}$ percentile, $75^{\text{th}}$ percentile, and $95^{\text{th}}$ percentiles of the distribution, as well as the current landscape proportion, the current percentile range of variability (\%RV) for that proportion, and the departure classification. For seral stage abbreviations, see Table \ref{condtable}. Departure is classified as follows: if the current landscape metric value falls within the $25^{\text{th}}-75^{\text{th}}$ percentile range (the box in the boxplots), it is considered not departed (Departure is ``none'' in the table). If it falls within the $5^{\text{th}}-25^{\text{th}}$ percentile range or the $75^{\text{th}}-95^{\text{th}}$ percentile range (the whiskers in the boxplots), it is moderately departed (Departure is ``moderate'' in the table). If it falls outside that range, it is completely departed (Departure is ``complete'' in the table).}
\label{tab:covcond_smcx}
\begin{tabular}{@{}lrrrrr|rrr@{}}
\toprule
\textbf{\begin{tabular}[c]{@{}l@{}}Seral \\ Stage\end{tabular}}  &  \textbf{$\mathbf{5}^{\text{th}}$} &   \textbf{$\mathbf{25}^{\text{th}}$} &   \textbf{$\mathbf{50}^{\text{th}}$} &   \textbf{$\mathbf{75}^{\text{th}}$} &   \textbf{$\mathbf{95}^{\text{th}}$}  &  \textbf{\begin{tabular}[c]{@{}l@{}}Current\\ \%cover\end{tabular}} &   \textbf{\begin{tabular}[c]{@{}l@{}}Current\\ \%RV \end{tabular}} &   \textbf{\begin{tabular}[c]{@{}l@{}}Departure\end{tabular}} \\ \midrule
 \textsc{early\_all}      &  25.2          &  29.63    &  34.53    &  38.95    &  42.82     &  19.48       &   0      &  complete    \\
 \textsc{mid\_cl   }      &  0.02          &  0.06     &  0.13     &  0.36     &  1.07      &  11.96       &   100    &  complete      \\
 \textsc{mid\_mod  }      &  0.9           &  1.62     &  2.88     &  4.35     &  7.6       &  14.92       &   100    &  complete    \\
 \textsc{mid\_op   }      &  26.55         &  30.59    &  33.79    &  36.58    &  39.36     &  11.48       &   0      &  complete    \\
 \textsc{late\_cl  }      &  1.19          &  2.51     &  3.81     &  5.99     &  8.69      &  24.72       &   100    &  complete      \\
 \textsc{late\_mod }      &  5.83          &  7.49     &  9.16     &  10.71    &  13.03     &  13.31       &   97     &  complete     \\
 \textsc{late\_op  }      &  9.39          &  12.4     &  15       &  17.42    &  22.45     &  4.13        &   0      &   complete  \\ \bottomrule 
\end{tabular}
\end{table}

\clearpage


%%%%%%%%%%%%%%%%%%%%%%%%%%%%%%%%%%%%%%%%%%%%%%%%%%%%%%%%%%%%%%%%%%%%%%%%%%%%%%%%%%%%%%%%%%%%%%%%
%%%%%%%%%%%%%%%%%%%%%%%%%%%%%%%%%%%%%%%%%%%%%%%%%%%%%%%%%%%%%%%%%%%%%%%%%%%%%%%%%%%%%%%%%%%%%%%%

\subsubsection*{Landscape Configuration}
I summarized the structure and patterns in the landscape using a suite of statistical measures calculated using \textsc{Fragstats}. Table \ref{tab:fragland} shows the range of variability for the simulation period as well as the current value, the current percentile range of variability (\%RV) for that proportion, and the departure classification. I show here a subset of metrics most useful for understanding patch characteristics in the study area; complete results are included in Appendix~\ref{app:full-land-results}.  Appendix~\ref{app:metricdescriptions} contains a detailed description of each \textsc{Fragstats} metric calculated for this project. At the landscape-level, most computed metrics have values outside the HRV. 

In Figures~\ref{fig:fragland1} and \ref{fig:fragland2} I graphically display the results from Table \ref{tab:fragland}. For these six metrics, the current landscape is fully departed from the historical range of variability. The average patch size is larger, and the average patch shape more complex, than the current landscape. Patches during the HRV had more edge, and on average, contained more core area than the current landscape. The landscape during the HRV is much more contagious than the current landscape. Values for Simpson's Evenness are near 1 during the HRV and in the present landscape, but the HRV values are well below the current conditions.

% plots updated 2015-09

\begin{figure}[!htbp]
  \centering
  \subfloat[][]{
    %\centering
    \includegraphics[width=0.5\textwidth]{/Users/mmallek/Documents/Thesis/Plots/fragland-hrv/ED1.png}
    \label{fig:fragland_ed}
  }%
  \subfloat[][]{
    %\centering
    \includegraphics[width=0.5\textwidth]{/Users/mmallek/Documents/Thesis/Plots/fragland-hrv/AREA_AM1.png}
    \label{fig:fragland_area}
  } \\
  \subfloat[][]{
    \includegraphics[width=0.5\textwidth]{/Users/mmallek/Documents/Thesis/Plots/fragland-hrv/SHAPE_AM1.png}
    \label{fig:fragland_shape}
  } 
  \subfloat[][]{
    \includegraphics[width=0.5\textwidth]{/Users/mmallek/Documents/Thesis/Plots/fragland-hrv/CORE_AM1.png}
    \label{fig:fragland_core}
    }
\caption{Landscape \textsc{Fragstats} Metrics. (a) Edge Density, a measure of patch perimeter complexity, (b) Area-weighted Mean Patch Area, a measure of patch size (c) Area-weighted Mean Shape, a measure of patch shape complexity (d) Area-weighted Mean Core Area, a measure of interior habitat available at the patch level. The red line indicates the metric value on the current landscape, the dotted lines indicate the 5th and 95th percentiles of the simulated data, the dashed line indicates the 50th percentile of the simulated data, and the blue line indicates the value for that metric at each timestep of the simulation.}
\label{fig:fragland1}
\end{figure}

\begin{figure}[!htbp]
  \centering
  \subfloat[][]{
    \includegraphics[width=0.5\textwidth]{/Users/mmallek/Documents/Thesis/Plots/fragland-hrv/CONTAG1.png}
    \label{fig:fragland_contag}
  } 
  \subfloat[][]{
    \includegraphics[width=0.5\textwidth]{/Users/mmallek/Documents/Thesis/Plots/fragland-hrv/SIEI1.png}
    \label{fig:fragland_siei}
  } 
\caption{Landscape \textsc{Fragstats} Metrics. (a) Contagion, a measure of patch dispersion and interspersion (b) Simpson's Evenness Index, a measure of diversity, or evenness, across all landscape patches. The red line indicates the metric value on the current landscape, the dotted lines indicate the 5th and 95th percentiles of the simulated data, the dashed line indicates the 50th percentile of the simulated data, and the blue line indicates the value for that metric at each timestep of the simulation.}
\label{fig:fragland2}
\end{figure}

%\clearpage

% repaired table 9/13
%\begin{landscape}
\begin{table}[!htbp]
\footnotesize
\centering
\caption{Range of variability during the simulation for selected landscape configuration metrics. See Appendix~\ref{app:metricdescriptions} for descriptions. Abbreviations are: 
\textsc{ed} = edge density;
\textsc{area\_am} = area-weighted mean patch size; 
\textsc{shape\_am} = area-weighted mean patch shape index; 
\textsc{core\_am} = area-weighted mean patch core area; 
\textsc{contag} = contagion; 
\textsc{siei} = Simpson's evenness index.
Included are the $5^{\text{th}}$ percentile, $25^{\text{th}}$ percentile, $50^{\text{th}}$ percentile, $75^{\text{th}}$ percentile, and $95^{\text{th}}$ percentiles of the distribution, as well as the current landscape proportion, the current percentile range of variability (\%RV) for that proportion, and the departure classification.
} 
\label{tab:fragland}
\begin{tabular}{@{}lrrrrr|rrr@{}}
\toprule
\textbf{\begin{tabular}[c]{@{}l@{}}Landscape\\ Metric\end{tabular}}  &   
\textbf{$5^{\text{th}}$ } &   
\textbf{$25^{\text{th}}$ } &   
\textbf{$50^{\text{th}}$ } &   
\textbf{$75^{\text{th}}$ } &   
\textbf{$95^{\text{th}}$ }  &  
\textbf{\begin{tabular}[c]{@{}l@{}}Current\\ Value\end{tabular}} &   
\textbf{\begin{tabular}[c]{@{}l@{}}Current\\ \%RV\end{tabular}} &   
\textbf{\begin{tabular}[c]{@{}l@{}}Departure\end{tabular}} \\ 
\midrule
\textsc{ed}         & 120.581         & 121.880           & 122.903          & 123.691          & 124.813          & 128.875     & 100     & complete  \\
\textsc{area\_am}   & 156.549         & 166.016          & 174.884          & 184.448          & 205.209          & 119.985     & 0       & complete \\
\textsc{shape\_am}  & 3.560            & 3.621            & 3.667            & 3.727            & 3.847            & 3.243       & 0       & complete \\
\textsc{core\_am}   & 135.146         & 141.964          & 149.582          & 157.587          & 169.545          & 106.710      & 0       & complete \\
\textsc{contag}     & 53.943          & 54.455           & 54.744           & 55.064           & 55.523           & 51.172      & 0       & complete \\
\textsc{siei}       & 0.946           & 0.949            & 0.951            & 0.953            & 0.956            & 0.971       & 100     & complete  \\
\bottomrule
\end{tabular}
\end{table}
%\end{landscape}




\clearpage

%%%%%%%%%%%%%%%%%%%%%%%%%%%%%%%%%%%%%%%%%%%%%%%%%%%%%%%%%%%%%%%%%%%%%%%%%%%%%%%%%%%%%%%%%%%%%%%%

\paragraph*{Class-level Results}

In addition to the landscape-level results, I also summarized structure and patterns at the cover type level. Figures~\ref{fig:fragclass-smcm} and \ref{fig:fragclass-smcx} show a subset of metrics most useful to understanding patch characteristics at the cover type-seral stage level for the two most prevalent cover types, Sierran Mixed Conifer - Mesic and Sierran Mixed Conifer - Xeric. Boxplots depict the range of variability for the simulation period as well as the current value. See Appendix~\ref{app:full-class-results} for full tabular results for the nine focal cover types.


\subparagraph*{Sierran Mixed Conifer - Mesic} %updated analysis 2015-09-20
The spatial configuration of stand conditions fluctuated markedly over time, although there was considerable variation in the magnitude of variability among configuration metrics (see Appendix~\ref{app:full-class-results}, Table~\ref{tab:fragclass_smcm}). Early and Mid Development patches in this cover type tended to have wide ranges of variability in metric outcomes, and were larger, less fragmented, more geometrically complex, and had more core area during the HRV than during the current conditions (Figure~\ref{fig:fragclass_smcm}). Metric values for these seral stages tended to be completely or nearly outside the simulated HRV. 
% B liked my writing here
In contrast, the other seral stages all fall within the simulated HRV in terms of patch size and core area. Results for geometric complexity and fragmentation were less consistent across the other seral stages. While Late--Open stands were more geometrically complex during the HRV than on the current landscape, Mid--Moderate, Mid--Late, and Late--Moderate patches were all less geometrically complex. Late--Closed patches currently fall within the simulated HRV. Meanwhile, the open canopy seral stages are currently within the HRV in terms of fragmentation, while the Mid--Moderate, Late--Closed, and Late--Moderate stages are all currently completely departed the range of variability and more fragmented today than during the simulated HRV.  

\begin{figure}[!htbp]
  \centering
  \subfloat[][]{
    %\centering
    \includegraphics[width=0.8\textwidth]{/Users/mmallek/Documents/Thesis/Plots/fragclass-bymetrics/HRV/SMC_M-AREA_AM-boxplots.png}
    \label{fig:smcm_areaam}
  } \\
  \subfloat[][]{
    %\centering
    \includegraphics[width=0.8\textwidth]{/Users/mmallek/Documents/Thesis/Plots/fragclass-bymetrics/HRV/SMC_M-CORE_AM-boxplots.png}
    \label{fig:smcm_coream}
  } \\
  \subfloat[][]{
    \includegraphics[width=0.8\textwidth]{/Users/mmallek/Documents/Thesis/Plots/fragclass-bymetrics/HRV/SMC_M-SHAPE_AM-boxplots.png}
    \label{fig:smcm_shapeam}
  } \\
  \subfloat[][]{
    %\centering
    \includegraphics[width=0.8\textwidth]{/Users/mmallek/Documents/Thesis/Plots/fragclass-bymetrics/HRV/SMC_M-CLUMPY-boxplots.png}
    \label{fig:smcm_clumpy}
  }%
\caption{Fragstats class-level results for Sierran Mixed Conifer - Mesic. (a) area-weighted mean patch area (AREA\_AM) (b) area-weighted mean core area (CORE\_AM) (c) area-weighted mean shape index (SHAPE\_AM) (d) clumpiness (CLUMPY). Boxplot whiskers extend from the $5^{\text{th}}-95^{\text{th}}$ percentile of the observed distribution. The thick grey bar denotes the metric value on the current landscape.}
\label{fig:fragclass_smcm}
\end{figure}


%\clearpage
%%%%%%%%%%%%%%%%%%%%%%%%%%%%%%%%%%%%%%%%%%%%%%%%%%%%%%%%%%%%%%%%%%%%%%%%%%%%%%%%%%%%%%%%%%%%%%%%


\subparagraph*{Sierran Mixed Conifer - Xeric}
The spatial configuration of stand conditions fluctuated markedly over time as well, although there was considerable variation in the magnitude of variability among configuration metrics (see Appendix~\ref{app:full-class-results}, Table~\ref{tab:fragclass_smcx}). Early Development and Mid--Open had wide ranges of variability in patch and core area size, while Mid--Closed had a wide range of variability in geometric complexity and fragmentation. In contrast to the mesic mixed conifer forests, results in this cover type were consistent across different metrics. Mid--Closed, Mid--Moderate, and Late--Moderate stages currently fall within the simulated HRV in terms of area-weighted mean patch size and core area, as well as for the shape and clumpiness indices. However, the other stages were generally currently completely departed from the simulated HRV or moderately departed within the simulated HRV. Early successional and open canopy stands are currently smaller, more fragmented, less geometrically complex, and have less core area than during the simulated HRV, while the opposite is true for Late--Closed patches (Figure~\ref{fig:fragclass_smcx}).

% figures updated 2015-09-20
\begin{figure}[!htbp]
  \centering
  \subfloat[][]{
    %\centering
    \includegraphics[width=0.7\textwidth]{/Users/mmallek/Documents/Thesis/Plots/fragclass-bymetrics/HRV/SMC_X-AREA_AM-boxplots.png}
    \label{fig:smcx_areaam}
  } \\
  \subfloat[][]{
    %\centering
    \includegraphics[width=0.7\textwidth]{/Users/mmallek/Documents/Thesis/Plots/fragclass-bymetrics/HRV/SMC_X-CORE_AM-boxplots.png}
    \label{fig:smcx_coream}
  } \\
  \subfloat[][]{
    \includegraphics[width=0.7\textwidth]{/Users/mmallek/Documents/Thesis/Plots/fragclass-bymetrics/HRV/SMC_X-SHAPE_AM-boxplots.png}
    \label{fig:smcx_shapeam}
  } \\
  \subfloat[][]{
    %\centering
    \includegraphics[width=0.7\textwidth]{/Users/mmallek/Documents/Thesis/Plots/fragclass-bymetrics/HRV/SMC_X-CLUMPY-boxplots.png}
    \label{fig:smcx_clumpy}
  }%
\caption{Fragstats class-level results for Sierran Mixed Conifer - Xeric. (a) area-weighted mean patch area (AREA\_AM) (b) area-weighted mean core area (CORE\_AM) (c) area-weighted mean shape index (SHAPE\_AM) (d) clumpiness (CLUMPY). Boxplot whiskers extend from the $5^{\text{th}}-95^{\text{th}}$ percentile of the observed distribution. The thick grey bar denotes the metric value on the current landscape.}
\label{fig:fragclass_smcx}
\end{figure}

\clearpage
%
% !TEX root = master.tex

\section{Discussion and Management Implications}
\label{sec:hrvdiscussion}




%%%%%%%%%%%%%%%%%%%%%%%%%%%%%%%%%%%%%%%%%%%%%%%%%%%%%%%%%%%%%%%%%%%%%%%%%%%%%
%%%%%%%%%%%%%%%%%%%%%%%%%%%%%%%%%%%%%%%%%%%%%%%%%%%%%%%%%%%%%%%%%%%%%%%%%%%%%


\subsection{Overall Landscape Assessment}

Fires during the simulated historical period burned far more frequently and across larger extents than at any time since record keeping began (according to available fire history data \citep{calfire2012,usgs-fire-data2012}) within the Yuba River watershed study area (see Tables~\ref{tab:darea_atleast} and \ref{tab:darea}, and Figure~\ref{fig:distid_median} and \ref{fig:distid_mean}). Most of the simulated individual fires were small and predominantly low mortality. However, large fires did occur, which indirectly affected patch configuration and directly affected the seral stage composition of cover types. Five-year periods (timesteps) in which over half the study area burned were very rare, indicating that occasional extremely widespread fire is a characteristic of the fire regime in this study area. Fire size was weakly but positively associated with the climate parameter (bigger fires when drought conditions were more severe). However, it is also influenced by vegetation susceptibility and the specified distribution of disturbance size. For this reason, large areas burned in relatively ``wet'' timesteps. 

While the resulting wildfire regime may seem dramatic, it is important to recall that I simulated the historical period, prior to modern fire management. Thus, in the model, fires are regulated only by local vegetation, topography, barriers, and weather (as represented by stochastically generated potential fire sizes). Although this can and does result in some very large fires, extending across tens of thousands of hectares, more typically I observed fairly small fires. This agrees with empirical data showing that when fire is frequent, it tends to be low severity and is often small in extent \citep{Kilgore1979,Taylor2012}. The large fire sizes observed reflect the fact that the simulation covers a very large spatial extent, and thus results in many more acres of fire burning per timestep than typically burn today \citep{calfire2012,usgs-fire-data2012}.


Regardless, because I had such a high degree of control over the fire regime, the particular disturbance regime results primarily confirm that the model was functioning appropriately and as designed. The bulk of the landscape assessment focuses on the composition and configuration results. 
% come up with better word for "droughtier"

% rewrote 2015-09-20
With respect to the seral stage distribution, a few patterns emerged (Full results for the nine focal cover types can be found in Appendix~\ref{sec:indiv_cov_results}.) Overall, the current study area composition departs from the HRV not only at the landscape scale, but also at the cover type and seral stage level. Complete departure from the simulated HRV is the norm for each seral stage of the xeric mixed conifer type. Mesic mixed conifer forests are less departed overall. Three of the seral stages (Mid--Open, Late--Closed, and Late--Moderate) have distributions that indicate that the current landscape is moderately departed from the HRV, and the current proportion of Early Development mesic mixed conifer forest is close to the median value observed over the course of the simulation. However, the current proportion of the landscape covered by the three remaining seral stages falls completely outside the simulated historical distributions.

%rewrote again to remove non-smc types on 2016-01-14
In both the mesic and xeric mixed conifer types, as well as the other cover types studied closely, Late--Open conditions were far more common during the HRV than on the current landscape. Interestingly, early seral conditions were more common during the simulated historical period than on the current landscape for xeric mixed conifer forests. This increase in the proportion of the study area belonging to early development and open canopy forests is directly connected to the frequent and extensive pattern of wildfires burning in the HRV scenario (Figure~\ref{fig:darea_smcx}, Tables~\ref{tab:darea_smcx} and \ref{tab:darea_atleast_smcx}). Mesic mixed conifer forests experienced somewhat less fire than the xeric type, but wildfires were still extensive (Figure~\ref{fig:darea_smcm} and Tables~\ref{tab:darea_smcm} and \ref{tab:darea_atleast_smcm}). The large amount of Mid--Closed in this cover type may be due to a combination of its relatively low susceptibility, relatively short early period (averaging 26 years), and the fact that Early Development is twice as likely to succeed to Mid--Closed as to Mid--Open (Appendix~\ref{smc-description}).


%rewrote this 2015-09-20
%Interestingly, early seral conditions were less common during the simulation than on the current landscape for the mixed evergreen and ultramafic mixed forests, but more common for the xeric mixed conifer forests; for the other types, however, the proportion of early seral is within the HRV. Both mixed evergreen types and both oak-conifer types have a smaller proportion of forest in late development now than during the simulation. Late--Open conditions were far more common during the HRV. For all cover types except for mesic mixed evergreen forests, Late--Open conditions were far more common during the HRV, and the current landscape is fully departed from the HRV. The Mid--Open was more common for xeric and ultramafic mixed conifer forests during the simulation than today, while Mid--Closed was dominant during the HRV for mesic mixed conifer forests, but now is not. Finally, closed canopy conditions were much more dominant in mesic red fir forests than on the present landscape; these stages are also completely outside the HRV.

%-- Closed canopy ocfw dominant now, but not during HRV.  (didn't use)
%-- Closed conditions were much more dominant in mesic red fir forests.
%-- Mid open more common for smcx and smcm
%-- Mid closed dominant during hrv for Smcm, but not now
%-- Mixed evergreen forests and oak-conifer forests and woodlands generally have less area in late development now than during the simulated HRV.
%-- Early seral was much more prevalent in SMCX and somewhat more prevalent in RFRX, but consistent with the other cover types.
%-- Early more common for smcx
%-- Late open more common for all cover types except rfrm, always outside of hrv
%-- Mixed evergreen and xeric mixed conifer had no seral stages within the HRV.
%-- 12 of 63 cover type - seral stage combinations with the HRV
%-- Only OCFWU-Early, RFRX-LDC, SMCM-Early, SMCU-MDM, were within interquartile range.
%-- Being completely outside the simulated HRV is the norm for seral stages with the focal nine areas. In many cases, the current proportion of a given seral stage is within the HRV, but this generally means either between the 5th and 25th percentiles, or between the 75th and 95th percentiles. In just four cases (\textsc{ocfw\_u} early, \textsc{rfr\_x} late closed, \textsc{smc\_m} early, and \textsc{smc\_u} mid moderate) was the current proportion within the 25th-75th interquartile range.


An analysis of \textsc{Fragstats} metrics at the landscape scale yields insight into the historical period. First, I note that compared to the present conditions, during the HRV the study area was composed of larger and more extensive patches, as illustrated by Figure~\ref{fig:fragland1}. This trend was heavily influenced by the presence of wildfires on the landscape, as high mortality fire in particular created large areas of Early Development vegetation (Figure~\ref{fig:patchmaps1-early}). However, I also observed large patches in the other seral stages, which were more likely to form long or convoluted patches that were nonetheless extensive (Figure~\ref{fig:patchmaps1-mid}).

% figure redone 2015-09-20
\begin{figure}[!htbp]
  \centering
  \subfloat[][]{
    \centering
    \includegraphics[width=0.45\textwidth]{/Users/mmallek/Documents/Thesis/maps/hrv-largepatch2410.pdf}
    \label{fig:patchmaps1-early}
    }%
  \subfloat[][]{
    \includegraphics[width=0.45\textwidth]{/Users/mmallek/Documents/Thesis/maps/hrv-largepatch2430.pdf}
    \label{fig:patchmaps1-mid}
    }
  \caption{(a) A large patch, highlighted with dark grey, of Sierran Mixed Conifer - Mesic in Early Development. The patch is 5,780 hectares, one of the second-largest patch during this timestep. (b) Two patches of Sierran Mixed Conifer - Mesic in Late--Closed. These are two of the largest patches during that timestep, at 10,930 hectares and 4,750 hectares.} 
  \label{fig:patchmaps1}
\end{figure}


% big 2622 12,600

In addition, I observed increased dominance by certain cover type-seral stage types during the HRV, which is likely what led to smaller values for the Simpson's Evenness Index (Figure~\ref{fig:fragland_siei}). For example, within the Sierran Mixed Conifer - Xeric cover type, Early Development and Mid--Open were much more widespread during the simulated HRV than in the current landscape (Figure~\ref{fig:patchmaps2}). Because Sierran Mixed Conifer - Xeric is so widespread, this shift would directly influence Simpson's Evenness, lowering its value.

% figure redone 2015-09-20
\begin{figure}[!htbp]
  \centering
  \subfloat[][]{
    \centering
    \includegraphics[width=0.45\textwidth]{/Users/mmallek/Documents/Thesis/maps/hrv-dominance-ts0.pdf}
    \label{fig:patchmaps2-ts0}
    }%
  \subfloat[][]{
    \includegraphics[width=0.45\textwidth]{/Users/mmallek/Documents/Thesis/maps/hrv-dominance-26.pdf}
    \label{fig:patchmaps2-ts570}
    }
  \caption{Cover type-Seral stage map focused on patches from Sierran Mixed Conifer - Xeric, showing increased dominance by certain cover type-seral stage types during the HRV. (a) The current landscape. (b) The same region of the map during a randomly selected timestep after the equilibration period. Note the contrast between the two maps with respect to the seral stages and size of individual patches.} 
  \label{fig:patchmaps2}
\end{figure}


Third, I find that patches on the landscape were more aggregated at the cell-level during HRV, which is illustrated by the Contagion metric (Figure~\ref{fig:fragland_contag}). In general, patches have low levels of both dispersion and interspersion. Of course, there are many ``edgy'' areas on the landscape, but this metric indicates that across the full landscape aggregation is more typical, particularly in comparison to the current landscape. Again, the homogeneity of post-fire early successional stands likely aids in increasing the contagion value (Figure~\ref{fig:patchmaps3}). 

% figure redone 2015-09-20
\begin{figure}[!htbp]
  \centering
  \subfloat[][]{
    \centering
    \includegraphics[width=0.45\textwidth]{/Users/mmallek/Documents/Thesis/maps/hrv-agg-ts0.pdf}
    \label{fig:patchmaps3-ts0}
    }%
  \subfloat[][]{
    \includegraphics[width=0.45\textwidth]{/Users/mmallek/Documents/Thesis/maps/hrv-agg-24.pdf}
    \label{fig:patchmaps3-ts615}
    }
  \caption{Cover type-Seral stage map focused on patches from Sierran Mixed Conifer - Mesic. (a) The current landscape. (b) The same region of the map during a randomly selected timestep after the equilibration period. Note the contrast between the two maps with respect to the contagion (at the cell level).} 
  \label{fig:patchmaps3} %year 1075 is timestep 215; year 675 is timestep 135
\end{figure}

Despite having a higher edge-to-area ratio, and being more geometrically complex, patches during the simulated HRV still show an increase in core area over the present landscape (Figure~\ref{fig:fragland_core}). This indicates that the large patches that contain core area are sufficiently large to surpass the relatively high amount of core area on the present landscape. I expected that the current landscape might have large amounts of core area because human-designed management unit boundaries are likely to create simple shapes, especially when they incorporate linear features such as raods or streams. Since this is an area-weighted measure, I conclude that it is the presence of many large patches that are large enough to contain significant core area that resulted in a simulated historical range of variability that does not overlap the current landscape. The results for the shape complexity metric confirm this analysis (Figure~\ref{fig:fragland_shape}). Especially among the largest patches on the landscape, convoluted shapes are common. Again, this is not to say that large and simple shapes do not occur---they do---but in comparison to the current landscape, complex shapes were characteristic of the simulated HRV (Figure~\ref{fig:patchmaps4}).

% figure redone 2015-09-20
\begin{figure}[!htbp]
  \centering
  \subfloat[][]{
    \centering
    \includegraphics[width=0.45\textwidth]{/Users/mmallek/Documents/Thesis/maps/hrv-complexpatch-2622.pdf}
    \label{fig:patchmaps4-ts555}
    }%
  \subfloat[][]{
    \includegraphics[width=0.45\textwidth]{/Users/mmallek/Documents/Thesis/maps/hrv-simplepatch2422.pdf}
    \label{fig:patchmaps4-ts690}
    }
  \caption{Cover type-Seral stage map focused on patches from Sierran Mixed Conifer. (a) A large patch of Sierran Mixed Conifer - Xeric in the mid development open seral stage illustrates how very large patches (this one is 12,600 hectares) may have relatively small amounts of core for their shape, yet accumulate a lot of core because of their overall size. (b) Simpler shapes do exist, such as this patch of Sierran Mixed Conifer - Mesic in the mid development open seral stage, which has a lot of core area. However, they are often much smaller (this one is 2,320 hectares).}
  \label{fig:patchmaps4}
\end{figure}

Examining some of the class-level results for similar \textsc{Fragstats} metrics as for the landscape as a whole, I observed many consistencies and a few interesting diversions (Figures~\ref{fig:smcm_areaam}--\ref{fig:smcx_clumpy}). In general, for the area-weighted metrics, if higher proportions of the landscape are occupied by a given seral stage, that cover type-seral stage combination is likely to have high values for a given metric. Again, results for area-weighted mean patch area and mean core area are consistent. Early Development patches are typically characterized by more complex shapes that are less aggregated during the HRV, as compared to the current landscape. This is due to the fact that during the simulation early successional patches are created by fires that are allowed to burn naturally, rather than by vegetation treatments with predetermined, linear boundaries. In addition, seral stages that became rare during the HRV tend to be smaller, aggregated and to have simple shapes. Larger shapes have a greater potential to have complicated shapes and to be disaggregated. This result may reflect the nature of the metrics at the class level as much as the seral stage structure. Overall, the class-level metrics reflect the interplay between wildfire and succession, and can best be used when considering management alternatives at the class level. However, the landscape-level interpretation is the most suitable for evaluating the study area and the potential impact of management actions, including restoration. 

\clearpage



\clearpage
\subsection{Management Implications}\todo{need to outline this section, what is it saying? a bit jumbled now}

% becky said delete the intro sentence, move limitaitons elsewhere
%The results of this study include a detailed quantitative and qualitative description of the historic reference period. Using these results I can next assess the extent to which human actions have led to ecological changes, such that landscapes and their functions are outside of their historic range of variability \citep{Landres1999,Swetnam1999}. Considering the alignment or gap between past and current fire frequencies is a potential basis for developing priorizations of forest management activities, including ecological restoration, fuels reduction, and habitat management \citep{Fule2008,Safford2014}. 

%"Drawing comparisons between past and current fire frequencies can assist resource managers in prioritizing areas for ecological restoration, fuels reduction, certain fire or habitat management practices, and other activities. " (Hugh)


In this study I leveraged current scientific knowledge and understanding of vegetation dynamics and disturbance processes to simulate changes to landscape composition and configuration over time under a historical reference framework. I then compared those dynamics to observations of current conditions, and assessed the departure from the historic range of variability. My landscape-level conclusions are that both composition and configuration deviate substantially from the HRV. In general, the current landscape is dominated by Mid and Late Development forest and lacks the fire-dependent open canopy stand conditions (Mid-Open, Late-Open) and spatial heterogeneity in vegetation that were maintained by natural disturbances during the reference period. The departure from the HRV is probably due in large part to land management practices, especially fire suppression and timber management, over the last 150 years \citep{Safford2014,Stephens2007}. 

Substantial changes to the ecology and landscape function in western forest landscapes have been documented not only in the Sierra Nevada, but also in the Cascades and Rocky Mountains \citep{Hessburg2005,Baker2012,Baker2014,Mallek2013,Agee1993}. In one recent study, \citet{Collins2011} resurveyed a 1911 timber inventory and compared conditions from 2005--2007 to those of 2011. The resurveyed plots were characterized by higher canopy cover and tree densities than in the 1911 survey. They also inferred that historically, fires burned at varying densities, but that large high severity patches were extremely rare \citep{Collins2011}. \citet{Baker2014} conducted an extensive review of the 1865--1885 General Land Office surveys, using the quantitative and qualitative data found in them to try and reconstruct historical forest structure and fire in Sierran mixed conifer forests. \citet{Baker2014} concludes that evidence of high severity fire was present across 31--39\% of the Sierra, a point disputed by \citet{Fule2014} based most strongly on the fact that the \citet{Baker2014} study assumed that the presence of dense stands of small trees were evidence of high severity fire. My study is unique in that it incorporates spatially explicit disturbance and succession modeling in combination with analyses of landscape structure; the studies above did not include any simulations. In addition, my study focused on the Tahoe National Forest specifically, an area often not included in other studies and with a management history that makes field-based studies difficult to use for making inferences about historical systems.

I suggest that my study may serve as a bridge between the \citet{Baker2014} and \citet{Collins2011} work and somewhat resolve the ongoing dispute over historical high severity fire in the region \citep{Baker2014,Fule2014}. To return to the \citet{Fule2014} article, the authors raise several issues with the \citet{Baker2014} methodology that they assert undermine conclusions drawn from them. First, \citet{Fule2014} claim that the Baker papers \citep{Baker2012,Baker2014} conclude that ``present-day large, high-severity fires are not distinguishable from historical patterns.'' It is important to distinguish patterns from aggregations. The claim in \citet{Baker2012} is that the proportion of high severity fire has not changed, not that the pattern of it is the same. \citet{Fule2014} also take issue with the use of tree size to infer stand age, and with it evidence of past disturbance. They support this claim with references to several studies of various forest types for which this is true. However, associating stand age with tree size is a widely used hueristic \citep{WHR1988,Landfire2007,USDAForestService2009}, also used in my own research, and in the absence of dendrochronology, I am unaware of other methods of inferring age from timber surveys. 

Critiques from \citet{Fule2014} about fire severity and comparing Monitoring Trends in Burn Severity (MTBS, \burl{mtbs.gov}), are also critiques of all historical studies. They in effect are saying that a comparison of recent fire severity to historical fire severity cannot be done using the MTBS data, since no historical study could hope to conduct a severity assessment that would be analagous to MTBS. \citet{Fule2014} also complain that a 70\% overstory mortality threshold for designating high severity fire is inappropriate, yet this is the definition used in the seminal work by \citet{Agee1993} in his fire ecology study, as well as that employed when creating the Vegetation Dynamics models used in \citet{Landfire2007}. While this critique may be valid, the Baker papers use of this threshold is not unique in the field, and should be further examined before being dismissed.

I also wish to comment on the fact that the \citet{Fule2014} does not mention the incorporation of chaparral as an indicator of high severity fire. The Baker methodology uses chaparral existence as evidence of past fire, which I also use. Incorporating chaparral into a cover type map as an early stage of forest by definition increases the area in an early successional state, as well as the area assumed to have been affected by high severity fire. Reburning of chaparral also usually results in high severity (high mortality in my model, see Appendix~\ref{app:covertypedesc}) fire. My results are somewhat similar to those in \citet{Baker2014}, perhaps because of the fact that we treated chaparral similarly. This is an area for potential future research. In fact, my high mortality fire rotations (113 years and 72 years for mesic and xeric mixed conifer forests, respectively) are much, much shorter than the 281 year rotation calculated by \citet{Baker2014} for mixed conifer forests in the northern Sierra Nevada. Before making inferences about the implications of the fire rotations found by myself or \citet{Baker2014}, other researchers should consider viewing the seral stage distribution dynamics, and potentially the output maps from simulations like mine. Based on my results, I would predict that fire rotations even longer than the 281 years proposed by Baker would yield a mixed conifer forest almost entirely composed of old-growth forest, which conflicts with the understanding of Sierran forests as being extremely patchy \citep{Franklin1996,SNEP1996}  

Management actions on the Tahoe National Forest that directly altered the landscape can explain the observed departure from the HRV of the current landscape. First, fire suppression facilitated the closure of forest canopies, leading to more closed and less open canopy cover across forests of varying species composition \citep{Beaty2007}. Second, it indirectly reduced the amount of the Early Development seral stage, particularly in the xeric mixed conifer forest (as evidenced by dominance of Late Development forest in the 2010 cover types map \citep{USDAForestService2009}). Third, I infer that it contributed to a reduction in patch complexity over time because the irregular borders of a natural fire do not become irregular borders of cover type-seral stage patches. Timber management could also have promoted departure from the HRV. Boundaries of timber sales tend to be linear, rather than irregular, affecting patch complexity \citep{USDAForestService2012}. Fine-scale heterogeneity would also be reduced in forests managed primarily to increase the proportional representation of the most valuable timber species, with structures designed to be easily accessible and harvested, rather than managed for complex structures that would provide habitat for a range of plants and animals \citep{Franklin2002}. While substantial, thinning and harvest operations affected only a small fraction of the amount of forest that probably burned annually during the historical period, so the likelihood that timber management activities compensated for the loss of stand opening and stand initiation effects from fire is low \citep{USDAForestService2012,Cushman2011}. Active replanting after cutting or after fire would also reduce the average area in the Early Development stage at any given moment because it would reduce the time spent in the Early Development stage, and because planting and managing trees would accelerate the time to Mid Development \citep{Dellasala2014}. In addition to these impacts, the Tahoe National Forest has been managed to promote the development of late successional, old growth forest habitat, in order to provide habitat for species dependent on it, like the spotted owl \citep{USDAForestService2004a}. This also pushes the landscape into an older, more closed condition, since both fire suppression and a lack of timber harvest affect the character of these focal old growth areas \citep{Franklin1996}.

%Based on our results, it might be tempting for managers to reach the simple conclusion that the landscape is less fragmented today than during the reference period. For example, today's landscape is simpler (lower \textsc{shape} values), contains less core area (based on \textsc{core\_am}), and has less contrast between patch types (\textsc{econ\_am}) (Figures~\ref{fig:fragland} and \ref{fig:patchmaps1}--\ref{fig:patchmaps4}). However, this conclusion is not as straightforward as it might seem. Fragmentation is a landscape-level process in which a specific habitat is progressively sub-divided into smaller, geometrically altered, and more isolated fragments as a result of both natural and human activities. This process involves changes in landscape composition, structure, and function at many scales and occurs on a backdrop of a natural patch mosaic created by vegetation transitions, both those mediated by and independent from natural disturbances \citep{McGarigal1995}. The scale at which fragmentation occurs is at the level of a specific habitat type; it is the habitat, rather than the landscape, that becomes fragmented. In this study we evaluated the spatial pattern---and by implication, the fragmentation---of many different patch types (defined by unique combinations of cover type and condition class). Certaintly, some of these patch types are less fragmented in the current landscape than they were under the simulated HRV\todo{B: What is train of thought? Like overview of configuration and why not simple to implement at management level}. 



%"Drawing comparisons between past and current fire frequencies can assist resource managers in prioritizing areas for ecological restoration, fuels reduction, certain fire or habitat management practices, and other activities. " (Hugh)

My results imply that, even if climate change were discounted, using the simulated historical range of variability as a restoration target would be challenging and take many decades to implement. The model equilibration used in this study (see Chapter~\ref{sec:hrvmethods} for more details) was 200 years long. Even though the seral stage distribution of some cover types equilibrated in less than that time, given that forest planning horizons are on timescales of 5--30 years, additional analyses would need to be completed to determine reasonable benchmarks for landscape change at those shorter timescales \citep{Millar1999,Millar2014}. It is true that since I simulated natural disturbance processes, accelerated management could reduce the time it takes to restore the HRV characteristics to the study area. However, at the same time, there are existing social, economic, and political challenges that would simultaneously retard progress. The extent and intensity of disturbance required to emulate the natural disturbance regime is significant, and a simple restoration of the historical fire regimes would not be possible with public input and environmental consultation. For these reasons a more practical use of these results are to guide the prioritization of areas for restoration, design fire and fuels management projects, or describe desired future conditions \citep{Keeley2000,Fule2008,Safford2014}.

\subsection{Limitations}
Several limitations apply to my results. First, my results are based on an initial cover type map that does not include roads because roads were not part of the historical landscape. I excluded roads by overwriting the land cover type of ``road'' with the nearest alternative cover type. Figure \ref{fig:roadcovermap} shows the cover type layer with roads overlaid. However, extensive research shows that roads have significant impacts on Sierran ecosystems \citep{Karr2004,Trombulak2000,Gucinski2001,Theobald2011}. The primary impact of roads is that they break up patches (the unit of analysis when using \textsc{Fragstats} and are associated with high-contrast seral stages adjacent to one another. Some researchers have found that the impacts from roads in terms of patch characteristics may be greater than that due to management because roads occur so extensively across the study area \citep{Gucinski2001,Tinker1998,Mcgarigal2001}. Given the large amount of roads within the study area and their disproportionate influence on landscape structure and function, it is important to consider the likely impact of roads on configuration metrics. In many cases, at least at the local scale, having included roads would probably have led to results in which the difference between the current landscape and the HRV was exacerbated.
%
\begin{figure}[!htbp]
  \centering
  \includegraphics[height=.4\textwidth]{/Users/mmallek/Tahoe/Report2/images/roads_cover.png}
  \caption{The core study area superimposed with the cover type map and all roads in black. There are a two designated roadless areas, but in general roads are common throughout the watershed. The closeup area is just northeast of the Pendola reservoir.} 
  \label{fig:roadcovermap}
\end{figure}
%

In addition, results are constrained by certain limitations related to the nature of this study as one that simulates landscape-level dynamics. For example, the vegetation cover layer is subject to human interpretation errors and objective classification errors, and is further limited by the spatial resolution of the grid. Still, my results can be used to help identify the most influential factors driving landscape change, the implications of the simulated disturbance and succession regime, and areas where further research is needed to delineate key parameters. The estimate of the HRV described here could change if new scientific understanding or better data that would affect model parameterization becomes available.

An issue brought up by \citet{Fule2014} is that the \citep{Baker2014} does not incorporate other disturbances into his assumptions about what processes produced the observed forest structure. A similar claim could be brought against this study, which only explicitly incorporates wildfire. In the case of my study, this would mean that even more disturbance occurred on the historical landscape, since my fire regime parameters were based off empirical data that only incorporated specific fire evidence.


\subsection{Using results for management at various scales}\todo{should this be a subsubsection?}
It is important to understand that this study was not designed to address questions below the landscape level. Managers should carefully view subregions through the lens of the results through a comparative or relative lens. It would not be appropriate, for example, to use the landscape-level statistical results as the template for a project-level target. Instead, the landscape-level results should guide the development of project-level targets. This is a subtle but important distinction in how to use the outputs of my simulations. When developing projects, the results of this study can inform the larger scale and context in which the project will alter the lanscape, which may affect details of project design. In addition, the success of a project should be measured not by whether the results of the project mirror the results, but by whether the project contributed to a landscape-level shift set as a goal based on the landscape-level results of my study.

To be more specific, I outline a simple example. The Tahoe National Forest could draw a boundary over an area identified as being appropriate for fire restoration. It would not be appropriate to assume that because across the full study area, late development, open canopy xeric mixed conifer forests ranged from about 25\% to 35\% of the total xeric mixed conifer forest, that within the newly identified project area, similar proportions should prevail. Many other variables should be taken into account, including environmental (topographic position) and social (wildland-urban interface) considerations. My results rely on the generation and analysis of a large quantity of data. When the scale of analysis is reduced, so is the quantity of data produced by the model, and with it, my confidence in the statistical validity of the results and their implications. For similar reasons, I limited faith in model outputs to cover types of at least 1000 ha, and focused my analysis on the two dominant cover types, which extend across an area many times that minimum. Planning using these results will be more effective moving from larger to smaller scales. Taking the existing example, the Forest could set a management target for late development, open canopy xeric mixed conifer forests as the above range. If the current proportion on the forest is 15\%, an individual project could seek to maintain or create additional acre in that seral stage, but my results themselves would not tell managers how much late development, open canopy xeric mixed conifer forest should exist within that project boundary. An individual project area can have 0\% or 100\% late development, open canopy xeric mixed conifer forest. The key is for managers to place individual projects in a landscape context, and understand the contribution different places on the forest contribute to the overall landscape pattern (in this example, the overall seral stage distribution).

\textsc{RMLands} has been optimized to perform well statistically when used to examine landscape change over large extents and long time periods. While spatially explicit in its dynamics, it is a simulation, producing many potential outcomes, and is not intended to predict specific outcomes at specific places and points in time. Rather, the data produced are aggregated in order to develop an understanding of the area as a whole. Because this model should not be used to predict an actual fire rotation value for a particular point on the landscape, it should also not be used to predict the effect of a particular vegetation management strategy \emph{at a particular point on the landscape.} Other forest and fire simulation models are designed for this purpose. \textsc{RMLands} is designed to produce outputs that facilitate measuring the effect of vegetation management implemented across space and time, which in turn enables assessing and understanding the potential impacts of such actions at scales larger than is practical or possible to measure experimentally.  

Although not conducted for this study, there are additional ways to use the results of this study that maintain a landscape-level focus. For example, a future project could conduct a comparable analysis on another similar sized and ecologically analogous area. The results from the two landscape could then be compared, and the results used to prioritize work based on the relative degree of departure from the HRV of a particular landscape in its composition or configuration. Another option would be to use the average canopy cover map (Figure~\ref{fig:averagecc}) to make the spatial aspect of the range of variability explicit, since the boxplots as presented summarize both over space and time. A comparison of the average canopy structure to the current canopy structure could serve as a starting point from which to prioritize survey and monitoring work. The field work, aimed at ground-truthing the results, would then feed into an analysis to determine where to implement management designed to shift the overall landscape toward a particular state. Such a strategy could work as long as decision-makers do not try to impose the median condition for individual cells during the simulation on all cells within the study area. The spatial information should inform targets, not \emph{be} the target.

\subsection{Recommendations for Sierran Mixed Conifer - Mesic and - Xeric}
My study methodology makes it possible to single out separate cover types and seral stages for analysis. In this section I provide some interpretation of those results that is targeted toward providing recommendations for specific strategies that could be used to push landscape structure toward the HRV, and other recommendations of management strategies to avoid. Because in reality cover types are interspersed with one another, managers will need to consider the recommendations for individual cover types within the context of the vegetation actually present within a given management unit. In some cases, the motivation for a given vegetation treatment may focus on the cover type involved, as well as the seral stage. However, because cover types in the real world blend into one another (as opposed to being separated by stark lines such as those in cover type maps), treatment unit boundaries are likely to include different cover types near those boundaries. The treatment prescribed may have been designed mostly for one cover type, but will affect adjacent cover types. In these situations, managers should have the flexibility to consider what landscape-level objectives can be achieved through adjusting the treatment boundaries to add more of or exclude adjacent cover types.

The observed fire rotation for mesic mixed conifer forests during the simulated historical period was 27 years, while the observed fire rotation for xeric mixed conifer forests was 23 years. In both, low mortality fire was much more common than high mortality fire. However, fires having both outcomes are necessary to create stand structure and composition similar to that observed during the simulation. 

In mesic mixed conifer forests, all seven seral stages were well represented during the simulated historical period. The most common condition (highest median value) during the simulation was Mid--Closed. The proportion of mesic mixed conifer on the current landscape is close to the median values for Early, Mid--Open, and Late--Moderate. There is much more Mid--Moderate and Late--Closed, and much less Mid--Closed and Late--Open, in the 2010 landscape compared to the simulated HRV. Based on these observations, in the short term I recommend maintaining existing open stands of mesic mixed conifer though prescribed fire and other understory vegetation treatments. Because patch sizes are much smaller now on average, when identifying stands to push towards, for example, Mid--Closed or Late--Open, I recommend looking for treatment units near or adjacent to existing patches of the target seral stage, in order to create large patches in that type. Overall, the relative proportion closed canopy forest across development classes is near the median values for closed canopy seral stages during the simulated HRV. I do not believe more closed canopy forest is needed to bring the landscape toward HRV conditions. That said, it probably makes sense to work to create conditions for younger closed canopy forests, so that all the closed canopy stands on the forest do not belong to the oldest age class.

In contrast to the results for Sierran Mixed Conifer - Mesic, I observed a consolidation of the area in Sierran Mixed Conifer - Xeric into the Early, Mid--Open, and Late--Open seral stages. Mid--Closed, Mid--Moderate, and Late--Closed were present in very low amounts, while Late--Moderate was well established but not dominant. Due to these shifts, this cover type is completely departed from the HRV. Based on these observations, in the short term I recommend maintaining existing Early Development, Mid--Open, and Late--Open stands of xeric mixed conifer through prescribed fire and other understory vegetation treatments. Harvest treatments should focus on areas now in closed canopy conditions. As with the mesic mixed conifer forests, because patch sizes are much smaller now on average, when identifying stands to push towards and open canopy structure, I recommend looking for treatment units near or adjacent to existing open patches in order to create larger patches overall. Of course, it is also important for fuels specialists, foresters, and biologists to work together to include fine-scale structural complexity in these large patches of open canopy stands.
 
In the medium term, for both cover types I recommend the restoration of fire wherever practicable. Fire was quite common on the simulated historical landscape and was probably the main driver of the complexity of patches I see in the simulated landscape, which is in agreement with empirically and modeled estimates of historical forest structure \citep{Franklin2002,Nonaka2005,Mallek2013}. The point-specific fire rotation for low mortality fire ranged greatly for both mesic mixed conifer forests (which ranged from 18 years to about 100 years) and xeric mixed conifer forests (which ranged from 17 years to about 80 years). Managers can transfer this variability into flexibility when planning and executing vegetation treatments and wildfire response (Figure~\ref{fig:pfire_comp_HCRD}). In fact, the spatial variability of fire was instrumental in creating the spatial variability of forests and plants observed as outputs of the simulation. Managers who are charged with focusing fuels reduction on certain areas within the forest could set goals of carrying out vegetation treatments somewhat more frequently in these parts of the forest and less frequently elsewhere, such as in spotted owl activity centers, thereby also contributing to the overall more complex landscape pattern observed during the simulated HRV. Because fuels reductions along roads offers benefits that include enhancing the ability of the road to serve as a barrier to fire spread and increasing the safety moving through forested areas during wildfire incidents, among others, they may be priority restoration sites and early targets for promoting open canopy conditions (Figure~\ref{fig:pfire_comp_EDNF}). 

\begin{figure}[!htbp]
  \centering
  \subfloat[][]{
    \centering
    \includegraphics[width=0.5\textwidth]{/Users/mmallek/Documents/Thesis/Seminar/pfirebefore.jpg}
    }%
  \subfloat[][]{
    \includegraphics[width=0.5\textwidth]{/Users/mmallek/Documents/Thesis/Seminar/pfireafter.jpg}
    }
  \caption{Before and after a prescribed fire treatment on the Hat Creek Ranger District, Lassen National Forest. The outcome of this burn was primarily low mortality, although some under and middlestory trees were killed as a result of the fire. Images from the Sierra Nevada Avian Monitoring Information Network.} 
  \label{fig:pfire_comp_HCRD}
\end{figure}

\begin{figure}[!htbp]
  \centering
  \subfloat[][]{
    \centering
    \includegraphics[width=0.5\textwidth]{/Users/mmallek/Documents/Thesis/Seminar/EDNF_mech_after.jpg}
    }%
  \subfloat[][]{
    \includegraphics[width=0.5\textwidth]{/Users/mmallek/Documents/Thesis/Seminar/EDNF_mech_before.jpg}
    }
  \caption{Before and after a mechanized fuels treatment on the Eldorado National Forest. The timing of mechanized treatments is less restricted than prescribed fires, which can only take place under certain weather and fuels conditions. When mechanized treatments are available, fuels managers have more flexibility in selecting treatment options. A mechanized treatment like the one pictured here could be followed by a prescribed burn.} 
  \label{fig:pfire_comp_EDNF}
\end{figure}


With respect to the landscape structure metrics that characterize the mixed conifer forest type, results differed between the mesic and xeric variants. In mesic mixed conifer forests, I did not find consistent patterns in how or if the current landscape, at the seral stage level for this cover type, departed from the simulated historical results. This reflects the fact that the mesic mixed conifer type is inherently complex and not necessarily dominated by a particular average patch size or level of geometric complexity. I suggest that the finding of larger, less fragmented, and more geometrically complex patches of Early Development be used to guide the planning and execution of post-fire management of areas that burn at high severity. Promoting patches of Early Development, especially in conjunction with restoration practices that encourage specific patch configuration, would be one way to manage and guide forests toward the HRV.
%
Some of the seral stages for the xeric mixed conifer forest type were nearly absent during the simulated historical period. As a result, I cannot make generalizations about them and will instead focus on the most common stages: Early, Mid--Open, and Late--Open. Patches of these seral stages in xeric mixed conifer forests in 2010 were smaller, more fragmented, less geometrically complex, and contain less core area than during the simulated HRV. 
%
Restoration of these forests to patches that reflect a more natural succession process may be challenging for managers, given practical needs like using roads and riparian buffers as the edges of treatment units. It may not be practical to perform mechanical treatments over large areas within this cover type. However, when conducting treatments using prescription fires, creative solutions should be sought to generate more complex edges and to complete burns over sufficiently large areas that large core areas are a byproduct of the treatment.











