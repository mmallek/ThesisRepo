% !TEX root = master.tex
\newpage
\section{Oak-Conifer Forest and Woodland (OCFW)}
\label{ocfw-description}

\subsection*{General Information}

\subsubsection{Cover Type Overview}

\textbf{Oak-Conifer Forest and Woodland (OCFW)}
\newline
\textbf{Crosswalks}
\begin{itemize}
	\item East of the Sierra Crest
	\begin{itemize}
		\item Eveg: Regional Dominance Type 1
		\begin{itemize}
			\item Black Oak
			\item Eastside Pine
			\item Jeffrey Pine
			\item Ponderosa Pine
		\end{itemize}
		\emph{And}
		\item Eveg: Regional Dominance Type 2
		\begin{itemize}
			\item Black Oak
			\item Canyon Live Oak
			\item Madrone
			\item Montane Mixed Hardwood
			\item Scrub Oak
		\end{itemize}
	\end{itemize}

	\item West of the Sierra Crest
	\begin{itemize}
		\item Eveg: Regional Dominance Type 1
		\begin{itemize}
			\item Black Oak
			\item Eastside Pine
			\item Jeffrey Pine
			\item Ponderosa Pine
		\end{itemize}

		\item LandFire BpS Model
		\begin{itemize}
			\item 0610300 Mediterranean California Lower Montane Black Oak-Conifer Forest and Woodland
		\end{itemize}
		
		\item Presettlement Fire Regime Type
		\begin{itemize}
			\item Yellow Pine
		\end{itemize}
\end{itemize}
\end{itemize}

Modifiers
\begin{itemize}
	\item Ultramafic: This type is created by intersecting an ultramafic soils/geology layer with the existing vegetation layer. Where cells intersect with OCFW they are assigned to the ultramafic modifier.
\end{itemize}

\noindent Reviewed by Becky Estes, Central Sierra Province Ecologist, USDA Forest Service; Kyle Merriam, Sierra-Cascade Province Ecologist, USDA Forest Service


\subsubsection{Vegetation Description}
\textbf{Oak-Conifer Forest and Woodland (OCFW)} The Oak-Conifer Forest and Woodland landcover type is characterized by woodlands or forests of \emph{Pinus ponderosa} or \emph{Pinus jeffreyi} with one or more oaks, such as \emph{Quercus kelloggii}, \emph{Quercus garryana}, \emph{Quercus wislizeni}, or \emph{Quercus chrysolepis}. \emph{Pseudotsuga menziesii} and other conifer species are uncommon but may co-occur, especially after long-term fire suppression (LandFire 2007a). \emph{Pinus jeffreyi} tends to dominate on ultramafic sites (Fitzhugh 1988). In some areas, sites are dominated initially by oaks, which form a dense subcanopy. Eventually, and especially on locally mesic sites, conifers will form a persistent emergent canopy over the oak as a bi-layered canopy (LandFire 2007a). In other cases, characteristic species occur in a mosaic-like pattern with small pure stands of conifers interspersed with small stands of broad-leaved trees. Most of the broad-leaved trees are schlerophyllous evergreen, but winter-deciduous species also occur (Anderson 1988). The understory is composed of shrubs such as \emph{Arctostaphylos}, \emph{Ceanothus, Chamaebatia, Cornus, Eriodictyon, Garrya, Prunus, Rhamnus, Ribes,} and \emph{Toxicodendron diversilobum}. Grasses and forbs are diverse and include \emph{Bromus}, \emph{Melica}, \emph{Poa}, \emph{Elymus}, \emph{Carex}, \emph{Collinsia}, \emph{Saltugilia}, \emph{Iris}, \emph{Lupinus}, \emph{Streptanthus}, \emph{Viola}, and \emph{Pteridium aquilnum} (LandFire 2007a, Fitzhugh 1988).

\begin{adjustwidth}{2cm}{}

\textbf{Ultramafic Modifier (OCFW\_U)}  \emph{P. ponderosa} or \emph{P. jeffreyi} woodlands occur mainly on low-elevation ultramafics. They grow on strongly serpentinized soil, and are typically adjacent to the non-ultramafic form of the cover type. While \emph{P. ponderosa} or \emph{P. jeffreyi} dominates, it may be associated with \emph{Calocedrus decurrens, Pinus attentuata, Pinus lambertiana, P. sabiniana}, and \emph{Q. chysolepis} (O'Geen et al. 2007). \emph{Q. kelloggi} is rare on ultramafic soils (Fryer 2007). The shrub layer is dominated by \emph{Arctostaphylos, Ceanothus, Eriodictyon, Heteromeles}, and \emph{Pickeringia}. The herb layer is a mix of sparse perennials and many annual grasses and forbs (O'Geen et al. 2007). 

\end{adjustwidth}


\subsubsection{Distribution}
This type occurs in the valleys and lower slopes of mountainous terrain, on a variety of parent materials including granitics, metamorphic and Franciscan metasedimentary parent material and deep, well developed soils, although rocky soils are also possible. Slopes are generally steep and all aspects are included. In the northern Sierra Nevada the elevational range is 240 to 1800 m (800 to 5000 ft) (LandFire 2007a, Anderson 1988).

\begin{adjustwidth}{2cm}{}

\textbf{Ultramafic Modifier} Ultramafics have been mapped at various spatial densities throughout the elevational range of the OCFW landcover type. Low to moderate elevations in ultramafic and serpentinized areas often produce soils low in essential minerals like calcium potassium, and nitrogen, and have excessive accumulations of heavy metals such as nickel and chromium. These sites vary widely in the degree of serpentinization and effects on their overlying plant communities (``CalVeg Zone 1'' 2011). Note, the terms ``ultramafic rock'' and ``serpentine'' are broad terms used to describe a number of different but related rock types, including serpentinite, peridotite, dunite, pyroxenite, talc and soapstone, among others (O'Geen et al. 2007).


\end{adjustwidth}

\subsection*{Disturbances}

\subsubsection{Wildfire}
Wildfires are common and frequent; mortality depends on vegetation vulnerability and wildfire intensity. Low mortality fires kill small trees and consume above-ground portions of shrubs and herbs, but do not kill large trees or below-ground organs of most shrubs and herbs which promptly re-sprout. High mortality fires kill large as well as small trees, and may kill many of the shrubs and herbs as well. Fire kills the above-ground portions of the shrubs and herbs, but most shrubs and herbs promptly resprout from surviving below-ground organs. Wildfires may trigger transitions between seral stages.

OCFW sites are fire-adapted and had frequent, low severity surface fires prior to fire exclusion in the late nineteenth century. Historically, fire return intervals (FRIs) in \emph{P. ponderosa-Q. kelloggii} forests increased with increasing elevation in the Sierra Nevada, with a tendency towards shorter mean FRIs (5-15 years) on dry, west- and south-facing slopes and longer FRIs (15-25 years) on mesic, east- and north-facing slopes. Mid-elevation forests typically had mixed-severity fires that created patchy mosaics (Fryer 2007).

Estimates of fire rotations for these variants are available from the LandFire project and a few review papers. The LandFire project’s published fire return intervals are based on a series of associated models created using the Vegetation Dynamics Development Tool (VDDT). In VDDT, fires are specified concurrently with the transition that follows them. For example, a replacement fire causes a transition to the early development stage. In the RMLands model, such fires are classified as high mortality. However, in VDDT mixed severity fires may cause a transition to early development, a transition to a more open seral stage, or no transition at all. In this case, we categorize the first example as a high mortality fire, and the second and third examples as a low mortality fire. Based on this approach, we calculated fire rotations and the probability of high mortality fire for each of the OCFW seral stages (including the ultramafic modifier) (Tables~\ref{tab:ocfwdesc_fire} and \ref{tab:ocfwudesc_fire}). We computed overall target fire rotations based on expert input from Safford and Estes, and values from Mallek et al. (2013), and Van de Water and Safford (2011). 




\begin{table}[!htbp]
\footnotesize
\centering
\caption{Fire rotation index values and probability of high severity fire (at least 75\% overstory tree mortality) probabilities for Oak-Conifer Forest and Woodland. The seral stage that is most susceptible to fire (i.e., has the lowest predicted fire rotation) has a fire rotation index value of 1. Higher values correspond with lower likelihoods of experiencing wildfire. The values are relative only within an individual seral stage and should not be compared against other land cover types. Values were derived from VDDT model 0610300 (LandFire 2007a), Mallek et al. (2013), and Safford and Estes (personal communication). }
\label{tab:ocfwdesc_fire}
\begin{tabular}{@{}lcc@{}}
\toprule
 \textbf{Seral Stage}    & \textbf{\begin{tabular}[c]{@{}c@{}}Fire Rotation \\ Index\end{tabular}} & \textbf{\begin{tabular}[c]{@{}c@{}}Probability of \\ High Severity Fire\end{tabular}} \\ \hline
Early (All)     		 & 3.8            & 1                             \\
Mid--Closed    			 & 1.4            & 0.26                          \\
Mid--Moderate  			 & 1.2           & 0.14                          \\
Mid--Open      			 & 1.0           & 0.05                          \\
Late--Closed   			 & 1.9           & 0.20                          \\
Late--Moderate 			 & 1.3           & 0.08                          \\
Late--Open     			 & 1.0           & 0.01        \\ 
\emph{Target Fire Rotation}    			& \emph{21 years}  &   \\ 
\bottomrule
\end{tabular}
\end{table}

\begin{table}[!htbp]
\footnotesize
\centering
\caption{Fire rotation index values and probability of high severity fire (at least 75\% overstory tree mortality) probabilities for Oak-Conifer Forest and Woodland - Ultramafic. The seral stage that is most susceptible to fire (i.e., has the lowest predicted fire rotation) has a fire rotation index value of 1. Higher values correspond with lower likelihoods of experiencing wildfire. The values are relative only within an individual seral stage and should not be compared against other land cover types. Values were derived from VDDT model 0610210 (LandFire 2007b), Mallek et al. (2013), and Safford and Estes (personal communication).}
\label{tab:ocfwudesc_fire}
\begin{tabular}{@{}lcc@{}}
\toprule
 \textbf{Seral Stage}    & \textbf{\begin{tabular}[c]{@{}c@{}}Fire Rotation \\ Index\end{tabular}} & \textbf{\begin{tabular}[c]{@{}c@{}}Probability of \\ High Severity Fire\end{tabular}} \\ \hline
Early (All)     		 & 3.8            & 1                             \\
Mid--Closed    			 & 1.4           & 0.26                          \\
Mid--Moderate  			 & 1.2           & 0.14                          \\
Mid--Open      			 & 1.0           & 0.05                          \\
Late--Closed   			 & 1.9           & 0.20                          \\
Late--Moderate 			 & 1.3           & 0.08                          \\
Late--Open     			 & 1.0           & 0.01        \\ 
\emph{Target Fire Rotation}    			& \emph{21 years}  &   \\ 
\bottomrule
\end{tabular}
\end{table}

\subsubsection{Other Disturbance}

\subsection*{Vegetation Seral Stages}
We recognize seven separate seral stages for OCFW and OCFW\_U: Early Development (ED), Mid Development - Open Canopy Cover (MDO), Mid Development - Moderate Canopy Cover, Mid Development - Closed Canopy Cover (MDC), Late Development - Open Canopy Cover (LDO), Late Development - Moderate Canopy Cover (LDM), and Late Development - Closed Canopy Cover (LDC) (Figure~\ref{transmodel_ocfw}). Our seral stages are an alternative to ``successional'' classes that imply a linear progression of states and tend not to incorporate disturbance. The seral stages identified here are derived from a combination of successional processes and anthropogenic and natural disturbance, and are intended to represent a composition and structural condition that can be arrived at from multiple other conditions described for that landcover type. Thus our seral stages incorporate age, size, canopy cover, and vegetation composition. In general, the delineation of stages has originated from the LandFire biophysical setting model descriptive of a given landcover type; however, seral stages are not necessarily identical to the classes identified in those models.

\begin{figure}[htbp]
\centering
\includegraphics[width=0.8\textwidth]{/Users/mmallek/Documents/Thesis/statetransmodel/StateTransitionModel/7class.png}
\caption{State and Transition Model for Oak-Conifer Forest and Woodland. Each dark grey box represents one of the seven seral stages for this landcover type. Each column of boxes represents a stage of development: early, middle, and late. Each row of boxes represents a different level of canopy cover: closed (70-100\%), moderate (40-70\%), and open (0-40\%). Transitions between states/seral stages may occur as a result of high mortality fire, low mortality fire, or succession. Specific pathways for each are denoted by the appropriate color line and arrow: red lines relate to high mortality fire, orange lines relate to low mortality fire, and green lines relate to natural succession.} 
\label{transmodel_ocfw}
\end{figure}

\paragraph{Early Development (ED)} 


\paragraph{Description}
The early seral stage is the initial post-disturbance community dominated by coppicing oak sprouts (predominantly \emph{Q. kelloggi}, but potentially also \emph{Q. chrysolepis}). \emph{T. diversilobum} may be abundant. Bunchgrasses and associated forbs dominate understory. Localized native herbivory may maintain oak sprouts in ``shrub'' form for extended period. Vegetation may also include conifer seedling/saplings (LandFire 2007a).

On sites or areas that are dry or of low quality, significant pine regeneration may depend on concurrent disturbance of shrub species and a good pine seed crop with favorable weather. Thus, it may require 50-100 years for significant pine regeneration in the absence of intervention. Dense brush is typical in young stands and an herbaceous layer may develop on some sites. On drier sites, there is less tendency for succession toward shade-adapted species. As young, dense stands age and attain a closed canopy, they exclude most undergrowth. When other adapted conifers occur in moist pine stands of medium to high site quality, they may form a significant understory in about 20 years in the absence of fire (Fitzhugh 1988).

\paragraph{Succession Transition} In the absence of disturbance, patches in this seral stage will begin transitioning to a mid development seral stage at 20 years. The rate of succession per time step is 0.7. The transition may be to either MDC or MDO. The secondary rate of succession to MDO is 0.4 and to MDC is 0.6. At 50 years, all patches will have succeeded to either MDC or MDO. On average, patches remain in ED for 27 years.
\begin{adjustwidth}{2cm}{}
\medskip
\textbf{Ultramafic Modifier} Succession may be substantially delayed. Thus, in the absence of disturbance, patches in this seral stage will begin transitioning to MDO at 50 years and may be delayed in the ED seral stage for as long as 100 years. A patch in this seral stage succeeds at a rate of 0.2 per time step. 

\end{adjustwidth}
\paragraph{Wildfire Transition}
High mortality wildfire (100\% of fires in this seral stage) recycles the patch through the Early Development seral stage, regardless of soil type. Low mortality wildfire is not modeled for this seral stage.


\noindent\hrulefill


\paragraph{Mid Development - Open Canopy Cover (MDO)}

\paragraph{Description} The mid-seral, open seral stage has hardwoods dominating the canopy and may have sporadic conifer presence at low coverage levels. Oaks are pole-sized to very large. Bunchgrasses and shade-intolerant shrubs, most notably, will be prominent on the majority of sites. This seral stage is distinguished from MDM and MDC primarily by its reduced conifer presence (LandFire 2007a).

\paragraph{Succession Transition} Patches in this seral stage will maintain under low mortality disturbance, but after 15 years without fire they begin transitioning to MDM at a rate of 0.7 per timestep. At 150 years since transitioning to a mid development seral stage, succession to LDO occurs at a rate of 0.3 per timestep. All remaining patches transition at 230 years. 
\begin{adjustwidth}{2cm}{}
\medskip

\textbf{Ultramafic Modifier}  In the absence of low mortality disturbance, patches will begin transitioning to MDC at 30 years at a rate of 0.1 per timestep. At 200 years in the mid development seral stage, succession to LD occurs at a rate of 0.3 per timestep. All remaining patches transition at 280 years.

\end{adjustwidth}
\paragraph{Wildfire Transition}
High mortality wildfire (5\% of fires in this seral stage) recycles the patch through the ED seral stage. Low mortality wildfire (95\%) maintains the patch in MDO.

\noindent\hrulefill

\paragraph{Mid Development - Moderate Canopy Cover (MDM)}

\paragraph{Description} The mid-seral, moderate canopy cover seral stage may represent a drier, hardwood dominated site that has gone without fire for an extended period, or a mesic site supporting both oak and yellow pine species that has been opened up by fire. \emph{P. menziesii} may occur. Oaks are pole to medium sized with moderate crown closure. Conifers are generally medium to large, depending on stand age. Overall canopy cover ranges from 40-70\%. Sod-forming grasses and shade-tolerant shrubs will be prominent on the majority of sites. Species from more arid sites may be remnants of earlier, more open post-fire communities (LandFire 2007a).

\paragraph{Succession Transition} Patches in this seral stage may maintain under low mortality disturbance, but after 15 years without fire they begin transitioning to MDC at a rate of 0.7 per timestep. At 110 years since transitioning to a mid development seral stage, succession to LDO occurs at a rate of 0.3 per timestep. All remaining patches transition at 180 years.
\begin{adjustwidth}{2cm}{}

\medskip
\textbf{Ultramafic Modifier}  In the absence of low mortality disturbance, patches will begin transitioning to MDC at 30 years at a rate of 0.1 per timestep. At 130 years in the mid development seral stage, succession to LDM occurs at a rate of 0.2 per timestep. All remaining patches transition at 250 years.

\end{adjustwidth}
\paragraph{Wildfire Transition}
High mortality wildfire (14\% of fires in this seral stage) recycles the patch through the ED seral stage. Low mortality wildfire (86\%) triggers a transition to MDO 32\% of the time; otherwise the patch remains in MDC.

\noindent\hrulefill

\paragraph{Mid Development - Closed Canopy Cover (MDC)}

\paragraph{Description} The mid-seral, closed seral stage is representative of the more mesic end of the environmental gradient and supports a dense canopy of oak and \emph{P. ponderosa} and/or \emph{P. jeffreyi}. Occasional \emph{P. menziesii} may occur. Oaks are pole to medium sized with crown closure approaching 70\%. Conifers are generally medium to large, depending on stand age. Overall canopy cover is at least 50\%. Sod-forming grasses and shade-tolerant shrubs will be prominent on the majority of sites. Species from more arid sites may be remnants of earlier, more open post-fire communities (LandFire 2007a).

\paragraph{Succession Transition} In the absence of stand-replacing disturbance, patches in this seral stage will begin transitioning to LDC at 80 years in an mid development seral stage at a rate of 0.3 per time step. At 150 years, all remaining patches succeed to LDC.
\begin{adjustwidth}{2cm}{}

\medskip
\textbf{Ultramafic Modifier}  Transition to late seral seral stages may be substatially delayed. Thus, in the absence of stand-replacing disturbance, patches in this seral stage will begin transitioning to LDC after 80 years at a rate of 0.2 per time step and may be delayed in a mid development seral stage for up to 300 years.

\end{adjustwidth}
\paragraph{Wildfire Transition} High mortality wildfire (15\% of fires in this seral stage) recycles the patch through the ED seral stage. Low mortality wildfire (85\%) triggers a transition to MDM 60\% of the time; otherwise the patch remains in MDC.


\noindent\hrulefill


\paragraph{Late Development - Open Canopy Cover (LDO)}

\paragraph{Description} The late-seral seral stage occurs when stand-replacing fire has been excluded from a patch for an extended period of time. Oaks are being overtopped by conifers. Thus, in this seral stage, oaks comprise a smaller proportion of the stand. Oaks and conifers are mature and large (LandFire 2007a). In general, sites are relatively open (Estes 2013).

\paragraph{Succession Transition} Patches in this seral stage will maintain under low mortality disturbance, but after 15 years without fire they begin transitioning to LDM at a rate of 0.7 per timestep. 
\begin{adjustwidth}{2cm}{}

\medskip
\textbf{Ultramafic Modifier}  In the absence of disturbance, patches in LDO will maintain.

\end{adjustwidth}
\paragraph{Wildfire Transition}
High mortality wildfire (1\% of fires in this seral stage) recycles the patch through the ED seral stage. Low mortality wildfire (99\%) maintains the patch in LDO.

\noindent\hrulefill

\paragraph{Late Development - Moderate Canopy Cover (LDM)}

\paragraph{Description} The late-seral seral stage occurs when stand-replacing fire has been excluded from a patch for an extended period of time. Oaks are being overtopped by conifers, including shade-tolerant conifers such as \emph{P. menziesii}. Thus, in this seral stage, oaks and even pines comprise a smaller proportion of the stand. Oaks and conifers are mature and large (LandFire 2007a). 

\paragraph{Succession Transition} Patches in this seral stage will maintain under low mortality disturbance, but after 15 years without fire they begin transitioning to LDC at a rate of 0.7 per timestep.
\begin{adjustwidth}{2cm}{}

\medskip
\textbf{Ultramafic Modifier}  In the absence of disturbance, patches in LDM will maintain.

\end{adjustwidth}
\paragraph{Wildfire Transition} High mortality wildfire (8\% of fires in this seral stage) recycles the patch through the ED seral stage. Low mortality wildfire (92\%) triggers a transition to LDO 18\% of the time; otherwise the patch remains in LDC.

\noindent\hrulefill

\paragraph{Late Development - Closed Canopy Cover (LDC)}

\paragraph{Description} The late-seral seral stage occurs when stand-replacing fire has been excluded from a patch for an extended period of time. Oaks are being overtopped by conifers, especially shade-tolerant conifers such as \emph{P. menziesii}. Thus, in this seral stage, oaks and even pines comprise a smaller proportion of the stand (LandFire 2007a). 

\paragraph{Succession Transition} In the absence of transition-causing disturbance, patches in this seral stage will maintain, regardless of soil characteristics.

\paragraph{Wildfire Transition} High mortality wildfire (20\% of fires in this seral stage) recycles the patch through the ED seral stage. Low mortality wildfire (80\%) triggers a transition to LDM 58\% of the time; otherwise the patch remains in LDC.

\noindent\hrulefill


\newpage
\subsection*{Seral Stage Classification}
\begin{table}[!htbp]
\footnotesize
\centering
\caption{Classification of cover seral stage for OCFW, for early and mid development stages. Diameter at Breast Height (DBH) and Cover From Above (CFA) values taken from EVeg polygons. DBH categories are: null, 0-0.9'', 1-4.9'', 5-9.9'', 10-19.9'', 20-29.9'', 30''+. CFA categories are null, 0-10\%, 10-20\%, \dots , 90-100\%. Each row in the table below should be read with a boolean AND across each column of a row.}
\label{ocfw_classification}
\begin{tabular}{@{}lrrrrr@{}}
\toprule
\textbf{\begin{tabular}[l]{@{}l@{}}Cover \\ Condition\end{tabular}} & \textbf{\begin{tabular}[r]{@{}r@{}}Overstory Tree \\ Diameter 1 \\ (DBH)\end{tabular}} & \textbf{\begin{tabular}[r]{@{}r@{}}Overstory Tree \\ Diameter 2 \\ (DBH)\end{tabular}} & \textbf{\begin{tabular}[r]{@{}r@{}}Total Tree\\ CFA (\%)\end{tabular}} & \textbf{\begin{tabular}[r]{@{}r@{}}Conifer \\ CFA (\%)\end{tabular}} & \textbf{\begin{tabular}[r]{@{}r@{}}Hardwood \\ CFA (\%)\end{tabular}} \\ \midrule
Early All        & null           & null    & any    & any    & any    \\
Early All        & 0-4.9''         & 0-4.9''  & any    & any    & any    \\
Early All        & 0-4.9''         & null    & any    & any    & any    \\
Mid Open         & 0-4.9''         & 5-29.9'' & 0-40   & any    & any    \\
Mid Open         & 5-29.9''        & null    & 0-40   & any    & any    \\
Mid Open         & 5-29.9''        & null    & null   & 0-40   & null   \\
Mid Open         & 5-29.9''        & null    & null   & null   & 0-40   \\
Mid Open         & 5-29.9''        & null    & null   & 0-40   & 0-40   \\
Mid Open         & 5-29.9''        & 0-29.9'' & 0-40   & any    & any    \\
Mid Open         & 5-29.9''        & 0-29.9'' & null   & 0-40   & 0-40   \\
Mid Moderate     & 0-4.9''         & 5-29.9'' & 40-70  & any    & any    \\
Mid Moderate     & 5-29.9''        & null    & 40-70  & any    & any    \\
Mid Moderate     & 5-29.9''        & null    & null   & 40-70  & null   \\
Mid Moderate     & 5-29.9''        & null    & null   & null   & 40-70  \\
Mid Moderate     & 5-29.9''        & null    & null   & 40-70  & 0-70   \\
Mid Moderate     & 5-29.9''        & null    & null   & 0-70   & 40-70  \\
Mid Moderate     & 5-29.9''        & 0-29.9'' & 40-70  & any    & any    \\
Mid Moderate     & 5-29.9''        & 0-29.9'' & null   & 40-70  & 0-70   \\
Mid Moderate     & 5-29.9''        & 0-29.9'' & null   & 0-70   & 40-70  \\
Mid Closed       & 0-4.9''         & 5-29.9'' & 70-100 & any    & any    \\
Mid Closed       & 5-29.9''        & null    & 70-100 & any    & any    \\
Mid Closed       & 5-29.9''        & null    & null   & 70-100 & any    \\
Mid Closed       & 5-29.9''        & null    & null   & any    & 70-100 \\
Mid Closed       & 5-29.9''        & 0-29.9'' & 70-100 & any    & any    \\
Mid Closed       & 5-29.9''        & 0-29.9'' & null   & 70-100 & any    \\
Mid Closed       & 5-29.9''        & 0-29.9'' & null   & any    & 70-100 \\ \bottomrule
\end{tabular}
\end{table}

\begin{table}[!htbp]
\footnotesize
\centering
\caption{Classification of cover seral stage for OCFW, for late development stages. Diameter at Breast Height (DBH) and Cover From Above (CFA) values taken from EVeg polygons. DBH categories are: null, 0-0.9'', 1-4.9'', 5-9.9'', 10-19.9'', 20-29.9'', 30''+. CFA categories are null, 0-10\%, 10-20\%, \dots , 90-100\%. Each row in the table below should be read with a boolean AND across each column of a row.}
\label{ocfw_classification2}
\begin{tabular}{@{}lrrrrr@{}}
\toprule
\textbf{\begin{tabular}[l]{@{}l@{}}Cover \\ Condition\end{tabular}} & \textbf{\begin{tabular}[r]{@{}r@{}}Overstory Tree \\ Diameter 1 \\ (DBH)\end{tabular}} & \textbf{\begin{tabular}[r]{@{}r@{}}Overstory Tree \\ Diameter 2 \\ (DBH)\end{tabular}} & \textbf{\begin{tabular}[r]{@{}r@{}}Total Tree\\ CFA (\%)\end{tabular}} & \textbf{\begin{tabular}[r]{@{}r@{}}Conifer \\ CFA (\%)\end{tabular}} & \textbf{\begin{tabular}[r]{@{}r@{}}Hardwood \\ CFA (\%)\end{tabular}} \\ \midrule
Late Open        & 30''+           & any     & 0-40   & any    & any    \\
Late Open        & 30''+           & any     & null   & 0-40   & null   \\
Late Open        & 30''+           & any     & null   & null   & 0-40   \\
Late Open        & 30''+           & any     & null   & 0-40   & 0-40   \\
Late Open        & any            & 30''+    & 0-40   & any    & any    \\
Late Open        & any            & 30''+    & null   & 0-40   & null   \\
Late Open        & any            & 30''+    & null   & null   & 0-40   \\
Late Open        & any            & 30''+    & null   & 0-40   & 0-40   \\
Late Moderate    & 30''+           & any     & 40-70  & any    & any    \\
Late Moderate    & 30''+           & any     & null   & 40-70  & null   \\
Late Moderate    & 30''+           & any     & null   & null   & 40-70  \\
Late Moderate    & 30''+           & any     & null   & 40-70  & 0-70   \\
Late Moderate    & 30''+           & any     & null   & 0-70   & 40-70  \\
Late Moderate    & any            & 30''+    & 40-70  & any    & any    \\
Late Moderate    & any            & 30''+    & null   & 40-70  & null   \\
Late Moderate    & any            & 30''+    & null   & null   & 40-70  \\
Late Moderate    & any            & 30''+    & null   & 40-70  & 0-70   \\
Late Moderate    & any            & 30''+    & null   & 0-70   & 40-70  \\
Late Closed      & 30''+           & any     & 70-100 & any    & any    \\
Late Closed      & 30''+           & any     & null   & 70-100 & any    \\
Late Closed      & 30''+           & any     & null   & any    & 70-100 \\
Late Closed      & any            & 30''+    & 70-100 & any    & any    \\
Late Closed      & any            & 30''+    & null   & 70-100 & any    \\
Late Closed      & any            & 30''+    & null   & any    & 70-100 \\ \bottomrule
\end{tabular}
\end{table}


\clearpage

\subsection*{References}
\begin{hangparas}{.25in}{1} 
\interlinepenalty=10000
Anderson, Richard. ``Montane Hardwood-Conifer (MHC).'' \emph{A Guide to Wildlife Habitats of California}, edited by Kenneth E. Mayer and William F. Laudenslayer. California Deparment of Fish and Game, 1988. \burl{http://www.dfg.ca.gov/biogeodata/cwhr/pdfs/MHC.pdf}. Accessed 4 December 2012.

``CalVeg Zone 1.'' Vegetation Descriptions. \emph{Vegetation Classification and Mapping}.  11 December 2008. U.S. Forest Service. \burl{http://www.fs.usda.gov/Internet/FSE_DOCUMENTS/fsbdev3_046448.pdf}. Accessed 2 April 2013.
Estes, Becky L. Personal communication, 21 June 2013.

Fitzhugh, E. Lee. ``Ponderosa Pine (PPN).'' \emph{A Guide to Wildlife Habitats of California}, edited by Kenneth E. Mayer and William F. Laudenslayer. California Deparment of Fish and Game, 1988. \burl{http://www.dfg.ca.gov/biogeodata/cwhr/pdfs/PPN.pdf}. Accessed 4 December 2012.

Fryer, Janet L. ``Quercus kelloggii.'' \emph{Fire Effects Information System}, U.S. Department of Agriculture, Forest Service,  Rocky Mountain Research Station, Fire Sciences Laboratory, 2007. \burl{http://www.fs.fed.us/database/feis/plants/tree/quekel/all.html}. Accessed 21 December 2012.

LandFire. ``Biophysical Setting Models.'' Biophysical Setting 0610300: Mediterranean California Lower Montane Black Oak-Conifer Forest and Woodland. 2007a. LANDFIRE Project, U.S. Department of Agriculture, Forest Service; U.S. Department of the Interior. \burl{http://www.landfire.gov/national_veg_models_op2.php}. Accessed 9 November 2012.

LandFire. ``Biophysical Setting Models.'' Biophysical Setting 0610210: Klamath-Siskiyou Lower Montane Serpentine Mixed Conifer Woodland. 2007b. LANDFIRE Project, U.S. Department of Agriculture, Forest Service; U.S. Department of the Interior. \burl{http://www.landfire.gov/national_veg_models_op2.php}. Accessed 9 November 2012.

LandFire. ``Biophysical Setting Models.'' Biophysical Setting 0711700: Klamath-Siskiyou Xeromorphic Serpentine Savanna and Chaparral. 2007c. LANDFIRE Project, U.S. Department of Agriculture, Forest Service; U.S. Department of the Interior. \burl{http://www.landfire.gov/national_veg_models_op2.php}. Accessed 30 November 2012.

O'Geen, Anthony T., Randy A. Dahlgren, and Daniel Sanchez-Mata. ``California Soils and Examples of Ultramafic Vegetation.'' In \emph{Terrestrial Vegetation of California, 3rd Edition}, edited by Michael Barbour, Todd Keeler-Wolf, and Allan A. Schoenherr, 71-106. Berkeley and Los Angeles: University of California Press, 2007. 

Skinner, Carl N. and Chi-Ru Chang. ``Fire Regimes, Past and Present.'' \emph{Sierra Nevada Ecosystem Project: Final report to Congress, vol. II, Assessments and scientific basis for management options}. Davis: University of California, Centers for Water and Wildland Resources, 1996.

Van de Water, Kip M. and Hugh D. Safford. ``A Summary of Fire Frequency Estimates for California Vegetation Before Euro-American Settlement.'' \emph{Fire Ecology} 7.3 (2011): 26-57. doi: 10.4996/fireecology.0703026.

\end{hangparas}

