% !TEX root = master.tex

\begin{abstract}
In California's northern Sierra Nevada mountains, the fire-dependent processes of forest ecosystems have been interrupted and altered by human land use and fire suppression. Forest Service policy directs land managers to plan for a future that includes multiple use and the restoration of resilient ecosystems. Planning decisions are to be informed by an analysis of the range of variability of ecological processes at multiple scales. Current climate trends in the northern Sierra are of increasing temperatures, increased precipitation, and earlier snowmelt, as well as changes to the frequency and duration of drought. These climate changes have and continue to influence fire frequency, extent, and severity.

For this thesis,  project partners and I adapted the Rocky Mountain Landscape Simulator (\textsc{RMLands}), a spatially explicit, stochastic, landscape disturbance and succession model, for use in the Sierra Nevada. \textsc{RMLands} was used to simulate wildfires and vegetation dynamics on a portion of the Tahoe National Forest in California, first under  historical climate settings and then under alternative climate trajectories based on the Representative Concentration Pathway RCP8.5 projections. I then quantified the historical and the future range of variability in the disturbance regime, seral stage distrbution, and patch configuration, and assessed departure of the current landscape from the ranges of variability using R and \textsc{Fragstats}.

My results suggest more frequent and extensive high severity fire, as well as higher canopy closure, than most other studies focused on mixed conifer Sierran forests. Although these results differ, they typically agree qualitatively with other research. Some differences may be accounted for by differences in study design. Under warmer and drier future climate scenarios, the total area burned, and the proportion burned at high severity, increased. Due to fire's effects on vegetation, the current landscape departs from either historical or future conditions by several statistical measures. Based on these findings, I recommend that managers implement aggressive restoration efforts, utilize mitigation measures where the consequences of changing fire regimes are socially unacceptable, and carefully balance the needs of different ecosystems and of the resident communities. My study can be used to inform goals and specific strategies in restoration planning and help project planners think about impacts at the landscape scale.
 [359 words, max is 350]
\end{abstract}



